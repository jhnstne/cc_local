\documentclass[12pt]{article}
\usepackage{times}
\usepackage[pdftex]{graphicx}
\makeatletter
\def\@maketitle{\newpage
 \null
 \vskip 2em                   % Vertical space above title.
 \begin{center}
       {\Large\bf \@title \par}  % Title set in \Large size. 
       \vskip .5em               % Vertical space after title.
       {\lineskip .5em           %  each author set in a tabular environment
        \begin{tabular}[t]{c}\@author 
        \end{tabular}\par}                   
  \end{center}
 \par
 \vskip .5em}                 % Vertical space after author
\makeatother

% default values are 
% \parskip=0pt plus1pt
% \parindent=20pt

\newcommand{\SingleSpace}{\edef\baselinestretch{0.9}\Large\normalsize}
\newcommand{\DoubleSpace}{\edef\baselinestretch{1.4}\Large\normalsize}
\newcommand{\Comment}[1]{\relax}  % makes a "comment" (not expanded)
\newcommand{\Heading}[1]{\par\noindent{\bf#1}\nobreak}
\newcommand{\Tail}[1]{\nobreak\par\noindent{\bf#1}}
\newcommand{\QED}{\vrule height 1.4ex width 1.0ex depth -.1ex\ \vspace{.3in}} % square box
\newcommand{\arc}[1]{\mbox{$\stackrel{\frown}{#1}$}}
\newcommand{\lyne}[1]{\mbox{$\stackrel{\leftrightarrow}{#1}$}}
\newcommand{\ray}[1]{\mbox{$\vec{#1}$}}          
\newcommand{\seg}[1]{\mbox{$\overline{#1}$}}
\newcommand{\tab}{\hspace*{.2in}}
\newcommand{\se}{\mbox{$_{\epsilon}$}}  % subscript epsilon
\newcommand{\ie}{\mbox{i.e.}}
\newcommand{\eg}{\mbox{e.\ g.\ }}
\newcommand{\figg}[3]{\begin{figure}[htbp]\vspace{#3}\caption{#2}\label{#1}\end{figure}}
\newcommand{\be}{\begin{equation}}
\newcommand{\ee}{\end{equation}}
\newcommand{\prf}{\noindent{{\bf Proof}:\ \ \ }}
\newcommand{\choice}[2]{\mbox{\footnotesize{$\left( \begin{array}{c} #1 \\ #2 \end{array} \right)$}}}      
\newcommand{\scriptchoice}[2]{\mbox{\scriptsize{$\left( \begin{array}{c} #1 \\ #2 \end{array} \right)$}}}
\newcommand{\tinychoice}[2]{\mbox{\tiny{$\left( \begin{array}{c} #1 \\ #2 \end{array} \right)$}}}
\newcommand{\ddt}{\frac{\partial}{\partial t}}
\newcommand{\Sn}[1]{\mbox{{\bf S}$^{#1}$}}
\newcommand{\calP}[1]{\mbox{{\bf {\cal P}}$^{#1}$}}

\newtheorem{theorem}{Theorem}	
\newtheorem{rmk}[theorem]{Remark}
\newtheorem{example}[theorem]{Example}
\newtheorem{conjecture}[theorem]{Conjecture}
\newtheorem{claim}[theorem]{Claim}
\newtheorem{notation}[theorem]{Notation}
\newtheorem{lemma}[theorem]{Lemma}
\newtheorem{corollary}[theorem]{Corollary}
\newtheorem{defn2}[theorem]{Definition}
\newtheorem{observation}[theorem]{Observation}
\newtheorem{implementation}[theorem]{Implementation note}

% \font\timesr10
% \newfont{\timesroman}{timesr10}
% \timesroman
  % remove if it causes you grief: you can always add
%in your own commands right her

\setlength{\oddsidemargin}{0pt}
\setlength{\topmargin}{0in}
\setlength{\textheight}{8.6in}
\setlength{\textwidth}{6.875in}
\setlength{\columnsep}{5mm}
\markright{\hfill \today \hfill}
\pagestyle{myheadings}

% -----------------------------------------------------------------------------

\title{Controlling a Virtual Camera}
\author{Ross Ptacek}
\begin{document}
\maketitle

\begin{abstract}

\end{abstract}

\section{Setup}
\subsection{The Path}
The path followed by our virtual camera has two distinct components, position
and orientation.
Since we are trying to precompute the entire path of the object, the keyframes
that we use to specify the path must also have a position and orientation 
component.
The choice of representation for position in each keyframe is simply the x,y,
and z coordinates of the point we want the camera to pass through.
For orientation, we choose to use a quaternion to represent the orientation at
each keyframe.
Remember that a quaternion is a 4-vector that encodes a rotation around an
arbitrary vector.
We choose quaternions over the more typical choice of rotation matrices because
quaternions allow us to develop rational motion.
In between every two consecutive keyframes, we have to interpolate between the
positions and orientations specified by the two keyframes.
To interpolate position, we use a cubic Bezier curve.
To interpolate orientations we use a method of rationally interpolating 
quaternions proposed in ... .
Let the Bezier spline formed by connecting the curves between consecutive
keyframes be $B(t)$, and let the rational quaternion spline formed by 
rationally interpolating through the orientations of the keyframes be $Q(t)$.
Note that these two curves are parameterized the same.
So at any parameter value $t_{i}$, the position of the camera is given by
$B(t_{i})$, and the orientation of the camera is given by $Q(t_{i})$.
The path specified by $B(t)$ and $Q(t)$ is our first guess at the path of the
camera.
Now we must ensure that this path does not collide with objects in the scene
and modify the path if it does.

%does this deserve a subsection?
\subsection{The Scene}
The scene can be any polygonal mesh.
If the mesh is not already a Triangular mesh, then it will be converted to one.
Optimally, a Delauney triangulation of the non-triangular faces will be used to
break such faces into triangles.
Triangles are preferred both because they are nicer to deal with but also
because the octree data structure that we will use in collision decection works
more simply with triangles.
The set of triangles that represent the scene will be called $T_{s}$.


\section{Camera Model}
The path of the camera itself is insufficient to determine collisions.
The virtual camera is infinitely small.
In fact, collisions of the virtual camera with the scene are not necessarily
collisions at all.
The visible volume is a rectangular frustum.
This frustum, the view frustum, is determined by the parameters $fovx$,$ fovy$,
$near$, and $far$.
$fovx$ and $fovy$ are angles defining the field of view in x and y.
$near$ and $far$ are distances in z where the frustum is cut.
The diagram below illustrates this.

%insert figure here

The diagram makes it clear that the near clipping plane of the view frustum
is what we should be checking for collisions with.
In effect, it is the ``lens'' of our virtual camera.
\section{Collision Detection}
\subsection{Calculating the Near Clipping Plane Corners}
Our first task in computing collisions is to calculate exactly where the
corners of the near clipping plane, NCP from now on, are at any time $t$ of the
path.

%diagram

From the diagram we see there is simple arithmetic to find a vector from the
camera position to any of the four corner points.

%math from powerpoint slides

It turns out that this vector is all that we really need.
\subsection{Calculating Corner Curves}
Using $B(t)$, $Q(t)$, and the vector from $B(t)$ to a corner of the NCP, $V$,
we can construct the path of the corner curve as the camera moves along the
path specified by $B(t)$ and $Q(t)$.
The procedure to do so is outlined in (Johnstone \& Williams '95 
``A Rational Model of the Surface Swept by a curve'').
In general, this algorithm takes the quaternion spline and transforms it into a
spline that interpolates rotation matrices.
So its control points are rotation matrices rather than quaternions.
Then the control matrices are all multiplied by $V$ to produce control points.
These control points are finally translated by the control points of $B(t)$.
The result is a degree 12 rational Bezier spline that specifies that path
of a corner point.
We will call these curves $C_{1}(s) \ldots C_{4}(s)$ where $C_{1}(s)$ is the 
upper left corner of the NCP and the rest follow in counterclockwise order.
Note that there is no guarantee that the corner curves are parameterized the
same was as $B(t)$ and $Q(t)$.
\subsection{Finding Collisions}




\end{document}




