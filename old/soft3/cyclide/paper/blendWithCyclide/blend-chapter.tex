% --------------------------------------------------------------------
%                          Introduction
% --------------------------------------------------------------------

\section{Introduction}
\label{section:cyclide-intro}

     Traditional solid modeling systems include primitives such as the popular
natural quadrics and torus. However, from
early stages of development, people have realized that natural
quadrics are not powerful enough and have tried to enlarge the set of 
primitives by adding some other powerful yet easy to handle surfaces.
Dupin cyclides have recently been proposed and studied from a couple of 
different perspectives.  A group at Cambridge University, including 
de Pont~\cite{depont:1984},\index{de Pont}
Martin~\cite{martin:1982}, \index{Martin}
Nutbourne~\cite{nutbourne-martin:1988}\index{Nutbourne} and 
Sharrock~\cite{sharrock:1985},\index{Sharrock} has 
studied the use of the cyclide as a patch.  
Boehm~\cite{boehm:1990},\index{Boehm}
Pratt~\cite{pratt:1990},\index{Pratt} and 
Chandru,\index{Chandru} Dutta\index{Dutta} and 
Hoffmann~\cite{chandru-dutta-hoffmann:1990}\index{Hoffmann} 
have looked at the cyclide as a
blending surface.

     From Hoffmann and 
Hopcroft's~\cite{hoffmann-hopcroft:1986,hoffmann-hopcroft:1987,hoffmann-hopcroft:1988} 
\index{Hopcroft} and Warren's~\cite{warren:1986,warren:1989}\index{Warren} 
fundamental work, we know that 
blending two quadrics, in general, requires a quartic surface.  However,
if two quadric surfaces have planar intersection, Warren shows how to blend 
them with quadric or cubic surfaces after imposing additional restrictions
on the clipping surfaces of the blending surfaces.   The most recent
result comes from Pratt~\cite{pratt:1990},\index{Pratt} credited to 
Sabin,\index{Sabin} that a necessary and sufficient condition for two cones, 
with intersecting axes, to have cyclide blending surfaces is that the cones 
have a common inscribed sphere\footnote{The original version of the 
Pratt-Sabin' result assumes
intersecting axes implicitly.  Note that the center of the common inscribed
sphere is the intersection point of the axes.}.  However, Pratt gives only
the proof of the sufficient part, using the offset properties of the cyclide.
With the present author's study of planar intersection of axial natural 
quadrics, we know immediately that the existence of a common inscribed sphere 
is equivalent to the cones, with intersecting axes, having planar 
intersection.  Therefore, Pratt and Sabin's result has an immediate extension
by replacing the condition of having a common inscribed sphere by the
condition of having planar intersection.  But, this direct replacement renders
Pratt and Sabin's original proof inappropriate.  Moreover, the construction
algorithm in Pratt's work is not powerful enough to uncover some of the hidden
geometry of the blending cyclide.  Therefore, in this chapter,
the present author sets out to rework Pratt and Sabin's result from a rather
different point of view.  Our results include the following:
\begin{enumerate}
     \item A more controllable construction algorithm for the blending 
          cyclides that uncovers some useful geometry.
     \item A more precise specification of the blending cyclide.  In
          Boehm's and Pratt's algorithm, some conditions for specifying
          a unique blending cyclide are implicitly assumed.  Our result
          makes these hidden conditions clearer and shows that they are always
          satisfied.
     \item For cones with planar intersection, the blending cyclides are not
          in a family, but can be organized into two families.  Each of these
          families corresponds to a diagonal of the skeletal quadrilateral
          on the axial plane.
     \item If two cones, with intersecting axes, have a double line in their
          intersection, we show that all blending cyclides are in a unique
          family of parabolic cyclides.  Furthermore, if the axes are
          parallel, no blending cyclide exists.
     \item If two cones have linear intersection, there are at most four and
          at least two families of blending cyclides.
\end{enumerate}

     This chapter has eleven sections and an appendix.  
The following section discusses 
previous work, while Section~\ref{section:prelim} lists some basic facts
to be used from elementary geometry.  Section~\ref{section:basic} presents
the necessary materials for uniquely specifying a Dupin cyclide.
Section~\ref{section:interpolating} explains the meaning of a blending
cyclide and shows that the common tangent curves of the cyclide and 
the given axial natural quadrics must be circles from the same family of
lines of curvature.
In Section~\ref{section:planar}, we give the missing necessary part of the
Pratt-Sabin proof: if two axial natural quadrics have a blending cyclide,
they have planar intersection.
In Section~\ref{section:cyclide-non-degen}, starting with our version of the 
Boehm-Pratt construction algorithm, we establish the converse:
If two axial natural quadrics have conic intersection,
they have a blending cyclide.  We also prove that any blending 
cyclide belongs to one of the two families.
Section~\ref{section:line-conic} handles the first  degenerate case and proves 
that the necessary and sufficient condition for two axial natural quadrics to
intersect in a double line and a finite conic is that they have a
blending parabolic cyclide.  
Section~\ref{section:cyc-linear} deals with the second degenerate case: the
linear intersection case in which all components are lines.  We show that,
in this case, there are at most four and at least two families of blending
cyclide. Section~\ref{section:cyc-concl} gives our conclusion.
Finally, the appendix contains a simple geometric proof of the offset property
of a Dupin cyclide: any offset of a Dupin cyclide is a Dupin cyclide.

% --------------------------------------------------------------------
%                          Previous Work
% --------------------------------------------------------------------

\section{Previous Work}
\label{section:cyc-prev}

      The Dupin cyclide was discovered by Charles Dupin\index{Dupin} in 1801 
(Dupin~\cite{dupin:1822}) when he was seventeen years old, using the concept of
lines of curvature invented by his teacher Gaspard Monge.\index{Monge}  
This is the first non-spherical surface whose lines of curvatures are all 
circles.

     The next important work came from 
Maxwell~\cite{maxwell:1868}\index{Maxwell} nearly seventy years later.  
In 1873, Cayley's\index{Cayley} important paper~\cite{cayley:1873} 
unified and simplified Dupin's and Maxwell's work.  
Casey~\cite{casey:1871}\index{Casey}
is another interesting and thorough study.  Since then, so many papers
were published at the end of the last and the beginning of this century, 
making cyclide a standard textbook topic 
(Baker~\cite{baker:1925},\index{Baker} 
Basset~\cite{basset:1910},\index{Basset}
Coolidge~\cite{coolidge:1916},\index{Coolidge} 
Darboux~\cite{darboux:1896,darboux:1917},\index{Darboux}
Fladt\index{Fladt} and 
Baur~\cite{fladt-baur:1975},\index{Baur} 
Forsyth~\cite{forsyth:1912},\index{Forsyth} 
Hilbert\index{Hilbert} and
Cohn-Vossen~\cite{hilbert-vossen:1952},\index{Cohn-Vossen} 
Jessop~\cite{jessop:1916},\index{Jessop}
Lilienthal~\cite{lilienthal:1913},\index{Lilienthal} 
Snyder\index{Snyder} and Sisam~\cite{snyder-sisam:1914},\index{Sisam}
and Woods~\cite{woods:1922}).\index{Woods}

     After a long quiet period, the study of cyclide is back, both in 
mathematics (Cecil~\cite{cecil:1992}\index{Cecil} and 
Cecil\index{Cecil} and Ryan~\cite{cecil-ryan:1985})\index{Ryan}
and CAGD.  In CAGD, the study of cyclide can be separated into two
different categories.  The first is its use as a patch design.  
According to Martin, a {\em principal patch}\index{principal patch} is a patch
whose boundary lines  are lines of curvature.  
A {\em cyclide patch}\index{cyclide patch} is
a principal patch whose two families of lines of curvature are circular arcs.
This approach was initiated in 
Martin's Ph.D. thesis~\cite{martin:1982}\index{Martin} and 
continued by de Pont~\cite{depont:1984}\index{de Pont} and 
Sharrock~\cite{sharrock:1985},\index{Sharrock} 
all from Cambridge University.  Their achievements as well as some others are 
reported in 
Nutbourne\index{Nutbourne} and Martin~\cite{nutbourne-martin:1988}.

     The second approach, using a larger piece of a cyclide as blending
surface, did not appear until 1989 when Pratt\index{Pratt} published his 
investigation in~\cite{pratt:1989,pratt:1990}.
The next year, Boehm~\cite{boehm:1990}\index{Boehm}
announced similar results on blending 
two cones with cyclides.  
Pratt~\cite{pratt:1990}\index{Pratt} also gives a proof, credited to
Sabin, that two cones with intersecting axes have a family of blending 
cyclides if and only if they
have a common inscribed sphere.  However, only the proof of the sufficient
part is given.  The construction algorithm of blending cyclides presented in 
Pratt's and Boehm's paper is simple; however, it is not powerful
enough to uncover some important insights of the constructed cyclide.
In particular, from the algorithm, it is not clear how the set of all
blending cyclides is organized.

     In this chapter, we present a more controllable algorithm, and show that 
blending cyclides belong to two distinct families, each controlled by a 
diagonal of the skeletal quadrilateral.  We also show that, except for a very
special case,  a necessary and
sufficient condition for two axial natural quadrics to have a blending
cyclide is they must have planar intersection, not only the existence of a
common inscribed sphere.  It is interesting to point out that the parabolic
cyclide can be used only when two axial natural quadrics have a double line
in their intersection.  Furthermore, in our algorithm, the control of the 
shape of the blending cyclide is obvious and is easy to use.

     Dutta\index{Dutta} in his Ph.D. thesis~\cite{dutta:1989} investigated the
use of piecewise Dupin cyclides\index{piecewise Dupin cyclide} as an 
approximation of blending surface.  His results along with some others are 
published in Chandru,\index{Chandru} Dutta\index{Dutta} and
Hoffmann~\cite{chandru-dutta-hoffmann:1989,chandru-dutta-hoffmann:1990}.
\index{Hoffmann}
By extracting part of the directrix
conic of a cyclide as a piece of a biarc, Dutta developed an algorithm to 
join several pieces of cyclides together as a way of 
approximating blending surface design.  There is no goodness-of-fit 
criterion in Dutta's study and hence, it is difficult to measure how well the 
piecewise cyclide can be used as blending surface.  
% --------------------------------------------------------------------
%                            Preliminaries
% --------------------------------------------------------------------

\section{Preliminaries}
\label{section:prelim}

     In this section, we will list all of the basic geometric preliminaries
that are used in the subsequent sections.  These materials include Menelaus' 
theorem and Desargues' theorem, which are powerful tools for proving 
concurrency of three lines and collinearity of three points,  
and the radical axis and the centers of similitude of two circles.
The reader may find more detailed discussion in 
Court~\cite{court:1925},\index{Court}
Davis~\cite{davis:1949},\index{Davis,D.R.} 
Johnson~\cite{johnson:1929},\index{Johnson} and especially Pedoe's\index{Pedoe}
lovely little book~\cite{pedoe:1957}.

     Menelaus' theorem\footnote{This theorem was discovered by Menelaus of
Alexandria (about 100 A.D.),\index{Menelaus of Alexandria} 
but was forgotten and rediscovered 1600 years later by an Italian 
mathematician, Giovanni Ceva,\index{Ceva} who published
this theorem in 1678 along with another that now bears his name,
the Ceva theorem.} 
and its converse are powerful tools for proving that three points, one on each 
side of a triangle, are collinear.  The Menelaus theorem can be stated as 
follows, using unsigned distance.

\begin{theorem}[Menelaus]
\label{thm:menelaus}\index{Menelaus theorem}
     Suppose a line intersects the sides of a triangle $\bigtriangleup ABC$ at
$X\in\stackrel{\longleftrightarrow}{AB}$,
$Y\in\stackrel{\longleftrightarrow}{BC}$ and
$Z\in\stackrel{\longleftrightarrow}{CA}$ (Figure~\ref{fig:menelaus}).
Then, the following identity holds:
\[ \frac{|\overline{AX}|}{|\overline{XB}|}\cdot
     \frac{|\overline{BY}|}{|\overline{YC}|}\cdot
     \frac{|\overline{CZ}|}{|\overline{ZA}|} = 1. \]
\end{theorem}
\begin{figure}
\vspace{4cm}
\caption{The Menelaus Theorem}
\label{fig:menelaus}
\end{figure}

     Desargues' theorem was published by Girard Desargues\index{Desargues} 
in his {\em M\'{e}thode Universelle de mettre en Perspective} (1636) 
(Smith~\cite{smith:1959}) and was forgotten until Brianchon\index{Brianchon} 
and Poncelet\index{Poncelet}  recognized its importance in projective geometry
in 1817 and 1822 respectively.  Desargues' theorem can be proven with the
Menelaus theorem and its converse. See, for example, 
Johnson~\cite{johnson:1929}\index{Johnson} or any
classic projective geometry text.\footnote{In modern projective geometry,
the Desargues theorem is usually listed as an axiom.}

\begin{theorem}[Desargues Theorem]
\label{thm:desargues}\index{Desargues theorem}
     Given two triangles $\bigtriangleup A_1A_2A_3$ and 
$\bigtriangleup B_1B_2B_3$ with no two of the six sides being identical 
(Figure~\ref{fig:desargues}),
$\stackrel{\longleftrightarrow}{A_1B_1}$,
$\stackrel{\longleftrightarrow}{A_2B_2}$ and
$\stackrel{\longleftrightarrow}{A_3B_3}$ are concurrent if and only if
$\stackrel{\longleftrightarrow}{A_1A_2}\cap\stackrel{\longleftrightarrow}{B_1B_2}$,
$\stackrel{\longleftrightarrow}{A_2A_3}\cap\stackrel{\longleftrightarrow}{B_2B_3}$ and
$\stackrel{\longleftrightarrow}{A_3A_1}\cap\stackrel{\longleftrightarrow}{B_3B_1}$
are collinear.
\end{theorem}
\begin{figure}
\vspace{6cm}
\caption{The Desargues Theorem}
\label{fig:desargues}
\end{figure}

\begin{definition}[Radical Axis]
\index{radical axis}
     Given two non-concentric circles, there exists a unique line from any
point of which the lengths of the tangents to both circles are equal.  
This line is called the radical axis of the given circles.  
\end{definition}

\begin{definition}[Coaxal Circles]
\index{coaxal circles}
\index{coaxal circles system}
\index{system of coaxal circles}
\index{line of centers}
     A system of circles is said to be coaxal if the radical axis of any two 
circles which belong to the system is the same as the radical axis of any 
other pair of circles of the system.  This system is called a coaxal circles
system, or system of coaxal circles.  Moreover, all of the centers of the
circles of a coaxal circles system lie on the same line, the line of centers.
\end{definition}

     In a coaxal circles system, if any two circles intersect in two points 
(resp., are tangent or are disjoint) then all circles intersect at the same 
points (resp., are tangent at the same point or are disjoint).  
See Figure~\ref{fig:radical-axis}.  In fact, it is not difficult to see that 
all of the circles are members of the pencil formed by any two fixed circles
in the system, and the radical axis is the linear member of the pencil.
\begin{figure}
\vspace{9.5cm}
\caption{Coaxal Circles System and Its Radical Axis}
\label{fig:radical-axis}
\end{figure}

     For each non-intersecting coaxal circles system, there are two point 
circles, the {\em limiting points}\index{limiting points} of the coaxal 
circle system.  The segment between these two points is the diameter of a 
circle that cuts all circles in the system orthogonally; and this circle, 
in turn, defines another coaxal circles system that intersect at the two 
limiting points.

     There is an important fact relating the coaxal circles system and the
Dupin cyclides.  Given a non-intersecting coaxal circles system with limiting 
points $L$ and $L^\prime$, if $L$ or $L^\prime$ is chosen to be the center of 
inversion, the circles in the system are inverted to concentric circles.  
This can be used to invert the two principal circles of a Dupin cyclide to 
two concentric circles, thus inverting the cyclide to a torus.

\begin{definition}[Centers of Similitude]
\index{centers of similitude}
     Given two non-concentric circles, with centers $O_1$ and $O_2$ and 
distinct radii $r_1$ and $r_2$, the point $S\in\overline{O_1O_2}$
(resp., $S\in\stackrel{\longleftrightarrow}{O_1O_2}-\overline{O_1O_2}$) 
satisfying 
$\frac{|\overline{O_1S}|}{|\overline{SO_2}|}=\frac{r_1}{r_2}$ is called the
internal (resp., external) center of similitude.  
\index{internal center of similitude}
\index{external center of similitude}
\end{definition}

       The construction of these centers of similitude is simple.
>From each center, construct a ray with the same direction meeting the 
respective circle at $P_1$ and $P_2$.  Then, 
$S=\stackrel{\longleftrightarrow}{P_1P_2}\cap
\stackrel{\longleftrightarrow}{O_1O_2}$ is a fixed point lying outside
of $\overline{O_1O_2}$, the {\em external center of similitude} 
(Figure~\ref{fig:similitude-centers}).
If the rays have opposite directions, $S$ lies in $\overline{O_1O_2}$
and is called the {\em interior center of similitude} 
(Figure~\ref{fig:similitude-centers}).
\begin{figure}
\vspace{3.5cm}
\caption{The Centers of Similitude of Two Circles}
\label{fig:similitude-centers}
\end{figure}

     In general, the external (resp., internal) center of similitude is the 
intersection points of the two outer (resp., inner) common tangents.
Due to the fact that the inner common tangents may not exist, the 
internal center of similitude may not be obtained this way.

     The following lemma is important to our development. 

\begin{lemma}
\label{lemma:tangent-center}
     Given two distinct fixed circles $C_1$ and $C_2$ and a third circle $C$
tangent to $C_1$ and $C_2$ at $P_1$ and $P_2$ respectively, the line 
$\stackrel{\longleftrightarrow}{P_1P_2}$ passes through one of the centers of
similitude (Figure~\ref{fig:tangent-center}).
\end{lemma}
{\bf Proof:}  Suppose the circles $C_1,C_2$ and $C$ have centers $O_1,O_2$ 
and $O$, and radii $r_1,r_2$ and $r$ respectively.
If $\stackrel{\longleftrightarrow}{P_1P_2}$ and
$\stackrel{\longleftrightarrow}{O_1O_2}$ are parallel, the radii of $C_1$ and
$C_2$ must be equal and hence one center of similitude is at infinity and 
the lemma holds.  If the lines are identical, then all centers of similitude
lie on both lines.  Thus, we only have to consider the case of intersecting
lines.  Let $S$ be their intersection point.   
We will show that $S$ is a center of similitude.

     Let $\stackrel{\longleftrightarrow}{P_1P_2}$ intersect $C_1$ and $C_2$
at $Q_1$ and $Q_2$ as shown in Figure~\ref{fig:tangent-center}.  
Let the tangents of circle $O$ at $P_1$ and $P_2$ meet at $X$.  $X$ cannot be 
at infinity otherwise $P_1$ and $P_2$ would lie on 
$\stackrel{\longleftrightarrow}{O_1O_2}$.
\begin{figure}
\vspace{4.5cm}
\caption{The Line Through the Tangent Points Passes a Center of Similitude}
\label{fig:tangent-center}
\end{figure}

     Now we want to show that $\stackrel{\longleftrightarrow}{O_1Q_1}$ is
parallel to $\stackrel{\longleftrightarrow}{O_2P_2}$.  Then, from
$\bigtriangleup SO_1Q_1\sim\bigtriangleup SO_2P_2$, we have
$\frac{|\overline{SO_1}|}{|\overline{SO_2}|}=
\frac{|\overline{O_1Q_1}|}{|\overline{O_2P_2}|}=\frac{r_1}{r_2}$.
Thus, $S$ divides $\overline{O_1O_2}$ into the ratio of radii of the fixed
circles and $S$ is a center of similitude.

     Since $|\overline{XP_1}|=|\overline{XP_2}|$, 
$\angle XP_1P_2=\angle XP_2P_1$.  Thus, $\angle O_1Q_1P_1=\angle O_1P_1Q_1=
\angle 90^\circ-\angle XP_1P_2=90^\circ-\angle XP_2P_1=\angle O_2P_2Q_2$.
Hence, $\stackrel{\longleftrightarrow}{O_1Q_1}$ is parallel to 
$\stackrel{\longleftrightarrow}{O_2P_2}$ and our lemma is proven. \QED

     The following corollary converts the above lemma to a form related to 
the radical axis of the fixed circles.

\begin{corollary}
\label{cor:tangent-center}
     From any point on the radical axis, construct a tangent to each fixed
circle giving two tangent points.  The line joining these two points passes
through a center of similitude.
\end{corollary}
{\bf Proof:}  Let $H$ be the intersection point of the lines through the
tangent points and perpendicular to the tangent lines.  Recall that from any 
point on the radical axis the tangent lengths are equal.  
It is not difficult to see that $H$ is equidistant to the tangent lines.
Therefore, the circle with this distance as radius and $H$ the center is
tangent to the tangent lines and the fixed circles at the two tangent points.  
Using the above lemma, this corollary follows immediately. \QED

% --------------------------------------------------------------------
%                     Basics of Dupin Cyclides
% --------------------------------------------------------------------

\section{Basics of Dupin Cyclides}
\label{section:basic}

     In this section, we will review some useful properties of Dupin cyclides.
In particular, we shall establish two important results for specifying 
a cyclide correctly.  In the following, Section~\ref{section:type-cyclide} 
reviews the possible types of cyclides. 
Section~\ref{section:non-degen-cyclide-gen} focuses on non-degenerate Dupin 
cyclides, addresses some useful geometric properties,
and proves a correct specification
(Lemma~\ref{lemma:non-degen-construction}).
Section~\ref{section:para-cyclide-gen} concentrates on parabolic cyclide and
presents a specification of parabolic cyclides.
Since in this chapter only Dupin cyclide and its degenerate form are used, in
the sequel, cyclide and parabolic cyclide shall mean the quartic (i.e.,
non-degenerate) and cubic (i.e., degenerate) cyclide respectively.

% ====================================================================
%                      Types of Dupin Cyclides
% ====================================================================

\subsection{Types of Dupin Cyclides}
\label{section:type-cyclide}

     We begin with a definition of a Dupin cyclide.

\begin{definition}[Dupin cyclide]
     A Dupin cyclide is a quartic surface whose two families of lines of 
curvature are all circles.  
\end{definition}

     Equations of the Dupin cyclide, in parametric or implicit form, can be found 
in Fladt\index{Fladt} and Baur~\cite{fladt-baur:1975},\index{Baur} 
Forsyth~\cite{forsyth:1912},\index{Forsyth}
Lilienthal~\cite{lilienthal:1913},\index{Lilienthal} 
Pratt~\cite{pratt:1990}\index{Pratt} and
Woods~\cite{woods:1922}.\index{Woods}  
Kommerell~\cite{kommerell:1918}\index{Kommerell} has more detailed
discussions of materials presented in this section.
Figure~\ref{fig:cyclides} gives
all possible shapes of a cyclide.
Figure~\ref{fig:cyclides}(a) is a {\em ring} cyclide.
\index{ring cyclide}\index{cyclide,ring}
Figure~\ref{fig:cyclides}(b) and Figure~\ref{fig:cyclides}(c) are
{\em horned} cyclides,\index{horned cyclide}\index{cyclide,horned} 
usually called the {\em singly} horned
\index{singly horned cyclide}\index{cyclide,singly horned} and
{\em doubly} horned cyclides\index{doubly horned cyclide}
\index{cyclide,doubly horned} respectively.
Figure~\ref{fig:cyclides}(d) and Figure~\ref{fig:cyclides}(e)
are {\em spindle} cyclides\index{spindle cyclide}
\index{cyclide,spindle} with one and two singularities.  
\begin{figure}
\vspace{18.5cm}
\caption{Three Types of Dupin Cyclides (taken from [39] with permission)}
\label{fig:cyclides}
\end{figure}

     There are also three types of degenerate Dupin cyclides.
\index{degenerate cyclide}\index{cyclide,degenerate}  All of
them are of degree three.  Sometimes, they are called
{\em cubic} cyclides\index{cubic cyclide}\index{cyclide,cubic}
or {\em parabolic} cyclides.\index{parabolic cyclide}\index{cyclide,parabolic}
Figure~\ref{fig:den-cyclides} shows all of them.
Figure~\ref{fig:den-cyclides}(a) is a parabolic ring cyclide,
\index{parabolic ring cyclide}\index{cyclide,parabolic,ring} and
Figure~\ref{fig:den-cyclides}(b) and (c) are parabolic singly horned
\index{parabolic singly horned cyclide}\index{cyclide,parabolic,singly horned}
and doubly horned cyclides
\index{parabolic doubly horned cyclide}\index{cyclide,parabolic,doubly horned}
respectively.  Note that in a parabolic 
cyclide, one circle from each family degenerates to a straight line and
therefore, there are two straight lines on it.
Fischer~\cite{fischer:1986}\index{Fischer} and 
Hilbert\index{Hilbert} and 
Cohn-Vossen~\cite{hilbert-vossen:1952}\index{Cohn-Vossen} 
contain some photographs of plastic models, 
while Chandru,\index{Chandru} Dutta\index{Dutta} and 
Hoffmann~\cite{chandru-dutta-hoffmann:1989}\index{Hoffmann} show
some computer generated images.
\begin{figure}
\vspace{12cm}
\caption{Three Types of Parabolic Dupin Cyclides (taken from [39] with permission)}
\label{fig:den-cyclides}
\end{figure}

% ====================================================================
%                  Non-Degenerate Cyclide Generation
% ====================================================================

\subsection{Dupin Cyclide Generation}
\label{section:non-degen-cyclide-gen}

     Many methods are available for generating cyclides.  Maxwell's
\index{Maxwell} 1868
paper~\cite{maxwell:1868} provided the first geometric method, which has been
very popular in the geometric modeling community 
(Boehm~\cite{boehm:1990},\index{Boehm}
Chandru,\index{Chandru} Dutta\index{Dutta} and 
Hoffmann~\cite{chandru-dutta-hoffmann:1990}).\index{Hoffmann}
Then, about ten years later, Cayley\index{Cayley} published 
several equivalent definitions in~\cite{cayley:1873}.  One of his
methods will be used in this paper.

     Any cyclide has two planes of symmetry.  One of them intersects a ring, 
a singly horned, a doubly horned, a one singularity spindle cyclide and
a two singularity spindle cyclide in two circles as shown in 
Figure~\ref{fig:hor-circle}(a), Figure~\ref{fig:hor-circle}(b), 
Figure~\ref{fig:hor-circle}(c), Figure~\ref{fig:hor-circle}(d) and
Figure~\ref{fig:hor-circle}(e) respectively.
Note that in a one singularity spindle cyclide,
one of these two circles degenerates to a point.  This plane 
and the circles on it will be arbitrarily called the {\em horizontal symmetric
plane}\index{horizontal symmetric plane}
\index{symmetric plane,horizontal} and the 
{\em horizontal principal circles}\index{horizontal principal circle}
\index{principal circle,horizontal} respectively.
\begin{figure}
\vspace{7.5cm}
\caption{Horizontal Symmetric Plane and Horizontal Principal Circles}
\label{fig:hor-circle}
\end{figure}

     Cayley's method states that a Dupin cyclide is the envelope of a moving
sphere with center on a symmetric plane and tangent to the two principal 
circles.  Note that the horizontal symmetric plane cuts the moving sphere in 
a circle which is tangent to both principal horizontal circles.  
If $O_1$ and $r_1$ (resp., $O_2$ and $r_2$) are the center and radius of the
larger (resp., smaller) principal circle, and $O$ is the center 
of the intersection circle of the horizontal symmetric plane and a moving 
sphere, then we have
$|\overline{OO_1}|+|\overline{OO_2}|=r_1+r_2$ 
(Figure~\ref{fig:hor-circle}(a),\ref{fig:hor-circle}(b) 
and~\ref{fig:hor-circle}(c)), or 
$|\overline{OO_1}|+|\overline{OO_2}|=r_1-r_2$ 
(Figure~\ref{fig:hor-circle}(d) and~\ref{fig:hor-circle}(e)).
Therefore, the locus of the center $O$ is an ellipse, 
the {\em directrix ellipse},\index{directrix ellipse} with foci $O_1$ and 
$O_2$.  This directrix ellipse is uniquely defined by the two principal circles
and one of the relations giving the sum.  Table~\ref{tbl:characteristic-hor}
summarizes our finding.  

     The common tangent circle of the moving sphere
and the cyclide projects to the segment joining the two tangent points on the
horizontal symmetric plane.  Furthermore, with the help from 
Lemma~\ref{lemma:tangent-center}, the line containing this segment passes 
through one of the centers of similitude.  
In Figure~\ref{fig:hor-circle}, this is the thick segment.
Simple verification shows that for 
ring cyclide and horned cyclides (resp., spindle cyclides) this is the 
internal (resp., external) center of similitude.  Note that, for the case of 
one singularity spindle cyclide, the inner circle degenerates to a point 
and the internal and the external center of similitude coincide.  Let the line
through the respective center of similitude and perpendicular to the horizontal
symmetric plane be called the {\em vertical line of similitude}.
\index{vertical line of similitude}\index{line of similitude,vertical}  Thus,
any plane through this vertical line of similitude cuts the cyclide in 
two circles.  These circles are called {\em vertical circles},
\index{vertical circle} or simply $V$-circles.\index{V-circle}  
The radical axis of the two horizontal principal circles is 
called the {\em horizontal radical axis}.
\index{horizontal radical axis}\index{radical axis,horizontal} 
\begin{table}
\caption{Characteristics of the Directrix Conics of a Cyclide}
\label{tbl:characteristic-hor}
$$
\BeginTable
     \BeginFormat
     |5 l | c | c |5
     \EndFormat
     \_5
     | \Lower{\it Cyclide Type}  | \it Horizontal | \it Vertical | \\ 
     |      | $\VBar\overline{OO_1}\VBar+\VBar\overline{OO_2}\VBar$
  | $\VBar\ \VBar\overline{OO_1}\VBar-\VBar\overline{OO_2}\VBar\ \VBar$| \\ \_3
     | Ring | $r_1+r_2$ | $\VBar r_1-r_2\VBar$ | \\ \_
     | Singly Horned | \Lower{$r_1+r_2$} | $r_1$ | \\ 
     | Doubly Horned |                   | $r_1+r_2$ | \\ \_
  | Spindle with one singularity | $r_1$ | \Lower{$\VBar r_1-r_2\VBar$} | \\
     | Spindle with two singularities | $r_1-r_2$ |               | \\ \_5
\EndTable
$$
\end{table}

     For the other symmetric plane, the {\em vertical symmetric plane},
\index{vertical symmetric plane}\index{symmetric plane,vertical} which 
is perpendicular to the horizontal symmetric plane, it cuts the cyclide in two 
circles, the {\em vertical principal circles}
\index{vertical principal circle}\index{principal circle,vertical}
(Figure~\ref{fig:ver-circle}).
Note that one of the vertical circles of a singly horned cyclide degenerates
to a point.  These circles define two centers of similitude.  The one we need 
is the external (resp., internal) center of similitude for ring and spindle 
(resp., horned) cyclide as shown in Figure~\ref{fig:ver-circle}.  The line 
through this center and perpendicular 
to the vertical symmetric plane is called the {\em horizontal line of 
similitude}.\index{horizontal line of similitude}
\index{line of similitude,horizontal}  
Any plane through this line cuts the cyclide into two circles, 
the {\em horizontal circles}\index{horizontal circle} or 
$H$-circles.\index{H-circle}  The vertical principal circles 
also define a radical axis, the {\em vertical radical axis}.
\index{vertical radical axis}\index{radical axis,vertical}  Note that 
a cyclide can also be generated by a moving sphere with center $O$ on the 
vertical symmetric plane and tangent to the vertical principal circles.
The locus of $O$ is a hyperbola, the {\em directrix hyperbola}.
\index{directrix hyperbola}  The characteristics of this directrix hyperbola
are listed in Table~\ref{tbl:characteristic-hor}.  Furthermore, 
this hyperbola contains the foci of the ellipse on the horizontal symmetric 
plane mentioned earlier and vice versa.  Thus, they are perpendicular focal 
conics.
\begin{figure}
\vspace{8.5cm}
\caption{Vertical Symmetric Plane and Vertical Principal Circles}
\label{fig:ver-circle}
\end{figure}

     Note that each $H$-circle (or $V$-circle) is a line of curvature of the 
cyclide.  Therefore, the two families of lines of curvature are exactly the 
sets of all $H$-circles and the set of all $V$-circles.

     There is a very useful relationship between the $H$-circles and the
vertical radical axis (as well as between the $V$-circles and the horizontal
radical axis). 

\begin{lemma}
\label{lemma:on-radical-axis}
     All of the tangent planes of a cyclide along a $H$-circle 
(resp., $V$-circle) meet at a point on the vertical (resp., horizontal)
radical axis.
\end{lemma}
{\bf Proof:}  Consider a horizontal (resp., vertical) circle $D$.  There is a
sphere tangent to the cyclide along $D$.  Since this sphere and the cyclide are
tangent along $D$, any tangent plane of the cyclide is also a tangent plane of
the sphere.  However, all tangent planes along a circle on a sphere envelop a
cone and thus meet at a point $V$.  $V$ may be at
infinity, if $D$ is a great circle of the sphere.  The tangent lengths from
$V$ to the sphere and thus to the cyclide are all equal.  Because of symmetry,
$V$ lies on the vertical (resp., horizontal) plane and hence must be a point
on the vertical (resp., horizontal) radical axis. \QED

     Conversely, this configuration defines a cyclide uniquely.

\begin{lemma}[Dupin Cyclide Specification]
\label{lemma:non-degen-construction} \ \\
\begin{enumerate}
     \item Let $C_1$ and $C_2$ be two circles on a plane ${\cal P}$.
     \item Let $\ell$ be a line perpendicular to ${\cal P}$ and through one of
          the two centers of similitude of $C_1$ and $C_2$.  
     \item Let $D$ be a circle, with center in ${\cal P}$ and lying in a plane
          through $\ell$, meeting $C_1$ and $C_2$ at $E_1$ and $E_2$, 
          respectively, such that the tangents of $C_i$ at $E_i$ ($i=1,2$) 
          meet at a point on the radical axis of $C_1$ and $C_2$.  
\end{enumerate}
Then, there exists a unique cyclide containing $C_1,C_2$ and $D$.  This cyclide
is denoted by ${\cal Z}(C_1,C_2,D,\ell)$.  Moreover, ${\cal P}$ is 
a symmetric plane in which $C_1$ and $C_2$ are principal circles and $\ell$ is
a line of similitude.
\end{lemma}
{\bf Proof:}  Let $V$ be the intersection point of the two tangents at $E_1$
and $E_2$ (Figure~\ref{fig:specification-nondegen}).  
If $V$ is at infinity, then the tangent lines at $E_1$ and $E_2$ are parallel 
to the radical axis of $C_1$ and $C_2$, and $E_1$ and $E_2$ are points on the
line of centers.  Therefore, by taking ${\cal P}$ and the plane containing 
$D$ as the two symmetric planes, a cyclide is determined uniquely.  Thus, in
the following, we will assume that $V$ is a finite point.

     Since $V$ lies on the radical axis, we have
$|\overline{VE_1}|=|\overline{VE_2}|$.  Thus, there exists a circle $C$
tangent to $\stackrel{\longleftrightarrow}{VE_i}$ at $E_i$.
Let $O$ and $r$ be the center and radius of circle $C$ respectively.
Let circle $C_i$ have center $O_i$ and radius $r_i$, $r_1>r_2$.  

     Consider the relative position of circle $C$ with respect to circles
$C_1$ and $C_2$.  We have the following cases:
\begin{enumerate}
     \item $C$ is not contained in $C_1$ or $C_2$:
     \begin{itemize}
          \item If $C$ lies outside of (resp., contains) both $C_1$ and $C_2$,
               $|\ |\overline{OO_1}|-|\overline{OO_2}|\ |$ is $r_1-r_2$
               (resp., $-(r_1-r_2)$).
          \item If $C$ contains exactly one of $C_1$ and $C_2$, we have
               $|\ |\overline{OO_1}|-|\overline{OO_2}|\ |=r_1+r_2$.
     \end{itemize}
          Hence, $O$ lies on a hyperbola with foci $O_1$ and $O_2$.  From
          Table~\ref{tbl:characteristic-hor}, this hyperbola is the
          directrix hyperbola on the vertical symmetric plane.  Based on
          $C_1$ and $C_2$ being disjoint, intersecting, and tangent, this
          hyperbola is the directrix hyperbola of a ring, two singularity
          spindle, and one singularity spindle cyclide respectively
          (Figure~\ref{fig:ver-circle}).
     \item $C$ is contained in exactly one of $C_1$ and $C_2$:\\
          If the other circle lies outside of $C$, we have
          $|\overline{OO_1}|+|\overline{OO_2}|=r_1+r_2$; otherwise,
          $|\overline{OO_1}|+|\overline{OO_2}|=r_1-r_2$.
          Therefore, $O$ lies on an ellipse with foci $O_1$ and $O_2$.
          From Table~\ref{tbl:characteristic-hor}, this ellipse is the
          directrix ellipse on the horizontal symmetric plane.
          $C_1$ and $C_2$ being disjoint, intersecting, and tangent, this
          ellipse is the directrix ellipse of a ring, doubly horned and
          singly horned cyclide respectively
          (Figure~\ref{fig:hor-circle}).
     \item $C$ is contained in both circles:\\
          In this case, we have 
          $|\ |\overline{OO_1}|-|\overline{OO_2}|\ | = |r_1-r_2|$ and hence,
          $O$ lies on the directrix hyperbola of a two singularity spindle
          cyclide.
\end{enumerate}
>From the above discussion, $O$ lies on a directrix conic defined by $C_1$ and
$C_2$ and thus, the sphere generated by revolving the circle $C$ about
$\stackrel{\longleftrightarrow}{VO}$ contains $D$
and is tangent to $C_1$ and $C_2$ at $E_1$ and $E_2$.  
Hence, this is a generating sphere of the cyclide defined by $C_1$, $C_2$ and
the directrix conic.  That is, there exists a
cyclide containing $C_1,C_2$ and $D$.  The uniqueness is easy to see. \QED
\begin{figure}
\vspace{4.5cm}
\caption{A Specification of a Dupin Cyclide}
\label{fig:specification-nondegen}
\end{figure}

     The above lemma presents a specification of a cyclide.  Simply speaking,
in order to specify a cyclide uniquely, we need two circles on a plane 
${\cal P}$ and a third circle, with center in ${\cal P}$ and lying in a plane 
perpendicular to ${\cal P}$, satisfying some additional conditions: 
(1) the third circle intersects each of
the first two circles at a point, (2) the line joining the intersection points
passes through a center of similitude of the first two circles, and
(3) the intersection of the tangents at the intersection points lies on the
radical axis of the first circles.  
In Section~\ref{section:cyclide-non-degen}, these conditions are used to show
the existence of a specific cyclide.

\begin{remark} \rm
     In the construction of blending cyclides for two cones, 
Boehm~\cite{boehm:1990} and Pratt~\cite{pratt:1990} assume 
the last two conditions implicitly.  Although as we will show in
Section~\ref{section:cyclide-non-degen} that, without
noticing conditions (2) or (3), a cyclide usually cannot be specified 
correctly.  For example, in Figure~\ref{fig:wrong-example}, we have two 
vertical circles $C_1$ and $C_2$ on a plane ${\cal P}$.  
Let $D$ be a third circle, with center in ${\cal P}$ and lying in a plane
perpendicular to ${\cal P}$, meeting $C_i$ at $E_i$ ($i=1,2$).
Suppose $\stackrel{\longleftrightarrow}{E_1E_2}$ contains the external center
of similitude of $C_1$ and $C_2$.  Then, this configuration meets the first two
conditions, but not the third one.  $D$ cannot be a member of the family of 
$H$-circles, since the tangents at the intersection points do not meet at a 
point of the radical axis of $C_1$ and $C_2$.  $\Box$
\end{remark}
\begin{figure}
\vspace{5.5cm}
\caption{A Specification That Does Not Meet Condition (3)}
\label{fig:wrong-example}
\end{figure}


% ====================================================================
%                Parabolic Cyclide Generation
% ====================================================================

\subsection{Parabolic Cyclide Generation}
\label{section:para-cyclide-gen}

     This section presents some properties of parabolic cyclides 
related to their generation.  Our discussion will be parallel to the 
last section; however, it is simpler for parabolic case.

     As shown in Figure~\ref{fig:den-cyclides}, each parabolic cyclide 
(in normal form) has two planes of symmetry, namely the $xz$-plane and the
$yz$-plane.  In the following, we shall arbitrarily call the $yz$-plane the 
{\em horizontal symmetric plane},\index{horizontal symmetric plane}
\index{symmetric plane,horizontal} and the $xz$-plane the 
{\em vertical symmetric plane}.\index{vertical symmetric plane}
\index{symmetric plane,vertical}  
Each plane of symmetry intersects the parabolic cyclide in 
a line, the {\em horizontal}\index{horizontal principal line}
\index{principal line,horizontal} or 
{\em vertical principal line},\index{vertical principal line}
\index{principal line,vertical} and a circle, 
the {\em horizontal}\index{horizontal principal circle}
\index{principal circle,horizontal} or 
{\em vertical principal circle}.\index{vertical principal circle}
\index{principal circle,vertical}
Figure~\ref{fig:para-line-circle} shows the principal line and principal
circle on horizontal and vertical symmetric planes as well as their relations.
Note that for the singly horned case (Figure~\ref{fig:para-line-circle}(b)), 
the vertical principal circle degenerates to a point.  
\begin{figure}
\vspace{13cm}
\caption{Principal Line and Circle of Parabolic Cyclides}
\label{fig:para-line-circle}
\end{figure}

     Cayley's method also works for this case.  In fact, the horizontal and 
vertical directrix conics are parabolas with focal point the center of the 
principal circle and directrix a line parallel to the corresponding principal 
line as shown in Figure~\ref{fig:para-line-circle}.  It is not difficult to 
verify that these directrix parabolas are perpendicular focal conics.

     Consider a symmetric plane.  It intersects the cyclide in a principal 
line and a principal circle, and a moving sphere generating the cyclide in a 
circle tangent to the former two.  The other symmetric plane intersects the
principal circle in two diametrical point.  It is not difficult to verify 
that the line joining the two tangent points passes through a fixed diametrical
points mentioned above.  In fact, this fixed point is exactly the intersection
point of the other principal line and this symmetric plane.  Therefore, the
principal lines serve the same purpose as the lines of similitude in the
non-degenerate case.  Thus, we also call the horizontal and vertical principal
lines the {\em horizontal}\index{horizontal line of similitude}
\index{line of similitude,horizontal} and 
{\em vertical lines of similitude}.\index{vertical line of similitude}
\index{line of similitude,vertical}  Any plane containing the horizontal
(resp., vertical) line of similitude intersects the cyclide in a line (
the horizontal (resp., vertical) line of similitude itself), and a circle,
the {\em horizontal}\index{horizontal circle} or $H$-circle\index{H-circle}
(resp., {\em vertical}\index{vertical circle} or $V$-circle\index{V-circle}).

     In the parabolic cyclide case, the radical axis on each symmetric plane
does not exist.  However, a result similar to Lemma~\ref{lemma:on-radical-axis}
is possible.  

\begin{lemma}
\label{lemma:on-principal-line}
     All of the tangent planes of a cyclide along a $H$-circle (resp., 
$V$-circle) meet at a point on the vertical (resp., horizontal) principal line.
\end{lemma}
{\bf Proof:} Consider a sphere tangent to a parabolic cyclide along a circle.
Because one of the two principal lines is also tangent to the sphere and 
because all tangent planes of a sphere along a circle envelop a cone, the cone
vertex lies on the principal line that is tangent to the sphere. \QED

     Using the above information and parallel to the specification of a
non-degenerate cyclide, we have a similar specification result for parabolic 
cyclides.

\begin{lemma}[Parabolic Cyclide Specification]
\label{lemma:para-construction} \ \\
\begin{enumerate}
     \item Let $\ell$ and $C$ be a line and a circle on a plane ${\cal P}$.  
     \item Let $P\not\in\ell$ be an intersection point of $C$ and the 
          line through the center of $C$ and perpendicular to $\ell$.  
\end{enumerate}
Then, there exists a unique parabolic cyclide containing $\ell$ and $C$.
This cyclide is denoted by ${\cal Z}_P(\ell,C,P)$.
Moreover, ${\cal P}$ is a symmetric plane in which $\ell$ and $C$ are the 
principal line and principal circle respectively.  The other principal line
is the line perpendicular to ${\cal P}$ and through $P$.
\end{lemma}
{\bf Proof:}  Through $P$ construct a line meeting $\ell$ and $C$ at $A$ and 
$B$ respectively (Figure~\ref{fig:para-const}).  
Let $D$ be the circle with diameter
$\overline{AB}$ and perpendicular to ${\cal P}$.
Let the tangent at $B$ meet $\ell$ at $V$.  It is not
difficult to show that in $\bigtriangleup VAB$, we have
$|\overline{VA}|=|\overline{VB}|$.  Hence, there exists a circle $E$, with
center $T$,  tangent to $\ell$ and $C$ at $A$ and $B$ respectively.  If $E$ is
revolved about $\stackrel{\longleftrightarrow}{VT}$, we have a sphere 
${\cal S}$ containing $D$ and tangent to $\ell$ and $C$ at $A$ and $B$.

     Let $r$ and $O$ be the radius and center of $C$.  Let $d$ be the line
parallel to $\ell$ at distance $r$, lying in the halfplane that contains $P$
if $P\in\overline{QR}$ and in the halfplane that does not contain $P$ if
$P\not\in\overline{QR}$, where the line $\stackrel{\longleftrightarrow}{OP}$
intersects $C$ and $\ell$ at $P$, $Q$, and $R$, respectively.  Then $T$ lies 
on the parabola with focus $O$ and directrix $d$, since $T$ is equidistant
from $O$ and $d$.  This shows that $S$ generates a parabolic cyclide
containing $\ell$, $C$ and $D$.  The proof of uniqueness is obvious and is
omitted.  \QED
\begin{figure}
\vspace{5.5cm}
\caption{A Specification of a Parabolic Cyclide}
\label{fig:para-const}
\end{figure}

% --------------------------------------------------------------------
%                      Interpolating Cyclides
% --------------------------------------------------------------------

\section{Blending Cyclides}
\label{section:interpolating}

\begin{definition}[Blending Surface]
     A {\em blending surface}\index{blending surface} of two surfaces
${\cal S}_1$ and ${\cal S}_2$ is a surface tangent to ${\cal S}_i$ along
$C_i$, where $C_i$ is a {\em clipping curve}\index{clipping curve} on
${\cal S}_i$.   $C_i$ is usually specified as the intersection curve of 
${\cal S}_i$ and some {\em clipping surfaces}.\index{clipping surface}
\end{definition}

     We will show that the clipping curves for the blend of two quadrics by a
cyclide must be circles.  For a proof of the following theorem, see 
Humbert~\cite{humbert:1885} or Jessop~\cite{jessop:1916}.
\index{Humbert}\index{Jessop}

\begin{theorem}[Humbert 1885]
\label{thm:humbert-thm}
\index{Humbert theorem}
     The locus of the tangent points with a cyclide of a plane, which is 
tangent to the cyclide and a fixed quadric, is a line of curvature of the
cyclide.
\end{theorem}

\begin{remark} \rm
     This theorem applies to a more general class of cyclides: all
quartic surfaces with equation $(x^2+y^2+z^2)^2+{\cal Q}=0$, 
where ${\cal Q}$ is a quadric equation.  $\Box$
\end{remark}

\begin{corollary}
\label{cor:humbert}
     If a quadric and a Dupin cyclide are tangent along a curve, this
curve must be a circle.  In particular, this curve is an $H$-circle or 
$V$-circle of the cyclide.
\end{corollary}
{\bf Proof:}  If the quadric and cyclide are tangent along a curve, the tangent
planes of the surfaces certainly coincide along this curve, and the curve must
be a line of curvature.  For a Dupin cyclide, as mentioned in the last section,
the lines of curvature are exactly the $H$-circles or $V$-circles. \QED

\begin{remark} \rm
     The $H$-circles and $V$-circles are not the only circles on a cyclide,
but they account for all of the lines of curvatures.   In fact, there are
four circles through each point of a ring cyclide, two of which are not 
$H$- or $V$- circles.  It is well-known that for each point on a non-degenerate
torus, there passes exactly four circles.  Among these four circles, one is 
from the $H$-family, one is from the $V$-family, and the other two are
Villarceau circles (discovered by Yvon Villarceau~\cite{villarceau:1848}).
\index{Villarceau}\index{Villarceau circles}
If we choose a vertical plane of symmetry of the torus, the plane
through one of the two internal tangents of the two vertical principal circles
and perpendicular to the vertical symmetric plane intersects the torus in two
circles and these are the Villarceau circles.  Because a ring cyclide can be
inverted to a torus (Section~\ref{section:prelim}) and because inversion
preserves circles, there are also four circles through each point of a ring
cyclide, two of which are not $H$- or $V$- circles.  However, the 
Villarceau circles are not lines of curvature.  For a proof and more general
results, see Blum~\cite{blum:1980}.\index{Blum}   $\Box$
\end{remark}

     If a cyclide blends two quadrics, then the two curves of tangency
must be $H$- or $V$-circles.  We can strengthen this: the two curves of
tangency must be from the same family of circles on the cyclide (i.e., both
$H$-circles or both $V$-circles).  Based on these discussions, we make the 
following definition of blending cyclides.

\begin{definition}[Blending Cyclides]
\label{def:interpolating-cyclides}
\index{blending cyclide}
     A blending cyclide of two axial natural quadrics ${\cal Q}_1$ and 
${\cal Q}_2$ is a cyclide ${\cal Z}$ that is tangent to ${\cal Q}_i$ along a 
circle, where the tangent circles are both $H$-circles or both $V$-circles of 
${\cal Z}$.  
\end{definition}

     The choice of a cyclide as blending surface also demands that the two 
axial natural quadrics to be blended must be in a special relative position.
The following lemma describes a restriction on the axes.

\begin{lemma}
\label{lemma:coplanar}
     Two axial natural quadrics can be blended by a cyclide only if their axes 
are coplanar.
\end{lemma}
{\bf Proof:}  Suppose that two axial natural quadrics can be blended by a
cyclide.  Consider the case of two cones first.  The cones are tangent to the
cyclide along two circles $C_1$ and $C_2$ from the same family.  We can assume
that both of these are $H$-circles.  From Lemma~\ref{lemma:on-radical-axis} 
(or Lemma~\ref{lemma:on-principal-line} if the cyclide is parabolic),
we know that the cone's vertex lies on a vertical radical axis 
(or a vertical principal line) of the cyclide.  Therefore, the cone vertices 
lie in the vertical symmetric plane.  The vertical symmetric plane cuts
each $H$-circle including $C_1$ and $C_2$ symmetrically.  
Thus, the vertical symmetric plane contains the centers
of $C_1$ and $C_2$.  Because the axis of a cone is the line joining the cone's
vertex and the center of any circle on the surfaces, and because the centers
and vertices lie in the vertical symmetric plane, so do the cones' axes.

     If one of the natural quadrics is a cylinder, its vertex is a point at
infinity on the radical axis (or the principal line) and the same argument
works.  The case of two cylinders is trivial.  \QED

     Therefore, the axial plane is well defined.  We shall reduce most of our
reasoning and construction to this plane.

     Two results can be extracted from the proof of the lemma.

\begin{corollary}
\label{cor:on-radical-axis}
     If two axial natural quadrics can be blended by a cyclide, the plane of
their axes is a symmetric plane of the cyclide.  In the case of cones, the
cone vertices also lie on the radical axis in this plane.
\end{corollary}

     If the blending cyclide is parabolic, we have the following stronger 
result.

\begin{corollary}
\label{cor:para->common-line}
     Two axial natural quadrics can be blended by a parabolic cyclide only if
all three surfaces share a common line.  Therefore, the axial natural quadrics
must have planar intersection.
\end{corollary}
{\bf Proof:} By Lemma~\ref{lemma:coplanar}, the axes are coplanar and all
surfaces are symmetric about the axial plane.  This plane intersects the 
parabolic cyclide in a line $\ell$ and a circle $C$.  The tangent circle of 
the cyclide and one of the surfaces, ${\cal C}_1$, meets $\ell$ at a point 
(Figure~\ref{fig:para->plane}), say $P$.  Since the cyclide and ${\cal C}_1$ 
are tangent at $P$, they share the same tangent plane at $P$.  
This tangent plane contains $\ell$ and the ruling through $P$ of ${\cal C}_1$
(and because of symmetry, it is also perpendicular to the axial plane).  
Therefore, $\ell$ must be a common line of the cyclide and ${\cal C}_1$.  
The same argument shows that $\ell$ is also a common line of the cyclide 
and ${\cal C}_2$, the other axial natural quadric.  Hence, the axial natural
quadrics share a common line.  Because the tangent plane along this line is
perpendicular to the axial plane, this common line becomes a double line
in the intersection of the given axial natural quadrics and hence,
they have planar intersection.  \QED
\begin{figure}
\vspace{4.5cm}
\caption{A Common Line on Both Surfaces}
\label{fig:para->plane}
\end{figure}

     This corollary shows that if two axial natural quadrics have a blending
parabolic cyclide, their intersection contains a double line.  Therefore, the 
use of parabolic cyclide is limited.

% --------------------------------------------------------------------
%                        Planar Intersection
% --------------------------------------------------------------------

\section{The Existence of Planar Intersection}
\label{section:planar}

     In this section, we strengthen the restriction on the relative position
of two axial natural quadrics that can be blended by a cyclide: they not only 
must have coplanar axes, they must have planar intersection.  This restriction
finally captures the blending of axial natural quadrics by cyclides.
That is, this restriction is also sufficient: except for a very special case
(two cones with parallel axes and a double line in their intersection),
a pair of natural quadrics with planar intersection can be blended by a
cyclide.  The sufficiency is proven in the next section.

     Suppose two axial natural quadrics are blended by a cyclide.  From
Lemma~\ref{lemma:coplanar}, their axes are coplanar.  For two axial natural
quadrics with coplanar axes, some conditions immediately imply planar
intersection and can thus be eliminated from consideration:
\begin{enumerate}
     \item If they have common vertices:  the axial natural quadrics
          have linear intersection.
     \item If they have a common skeletal line:  it is a double line in their 
          intersection and the surfaces have planar intersection.
     \item If the skeletal lines in a skeletal pair are parallel to the lines 
          in the other pair; by Theorem~\ref{theorem:parallel-axes}
          (see also Theorem~\ref{theorem:parallel-axes-geo}), the
          quadrics have conic intersection.
\end{enumerate}
>From Corollary~\ref{cor:para->common-line}, we may further assume that the
blending cyclide is not parabolic.  Therefore, in the rest of this section, 
we will assume a non-parabolic blending cyclide, distinct vertices, distinct 
skeletal lines, and at most one line from each skeletal pair are parallel.

     In order to prove that two cones with a blending non-parabolic cyclide
have planar intersection, our strategy consists of the following steps:
(1) prove that on the axial plane, the skeletal quadrilateral is well-defined
(Lemma~\ref{lemma:not-on-skeletal-line}),
(2) this skeletal quadrilateral has a diagonal with two finite diagonal points
(Lemma~\ref{lemma:finite-d-points}),
(3) the diagonal plane through this diagonal intersects
both cones in the same type of conic (Lemma~\ref{lemma-1}),
(4) these conics have the same focal length (Lemma~\ref{lemma-2}), 
and finally 
(5) using the focal length characterization theorem
(Theorem~\ref{thm:focal-length}), the cones have planar intersection
(Lemma~\ref{lemma:cyclide->planar}).  The cone-cylinder and cylinder-cylinder
cases are treated in Lemma~\ref{lemma:cyclide-planar-for-cys}.

\begin{lemma}
\label{lemma:not-on-skeletal-line}
     Under the assumptions above, no vertex can lie on the other surface's 
skeletal lines.
\end{lemma}
{\bf Proof:}  If the surfaces are cylinders, since they are tangent to 
principal circles from the same family, their axes are parallel and thus, meet
at the same point at infinity.  This violates the distinct vertices assumption.

     Consider the cone-cone case.  Let cone ${\cal C}_i$ have vertex $V_i$,
$i=1,2$.  If $V_2$ lies on a skeletal line of ${\cal C}_1$, then
$\stackrel{\longleftrightarrow}{V_1V_2}$, the radical axis, is identical
to one of the skeletal lines of ${\cal C}_1$, and therefore, is tangent to
one of the two principal circles on the axial plane.  In a coaxal circles 
system, if the radical axis is tangent to one circle, all circles in the
system must also be
tangent to the radical axis at the same point.  This implies that the two
principal circles are tangent to each other and the blending cyclide is an
one singularity spindle cyclide (the principal circles are vertical,
Figure~\ref{fig:not-on-other-skeletal-line}(a)) or a singly horned cyclide 
(the principal circles are horizontal, 
Figure~\ref{fig:not-on-other-skeletal-line}(b)).  However, this is impossible
since the cyclide is not tangent to ${\cal C}_1$ along an $H$- or $V$- 
circle.
\begin{figure}
\vspace{6cm}
\caption{Vertex Cannot Lie on the Other Surface's Skeletal Lines}
\label{fig:not-on-other-skeletal-line}
\end{figure}

     Next, consider the cone-cylinder case.  Since cylinder can only be tangent
to principal circles, its skeletal lines are perpendicular to the line of
centers of the principal circles in the axial plane.  If the cylinder's vertex,
which is a point at infinity, lies on a skeletal line of the cone, this 
skeletal line is parallel to the cylinder's skeletal lines and hence,
is perpendicular to the line of centers.  Since $V$, the cone's vertex, lies on
the radical axis of the principal circles in the axial plane, the cone's
skeletal line mention above must be the radical axis. Since this skeletal line
is tangent to one of the principal circles, the radical axis is also tangent
to a that circle (Figure~\ref{fig:not-on-other-skeletal-line}(c)).  
Therefore, the principal circles are tangent to each other.
With the same argument as above, the cyclide does not tangent to the cone
along an $H$- or $V$- circle.  The same reasoning works for the case of $V$ 
lying on a skeletal line of the cylinder.  \QED

     The above lemma shows that the skeletal quadrilateral is well-defined.
Furthermore, from our assumption, this skeletal quadrilateral has at most one
vertex at infinity.  Therefore, the following lemma is obvious.

\begin{lemma}
\label{lemma:finite-d-points}
     At least one of the two diagonals of the skeletal quadrilateral has two 
finite diagonal points.
\end{lemma}

     In the following, $V_1$ and $V_2$ denote the vertices of the
axial natural quadrics, and $\stackrel{\longleftrightarrow}{RS}$ is a diagonal
of the skeletal quadrilateral, where $R$ and $S$ are finite diagonal points.

\begin{lemma}
\label{lemma-1}
     For two cones, the diagonal plane through 
$\stackrel{\longleftrightarrow}{RS}$
intersects the surfaces in the same type of conics.
\end{lemma}
{\bf Proof:}  Consider two cones.  It is not difficult to see that if the 
axial plane is the
vertical (resp., horizontal) symmetric plane, the radical axis lies in the
interiors (resp., exteriors) of the cone (Figure~\ref{fig:int-ext}).  
Therefore, the radical axis lies in both interiors or both exteriors of the 
cones.
\begin{figure}
\vspace{7.5cm}
\caption{The Radical Axis Lies in Both Interiors or Both Exteriors}
\label{fig:int-ext}
\end{figure}

     On the axial plane, there are only six different ways to place the radical
axis and  cone angles (Figure~\ref{fig:configs}; the radical axis is not
shown  since it must be $\stackrel{\longleftrightarrow}{V_1V_2}$ by
Corollary~\ref{cor:on-radical-axis}).  The supporting segment on each diagonal
is displayed in Figure~\ref{fig:configs}.  It is clear that the intersection
types of the diagonal planes with the two cones are identical.  A similar 
argument works for the cone and cylinder case.  \QED
\begin{figure}
\vspace{6cm}
\caption{All Possible Placements of $\stackrel{\longleftrightarrow}{V_1V_2}$ and Cone Angles}
\label{fig:configs}
\end{figure}

\begin{lemma}
\label{lemma-2}
     For two cones, the focal lengths of the intersection conics in the 
diagonal plane through $\stackrel{\longleftrightarrow}{RS}$ are equal.
\end{lemma}
{\bf Proof:}  Since diagonal points $R$ and $S$ are finite, the intersection 
conics in the diagonal plane through $\stackrel{\longleftrightarrow}{RS}$ must
be an ellipse or a hyperbola with focal length
$f_E=\frac{1}{2}|\ |\overline{V_iR}|-|\overline{V_iS}|\ |$ or
$f_H=\frac{1}{2}(|\overline{V_iR}|+|\overline{V_iS}|)$ respectively
(Lemma~\ref{lemma:f-central}).  

     Let $\stackrel{\longleftrightarrow}{V_1R}$ and
$\stackrel{\longleftrightarrow}{V_2R}$ (resp.,
$\stackrel{\longleftrightarrow}{V_1S}$ and
$\stackrel{\longleftrightarrow}{V_2S}$) be tangent to $C_1$ (resp., $C_2$) at
$A$ and $C$ (resp., $B$ and $D$) respectively, where $C_1$ and $C_2$ are the 
principal circles in the axial plane (Figure~\ref{fig:Q-RS}).  Our lemma
follows immediately from the following identities:
\[ |\overline{V_1A}|=|\overline{V_1B}|,\ \ \ \ \ 
   |\overline{V_2C}|=|\overline{V_2D}|,\ \ \ \ \ 
   |\overline{RA}|=|\overline{RC}|\ \ \mbox{and}\ \  
   |\overline{SB}|=|\overline{SD}|. \QED \]

\begin{figure}
\vspace{6.5cm}
\caption{The Skeletal Quadrilateral $V_1RV_2S$}
\label{fig:Q-RS}
\end{figure}

     Finally, we can prove one of our main result of this section.

\begin{lemma}
\label{lemma:cyclide->planar}
     If two cones have a blending cyclide, they have planar intersection.
\end{lemma}
{\bf Proof:}  If the blending cyclide is parabolic, 
Corollary~\ref{cor:para->common-line} delivers the desired result. 
Suppose the cyclide is not parabolic.
>From Lemma~\ref{lemma:coplanar}, the axes must be coplanar and
lie in one of the symmetric planes.  On the axial plane, we have a skeletal
quadrilateral $V_1RV_2S$ as shown in Figure~\ref{fig:Q-RS}, where $R$ and $S$
are finite diagonal points.  Since the diagonal plane through 
$\stackrel{\longleftrightarrow}{RS}$ intersects both cones in the same type of
conics (Lemma~\ref{lemma-1}) with equal focal lengths
(Lemma~\ref{lemma-2}), the conics must be identical
and hence, the cones have planar intersection
(Theorem~\ref{thm:focal-length}).  \QED

     The cone-cylinder and cylinder-cylinder cases are simpler.

\begin{lemma}
\label{lemma:cyclide-planar-for-cys}
     If a cone and a cylinder or two cylinders have a blending cyclide,
they have planar intersection.
\end{lemma}
{\bf Proof:}  Consider two cylinders.  A cylinder can only be tangent to the 
cyclide along one of the four principal circles.  Therefore, if two cylinders 
have a blending cyclide, their axes must be parallel and hence have planar
intersection.

     For the cone-cylinder case, based on how the cyclide is tangent to the
cylinder and the type of the cyclide, we have two cases to consider
(Figure~\ref{fig:cone-cy-cyclide}).  If the cylinder is tangent to the cyclide
along a horizontal (resp., vertical) principal circle, on the axial plane, we 
have Figure~\ref{fig:cone-cy-cyclide}(a) 
(resp., Figure~\ref{fig:cone-cy-cyclide}(b)), where $V$ is the cone's
vertex.  In the figure, $A$ and $B$ (resp., $C$ and $D$) are the tangent 
points of the cone's (resp., cylinder's) skeletal lines and the principal
circles.  We assume that $A$ and $C$ lie on one circle, while $B$ and $D$ lie
on the other.  Let $\stackrel{\longleftrightarrow}{RS}$ be a diagonal, where 
$R$ and $S$ are finite diagonal points.  
We shall present a proof for the case shown in 
Figure~\ref{fig:cone-cy-cyclide}(a) only; other cases can be proven the same
way.
\begin{figure}
\vspace{6cm}
\caption{The Intersection of a Cone and a Cylinder}
\label{fig:cone-cy-cyclide}
\end{figure}

     It is obvious that the diagonal plane through 
$\stackrel{\longleftrightarrow}{RS}$ intersects both surfaces in ellipses.
Therefore, in order to establish planar intersection of the cone and cylinder,
what remains is to show that these ellipses have equal focal lengths
(Theorem~\ref{thm:focal-length}).   Since their axes are identical, it 
suffices to show that the two ellipses have the same focus point.  
Let $F$ be one of the foci of the
intersection ellipse of the cylinder and the diagonal plane through
$\stackrel{\longleftrightarrow}{RS}$.  Then, there exists a Dandelin sphere
inscribed in the cylinder and tangent to the diagonal plane at $F$.
Let this Dandelin sphere be tangent to the two skeletal lines of the cylinder
at $C^\prime$ and $D^\prime$.  
Thus, $|\overline{CC^\prime}|=|\overline{DD^\prime}|$.
Therefore, we have the following:
\begin{eqnarray*}
     |\ |\overline{VR}|-|\overline{VS}|\ | 
          &=& |(|\overline{VA}|+|\overline{AR}|) -
               (|\overline{VB}|+|\overline{BS}|)| \\
          &=& |\ |\overline{AR}|-|\overline{BS}|\ |
           =  |\ |\overline{RC}|-|\overline{SD}|\ | \\
          &=& |(|\overline{RC}|+|\overline{CC^\prime}|)-
               (|\overline{SD}|+|\overline{DD^\prime}|)| \\
          &=& |\ |\overline{RC^\prime}|-|\overline{SD^\prime}|\ |
           =  |\ |\overline{RF}|-|\overline{SF}|\ |.
\end{eqnarray*}

     Since the focal length of the intersection ellipse of the diagonal plane
and the cone is $\frac{1}{2}|\ |\overline{VR}|-|\overline{VS}|\ |$
(Lemma~\ref{lemma:f-central}), this implies that $F$ is a focus of this 
ellipse.  Thus, we have planar intersection. \QED

     With the above two lemmas, we have established the following theorem.

\begin{theorem}
\label{thm:cyclide->planar}
     If two axial natural quadrics have a blending cyclide, they must
have planar intersection.
\end{theorem}

% --------------------------------------------------------------------
%                The Existence of a Blending Cyclide
%                       (Non-Degenerate Case)
% --------------------------------------------------------------------

\section{The Existence of a Blending Cyclide}
\label{section:cyclide-non-degen}

     In this section, our goal is to show that the restriction of the previous
section is sufficient.  That is, we will prove that (except for a very special
case, two cones with parallel axes and a double line in their intersection),
if two axial natural quadrics have planar intersection then there exists 
a blending Dupin cyclide.  Since our proof technique is by direct construction,
an algorithm for the creation of the blending cyclide is immediately available.
We will show that in general there are two families of blending cyclides, not
just one.

     We will deal with the three cases of planar intersection separately:
a planar intersection of two conics in this section, a line and a conic in
Section~\ref{section:line-conic}, and a linear intersection in which all 
intersection curves are lines in Section~\ref{section:cyc-linear}.

     Suppose that two axial natural quadrics have planar intersection.  In
general, this intersection will consist of two conics (including possibly an
infinite one): in particular, if the four skeletal lines in the axial plane
are distinct, the intersection will be two conics.  This is the 
{\em non-degenerate conic intersection}
\index{non-degenerate conic intersection}
\index{conic intersection,non-degenerate}
(Section~\ref{section:NQ-NQ-conic}).
We shall deal with this case in this section.  In the following subsection, 
we also assume that the quadrics' axes are intersecting.  The parallel axes 
case is dealt with in Section~\ref{section:parallel-axes}.

% ====================================================================
%                        Intersecting Axes
% ====================================================================

\subsection{The Intersecting Axes Case}
\label{section:intersecting-axes}

     We first fix some notation.
Let ${\cal Q}_1$ and ${\cal Q}_2$ be two axial natural quadrics with
vertices $V_1$ and $V_2$, intersecting axes $\ell_1$ and $\ell_2$, and
planar intersection which is two conics.  Let $O=\ell_1\cap\ell_2$.  
Since the surfaces have conic intersection, they must 
have a common inscribed sphere 
(Theorem~\ref{theorem:intersecting} and 
Theorem~\ref{theorem:intersecting-axes-characterization}).
The axial plane intersects the surfaces in 
two pairs of skeletal lines and the sphere in a circle.
Let $\stackrel{\longleftrightarrow}{RS}$ be a diagonal of the skeletal 
quadrilateral with $R$ and $S$ being diagonal points.  Note that one of
$R$ and $S$ may be at infinity and in this case the diagonal is parallel to
one skeletal line from each skeletal pair.  Let the skeletal lines
$\stackrel{\longleftrightarrow}{V_1R}$, $\stackrel{\longleftrightarrow}{V_1S}$,
$\stackrel{\longleftrightarrow}{V_2R}$ and 
$\stackrel{\longleftrightarrow}{V_2S}$ be tangent to the circle at $A, B, C$ 
and $D$ respectively (Figure~\ref{fig:int-config}).  By Brianchon's theorem, 
$\stackrel{\longleftrightarrow}{AB}$, $\stackrel{\longleftrightarrow}{CD}$,
$\stackrel{\longleftrightarrow}{RS}$ and
$\stackrel{\longleftrightarrow}{R^\prime S^\prime}$ meet at a point, say $X$,
where $R^\prime$ and $S^\prime$ are the other two diagonal points.
This is the basic configuration on the axial plane for our construction.
\begin{figure}
\vspace{6.5cm}
\caption{Basic Configuration in the Axial Plane--Intersecting Axes Case}
\label{fig:int-config}
\end{figure}

     We next present a construction of the blending cyclide.  There is a 
degree of freedom in the choice of blending cyclide: any point on either
diagonal leads to a blending cyclide.  Our construction algorithm consists 
of three simple steps as follows:
\begin{enumerate}
     \item Let $X^\prime$ be any point on $\stackrel{\longleftrightarrow}{RS}$.
     \item Through $X^\prime$ construct a line perpendicular to $\ell_1$ 
          meeting $\stackrel{\longleftrightarrow}{V_1A}$ and
          $\stackrel{\longleftrightarrow}{V_1B}$ at $A^\prime$ and $B^\prime$
          respectively.
     \item Through $X^\prime$ construct a line perpendicular to $\ell_2$ 
          meeting $\stackrel{\longleftrightarrow}{V_2C}$ and
          $\stackrel{\longleftrightarrow}{V_2D}$ at $C^\prime$ and $D^\prime$
          respectively.
\end{enumerate}

     If construction is of parabolic cyclide, then $A$ and $C$ define the 
principal line instead of a principal circle.

     In the rest of this section, we shall show that there exists two circles, 
one of them tangent to $\stackrel{\longleftrightarrow}{V_1R}$ and
$\stackrel{\longleftrightarrow}{V_2R}$ at $A^\prime$ and $C^\prime$, and the
other tangent to $\stackrel{\longleftrightarrow}{V_1S}$ and
$\stackrel{\longleftrightarrow}{V_2S}$ at $B^\prime$ and $D^\prime$, which
along with the circle defined by the intersection of 
the plane through $\stackrel{\longleftrightarrow}{A^\prime B^\prime}$
(resp., $\stackrel{\longleftrightarrow}{C^\prime D^\prime}$) and
perpendicular to the axial plane with the first (resp., the second) surface
define a unique blending cyclide that contains all four circles
(using Lemma~\ref{lemma:non-degen-construction}).

     Since each point $X^\prime$ on a diagonal 
$\stackrel{\longleftrightarrow}{RS}$ determines a unique blending cyclide, 
we have an infinite number of blending cyclides.  

\begin{definition}
     The cyclide constructed from 
$X^\prime\in\stackrel{\longleftrightarrow}{RS}$, where $R$ and $S$ are diagonal
points, is denoted by 
${\cal Z}[\stackrel{\longleftrightarrow}{RS}|X^\prime]$.
The set of all cyclides constructed from points on the diagonal
$\stackrel{\longleftrightarrow}{RS}$, the 
$\stackrel{\longleftrightarrow}{RS}$ family, is denoted by
${\cal Z}[\stackrel{\longleftrightarrow}{RS}]$.
\end{definition}

\begin{lemma}
\label{lemma:int-cyclide}
     If two axial natural quadrics with intersecting and distinct axes have
non-degenerate conic intersection, they have a Dupin blending cyclide.
\end{lemma}
{\bf Proof:}  There are four cases to consider.  The first two cases deal with
the blending of two cones: the first is the general case in which the diagonal
points of the diagonal being used are both finite, and the second case when
one of these diagonal points is at infinity.  The third case and fourth deal
with cone-cylinder and cylinder-cylinder blends respectively.

     Consider the first case: two cones with no two parallel skeletal lines.
Because $A, B, C$ and $D$ are tangent 
points, we have $|\overline{RA}|=|\overline{RC}|$ and 
$|\overline{SB}|=|\overline{SD}|$ (Figure~\ref{fig:int-config}).
Since $\stackrel{\longleftrightarrow}{AX}$ is parallel to
$\stackrel{\longleftrightarrow}{A^\prime X^\prime}$, we have
$\bigtriangleup XRA \sim\bigtriangleup X^\prime RA^\prime$ and
$\frac{|\overline{RA}|}{|\overline{RA^\prime}|}=
\frac{|\overline{RX}|}{|\overline{RX^\prime}|}$.  Similarly, from
$\bigtriangleup XRC$ and $\bigtriangleup X^\prime RC^\prime$, we have
$\frac{|\overline{RC}|}{|\overline{RC^\prime}|}=
\frac{|\overline{RX}|}{|\overline{RX^\prime}|}$.  From these two identities
and $|\overline{RA}|=|\overline{RC}|$, we have 
$|\overline{RA^\prime}|=|\overline{RC^\prime}|$.  
This establishes that there is a (unique) circle $O_1$ on the axial plane 
tangent to $\stackrel{\longleftrightarrow}{V_1R}$ and 
$\stackrel{\longleftrightarrow}{V_2R}$ at $A^\prime$ and $C^\prime$.
With the same argument, we can show 
$|\overline{SB^\prime}|=|\overline{SD^\prime}|$ and that
there is a (unique) circle
$O_2$ tangent to $\stackrel{\longleftrightarrow}{V_1S}$ and
$\stackrel{\longleftrightarrow}{V_2S}$ at $C^\prime$ and $D^\prime$.

     Since $|\overline{V_1A}|=|\overline{V_1B}|$ and
$\stackrel{\longleftrightarrow}{AB}$ is parallel to
$\stackrel{\longleftrightarrow}{A^\prime B^\prime}$,
$|\overline{V_1A^\prime}|=|\overline{V_1B^\prime}|$, and similarly,
$|\overline{V_2C^\prime}|=|\overline{V_2D^\prime}|$.
Therefore, $V_1$ and $V_2$ lies on the radical axis of these circles and
$X^\prime$ is a center of similitude (Corollary~\ref{cor:tangent-center}).
Let $D_{\overline{A^\prime B^\prime}}$ be the circle with diameter
$\overline{A^\prime B^\prime}$ and perpendicular to the axial plane.  
Let $\ell_{X^\prime}$ be the line through $X^\prime$ and perpendicular to the 
axial plane.  By Lemma~\ref{lemma:non-degen-construction}, there is a unique 
Dupin cyclide,
${\cal Z}(O_1,O_2,D_{\overline{A^\prime B^\prime}},\ell_{X^\prime})$,
containing the circles $O_1, O_2$ and $D_{\overline{A^\prime B^\prime}}$.  
Applying Lemma~\ref{lemma:non-degen-construction} again, there is another 
cyclide, ${\cal Z}(O_1,O_2,D_{\overline{C^\prime D^\prime}},\ell_{X^\prime})$,
containing the circles $O_1,O_2$ and $D_{\overline{C^\prime D^\prime}}$,
where $D_{\overline{C^\prime D^\prime}}$ is the circle with diameter
$\overline{C^\prime D^\prime}$ and perpendicular to the axial plane.
Since these two cyclides have the same principal circles and center of
similitude (i.e., $X^\prime$), they are identical.  Thus, this lemma is
established for the first case.

     If one of the diagonal points, say $S$, goes to infinity, the construction
is the same except for the construction for the second circle tangent to
$\stackrel{\longleftrightarrow}{V_1S}$ and 
$\stackrel{\longleftrightarrow}{V_2S}$ at $B^\prime$ and $D^\prime$.
The problem is that the diagonal point $S$ of the diagonal
$\stackrel{\longleftrightarrow}{RS}$ is now at infinity and so we cannot
establish the existence of the circle satisfying
$|\overline{SB^\prime}|=|\overline{SD^\prime}|$.  Instead, we will show that
$\stackrel{\longleftrightarrow}{B^\prime D^\prime}$ is parallel to
$\stackrel{\longleftrightarrow}{BD}$, which shows that there is a circle with
diameter $\overline{B^\prime D^\prime}$ that is tangent to
$\stackrel{\longleftrightarrow}{V_1S}$ and
$\stackrel{\longleftrightarrow}{V_2S}$ at $B^\prime$ and $D^\prime$
respectively (Figure~\ref{fig:int-config-para}).
\begin{figure}
\vspace{8cm}
\caption{Basic Configuration in the Axial Plane--Two Parallel Skeletal Lines Case}
\label{fig:int-config-para}
\end{figure}

     Consider $\bigtriangleup XBD$ and 
$\bigtriangleup X^\prime B^\prime D^\prime$
(Figure~\ref{fig:int-config-para}).  
Since $\stackrel{\longleftrightarrow}{XX^\prime}$,
$\stackrel{\longleftrightarrow}{BB^\prime}$ and
$\stackrel{\longleftrightarrow}{DD^\prime}$ are parallel to each other,
from Desargues' theorem, the intersection points of corresponding sides must 
be collinear.  Since $\stackrel{\longleftrightarrow}{XB}$ and
$\stackrel{\longleftrightarrow}{X^\prime B^\prime}$ are parallel, their 
intersection point is at infinity.  So is the intersection of
$\stackrel{\longleftrightarrow}{XD}$ and
$\stackrel{\longleftrightarrow}{X^\prime D^\prime}$.  Hence, the intersection
point of the third line pair, $\stackrel{\longleftrightarrow}{BD}$ and
$\stackrel{\longleftrightarrow}{B^\prime D^\prime}$, is also at infinity.
That is, $\stackrel{\longleftrightarrow}{BD}$ and
$\stackrel{\longleftrightarrow}{B^\prime D^\prime}$ are parallel.  

     Suppose that the second surface is a cylinder rather than a cone.
The proof is the 
same except that we do not have $|\overline{V_2C}|=|\overline{V_2D}|$.  
Instead, both $\stackrel{\longleftrightarrow}{CD}$ and
$\stackrel{\longleftrightarrow}{C^\prime D^\prime}$ are perpendicular to
the skeletal lines of the cylinder.  The proof is similar to the first case and
is omitted.

     Finally, consider the cylinder-cylinder case.  If two cylinders with 
intersecting axes have conic intersection, their radii must be equal
(Corollary~\ref{cor:variations-of-cm-ins-sphere} or
Theorem~\ref{thm:cy-cy-conic}).  Therefore, the diagonal bisects the angle
containing the diagonal (Figure~\ref{fig:construct-torus}).  
For each point $X$ on a diagonal
$\stackrel{\longleftrightarrow}{RS}$, where $R$ and $S$ are diagonal points,
since $\bigtriangleup XBS\cong\bigtriangleup XDS$,
$|\overline{XB}|=|\overline{XD}|$ and hence $|\overline{XA}|=|\overline{XC}|$.
Therefore, the constructed cyclide is of torus type (i.e., the principal 
circles are concentric).  \QED
\begin{figure}
\vspace{4.5cm}
\caption{Two Cylinders with Equal Radii Are Blended with a Torus}
\label{fig:construct-torus}
\end{figure}

     Note that if $X^\prime$ is not identical to $R$ or $S$, the constructed
cyclide, in general, is a ring, a doubly horned, or even a spindle type;
if $X^\prime$ coincides with $R$ or $S$, the cyclide is singly horned
with $X^\prime$ its only singularity.

     In this non-degenerate conic intersection case, we have two distinct and 
finite diagonals, and each provides a family of blending cyclides.  It is not 
difficult to see that these  two families are distinct.  However, they have a 
common member, the common inscribed sphere, which is generated when 
$X^\prime\in\stackrel{\longleftrightarrow}{RS}\cap
\stackrel{\longleftrightarrow}{R^\prime S^\prime}$.
The following lemma shows that any blending 
cyclide belongs to one of these two families.

\begin{lemma}
\label{lemma:two-families}
     Each blending cyclide of two axial natural quadrics with intersecting
axes and non-degenerate conic intersection is a member of one of the two 
families.  That is, if $Z$ is a blending cyclide, there exists a point $X$ 
and a diagonal $\stackrel{\longleftrightarrow}{RS}$ such that
$Z\in{\cal Z}[\stackrel{\longleftrightarrow}{RS}|X]$.
\end{lemma}
{\bf Proof:} This proof is divided into three parts, one for the cone-cone 
case, one for the cone-cylinder case, and the other for the cylinder-cylinder
case.

     Suppose two cones, with vertices $V_1$ and $V_2$, have a blending
cyclide.  On the axial plane, we have a skeletal quadrilateral $V_1RV_2S$.
There are two cases to consider: (1) both $R$ and $S$ are finite, and 
(2) $S$ is at infinity.  Note that if both $R$ and $S$ are at infinity, the 
skeletal lines in each skeletal pair must be parallel to the corresponding 
lines in the other pair and hence, their axes must also be parallel.

     Suppose $R$ and $S$ are finite. 
Let $\stackrel{\longleftrightarrow}{V_1R}$ and 
$\stackrel{\longleftrightarrow}{V_2R}$ (resp., 
$\stackrel{\longleftrightarrow}{V_1S}$ and
$\stackrel{\longleftrightarrow}{V_2S}$) be tangent to one (resp., the other)
principal circle at $A$ and $C$ (resp., $B$ and $D$).  We claim that
$\stackrel{\longleftrightarrow}{V_1V_2}$,
$\stackrel{\longleftrightarrow}{AC}$ and
$\stackrel{\longleftrightarrow}{BD}$ are concurrent 
(Figure~\ref{fig:concurrent}(a)).  
Let $X=\stackrel{\longleftrightarrow}{AC}\cap
\stackrel{\longleftrightarrow}{V_1V_2}$.  If $X$ is a point at infinity, then
$\stackrel{\longleftrightarrow}{AC}$ is parallel to
$\stackrel{\longleftrightarrow}{V_1V_2}$, implying that both
$|\overline{V_1R}|=|\overline{V_2R}|$ and $V_1$ and $V_2$ are symmetric about 
the line joining the centers of the principal circles on the axial plane.  
This implies that $|\overline{V_1S}|=|\overline{V_2S}|$.  Therefore, 
$\stackrel{\longleftrightarrow}{V_1V_2}$,
$\stackrel{\longleftrightarrow}{AC}$ and
$\stackrel{\longleftrightarrow}{BD}$ are parallel to each other, and
they meet at the same point at infinity.
Suppose $X$ is finite.  Let $\bar{X}=\stackrel{\longleftrightarrow}{BD}\cap
\stackrel{\longleftrightarrow}{V_1V_2}$.  $\bar{X}$ must be finite, otherwise
$X$ will be at infinity.  In $\bigtriangleup V_1RV_2$,
$\stackrel{\longleftrightarrow}{AC}$ is a transversal and by Menelaus' theorem
we have 
\[ \frac{|\overline{V_1A}|}{|\overline{AR}|}\cdot
   \frac{|\overline{RC}|}{|\overline{CV_2}|}\cdot
   \frac{|\overline{V_2X}|}{|\overline{XV_1}|} = 1. \]
Since $|\overline{RA}|=|\overline{RC}|$,
\[ \frac{|\overline{XV_1}|}{|\overline{XV_2}|} =
   \frac{|\overline{V_1A}|}{|\overline{V_2C}|}. \]
Similarly, in $\bigtriangleup V_1SV_2$, using 
$\stackrel{\longleftrightarrow}{BD}$ as a transversal, we have
\[ \frac{|\overline{\bar{X}V_1}|}{|\overline{\bar{X}V_2}|} =
   \frac{|\overline{V_1A}|}{|\overline{V_2C}|}. \]
With these two identities, we finally have $X=\bar{X}$ and hence,
$\stackrel{\longleftrightarrow}{V_1V_2}$,
$\stackrel{\longleftrightarrow}{AC}$ and
$\stackrel{\longleftrightarrow}{BD}$ are concurrent.
\begin{figure}
\vspace{14cm}
\caption{Any Blending Cyclide Belongs to One of the Two Families}
\label{fig:concurrent}
\end{figure}

     In $\bigtriangleup RAC$ and $\bigtriangleup SBD$, their corresponding 
sides meet at three collinear points $V_1, V_2$ and $X=\bar{X}$.  By
Desargues' theorem, the lines joining corresponding vertices are concurrent.
That is, $\stackrel{\longleftrightarrow}{AB}$,
$\stackrel{\longleftrightarrow}{CD}$ and $\stackrel{\longleftrightarrow}{RS}$
meet at the same point, say $X^\prime$.  
>From Corollary~\ref{cor:tangent-center}, $X^\prime$ is a center of similitude
of the two principal circles.   Therefore, applying the construction 
algorithm to $X^\prime$ shows that the given blending cyclide 
is a member of the family generated by the diagonal 
$\stackrel{\longleftrightarrow}{RS}$, 
${\cal Z}[\stackrel{\longleftrightarrow}{RS}]$.

     Consider the second case: $S$ at infinity
(Figure~\ref{fig:concurrent}(b)).  Since $R$ is finite,
the following identity does not change:
\begin{equation}
\label{eqn:nondegen-equ-1}
   \frac{|\overline{XV_1}|}{|\overline{XV_2}|} =
   \frac{|\overline{V_1A}|}{|\overline{V_2C}|}.
\end{equation}
Note that $\stackrel{\longleftrightarrow}{V_1B}$ is parallel to
$\stackrel{\longleftrightarrow}{V_2D}$.  Because $\bigtriangleup \bar{X}V_1B
\sim\bigtriangleup \bar{X}V_2D$, we have
\begin{equation}
\label{eqn:nondegen-equ-2}
   \frac{|\overline{\bar{X}V_1}|}{|\overline{\bar{X}V_2}|} =
     \frac{|\overline{V_1B}|}{|\overline{V_2D}|}=
     \frac{|\overline{V_1A}|}{|\overline{V_2C}|}.
\end{equation}
>From (\ref{eqn:nondegen-equ-1}) and (\ref{eqn:nondegen-equ-2}), we have 
$X=\bar{X}$.  By the same reasoning as above,  the
given blending cyclide is a member of the family generated by the diagonal
$\stackrel{\longleftrightarrow}{RS}$.

     Next, consider the cone-cylinder case.  Let $V_2$ be at infinity
making the second cone a cylinder (Figure~\ref{fig:concurrent}(c)).  Let 
$\stackrel{\longleftrightarrow}{AC}$ and $\stackrel{\longleftrightarrow}{BD}$
meet $\stackrel{\longleftrightarrow}{V_1V_2}$ at $X$ and $\bar{X}$ 
respectively.  Since the two skeletal lines of the cylinder are parallel to the
radical axis of the two principal circles, we have
$\bigtriangleup RAC\sim\bigtriangleup V_1AX$ and
$\bigtriangleup SBD\sim\bigtriangleup V_1B\bar{X}$.  
Since $|\overline{RA}|=|\overline{RC}|$ and $|\overline{SB}|=|\overline{SD}|$,
$|\overline{V_1A}|=|\overline{V_1X}|$ and 
$|\overline{V_1B}|=|\overline{V_1\bar{X}}|$.  However, since
$|\overline{V_1A}|=|\overline{V_1B}|$, $X=\bar{X}$.  By the same reasoning
as above, the given blending cyclide is a member of the 
family generated by the diagonal $\stackrel{\longleftrightarrow}{RS}$.

     Finally, consider the cylinder-cylinder case.  Since these cylinders,
with intersecting axes, have a blending cyclide, they have planar intersection
and thus, their radii are equal.  On the axial plane, let the principal
circles be tangent to the skeletal lines of the first (resp., second)
cylinder at $A$ and $B$ (resp., $C$ and $D$) 
(Figure~\ref{fig:construct-torus}).  
Then, $|\overline{AB}|=|\overline{CD}|=2r$, where $r$ is the radius of the
cylinders.  Since
$\stackrel{\longleftrightarrow}{AB}$ is perpendicular to the tangents of the
circles, it passes through the centers of the principal circles.  So does
$\stackrel{\longleftrightarrow}{CD}$.  Therefore,
$X=\stackrel{\longleftrightarrow}{AB}\cap
\stackrel{\longleftrightarrow}{CD}$ is the common center of both principal
circles.  That is, the blending cyclide is a torus.  Now it is not
difficult to see that $X$, $R$ and $S$ are collinear.  By the same
reasoning as above, the given blending torus is member of the family generated
by the diagonal $\stackrel{\longleftrightarrow}{RS}$.  \QED

    Combining Lemma~\ref{lemma:int-cyclide} and Lemma~\ref{lemma:two-families},
we have the following theorem.

\begin{theorem}
\label{thm:cyclide<->planar-nondegen}
     If two axial natural quadrics, with intersecting and distinct axes, have
non-degenerate conic intersection, they have a blending cyclide.
Furthermore, any blending cyclide is a member of the two families of
cyclides generated by the two diagonals.
\end{theorem}

% ====================================================================
%                       Parallel Axes Case
% ====================================================================

\subsection{The Parallel Axes Case}
\label{section:cyc:parallel-axes}

     In this section, we shall examine the parallel and distinct axes case.  
If the axes coincide, it is easy to show that there are two families of
blending tori (Figure~\ref{fig:blending-tori}).
In the following, we shall assume the axes are distinct.
\begin{figure}
\vspace{6.5cm}
\caption{Two Families of Blending Tori}
\label{fig:blending-tori}
\end{figure}

     If the axes are distinct, in order to have a conic intersection, the 
corresponding skeletal lines must be parallel to each other.  In this case, we 
have only one finite diagonal with two finite diagonal points $R$ and $S$.  
Since we do not have a finite common inscribed sphere, a different proof
technique is needed.  The construction can still be performed with any point 
$X$ on the diagonal $\stackrel{\longleftrightarrow}{RS}$ as will be shown in 
Lemma~\ref{lemma:para-cyclide}.  Since each diagonal gives
one family of blending cyclides, it is reasonable to guess that for the
parallel axes case there is only one family of blending cyclides.
This fact is established in Lemma~\ref{lemma:para-1-family}.

\begin{lemma}
\label{lemma:para-cyclide}
     If two axial natural quadrics, with parallel and distinct axes, 
have non-degenerate conic intersection, they have a blending cyclide.
\end{lemma}
{\bf Proof:}  Since a cone and a cylinder with parallel axes can never have
planar intersection (Theorem~\ref{thm:cone-cylinder-no-parallel}),
in this proof, we shall only consider the cone-cone and 
cylinder-cylinder cases.

     Suppose ${\cal C}_1$ and ${\cal C}_2$ are cones with vertices $V_1$ and
$V_2$ and parallel axes $\ell_1$ and $\ell_2$.  Since they have conic 
intersection, the corresponding skeletal lines on the axial plane 
must be parallel (Theorem~\ref{theorem:parallel-axes} and
Theorem~\ref{theorem:parallel-axes-geo}).  Let the finite diagonal be
$\stackrel{\longleftrightarrow}{RS}$, where $R$ and $S$ are diagonal points
(Figure~\ref{fig:para-cyclide}).  Let $X\in\stackrel{\longleftrightarrow}{RS}$.
>From $X$ construct a line perpendicular to $\ell_1$ and $\ell_2$ meeting
$\stackrel{\longleftrightarrow}{V_1R}, \stackrel{\longleftrightarrow}{V_1S},
\stackrel{\longleftrightarrow}{V_2R}$ and 
$\stackrel{\longleftrightarrow}{V_2S}$ at $A, B, C$ and $D$ respectively.
\begin{figure}
\vspace{6cm}
\caption{Basic Configuration in the Axial Plane--Parallel Axes Case}
\label{fig:para-cyclide}
\end{figure}

     Now it is not difficult to see that $|\overline{RA}|=|\overline{RC}|$ and
$|\overline{SB}|=|\overline{SD}|$.  Hence, there is a unique circle tangent to
$\stackrel{\longleftrightarrow}{V_1R}$ and 
$\stackrel{\longleftrightarrow}{V_2R}$ at $A$ and $C$ respectively.  Similarly,
there is a unique circle tangent to $\stackrel{\longleftrightarrow}{V_1S}$ and 
$\stackrel{\longleftrightarrow}{V_2S}$ at $B$ and $D$ respectively.
Again since $\stackrel{\longleftrightarrow}{AB}$ is perpendicular to
both axes, we have
$|\overline{V_1A}|=|\overline{V_1B}|$ and 
$|\overline{V_2C}|=|\overline{V_2D}|$, and hence $V_1$ and $V_2$ lie on the
radical axis of these circles.

     Next, we have to show that $X$ is a center of similitude of the two
principal circles.  Let $O_1$ and $r_1$ (resp., $O_2$ and $r_2$) be the center
and radius of the circle tangent to
$\stackrel{\longleftrightarrow}{V_1R}$ and
$\stackrel{\longleftrightarrow}{V_2R}$ (resp.,
$\stackrel{\longleftrightarrow}{V_1S}$ and
$\stackrel{\longleftrightarrow}{V_2S}$) (Figure~\ref{fig:para-1-family}).
Since $\bigtriangleup O_1RC\sim\bigtriangleup O_2SB$, 
$r_1/r_2=|\overline{RC}|/|\overline{SB}|$.  
>From $\bigtriangleup XRC\sim\bigtriangleup XSB$, we have
$|\overline{XC}|/|\overline{XB}|=|\overline{RC}|/|\overline{SB}|=r_1/r_2$.
Therefore, $\bigtriangleup XO_1C\sim \bigtriangleup XO_2B$ holds.
This implies that $\angle XO_1C=\angle XO_2B$ and
$|\overline{XO_1}|/|\overline{XO_2}|=r_1/r_2$.  Hence, $X, O_1$ and $O_2$ are
collinear and $X$ is a center of similitude.  
Again, from Lemma~\ref{lemma:non-degen-construction}, a unique cyclide can be
constructed.
\begin{figure}
\vspace{6cm}
\caption{There Is Only One Family of Blending Cyclides for Parallel Axes}
\label{fig:para-1-family}
\end{figure}

     The case of two cylinders with parallel axes is simple.
Recall that a cylinder can only be tangent to a cyclide along
one of the four principal circles.  Let $P$ be any plane perpendicular to
both axes.  $P$ intersects the cylinders in two circles
(Figure~\ref{fig:para-cy-cyclide}).  These are the
principal circles of a blending cyclide, horizontal in case (a) to (c), and
vertical in case (d) and (e).  \QED
\begin{figure}
\vspace{3cm}
\caption{Basic Configuration in the Axial Plane--Two Cylinders Case}
\label{fig:para-cy-cyclide}
\end{figure}

     Next, we shall prove that all blending cyclides belong to the 
family constructed in Lemma~\ref{lemma:para-cyclide}.

\begin{lemma}
\label{lemma:para-1-family}
     Each blending cyclide of two cones with parallel and distinct axes and 
non-degenerate conic intersection is a member of the family of cyclides
generated by the unique diagonal.  That is, it is a member of 
${\cal Z}[\stackrel{\longleftrightarrow}{RS}]$, where $R$ and $S$ are 
finite diagonal points.
\end{lemma}
{\bf Proof:}  Let ${\cal C}_1$ and ${\cal C}_2$ be two cones with 
vertices $V_1$ and $V_2$, parallel axes $\ell_1$ and $\ell_2$,
non-degenerate conic intersection, and thus
equal cone angles.  Let $R$ and $S$ be the
two finite diagonal points.
If there is a blending cyclide, the axial plane intersects
this cyclide in two principal circles.  
Let the principal circles be tangent to 
$\stackrel{\longleftrightarrow}{V_1R}$ and
$\stackrel{\longleftrightarrow}{V_2R}$ (resp.,
$\stackrel{\longleftrightarrow}{V_1S}$ and
$\stackrel{\longleftrightarrow}{V_2S}$) at $A$ and $C$ (resp., $B$ and $D$)
respectively (Figure~\ref{fig:para-1-family}).

     Since the skeletal lines are parallel, 
$\bigtriangleup V_1AB\sim\bigtriangleup V_2CD$.  Since two 
pairs of corresponding sides are parallel to each other, the third pair
(i.e., $\stackrel{\longleftrightarrow}{AB}$ and
$\stackrel{\longleftrightarrow}{CD}$) must be (1) perpendicular, (2) parallel,
or (3) identical.  The first case is impossible since the axes are parallel.  
For case (2),
it is easy to prove that we have two circles with equal radii and hence the
blending cyclide becomes a torus.  This implies in turn that the cone
axes are identical, a contradiction.  Therefore, 
$\stackrel{\longleftrightarrow}{AB}$ and $\stackrel{\longleftrightarrow}{CD}$
are identical and $A,B,C$ and $D$ are collinear.  Let this line meet the 
diagonal $\stackrel{\longleftrightarrow}{RS}$ at $X$.  
>From the proof in Lemma~\ref{lemma:para-cyclide}, $X$ must be a center of
similitude of the principal circles.  Therefore, this
cyclide can be constructed using $X\in\stackrel{\longleftrightarrow}{RS}$,
and thus, is a member of ${\cal Z}[\stackrel{\longleftrightarrow}{RS}]$. \QED

     In summary, we have established the following theorem.

\begin{theorem}
\label{thm:have-cyclide-parallel-axes}
     If two axial natural quadrics with parallel and distinct axes have 
non-degenerate conic intersection, they have a blending cyclide.  Furthermore,
any blending cyclide of two cones is a member of the
family generated by the unique diagonal.
\end{theorem}

% ====================================================================
%                        The Degenerate Case
% ====================================================================

\section{The Degenerate Conic Intersection Case}
\label{section:line-conic}

     In this section, we shall consider the {\em degenerate conic intersection}
\index{degenerate conic intersection} case in which the 
complete planar intersection consists of a double line and a conic. 
In this degenerate case, we still have two subcases: 
(1) parallel axes and (2) intersecting axes.  In the following, we will 
establish the converse of Corollary~\ref{cor:para->common-line} by showing 
that for the intersecting axes case, any blending cyclide must be 
parabolic (Lemma~\ref{lemma:only-one-family}).  As a simple result of this 
property, we prove that for the parallel axes case, two cones can never have 
any blending cyclide (Corollary~\ref{cor:no-cyclide-at-all}).  Then, by 
the construction in the last section, we prove the existence of blending 
cyclides for the intersecting axes case (Lemma~\ref{lemma:three-line-cyclide}).
Finally, Lemma~\ref{lemma:1-parabolic} establishes the uniqueness of this
family.

     If two cones have a double line and a conic in their intersection, then
in the axial plane two skeletal lines, one from each skeletal pair, coincide 
and the common line is a double line.  The vertices of the surfaces cannot 
coincide; otherwise, we have a linear intersection.  If the axes intersect,
the skeletal quadrilateral reduces to a triangle and one of the two 
diagonals becomes the common line (Figure~\ref{fig:three-lines}).
As we have shown in Chapter~\ref{chapter:NQ-NQ}, the other diagonal is the 
line joining the intersection point $S$ of the two non-coincident lines and 
the tangent point $R$ of the common inscribed sphere on the common line
(Figure~\ref{fig:three-lines}(a) to (c)).  Note that $S$ may be at infinity if
the sum of cone angles is equal to $90^\circ$ 
(Figure~\ref{fig:three-lines}(d)).  The case of a cone and a cylinder is 
totally analogous to two cones 
(Figure~\ref{fig:three-lines}(e)).\footnote{Note that two cylinders
with a double line in their intersection have a linear intersection, not a
double line and a conic; and so this case is not covered here.}  
In all of the above cases, we always have a diagonal.
However, if the axes are parallel (Figure~\ref{fig:three-lines}(f)),
there is no diagonal.  Intuitively, since a diagonal controls one family of 
blending cyclides, each of the first five cases has one family, while
the last case has no blending cyclide at all.  
\begin{figure}
\vspace{7.5cm}
\caption{The Only Diagonal in the Double Line-Conic Case}
\label{fig:three-lines}
\end{figure}

     The following lemma is our first step.  It shows that if two axial natural
quadrics have a blending cyclide, it cannot be a non-degenerate one.

\begin{lemma}
\label{lemma:only-one-family}
     A blending cyclide of two axial natural quadrics with degenerate conic
intersection (i.e., a double line and a conic in their intersection) must be
parabolic.
\end{lemma}
{\bf Proof:} We consider the cone-cone case first.  Suppose the cones 
intersect in a double line and a conic.  The axial plane intersects the 
surfaces in four lines in which two of them, one from each skeletal pair, 
coincide.  If these cones have a non-parabolic blending cyclide, the axial 
plane intersects this cyclide in two principal circles and their radical axis 
contains the vertices.  Hence, the radical axis is the double line, since the 
double line also contains the cones' vertices.  Because a principal circle is 
tangent to a skeletal line from each skeletal pair, one of the two principal 
circles is tangent to the radical axis.  In a coaxal circles system, if one of 
the circles is tangent to the radical axis, all of the other circles in the 
system are also tangent to the radical axis at the same point.  Thus, the 
principal circles must be tangent to the radical axis at the same point.

     We claim that these two principal circles are not point circles.
Recall that exactly two non-degenerate cyclides can have a point singularity:
the singly horned cyclide and the spindle cyclide with one singularity.  
However, the point principal circle does not lie on the radical axis in either
case.  Thus, neither principal circle is a point, and either
(1) one of the circles contains the other (Figure~\ref{fig:all-tangent}(a)), or
(2) they exclude each other (Figure~\ref{fig:all-tangent}(b)).  In case (1),
the principal circles are obviously the horizontal principal circles of a
singly horned cyclide.  However, this is not a blending cyclide, since
it is not tangent to the cones along circles.  In case (2), the circles are
the vertical principal circles of a spindle cyclide with one singularity at
the common tangent point.  This cyclide does not blend the given
cones either.  Since both cases give contradictions, the blending cyclide
cannot be non-parabolic.  A cylinder can be treated as a cone with its vertex
at infinity and the above argument still holds. \QED
\begin{figure}
\vspace{5cm}
\caption{The Principal Circles Are Tangent to Their Radical Axis}
\label{fig:all-tangent}
\end{figure}

     An immediate result of this lemma is to show that there exists a special
class of cones with planar intersection that have no blending cyclide.  This
is the only exception to the necessity and sufficiency of planar intersection
for a blending cyclide.

\begin{corollary}
\label{cor:no-cyclide-at-all}
     Two cones with parallel axes and degenerate conic intersection 
have no blending cyclide.
\end{corollary}
{\bf Proof:}  Suppose these two cones have a blending parabolic 
cyclide.  Let it be tangent to the two cones along the circles $C_1$ and $C_2$.
Since the planes containing $C_1$ and $C_2$ are perpendicular
to the parallel axes, they are parallel to each other.
However, $C_1$ and $C_2$ are circles from the same family of a parabolic
cyclide, and all planes containing circles from the same family pass through
a finite common line, one of the two principal lines of the parabolic cyclide.
This contradiction shows that the cones do not have any blending cyclide. \QED

     We now establish the converse of Corollary~\ref{cor:para->common-line}.
The algorithm of the last section is used to construct the blending parabolic 
cyclide.

\begin{lemma}
\label{lemma:three-line-cyclide}
     If two axial natural quadrics with intersecting axes have degenerate conic
intersection, they have a blending parabolic cyclide.
\end{lemma}
{\bf Proof:}  Let ${\cal C}_1(V_1,\ell_1,\alpha_1)$ and
${\cal C}_2(V_2,\ell_2,\alpha_2)$ be two cones with
$\stackrel{\longleftrightarrow}{V_1V_2}$ the only double line in the
intersection curve.  Let $\stackrel{\longleftrightarrow}{RS}$ be the diagonal,
where $R$ is the tangent point of the common inscribed 
sphere on $\stackrel{\longleftrightarrow}{V_1V_2}$ and $S$ is the intersection
point of the other two non-coincident skeletal lines.  Note that $S$ may be
at infinity.  Let $M$ and $N$ be the tangent points of the common inscribed 
sphere on $\stackrel{\longleftrightarrow}{V_1S}$ and
$\stackrel{\longleftrightarrow}{V_2S}$ respectively.
Let $X\in\stackrel{\longleftrightarrow}{RS}$ and 
$X\not\in\stackrel{\longleftrightarrow}{V_1V_2}$.\footnote{If $X$ lies on
$\stackrel{\longleftrightarrow}{V_1V_2}$, the blending surface is the common
inscribed sphere.}
Applying the construction algorithm at $X$ gives a
line perpendicular to $\ell_1$ (resp., $\ell_2$) meeting
$\stackrel{\longleftrightarrow}{V_1R}$ and
$\stackrel{\longleftrightarrow}{V_1S}$ (resp.,
$\stackrel{\longleftrightarrow}{V_2R}$ and
$\stackrel{\longleftrightarrow}{V_2S}$) at $A$ and $B$ (resp., $C$ and $D$)
respectively (Figure~\ref{fig:parabolic-construct}).  Let $O$ be the 
circumscribed circle of $\bigtriangleup BDX$.  We claim that 
$\stackrel{\longleftrightarrow}{V_1V_2}$, $O$ and $X$ define a blending
parabolic cyclide, ${\cal Z}_P(\stackrel{\longleftrightarrow}{V_1V_2},O,X)$.
We must first show that this cyclide is well defined by showing that the line 
joining $X$ and the center of $O$ is perpendicular to
$\stackrel{\longleftrightarrow}{V_1V_2}$.
\begin{figure}
\vspace{5.5cm}
\caption{The Construction of a Blending Parabolic Cyclide}
\label{fig:parabolic-construct}
\end{figure}

     We first show that $\stackrel{\longleftrightarrow}{BD}$ and
$\stackrel{\longleftrightarrow}{MN}$ are parallel.
In $\bigtriangleup RMN$ and $\bigtriangleup XBD$, since
$\stackrel{\longleftrightarrow}{BM}$,
$\stackrel{\longleftrightarrow}{XR}$ and
$\stackrel{\longleftrightarrow}{DN}$ are concurrent at $S$, by Desargues'
theorem, 
$\stackrel{\longleftrightarrow}{XB}\cap\stackrel{\longleftrightarrow}{RM}$,
$\stackrel{\longleftrightarrow}{XD}\cap\stackrel{\longleftrightarrow}{RN}$ and
$\stackrel{\longleftrightarrow}{BD}\cap\stackrel{\longleftrightarrow}{MN}$
are collinear.  Since $\stackrel{\longleftrightarrow}{XB}$ (resp.,
$\stackrel{\longleftrightarrow}{XD}$) is parallel to
$\stackrel{\longleftrightarrow}{RM}$ (resp.,
$\stackrel{\longleftrightarrow}{RN}$), the intersection point is a point at
infinity and thus, the line determined by the three collinear points is the
line at infinity.  Therefore, 
$\stackrel{\longleftrightarrow}{BD}$ and
$\stackrel{\longleftrightarrow}{MN}$ meet at a point at infinity and thus,
they are parallel.

     Consider a tangent of circle $O$ at $X$.  Let this tangent
meet $\stackrel{\longleftrightarrow}{V_2S}$ at $Y$.  Then, we have
$\angle YXD=\angle XBD=\angle RMN=\angle V_2RN$ and hence, this tangent line
is parallel to $\stackrel{\longleftrightarrow}{V_1V_2}$.  Since the line
joining $X$ and $O$'s center is perpendicular to the tangent line, which is
parallel to $\stackrel{\longleftrightarrow}{V_1V_2}$, it must also be 
perpendicular to $\stackrel{\longleftrightarrow}{V_1V_2}$.
Therefore, we have a unique parabolic cyclide
${\cal Z}_P(\stackrel{\longleftrightarrow}{V_1V_2},O,X)$.  In this cyclide,
$\stackrel{\longleftrightarrow}{V_1V_2}$ and $O$ is a principal line and
principal circle.  

    Next, we show that ${\cal Z}_P(\stackrel{\longleftrightarrow}{V_1V_2},O,X)$
blends the cones.
Consider the circumscribed circle of $\bigtriangleup RMN$, which is the
intersection circle of the common inscribed sphere and the axial plane.
Since $\stackrel{\longleftrightarrow}{V_1S}$ is tangent to it at $M$, a
geometry theorem gives $\angle V_1MR=\angle MNR$.  Therefore, in
$\bigtriangleup XBD$, since the corresponding sides are parallel to each other,
we have $\angle V_1XB=\angle V_1MR=\angle MNR=\angle BDX$.  Hence,
$\stackrel{\longleftrightarrow}{V_1S}$ is a tangent of circle $O$ at $B$.
Similarly, $\stackrel{\longleftrightarrow}{V_2S}$ is a tangent of circle $O$
at $D$.

     Since from the construction 
$|\overline{V_1A}|=|\overline{V_1B}|$, there must be a circle tangent to 
$\stackrel{\longleftrightarrow}{V_1V_2}$ and 
$\stackrel{\longleftrightarrow}{V_1S}$ at $A$ and $B$.
Since $O$ is tangent to $\stackrel{\longleftrightarrow}{V_1S}$ at $B$, this 
circle is also tangent to $O$.  Therefore, the cone ${\cal C}_1$ is tangent
to the parabolic cyclide along the circle with diameter $\overline{AB}$ and
perpendicular to the axial plane.  Similarly, the cone ${\cal C}_2$ is
also tangent to the cyclide along the circle with diameter $\overline{CD}$
and perpendicular to the axial plane.  Hence, the parabolic cyclide
is a blending surface of ${\cal C}_1$ and ${\cal C}_2$.
The proof of the cone-cylinder case is completely analogous and is 
omitted. \QED

     At this point, we know that two axial natural quadrics with intersecting
axes always have a blending parabolic cyclide.
However, are there some other families of parabolic cyclides
different from the family constructed in Lemma~\ref{lemma:three-line-cyclide}?
The following lemma shows that there are not.

\begin{lemma}
\label{lemma:1-parabolic}
     Each blending cyclide of two axial natural quadrics with intersecting
axes and degenerate conic intersection is a member of the family constructed
in Lemma~\ref{lemma:three-line-cyclide}.
\end{lemma}
{\bf Proof:} Suppose two cones ${\cal C}_1$ and ${\cal C}_2$ have a blending
cyclide.  The cone-cylinder case is similar.
By Lemma~\ref{lemma:only-one-family}, this cyclide must be
parabolic.  On the axial plane, which intersects the cyclide in the principal
circle $O$ and the principal line $\stackrel{\longleftrightarrow}{V_1V_2}$,
let $S$ be the intersection point of the two non-identical skeletal lines
(Figure~\ref{fig:parabolic-construct}).  Note that $S$ is at infinity if the 
sum of the  cone angles is $90^\circ$.  Since the cyclide blends ${\cal C}_1$ 
and ${\cal C}_2$, the planes containing the two common tangent circles  meet in
a principal line perpendicular to the axial plane and through a fixed point $X$
on the principal circle $O$.  Let the plane containing the
common tangent circle on ${\cal C}_1$ (resp., ${\cal C}_2$) intersect
$\stackrel{\longleftrightarrow}{V_1V_2}$ at $A$ (resp., $C$) and
$\stackrel{\longleftrightarrow}{V_1S}$ (resp.,
$\stackrel{\longleftrightarrow}{V_2S}$) at $B$ (resp., $D$).  Then,
$\stackrel{\longleftrightarrow}{AB}$ and $\stackrel{\longleftrightarrow}{CD}$
meet at $X$.  Note that $O$ is tangent to 
$\stackrel{\longleftrightarrow}{V_1S}$ and 
$\stackrel{\longleftrightarrow}{V_2S}$ at $B$ and $D$ respectively and
is the circumscribed circle of $\bigtriangleup XBD$.

     Let the common inscribed sphere be tangent to
$\stackrel{\longleftrightarrow}{V_1V_2}$,
$\stackrel{\longleftrightarrow}{V_1S}$ and
$\stackrel{\longleftrightarrow}{V_2S}$ at $R$, $M$ and $N$ respectively.
It is not difficult to see that $\stackrel{\longleftrightarrow}{XB}$ and
$\stackrel{\longleftrightarrow}{RM}$ (resp., 
$\stackrel{\longleftrightarrow}{XD}$ and
$\stackrel{\longleftrightarrow}{RN}$) are parallel.  Thus, we have
$\angle MRN=\angle BXD$.  Since 
$\stackrel{\longleftrightarrow}{V_1S}$ is tangent to the circumscribed circle
of $\bigtriangleup RMN$ (resp., $\bigtriangleup XBD$) at $M$ (resp., $B$), 
$\angle SMN=\angle MRN$ (resp., $\angle SBD=\angle BXD$).
Hence, $\stackrel{\longleftrightarrow}{MN}$ is parallel to
$\stackrel{\longleftrightarrow}{BD}$.  In $\bigtriangleup RMN$ and
$\bigtriangleup XBD$, since corresponding lines are parallel (i.e., meeting at
points on the line at infinity), by Desargues' theorem, the lines joining 
corresponding vertices are concurrent.  That is, 
$\stackrel{\longleftrightarrow}{BM}$,
$\stackrel{\longleftrightarrow}{DN}$ and
$\stackrel{\longleftrightarrow}{XR}$ meet at $S$ and thus, $R$, $S$ and $X$ are
collinear.  This show that $X$ lies on the diagonal and the given blending
cyclide is generated by applying the construction algorithm at $X$.
In other words, the given cyclide is 
${\cal Z}_P(\stackrel{\longleftrightarrow}{V_1V_2},O,X)$, where $O$ is the
circumscribed circle of $\bigtriangleup XBD$.  \QED

     Combining all results obtained in this section, we have the following 
theorem for the degenerate case.

\begin{theorem}
\label{thm:cyclide<->planar-degen}
     Two axial natural quadrics with intersecting axes and degenerate conic
intersection always have a blending cyclide.  Any blending 
cyclide is a member of the family of parabolic cyclides generated by the only 
diagonal.  Furthermore, two cones with parallel axes and degenerate conic
intersection can never have any blending cyclide.
\end{theorem}

% ====================================================================
%                   The Linear Intersection Case
% ====================================================================

\section{The Linear Intersection Case}
\label{section:cyc-linear}

     The construction of the previous sections does not work for the linear
intersection case, because the skeletal quadrilateral does not exist.  
In this section, a technique similar to Boehm's~\cite{boehm:1990}
\index{Boehm}
is used to fully investigate this case.  Only the case of two cones 
will be considered, since a cone and a cylinder cannot have linear 
intersection, and the case of two cylinders has been solved completely in 
Lemma~\ref{lemma:para-cyclide}.

\begin{lemma}
\label{lemma:all-dist-equal}
     If two cones with a common vertex $V$ have a blending cyclide, then 
in the axial plane the distances from $V$ to the tangent points of the two 
principal circles and the skeletal lines are equal.
\end{lemma}
{\bf Proof:} Suppose that two cones with a common vertex $V$ have a 
blending cyclide.  The axial plane intersects the cyclide in two principal 
circles.  Since $V$ lies on the radical axis of these two circles, the 
distances from $V$ to all four tangent points of these circles and the 
skeletal lines are equal. \QED

     This lemma provides an upper bound on the number of families of
blending cyclides.  

\begin{lemma}
\label{lemma:upper-bound}
     There are at most four distinct families of blending cyclides for
two cones with a common vertex.
\end{lemma}
{\bf Proof:}  Let $V$ be the common vertex.  On the axial plane, draw a 
circle with any radius $r>0$ and center $V$ meeting the skeletal lines of the 
first cone at $A_1, B_1, C_1$ and $D_1$ (and then $\overline{A_1B_1}$ and 
$\overline{C_1D_1}$ are perpendicular to the axis of the first cone), and 
similarly, $A_2, B_2, C_2$ and $D_2$ on the second cone
(Figure~\ref{fig:upper-bound}).  Let the circle tangent to two skeletal lines,
one from each cone, at $X$ and $Y$ be denoted by $C_{(X,Y)}$.  
Consider $C_{(A_1,A_2)}$ and $C_{(B_1,B_2)}$.  Since $V$ lies on the radical 
axis of these two circles,
$S=\stackrel{\longleftrightarrow}{A_1B_1}\cap
\stackrel{\longleftrightarrow}{A_2B_2}$ is a center of similitude
(Corollary~\ref{cor:tangent-center}).  Let $C_{\overline{A_1B_1}}$ and
$C_{\overline{A_2B_2}}$ be the circles with diameters $\overline{A_1B_1}$ and
$\overline{A_2B_2}$, respectively, and perpendicular to the axial plane.
Then, by Lemma~\ref{lemma:non-degen-construction},
${\cal Z}(C_{(A_1,A_2)},C_{(B_1,B_2)},C_{\overline{A_1B_1}},S)$ is a blending
cyclide.  Note that
${\cal Z}(C_{(A_1,A_2)},C_{(B_1,B_2)},C_{\overline{A_2B_2}},S)$ gives another.
Since these two cyclides share the same principal circles
$C_{(A_1,A_2)}$ and $C_{(B_1,B_2)}$, and the same center of similitude $S$,
they are identical.

     Similarly, $C_{(A_1,B_2)}$ and $C_{(B_1,A_2)}$ determine the second 
cyclide, while the third is given by $C_{(A_1,D_2)}$ and $C_{(B_1,C_2)}$.  
Finally, the fourth one is determined by $C_{(A_1,C_2)}$ and $C_{(B_1,D_2)}$.
Since these cyclides are parameterized by $r$, there are four distinct 
families of blending cyclides. \QED
\begin{figure}
\vspace{4cm}
\caption{Determining the Four Interpolating Cyclides}
\label{fig:upper-bound}
\end{figure}

\begin{remark} \rm
     For two cones with a common vertex and distinct axes, the above 
construction can be
formulated by two lines mimicking the two diagonals.  Suppose all skeletal 
lines are distinct.  Then, the locus of the intersection point of
$\stackrel{\longleftrightarrow}{A_1B_1}$ and
$\stackrel{\longleftrightarrow}{A_2B_2}$ is a line $e_1$ through the common 
vertex.  Similarly, there is another line $e_2$ generated from the intersection
point of $\stackrel{\longleftrightarrow}{A_1B_1}$ and
$\stackrel{\longleftrightarrow}{C_2D_2}$.  Now it is not difficult to see that
the intersection point of $\stackrel{\longleftrightarrow}{C_1D_1}$ and
$\stackrel{\longleftrightarrow}{C_2D_2}$ (resp.,
$\stackrel{\longleftrightarrow}{A_2B_2}$ and
$\stackrel{\longleftrightarrow}{C_1D_1}$) lies on $e_1$ (resp., $e_2$).
Therefore, $e_1$ and $e_2$ play the role of the diagonals.  However,
we prefer to use a circle instead of $e_1$ and $e_2$ simply because
the former case has no special cases to consider.  $\Box$
\end{remark}

     The rest of this section is devoted to constructing these blending
cyclides.  We shall show that if the cones have no double line the upper bound
of Lemma~\ref{lemma:upper-bound} can always be attained.  
If the cones have one (resp., two) double line we have 
three (resp., two) families of blending cyclides.  

     If there is no double line in the intersection, the linear intersection
consists of four, two or zero real lines.  Figure~\ref{fig:four-real} displays
the construction of all four families of blending cyclides.  In the figure,
the circle whose center is the common vertex defines the families, and the  
line segments joining the two points, one on each skeletal line from different 
skeletal pairs, are the perpendicular projections of the common tangent 
circles of the blending cyclide and the cones.  
The other two circles are principal circles.
In this case, all blending cyclides are of double horned type defined by two
vertical principal circles.  These results are collected in
Table~\ref{tbl:no-double-line}, where the types of the principal circle and 
the blending cyclides are given.
\begin{figure}
\vspace{4cm}
\caption{The Case of Four Real Lines}
\label{fig:four-real}
\end{figure}
\begin{table}
\caption{Types of Blending Cyclides and Principal Circles--No Double Line}
\label{tbl:no-double-line}
$$
\BeginTable
     \def\C{\JustLeft}
     \ninepoint
     \BeginFormat
     |5 c | c | c | c | c | c | c |5
     \EndFormat
     \_5
     | \use2 \it Real Lines |
          \it Figure | (a) | (b) | (c) | (d) |  \\ \_3
     | 4 "     | \ref{fig:four-real} | Vertical\ ${\cal H}_2$
               | Vertical  \ ${\cal H}_2$
               | Vertical  \ ${\cal H}_2$
               | Vertical  \ ${\cal H}_2$ | \\
     | 2 "     | \ref{fig:two-real}  | Vertical  \ ${\cal H}_2$
               | Horizontal  \ ${\cal S}_2$
               | Vertical  \ ${\cal R}$
               | Horizontal \ ${\cal H}_2$ | \\
     |\Lower{0} " Containing | \ref{fig:zero-real-1} | Vertical \ ${\cal R}$
               | Vertical \ ${\cal S}_2$
               | Vertical \ ${\cal R}$
               | Vertical \ ${\cal S}_2$ | \\
     |         " Excluding | \ref{fig:zero-real-2} | Horizontal \ ${\cal R}$
               | Horizontal \ ${\cal H}_2$
               | Horizontal \ ${\cal R}$
               | Horizontal \ ${\cal H}_2$ | \\ \_5
     " \use7 \C ${\cal R}$: ring, ${\cal H}_2$: double horned, 
               and ${\cal S}_2$: two singularity spindle "  \\
\EndTable
$$
\end{table}

     Figure~\ref{fig:two-real} gives the desired result for the case of two 
real lines.  In order to have zero real lines, 
one cone must lie in the interior of 
the other (Figure~\ref{fig:zero-real-1}), or they must exclude each other
(Figure~\ref{fig:zero-real-2}).  In both cases, we have two ring cyclides,
and two spindle cyclides with two singularities or two double horned cyclides.
\begin{figure}
\vspace{4cm}
\caption{The Case of Two Real Lines}
\label{fig:two-real}
\end{figure}
\begin{figure}
\vspace{4cm}
\caption{The Case of Zero Real Line -- I}
\label{fig:zero-real-1}
\end{figure}
\begin{figure}
\vspace{4cm}
\caption{The Case of Zero Real Line -- II}
\label{fig:zero-real-2}
\end{figure}

\begin{remark} \rm
     Boehm~\cite{boehm:1990}\index{Boehm}
presents a construction identical to the cases
shown in Figure~\ref{fig:zero-real-2}(a) and (c). $\Box$
\end{remark}

     The double line in a linear intersection is counted twice.  Therefore,
either we have one double line with two real lines or one double line with zero
real lines.  The former case can be treated as a degenerate case of the four
real line case, by rotating one of the skeletal lines, $s$ in
Figure~\ref{fig:four-real},  about the common vertex 
clockwise until it is identical to 
a skeletal line from the other pair (Figure~\ref{fig:1-d-2-real}).  
The latter case is obtained from the two
real line case by rotating the line $s_1$ (resp., $s_2$) in
Figure~\ref{fig:two-real} clockwise (resp., counter clockwise) until it 
coincides with a skeletal line.  The resulting blending cyclides are shown
in Figure~\ref{fig:1-d-in} (resp., Figure~\ref{fig:1-d-out}).  
Note that in Figure~\ref{fig:four-real}(b), 
rotating $s$ makes the two principal circles tangent to the common line at the
same point.  This does not give a valid blending cyclide.  
Similarly, rotating $s_1$ (resp., $s_2$) in Figure~\ref{fig:two-real}(b)
(resp., Figure~\ref{fig:two-real}(a)), does not deliver a valid blending
cyclide either.  Therefore, each of these three cases has three families of
blending cyclides.  
These findings are summarized in Table~\ref{tbl:one-double-line}.
\begin{figure}
\vspace{4cm}
\caption{The Case of One Double Line and Two Real Lines}
\label{fig:1-d-2-real}
\end{figure}
\begin{figure}
\vspace{4cm}
\caption{The Case of One Double Line -- I}
\label{fig:1-d-in}
\end{figure}
\begin{figure}
\vspace{4cm}
\caption{The Case of One Double Line -- II}
\label{fig:1-d-out}
\end{figure}
\begin{table}
\caption{Types of Blending Cyclides and Principal Circles--One Double Line}
\label{tbl:one-double-line}
$$
\BeginTable
     \def\C{\JustLeft}
     \ninepoint
     \BeginFormat
     |5 c | c | c | c | c | c |5
     \EndFormat
     \_5
     | \use2 \it Real Lines |
          \it Figure | (a) | (b) | (c) |  \\ \_3
     | 2 "     | \ref{fig:1-d-2-real} | Vertical\ ${\cal H}_1$
               | Vertical  \ ${\cal H}_2$
               | Vertical  \ ${\cal H}_2$ | \\
     |\Lower{0} " Containing | \ref{fig:1-d-in} | Vertical \ ${\cal H}_1$
               | Vertical \ ${\cal R}$
               | Parabolic \ ${\cal H}_2$ | \\
     |         " Excluding | \ref{fig:1-d-out} | Horizontal \ ${\cal S}_1$
               | Parabolic \ ${\cal R}$
               | Horizontal \ ${\cal H}_2$ | \\ \_5
     " \use6 \C ${\cal R}$: ring, ${\cal H}_1$: single horned,
               ${\cal H}_2$: double horned, 
               and ${\cal S}_2$: two singularity spindle "  \\
\EndTable
$$
\end{table}

     If two cones have two double lines in their intersection, they must
exclude each other; otherwise, we have two identical cones.  Thus, this can be
viewed as a degenerate case of Figure~\ref{fig:1-d-out}.  Let one of the two
skeletal lines, which is not part of the double line, be rotated until it
coincides with the other.  Obviously, the cases in Figure~\ref{fig:1-d-out}(a)
and (c) do not generate a cyclide.  However, for Figure~\ref{fig:1-d-out}(b), 
the only principal circle degenerates to a point (Figure~\ref{fig:2-d}(a)).  
In this case, we have a parabolic single horned cyclide with that point 
as its only singularity (Figure~\ref{fig:2-d}).  
The other double line gives the second family.  
Therefore, for two cones with two double lines, there are two symmetric
families of blending cyclides and all of them are of 
parabolic singled horned type.
\begin{figure}
\vspace{4cm}
\caption{The Case of Two Double Lines}
\label{fig:2-d}
\end{figure}

     In summary, we have the following result.

\begin{theorem}
\label{thm:linear-inter-blending}
     Two cones with linear intersection always have a blending cyclide.
Any blending cyclide is a member of one of four (resp., three or two) families
of cyclides if the intersection contains zero (resp., one or two) double
lines.  These cyclides may be Dupin or parabolic.
\end{theorem}

% --------------------------------------------------------------------
%                            Conclusion
% --------------------------------------------------------------------

\section{Conclusion}
\label{section:cyc-concl}

     In this chapter, we have successfully presented a complete investigation
of the existence, use and organization of blending cyclides of two axial
natural quadrics.  The main contributions of this chapter are a new 
construction algorithm for blending cyclides, a complete and new proof that 
for almost all cases, two axial natural quadrics have planar intersection 
if and only if they have a blending cyclide, and a thorough study of how all 
of these blending cyclides are organized into families.
The necessary and sufficient conditions show
when and how we can use Dupin cyclides as blending surfaces.

% --------------------------------------------------------------------
%                    Offsets of a Dupin Cyclide
% --------------------------------------------------------------------

\section{Appendix: Offsets of a Dupin Cyclide}
\label{section:offsets}

     Offset surfaces\index{offset surface}\index{offset} 
have many applications such as
cutter path generation, motion planning, tolerance zones in mechanical design,
and blending.  In general, offset surfaces are more complicated than the
original surface; 
however, offsets of a Dupin cyclide are always Dupin cyclides, 
although they may be of different types.  In the geometric modeling community, 
R. R. Martin's\index{Martin} 
1982 Ph.D. thesis~\cite{martin:1982} is 
the first study of this property, 
while Pratt~\cite{pratt:1990}\index{Pratt} contains
a different proof (see also Zhou\index{Zhou} and 
Stra{\ss}er~\cite{zhou-strasser:1991}).\index{Stra{\ss}er}
All of these proofs are algebraic, based on the
equation(s) defining the cyclide.  In this appendix, as a simple application of
Cayley's\index{Cayley} method, 
we present a purely geometric proof that any offset of
a cyclide is a cyclide.  With the intuition used in this proof, the type 
variation of the offsets can be determined precisely.

     In the following, we shall prove the offset property with horizontal
principal circles.  By changing horizontal to vertical, the proof still works.
Furthermore, it also works for parabolic cyclides.

\begin{lemma}
\label{lemma:offset}
     The offset surfaces of a Dupin cyclide are Dupin cyclides.  Moreover,
all offsets have the same directrix conic.
\end{lemma}
{\bf Proof:}  Consider any $V$-circle of a cyclide.  By Cayley's method, there
must be a sphere, with radius $r$ and center $O$ on the directrix conic of the
horizontal symmetric plane, tangent to the cyclide along this circle.  
All lines joining $O$ and points on the common tangent circle give a right 
cone.  This cone may degenerate to a plane when $O$ becomes  the center of 
one of the two principal vertical circles.  Note that if $P$ is a point of the
circle, the normalized vector $\stackrel{\longrightarrow}{OP}$ is the normal 
vector of the sphere at $P$.  Since the sphere and the cyclide are tangent at
$P$, this vector is also the normal vector of the cyclide at $P$.
Thus, if $\epsilon$ is a signed distance, the offset point of $P$ is 
\[ \arrow{Q} = \arrow{P}+\frac{\epsilon}{|\stackrel{\longrightarrow}{OP}|}
          \stackrel{\longrightarrow}{OP}, \]
where $\arrow{P}$ and $\arrow{Q}$ are points $P$ and $Q$, respectively,
treated as vectors.  As $P$ moves along the circle, $Q$ generates another 
circle on the same cone.  Hence, as the sphere moves, another sphere, with the
same center and radius $r+\epsilon$, generates a cyclide which is the offset 
of the original.  Note that on the horizontal symmetric plane, one of the
principal circle's radius is increased by $\epsilon$, while the other's is
decreased by $\epsilon$.  Therefore, we have established the lemma. \QED

     Knowing how the offset surfaces of a given Dupin cyclide vary is
important to a designer, since this information may help him to pick up
a cyclide with the right type (e.g., no singularity, single horned, and
so on).  This problem can be studied with either horizontal or vertical
principal circles. 

     Consider the two vertical principal circles, with radii $r_1>r_2$ and
$d$ the distance between the centers.  
If the offset distance $\epsilon$ is positive, as $\epsilon$
is increasing, the offset principal circles will first become tangent to each
other and then always intersect in two points.  For the case of tangency,
$\epsilon=\frac{1}{2}[d-(r_1+r_2)]$, which is a one singularity spindle 
cyclide.  If $\epsilon$ is greater than this value, the cyclide becomes a two
singularity spindle cyclide.

     Note that for the case of $\epsilon<0$, if $r_i+\epsilon<0$, the actual
radius of the offset principal circle is $|r_i+\epsilon|$.  We have two 
critical cases: (1) $\epsilon=-r_2$ and (2) $\epsilon=-r_1$.
Both cases give a single horned cyclide.  If $-r_1<\epsilon<-r_2$, the
offset is a double horned cyclide.  There is a special case in this range,
namely $r_1+\epsilon=|r_2+\epsilon|$.  Or, equivalently, 
$\epsilon=-\frac{1}{2}(r_1+r_2)$.  The result is a double horned torus in
which the two principal circles have equal radii.  The type spectrum is 
symmetric about this point.  Table~\ref{tbl:type-var-V} summarizes our finding.
In the table, ${\cal R}$, ${\cal H}_1$, ${\cal H}_2$, ${\cal T}_2$,
${\cal S}_1$ and ${\cal S}_2$ denote a ring, single horned, double horned,
double horned torus, one singularity spindle and two singularity spindle
cyclide.  Using this table, with input $r_1,r_2,d$ and $\epsilon$, the type
of the offset for a ring cyclide
can be determined immediately by a simple table search.
If the given cyclide is not of ring type, simple adjustment is required and
its discussion is omitted.
\begin{table}
\caption{Type Changes of the Offsets of a Ring Cyclide Using Vertical Principal Circles}
\label{tbl:type-var-V}
$$
\BeginTable
     \OpenUp22
     \ninepoint
     \def\R{$\cal R$}
     \def\H{${\cal H}_1$}     
     \def\h{${\cal H}_2$}
     \def\S{${\cal S}_1$}
     \def\s{${\cal S}_2$}
     \def\T{${\cal T}_2$}
     \BeginFormat
     |5 c|c|c|c|c|c|c|c|c|c|c|c|5
     \EndFormat
     \_5
     | $\epsilon$ | $\cdots$ " $-\frac{1}{2}(d+\sigma)$ " $\cdots$ "
          $-r_1$ " $\cdots$ " $-\frac{1}{2}\sigma$ " $\cdots$ "
          $-r_2$ " $\cdots$ " $\frac{1}{2}(d-\sigma)$ " $\cdots$ | \\ \_3
     |\it Type | \s " \S " \R " \H " \h " \T " \h " \H " \R " \S " \s | \\
     \_5
     " \use{12} \JustLeft $\sigma=r_1+r_2$ " \\
\EndTable
$$
\end{table}

     If only horizontal principal circles are known, a similar table can be
constructed.  
Consider the two horizontal principal circles, with radii $R_1>R_2$ and
$D$ the distance between the centers.  Then, simple calculation gives
$r_1=\frac{1}{2}[D+(R_1-R_2)]$, $r_2=\frac{1}{2}[-D+(R_1-R_2)]$ and
$d=R_1+R_2$.  Therefore, $r_1+r_2=R_1-R_2$, $d+(r_1+r_2)=2R_1$ and
$d-(r_1+r_2)=2R_2$.  Plugging these identities into Table~\ref{tbl:type-var-V}
gives Table~\ref{tbl:type-var-H}.

\begin{table}
\caption{Type Changes of the Offsets of a Ring Cyclide Using Horizontal Principal Circles}
\label{tbl:type-var-H}
$$
\BeginTable
     \OpenUp22
     \ninepoint
     \def\R{$\cal R$}
     \def\H{${\cal H}_1$}     
     \def\h{${\cal H}_2$}
     \def\S{${\cal S}_1$}
     \def\s{${\cal S}_2$}
     \def\T{${\cal T}_2$}
     \BeginFormat
     |5 c|c|c|c|c|c|c|c|c|c|c|c|5
     \EndFormat
     \_5
     | $\epsilon$ | $\cdots$ " $-R_1$ " $\cdots$ " $-\frac{1}{2}(D+\Sigma)$ " 
          $\cdots$ " $-\frac{1}{2}\Sigma$ " $\cdots$ "
          $-\frac{1}{2}(-D+\Sigma)$ " $\cdots$ " $R_2$ " $\cdots$ | \\ \_3
     |\it Type | \s " \S " \R " \H " \h " \T " \h " \H " \R " \S " \s | \\
     \_5
     " \use{12} \JustLeft $\Sigma=R_1-R_2$ " \\
\EndTable
$$
\end{table}

     The case of parabolic is simple.  Let $r$ be the radius of
the principal circle and $d$ the distance from this principal circle to the
principal line on the same symmetric plane.  Table~\ref{tbl:type-var-para} 
summarizes the desired result.
\begin{table}
\caption{Type Changes of the Offsets of a Parabolic Ring Cyclide}
\label{tbl:type-var-para}
$$
\BeginTable
     \OpenUp22
     \ninepoint
     \def\R{$\cal R$}
     \def\H{${\cal H}_1$}     
     \def\h{${\cal H}_2$}
     \BeginFormat
     |5 c|c|c|c|c|c|5
     \EndFormat
     \_5
     | $\epsilon$ | $\cdots$ " $-\frac{d}{2}$ " $\cdots$ " $r$ " $\cdots$ |
                                                  \\ \_3
     |\it Type | \h " \H " \R " \H " \h | \\
     \_5
\EndTable
$$
\end{table}
