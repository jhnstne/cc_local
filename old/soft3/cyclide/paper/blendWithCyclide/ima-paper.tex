\documentstyle[conf]{article}

\newtheorem{remark}{Remark}[section]
\newtheorem{cor}{Corollary}[section]
\newcommand{\QED}{\vrule height 1.4ex width 1.0ex depth -.1ex\ } % square box

\input{table}

\title{
Dupin cyclides as blending surfaces for cones
}

\author{
John~K.~Johnstone \and C.-K.~Shene
\affil The Johns Hopkins University
}

\shortauthor{Johnstone and Shene}

\begin{document}

\maketitle

\section{Introduction}

A surface blends two surfaces if it is tangent to them, each along a curve.
Blending surfaces are used to smooth the intersection of two surfaces
or to connect two disjoint surfaces.
Two quadric surfaces can always be blended by a quartic 
surface \cite{hoffmann-hopcroft:1986,hoffmann-hopcroft:1987}.
However, an arbitrary quartic blending surface is not necessarily optimal,
although its degree is minimal in general \cite{hoffmann-hopcroft:1986}, 
since it will not be clear what its geometry is.
The Dupin cyclide is a quartic surface that has been recognized as a
good surface for blending 
\cite{boehm:1990,chandru-dutta-hoffmann:1990,dutta:1989,pratt:1989,pratt:1990} 
(e.g., as a variable radius blend or a natural candidate for blending pipes)
and, importantly, it is 
very well understood geometrically and algebraically.
This is useful in order to anticipate what the blending surface will 
look like.
The Dupin cyclide is in general nonsingular (and has at worst two
singularities), which is another advantage for blending.
Finally, it is feasible to include Dupin cyclides 
as primitives in a solid modeling vocabulary 
along with the quadric and torus (see \cite{jj92}),
more so than it is to include all quartic surfaces.
% we explore blending cones

In this paper, we fully explore the use of Dupin cyclides to blend cones
(one of the most widely used surfaces).
We characterize when this is possible, and show how to construct all of the
possible blends.
The blending of a cone 
by a cyclide has been studied by Pratt, Sabin, and Boehm
\cite{pratt:1989,pratt:1990,boehm:1990}.
Two cones cannot always be blended by a cyclide.
Pratt \cite{pratt:1990} 
examined a necessary and sufficient condition for two cones
to have a cyclide blending surface (credited to Sabin), 
using a common inscribed sphere condition.
However, Pratt includes only the proof for sufficiency, using the offset 
properties of the cyclide.
The common inscribed sphere condition also implicitly assumes nondegeneracy:
thus, two cones {\em with nonparallel axes and distinct
vertices} can be blended by a Dupin cyclide if and only if
the cones have a common inscribed sphere.

We reexplore this problem, yielding some interesting new results.
We find a slightly different necessary and sufficient condition 
that deals with all cases (including degeneracies), 
and give a full proof of its necessity and sufficiency.
In particular, we prove that two cones can be blended by a Dupin cyclide
(either nondegenerate or degenerate)
if and only if they have planar intersection.
Our condition relates to Pratt's condition as follows:
for two cones with intersecting axes,
the existence of a common inscribed sphere
is equivalent to planar intersection that contains at least one conic.
We also show that the condition has one interesting exception.
We illustrate the need for parabolic cyclides to cover degenerate cases.

Given two cones with planar intersection, there is not just one cyclide
that blends them, there are an infinite number.
The second major contribution of this paper is a method of controlling
the choice of blending cyclide,
which is useful in choosing an optimal blending cyclide for the desired shape.
We identify all of the possible blending cyclides, classify them 
into families,
and give a method for constructing any specific member from a family.
This is done by making a constructive proof of the sufficiency condition.

This paper shows both the strengths and weaknesses of the Dupin cyclide as
a blending surface.
Its weakness is that, as a restricted class of surfaces, it can only blend
in restricted cases.
(Note however, the importance of the restricted class of `cones with planar
intersection' in practical design, and the potential for blending more general
cases by using more than one cyclide.)
Its strength is that one has a large amount of control over the final result,
in both anticipating the geometry of the blending surface and 
controlling the choice of cyclide from the families of potential blending
cyclides.
Its other strength is that the cyclide is a more tractable primitive in
a solid model than the general quartic surface.

We should note that two cones with planar intersection can also be blended by 
a quadric \cite{warren:1987}.
However, there are many configurations for the blend that can only be
achieved by cyclide blends.
Examples of such blends will be given in Section~\ref{section:eg}.

All of our results also apply directly 
to the blending of a cone and a cylinder, or two
cylinders, but we restrict to cones for brevity.
We also sketch or omit some proofs for this paper.
A full development can be found in Shene \cite{shenethesis}.

Section~\ref{section:basic} reviews the basic properties of both
the quartic and (degenerate) cubic Dupin cyclide,
and presents some useful technical lemmas.
Section~\ref{section:interpolating} develops several 
restrictions on the relative positions of the cones and cyclide
when the cones can be blended by a cyclide, ending with the necessity
of planar intersection.
Section~\ref{section:cyclide-non-degen} proves the
sufficiency of planar intersection,
with the sole exception highlighted in Section~\ref{section:line-conic}.
The proof is constructive: the blending cyclide is actually created.
In this section, we also show how the families of blending
cyclides that blend two given cones can be controlled by diagonals.
A definition of a nondegenerate Dupin cyclide by three circles is developed in 
Section~\ref{section:spec} and used to construct the blending cyclide.
Section~\ref{section:eg} ends with a number of examples of cones blended
by Dupin cyclides.

\section{Dupin Cyclides}
\label{section:basic}

The Dupin cyclide is a generalization of the torus (Figure~\ref{fig:cyclides}).
It has been the subject of a substantial amount of recent
research, including the work of Martin and others on cyclide patches
\cite{depont:1984,martin:1982,nutbourne-martin:1988,sharrock:1985}, 
the work of Pratt and Boehm on blending with cyclides 
\cite{boehm:1990,pratt:1989,pratt:1990}, 
the work of Chandru, Dutta and Hoffmann 
on the more general blending of arbitrary surfaces by a network of cyclides 
\cite{chandru-dutta-hoffmann:1989,chandru-dutta-hoffmann:1990,dutta:1989},
the work of one of the present authors on intersection with cyclides
\cite{jj92}, and many other papers.

\begin{definition}
A {\bf Dupin cyclide} is a quartic surface whose two families of lines of 
curvature are all circles.  
\end{definition}

Dupin's original definition is the envelope of a sphere that is tangent to
three fixed spheres in a continuous manner \cite{maxwell:1868}.

% use entire figure, but shrunk
\begin{figure}
\vspace{18.5cm}
\caption{Dupin Cyclides (taken from [39] with permission)}
\label{fig:cyclides}
\end{figure}

\begin{definition}
The Dupin cyclide has two planes of symmetry, which both intersect
the cyclide in two circles.\footnote{One of the circles may degenerate
	to a point.}
In one of these planes, which we call the {\bf horizontal symmetric plane},
the circles either intersect or lie one inside the other;
while in the {\bf vertical symmetric plane}, the circles either intersect
or lie one outside the other.\footnote{There is no trouble distinguishing
	the two planes since, in at least one of the planes, the circles
	do not intersect.}
The horizontal and vertical symmetric planes are perpendicular.
The circles in the horizontal (resp., vertical) symmetric plane are called
{\bf horizontal} (resp., {\bf vertical}) {\bf principal circles} in this 
paper, and their radical axis\footnote{The radical axis of two 
	nonconcentric circles is the line such that
	the lengths of the tangents from any point of the line to both
	circles are equal.}
is called the {\bf horizontal} (resp., {\bf vertical}) {\bf radical axis}.
\end{definition}

There are several types of Dupin cyclide, based on the configuration of
the principal circles in the symmetric planes:
ring, singly horned, doubly horned, and spindle (with one or two 
singularities).  See Figure~\ref{fig:cyclides}(a)-(e), respectively.

\begin{definition}
\label{defn:HV}
The two families of lines of curvature of a Dupin cyclide will be called
{\bf horizontal} (or {\bf $H$-}) {\bf circles} and
{\bf vertical} (or {\bf $V$-}) {\bf circles}.
The planes that contain the $H$-circles (resp., $V$-circles)
all intersect in a line that is perpendicular to the vertical 
(resp., horizontal) symmetric plane and passes through a center of
similitude\footnote{See Definition~\ref{defn:cofs}.}
of the vertical (resp., horizontal) circles.
(This leads to a method of generating the cyclide.)
\end{definition}

There is a useful relationship between the $H$-circles and the vertical
radical axis.

\begin{lemma}
\label{lemma:on-radical-axis}
     All of the tangent planes of a Dupin cyclide along an $H$-circle 
(resp., $V$-circle) meet at a point on the vertical (resp., horizontal)
radical axis.
\end{lemma}
{\bf Proof:}
Consider a horizontal (resp., vertical) circle $D$.  There is a
sphere tangent to the cyclide along $D$.  Since this sphere and the cyclide are
tangent along $D$, any tangent plane of the cyclide is also a tangent plane of
the sphere.  However, all tangent planes along a circle on a sphere envelope a
cone and thus meet at a point $V$.  $V$ may be at
infinity, if $D$ is a great circle of the sphere.  The tangent lengths from
$V$ to the sphere and thus to the cyclide are all equal.  Because of symmetry,
$V$ lies on the vertical (resp., horizontal) plane and hence must be a point
on the vertical (resp., horizontal) radical axis. 
\QED

A full treatment of the blending of cones by Dupin cyclides requires
parabolic cyclides as well.

\begin{definition}
A {\bf parabolic} (or {\bf cubic}) cyclide is a degenerate Dupin cyclide
of degree three, where one circle from each family of lines of curvature
degenerates to a line (Figure~\ref{fig:den-cyclides}).
\end{definition}

% use entire figure, but shrink
\begin{figure}
\vspace{12cm}
\caption{Parabolic Dupin Cyclides (taken from [39] with permission)}
\label{fig:den-cyclides}
\end{figure}

\begin{definition}
The parabolic cyclide again has two planes of symmetry.
When a parabolic cyclide is placed in normal form
these are the $yz$- and $xz$-planes, which we call the {\bf horizontal}
and {\bf vertical symmetric planes}, respectively.
Each plane of symmetry intersects the parabolic cyclide in a circle,
the {\bf horizontal} (or {\bf vertical}) {\bf principal circle},
and a line, the {\bf horizontal} (or {\bf vertical}) {\bf principal line}.
We can again distinguish two families of circles on the parabolic cyclide,
the {\bf $H$-} and {\bf $V$-circles}.
The planes that contain the $H$-circles (resp., $V$-circles)
all intersect in the vertical (resp., horizontal) principal line.
\end{definition}

Because of the lack of a radical axis, 
a different form of Lemma~\ref{lemma:on-radical-axis} is needed.

\begin{lemma}
\label{lemma:on-principal-line}
     All of the tangent planes of a parabolic 
cyclide along an $H$-circle (resp., 
$V$-circle) meet at a point on the vertical (resp., horizontal) principal line.
\end{lemma}
{\bf Proof:} See \cite[Lemma~5.4]{shenethesis}. 
\QED

In the rest of the paper, we will use the shorthand term
`cyclide' for the quartic Dupin cyclide and
`parabolic cyclide' for the cubic degenerate Dupin cyclide.
We end this section with a technical lemma.

\begin{definition}
\label{defn:cofs}
     Given two nonconcentric circles, with centers $O_1$ and $O_2$ and 
distinct radii $r_1$ and $r_2$, the point $S\in\overline{O_1O_2}$
(resp., $S \in \stackrel{\longleftrightarrow}{O_1O_2}-\overline{O_1O_2}$) 
satisfying 
$\frac{|\overline{O_1S}|}{|\overline{SO_2}|}=\frac{r_1}{r_2}$ is called the
{\bf internal} (resp., {\bf external}) {\bf center of similitude}.  
\end{definition}

\begin{lemma}
\label{cor:tangent-center}
From any point on the radical axis of two circles, construct a tangent to each
circle giving two tangent points.  
The line joining these two tangent points passes through a center of 
similitude of the circles.
\end{lemma}
{\bf Proof:}
See \cite[Lemma~5.1 and Corollary~5.1]{shenethesis}.
\QED

\section{Necessary Conditions}
\label{section:interpolating}

If a cyclide is to blend two cones, the cones must be in a special position.
In this section, we establish many useful restrictions on the relative
positions of the cones and cyclide, culminating with the condition that
the two cones must have planar intersection.
We begin with a restriction on the tangent curves of the blend.

\begin{lemma}
\label{lemma:tan}
If a quadric and a Dupin\footnote{The Dupin 
		cyclide can be nondegenerate or degenerate.}
cyclide are tangent along a curve, this curve
must be a circle (in particular, an $H$-circle or $V$-circle of the cyclide).
\end{lemma}
{\bf Proof:}
Given a cyclide and a quadric, consider all of the planes that are tangent
to both surfaces.
Humbert has shown that on the cyclide, the tangent points of these planes
define a line of curvature of the cyclide \cite{humbert:1885,jessop:1916}.
Thus, if the quadric and cyclide are tangent along a curve, since
the tangent planes of the surfaces must coincide along this curve, the curve
must be a line of curvature.
The lines of curvature of a Dupin cyclide are the $H$- and $V$-circles.
\QED

\begin{cor}
If a Dupin cyclide blends two quadrics, the curves of tangency must be two
$H$-circles or two $V$-circles of the cyclide.
\end{cor}

Note that the curves of tangency cannot be from different families of circles,
otherwise they intersect.
This result limits the type of clipping curves that can be used in blending
cones with cyclides.
Restrictions on clipping curves are not new: they also arise in the blending
of quadrics by quadrics, as demonstrated by Warren \cite{warren:1987}.

\begin{remark}
Lemma~\ref{lemma:tan} verifies Pratt's observation that the 
cone/torus intersection of the Cranfield object
cannot be blended by a Dupin cyclide.
The cyclide cannot blend as desired if it must meet the cone in a circle.
See Pratt~\cite{pratt:1990} for pictures and a discussion of the Cranfield 
object.
\end{remark}

The choice of a cyclide as blending surface also forces the two 
cones into a special relative position.

% START HERE

\begin{lemma}
\label{lemma:coplanar}
Two cones can be blended by a Dupin\footnote{The Dupin 
		cyclide can be nondegenerate or degenerate.}
cyclide only if their axes are coplanar.
\end{lemma}
{\bf Proof:}  Suppose that two cones can be blended by a
cyclide.  The cones are tangent to the
cyclide along two circles $C_1$ and $C_2$ from the same family,
say both $H$-circles.  From Lemma~\ref{lemma:on-radical-axis} 
(resp., Lemma~\ref{lemma:on-principal-line} if the cyclide is parabolic),
each cone's vertex lies on the vertical radical axis 
(resp., vertical principal line) of the cyclide.  
In particular, the cone vertices 
lie in the vertical symmetric plane.  This plane cuts
each $H$-circle symmetrically, and thus
contains the centers of $C_1$ and $C_2$.  
Therefore, the vertical symmetric plane contains both cone axes,
since the axis of a cone is the line joining the cone's
vertex and the center of any circle on the cones (say $C_1$ and $C_2$),
both of which lie in the vertical symmetric plane.
\QED

Therefore, when blending with cyclides, we can speak of the
{\bf axial plane} of two cones.
We shall reduce most of our reasoning and construction to this plane.

The following result can be extracted from the proof of the lemma.
It shows that, once the cones are defined, the position of the blending
cyclide is already quite constrained.

\begin{cor}
\label{cor:on-radical-axis}
If two cones are blended by a Dupin cyclide,
the axial plane of the cones must be a symmetric plane of the cyclide,
and the cone vertices must lie on the radical axis of the symmetric plane.
\end{cor}

\noindent In Section~\ref{section:cyclide-non-degen}, 
we shall identify the degrees of freedom of 
the cyclide that remain once one of the symmetric planes and radical axes
are fixed by Corollary~\ref{cor:on-radical-axis}.

A stronger restriction to the relative position of the cones
is possible if the blending cyclide is parabolic.

\begin{lemma}
\label{cor:para->common-line}
Two cones can be blended by a parabolic cyclide only if
all three surfaces share a common line.
\end{lemma}
{\bf Proof:} 
See \cite[Corollary~5.4]{shenethesis}.
\QED

We can go one more step in strengthening the restriction on the relative
position of two cones that can be blended by a cyclide: they must have
planar intersection.
As we shall show in later sections, this is the strongest restriction
possible.
We first need to define the geometry of the cones' axial plane.

\begin{definition}
If the axes of two cones are distinct and coplanar, they determine a plane
called the {\bf axial plane}.
This plane cuts each cone in a pair of straight lines, called the 
{\bf skeletal pair}.
The lines in a skeletal pair are called {\bf skeletal lines}.
If neither cone vertex lies on a skeletal line, the skeletal lines have
four non-vertex intersection points, called {\bf diagonal points},
which define two non-skeletal lines, called {\bf diagonals}.
See Figure~\ref{fig3.1}.
The plane through a diagonal and perpendicular to the axial plane is called a
{\bf diagonal plane}.
\end{definition}

% This is figure 3.1 from thesis: use (a), (b) and (c) (last two are overkill)
\begin{figure}
\vspace{5cm}
\caption{Skeletal lines and diagonals}
\label{fig3.1}
\end{figure}

\begin{theorem}
Two cones can be blended by a Dupin cyclide only if they have 
planar intersection.
\end{theorem}
{\bf Proof (sketch):}
By Lemma~\ref{lemma:coplanar}, we can assume that the cones have coplanar axes.
Some additional conditions immediately imply planar intersection:
common vertices, common skeletal line,
or both lines in one skeletal pair parallel to those in the other pair
(see \cite[Section~5.6]{shenethesis}).
Therefore, we will assume distinct vertices, distinct skeletal lines, and
at most one line from each skeletal pair are parallel.
From Lemma~\ref{cor:para->common-line}, we may further assume that the cyclide
is not parabolic, since the common line also implies planar intersection.

It can be shown that at least one of the two diagonals, say 
$\stackrel{\longleftrightarrow}{RS}$,
has two finite diagonal points, say $R$ and $S$ 
\cite[Lemmas~5.7 and 5.8]{shenethesis}.
We will show that the diagonal plane through 
$\stackrel{\longleftrightarrow}{RS}$ intersects the two cones in the same 
conic.
Since diagonal points $R$ and $S$ are finite, the intersection 
conics in the diagonal plane through $\stackrel{\longleftrightarrow}{RS}$ 
must be an ellipse or a hyperbola.
Since these intersection conics lie in the diagonal plane and are central,
it suffices to show that the conics have the same type and the same 
focal length (see \cite[Theorem 4.2]{shenethesis}).
(They automatically have the same major and minor axes by virtue
of the fact that they lie in the diagonal plane.)

Recall that the axial plane of the cones must be a symmetric plane of the
cyclide (Lemma~\ref{cor:on-radical-axis}).
It is easy to see that the radical axis of this symmetric plane
lies inside both of the cones
if the axial plane is the vertical symmetric plane, and outside both
otherwise (Figure~\ref{fig:int-ext}).
Thus, the line between the cone vertices $V_1$ and $V_2$, 
which is this radical axis by
Lemma~\ref{cor:on-radical-axis}, lies inside both or outside both cones.
This reduces the number of possible configurations of the two cones,
and in turn makes it clear that the diagonal plane through 
$\stackrel{\longleftrightarrow}{RS}$ intersects both cones in the same type
of conic (see Figure~\ref{fig:configs}).

% use only the ring column
\begin{figure}
\vspace{8cm}
\caption{The Radical Axis Lies in Both Interiors or Both Exteriors}
\label{fig:int-ext}
\end{figure}

% use entire thing: if we get stretched for space, use only first column
\begin{figure}
\vspace{7cm}
\caption{The six possible placements of $\stackrel{\longleftrightarrow}{V_1V_2}$}
\label{fig:configs}
\end{figure}

We finish by showing that the focal lengths of these conics
are equal.
It can be shown that the focal lengths of the ellipse and hyperbola are
$f_E=\frac{1}{2}|\ |\overline{V_iR}|-|\overline{V_iS}|\ |$ and
$f_H=\frac{1}{2}(|\overline{V_iR}|+|\overline{V_iS}|)$, respectively
\cite[Lemma~4.1]{shenethesis}.
Let $\stackrel{\longleftrightarrow}{V_1R}$ and
$\stackrel{\longleftrightarrow}{V_2R}$ (resp.,
$\stackrel{\longleftrightarrow}{V_1S}$ and
$\stackrel{\longleftrightarrow}{V_2S}$) be tangent to $C_1$ (resp., $C_2$) at
$A$ and $C$ (resp., $B$ and $D$) respectively, where $C_1$ and $C_2$ are the 
principal circles in the axial plane (Figure~\ref{fig:Q-RS}). 
The focal length equality follows immediately from the following identities:
$|\overline{V_1A}|=|\overline{V_1B}|$ and 
$|\overline{V_2C}|=|\overline{V_2D}|$ (since $V_1$ and $V_2$ lie on the radical
axis of $C_1$ and $C_2$ by Corollary~\ref{cor:on-radical-axis}),
$|\overline{RA}|=|\overline{RC}|$ and $|\overline{SB}|=|\overline{SD}|$.
(Consider the ellipse example in Figure~\ref{fig:Q-RS}.)
\QED

% use only first of two pictures
\begin{figure}
\vspace{7cm}
\caption{The Axial Plane, with Principal Circles}
\label{fig:Q-RS}
\end{figure}

\begin{cor}
In general,
%(nonparabolic blending cyclide, distinct vertices,
%distinct skeletal lines, skeletal lines in one pair not both parallel
%to the lines in other pair),
the plane that contains the intersection
is the diagonal plane through the diagonal with two finite diagonal points.
\end{cor}

\section{Planar Intersection is Sufficient}
\label{section:cyclide-non-degen}

We now want to show that the planar intersection condition is also sufficient:
two cones can be blended by a Dupin cyclide whenever
they have planar intersection, with only one exception.
This exception is
two cones with parallel axes and a double line and conic in their intersection.
We will deal with the three cases of planar intersection separately:
a planar intersection of two conics in Section~\ref{ssection:nondegn}, 
a line and a conic in Section~\ref{section:line-conic}, 
and a linear intersection in which all 
intersection curves are lines in Section~\ref{section:cyc-linear}.
We start by showing how to define the blending cyclide in the next subsection.

\subsection{Specifying a cyclide}
\label{section:spec}

Given two cones with planar intersection, we need to be able to specify
a Dupin cyclide that will blend them.  
A quartic Dupin cyclide is uniquely defined by its principal circles in one
symmetric plane, the type\footnote{This type may be fully determined by the
	principal circles ({\it e.g.}, if one circle lies inside the other, 
	the type must be horizontal).}
of this symmetric plane (horizontal or vertical),
and the type\footnote{Again, some of these types may be ruled out 
	by the principal circles ({\it e.g.}, if they intersect, the cyclide
	cannot be ring).} 
of the cyclide (ring, singly horned, etc.) \cite{jj92}.
However, we do not want to involve type information in the definition:
we want a clean geometric definition by three circles, the third circle
replacing the type information.
The following lemma shows how.

\begin{lemma}
\label{lemma:non-degen-construction}\ 
\begin{enumerate}
     \item Let $C_1$ and $C_2$ be two circles on a plane ${\cal P}$
	(Figure~\ref{fig:specification-nondegen}).
     \item Let $\ell$ be a line perpendicular to ${\cal P}$ and through one of
          the two centers of similitude of $C_1$ and $C_2$.  
     \item Let $D$ be a circle, with center in ${\cal P}$ and lying in a plane
          through $\ell$, meeting $C_1$ and $C_2$ at $E_1$ and $E_2$, 
          respectively, such that the tangents of $C_i$ at $E_i$ ($i=1,2$) 
          meet at a point on the radical axis of $C_1$ and $C_2$.  
\end{enumerate}
Then, there exists a unique cyclide containing $C_1,C_2$ and $D$.  This cyclide
is denoted by ${\cal Z}(C_1,C_2,D,\ell)$.  
${\cal P}$ is a symmetric plane in which $C_1$ and $C_2$ are principal circles.
\end{lemma}
{\bf Proof (sketch):}
Consider all of the cyclides with principal circles $C_1$ and $C_2$.
We must show that exactly one contains $D$.
Another way of defining a Dupin cyclide is as follows:
the envelope of a sphere with
center on a plane (which becomes a symmetric plane) and tangent to two
circles in this plane (which become the principal circles) 
\cite{chandru-dutta-hoffmann:1989}.
The locus of the sphere's center describes a conic, called a 
{\bf directrix conic}.
It can be shown that, if the two principal circles $C_1$ and $C_2$ are fixed
(with centers $O_1$ and $O_2$, radii $r_1$ and $r_2$),
there are at most four choices of directrix conic: 
if $C_1$ and $C_2$ are horizontal principal circles, the directrix conic
can be an ellipse with foci $O_1$ and $O_2$ (where $O_1$ and $O_2$ are the
centers of $C_1$ and $C_2$) such that the sum of distances from the foci
is $r_1 + r_2$ or $|r_1 - r_2|$;
if $C_1$ and $C_2$ are vertical principal circles, the directrix conic
can be a hyperbola with foci $O_1$ and $O_2$
and difference of distances $r_1 + r_2$ or $|r_1 - r_2|$ 
(see \cite[Section~5.4.2]{shenethesis}).
We will use this definition to show that there is a unique cyclide with
principal circles $C_1$ and $C_2$ that contains $D$.

% use it exactly the same 
\begin{figure}
\vspace{4.5cm}
\caption{Specifying a Dupin Cyclide}
\label{fig:specification-nondegen}
\end{figure}

Let $V$ be the intersection point of the two tangents at $E_1$ and $E_2$
(Figure~\ref{fig:specification-nondegen}).  
Since $V$ lies on the radical axis, 
$|\overline{VE_1}|=|\overline{VE_2}|$ and there exists a circle $C$
tangent to $\stackrel{\longleftrightarrow}{VE_i}$ at $E_i$,
say with center $O$ and radius $r$.
By revolving the circle $C$ about $\stackrel{\longleftrightarrow}{VO}$,
we get a sphere $S$ tangent to $C_1$ and $C_2$ at $E_1$ and $E_2$,
which also has center $O$.
It can be shown that $O$ lies on exactly one of the four possible directrix
conics for the principal circles $C_1$ and $C_2$ (see above),
depending on the relative position of $C$, $C_1$ and $C_2$ 
\cite[Lemma~5.3]{shenethesis}.
Let this directrix conic be $DC$, and let the cyclide defined by
$C_1$, $C_2$, and $DC$ be ${\cal C}$.
Notice that the sphere $S$ is one of the spheres with center on $DC$
and tangent to $C_1$ and $C_2$ (at $E_1$ and $E_2$).
$D$ lies on $S$, and $D$ is also tangent to $C_1$ and $C_2$ 
at $E_1$ and $E_2$.
Thus, it is easy to show that $D$ lies on the cyclide ${\cal C}$ and
${\cal Z}(C_1,C_2,D,\ell) = {\cal C}$.
The uniqueness of the cyclide containing $C_1$, $C_2$ and $D$ is simple,
and its proof is omitted.
% This is where choice of center of similitude comes in
\QED

\begin{remark} \rm
The conditions on the third circle $D$ are important.
It is simple to find a cyclide that cannot contain all three circles if
the conditions are violated (Figure~\ref{fig:wrong-example}).
\end{remark}

% use it (Figure 5.11 in red book called '... does not meet condition (3)')
\begin{figure}
\vspace{5.5cm}
\caption{No Cyclide Contains These Three Circles}
\label{fig:wrong-example}
\end{figure}

\noindent We have a similar method of defining parabolic cyclides.

\newpage

\begin{lemma}
\label{lemma:para-construction} \ 
\begin{enumerate}
     \item Let $\ell$ and $C$ be a line and a circle on a plane ${\cal P}$.  
     \item Let $P\not\in\ell$ be an intersection point of $C$ and the 
          line through the center of $C$ and perpendicular to $\ell$.  
\end{enumerate}
Then, there exists a unique parabolic cyclide that contains $\ell$ and $C$,
and has a principal line through $P$ and perpendicular to ${\cal P}$.  
This cyclide is denoted by ${\cal Z}_P(\ell,C,P)$.
${\cal P}$ is a symmetric plane in which $\ell$ and $C$ are the 
principal line and principal circle, respectively.  
\end{lemma}
{\bf Proof:}
See \cite[Lemma~5.5]{shenethesis}.
\QED

% **********************************************************************

\subsection{Nondegenerate Planar Intersection}
\label{ssection:nondegn}

Suppose that two cones have planar intersection.  
In general ({\it i.e.}, if the four skeletal lines in the axial plane
are distinct), this intersection will consist of two conics
(including possibly an infinite one).
We shall deal with this case in this section.  
In the following subsection, 
we also assume that the quadrics' axes are intersecting.
The parallel axes case is dealt with in 
Section~\ref{section:cyc:parallel-axes}.

\subsubsection{Intersecting Axes}
\label{section:intersecting-axes}

A major advantage of the following proof is that it directly reveals
a construction algorithm for the blending cyclide.

\begin{lemma}
\label{lemma:int-cyclide}
If two cones with intersecting and distinct\footnote{If the axes coincide, 
	it is easy to show that there are two families of
	blending tori.}
axes have planar intersection consisting of two conics, they can be blended
by a Dupin cyclide.
\end{lemma}
{\bf Proof:}
For brevity, we shall only consider the general case where the two cones
do not have two parallel skeletal lines
(i.e., the diagonal points R and S below are both finite).
If one of the diagonal points is at infinity, the construction is similar
(see \cite[Lemma~5.13]{shenethesis}).

We will define two cyclides, one tangent to each cone, and then show that
they are the same cyclide, which therefore blends the cones.
First, we introduce the basic configuration on the axial plane for our
construction.
Consider two cones that intersect in two conics, with vertices $V_1$ and
$V_2$ and intersecting axes $\ell_1$ and $\ell_2$.
Two cones with conic intersection must have coplanar axes 
\cite{shene-johnstone:1991a}
and thus the axial plane is well defined.
Let $\stackrel{\longleftrightarrow}{RS}$ and 
$\stackrel{\longleftrightarrow}{R^\prime S^\prime}$ be the diagonals in the 
axial plane, where $R$, $S$, $R^\prime$ and $S^\prime$ are diagonal 
points.\footnote{One of $R$ and $S$ (or one of $R^\prime$ and $S^\prime$)
	may be at infinity, in which case the diagonal is parallel to one
	skeletal line from each skeletal pair.}
Since the cones have conic intersection, they must have a common inscribed
sphere \cite{shene-johnstone:1991a}, 
which the axial plane intersects in a circle.
Let the skeletal lines 
$\stackrel{\longleftrightarrow}{V_1R}$, $\stackrel{\longleftrightarrow}{V_1S}$,
$\stackrel{\longleftrightarrow}{V_2R}$ and 
$\stackrel{\longleftrightarrow}{V_2S}$ be tangent to the circle at $A, B, C$ 
and $D$ respectively (Figure~\ref{fig:int-config}).  By Brianchon's 
theorem,\footnote{Brianchon's theorem states that the 
	three opposite diagonals of a
	hexagon circumscribing a conic intersect in a point 
	\cite{johnson:1929}.}
$\stackrel{\longleftrightarrow}{AB}$, $\stackrel{\longleftrightarrow}{CD}$,
$\stackrel{\longleftrightarrow}{RS}$ and
$\stackrel{\longleftrightarrow}{R^\prime S^\prime}$ meet at a point, say $X$.

% use both pictures in figure (to show that we cover vertices inside or out)
\begin{figure}
\vspace{6.5cm}
\caption{The Basic Configuration in the Axial Plane}
\label{fig:int-config}
\end{figure}

Next, we present our construction of the blending cyclide.
For each cone, we must find three circles that define a cyclide tangent
to the cone, and then we must show that these two cyclides are equivalent.
Let $X^\prime$ be any point on $\stackrel{\longleftrightarrow}{RS}$
(Figure~\ref{fig:int-config}).
Through $X^\prime$ construct a line perpendicular to $\ell_1$ 
meeting $\stackrel{\longleftrightarrow}{V_1A}$ and
$\stackrel{\longleftrightarrow}{V_1B}$ at $A^\prime$ and $B^\prime$
respectively.
Through $X^\prime$ construct another line perpendicular to $\ell_2$ 
meeting $\stackrel{\longleftrightarrow}{V_2C}$ and
$\stackrel{\longleftrightarrow}{V_2D}$ at $C^\prime$ and $D^\prime$
respectively.

We claim that $|\overline{RA^\prime}|=|\overline{RC^\prime}|$
and thus there is a (unique) circle $O_1$ on the axial plane 
tangent to $\stackrel{\longleftrightarrow}{V_1R}$ and 
$\stackrel{\longleftrightarrow}{V_2R}$ at $A^\prime$ and $C^\prime$.
Since $\bigtriangleup XRA \sim\bigtriangleup X^\prime RA^\prime$,
$\frac{|\overline{RA}|}{|\overline{RA^\prime}|}=
\frac{|\overline{RX}|}{|\overline{RX^\prime}|}$.
Similarly, from
$\bigtriangleup XRC \sim \bigtriangleup X^\prime RC^\prime$, 
$\frac{|\overline{RC}|}{|\overline{RC^\prime}|}=
\frac{|\overline{RX}|}{|\overline{RX^\prime}|}$.
Since $A$ and $B$ are both tangent points from $R$, 
$|\overline{RA}|=|\overline{RC}|$.
Thus, $|\overline{RA^\prime}|=|\overline{RC^\prime}|$.
By the same argument, there is also a (unique) circle 
$O_2$ tangent to $\stackrel{\longleftrightarrow}{V_1S}$ and
$\stackrel{\longleftrightarrow}{V_2S}$ at $C^\prime$ and $D^\prime$.

Let $D_{\overline{A^\prime B^\prime}}$, the third circle in the definition of
the cyclide, be the circle with diameter
$\overline{A^\prime B^\prime}$ and perpendicular to the axial plane.
Finally, we show that $X^\prime$ is a center of similitude
of the circles $O_1$ and $O_2$.
Since $|\overline{V_1A}|=|\overline{V_1B}|$ and
$\stackrel{\longleftrightarrow}{AB}$ is parallel to
$\stackrel{\longleftrightarrow}{A^\prime B^\prime}$,
$|\overline{V_1A^\prime}|=|\overline{V_1B^\prime}|$; and similarly,
$|\overline{V_2C^\prime}|=|\overline{V_2D^\prime}|$.
Therefore, $V_1$ and $V_2$ lie on the radical axis of $O_1$ and $O_2$.
Using Lemma~\ref{cor:tangent-center}, a center of similitude of 
$O_1$ and $O_2$ lies on both 
$\stackrel{\longleftrightarrow}{A^\prime B^\prime}$ and
$\stackrel{\longleftrightarrow}{C^\prime D^\prime}$,
which intersect at $X^\prime$.
Thus, $X^\prime$ is a center of similitude.
Let $\ell_{X^\prime}$ be the line through $X^\prime$ and perpendicular to the 
axial plane. 

By Lemma~\ref{lemma:non-degen-construction}, there is a unique 
Dupin cyclide
${\cal Z}(O_1,O_2,D_{\overline{A^\prime B^\prime}},\ell_{X^\prime})$
that contains the circles $O_1, O_2$ and $D_{\overline{A^\prime B^\prime}}$.
This cyclide is tangent to the first cone along the circle 
$D_{\overline{A^\prime B^\prime}}$:
it is clear that the cone contains $D_{\overline{A^\prime B^\prime}}$
and, by Lemma~\ref{lemma:on-radical-axis}, 
the cyclide is tangent to the cone along 
$D_{\overline{A^\prime B^\prime}}$.

By the same argument, there is a Dupin cyclide
${\cal Z}(O_1,O_2,D_{\overline{C^\prime D^\prime}},\ell_{X^\prime})$
tangent to the second cone along the circle $D_{\overline{C^\prime D^\prime}}$ 
with diameter 
$\overline{C^\prime D^\prime}$ and perpendicular to the axial plane.
Since these two cyclides have the same principal circles and center of
similitude $X^\prime$, it can be shown that they are identical.
Therefore, there is a cyclide that blends the two cones.
\QED

In the construction of the blending cyclide in the above lemma, each point
$X^\prime$ on a diagonal determines a unique blending cyclide: thus, there
are an infinite number of blending cyclides controlled by the two diagonals.

\begin{definition}
\label{defn:family}
The cyclide 
${\cal Z}(O_1,O_2,D_{\overline{A^\prime B^\prime}},\ell_{X^\prime})$
constructed from 
$X^\prime$ on the diagonal $\stackrel{\longleftrightarrow}{RS}$, 
as defined in the proof of Lemma~\ref{lemma:int-cyclide},
will be denoted \mbox{${\cal Z}[\stackrel{\longleftrightarrow}{RS}|X^\prime]$},
and the family of cyclides on this diagonal will be denoted 
\mbox{${\cal Z}[\stackrel{\longleftrightarrow}{RS}] =$} \\
\mbox{$\{ {\cal Z}[\stackrel{\longleftrightarrow}{RS}|X^\prime]: X^\prime \in 
\stackrel{\longleftrightarrow}{RS}\}$}.
\end{definition}

\begin{remark}
In this nondegenerate conic intersection case, we have two distinct
and finite diagonals, and each provides a family of blending cyclides.
These families have a common member, the common inscribed sphere of the cones,
which is generated when $X^\prime\in\stackrel{\longleftrightarrow}{RS}\cap
\stackrel{\longleftrightarrow}{R^\prime S^\prime}$.
\end{remark}

The following lemma shows that the two families 
${\cal Z}[\stackrel{\longleftrightarrow}{RS}]$ and 
${\cal Z}[\stackrel{\longleftrightarrow}{R^\prime S^\prime}]$
account for all of the blending cyclides.

\begin{lemma}
\label{lemma:two-families}
If $Z$ is a cyclide that blends two cones with intersecting
axes and planar intersection consisting of two conics, 
there exists a point $X$ on a diagonal $\stackrel{\longleftrightarrow}{RS}$
such that $Z = {\cal Z}[\stackrel{\longleftrightarrow}{RS}|X]$.
\end{lemma}
{\bf Proof:} 
Let $C_1$ and $C_2$ be the two cones, with vertices $V_1$ and $V_2$.
Since they can be blended by a cyclide, their axes are coplanar;
and the axial plane is a symmetric plane of the cyclide.
We wish to choose the proper diagonal 
$\stackrel{\longleftrightarrow}{RS}$.
Let $R$ (resp., $S$) be the intersection of the skeletal lines that are
tangent to the first (resp., second) principal circle 
(Figure~\ref{fig:concurrent}).
We only consider the case where $R$ and $S$ are finite: the other cases
are similar (see \cite[Lemma~5.14]{shenethesis}).

Let $\stackrel{\longleftrightarrow}{V_1R}$ and 
$\stackrel{\longleftrightarrow}{V_2R}$ (resp., 
$\stackrel{\longleftrightarrow}{V_1S}$ and
$\stackrel{\longleftrightarrow}{V_2S}$) be tangent to one (resp., the other)
principal circle at $A$ and $C$ (resp., $B$ and $D$).
See Figure~\ref{fig:concurrent}.
It suffices to show that $\stackrel{\longleftrightarrow}{AB}$,
$\stackrel{\longleftrightarrow}{CD}$ and $\stackrel{\longleftrightarrow}{RS}$
meet at the same point, say $X^\prime$.  
From Lemma~\ref{cor:tangent-center}, $X^\prime$ is a center of similitude
of the two principal circles.   Therefore, applying the construction 
algorithm to $X^\prime$ shows that the given blending cyclide 
is a member of the family generated by the diagonal 
$\stackrel{\longleftrightarrow}{RS}$, 
${\cal Z}[\stackrel{\longleftrightarrow}{RS}]$.
($\stackrel{\longleftrightarrow}{AB}$ is perpendicular to the axis of $C_1$
since $|\overline{V_1A}|=|\overline{V_1B}|$ and the axis is the angle bisector
of the cone angle.   Similarly, 
$\stackrel{\longleftrightarrow}{CD}$ is perpendicular to the axis of $C_2$.)

We will first show that 
$\stackrel{\longleftrightarrow}{V_1V_2}$,
$\stackrel{\longleftrightarrow}{AC}$ and
$\stackrel{\longleftrightarrow}{BD}$ meet at the same point
(Figure~\ref{fig:concurrent}).  
Let $X=\stackrel{\longleftrightarrow}{AC}\cap
\stackrel{\longleftrightarrow}{V_1V_2}$.  If $X$ is a point at infinity, then
$\stackrel{\longleftrightarrow}{AC}$ is parallel to
$\stackrel{\longleftrightarrow}{V_1V_2}$, implying that both
$|\overline{V_1R}|=|\overline{V_2R}|$ and $V_1$ and $V_2$ are symmetric about 
the line joining the centers of the principal circles on the axial plane.  
This implies that $|\overline{V_1S}|=|\overline{V_2S}|$.  Therefore, 
$\stackrel{\longleftrightarrow}{V_1V_2}$,
$\stackrel{\longleftrightarrow}{AC}$ and
$\stackrel{\longleftrightarrow}{BD}$ are parallel to each other, and
they meet at the same point at infinity.
Suppose $X$ is finite.  Let $\bar{X}=\stackrel{\longleftrightarrow}{BD}\cap
\stackrel{\longleftrightarrow}{V_1V_2}$.  $\bar{X}$ must be finite, otherwise
$X$ will be at infinity.  In $\bigtriangleup V_1RV_2$,
$\stackrel{\longleftrightarrow}{AC}$ is a transversal and by Menelaus' 
Theorem\footnote{Menelaus' theorem \cite{johnson:1929} states that 
	if a line intersects the sides of a triangle 
	$\bigtriangleup ABC$ at $X\in\stackrel{\longleftrightarrow}{AB}$,
	$Y\in\stackrel{\longleftrightarrow}{BC}$ and
	$Z\in\stackrel{\longleftrightarrow}{CA}$,
	then $\frac{|\overline{AX}|}{|\overline{XB}|}\cdot
	     \frac{|\overline{BY}|}{|\overline{YC}|}\cdot
	     \frac{|\overline{CZ}|}{|\overline{ZA}|} = 1$.}
we have 
$\frac{|\overline{V_1A}|}{|\overline{AR}|}\cdot
   \frac{|\overline{RC}|}{|\overline{CV_2}|}\cdot
   \frac{|\overline{V_2X}|}{|\overline{XV_1}|} = 1$.
Since $|\overline{RA}|=|\overline{RC}|$,
$\frac{|\overline{XV_1}|}{|\overline{XV_2}|} =
   \frac{|\overline{V_1A}|}{|\overline{V_2C}|}$.
Similarly, in $\bigtriangleup V_1SV_2$, using 
$\stackrel{\longleftrightarrow}{BD}$ as a transversal, we have
$\frac{|\overline{\bar{X}V_1}|}{|\overline{\bar{X}V_2}|} =
   \frac{|\overline{V_1A}|}{|\overline{V_2C}|}$.
With these two identities, we finally have $X=\bar{X}$ and hence,
$\stackrel{\longleftrightarrow}{V_1V_2}$,
$\stackrel{\longleftrightarrow}{AC}$ and
$\stackrel{\longleftrightarrow}{BD}$ are concurrent.

% only use (a) and shrink (very tall)
\begin{figure}
\vspace{11cm}
\caption{Any Blending Cyclide Belongs to One of the Two Families}
\label{fig:concurrent}
\end{figure}

     In $\bigtriangleup RAC$ and $\bigtriangleup SBD$, their corresponding 
sides meet at three collinear points $V_1, V_2$ and $X=\bar{X}$.  By
Desargues' theorem,\footnote{Desargues' theorem \cite{johnson:1929} states 
	that, given two triangles $\bigtriangleup A_1A_2A_3$ and 
	$\bigtriangleup B_1B_2B_3$ with no two of the six sides identical,
	$\stackrel{\longleftrightarrow}{A_1B_1}$,
	$\stackrel{\longleftrightarrow}{A_2B_2}$ and
	$\stackrel{\longleftrightarrow}{A_3B_3}$ are concurrent if and only if
	$\stackrel{\longleftrightarrow}{A_1A_2}\cap\stackrel{\longleftrightarrow}{B_1B_2}$,
	$\stackrel{\longleftrightarrow}{A_2A_3}\cap\stackrel{\longleftrightarrow}{B_2B_3}$ and
	$\stackrel{\longleftrightarrow}{A_3A_1}\cap\stackrel{\longleftrightarrow}{B_3B_1}$
	are collinear.} 
the lines joining corresponding vertices are concurrent.
That is, $\stackrel{\longleftrightarrow}{AB}$,
$\stackrel{\longleftrightarrow}{CD}$ and $\stackrel{\longleftrightarrow}{RS}$
meet at the same point.
\QED

\subsubsection{Parallel Axes}
\label{section:cyc:parallel-axes}

     In this section, we shall examine the parallel axes case.  
In this case, in order to have a conic intersection, the 
corresponding skeletal lines must be parallel to each other. 
Thus, we have only one finite diagonal with two finite diagonal points $R$ 
and $S$.  
The result is the same as Section~\ref{section:intersecting-axes},
but with only one diagonal.
(Since there is no finite common inscribed sphere, a slightly different proof
technique is needed.)

\begin{lemma}
\label{lemma:para-cyclide}\ 
\begin{enumerate}
\item
     If two cones, with parallel and distinct axes,
have planar intersection consisting of two conics, 
they can be blended by a cyclide.
\item
     There is an entire family of blending cyclides.
If $\stackrel{\longleftrightarrow}{RS}$ is the only diagonal with two finite
diagonal points $R$ and $S$, the family of cyclides is 
${\cal Z}[\stackrel{\longleftrightarrow}{RS}]$ (exactly as defined in 
Definition~\ref{defn:family}), {\it i.e.}, the typical cyclide is
${\cal Z}(O_1,O_2,D_{\overline{A^\prime B^\prime}},\ell_{X^\prime})$
as defined in Lemma~\ref{lemma:int-cyclide}, where 
$X^\prime \in \stackrel{\longleftrightarrow}{RS}$.
\item
	Any cyclide that blends these cones is a member of this family.
\end{enumerate}
\end{lemma}
{\bf Proof:}  
See \cite[Lemmas~5.15 and 5.16]{shenethesis}.
\QED

\subsection{Degenerate Conic Intersection}
\label{section:line-conic}

     In this section, we shall consider the degenerate conic intersection
case in which the 
complete planar intersection consists of a double line and a conic. 
If two cones have a double line and a conic in their intersection, then
in the axial plane two skeletal lines, one from each skeletal pair, coincide 
and the common line is a double line.  
One of the two diagonals disappears, 
becoming the common line and leaving a triangle rather than the typical
quadrilateral in the axial plane (Figure~\ref{fig:three-lines}).
Therefore, we only expect one family of blending cyclides.

% use just the 1st col (which is a repeat of next figure, but less cluttered)
% and (f) (for the parallel axes case)
\begin{figure}
\vspace{7.5cm}
\caption{There is Only One Diagonal in the Line-Conic Case}
\label{fig:three-lines}
\end{figure}

We first show that any blending cyclide must be parabolic.

\begin{lemma}
\label{lemma:only-one-family}
     A cyclide that blends two cones with degenerate conic
intersection (i.e., a double line and a conic in their intersection) must be
parabolic.
\end{lemma}
{\bf Proof (sketch):} 
If the blending cyclide has two principal circles in the axial plane,
it can be shown that they must be tangent to the double line at the same point.
Both of the resulting cases (one circle inside the other, or both circles
outside the other) lead to cyclides that 
cannot be tangent to the cones along circles.
See \cite[Lemma~5.17]{shenethesis}.
\QED

An immediate result of this lemma is that there exists a special
class of cones with planar intersection that have no blending cyclide.  This
is the only exception to the necessity and sufficiency of planar intersection
for a blending cyclide.

\begin{cor}
\label{cor:no-cyclide-at-all}
Two cones with parallel axes and degenerate conic intersection 
have no blending cyclide.
\end{cor}
{\bf Proof:}  Suppose these two cones have a blending parabolic 
cyclide.  Let it be tangent to the two cones along the circles $C_1$ and $C_2$.
Since the planes containing $C_1$ and $C_2$ are perpendicular
to the parallel axes, they are parallel to each other.
However, $C_1$ and $C_2$ are circles from the same family of a parabolic
cyclide, and all planes containing circles from the same family pass through
a finite common line, one of the two principal lines of the parabolic cyclide.
This contradiction shows that the cones do not have any blending cyclide. \QED

Note that if the axes are parallel (Figure~\ref{fig:three-lines}c),
there is no diagonal.  
We now establish the converse of Lemma~\ref{cor:para->common-line},
and the analogue of Lemma~\ref{lemma:int-cyclide}.

\begin{lemma}
\label{lemma:three-line-cyclide}
If two cones with intersecting axes have degenerate conic
intersection, they can be blended by a parabolic cyclide.
\end{lemma}
{\bf Proof (sketch):} 
We begin with some notation.
Let ${\cal C}_1$ and ${\cal C}_2$ be the two cones 
(with vertices $V_1$ and $V_2$ and axes $\ell_1$ and $\ell_2$) 
and $\stackrel{\longleftrightarrow}{V_1V_2}$ be the only double line in the
intersection curve.  
Let $\stackrel{\longleftrightarrow}{RS}$ be the only diagonal,
which is the 
line joining the tangent point $R$ of the common inscribed sphere on 
$\stackrel{\longleftrightarrow}{V_1V_2}$
and the intersection point $S$ of the two non-coincident skeletal lines
\cite{shene-johnstone:1991b} (Figure~\ref{fig:three-lines}).
Note that $S$ may be at infinity.  
Let $X\in\stackrel{\longleftrightarrow}{RS}$ and 
$X\not\in\stackrel{\longleftrightarrow}{V_1V_2}$.\footnote{If $X$ lies on
$\stackrel{\longleftrightarrow}{V_1V_2}$, the blending surface is the common
inscribed sphere.}

We construct the cyclide in a similar way to Lemma~\ref{lemma:int-cyclide},
but we now want to define a principal circle and a principal line rather
than two principal circles.
Through $X$, construct 
a line perpendicular to $\ell_1$ (resp., $\ell_2$) meeting
$\stackrel{\longleftrightarrow}{V_1R}$ and
$\stackrel{\longleftrightarrow}{V_1S}$ (resp.,
$\stackrel{\longleftrightarrow}{V_2R}$ and
$\stackrel{\longleftrightarrow}{V_2S}$) at $A$ and $B$ (resp., $C$ and $D$)
respectively (Figure~\ref{fig:parabolic-construct}).  
Let $O$ be the circumscribed circle of $\bigtriangleup BDX$.  
We claim that $O$ (as principal circle), 
$\stackrel{\longleftrightarrow}{V_1V_2}$ (as principal line), 
and $X$ define a blending
parabolic cyclide ${\cal Z}_P(\stackrel{\longleftrightarrow}{V_1V_2},O,X)$.

% use both figures (to show S at infinity case too)
\begin{figure}
\vspace{5.5cm}
\caption{The Construction of a Blending Parabolic Cyclide}
\label{fig:parabolic-construct}
\end{figure}

It can be shown that this cyclide is well defined ({\it i.e.}, the line 
joining $X$ and the center of $O$ is perpendicular to
$\stackrel{\longleftrightarrow}{V_1V_2}$).
We need to show that \mbox{${\cal Z}_P(\stackrel{\longleftrightarrow}{V_1V_2},O,X)$}
blends the cones.
Let $M$ and $N$ be the tangent points of the common inscribed 
sphere on $\stackrel{\longleftrightarrow}{V_1S}$ and
$\stackrel{\longleftrightarrow}{V_2S}$, respectively.
Consider the circumscribed circle of $\bigtriangleup RMN$, which is the
intersection circle of the common inscribed sphere and the axial plane.
Since $\stackrel{\longleftrightarrow}{V_1S}$ is tangent to it at $M$, a
geometry theorem gives $\angle V_1MR=\angle MNR$.  Therefore, in
$\bigtriangleup XBD$, since the corresponding sides are parallel to each other,
we have $\angle V_1BX=\angle V_1MR=\angle MNR=\angle BDX$. 
Hence, $\stackrel{\longleftrightarrow}{V_1S}$ is tangent to circle $O$ at $B$.

Since $|\overline{V_1A}|=|\overline{V_1B}|$ (by construction),
there is a circle tangent to 
$\stackrel{\longleftrightarrow}{V_1V_2}$ and 
$\stackrel{\longleftrightarrow}{V_1S}$ at $A$ and $B$,
which is thus tangent to the circle $O$.
Therefore, the cone ${\cal C}_1$ is tangent
to the parabolic cyclide along the circle with diameter $\overline{AB}$ and
perpendicular to the axial plane: the cyclide and cone both contain this circle
and, by Lemma~\ref{lemma:on-principal-line}, 
the cyclide is tangent to the cone along the circle.
Similarly, the cone ${\cal C}_2$ is
also tangent to the cyclide along the circle with diameter $\overline{CD}$
and perpendicular to the axial plane.  Hence, the parabolic cyclide
${\cal Z}_P(\stackrel{\longleftrightarrow}{V_1V_2},O,X)$
is a blending surface of ${\cal C}_1$ and ${\cal C}_2$.
\QED

\begin{definition}
The parabolic cyclide ${\cal Z}_P(\stackrel{\longleftrightarrow}{V_1V_2},O,X)$
constructed from $X$ on the diagonal $\stackrel{\longleftrightarrow}{RS}$, 
as defined in the proof of Lemma~\ref{lemma:three-line-cyclide},
will be denoted 
\mbox{${\cal Z}_P[\stackrel{\longleftrightarrow}{RS}|X]$},
and the family of cyclides on this diagonal will be denoted\\
${\cal Z}_P[\stackrel{\longleftrightarrow}{RS}] =$
\mbox{$\{ {\cal Z}_P[\stackrel{\longleftrightarrow}{RS}|X]: X \in 
\stackrel{\longleftrightarrow}{RS}\}$}.
\end{definition}

\begin{lemma}
\label{lemma:1-parabolic}
     Each blending cyclide of two cones with intersecting
axes and degenerate conic intersection is a member of the family
${\cal Z}_P[\stackrel{\longleftrightarrow}{RS}]$,
where $\stackrel{\longleftrightarrow}{RS}$ is the unique diagonal.
\end{lemma}
{\bf Proof:} 
See \cite[Lemma~5.19]{shenethesis}.
\QED

\subsection{Linear Intersection}
\label{section:cyc-linear}

For the linear intersection case, we simply state the result.
A technique similar to Boehm's \cite{boehm:1990} is used.

\begin{lemma}
\label{thm:linear-inter-blending}
     Two cones with linear intersection always have a blending cyclide.
Any blending cyclide is a member of one of four (resp., three or two) families
of cyclides if the intersection contains zero (resp., one or two) double
lines.  These cyclides may be nondegenerate or parabolic.
\end{lemma}
{\bf Proof:} See \cite[Section~5.9]{shenethesis}.
\QED

We have now fully established our necessary and sufficient condition.

\begin{theorem}\ 
\begin{enumerate}
\item
Two cones with planar intersection
that consists of a double line and a conic, and parallel axes, 
cannot be blended by any Dupin cyclide.
\item
Otherwise, two cones can be blended by a Dupin cyclide if and only if 
they have planar intersection.
\item
The blending Dupin cyclide is nondegenerate if the planar
intersection is nondegenerate (two conics).
It is parabolic if the planar 
intersection contains a double line and a conic.
\item
With the exception of the linear intersection case,
there are one or two families of blending cyclides, each family 
controlled by one of the diagonals in the axial plane. 
\end{enumerate}
\end{theorem}

\section{Examples}
\label{section:eg}

We now give some examples of cones blended by cyclides.
Figure~\ref{fig:1} shows two typical cone-cone blends.
Figure~\ref{fig:2} is important, since it illustrates
a configuration of the blend that cannot be achieved by a blending quadric.
The blending Dupin cyclide can reside in the interior of one cone and
in the exterior of the other.
This configuration is impossible for quadric blending surfaces, since
the blending surface must have a hole.
Figure~\ref{fig:3} shows two parabolic cyclide blends, and Figure~\ref{fig:4}
shows a typical pipe blend.

\section{Conclusions}
\label{section:cyc-concl}

In this paper, we have successfully presented a complete investigation
of the existence, use and organization of blending cyclides of two cones.  
The main contributions are a new 
construction algorithm for blending cyclides, a complete and new proof that 
for almost all cases, two cones have planar intersection 
if and only if they have a blending cyclide, and a thorough study of how all 
of these blending cyclides are organized into families.
The necessary and sufficient conditions show
when and how we can use Dupin cyclides as blending surfaces.

\begin{figure}
\vspace{3.5in}
\caption{Two typical cone-cone blends}
\label{fig:1}
\end{figure}

\begin{figure}
\vspace{3.5in}
\caption{A blend that cannot be achieved by a quadric}
\label{fig:2}
\end{figure}

\begin{figure}
\vspace{3.5in}
\caption{Two parabolic cyclide blends}
\label{fig:3}
\end{figure}

\begin{figure}
\vspace{3.5in}
\caption{A pipe blend}
\label{fig:4}
\end{figure}

\clearpage

\section{Acknowledgements}

This work was supported in part by NSF Grant \#IRI-8910366.

\begin{thebibliography}{999}

\bibitem{boehm:1990}
     Wolfgang Boehm,
     On Cyclides in Geometric Modeling,
     {\em Computer Aided Geometric Design},
     Vol. 7 (1990), pp. 243--255.

\bibitem{cayley:1873}
     Arthur Cayley,
     On the Cyclide,
     {\em Quarterly Journal of Pure and Applied Mathematics},
     Vol. 12 (1873), pp. 148--165.

\bibitem{chandru-dutta-hoffmann:1989}
     Vijaya Chandru, Debasish Dutta and Christoph M. Hoffmann,
     On the Geometry of Dupin Cyclides,
     {\em The Visual Computer},
     Vol. 5 (1989), pp. 277--290.

\bibitem{chandru-dutta-hoffmann:1990}
     Vijaya Chandru, Debasish Dutta and Christoph M. Hoffmann,
     Variable Radius Blending Using Dupin Cyclides,
     in {\em Geometric Modeling for Product Engineering},
     edited by M. J. Wozny, J. U. Turner and K. Preiss,
     Elsevier Science Publishers B. V., 1990, pp. 39--57.

\bibitem{depont:1984}
     John James de Pont,
     Essays on the Cyclide Patch,
     Ph.D. Thesis,
     Engineering Department, Cambridge University, 1984.

\bibitem{degen:1990}
     W. Degen,
     Generalized Cyclides for Use in {\em CAGD},
     Mathematisches Institut B, Universit\"{a}t Stuttgart,
     preprint, 1990.

\bibitem{dupin:1822}
     Charles Dupin,
     {\em Applications de G\'{e}om\'{e}trie et de M\'{e}chanique},
     Bachelier, Paris, 1822.
     
\bibitem{dutta:1989}
     Debasish Dutta,
     {\em Variable Radius Blends and Dupin Cyclides},
     Ph. D. Dissertation,
     School of Industrial Engineering, Purdue University, 1989.

\bibitem{fischer:1986}
     Gerd Fischer, editor,
     {\em Mathematical Models}, two volumes,
     Friedr. Vieweg \& Sohn,
     Braunschweig/Wiesbaden, Germany, 1986.

\bibitem{hilbert-vossen:1952}
     David Hilbert and S. Cohn-Vossen,
     {\em Geometry and the Imagination},
     translated from the German edition {\em Anschauliche Geometrie} by
          P. Nemenyi,
     Chelsea, New York, 1952.

\bibitem{hoffmann-hopcroft:1986}
     Christoph M. Hoffmann and John Hopcroft,
     Quadratic Blending Surfaces,
     {\em Computer Aided Design},
     Vol. 18 (1986), pp. 301--307.

\bibitem{hoffmann-hopcroft:1987}
     Christoph M. Hoffmann and John Hopcroft,
     The Potential Method for Blending Surfaces and Cornors,
     in {\em Geometric Modeling}, edited by G. Farin,
     SIAM, Philadelphia, 1987.

\bibitem{hoffmann-hopcroft:1988}
     Christoph M. Hoffmann and John Hopcroft,
     The Geometry of Projective Blending Surfaces,
     {\em Artificial Intelligence},
     Vol. 37 (1988), pp. 357--376.

\bibitem{humbert:1885}
     G. Humbert,
     Sur Les Surfaces Cyclides,
     {\em Journal de L'\'{e}cole Polytechnique},
     Vol. 55 (1885), pp. 127--252.

\bibitem{jessop:1916}
     Charles Jessop,
     {\em Quartic Surfaces with Singular Points},
     Cambridge University Press, Cambridge, England, 1916.

\bibitem{johnson:1929}
     Roger A. Johnson,
     {\em Modern Geometry},
     Houghton Mifflin, Boston, 1929.

\bibitem{jj92}
     John K. Johnstone,
     {\em A New Intersection Algorithm for Cyclides and
	Swept Surfaces using Circle Decomposition},
     Computer Aided Geometric Design, to appear.

\bibitem{martin:1982}
     R. R. Martin,
     Principal Patches for Computational Geometry,
     Ph.D. Thesis,
     Cambridge University, Engineering Department, 1982.

\bibitem{maxwell:1868}
     J. Clerk Maxwell,
     On the Cyclide,
     {\em Quarterly Journal of Pure and Applied Mathematics},
     Vol. 8 (1868), pp. 111-126.

\bibitem{nutbourne-martin:1988}
     Anthony W. Nutbourne and Ralph R. Martin,
     {\em Differential Geometry Applied to Curve and Surface Design, 
          Volume 1: Foundations},
     Ellis Horwood Limited, Chichester, West Sussex, England, 1988.

\bibitem{pratt:1989}
     M. J. Pratt,
     Application of Cyclide Surfaces in Geometric Modeling,
     in {\em The Mathematics of Surfaces III},
     edited by D. C. Handscomb,
     Clarendon Press, Oxford, England, 1989, pp. 405--428.

\bibitem{pratt:1990}
     M. J. Pratt,
     Cyclides in Computer Aided Geometric Design,
     {\em Computer Aided Geometric Design},
     Vol. 7 (1990), pp. 221-242.

\bibitem{sharrock:1985}
     Timothy John Sharrock,
     Surface Design with Cyclide Patches,
     Ph.D. Thesis,
     Engineering Department, Cambridge University, 1985.

\bibitem{shenethesis}
     Ching-Kuang Shene,
     Planar Intersection and Blending of Natural Quadrics,
     Ph.D. thesis, Department of Computer Science, The Johns Hopkins
     University, 1992.

\bibitem{shene-johnstone:1991a}
     Ching-Kuang Shene and John K. Johnstone,
     On the Planar Intersection of Natural Quadrics,
     {\em ACM Symposium on Solid Modeling Foundations and
     CAD/CAM Applications}, 
     June 1991, pp. 233--242.

\bibitem{shene-johnstone:1991b}
     Ching-Kuang Shene and John K. Johnstone,
     On the Lower Degree Intersections of Two Natural Quadrics I: Algorithms,
     Technical Report JHU-91/15, Department of Computer Science,
     Johns Hopkins University, September 1991.

\bibitem{warren:1986}
     Joe Warren,
     On Algebraic Surfaces Meeting with Geometric Continuity,
     Ph.D. thesis, 
     Department of Computer Science, Cornell University, 1986.

\bibitem{warren:1987}
     Joe Warren,
     Blending Quadric Surfaces with Quadric and Cubic Surfaces,
     {\em Proceedings of the Third Annual Symposium on Computational Geometry},
     Waterloo, Ontario, Canada 1987, pp. 341--347.

\bibitem{warren:1989}
     Joe Warren,
     Blending Algebraic Surfaces,
     {\em ACM Transactions on Graphics},
     Vol. 8 (1989), No. 4 (October), pp. 263--278.

\end{thebibliography}

\end{document}

\begin{figure}
\vspace{18.5cm}
\caption{Dupin Cyclides (taken from [39] with permission)}
\label{fig:cyclides}
\end{figure}

\begin{figure}
\vspace{12cm}
\caption{Parabolic Dupin Cyclides (taken from [39] with permission)}
\label{fig:den-cyclides}
\end{figure}

\begin{figure}
\vspace{2in}
\caption{Skeletal lines and diagonals}
\label{fig3.1}
\end{figure}

\begin{figure}
\vspace{2in}
\caption{The Radical Axis Lies in Both Interiors or Both Exteriors}
\label{fig:int-ext}
\end{figure}

\begin{figure}
\vspace{3cm}
\caption{The six possible placements of $\stackrel{\longleftrightarrow}{V_1V_2}$}
\label{fig:configs}
\end{figure}

\begin{figure}
\vspace{6.5cm}
\caption{The Axial Plane, with Principal Circles}
\label{fig:Q-RS}
\end{figure}

\begin{figure}
\vspace{4.5cm}
\caption{Specifying a Dupin Cyclide}
\label{fig:specification-nondegen}
\end{figure}

\begin{figure}
\vspace{5.5cm}
\caption{No Cyclide Contains These Three Circles}
\label{fig:wrong-example}
\end{figure}

\begin{figure}
\vspace{6.5cm}
\caption{Basic Configuration in the Axial Plane}
\label{fig:int-config}
\end{figure}

\begin{figure}
\vspace{14cm}
\caption{Any Blending Cyclide Belongs to One of the Two Families}
\label{fig:concurrent}
\end{figure}

\begin{figure}
\vspace{7.5cm}
\caption{There is Only One Diagonal in the Line-Conic Case}
\label{fig:three-lines}
\end{figure}

\begin{figure}
\vspace{5.5cm}
\caption{The Construction of a Blending Parabolic Cyclide}
\label{fig:parabolic-construct}
\end{figure}

% ***************

\begin{figure}
\vspace{3.5in}
\special{psfile=/users/cogito/jj/Research/blend/figs/---
	hscale= vscale= hoffset= voffset=}
\caption{Two typical cone-cone blends}
\label{fig:1}
\end{figure}

\begin{figure}
\vspace{3.5in}
\special{psfile=/users/cogito/jj/Research/blend/figs/---
	hscale= vscale= hoffset= voffset=}
\caption{A blend that cannot be achieved by a quadric}
\label{fig:2}
\end{figure}

\begin{figure}
\vspace{3.5in}
\special{psfile=/users/cogito/jj/Research/blend/figs/---
	hscale= vscale= hoffset= voffset=}
\caption{Two parabolic cyclide blends}
\label{fig:3}
\end{figure}

\begin{figure}
\vspace{3.5in}
\special{psfile=/users/cogito/jj/Research/blend/figs/---
	hscale= vscale= hoffset= voffset=}
\caption{A pipe blend}
\label{fig:4}
\end{figure}


\end{document}

