\documentstyle [12pt]{article}

\title{Dupin cyclides as blending surfaces for cones}

\author{Ching-Kuang Shene and John K. Johnstone \\
              Department of Computer Science \\
              The Johns Hopkins University, Baltimore, MD 21218--2691}

\date{}

\setlength{\oddsidemargin}{0pt}
\setlength{\evensidemargin}{0pt}
\setlength{\headsep}{0pt}
\setlength{\topmargin}{0pt}
\setlength{\textheight}{8.75in}
\setlength{\textwidth}{6.5in}

\begin{document}
\maketitle

We prove that two cones can be blended by a Dupin cyclide
(either cubic or quartic) if and only if they have planar intersections.
The problem of defining a necessary and sufficient condition for
blending two cones by a cyclide was first studied by Sabin and Pratt.
% statement without proof by Sabin, sufficiency by Pratt
% iff they contain a common inscribed sphere
% incorrect if cones have parallel axes, unless common inscribed sphere
% at infinity is included
We show that, given two cones satisfying the above condition,
there are exactly two families of cyclides that blend them.
The proof is constructive and leads directly to a method of generating
all of the cyclides.

Consider the necessity of the planar intersection condition.
If a cyclide blends two cones, the cones must meet the cyclide at two circles
from the same family of lines of curvature.
This forces the cone axes to be coplanar,
and allows us to study the problem in the plane defined by the cone axes
(the axial plane).
We show that the distances of the cone vertices to two fixed points in this
plane are in a fixed relationship (their sum or difference is constant).
By applying recent results on necessary and sufficient conditions for planar
intersection of two cones, this implies that the intersection of the
cones must be planar.

Our proof of the sufficiency of the condition is constructive, and
leads to an elegant method of generating the two families of blending cyclides.
This is done by establishing a one-to-one correspondence between the points
on the (two) diagonals of the axial plane and (the two families of)
blending cyclides.
The diagonals are the intersection lines of the axial plane and the planes
containing the planar intersection.
From this construction, we can classify/enumerate all possible blending
cyclides, and control the type and position of the blend.
% can find common inscribed sphere, spindle, ...

We also clarify that the curve of tangency of the blending cyclide
with the cones must be circular, using a result of Humbert.
This not only simplifies the proof, but allows us to show
that the Dupin cyclide is inappropriate for blending the famous
Cranfield object (a cone/torus blend).
% specific example of how cyclide type changes as one moves along diagonal



\end{document}
