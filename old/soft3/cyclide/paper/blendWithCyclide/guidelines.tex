%From Ralph.Martin@cf.cm Thu May 14 10:11:58 1992
%Message-Id: <2932.9205140911@aifh.ed.ac.uk>
%Received: from cm.cf.ac.uk by aifh.ed.ac.uk; Thu, 14 May 92 10:11:57 BST
%Received: from ralph.cm.cf.ac.uk by sentinel.cm.cf.ac.uk with SMTP (PP) 
%          id <12897-0@sentinel.cm.cf.ac.uk>; Thu, 14 May 1992 10:10:59 +0100
%Date: Thu, 14 May 1992 10:11:00 +0000
%To: Bob Fisher <rbf@ed.aifh>
%From: Ralph Martin <Ralph.Martin@cf.cm>
%Subject: Proceedings Instructions (LaTeX)
%Status: RO
%
\documentstyle[conf]{article}
 
\title{Guidelines for Typing Contributions to\\
``Mathematics of Surfaces III''}
 
\author{D.~C.~Handscomb \affil University of Oxford}
 
\shortauthor{Handscomb}
 
\numberedbib
 
\begin{document}
 
\maketitle
 
\sloppy
 
\section{General remarks}\label{s1}
 
These guidelines are designed to ensure a degree of uniformity between the
way contributions appear in print.  I shall go through every \LaTeX\
sourcefile once it has been typed, and possibly make some amendments, but
these will be fewer and easier if the guidelines have been adhered to.
 
Follow the \LaTeX\ handbook\footnote{Leslie Lamport, {\em \LaTeX---A
Document Preparation System: User's Guide \& Reference Manual},
Addison--Wesley, 1985} closely in all matters not discussed in these notes,
but do not try anything too clever!  You may, if you like, define a command
or `box' to represent something that recurs many times during the paper, to
save typing, as long as you keep such definitions to the beginning of the
file.
 
Do not try to set pictures.
 
Correct any {\em obvious} misprints in the text ({\em e.g.}
mis-spell\-ings---but do not necessarily replace U.S. by U.K. spellings);
draw my attention to anything more dubious.
 
\section{Working environment}\label{s2}
 
I use the standard documentstyle `article' (version dated 23 Sep 1985),
with modifications which are effected by calling my own documentstyle
option `conf.sty'.  This will take care of all detailed requirements for
page layout, and includes some extra facilities described below.  [It can
be used for other conference proceedings, if modified by changing the
`booktitle' running heading.]
 
\section{Preliminaries}\label{s3}
 
Each contribution should be set up as a separate sourcefile in ASCII
format, whose name should be `\verb!author.TEX!', where `\verb!author!'
will generally be the surname of the first or only author (but may be
abbreviated if too long) without spaces of hyphens.  Begin the sourcefile
as follows:
 
\begin{verbatim}
        \documentstyle[conf]{article}
\end{verbatim}
 
\begin{verbatim}
        \title{Title with Important Words Capitalized}
\end{verbatim}
 
\begin{verbatim}
        \author{Author \affil Affiliation}
\end{verbatim}
or
\begin{verbatim}
        \author{Author \and Author \affil Affiliation}
\end{verbatim}
or
\begin{verbatim}
        \author{Author \affil Affiliation
          \and Author \and Author \affil Affiliation}
\end{verbatim}
\&c.
 
\begin{itemize}
 
\item The name(s) of the author(s) should appear as on the manuscript,
{\em i.e.} either in full or as initials and surname.  Put a space after
the `.' after each initial.
 
\item The `affiliation' given for each author or group of authors should
be very brief, {\em e.g.} the name of a University or a firm.  Do not
include full addresses, which will appear elsewhere in the book.
 
\item Any other footnotes to the title or author(s) ({\em e.g.}
acknowledgments) should be moved to the end of the paper, before the
bibliography, with suitable wording (which I will supply if necessary) such
as `This work was supported by \ldots' or `One of us (A.~B.~C.) was
assisted by a grant from \ldots', and enclosed between `\verb!\begin{ackn}!
\verb!\end{ackn}!' (see page~\pageref{ack} below).
 
\end{itemize}
 
\begin{verbatim}
        \shortauthor{Surname(s)}
\end{verbatim}
\begin{itemize}
\item The argument to `\verb!\shortauthor!', used to supply a running head,
should be
\[\left\{\begin{array}{l}
\mbox{The author's Surname for a sole author}\\
\mbox{Surname 1 `\verb! \& !' Surname 2 for two joint authors}\\
\mbox{Surname 1 ` et al.' for more than two}
\end{array}\right.\]
\end{itemize}
 
\noindent\verb!        [\unnumberedbib]!\footnote{include this command only
if references are all cited by name and date, not by cross-reference; see
page~\pageref{ss52} below.}
 
\begin{verbatim}
        \begin{document}
\end{verbatim}
 
\begin{verbatim}
        \maketitle
\end{verbatim}
 
 
\section{Body of work}\label{s4}
 
If an abstract, summary or list of keywords has been supplied, ignore it.
 
Follow the author's divisions into sections, sub\-sec\-tions,
sub\-sub\-sec\-tions---not parts, chapters, paragraphs,
sub\-para\-graphs---but use \LaTeX's cross-ref\-er\-ence system rather than
copying the author's numbering; see Section~\ref{s5} below.
 
\subsection{Figures and tables}
 
Where there is a figure to be inserted, include the commands
\begin{verbatim}
        \begin{figure}[htbp]
        \vspace{2cm}
        \caption[]{...} \label{f...} \end{figure}
\end{verbatim}
to leave a token amount (2cm) of vertical space.  This I shall adjust by
editing the sourcefile later, when the printers have worked out the actual
sizing of each figure.  Type out the caption if one is supplied; otherwise
just say
\begin{verbatim}
        \caption[]{}
\end{verbatim}
 
To allow two figures to be inserted side by side, use the commands
\begin{verbatim}
        \begin{figure}[htbp]
        \vspace{2cm}
        \halfbox{\caption[]{...}\label{f...}}
        \halfbox{\caption[]{...}\label{f...}}
        \end{figure}
\end{verbatim}
 
        For example, Fig.~\ref{f1} and Figs~\ref{f2}
        and~\ref{f3}, on page~\pageref{f1}, together
        with this comment, were generated by the
        instructions:
 
        \begin{figure}[htbp] \vspace{2cm}
        \caption[]{First figure looks like this}
        \label{f1} \end{figure}
        \begin{figure}[htbp] \vspace{2cm}
        \halfbox{\caption[]{Second figure looks like
          this} \label{f2}}
        \halfbox{\caption[]{Third figure looks like
          this}\label{f3}}
        \end{figure}
 
\begin{verbatim}
        For example, Fig.~\ref{f1} and Figs~\ref{f2}
        and~\ref{f3}, on page~\pageref{f1}, together
        with this comment, were generated by the
        instructions:
\end{verbatim}
 
\begin{verbatim}
        \begin{figure}[htbp] \vspace{2cm}
        \caption[]{First figure looks like this}
        \label{f1} \end{figure}
        \begin{figure}[htbp] \vspace{2cm}
        \halfbox{\caption[]{Second figure looks like
          this} \label{f2}}
        \halfbox{\caption[]{Third figure looks like
          this}\label{f3}}
        \end{figure}
\end{verbatim}
 
Tables are dealt with similarly, except that you can type them out
explicitly instead of leaving a space; use `\verb!tabular!'. They should be
centred on the page. For example, here is Table~\ref{t1}:
        \begin{table}[htbp]\centering
        \begin{tabular}{|c|c|}
          \hline function & derivative\\ \hline
          $\ee^x$ & $\ee^x$\\ $\log x$ & $1/x$\\
          $\sin x$ & $\cos x$\\ $\cos x$ & $-\sin x$\\
          $x^n$ & $nx^{n-1}$\\
          \hline
        \end{tabular}
        \caption[]{Derivatives of simple functions}
        \label{t1} \end{table}
It was produced by the instructions
\begin{verbatim}
        \begin{table}[htbp]\centering
        \begin{tabular}{|c|c|}
          \hline function & derivative\\ \hline
          $\ee^x$ & $\ee^x$\\ $\log x$ & $1/x$\\
          $\sin x$ & $\cos x$\\ $\cos x$ & $-\sin x$\\
          $x^n$ & $nx^{n-1}$\\
          \hline
        \end{tabular}
        \caption[]{Derivatives of simple functions}
        \label{t1} \end{table}
\end{verbatim}
 
Both figures and tables should be inserted as early as possible after the
place where they are first referred to (although you may find it easier to
type them separately and edit them into position afterwards).   Note that
you refer to `Fig.~\ref{f1}' just like that, not as `figure' for example.
 
\subsection{Mathematics}
 
Don't forget to enclose any isolated mathematical symbols in the text
between `\verb!$!~\verb!$!' signs.
 
Any punctuation that is required by the mathematics should come {\em
outside} `\verb!$!~\verb!$!' signs, but {\em inside} the
`\verb!\[!~\verb!\]!' or `\verb!\begin{equation}! \verb!\end{equation}!'
signs that cause the material to be {\em displayed}.
 
Use `\verb!eqnarray!' or `\verb!eqnarray*!' where you can, to line up
parallel formulae nicely.
 
In displayed equations, use `\verb!\mbox{...}!' round any words like `for',
`if', `otherwise', `and so on' \&c., so that they appear in roman and not
italic, {\em e.g.} `\verb!\[i=1 \mbox{ to }n.\]!' (note the spaces either
side of `to'), which gives
\[i=1 \mbox{ to }n.\]
 
\subsubsection{Equation groups}
 
A group of equations (to be numbered (1.1a), (1.1b) \&c., for example)
should be enclosed between corresponding `\verb!\begin{eqngroup}!' and
`\verb!\end{eqngroup}!' commands; the equations will still need their own
`\verb!\begin/end{equation}!' or `\verb!\begin/end{eqnarray}!' commands,
but the group may include intervening text as well.  The group may itself
be labelled, and can then be referenced as a whole as (1.1), for example,
instead of (1.1a--c).
 
\begin{itemize}
\item An `\verb!eqngroup!' must be properly nested with other environments.
 
\item An `\verb!eqngroup!' may contain, but may not be contained in an
`\verb!eqnarray!'.
 
\item An `\verb!eqngroup!' must lie wholly within one \verb!section!.
 
\end{itemize}
 
\noindent For example:
        \begin{eqngroup}\label{eq234}
        \begin{equation}
        a \frac{\partial u}{\partial t} =
        \frac{\partial}{\partial x}
          K \frac{\partial u}{\partial x},
        \label{eq2}\end{equation}
        where $u = u(x,t)$,
        \begin{eqnarray}
        u(x,0) &=& \phi(x),
          \quad 0\leq x\leq \pi,\label{eq3}\\
        u(0,t) = u(\pi,t) &=& 0,
          \quad t>0.\label{eq4}
        \end{eqnarray}
        \end{eqngroup}
        Here (\ref{eq2}) refers to the differential
        equation, (\ref{eq3}) to the initial conditions,
        (\ref{eq4}) to the boundary conditions, and
        (\ref{eq234}) to the whole problem.
 
This example was produced by the commands:
\begin{verbatim}
        \begin{eqngroup}\label{eq234}
        \begin{equation}
        a \frac{\partial u}{\partial t} =
        \frac{\partial}{\partial x}
          K \frac{\partial u}{\partial x},
        \label{eq2}\end{equation}
        where $u = u(x,t)$,
        \begin{eqnarray}
        u(x,0) &=& \phi(x),
          \quad 0\leq x\leq \pi,\label{eq3}\\
        u(0,t) = u(\pi,t) &=& 0,
          \quad t>0.\label{eq4}
        \end{eqnarray}
        \end{eqngroup}
        Here (\ref{eq2}) refers to the differential
        equation, (\ref{eq3}) to the initial conditions,
        (\ref{eq4}) to the boundary conditions, and
        (\ref{eq234}) to the whole problem.
\end{verbatim}
 
\subsubsection{Other useful mathematical commands}
 
The environments `\verb!theorem!', `\verb!definition!', `\verb!lemma!',
`\verb!proposition!' and `\verb!proof!' are available if required---all
except `\verb!proof!' will be numbered.
 
Also the additional mathematical `log-like' functions
\begin{verbatim}
        `\trace',`\sgn', `\cosec',
\end{verbatim}
the commands
\begin{verbatim}
        `\dd', `\DD', `\ee',`\ii'
\end{verbatim}
for roman\footnote{Use roman $\dd$ in $\dd y/\dd x$ and $\int\cdots\dd x$,
roman $\DD$ in $\DD_xy$, roman $\ee$ for $2.718\ldots$ and roman $\ii$ for
$\sqrt{-1}$.}
$$\dd,\ \DD,\ \ee,\ \ii,$$
and the commands
\begin{verbatim}
        `\binom{.}{.}', `\abs{...}', `\norm{...}'
\end{verbatim}
for
$$\binom{\cdot}{\cdot},\ \abs{\cdots}, \ \norm{\cdots}.$$
 
\subsection{Dashes}
 
In the text, use:
\begin{itemize}
\item the single dash `\verb!-!' for a hyphen, to join two words or
part-words one of which is subordinate to the other, {\em e.g.}
`multi-index', `camera-ready', `$n$-dimensional';
 
\item the double dash `\verb!--!' for an `en-dash', to join words, names,
dates or numbers \&c. of equal status, {\em e.g.} `Bernstein--B\'ezier',
`1--10', `surface--surface intersection';
 
\item the triple dash `\verb!---!' for a punctuation dash---as here.
\end{itemize}
In none of the three cases should the dash be preceded or followed by a
space or carriage return.
 
In mathematics, the `\verb!-!' sign is used for `minus' ($-$) in  the usual
way, and spaces in the sourcefile are ignored.
 
\section{Cross-references} \label{s5}
 
\subsection{Equations, \&c.}
 
Numbering of sections, equations, figures, \&c. may alter when \LaTeX\ is
run.
 
\begin{itemize}
 
\item To ensure that the author's cross-references remain correct, replace
any equation reference in the text, {\em e.g.} `(19)', by a reference, {\em
e.g.} `\verb!(\ref{e19})!', labelling the corresponding equation by
`\verb!\begin{equation}! \ldots \verb!\label{e19}! \verb!\end{equation}!'
(or the corresponding thing with `\verb!eqnarray!').
 
\item Likewise, replace figure references, {\em e.g.} `Figure~1', by
references, {\em e.g.} `\verb!Fig.~\ref{f1}!', labelling the figure by
`\verb!\begin{figure}! \verb!\vspace{\ldots}! \verb!\caption[]{...}!
\verb!\label{f1}! \verb!\end{figure}!'; similarly tables
(`\verb!Table~\ref{t1}!'), theorems (`\verb!Theorem~\ref{th1}!'), \&c..
 
\item Replace  a section heading such as `Section 1. Introduction' by
`\verb!\section{Introduction}\label{s1}!'and section references by {\em
e.g.} `\verb!Section~\ref{s1}!', `\verb!Section~\ref{ss1.3}!'.
 
\end{itemize}
 
\subsection{Bibliography}\label{ss52}
 
I have not attempted to impose a totally uniform style of biliography on
these proceedings---this could have been done by the universal application
of {\sc Bib\TeX}, but this would have entailed much conversion work on the
extensive machine-readable bibliographies supplied by some authors.  It is
however worth putting all new bibliographic references in something like a
standard form.
 
\begin{itemize}
 
\item If bibliographical references are cited by number in the manuscript,
replace [2] in the text by `\verb!\cite{r2}!' and identify the item in the
bibliography by `\verb!\bibitem{r2}!';
 
\item if cited by label, then replace `[Bloggs82]' in the text by
`\verb!\cite{Bloggs82}!' and identify the item in the bibliography by
`\verb!\bibitem{Bloggs82}!';
 
\item if neither (cited explicitly by author and year), then include the
command `\verb!\unnumberedbib!' in the preamble, and identify the item
merely by `\verb!\bibitem{}!'.
 
\end{itemize}
 
Items in the bibliography should follow the style of the examples in this
document (page~\pageref{bib}, in which the original citation is assumed to
be by label); note that:
\begin{itemize}
\item the year follows the name(s) of the author(s);
\item italics (`\verb!{\it ...}!') are used for the titles of {\em books}
and
the names of {\em journals} only;
\item in titles of articles, capitals are used for first word and proper
names only;
\item boldface (`\verb!{\bf ...}!') is used for volume numbers of journals;
\item page numbers are separated by `en-dashes' (`\verb!--!');
\item punctuation and spacing is as shown.
\end{itemize}
The entries should appear in alphabetical order of the (first-named)
authors' surnames (\LaTeX\ will automatically (re)number them in that
order if they do not.)
 
The examples in this document were created with the instructions:
\begin{verbatim}
        \begin{thebibliography}{99}
 
        \bibitem{bajaj87} S. S. Abhyankar, C. Bajaj,
        1987: Automatic parameterization of rational
        curves and surfaces II: cubics and cubicoids.
        {\em Computer Aided Design} {\bf 19} 499--502.
 
        \bibitem{buchberger83} B. Buchberger, 1983:
        Gr\"{o}bner bases: an algorithmic method in
        polynomial ideal theory. In {\em Recent Trends
        in Multidimensional Systems}, N. K. Bose, editor,
        D. Reidel, 16--28.
 
        \bibitem{cameron84} S. Cameron, 1984:
        Modelling Solids in Motion.
        Ph. D. Thesis, University of Edinburgh.
 
        \bibitem{walker50} R. J. Walker, 1950;
        {\em Algebraic Curves}.
        Princeton University Press.
 
        \end{thebibliography}
\end{verbatim}
 
\section{Miscellaneous}\label{s6}
 
Do not \underline{underline} for emphasis (or foreign words or phrases,
booktitles in the text); use {\em italics} instead (`\verb!{\em ...}!').
In particular, use `\verb!{\em i.e.}!' and `\verb!{\em e.g.}!'.
 
Title, section headings, \&c. should not end with a full stop. In section
headings and captions, capitalize the first word and proper names only.
 
In text, as distinct from mathematics (where your spacing will be
disregarded), there should be a space (or equivalently a carriage return)
after but {\em not} directly before a punctuation mark, and outside but not
directly inside a bracket $\langle$of any shape\} or quotation mark.  [When
these rules conflict, because two such marks come together, the space is
omitted.]
 
Use single rather than double quotes, and distinguish between left and
right quotes. [If your keyboard has no left quote \{`\} use `\verb!\lq!'
instead.]
 
\begin{ackn}\label{ack}
Acknowledgments should be placed after the conclusion and before the
bibliography, headed by `\verb!\begin{ackn}!' and concluded by
`\verb!\end{ackn}!'.
\end{ackn}
 
        \begin{thebibliography}{99}\label{bib}
 
        \bibitem{bajaj87} S. S. Abhyankar, C. Bajaj,
        1987: Automatic parameterization of rational
        curves and surfaces II: cubics and cubicoids.
        {\em Computer Aided Design} {\bf 19} 499--502.
 
        \bibitem{buchberger83} B. Buchberger, 1983:
        Gr\"{o}bner bases: an algorithmic method in
        polynomial ideal theory. In {\em Recent Trends
        in Multidimensional Systems}, N. K. Bose, editor,
        D. Reidel, 16--28.
 
        \bibitem{cameron84} S. Cameron, 1984:
        Modelling Solids in Motion.
        Ph. D. Thesis, University of Edinburgh.
 
        \bibitem{walker50} R. J. Walker, 1950;
        {\em Algebraic Curves}.
        Princeton University Press.
 
        \end{thebibliography}
 
\appendix
 
\part{Appendices}\label{aA}
 
If there are any appendices, follow the bibliography by \verb!\appendix!,
and begin each appendix with a \verb!\part{...}! command, followed
if necessary by a command like `\verb!\label{aA}!'; thus this appendix
begins with
\begin{verbatim}
        \appendix
\end{verbatim}
 
\begin{verbatim}
        \part{Appendices}\label{aA}
\end{verbatim}
 
\section{Sectioning}\label{sA1}
 
An appendix may, if necessary, be broken down into sections \&c., just as
the main paper is. Cross-references from the paper to entries in an
appendix will need to be in a form like `\verb!Appendix~\ref{aA},!
\verb!equation~\ref{eA1}!'.   Here is an equation (\ref{eA1}) to show what
happens to the numbering:
\begin{equation}b^2=4ac.\label{eA1}\end{equation}
 
\begin{flushright}\today\end{flushright}
 
\end{document}


