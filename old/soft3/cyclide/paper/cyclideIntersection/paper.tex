% header directly from /rb/jj/Research/.cyclide/paper.tex

\documentstyle[12pt]{article} 

\newif\ifFull
\Fullfalse

\makeatletter
\def\@maketitle{\newpage
 \null
 %\vskip 2em                   % Vertical space above title.
 \begin{center}
       {\Large\bf \@title \par}  % Title set in \Large size. 
       \vskip .5em               % Vertical space after title.
       {\lineskip .5em           %  each author set in a tabular environment
        \begin{tabular}[t]{c}\@author 
        \end{tabular}\par}                   
  \end{center}
 \par
 \vskip .5em}                 % Vertical space after author
\makeatother

% non-indented paragraphs with xtra space
% set the indentation to 0, and increase the paragraph spacing:
\parskip=8pt plus1pt                             
\parindent=0pt
% default values are 
% \parskip=0pt plus1pt
% \parindent=20pt
% for plain tex.

\newenvironment{summary}[1]{\if@twocolumn
\section*{#1} \else
\begin{center}
{\bf #1\vspace{-.5em}\vspace{0pt}} 
\end{center}
\quotation
\fi}{\if@twocolumn\else\endquotation\fi}

\renewenvironment{abstract}{\begin{summary}{Abstract}}{\end{summary}}

\newcommand{\SingleSpace}{\edef\baselinestretch{0.9}\Large\normalsize}
\newcommand{\DoubleSpace}{\edef\baselinestretch{1.4}\Large\normalsize}
\newcommand{\Comment}[1]{\relax}  % makes a "comment" (not expanded)
\newcommand{\Heading}[1]{\par\noindent{\bf#1}\nobreak}
\newcommand{\Tail}[1]{\nobreak\par\noindent{\bf#1}}
\newcommand{\QED}{\vrule height 1.4ex width 1.0ex depth -.1ex\ } % square box
\newcommand{\arc}[1]{\mbox{$\stackrel{\frown}{#1}$}}
\newcommand{\lyne}[1]{\mbox{$\stackrel{\leftrightarrow}{#1}$}}
\newcommand{\ray}[1]{\mbox{$\vec{#1}$}}          
\newcommand{\seg}[1]{\mbox{$\overline{#1}$}}
\newcommand{\tab}{\hspace*{.2in}}
\newcommand{\se}{\mbox{$_{\epsilon}$}}  % subscript epsilon
\newcommand{\ie}{\mbox{i.e.}}
\newcommand{\eg}{\mbox{e.\ g.\ }}
\newcommand{\figg}[3]{\begin{figure}[htbp]\vspace{#3}\caption{#2}\label{#1}\end{figure}}
\newcommand{\be}{\begin{equation}}
\newcommand{\ee}{\end{equation}}
\newcommand{\prf}{\noindent{{\bf Proof} :\ }}

\newtheorem{rmk}{Remark}[section]
\newtheorem{example}{Example}[section]
\newtheorem{conjecture}{Conjecture}[section]
\newtheorem{claim}{Claim}[section]
\newtheorem{notation}{Notation}[section]
\newtheorem{lemma}{Lemma}[section]
\newtheorem{theorem}{Theorem}[section]
\newtheorem{corollary}{Corollary}[section]
\newtheorem{defn2}{Definition}

\ifFull                                          
\SingleSpace
\else
\DoubleSpace
\fi

\setlength{\oddsidemargin}{0pt}
\setlength{\evensidemargin}{0pt}
\setlength{\headsep}{0pt}
\setlength{\topmargin}{0pt}
\setlength{\textheight}{8.75in}
\setlength{\textwidth}{6.5in}

\input{clock}

\setlength{\headsep}{.2in}
\setclock              

\title{Drawing on the sphere
        \thanks{This work supported by National Science Foundation grant
        ---.}}                           
\author{John K. Johnstone\thanks{Dept. of Computer and Information 
	Sciences,
	The University of Alabama at Birmingham,
	125 Campbell Hall, 1300 University Boulevard,
	Birmingham, Alabama  35294-1170 USA, johnstone@cis.uab.edu.}
	\and James P. Williams\thanks{Dept. of Computer Science,
	The Johns Hopkins University, Baltimore, Maryland 21218 USA.}}
\date{Version: \today \clock}

\begin{document}

% HEADER ON EACH PAGE
% \markright{\fbox{{\bf Drawing}} \hrulefill 
%          \fbox{{\bf Johnstone, \today}} \hrulefill}
% \pagestyle{myheadings}

\maketitle

% *********************************************************************

\begin{abstract}                                 
This is a paper on animation and swept surface modeling.
\end{abstract}

\Comment{
The orientation of a sweeping object can be represented by
a curve on the 4-dimensional quaternion sphere.

We can define the orientation by manipulating orientation curves
directly (e.g., by moving control points) or by specifying orientations
for the orientation curve to interpolate.
The problem with the former is that general orientation curves are
not rational, so we must define a means of defining rational orientation
curves directly.
We therefore use the second method.

(Could we manipulate the orientation curve on the 4-sphere directly
by moving control points and automatically moving neighbouring
control points to maintain the curve as the image of some space curve
under M?)

Example of sweeping object in solid modeling context for which we would
want to define the orientation.
(Animation has the natural key-frame example, which may or may not
transfer to solid modeling applications.)

This work can be interpreted as the development of methods for
the interpolation of orientations rather than the classical
interpolation of points.
Many desirable criteria for point interpolation translate over
to orientation interpolation, such as continuity conditions.
(Challenge: point interpolation is the domain of CAGD, not
solid modeling.  However, swept surfaces are certainly solids.
May want point location and Boolean operations on these swept
solids too.)

We will certainly want translation mechanisms from the quaternion curve
back to at least parametric representations and preferably implicit
representations of the swept surface, which indeed could be the main topic
of the paper.


-------------------------------

The modeling of a sweep requires the representation of the position
of the object and the orientation of the object at all times.
For simple sweeps such as the ruled surface (the sweep of a line),
the orientation can be represented simply by a vector
(the direction of the line at that time).
[The position is represented by a curve c(t) and the orientation can be
represented either by an explicit vector v(t) at each point c(t)
or, as is commonly done (and called lofting), by a second curve d(t) that 
implicitly defines a vector d(t)-c(t) for c(t).]
However, the orientation of an object generally requires a more complex
representation: by a rotation matrix, a set of Euler angles, or a quaternion.
If the object is planar and rotationally symmetric (i.e., a circle),
its orientation can again be represented by a single vector, the normal
of its plane.

Representation of (single) orientation:
	line		vector
	circle		unit vector (plane's normal; 
				     unit so that rotation matrix is rational)
			actually vector of rational length is enough
	plane curve	vector (plane's normal) + angle of rotation = quaternion
				     i.e., non-rotationally plane curves = plane curves - circle)
	space curve	?
	rigid solid	Euler angles; (?) vector and angle of rotation about this vector = quaternion

Thus, full representation of orientation across sweep:

	line		space curve (either representing endpoint of vector from
				origin or from point c(t))
	circle		3-d spherical curve
			actually, hodograph of Pythagorean hodograph curve is enough
	plane curve	4-d quaternion curve


Ruled surface previous work (e.g., my previous work)
Ringed surface/canal surface previous work (e.g, Farouki and Sakkalis)
}

% *********************************************************************

\section{Conventions}

The homogeneous coordinate will always be listed first
(in some sources the homogeneous coordinate is last).
Thus, $(x_0,x_1,x_2,x_3,x_4)$ in projective space is equivalent to
$(x_1/x_0,x_2/x_0,x_3/x_0,x_4/x_0)$ in affine space.

\section{Problem definition}

The problem is, given n positions of a solid, to define
a path for the solid that interpolates the n positions,
for use in animation and swept surface modeling.
(Each position of the solid is specified by a position for a reference
vertex and an orientation.
Each orientation can be initially specified in any established way: 
by a rotation matrix, a set of Euler angles, or a quaternion.
It will be converted to a quaternion.)

\section{Algorithm}

\begin{rmk}
Explain why we do not interpolate arbitrary quaternions (which could
be done using traditional interpolation) but instead must interpolate
unit quaternions on the sphere,
in terms of the fact that the distance metric for quaternions on the sphere
is the same as the metric for orientation (see Misner), so that we can predict 
the speed of orientation and control it: smooth movement on the sphere
equates to smooth animation of the object.  Does this apply to swept
surface modeling too?

Also, one can directly read off the vector about which to rotate and the 
amount of rotation only from a unit quaternion.
\end{rmk}

We assume the input is a set of $n$ orientations of the solid, as unit
quaternions $p_1,\ldots,p_n$.
A preliminary translation (e.g., rotation matrix to unit quaternion)
may be necessary.

\begin{enumerate}
\item
	(Translate orientations to unit quaternions.)
\item
	Map the points $p_i$ by $M^{-1}$.
	This maps them off of the sphere into 4-space.
\item
	Design a curve $C$ to interpolate the $M^{-1}(p_i)$ in 4-space.
\item
	Map $C$ back to the sphere using $M$, Bezier segment by Bezier segment.
\end{enumerate}

\section{A map onto the 4-sphere}

We want a map from 4-space onto the unit 4-sphere $S^4$.
A formula from number theory yields a solution
\cite{Dickson52}[p. 318].\footnote{This is a special case of
	Euler's famous formula, as well as a special case of a formula of Aida.}

\begin{lemma}[Aida,Euler]
\begin{equation}
\label{eqn:aida}
(a^2 + b^2 + c^2 - d^2)^2 + (2ad)^2 + (2bd)^2 + (2cd)^2 = 
(a^2 + b^2 + c^2 + d^2)^2
\end{equation}
\end{lemma}
\prf
\QED

% This is a special form of Euler's formula \cite{Dickson52}[p. 277],
% $(a^2 + b^2 + c^2 + d^2)(p^2 + q^2 + r^2 + s^2) = 
%  (ap+bq+cr+ds)^2 + 
%  (aq-bp \pm cs \mp dr)^2 + 
%  (ar \mp bs - cp \pm dq)^2 + 
%  (as \pm br \mp cq - dp)^2$.
% Simply let $p=-a, q=b, r=c, s=d$.

\begin{corollary}
The map $M:P^4 \rightarrow S^4 \subset P^4$:
\begin{equation}
\label{eq:M}
	M(1,a,b,c,d) = (a^2+b^2+c^2+d^2,a^2+b^2+c^2-d^2,2ad,2bd,2cd)
\end{equation}
\begin{equation}
	M(1,x_1,x_2,x_3,x_4) = (x_1^2+x_2^2+x_3^2+x_4^2,x_1^2+x_2^2+x_3^2-x_4^2,
				2x_1x_4,2x_2x_4,2x_3x_4)
\end{equation}
sends any point in projective 4-space onto the unit 4-sphere in projective
4-space.
That is, $\| M(1,a,b,c,d) \| = 1$.
\end{corollary}
\prf
\QED

\begin{rmk}
To get a map from 4-space to the unit 4-sphere,
we are looking for any result from number theory 
of the form $A^2+B^2+C^2+D^2=E^2$
(that is, the sum of four squares equal to another square)
where A,B,C,D, and E are functions of at most four variables
and at least one is a function of exactly four variables.
Then a point in 4-space can be
mapped to (E,A,B,C,D) and $\|(E,A,B,C,D)\| = 1$.

(\ref{eqn:aida}) is the only one we have been able to find.
\end{rmk}

\begin{rmk}
In 3-space, mapping to the unit sphere is more elegant because
there is a necessary and sufficient condition for --- (see Dietz-Hoschek).
The above formula is `not a necessary and sufficient condition'
and we are not aware of one that is.
(Note: every sum of three squares is a sum of four squares:
see Dickson, p. 300.)
\end{rmk}

\section{$M^{-1}$: from the sphere to 4-space}

\begin{lemma}
\begin{equation}
\label{eq:invM}
M^{-1}(x_0,x_1,x_2,x_3,x_4) = (\pm 2 \sqrt{\frac{x_0 - x_1}{2}}, x_2,x_3,x_4,
								x_0-x_1)
\end{equation}
where $(x_0,x_1,x_2,x_3,x_4)$ lies on the unit 4-sphere
({\em i.e.}, $M^{-1}:S^4 \rightarrow P^4$).
\end{lemma}
\prf
See Aug 16 message to jimbo.
\QED

\section{Image of cubic Bezier segment under M}

\begin{lemma}
The image of a cubic Bezier segment (defined by the four control points 
$b_i = (1,b_{i1},b_{i2},b_{i3},b_{i4}) \in P^4$) under the map M 
(see (\ref{eq:M})) is a
Bezier segment of degree 6 defined by the following seven control points:
\begin{equation}
c_k = \sum_{i,j \geq 0, i+j=k} 
        \choose{3}{i}{} * \choose{3}{j}{} / \choose{6}{k}{}
                (b_{i1} b_{j1} + b_{i2} b_{j2} + b_{i3} b_{j3} + b_{i4} b_{j4},
                 b_{i1} b_{j1} + b_{i2} b_{j2} + b_{i3} b_{j3} - b_{i4} b_{j4},
                 2b_{i1} b_{j4},
                 2b_{i2} b_{j4},
                 2b_{i3} b_{j4})
\end{equation}
\end{lemma}
\prf
\QED

\section{Implementation}

\begin{itemize}
\item
	$M$ and $M^{-1}$ are implemented in Maple by jimbo,
	but must be implemented in GL.
\item
	Rotation matrix to quaternion and quaternion to rotation matrix,
	using Shoemake (or perhaps direct definition of the orientation
	in terms of quaternions is sufficient).
\item
	Cubic Bezier interpolation: using Farin's tridiagonal technique
	(possibly verbatim using C code on disk: need IBM PC to read it).
\item
	mapping of cubic Bezier segments to sextic Bezier segments
	under M
\item
	definition of cube (or other candidate for solid)
\end{itemize}

\bibliographystyle{alpha}
\bibliography{/rb/jj/bib/modeling}

\end{document}
