% I think that this is what I originally submitted to CAGD

\documentstyle[12pt]{article} 

\newif\ifFull
\Fullfalse

\makeatletter
\def\@maketitle{\newpage
 \null
 %\vskip 2em                   % Vertical space above title.
 \begin{center}
       {\Large\bf \@title \par}  % Title set in \Large size. 
       \vskip .5em               % Vertical space after title.
       {\lineskip .5em           %  each author set in a tabular environment
	\begin{tabular}[t]{c}\@author 
	\end{tabular}\par}                   
  \end{center}
 \par
 \vskip .5em}                 % Vertical space after author
\makeatother

% non-indented paragraphs with xtra space
% set the indentation to 0, and increase the paragraph spacing:
\parskip=8pt plus1pt
\parindent=0pt
% default values are 
% \parskip=0pt plus1pt
% \parindent=20pt
% for plain tex.

\newenvironment{summary}[1]{\if@twocolumn
\section*{#1} \else
\begin{center}
{\bf #1\vspace{-.5em}\vspace{0pt}} 
\end{center}
\quotation
\fi}{\if@twocolumn\else\endquotation\fi}

\renewenvironment{abstract}{\begin{summary}{Abstract}}{\end{summary}}

\newcommand{\SingleSpace}{\edef\baselinestretch{0.9}\Large\normalsize}
\newcommand{\DoubleSpace}{\edef\baselinestretch{1.4}\Large\normalsize}
\newcommand{\Comment}[1]{\relax}  % makes a "comment" (not expanded)
\newcommand{\Heading}[1]{\par\noindent{\bf#1}\nobreak}
\newcommand{\Tail}[1]{\nobreak\par\noindent{\bf#1}}
\newcommand{\QED}{\vrule height 1.4ex width 1.0ex depth -.1ex\ } % square box
\newcommand{\arc}[1]{\mbox{$\stackrel{\frown}{#1}$}}
\newcommand{\lyne}[1]{\mbox{$\stackrel{\leftrightarrow}{#1}$}}
\newcommand{\ray}[1]{\mbox{$\vec{#1}$}}
\newcommand{\seg}[1]{\mbox{$\overline{#1}$}}
\newcommand{\tab}{\hspace*{.2in}}
\newcommand{\se}{\mbox{$_{\epsilon}$}}  % subscript epsilon
\newcommand{\ie}{\mbox{i.e.}}
\newcommand{\eg}{\mbox{e.\ g.\ }}
\newcommand{\figg}[3]{\begin{figure}[htbp]\vspace{#3}\caption{#2}\label{#1}\end{figure}}
\newcommand{\be}{\begin{equation}}
\newcommand{\ee}{\end{equation}}
\newcommand{\prf}{\noindent{{\bf Proof} :\ }}

\newtheorem{rmk}{Remark}[section]
\newtheorem{example}{Example}[section]
\newtheorem{conjecture}{Conjecture}[section]
\newtheorem{claim}{Claim}[section]
\newtheorem{notation}{Notation}[section]
\newtheorem{lemma}{Lemma}[section]
\newtheorem{theorem}{Theorem}[section]
\newtheorem{corollary}{Corollary}[section]
\newtheorem{defn2}{Definition}

\ifFull
\SingleSpace
\else
\DoubleSpace
\fi

\setlength{\oddsidemargin}{0pt}
\setlength{\evensidemargin}{0pt}
\setlength{\headsep}{0pt}
\setlength{\topmargin}{0pt}
\setlength{\textheight}{8.75in}
\setlength{\textwidth}{6.5in}

\input{clock}

\setlength{\headsep}{.2in}
\setclock

% POSSIBLE TITLES (FOR THIS PAPER OR SPINOFFS)
	% A new approach to intersection through decomposition
	
\title{A New Intersection Algorithm using Circle Decomposition
%	Circle decomposition and its use in intersection,
%		with particular attention to cyclides
	\thanks{This work supported by National Science Foundation grant
	IRI-8910366.}}
% 	DON'T WANT CYCLIDES IN THE TITLE
% 	\\ Intersection from circles, a new exact method for an extended
%		vocabulary of primitives
%	\\(Preliminary version)
\author{John K. Johnstone\thanks{Department of Computer Science,
	The Johns Hopkins University,
	Baltimore, Maryland 21218 USA,
	jj@cs.jhu.edu.}}
\date{Version: \today \clock}

\begin{document}

% HEADER ON EACH PAGE
% \markright{\fbox{{\bf Cyclides}} \hrulefill 
% 	   \fbox{{\bf Johnstone, \today}} \hrulefill}
% \pagestyle{myheadings}

\maketitle

% *********************************************************************

% \input{newintro}
% % \input{intro}

\begin{abstract}
The present vocabulary of a solid modeler is canonically the plane,
(some subset of) the quadrics, and the torus.
The class of cyclides is also becoming important.
Quadrics and cyclides lie in the more general class of ringed surfaces:
surfaces that can be swept out by a circle.
This class also contains the important class of revolute surfaces.
We will present a method for the exact intersection of any ringed surface with
any quadric or cyclide.
This algorithm shows that it is feasible to expand the 
vocabulary of solid modeling primitives to include all ringed surfaces.
In solid modeling, surface intersection is crucial to the design of solids
and their subsequent analysis.

Our intersection algorithm is exact: 
that is, the intersection is computed symbolically rather than numerically.
For exact intersection, we must reduce to degree 4 computations.
We do this by concentrating on the decomposition of a surface into simpler
components.
Previous algorithmic development has centered around 
the degree of an algebraic surface.
Two keys to our algorithm are circle decomposition and inversion.
% We will fully develop 
% ALREADY DONE, SO DON'T EMPHASIZE
% the circle decomposition of a cyclide 
% (i.e., its description as a union of circles),
% and results on the inverse of any circle.
\end{abstract}

% Amy and I worked on the center of the inverse of a circle,
% and more globally on finding the directrix curve (curve of centers) 
% of a cyclide via mapping of directrix curve of torus or cylinder
% The theme was `directrix curves of lower degree than the surface'
% (e.g., line directrix on quadric ruled, conic directrix on cyclide)
% which I believed was a good way of reducing the complexity of the 
% computations on a surface.

Keywords: intersection, cyclides, circles, inversion, ringed surfaces


\section{Introduction}

In this paper, we develop a new method for intersection that can be used
to exactly compute the intersection of higher degree surfaces.
The surfaces that we concentrate on are swept surfaces called {\em ringed}
surfaces, which include any surface that can be swept out by a circle.
This class contains all revolute surfaces and all quadrics, surfaces that
are very important for design.
Note that a ringed surface can be of arbitrarily high degree.

We also concentrate on cyclides, which are special ringed surfaces 
and a natural extension of the quadric and torus.
Cyclides have received considerable attention by the solid modeling
community as a blending surface, a patch, and a tractable surface.

It is desirable to add both ringed surfaces and cyclides to a solid modeler,
which in its rudimentary form only includes planes, natural quadrics, and tori.
A key step in the integration of any primitive into the vocabulary of a solid
modeler is the development of an intersection algorithm.
This paper addresses this problem.

We develop a method for the intersection of any quadric or cyclide surface
with a ringed surface.
The new method uses circle decomposition and inversion to solve the 
intersection problem, rather than the traditional elimination, tracing, or
implicit/parametric methods.
This new intersection technique is powerful: a
complicated surface such as the quartic cyclide can be intersected exactly
with a surface of arbitrary degree, without resorting to approximate methods.
This defies the common conception that exact intersection is restricted to 
low degree surfaces.

In the next section, we establish that cyclides and ringed surfaces 
are the natural extensions of quadric surfaces and the natural primitives 
to include in the vocabulary of a solid modeler, after the quadric surface.
Section~\ref{prev-wk} reviews previous work on intersection and cyclides.
The next two sections formally introduce cyclides and the inversion map, 
respectively.
We present our new algorithm for the intersection of a ringed surface and 
a quadric or cyclide in Section~\ref{sec:nim}.
Section~\ref{sec:inv-cyclide} shows how to compute the inverse of a torus,
while the more difficult inversion from a cyclide to a torus is developed
in Section~\ref{sec:totorus}.
We discuss the mapping of a cyclide to a quadric in 
Section~\ref{sec:toquadric}.
The precise effect of inversion on a circle is examined in Section~\ref{sec1}.
Methods for the circle decomposition of a cyclide are presented in
Section~\ref{sec-decomp}, while the advantages of this decomposition 
are analyzed in Section~\ref{advs}.
Section~\ref{sec:conc} ends with conclusions and ideas for future work.

\section{Cyclides and ringed surfaces are the natural extensions of quadrics}

In this section, we would like to argue that the 
cyclide is the most natural extension of the present solid modeling
CSG (constructive solid geometry) vocabulary.
The plane, the quadric surfaces, and the torus are the most fundamental
primitives for design.
Revolute surfaces are also very popular, because they arise naturally
in manmade objects.
All of these surfaces (except the plane) are examples of ringed surfaces.

\begin{defn2}
\label{ringed}
A {\bf ringed}\footnote{
	% A ruler is `a strip of wood or metal having a straight edge used for
	% measuring or drawing lines'.
	Since we have not been able to find a term for this class of surfaces
	in the literature, we have introduced the term `ringed'.
	The following definition explains our choice of this term:
	{\em Ring}: anything having a circular form; v.,
	to move in a ring or a curving course (Random House Dictionary, 1980).}
{\bf surface} % (or cyclic surface, Gordon) : sounds stupid
is a surface generated by a circle sweeping through space.
% (i.e., a single-parameter family of circles).
The curve that the center of the circles sweeps along is 
a {\bf directrix curve}.
A {\bf radius function} and {\bf orientation function} 
give the radius and orientation of the circle at each point of the
directrix curve.
\Comment{
There may be more than one directrix curve,
since there may be more than one way to generate the ringed surface by
sweeping a circle through space.
}
\end{defn2}

\begin{rmk}
A well-known subclass of ringed surfaces is the canal surface, 
% (see \cite{rr84,rossignac85})
which is a surface generated by sweeping a sphere along a curve
% Hilbert, p. 219; Rossignac and Requicha on constant-radius blending surfaces
or, equivalently, a ringed surface where the 
plane of the circle is always perpendicular to the directrix.
% this last point is mentioned in Gordon too: p. 207
Also, note the similarity between ringed and ruled surfaces, 
which are surfaces generated by a line sweeping through space.
\end{rmk}

The Dupin cyclide is a special ringed surface that has received
considerable attention (see Section~\ref{prev-wk}).
It is very useful in blending and patching.
We shall introduce it formally in Section~\ref{sec-cy}.
We believe that the renewed interest in the cyclide is natural:
it is not only very useful, but it is the logical extension
of the quadric surface and the torus, which are of course the most
popular primitives for solid modeling.
To see this, note that the quadric is a ringed surface with
a line directrix and a quadratic radius function, 
a torus is a ringed surface with a circle directrix and a constant
radius function, 
while a Dupin cyclide is a ringed surface with a conic directrix 
and a quadratic radius function.

% *conic directrix* rather than line, but still *quadratic radius function*
% thus, can get degree 8 intersection of cyclides, 
% using circle x conic parameterization of cyclide
%
% can cyclide be input via conic directrix curve and radius function?
%
%	   Forget about quadratic radius function: it is a bell.
%	   (Determining the algebra of the radius function for the
%	    space circle in Corollary 5.1 is too difficult.)
%	   Having a conic directrix curve for sweeping circle is enough.

% *********************************************************************

% \input{prev-wk}

\section{Previous work}
\label{prev-wk}

Our method of intersection, using circle decomposition and inversion,
contrasts with more traditional techniques of intersection, such as
tracing and elimination \cite{HOF89}.
Tracing \cite{BHHL88} is an excellent technique 
when numeric computation of the intersection is necessary.
The output is a collection of points on the intersection 
of two surfaces.
Elimination theory provides a very general method for the intersection 
of two curves or surfaces, using the elimination of a variable by a 
resultant.
This method is not appropriate for the intersection of a ringed surface 
and a cyclide, because the resultant is of degree $4n$,
where $n$ is the degree of the ringed surface (e.g., $4n = 16$ if the ringed
surface is a cyclide).
% One must be careful about extraneous factors whenever using resultants.

Another popular method of intersection is the implicit/parametric method:
(1) the parametric equation of one surface is substituted into the implicit
equation of the other surface, (2) the resulting equation (in the
parameters $s$ and $t$) is solved for $s$ (in terms of $t$), 
and (3) this solution is substituted back into the original parametric 
equation.
It is important to note that de Pont and Martin \cite{DEP84,MDS86} use
this method to solve the intersection of cyclides with planes, quadrics,
and cyclides.
They observe that the implicit/parametric method is applicable to 
cyclide/cyclide intersection because the equation derived in step (2) 
is of degree 4 in each of the parameters, although it is of degree 8 overall.
Thus, it is possible to solve the equation symbolically for either $s$
or $t$.
Our method is an extension of their work to general ringed surfaces, as
well as an interesting alternative for cyclide/cyclide and cyclide/quadric
intersection.

% see circle-cyclide-int-is-tough file for elaboration

% LOOK IN HOFFMANN DISCUSSION OF INTERSECTION?  SHOWS SOME WEAKNESSES OF 
% IMPLICIT/PARAMETRIC THAT MAY NOT APPLY TO OUR METHOD.

% EVEN FOR THIS SUBCASE OF OUR PROBLEM, OUR METHOD MAY BE PREFERABLE
% IF THERE ARE SOME DISADVANTAGES OF IMPLICIT/PARAMETRIC LISTED IN HOFFMANN.

% **********************


% THIS COMMENT WAS REMOVED IN TECH REPORT VERSION
\begin{rmk}
Sabin \cite{Sabin89} comments on the tractability of cyclide intersection,
% p. 150
% [Something like `Two cyclides always share a component at infinity, 
% called the circle at infinity.']
noting that the real part of the intersection of two cyclides is a curve 
of degree 8 (rather than 16 as expected) and genus 3, 
the latter property implying that it has a 
parametric equation in terms of solutions of fourth order equations, 
and thus can be solved symbolically.

It is interesting that cyclide/cyclide intersection is exactly
solvable, reducing to a degree 4 computation, despite the fact that
computing the distance between two circles, which is related to 
detecting a null intersection between two tori, has no closed-form
solution \cite{neff90}.
% Note that the output of our intersection algorithm is a parameterization,
% which might have no real locus.
% That is, we do not directly state whether the intersection is or is not
% empty.
\end{rmk}

% *************************************************

There is a considerable body of work on the Dupin cyclide.
Two early classical mathematical studies are the papers 
of Maxwell \cite{MAX68} and Cayley \cite{CAY96}.
They develop the key properties of Dupin cyclides.
% e.g., anticonics, inversion of cyclide is cyclide, parabolic cyclides
% Another important development of cyclide theory is in Forsyth \cite{F12}.
%
A more recent discussion of the 
geometry, construction, and classification of cyclides is
available in Chandru, Dutta, and Hoffmann \cite{CDH89a}.
Cyclides are an object of interest for mathematicians,
because they are the only surfaces whose lines of curvature are all circular,
and because they are the only surfaces whose evolute (or surface of centers)
consists of two curves, rather than two surfaces \cite{H52}.
	% Hilbert, p. 217

The use of cyclides in solid modeling and design was first explored
at Cambridge, with a collection of work including the theses of Martin 
\cite{MAR82}, de Pont \cite{DEP84}, and Sharrock \cite{SHAR85}.
They develop a surface design system based on patching together cyclides.
The system uses principal patches, whose boundaries are lines of curvature, 
which makes the use of cyclides natural.

The use of cyclides as blending surfaces has been developed by Chandru, Dutta, 
Hoffmann, in Hoffmann \cite{H88}, then Chandru, Dutta, and Hoffmann 
\cite{CDH89b-ifip}, and Dutta \cite{Dutta89}.
Blending is the operation of smoothing off sharp corners and edges,
a fundamental operation in solid modeling.
Cyclides prove to be natural blending surfaces,
because they are the envelope of a sphere rolling about a conic.
The rolling of a sphere about the intersection of two surfaces is a good way
of blending them; and the spine of this spherical envelope can be approximated
by conic segments.
Pratt \cite{P89} also discusses blending with cyclides,
this time of cylinder/plane, 
cone/sphere, and cone-torus (the Cranfield object).
He also considers the Bezier formulation of cyclides and their offsets.
Other papers on cyclides have been written by McLean \cite{Mc85} and Degen
\cite{Degen90}.
% already mentioned above:
% and Martin, de Pont, and Sharrock \cite{MDS86}.
% (discussing intersection and blending).

The use of a circle decomposition of a ringed surface to solve
ringed/quadric and ringed/cyclide intersection is reminiscent of the use
of a line decomposition of a ruled surface to solve the intersection
of a ruled surface with any surface of degree four or less.
The decomposition of ruled surfaces to the $\mbox{line}(t)$ representation 
is covered in Johnstone \cite{JOH89}.
Levin recognized that it is easy to intersect with ruled surfaces: 
he reduced quadric intersection to the 
intersection of a quadric with a ruled \cite{LEVI76}.

% *********************************************************************

% structure of cyclide
% \input{cyclide}

\section{Dupin cyclides}
\label{sec-cy}

\subsection{Definitions}

There are many definitions in use for a cyclide,
so it is important to make it clear what our definition is.
A general definition of a cyclide is 
a quartic surface with a double curve at the circle at infinity\footnote{The
	circle at infinity is the curve $x^{2} + y^{2} + z^{2} + w^{2} = 0$.}
% (i.e., contains the circle at infinity twice)
or a cubic surface that contains the circle at infinity
(e.g., \cite{F12,SOM47}). % p. 408 of Sommerville, p. 324 of Forsyth
A Dupin cyclide is a special type of cyclide,
and the first type of cyclide studied.
The most general definition of a Dupin cyclide, 
and one that is often used in the literature (e.g., \cite{CDH89a,FISCH86}),
% Dutta on p. 288, Fischer on p. 28
is a surface whose lines of curvature are all circles or lines.
% we earlier called this a {\em general Dupin cyclide}
% plane, sphere, cylinder, cone, 
% parabolic cyclide, or a quartic Dupin cyclide.)
% e.g., the two families of lines of curvature of a cylinder are lines and 
% circles
Note that, under this definition, the Dupin cyclide includes 
the plane, sphere, cylinder, and cone: those key surfaces that are
already in the vocabulary of a solid modeler.
In this paper, we shall use a more restricted definition of the Dupin cyclide,
which agrees with Dupin's original definition 
(the envelope of a sphere
that is tangent to three fixed spheres in a continuous manner \cite{MAX68}).

\begin{defn2}
\label{ourdefn}
A {\bf Dupin cyclide} is a surface whose lines of curvature are all circles.
\end{defn2}

This definition removes surfaces such as the plane, cylinder, cone, and 
parabolic cyclide, some of whose lines of curvature are lines.
In the rest of the paper, `cyclide' will refer to the quartic Dupin cyclide
of Definition~\ref{ourdefn}.

\begin{rmk}
\label{rmk:conic}
The centers of the lines of curvature of a Dupin cyclide 
lie on a conic \cite{F12,H52}. 
% curve of centers: Hilbert, Forsyth; also Coolidge,p267
That is, the Dupin cyclide is a circle sweeping along a conic.
\end{rmk}

In this sense, a Dupin cyclide is the generalization of a quadric surface,
since any quadric surface is a circle sweeping along a line.

\figg{cyc-pic}{Cyclides: (a) Ring cyclide (b) Self-intersecting torus 
	[13]
	(c) Singly-horned cyclide (d) Doubly-horned cyclide [17]}{5in} % 5.5in

\subsection{Mechanisms for defining cyclides}

\subsubsection{Implicit equation}

Forsyth \cite{F12} 
developed a normal form for the implicit equation of a Dupin cyclide:

\be
\label{eqcy}
(x^{2} + y^{2} + z^{2} - \mu^{2} + b^{2})^{2} = 4(ax-c\mu)^{2} + 4b^{2}y^{2}
\ee
where $a,b > 0$, $c,\mu \geq 0$, and $a^{2} = b^{2} + c^{2}$.
% see equivalent parametric definition in Depont: c=0 is OK

	% not difficult to translate to this implicit form
	% e.g., if two circles are given, we can map them to normal form and
	% use the computation of the two circles below to determine what the 
	% normal form is; or if parameterization is used, use de Pont's 
	% parameterization equivalent on p. 31.

\subsubsection{The planes of symmetry of a cyclide}

The cyclide has two planes of symmetry, and its cross-section
by either of these planes is a pair of circles.
These circles yield a lot of information about the cyclide.

\figg{fig:sym}{(a) $y=0$, the exterior plane of symmetry
	       (b) $z=0$, the interior plane of symmetry}{5in} % 6in

\begin{lemma}
\label{lem:sym}
The planes of symmetry of the cyclide in (\ref{eqcy}) are $y=0$ and $z=0$
(Figure~\ref{fig:sym}).
The two circles in the plane $y=0$ are $(x-a)^2 + z^2 = (\mu - c)^2$
and $(x+a)^2 + z^2 = (\mu + c)^2$.
The two circles in the plane $z=0$ are $(x-c)^2 + y^2 = (\mu - a)^2$
and $(x+c)^2 + y^2 = (\mu + a)^2$.
\end{lemma}
\prf 
The pair of circles in $y=0$ are found by factoring 
$(x^{2} + z^{2} - \mu^{2} + b^{2})^{2} = 4(ax-c\mu)^{2}$
into $[(x^{2} + z^{2} - \mu^{2} + b^{2} - 2(ax-c\mu)]
      [(x^{2} + z^{2} - \mu^{2} + b^{2} + 2(ax-c\mu)] = 0$,
which is equivalent, using $a^2 = b^2 + c^2$, to
$[(x-a)^2 + z^2 - (\mu - c)^2][(x+a)^2 + z^2 - (\mu + c)^2]$.
The same technique is used for the pair of circles in $z=0$,
except the equation (\ref{eqcy}) for the cyclide is replaced by
the following equivalent equation \cite{F12}:
\be
(x^{2} + y^{2} + z^{2} - \mu^{2} - b^{2})^{2} = 4(cx-a\mu)^{2} - 4b^{2}z^{2}
\ee
\QED

\begin{rmk}
\label{rmk:ext}
The smaller circle in the $y=0$ plane never lies inside the larger one
(i.e., $2a + |\mu - c| > \mu + c$).
% easy to show 
Conversely, the smaller circle in the $z=0$ plane never lies outside the 
larger one (i.e., $2c < |\mu - a| + \mu + a$).
\end{rmk}

Remark~\ref{rmk:ext} motivates the following terminology.

\begin{defn2}
The {\bf exterior} (resp., {\bf interior}) {\bf plane of symmetry} of a cyclide
is the plane of symmetry whose smaller circle never lies inside 
(resp., outside) the larger circle.
The {\bf exterior} (resp., {\bf interior}) {\bf circles} are the two circles 
that lie in the exterior (resp., interior) plane of symmetry.
\end{defn2}

\begin{rmk}
$y=0$ is the exterior plane of symmetry and 
$z=0$ is the interior plane of symmetry for the normal form (\ref{eqcy}).
\end{rmk}

We can identify three configurations for the
exterior and interior circles of a cyclide:
\begin{enumerate}
\item
	the smaller exterior circle lies completely outside the larger exterior
	circle, and the smaller interior circle lies completely inside the
	larger interior circle;
\item
	the smaller exterior circle intersects the larger exterior circle,
	and the smaller interior circle lies completely inside the
	larger interior circle;
\item
	the smaller exterior circle lies completely outside the larger exterior
	circle, and the smaller interior circle intersects the
	larger interior circle.
\end{enumerate}

Associated with these three configurations are the three classes of cyclides:
{\em ring} cyclides, {\em self-intersecting} 
cyclides, and {\em horned} cyclides, respectively (see Figure~\ref{cyc-pic}).
The horned cyclide splits further into {\em singly-horned}
(when the smaller interior circle is tangent to the 
larger interior circle) and {\em doubly-horned} (otherwise).
The {\em horns} of a horned cyclide are the intersections of its interior
circles.

\begin{rmk}
\label{rmk:rel}
Using the normal form (\ref{eqcy}) and Figure~\ref{fig:sym},
we note that the exterior circles intersect if and only if $\mu \geq a$,
while the interior circles intersect if and only if $\mu \leq c$.
(For example, $2c + (a-\mu) \geq \mu+a \Rightarrow 2c \geq 2\mu$.)
Thus, a cyclide is horned if $\mu \leq c$, ring if $c < \mu < a$, 
and self-intersecting if $\mu \geq a$.
\end{rmk}

The above development gives us a very useful way of defining a cyclide.
The following result can be easily proved using Figure~\ref{fig:sym}.

\begin{notation}
Let $C(c,r,o)$ be the circle with center $c$ and radius $r$
that lies in the plane with normal $o$.
\end{notation}

\begin{lemma}
\label{lem:uni}
A cyclide is fully determined by its exterior circles and its type,
or its interior circles and its type.
In particular, the cyclide with exterior circles $C_1(c_1,r_1,o)$
and $C_2(c_2,r_2,o)$, $r_1 \geq r_2$
is the cyclide with normal form (\ref{eqcy}) where
\begin{itemize}
\item
	$a = \frac{\|c_2 - c_1\|}{2}$, 
\item
	$\mu = \left\{ \begin{array}{ll}
		\frac{r_1 - r_2}{2} &	\mbox{if the cyclide is horned} \\
		\frac{r_1 + r_2}{2} &	\mbox{otherwise}
	\end{array}
	\right. $
\item
	$c = r_1 - \mu$, and 
\item
	$b = \sqrt{a^2 - c^2}$,
\end{itemize}
under the rigid transformation
that rotates the $y$-axis to the
vector $o$, and translates the origin to $\frac{c_1 + c_2}{2}$.\footnote{Note 
	that the normal form 
	(\ref{eqcy}) assumes that the exterior circles
	lie in the plane $y=0$, centered at the origin.}
% That is, if $f(x,y,z)$ is equation of this normal form,
% the equation of the cyclide is $f(M1 M2 (x,y,z))$.

The cyclide with interior circles $C_1(c_1,r_1,o)$
and $C_2(c_2,r_2,o)$, $r_1 \geq r_2$
is the cyclide with normal form (\ref{eqcy}) where
\begin{itemize}
\item
	$c = \frac{\|c_2 - c_1\|}{2}$, 
\item
	$\mu = \left\{ \begin{array}{ll}
	\frac{r_1 + r_2}{2} & \mbox{if the cyclide is self-intersecting} \\
	\frac{r_1 - r_2}{2} & \mbox{otherwise}
	\end{array}
	\right.$
\item
	$a = r_1 - \mu$, and 
\item
	$b = \sqrt{a^2 - c^2}$,
\end{itemize}
under the rigid transformation that rotates the $z$-axis to the
vector $o$, and translates the origin to $\frac{c_1 + c_2}{2}$.
%
% CONDITIONS THAT CIRCLES BE PROPER INTERIOR AND EXTERIOR CIRCLES
%	(E.G., FOR INPUT MECHANISM OF CYCLIDES)
% circles $C_1$ and $C_2$ in one plane of symmetry 
% and $C_3$ and $C_4$ in the other plane of symmetry, 
% where $C_i$ is the circle\footnote{There are two special cases 
% 	where one of the
% 	circles degenerates to a point.
%	If $\mu = c$, the circle $C_2$ degenerates to a point.
% 	If $\mu = a$, the circle $C_4$ degenerates to a point.}
% with center $c_i$ and radius $r_i$, $i=1,2,3,4$,
% such that
% \begin{enumerate}
% \item
%	$C_1$ and $C_2$ lie in the same plane;
% \item
% 	$C_3$ and $C_4$ lie in the same plane, perpendicular to the plane of 
%	$C_1$ and $C_2$;
% \item	
%	$r_1 \geq r_2$ and $C_2$ does not lie completely inside $C_1$;
% \item	
%	$r_3 \geq r_4$ and $C_4$ does not lie completely outside $C_3$;
% \item
%	$\|c_1 - c_2\| > \|c_3 - c_4\|$;  % so that a > c
% \item
%	$r_3 = r_1 - \frac{\|c_3 - c_4\|}{2} + \frac{\|c_1 - c_2\|}{2}$
%	and $r_4 = |r_1 - \frac{\|c_3 - c_4\|}{2} - \frac{\|c_1 - c_2\|}{2}|$.
% \end{enumerate}
%
\end{lemma}

\begin{rmk}
In this paper, we are consistently able to represent a cyclide 
by two circles in a plane of symmetry.  All of the algorithms
that deal with a cyclide use this representation (Theorems~\ref{thm:invtorus} 
and \ref{cor:coi}).

Cayley also used two circles in a plane of symmetry to define the cyclide.
However, his method of deriving the cyclide from these two circles is 
quite different (see \cite[p. 285]{CDH89a}).

Chandru, Dutta, and Hoffmann \cite{CDH89a} observe that the input of a
cyclide by two (exterior) circles in its plane of symmetry is natural 
for blending, where the circles can encode two (extremal) positions
of a rolling ball blend.
\end{rmk}

\Comment{
\subsubsection{Cyclide as circle $\times$ conic}

Good representation because: 
	Simple point classification (like any implicit):
		find the point q of the ellipse such that p lies on its
		normal plane
			IS THIS EASY?
		p lies on cyclide $\Leftrightarrow$ 
		dist(p,q) = radius of circle at q
	Generate points (like any parametric)
	Don't need to store orientation function: tangent to conic at q 
		is always the normal of the plane of the circle at q


Many authors use this representation (e.g. Dutta (conic = spine for blending))

Maxwell's equation in terms of anticonic parameters: p. 286 of Dutta.

}

% *********************************************************************

% theory of inversion
% \input{inversion}

\section{Inversion}
\label{sec:inv}

In this section, we formally introduce the inversion map.
As suggested by Figure~\ref{fig1},
inversion is the generalization of reflection.
While a point is reflected in a line or plane, it is inverted with respect
to a circle or sphere.

In the following discussion, we shall present definitions and results
for the sphere; however, most results also apply to inversion in a 
circle, by simply replacing sphere by circle and plane by line.

\figg{fig1}{(a) Reflection in a line (b) Inversion in a circle}{2.5in}

\begin{defn2}
$\mbox{inv}_{S}(p)$, the {\bf inverse of a point} $p \in \Re^{3}$ in the 
sphere $S$ of radius $R$ centered at $c$, 
is the point $p' \in \ray{cp}$ such that
$\mbox{dist}(c,p)\ *\ \mbox{dist}(c,p') = R^{2}$.
$S$, $c$ and $R$ are called the {\bf sphere of inversion},
the {\bf center of inversion},
and the {\bf radius of inversion}, respectively.
As a special case, the inverse of a point in a plane is the reflection
of that point in the plane.
% Johnson, p. 44 \cite{J29}.
%
\Comment{
One may also define an inversion map restricted to a plane.
Let $C$ be a circle of radius $r$ centered at $c$ and lying in the plane $P$.
The {\em inverse} of a point $p \in P$ in the circle $C$, 
$\mbox{inv}_{C}(p)$, is the point $p' \in \ray{cp}$ such that
$\mbox{dist}(c,p)\ \mbox{dist}(c,p') = r^{2}$.
}
\end{defn2}

Note that, under inversion, the interior of the sphere of inversion
is mapped to its exterior and vice versa, while the sphere is mapped to itself.

\begin{lemma}
\label{lem-inv}
Let $S$ be the sphere of inversion with center $c$ and radius $R$.
$\mbox{inv}_{S}(p) = \frac{R^{2}}{||p-c||^{2}} (p-c) + c$.
\end{lemma}
\Heading{Proof:}
The inverse of $p$ is $p' = c + k(p-c)$, where $k\|p - c\|*\|p - c\| = R^{2}$.
\QED

The most important property of inversion is that, 
in general, it maps circles to circles.
\Comment{
% ***********************************
% THING TO DO: PROVE FOR GENERAL CASE; REQUIRES PROOF THAT INV(N-SPHERE)
% 		IS AN N-SPHERE
%
% IN JOURNAL: 
% STATE GENERAL CASE IN TERMS OF I-SPHERE.
% ELEGANT TO STATE MOST GENERAL CASE;   MAIN PROBLEM: GOOD DEFN OF I-SPHERE
% 	COLLAPSES SEVERAL CASES INTO ONE;
%	ALLOWS STATEMENT OF GENERALIZED PROOF;
%	FORCES ME TO UNDERSTAND CIRCLE -> CIRCLE PROOF SO THAT I CAN EXTEND
%	TO N-SPHERE -> N-SPHERE

\begin{description}
\item[(1.4a)]
\label{lem:inversion}
	The inverse of an $i$-sphere is an $i$-sphere or an $i$-plane,
	$i \leq n$
	\cite{D49,Coo71}.
	It is an $i$-plane exactly when the $i$-sphere 
	passes through the center of inversion.
	%
	% \cite[p. 210]{D49}: nice proof for plane circles.
	% \cite[p. 26, 228]{Coolidge} % statement for plane circles (p. 26)
			    % statement for spheres and space circles (p. 228)
PROOF: PROOF FOR PLANE CIRCLES (E.G., JOHNSON, P. 50) AND LINES (P. 49)
	GENERALIZES DIRECTLY TO HYPERSPHERES AND HYPERPLANES.
	NOW YOU CAN PROVE FOR $I$SPHERE BY INDUCTION, AND INTERSECTION.
\end{description}

\noindent where

\begin{defn2}
An $i$-sphere (resp., $i$-plane) is a hypersphere (resp., hyperplane) 
in $R^{i}$.
Thus, a 2-sphere is a circle and a 3-sphere is the typical sphere.
More technically, in $\Re^{n}$, 
a hypersphere is an $n$-sphere, a hyperplane is an $n$-plane,
an $i$-sphere is the intersection of a hypersphere and an $i+1$-plane ($i < n$)
and an $i$-plane is the intersection of a hyperplane and an $i+1$-plane 
($i < n$).
\end{defn2}

We are interested in the following subcase: 
\begin{description}
\item[]
	The inverse of a circle is a circle or a line.
	It is a line exactly when the circle 
	passes through the center of inversion.
\end{description}

}

\begin{description}
\item[(1)]
\label{lem:inversion}
	The inverse of a circle is a circle or a line.
% (e.g., \cite{Coo71})
	It is a line exactly when the circle 
	passes through the center of inversion.\footnote{This is
	the key to Peaucellier's linkage for drawing a straight line.}
\item[(2)]
	The inverse of a sphere is a sphere or a plane.
	It is a plane exactly when the sphere 
	passes through the center of inversion.
	%
	% \cite[p. 210]{D49}: nice proof for plane circles.
	% Johnson also has a nice proof (the same one) for plane circles **
	% \cite[p. 26, 228]{Coolidge} % statement for plane circles (p. 26)
			    % statement for spheres and space circles (p. 228)
\Comment{
Proof for space circles:
Any circle is the intersection of a sphere and a plane.
If the circle contains the center of inversion, then 
the sphere and plane both invert to a plane, which intersect in a line.
Otherwise, the sphere and plane both invert to a sphere,
which intersect in a circle.

Proof for lines: 
$\mbox{inv}(\mbox{inv}(p)) = p$.
}
\end{description}

% Inversion is useful as a dual map in computational geometry, 
% as discussed in Preparato and Shamos.

% IN JOURNAL: \figg{}{Circles map to circles}{1in}

Several other properties of inversion are indirectly important.

\begin{description}
\item[(3)]
	$\mbox{inv}_{S}^{-1} = \mbox{inv}_{S}$, 
	i.e., $\mbox{inv}_{S}(\mbox{inv}_{S}(p)) = p$.
	% Inversion is self-inverse; `reciprocal' (is used in Johnson)
\end{description}

Combining properties (1) and (3), we have

\begin{description}
\item[(4)]
	The inverse of a line $L$ is a circle through the center
	of inversion or a line.
	It is a line exactly when $L$
	passes through the center of inversion,
	in which case the inverse of $L$ is $L$.
\end{description}

(4) is also true for planes.

\begin{description}
\item[(5)]
\label{prop:conformal}
	Inversion is conformal.
	% \cite{Davis}
	% p. 214
	That is, the angle between two intersecting curves is 
	preserved under inversion.
\end{description}

\begin{defn2}
If the point $p$ lies outside the sphere $S$,
the {\bf polar} of $p$ with respect to $S$
is the plane that intersects $S$ in the points of tangency from $p$ 
(Figure~\ref{fig2}).
\end{defn2}

\begin{description}
% FOLLOWING FACT IS NEEDED TO FIND CENTER OF SPACE CIRCLE
\item[(6)]
\label{prop:polar}
If $p$ lies outside $S$, $\mbox{inv}_{S}(p)$ 
lies on the polar of $p$ with respect to $S$ \cite{Coo71}.
% IN JOURNAL: CHECK THIS: (ONLY STATED FOR CIRCLES S)
% p. 22
Thus, $\mbox{inv}_{S}(p)$ is the intersection of $p$'s polar
and the line to the center of $S$ (Figure~\ref{fig2}).
\end{description}

\figg{fig2}{The relation between inversion and polars}{3in}

\begin{description}
\item[(7)]
Lines of curvature are preserved under inversion \cite{FISCH86}. % p. 29
\end{description}

% That is, lines of curvature are mapped onto lines of curvature

A direct consequence of (7), and (1), is that cyclides are mapped to cyclides.

\begin{description}
\item[(8)]
The inverse of a Dupin cyclide is a Dupin cyclide,
if the center of inversion does not lie on the cyclide.\footnote{Under the
	more general definition of a Dupin cyclide as a surface whose lines
	of curvature are circles or lines,
	the inverse of a Dupin cyclide is always a Dupin cyclide.}
% Reference of this fact: p. 29 of Gerd Fischer: Dupin -> Dupin 
% parabolics arise from centers of inversion on cyclide
\end{description}

In special cases, the inverse of a cyclide can be simpler than a cyclide.

\begin{description}
\item[(9a)] Any Dupin cyclide can be inverted to a torus.
\item[(9b)] Any singly-horned Dupin cyclide can be inverted to a cylinder.
\item[(9c)] Any doubly-horned Dupin cyclide can be inverted to a cone.
\item[(9d)] Any self-intersecting Dupin cyclide can be inverted to a cone.
\end{description}

These important cases will be proved and discussed in 
Section~\ref{sec:totorus}.

% *********************************************************************

% \input{newintmethod}

\section{A new method for intersection}
\label{sec:nim}

% IT IS VERY GOOD TO PRESENT THE INTERSECTION RESULTS AT THE BEGINNING,
% BEFORE CIRCLE DECOMPOSITION, CIRCLE INVERSION, TORUS INVERSION, AND OTHER
% WEAPONRY THAT IS BEING BUILT FOR THE INTERSECTION METHOD.
% THE READER WILL UNDERSTAND IF WE INDICATE THE RESULTS OF LATER SECTIONS.
% THE READER WILL BE MOTIVATED TO READ THE LATER SECTIONS.
% THE READER WHO ONLY READS THE FIRST BIT WILL UNDERSTAND THE METHOD.

In this section, we present our method for the exact intersection of any
ringed surface with a quadric or cyclide.
%
% The fact that the canonical CSG primitives (quadric, torus, plane) 
% are the union of circles or lines
% is a weakness from the point of view of representational power, but
% a strength from the point of view of algorithmic unity and integration.
% Previous algorithmic development has not taken advantage of this unity of
% structure, centering instead around the degree of a primitive.
% 
We begin with intersection with a quadric surface, then intersection with
a horned or self-intersecting cyclide, and finally intersection with a ring
cyclide.
Our goal is to reduce everything to a degree 4 computation.
This is necessary for exact computation, since equations of higher degree
do not have closed form solutions \cite{He75} 
and must be solved numerically.

Consider the intersection of a ringed surface with a quadric surface,
which illustrates the importance of circle decomposition in intersection.
Using a circle decomposition of the ringed surface 
(that is, a decomposition of the surface into circles, 
$\cup_{t \in I} \rm{circle}(t)$),
the ringed/quadric intersection can be reduced
to a circle($t$)/quadric intersection.
% 	the only important fact is that we have a circle decomposition:
% 	i.e., circle x conic could be replaced by circle x arbitrary-curve
This can be interpreted as an infinite number of circle/quadric intersections.
Treating the parameter $t$ as a constant,
we can solve symbolically for the intersection of the circle with the quadric
(since it is a degree 4 computation).
The resulting parameterization in $t$ is the intersection curve.

% EXAMPLE: need Section~\ref{sec-decomp}

Circle decomposition also immediately simplifies the intersection
of a ringed surface with a cyclide,
by reducing it to circle($t$)/cyclide intersection, 
which is a symbolic degree 8 computation.
The secret to reducing the intersection to a degree 4 computation 
is to reduce the circles to lines.
In particular, exact ringed/cyclide intersection
requires inversion as well as circle decomposition.
%
\Comment{
REDUNDANT: EXACTLY THE SAME REASON THAT CIRCLE/CYCLIDE IS DEGREE 8.
Another way of seeing that circle decomposition yields a degree 8 computation
for cyclide/cyclide intersection is the following.
To elaborate on the circle decomposition of a cyclide,
it is possible to parameterize the cyclide as a circle sweeping about a conic.
We shall discuss the circle decomposition of a cyclide further
in Sections~\ref{sec-cy} and \ref{sec-decomp}.
Note that this parameterization of a cyclide is only degree 2 in $s$ 
(since the circle has a degree 2 parameterization)
and degree 2 in $t$ (since the conic has a degree 2 parameterization),
although it is still degree 4 in total.
Using this parameterization, the intersection of the cyclide
$f(x,y,z)=0$ with the cyclide $(x(s,t),y(s,t),z(s,t))$ can be computed via
$f(x(s,t),y(s,t),z(s,t)) = 0$, which is 
an equation of degree 8 in $t$ rather than degree 16.
}
%
Recall that inversion can reduce the degree, as in the mapping
of a circle to a line, or a horned cyclide to a cylinder.
% or a torus to a parabolic cyclide (a cubic surface).
% In all cases, inversion is with respect to a point on the object.

In some cases, the intersection of a ringed surface and a cyclide quickly 
reduces to the intersection of a ringed surface and a quadric.
Recall that a horned cyclide can be inverted to a cylinder or cone,
and a self-intersecting cyclide can be inverted to a cone
(Property (9b-9d) of inversion).
Suppose that the cyclide is horned or self-intersecting,
and the inversion map $f$ maps it to a cylinder or cone.
By circle decomposition, the ringed/cyclide intersection can be reduced to
a circle(t)/cyclide intersection.
Since circles are inverted to circles, this can 
be further reduced to a circle(t)/cylinder or circle(t)/cone intersection
by the inversion $f$:
$\cup_{t \in I} [f(\mbox{circle}(t))] \cap f(\mbox{cyclide})$.
This is the intersection of a ringed surface and a quadric,
which has been solved above.
The desired result is the inversion $f(C)$ of this curve $C$.

We can now restrict our attention to the
intersection of a ringed surface with a ring cyclide, which is the most
challenging problem.
The ringed/cyclide intersection 
is first reduced to circle$_1$($t$)/cyclide intersection.
The degree of the computation is then halved from 8 to 4 through inversion.
We first reduce the (ring) cyclide to a (ring) torus $T$ via inversion
(Property (9a) of inversion).
This inversion maps the circles of the ringed surface to circles.
That is, the circle$_1$($t$)/cyclide intersection is reduced via inversion
to a circle$_2$($t$)/torus intersection.
This is an (infinite) collection of circle/torus intersections,
still a degree 8 computation.
We reduce it further to an (infinite) collection of line intersections,
by inverting each circle to a line.
However, this reduction 
requires a different center of inversion for each circle, 
since the center of inversion must lie on the circle in order to invert it
to a line.
We shall use a symbolic center of inversion: circle$_2$($t$) 
will have the center of inversion (circle$_2$($t$))($k$).
$k$ must be a constant, since we must restrict to one free parameter, $t$.
Under this symbolic inversion, the torus is inverted to a symbolic cyclide.
Thus, the collection of circle/torus intersections,
$\cup_{t \in I} [{\rm circle}_{2}(t) \cap {\rm torus}\ T]$, 
is reduced to a collection of line/cyclide intersections: 
$\cup_{t \in I} [{\rm line}(t) \cap {\rm cyclide}(t)]$,
where ${\rm line}(i) = {\rm inv}_{({\rm circle}_{2}(i))(k)}{\rm circle}_{2}(i)$
and ${\rm cyclide}(i) = {\rm inv}_{({\rm circle}_{2}(i))(k)} T$.

\begin{rmk}
The reason that we first reduce the cyclide to a torus 
is that we know how to compute the inverse of a torus with respect to any 
point.
\end{rmk}

% The reason we reduce to $\mbox{circle}(t)$/torus 
% rather than staying with $\mbox{circle}(t)$/cyclide 
% is that we know how to compute the inverse of a torus with respect to any 
% point; but we only know how to invert a general cyclide with
% respect to points in its planes of symmetry.
% There is no way of restricting the symbolic center of inversion to two 
% planes.

We want the implicit equation of cyclide($t$),
so that we can substitute the parameterization of line($t$) into the
cyclide's implicit equation, yielding a degree 4 equation in $t$,
which can be solved symbolically, yielding the points of 
intersection of line($t$)/cyclide($t$).
Unfortunately, it is difficult to determine the exact implicit equation of 
cyclide($t$) (Section~\ref{sec:inv-cyclide}).
Instead, we find the implicit equations of two cyclides: 
$(c_1(x,y,z))(t)$ and $(c_2(x,y,z))(t)$, where
\[ {\rm cyclide}(t) = \left\{ \begin{array}{ll}
	c_1(t) & {\rm if\ } ({\rm circle}_2(t))(k) 
		\mbox{ lies outside the torus $T$} \\
	c_2(t) & {\rm if\ } ({\rm circle}_2(t))(k) 
		\mbox{ lies inside the torus $T$}.
   \end{array} \right. 
\]
That is, cyclide($t$) switches from $c_1(t)$ to $c_2(t)$ or from $c_2(t)$ to 
$c_1(t)$ when the center of inversion crosses the torus.
We compute both $\mbox{line}(t) \cap c_1(t)$ and $\mbox{line}(t) \cap c_2(t)$,
and then choose the appropriate one for each $t$, as follows.
By computing the intersections
of $\{({\rm circle}_2(t)(k) | \mbox{$t \in I$}\}$, 
the locus of the center of inversion, 
with the torus (say these intersections are at $t = t_1,t_2,\ldots,t_N$),
we can identify the intervals $(t_i,t_{i+1})$ where cyclide($t$) = $c_1(t)$
and the remaining intervals $(t_j,t_{j+1})$ where cyclide($t$) = $c_2(t)$.
Thus, we can correctly compute the intersection of line($t$)/cyclide($t$).

\Comment{
\begin{rmk}
For a cyclide, since the center of circle($t$) sweeps along a conic 
(Remark~\ref{rmk:conic}), a fixed point (circle($t$))($k$), 
which is at a constant offset from the center,
also sweeps along a conic.
Thus, the intersection of $\{({\rm circle}(t)(k) | t \in I\}$ and the torus
is a numeric degree 8 computation, which is done offline.

For ringed surfaces, step (3) of the cyclide/cyclide method may become 
a more complicated computation (i.e., the solution of an equation
of higher degree).
\end{rmk}
}

\begin{rmk}
\label{rmk:noton}
The intersections ${\rm circle}_2(t_i)(k)$ of 
$\{({\rm circle}_2(t)(k) | \mbox{$t \in I$}\}$ and the torus $T$ are special,
because ${\rm cyclide}(i) = {\rm inv}_{({\rm circle}_{2}(i))(k)} T$
and the inverse of a torus with respect to a point on the torus 
is a (cubic) parabolic cyclide, rather than a quartic cyclide.
Thus, cyclide(t) is $c_1(t)$ on some intervals $(t_i,t_{i+1})$,
$c_2(t)$ on the remaining intervals $(t_j,t_{j+1})$,
and $c_3(t)$ for $t = t_i, i = 1,\ldots,N$, where $c_3(t)$ is a parabolic 
cyclide.
To avoid the complication of computing $c_3(t)$,
we ignore the values $t=t_i$ when computing the line($t$)/cyclide($t$) 
intersections, thus ignoring
a finite number of points from the intersection.
In fact, since the intersection curve is continuous,
these `missing' points are limits of their neighbours,
and are automatically filled in.
\end{rmk}

The result of line($t$)/cyclide($t$) 
is the parameterization (in $t$) of the intersection curve of the ringed
surface and the original cyclide, after a series of two inversions
(Figure~\ref{fig:box}).
Thus, we need to invert this result,
first via the symbolic center of inversion (circle$_2$($t$))($k$)
and then via the center of inversion used for 
the original cyclide-to-torus inversion.
The final result is the desired intersection curve.

To review, here is our algorithm for ringed/cyclide intersection.

\ifFull
\vspace{.2in}
\else
\clearpage
\fi

\centerline{{\bf The Circle Method}}
\begin{description}
\item[1.]
	Express ringed $\cap$ cyclide as
		$\cup_{t \in I}[{\rm circle}(t) \cap {\rm cyclide}]$.
\item[2.]
	If the cyclide is horned or self-intersecting, 
	then:
\begin{description}
\item[(a)]
	Reduce the problem to 
	$\cup_{t \in I}[f({\rm circle}(t)) \cap f({\rm cyclide})]$,
	where $f$ is the inversion that maps the cyclide to a cylinder
 	or cone.  
	Sections~\ref{sec:toquadric} and \ref{sec1} 
	deal with the computation
	of the map $f$ and the inverse of each circle, respectively.
\item[(b)]
	The intersection of the circle $f({\rm circle}(t))$ with the
	quadric $f({\rm cyclide})$ can be computed symbolically.
	If this intersection is $\alpha(t)$, 
	then the ringed/cyclide intersection curve is 
	$f(\alpha(t))$.
\item[(c)]
	Exit.
\end{description}
\item[3.]
(Note that we now know that we are working with a ring cyclide.)\\
	Reduce $\cup_{t \in I}[{\rm circle}(t) \cap {\rm cyclide}]$
		to $\cup_{t \in I}[{\rm circle}(t) \cap {\rm torus}]$,
		by inverting the cyclide to a torus (Section~\ref{sec:totorus})
		and mapping each circle under this inversion 
		(Section~\ref{sec1}).
		Let the center of inversion be $c$.
\item[4.]
	Compute the intersections of the curve
		$[(\mbox{circle}(t))(k) | t \in I]$
		with the torus.  Let these intersections be
		$t_1,t_2,\ldots,t_N$.
		% this is a conic by Remark~\ref{rmk:conic}
\item[5.]
	Identify the intervals $(t_{i_1},t_{i_{1}+1}), (t_{i_2},t_{i_{2}+1}),
		\ldots, (t_{i_m},t_{i_{m}+1})$,
		where $(\mbox{circle}(t))(k)$ lies outside the torus.
\item[6.]
	Reduce $\cup_{t \in I}[{\rm circle}(t) \cap {\rm torus}]$
		to $\cup_{t \in I}[{\rm line}(t) \cap {\rm cyclide}(t)]$,
		where 
\[ 
	{\rm cyclide}(t) = \left\{ \begin{array}{ll}
		c_1(t) & t \in \cup_{j \in 1,\ldots,m}\{(t_{i_j},t_{i_{j}+1})\}
				\\
		c_2(t) & {\rm otherwise}.
	\end{array} \right.
\]
		by inverting the circle (Section~\ref{sec1}) and 
		torus (Section~\ref{sec:inv-cyclide}) with respect to the 
		center of inversion (circle($t$))($k$).
		Both $c_1(t)$ and $c_2(t)$ are implicit equations of cyclides.

	   % MUST MAKE SURE THAT WE CAN FIND PARAMETRIC EQUATION OF LINE(T)
\item[7.]
	Viewing $t$ as a symbolic constant,
		compute $\mbox{line}(t) \cap c_1(t)$
		for its four intersections $p_{1,1}(t)$, $p_{2,1}(t)$,
		$p_{3,1}(t)$, $p_{4,1}(t)$.
		That is, substitute the parameterization of line($t$) into
		$c_1(t)$, and solve the resulting degree 4 equation 
		symbolically.
		Similarly, compute the four intersections
		$p_{1,2}(t)$, $p_{2,2}(t)$, $p_{3,2}(t)$, $p_{4,2}(t)$
		of $\mbox{line}(t) \cap c_2(t)$.
		Let
\[ 
	p_i(t) = \left\{ \begin{array}{ll}
	p_{i,1}(t) & t \in \cup_{j \in 1,\dots,m}\{(t_{i_j},t_{i_{j}+1})\} 
				\\
		p_{i,2}(t) & {\rm otherwise}.
	\end{array} \right.
\]
\item[8.]
	Invert $p_1(t)$, $p_2(t)$, $p_3(t)$, and $p_4(t)$ to 
		$q_1(t)$, $q_2(t)$, $q_3(t)$, $q_4(t)$,
		using the symbolic center of inversion $(\mbox{circle}(t))(k)$
		(and the same radius of inversion as in step (6)).
\item[9.]
	Invert $q_1(t)$, $q_2(t)$, $q_3(t)$, and $q_4(t)$ to 
		$r_1(t)$, $r_2(t)$, $r_3(t)$, $r_4(t)$,
		using the center of \mbox{inversion $c$}
		(and the same radius of inversion as in step (3)).
\item[10.]
	The intersection of the original ringed surface and cyclide is
		$\cup_{t \in I} [r_1(t),r_2(t),r_3(t),r_4(t)]$.
\end{description}

\begin{figure}[h]
\[
\begin{array}{c}
\mbox{ringed} \cap \mbox{cyclide}  \\
\downarrow \\
\mbox{circle}(t) \cap \mbox{cyclide} \\
\downarrow \\
\mbox{circle}(t) \cap \mbox{torus} \\
\downarrow \\
\mbox{line}(t) \cap \mbox{cyclide}(t)
\end{array}
\]
\caption{Ringed/cyclide intersection (last two arrows are inversion maps)}
\label{fig:box}
\end{figure}


% The complexity of this computation is as follows:
% 	START HERE


\Comment{
Q. WHAT ABOUT INSTABILITY OF DEGREE 4 CLOSED FORM?
A. IF YOU DON'T WANT TO USE DEGREE 4 SYMBOLIC COMPUTATION,
	CAN STILL USE NUMERIC METHODS, BUT OF COURSE WILL ONLY GET 
	APPROXIMATE  INTERSECTION (I.E., WILL HAVE TO CHOOSE A FINITE
	SET OF CIRCLES TO INTERSECT)
If desired, rather than computing the closed form solution for the intersection
curve, it is of course possible to get a finite number of points on the 
intersection curve by substituting a constant value for $t$ in each of the 
above methods, and solving numerically.
You can choose an arbitrary granularity for these points on the intersection.
(This is an advantage of this method over the tracing method of finding
a finite number of points on the intersection.  The tracing method must
choose a fine granularity of points, since it must take care not to lose
the curve as it traces along it.)
}

% ********************BOGUS*******************************************

\Comment{
Actually, general cyclide/cyclide intersection can be solved (exactly)
using the mapping to a torus.
In particular, the cyclide/cyclide intersection can be reduced to a 
(circle(t) parameterization)/cyclide intersection,
which can be further reduced via inversion to a 
(circle(t) parameterization)/torus intersection.\footnote{Actually,
	any parameterization of the first cyclide may be used,
	but the circle(t) parameterization is the most natural one.}
Now interpret the circle(t) parameterization as a point parameterization
point(s,t).
Note that a point lies on the intersection of a surface with a torus
if and only if it lies on the surface and at distance $d$ from the circle $C$,
where the torus is the $d$-offset of $C$.
Thus, the [point(s,t)/(torus = offset(C,d))] intersection can be reduced to a
\be
\label{g}
dist(point(s,t),C) = d
\ee
calculation.
Since there is a degree 4 formula for point-circle distance,
the solution of (\ref{g}) for $t$ is 
a degree 4 * (degree of s in point(s,t)) computation.
	POINT(S,T) IS DEGREE 2 IN BOTH S AND T, 
	SO DIST(POINT(S,*),C) = D 
	IS A DEGREE EIGHT FORMULA IN S!
	WHICH CANNOT BE SOLVED SYMBOLICALLY
The resulting parameterization in $t$ is the intersection curve.

% IS POINT-CIRCLE DISTANCE FORMULA POSSIBLY DEGREE 1 OR 2 IN POINT,
%	(ALTHOUGH IT IS DEGREE 4 TOTAL) ? NO
}

% *******************END OF BOGUS************************************

% *********************************************************************

% inverse of torus
% \input{inv-torus}

\section{The inverse of a torus}
\label{sec:inv-cyclide}

In this section, we show how to compute the inverse of a torus.
This is needed in step (6) of the intersection algorithm.
Note that we don't need to consider the inverse of an arbitrary torus.
Like cyclides, we can distinguish between ring, horned, and self-intersecting
tori, and we only need to consider the inverse of ring tori 
(i.e., tori whose exterior circles are disjoint).
This is enough because 
the torus in step (6) of the algorithm is a ring torus,
being the inverse of a ring cyclide.

The inverse of a ring torus is an easy computation for the following reason.
We know that the inverse of a cyclide is a cyclide (Property (8) of inversion).
The inverse of a cyclide is easy to compute if the center of inversion
lies on a plane of symmetry.
This plane remains a plane of symmetry and the inverse cyclide is
determined by the effect of the inversion on the two circles in this plane
of symmetry (Lemma~\ref{lem:uni}).
For this reason, the inverse of a torus is always easy to compute,
since a torus has an infinite number of planes of symmetry which cover space.
That is, the center of inversion always lies in a plane of symmetry.
The only difficulty is deciding whether the plane of symmetry becomes 
an exterior or interior plane of symmetry.
The following lemma shows how to decide.

\begin{lemma}
\label{lem:intorext}
Consider the inverse of a ring torus.
Let $P$ be the exterior plane of symmetry of the torus that contains
the center of inversion.
The exterior circles in $P$ map to interior circles if and only if the
center of inversion lies inside one of the exterior circles.
\end{lemma}
\prf
Let $C_1$ and $C_2$ be the two exterior circles in the plane $P$.
We will prove that $\mbox{inv}(C_2)$ contains $\mbox{inv}(C_1)$ 
if and only if $C_2$ contains the center of inversion.
Suppose $C_2$ contains the center of inversion $c$.
In particular, in every direction from $c$, $C_2$ is encountered before $C_1$.
After inversion, in every direction from $c$,
$\mbox{inv}(C_1)$ will be encountered before $\mbox{inv}(C_2)$,
because the closer a point is to the center of inversion, 
the further away it is mapped from the center of inversion.
Since $\mbox{inv}(C_2)$ is a circle that contains the center of inversion,
this implies that $\mbox{inv}(C_2)$ contains $\mbox{inv}(C_1)$.

Suppose $C_2$ does not contain the center of inversion.
Circles are stretched radially out from the center of inversion,
or shrunk radially in.
Thus, in order for $C_1$ to map inside $C_2$, it must lie in the sector
defined by $C_2$ as well as the annulus defined by $C_2$ 
(Figure~\ref{fig:cant}).
It is impossible to satisfy both of these criteria,
since $C_1$ and $C_2$ are disjoint.
\QED

\figg{fig:cant}{$C_1$ cannot map inside $C_2$}{3.75in}

This leads immediately to the desired result.
Note that we can assume that the center of inversion does not lie on the torus,
since this will be the case in step (6) of the intersection algorithm.
(Remark~\ref{rmk:noton}).

\ifFull
\else
\newpage
\fi

\begin{theorem}
\label{thm:invtorus}
Let $T$ be a ring torus and let
$S$ be a sphere of inversion with center $c \not \in T$.
The inverse of $T$ is the ring cyclide with 
\[ \begin{array}{ll}
   {\rm interior\ circles\ } \mbox{inv}_S(E_1), \mbox{inv}_S(E_2) &
		\rm{if\ the\ center\ of\ inversion\ lies\ inside\ } T \\
   {\rm exterior\ circles\ } \mbox{inv}_S(E_1), \mbox{inv}_S(E_2) &
			\rm{otherwise}
   \end{array}
\]
where $E_1$ and $E_2$ are the exterior circles of $T$
that lie in the same plane as the center of inversion.
\end{theorem}
\Comment{
\prf
By Lemma~\ref{lem:intorext}, 
the exterior circles of the torus are mapped to interior circles
if and only if one of them contains the center of inversion.
\QED
}

Section~\ref{sec1} discusses how to compute the inverse of a circle
(for $\mbox{inv}_S(E_1)$ and $\mbox{inv}_S(E_2)$).
and Lemma~\ref{lem:uni} shows how to recover a cyclide from its interior or
exterior circles.
In the following section, 
we consider the map from a cyclide to a torus, which
is needed in step (3) of the intersection algorithm and in 
Section~\ref{sec-decomp} on circle decomposition.

% *********************************************************************

% reduction of cyclide to torus
% \input{totorus}

\section{The mapping of a cyclide to a torus}
\label{sec:totorus}

The inverse of a torus is always a cyclide, but the inverse of a cyclide
is rarely a torus.
Nevertheless, it is always possible to find a center of inversion that maps
a cyclide to a torus.
It is clear from Figure~\ref{fig:sym} that a cyclide is a torus if and only
if $c=0$, if and only if the interior circles are concentric.
	% since the exterior and interior circles identify a cyclide
Thus, a cyclide can be mapped to a torus by
inverting its interior circles (or its exterior circles) 
to concentric interior circles.
% THUS AGAIN WE CAN DIRECTLY USE THE REP BY CIRCLES IN PLANES OF SYMMETRY
We attack this problem in the following section.

% ********************************************************************

\subsection{Mapping to concentric circles}

There is a classical method for inverting two non-intersecting circles 
into concentric circles.

\begin{defn2}
\label{defn:power}
The {\bf power} of a point $P$ with respect to a circle $C$,
$\mbox{power}_{C}(P)$, is the product of the (signed) distances from
$P$ to the closest point and the furthest point 
of the circle.\footnote{The classical
	definition of $\mbox{power}_{C}(P)$ is  
	the product of signed distances from $P$ 
	to any pair of points $U$ and $V$
	on $C$, such that $U$, $V$, and $P$ are collinear.
	This product is well defined.
	Our definition chooses the closest and furthest points of the
	circle for $U$ and $V$.}
Thus, $\mbox{power}_{C}(P) = (||P-c|| - r)(||P-c|| + r)
= ||P-c||^2 - r^2$, where $c$ and $r$ are the 
center and radius of $C$.
\end{defn2}

% ADDS NOTHING
%\figg{fig-power1}{$\mbox{power}_{C}(P)=\mbox{dist}(P,U)\mbox{dist}(P,V)$}{1in}

\begin{rmk}
\label{rmk:pow}
The power of a point can be thought of as a distance measure of the point
from a circle. 
Both the closest and furthest points of the circle are used, 
rather than just the closest point.
This encodes more information: if only the closest point is used, 
the inside of the circle cannot be distinguished from the outside.
The power of a point is positive when the point lies outside the 
circle and negative when the point lies inside.

% See Johnson, p. 30, on power of line roughly equivalent 
% to ratio to distance to line = midpoint of their common chord.

% Since distance to a sphere is a distance metric, so is power.
% (NEED TO SHOW A BIT MORE CAREFULLY.  GIVE EXAMPLE OF A INSIDE AND 
% B OUTSIDE CIRCLE FOR TRIANGLE INEQUALITY.)
\end{rmk}

\begin{defn2}
The {\bf radical axis} of two coplanar circles $C_1$ and $C_2$
is the locus of points $P$ such that $\mbox{power}_{C_1}(P) = 
\mbox{power}_{C_2}(P)$ (Figure~\ref{fig:rad}).
A {\bf coaxal system} of circles is a collection of circles, 
each pair of which have the same radical axis.
\end{defn2}

\begin{rmk}
\label{rmk:coa}
The radical axis of two circles is a line perpendicular to the line 
between their centers \cite{J29}. % Johnson, p. 31
There are three types of coaxal systems of circles:
% \cite{J29}, Johnson, p. 36
\begin{description}
\item[(1)]
	a collection of concentric circles;
\item[(2)]
	a collection of circles through one or two fixed points;
\item[(3)]
	a collection of non-intersecting circles, all of which are orthogonal
	to a fixed circle, and all of whose centers are collinear.
\end{description}
We are interested in the latter type.
\end{rmk}

\figg{fig:rad}{A coaxal system with radical axis $A$, all orthogonal to the 
		circle $C$ [20]}{1.75in}
% Figure 10 from Johnson, p. 35

\begin{defn2}
Consider a coaxal system of type~(3),
all of whose circles are orthogonal to the circle $C$ and
all of whose centers lie on the line $L$.
The {\bf limiting points} of this coaxal system are the two points of 
intersection of $C$ and $L$.
\end{defn2}

The points $K$ and $K'$ are the limiting points of the coaxal system
in Figure~\ref{fig:rad}.

\begin{lemma}
\cite{J29} % p. 55
\label{lem:cntr}
Two non-intersecting circles are inverted to concentric circles
if a limiting point of their coaxal system is chosen as
the center of inversion.
\end{lemma}

The following lemma shows how to find this limiting point.

% FIGURE: (a) circles not inside one another, (b) one circle contains other.

\begin{lemma}
\cite{J29} % p. 35
\label{lem:sqrtp}
Let $C_1$ and $C_2$ be two non-intersecting circles, with centers $c_1$
and $c_2$.
Let $P$ be the intersection of their radical axis and $\lyne{c_1c_2}$.
The limiting points of the coaxal system of $C_1$ and $C_2$
lie on $\lyne{c_1c_2}$, at distance $\sqrt{\mbox{power$_{C_1}(P)$}} = 
\sqrt{\mbox{power$_{C_2}(P)$}}$ from $P$.
\end{lemma}

% to compute limiting point: (1) compute $P$, which is the unique point on 
% 		the line $\lyne{c_1c_2}$ with equal power to both centers.
%			     (2) compute the power of $P$
% 			     (3) compute the limiting point as the points
%			         	along the line of centers at this
%					distance from $P$.

It is simple to compute $P$, the unique point on the line between the centers
such that \mbox{power$_{C_1}(P) =$ power$_{C_2}(P)$}, 
and thus it is simple to compute the limiting point.

\subsection{Reduction of cyclide to torus}

We can now invert a cyclide to a torus using 
inversion with respect to a limiting point of the interior or exterior circles.

\begin{theorem}
\label{cor:coi}
Let $C$ be a cyclide.
Let $C_1(c_1,r_1,o)$ and $C_2(c_2,r_2,o)$, $r_1 < r_2$, be 
\[
\begin{array}{ll}
\mbox{the interior circles of $C$} & \mbox{\ \ \ \ if $C$ is a ring or 
			self-intersecting cyclide} \\
\mbox{the exterior circles of $C$} & \mbox{\ \ \ \ if $C$ is a horned cyclide}
\end{array} 
\]
The cyclide $C$ is mapped to a torus under an inversion 
whose center is the limiting point of $C_1$ and $C_2$.
Moreover, the interior circles of the torus are the inverses of $C_1$ 
and $C_2$.
\end{theorem}
\prf
We wish to map two non-intersecting circles in a plane of symmetry of $C$
to two concentric circles in a plane of symmetry.
This will successfully map $C$ to a torus, since the only cyclide with 
concentric interior circles is the torus.
By choosing the center of inversion to lie in a plane of symmetry of $C$,
we guarantee that this plane remains a plane of symmetry.
The non-intersecting condition enforces the use of interior circles
for the self-intersecting cyclide, and exterior circles for the horned
cyclide (Remark~\ref{rmk:rel}).
Note that we could have used either interior or exterior circles for the ring
cyclide.
\QED

% Agrees with Depont's special case on p. 37, when radii and centers of
% interior circles for Forsyth's normal form are plugged in.

\begin{rmk}
Note that the radius of inversion is arbitrary;
and that the torus is not necessarily a ring torus (e.g., it may be 
self-intersecting).
\end{rmk}

\begin{rmk}
The inversion of a singly-horned cyclide to a torus deserves additional 
comment, since it is a degenerate case.
Let the exterior circles of the singly-horned cyclide be $C_1$ and $C_2$.
One of these exterior circles is a point, a degenerate circle (say $C_2$).
We can still talk of the coaxal system of $C_1$ and $C_2$,
but $C_2$ is now a degenerate circle of the coaxal system.
In particular, $C_2$ is a limiting point of the coaxal system.
(This is to be expected, since given a circle of a coaxal system,
the circle always contains one of the limiting points; thus, when 
the circle degenerates to a point, it degenerates to the limiting point.)
This means that we have to take care with the choice 
of the center of inversion: if the two limiting points of 
the coaxal system of $C_1$ and $C_2$ are $C_2$ and 
$P$, we should choose $P$ as the center of inversion.
Otherwise, the process of inverting the cyclide to a torus is identical.
% following sentence added to technical report
The singly-horned cyclide will map to a `horned' torus, for which one
interior circle is a point.
\end{rmk}

Since a torus is fully defined by its interior circles, and Section~\ref{sec1}
will show how to compute the inverse of a circle,
Theorem~\ref{cor:coi} fully defines the mapping of a cyclide to a torus.

% Note that we do not find the torus by finding the inversion of the entire
% cyclide surface with respect to a center of inversion.
% Instead, we find key components of the image surface: interior circles of 
% torus, or vertex and a circle of cone.

% ************************************************************

\section{The mapping of a cyclide to a quadric}
\label{sec:toquadric}

The previous section showed that every cyclide can be simplified 
to a torus, through inversion.
In this section, we show that certain cyclides (the horned cyclide 
and the self-intersecting cyclide) can also be mapped to a cylinder or a cone
by inversion.
This is a very useful mapping, since the degree of the surface drops.
% It is also very easy to establish.

\subsection{Reduction of horned cyclides to cylinders and cones}
\label{sec:tocyl}

\begin{lemma}
% Fischer, p. 29, states first property; DePont in his thesis uses both facts
% but does not prove
\label{lem:horn}
A singly-horned cyclide is mapped to a circular cylinder 
if its horn is used as the center of inversion.
A doubly-horned cyclide is mapped to a circular cone
if one of its horns is used as the center of inversion.
\end{lemma}
\prf
Although this result is well-known, 
we provide the proof since it is an important result.
All circles of a horned cyclide pass through the horns.
Thus, if one of the horns is used as the center of inversion,
the circles are all mapped to lines and 
the inverse of the horned cyclide is a ruled surface.
This inverse must also be a general
Dupin cyclide (i.e., a surface whose lines of curvature are circles 
{\em or lines}).\footnote{See the footnote to property (8) of inversion.}
But the only general Dupin cyclides that are ruled are the cylinder and cone.

All of the circles of a doubly-horned cyclide pass through the second horn,
so this horn maps to a cone vertex and the cyclide maps to a cone.
The circles of a singly-horned cyclide only share the one horn,
so this cyclide maps to a cylinder.
\QED

\begin{rmk}
\label{rmk:interior}
The interior circles are fundamental in the reduction of a horned cyclide
to a cone/cylinder.
The center of inversion is a horn, which is an intersection of the interior
circles.
Moreover, since the center of inversion lies in an (interior) plane of
symmetry, the interior circles map to lines in a plane of symmetry;
and these lines clearly define the cone/cylinder.
For the doubly-horned cyclide, their intersection defines the cone vertex
and their angle defines the cone angle;
for the singly-horned cyclide, they are parallel and their bisector
is the cylinder's axis.
\end{rmk}

\Comment{
I DON'T WANT TO ATTACK DE PONT, SO DON'T MENTION THIS.
de Pont sketches a different method for mapping cyclides to tori,
(stressing the fact that the image circles must be equal
radius) It is not as general (is based upon Forsyth's  normal form (\ref{eqcy})
and it is only sketched: i.e., there are magic formula.
(Small point: he doesn't show how to find horns.)
Also note that dePont mistakenly says horned cyclides cannot be mapped to tori 
\cite[p. 37]{}.
}

\subsection{Reduction of self-intersecting cyclides to cones}
\label{sec:tocone}

By a similar argument, the self-intersecting cyclide can also 
be mapped to a cone.
In Lemma~\ref{lem:horn}, we mapped a cyclide to a cone/cylinder
by inverting with respect to an intersection of the interior circles.
For the self-intersecting cyclide, we use the exterior circles.

\begin{lemma}
A self-intersecting cyclide is mapped to a cone by an inversion centered
at one of the intersections of its exterior circles.
\end{lemma}
\Heading{Proof:}
Let $C$ be the self-intersecting cyclide, 
and let $i$ and $j$ be the two intersections of its exterior circles.
Let $i$ be the center of inversion.
It is not true that all the circles (lines of curvature)
on the cyclide pass through $i$,
but it is still true that the circles
(lines of curvature) of the cyclide that pass through $i$ cover the 
cyclide.\footnote{To see this, one can invert $C$
	to a self-intersecting torus, for which the statement is clear
	because of the symmetry of the torus.  
	Inversion preserves the property.}
This is enough to guarantee that $\mbox{inv}_{S}(C)$ is a ruled surface.
Moreover, by the same argument, 
the lines on $\mbox{inv}_{S}(C)$ all pass through 
$\mbox{inv}_{S}(j)$.
We conclude that $\mbox{inv}_{S}(C)$, which must be a general Dupin cyclide,
is a circular cone.
\QED

\begin{rmk}
Just as with the doubly-horned cyclide (Remark~\ref{rmk:interior}),
the vertex of the cone and the cone angle are defined by the image
of the two exterior circles.
\end{rmk}

% *********************************************************************

% inverse of arbitrary circle
% \input{inv-circle}

\section{The inverse of any circle}
\label{sec1}

The inverse of a circle is usually a circle, and this
inverse circle is well understood in two dimensions: i.e., 
when the plane of the circle contains the center of inversion and
the inversion is thus restricted to a plane.
However, the literature is silent on the general case of the
inverse of a circle in three dimensions.
In this section, we consider this inverse of a {\em space} circle in detail.

\begin{defn2}
A {\bf plane circle} is a circle whose plane contains the center of inversion. 
A {\bf space circle} is a circle whose plane does not contain the center of
inversion.
\end{defn2}

For the mapping of a cyclide to a torus,
only the inverse of a plane circle is needed, since
one can work completely in a plane of symmetry of the cyclide 
(Section~\ref{sec:totorus}).
However, the inverse of a space circle is needed in many other places of the
intersection algorithm, such as the reduction of 
circle(t)/cyclide intersection to circle(t)/torus intersection
(Step (3) of the algorithm in Section~\ref{sec:nim})
and our circle decomposition of a cyclide (Section~\ref{sec-decomp}).

A circle is fully defined by its center, its radius, and its
orientation in space.
In the following sections, we examine how inversion affects these 
three components.

\subsection{Radius}

\begin{theorem}
\label{thm:rad}
Let $S$ be the sphere of inversion with center $c$ and radius $R$.
\begin{enumerate}
\item
% \label{lem-radius1}
Let $D$ be a plane circle or a sphere, of radius $r$, $c \not \in D$.
The radius of $\mbox{inv}_{S}(D)$ is 
\be
\label{rad1} 
	\frac{R^{2}}{|\mbox{power}_{D}(c)|} r
\ee
% or $\frac{R^{2}}{|a^{2} + b^{2} + c^{2} - r^{2}|} r$, center $(a,b,c)$
\item
% \label{lem:radp}
Let $D$ be a line or plane, $c \not \in D$.
The radius of $\mbox{inv}_{S}(D)$ is 
\be
\label{rad2}
	\frac{R^{2}}{2\mbox{dist}(c,D)}
\ee
\item
% \label{thm:rad}
Let $D$ be a space circle of radius $r$ and center $d$,
\Comment{
lying in the plane $p$,
with equation $f(x,y,z)=0$ and normal $N$.
% $p$ with equation $f(X) = N \cdot X + \alpha = 0$.
}
lying in the plane $p$.
The radius of $\mbox{inv}_{S}(D)$ is 
\be
\label{eq:spacerad}
\begin{array}{ll}
 	{\rm radius\ of\ inv}_{S}(p) &	{\rm if\ } c\in E \\[.1in]
 	\frac{\rm{radius\ of\ inv}_{S}(E)\ *\ \rm{radius\ of\ inv}_{S}(p)}
        {\rm{dist(center\ of\ inv}_{S}(E)\ ,\ \rm{center\ of\ inv}_{S}(p))}
						&	{\rm otherwise}
\end{array}
\ee
%
\Comment{
or, equivalently,
\be
\label{eq:radsc}
\begin{array}{ll}
	\frac{R^{2}}{2\mbox{dist}(c,p)}		&	{\rm if\ } c\in E \\ 
	\frac{R^{2}r}{2f(c)(d-c) + \mbox{power}_E(c) N}
						&	{\rm otherwise}
\end{array}
\ee
}
%
where $E$ is the sphere with the same center and radius as $D$.
\end{enumerate}
\end{theorem}
\Heading{Proof:}
\begin{enumerate}
\item
% ****************************
% PLANE CIRCLE
% ****************************
The result for plane circles is classical \cite{Coo71,J29}.
% Coolidge: p. 30, 231 (sphere), no proof
% plane circle: Johnson, p. 51, proof included and relationship to power cited
% also Davis, p. 217
\Comment{
PROOF OF PLANE CIRCLE ON p. 51 OF JOHNSON.
NO PROOF IN COOLIDGE, BUT STATEMENT FOR SPHERE.
DO WE NEED A PROOF FOR SPHERE?  NAH.

\Heading{Proof:}
Consider the line $L$ connecting the center of inversion and the center of the
circle.
It intersects the circle in two points, $A$ and $B$, which form a diameter
of the circle.
The inverses of $A$ and $B$ are also a diameter of the inverse circle,
since the center of the inverse circle also lies on $L$ (Figure~\ref{fig-bar}).
Therefore, we want to compute the distance between $\mbox{inv}(A)$ and
$\mbox{inv}(B)$.
$A$ is a distance of $r$ from the center of the circle:
\[ A = (a,b,c) - r \frac{(a,b,c)}{\|(a,b,c)\|} 
     = (1 - \frac{r}{\sqrt{a^{2}+b^{2}+c^{2}}})*(a,b,c) \]
\[ B = (1 + \frac{r}{\sqrt{a^{2}+b^{2}+c^{2}}})*(a,b,c) \]
Applying Lemma~\ref{lem-inv} to get the inverses of $A$ and $B$,
and computing half of the distance between these two inverses,
we find that the radius of the inverse circle is
$\frac{R^{2}r}{|a^{2}+b^{2}+c^{2}-r^{2}|}$.
\QED
\figg{fig-bar}{}{1in}
}
\item
% ****************************
% LINE
% ****************************
Suppose $D$ is a line (or plane), $c \not \in D$.
Let $P$ be the closest point of $D$ to $c$.
Since the inverse of $D$ is a circle (or sphere) through $c$
and $\lyne{cP}$ is an axis of symmetry (Figure~\ref{fig:plrad}),
the diameter of $\mbox{inv}_S(P)$ is 
$\mbox{dist}(c,\mbox{inv}_S(P)) = \frac{R^{2}}{\mbox{dist}(c,P)}$.
\Comment{
% a more detailed proof
For simplicity, we restrict to a plane through $\lyne{cP}$.
Let $L$ be the line on $D$ in this plane (Figure~\ref{fig:plrad}).
The inverse of the plane $D$ is a sphere through $c$,
and the inverse of $L$ is a circle through $c$ (a great circle of the
sphere).
$L$ and $\mbox{inv}(L)$ are symmetric about $\lyne{cP}$.
Thus, the diameter of $\mbox{inv}(L)$, and the diameter of $\mbox{inv}(p)$,
is $\mbox{dist}(c,\mbox{inv}(P))$.
}

\item
% ****************************
% SPACE CIRCLE
% ****************************
Let $D$ be a space circle, the intersection of the sphere $E$ 
and the plane $p$.
$\mbox{inv}_{S}(D) = \mbox{inv}_{S}(E) \cap \mbox{inv}_{S}(p)$.
$\mbox{inv}_{S}(E)$ and $\mbox{inv}_{S}(p)$ are orthogonal,
since $E$ and $p$ are
% ($p$ goes through the center of $E$)
(property (5) of Section~\ref{sec:inv}).
% \ref{prop:conformal}
$\mbox{inv}_{S}(p)$ is a sphere (i.e., $c \not \in p$, because $D$ 
is a space circle).
There are two cases based on whether $\mbox{inv}_{S}(E)$ is a sphere or
a plane.
Suppose $c \in E$.
Then $\mbox{inv}_{S}(E)$ is a plane, which must pass through the center
of $\mbox{inv}_{S}(p)$ since 
$\mbox{inv}_{S}(E)$ and $\mbox{inv}_{S}(p)$ are orthogonal.
Thus, the radius of $\mbox{inv}_{S}(D)$ is identical to the radius of  
$\mbox{inv}_{S}(p)$.

Suppose $c \not \in E$.
Then $\mbox{inv}_{S}(E)$ is a sphere.
Let the center of $\mbox{inv}_{S}(E)$ be $C_{1}$, and
the center of $\mbox{inv}_{S}(p)$ be $C_{2}$.
Any plane containing the centers of the two spheres 
intersects the two spheres in two circles (Figure~\ref{fig:space-center}),
which intersect in $X$ and $Y$.
The center of $\mbox{inv}_{S}(D)$ is the midpoint of $\seg{XY}$,
i.e., $\seg{XY} \cap \seg{C_{1}C_{2}}$.
(The two spheres are generated by rotating the two circles about 
the axis $\seg{C_{1}C_{2}}$; 
thus, $\mbox{inv}_{S}(D)$ is generated by rotating $X$ and $Y$ about
$\seg{C_{1}C_{2}}$.)
% Thus, the center of $\mbox{inv}_{S}(D)$ must lie on $\seg{C_{1}C_{2}}$.
Let the center of $\mbox{inv}_{S}(D)$ be $C_{3}$,
so that
the desired radius of $\mbox{inv}_{S}(D)$ is $\mbox{dist}(C_{3},X)$.
Since $\mbox{inv}_{S}(E)$ and $\mbox{inv}_{S}(p)$ are orthogonal,
$\triangle X C_1 C_2$ is a right triangle.
Its area is $\frac{1}{2} C_1 X * C_2 X$.
Since $\angle C_1 C_3 X$ is also a right angle, $C_3 X$ is the altitude
of $\triangle X C_1 C_2$, whose area is therefore $\frac{1}{2} C_1 C_2 *C_3 X$.
Equating these two expressions for the area, we have
$C_3 X = \frac{C_1 X * C_2 X}{C_1 C_2}$.
\Comment{
The formula (\ref{eq:radsc}) 
is derived from 1., 2., and Theorem~\ref{thm:center} below.
}
%
%
\Comment{
CLUMSIER PROOF 
Since $\angle C_{1}C_{3}X$ is a right angle,
$C_{3}X = \sin \theta * C_{1}X$.
Also recall that the two circles intersect orthogonally, 
so $\angle C_{1}XC_{2}$ is a right angle.
Thus, $\cos \theta = \frac{C_{1}X}{C_{1}C_{2}}$ and
$(C_{1}C_{2})^{2} = (C_{1}X)^{2} + (C_{2}X)^{2}$.
We can use these to solve for $\sin \theta$: 
$\sin \theta 	= \sqrt{1 - \cos^{2} \theta}
		= \sqrt{1 - (\frac{C_{1}X}{C_{1}C_{2}})^{2}}
		= \sqrt{\frac{(C_{1}C_{2})^{2} - 
	(C_{1}X)^{2}}{(C_{1}C_{2})^{2}}}
		= \sqrt{\frac{(C_{2}X)^{2}}{(C_{1}C_{2})^{2}}}
		= \frac{C_{2}X}{C_{1}C_{2}}$.
Thus, $C_{3}X = \frac{C_{2}X * C_{1}X}{C_{1}C_{2}}$.
% Finally, $C_{1}X$ is the radius of $\mbox{inv}_{S}(E)$ and
% $C_{2}X$ is the radius of $\mbox{inv}_{S}(p)$.
}
\end{enumerate}
\QED

\figg{fig:plrad}{The inverse of a line}{2.5in}

\figg{fig:space-center}{$C_3$ is the center of the space circle}{2.1in}

\begin{rmk}
The formulas (\ref{rad1}) and (\ref{rad2}) both 
involve the factor $\frac{R^2}{d}$, 
where $d$ is the
distance of the object from the center of inversion (Remark~\ref{rmk:pow}).
This factor is also important in the mapping of a point under inversion
(Lemma~\ref{lem-inv}).
It encodes the amount of stretching of a point or plane circle 
under inversion.
%
% the power of a point is a stretching map:
% d^2 - k^2: radius k stays constant, while d grows linearly
% so d^2 - k^2 grows almost quadratically, which is a form of stretching.
\end{rmk}

% ***************************************************************

\subsection{Center}

The amount of stretching that a point undergoes during inversion is nonlinear.
The relative stretching (or shrinking) 
of a point far away from the circle of inversion 
is much more than that of a point close to the circle of inversion.
For this reason, the center of a circle does not map to the center of the new
circle under inversion.
However, it turns out that the new center 
can be found with two or three applications of the inversion map.

\begin{theorem}
\label{thm:center}
Let $S$ be the sphere of inversion with center $c$ and radius $R$.
\begin{enumerate}
\item
% \label{lem-twoappl1}
Let $D$ be a plane circle or a sphere, of center $d$, $c \not \in D$.
The center of $\mbox{inv}_{S}(D)$ is 
\be
\label{cen1}
	 \mbox{inv}_{S} ( \mbox{inv}_{D} (c))
% or THE FOLLOWING RESULT IS NEW  (ESPECIALLY RELATIONSHIP TO POWER)
% \label{lem-center}
\Comment{
	I'M SURE THIS IS WRONG (I got 
$c + \frac{R^{2}\mbox{power}_{D}(c)}{||\mbox{power}_{D}(c)||^{2}} (d-c)
	when I last checked it)

	= c + \frac{R^{2}}{\mbox{power}_{D}(c)} (d-c)
}
\ee
\item
% \label{lem:cenplane}
Let $D$ be a line or plane, with equation $f(x,y,z)=0$ and normal $N$,
$c \not \in D$.
The center of $\mbox{inv}_{S}(D)$ is 
\be
\label{cen2}
	\mbox{inv}_{S}(\mbox{reflection}_{D}(c))
% or
% \label{cor-plane}
	% the plane $Ax+By+Cz+\alpha=0$ $(\alpha \neq 0)$ 
	= c - \frac{R^{2}}{2f(c)} N
\ee
\item
% \label{thm:center}
Let $D$ be a space circle with center $d$,
% lying in the plane $p$ with equation $f(x,y,z)=0$ and normal $N$.
lying in the plane $p$.
The center of $\mbox{inv}_{S}(D)$ is
\be
\label{eq:spacecen}
\begin{array}{ll}
\rm{center\ of\ inv}_{S}($p$)			&	{\rm if\ } c \in E \\
%\mbox{inv}_{S}(\mbox{reflection}_{p}(c))	&	{\mbox if} c \in E \\
\mbox{inv}_{\mbox{inv}_{S}(p)} (\rm {center\ of\ inv}_{S}($E$)) 
						&	{\rm otherwise}
\end{array}
\ee
\Comment{
or, equivalently,
% \label{cor-cen}
\be
\begin{array}{ll}
	c - \frac{R^{2}}{2f(c)} N		&	{\rm if\ } c\in E \\ 
	(c - \frac{R^{2}}{2f(c)} N) + 
	\frac{R^{2}\rm{power}_E(c)}{2f(c)}
	\frac{2f(c)(d-c) + \mbox{power}_E(c) N}
	   {\|2f(c)(d-c) + \mbox{power}_E(c) N\|^2}
						&	{\rm otherwise}
\end{array}
\ee
}
where $E$ is the sphere with the same center and radius as $D$.
\end{enumerate}
\end{theorem}
\Heading{Proof:}
\begin{enumerate}
\item
% **************
% PLANE CIRCLE
% **************
This is a classical result \cite{Coo71,D49,J29}.
% Coolidge, p. 231 for sphere
% also Johnson, p. 56 for plane circle
% \cite[p. 216]{D49} is a good proof of the plane circle case.
% The formula can be found through the application of Lemma~\ref{lem-inv}.
%
\Comment{
% SINCE COOLIDGE HAS NO PROOF FOR SPHERE, HERE IS THE PROOF FOR SPHERE, 
%
\Heading{Proof:}
We provide a proof for spheres, since we have only seen a statement
without proof in the literature \cite[p. 231]{Coo71}.
NOTE: IN THE FOLLOWING PROOF THE SPHERE OF INVERSION IS CALLED C
AND THE SPHERE THAT IS BEING INVERTED IS CALLED D (FOR HISTORICAL REASONS)
Let $D$ be a sphere.
Consider any plane $p$ through the centers of both spheres $C$ and $D$.
This plane divides both spheres in half.
That is, it is a plane of symmetry of $D$.
It is clear that points on one side of the plane
stay on that side of the plane under inversion.
Therefore the plane remains a plane of symmetry for the inverse of $D$.
% 
\Comment{
(This can be seen by considering the image of diametral pairs.
If $p$ isn't a plane of symmetry, 
then, on the inverse sphere, there exist two points that define a diameter
(i.e., a diametral pair of points)
that lie strictly on one side of the plane.
But a diametral pair on inv($D$) is the inverse of a diametral pair
on $D$ (because they are extremal ELABORATE), and all diametral pairs 
of the original sphere either lie on opposite
sides of the plane or they both lie on the plane.
This leads to a contradiction.)
}
%
This means that the center of $\mbox{inv}(D)$ is the center of 
$\mbox{inv}(D) \cap p$.
Our problem now reduces to a two-dimensional problem on the plane $p$,
using the two circles $D \cap p$ and $\mbox{inv}_{C}(D) \cap p$.
\QED
}

\item
% ****************
% LINE/PLANE
% ****************
It is classical that the center of $\mbox{inv}_S(D)$ is 
$\mbox{inv}_{S}(\mbox{reflection}_{D}(c))$ \cite{Coo71}.
To get the formula, note that the reflection of $c$ in the plane $D$ is 
$c - 2 \frac{f(c)}{\|N\|} \frac{N}{\|N\|}$,
since $c$ lies at distance $\frac{|f(c)|}{\|N\|}$ from $D$ in
the direction $(\mbox{sign}(f(c))) N$ \cite{SS14}.
% p. 17, Snyder and Sisam
% i.e., $c$ lies in the halfspace pointed to by $sign(f(c)) N$ 
An application of Lemma~\ref{lem-inv} yields the result.
%
% [p. 30, 231]
% IN JOURNAL (POSSIBLY???): PROOF SINCE NO PROOF IN COOLIDGE
%
% IN THE FOLLOWING, THE SPHERE OF INVERSION IS CALLED C.
% The result for lines in 2D (i.e., line in same plane as circle $C$)
% is stated in Coolidge \cite[p. 30]{Coo71}.
% BUT NO PROOF IN COOLIDGE
% Same proof idea as that used in the proof of 
% Lemma~\ref{lem-twoappl} in Davis \cite{D49}.
% Let $A$ be the center of the inverse of $D$.
% We shall prove the equivalent result:
% $c = \mbox{reflection}_{D}(\mbox{inv}_{C}(A))$.
% VERY CLOSE TO A PROOF, BUT CAN'T SEEM TO PROVE THAT CENTER OF inv(d) lies
% on the line cA.

\item
% ****************
% SPACE CIRCLE
% ****************
Let $D$ be a space circle.
$\mbox{inv}_{S}(D) = \mbox{inv}_{S}(E) \cap \mbox{inv}_{S}(p)$.
Recall from the proof of Theorem~\ref{thm:rad} that,
if $c \in E$, $\mbox{inv}_{S}(E)$ is a plane that passes 
through the center of the sphere $\mbox{inv}_{S}(p)$,
so the center of $\mbox{inv}_{S}(D)$ is identical to the center of  
$\mbox{inv}_{S}(p)$.
Suppose $c \not \in E$.
$\mbox{inv}_{S}(D)$ is the intersection of the two spheres $\mbox{inv}_{S}(E)$
and $\mbox{inv}_{S}(p)$,
with centers $C_1$ and $C_2$.
Any plane containing the centers of the two spheres 
intersects the two spheres in two circles (Figure~\ref{fig:space-center}),
which intersect in $X$ and $Y$.
As we saw in Theorem~\ref{thm:rad}, the center of $\mbox{inv}_S(D)$ is 
$\seg{XY} \cap \seg{C_{1}C_{2}}$.
We can find this point using inversion, as follows.
$\mbox{inv}_{S}(E)$ and $\mbox{inv}_{S}(p)$ are orthogonal,
since $E$ and $p$ are.
That is, $C_1$'s and $C_2$'s tangents are normal at $X$ and $Y$.
Thus, $C_2$'s tangents at $X$ and $Y$ pass through $C_{1}$.
That is, $\lyne{XY}$ is the polar of $C_{1}$ with respect to the circle
centered at $C_{2}$.
By Property~(6) of Section~\ref{sec:inv},
$\lyne{XY} \cap \seg{C_{1}C_{2}}$ is
the inverse of $C_{1}$ in the circle centered at $C_{2}$.
That is, the center of $\mbox{inv}_{S}(D)$ is the inverse of $C_1$ in
$\mbox{inv}_{S}(p)$.
Note that by symmetry, the center of $\mbox{inv}_{S}(D)$ is also the inverse 
of $C_2$ in
$\mbox{inv}_{S}(E)$.
% The formula is determined by an application of Lemma~\ref{lem-inv}.
\end{enumerate}
\QED

\Comment{
The formulas (\ref{cen1}) and (\ref{cen2}) 
again involve the factor $\frac{R^2}{d}$, 
where $d$ is the distance from the center of inversion to 
the object.\footnote{The
	distance from the center of inversion to the line is actually 
	$\frac{|f(c)|}{\|N\|}$, but we can
	assume that $N$ is a unit normal.}
Recall that we saw this factor with the mapping of the radius by inversion.
}

It is straightforward to reduce the space circle formulas 
(\ref{eq:spacerad}) and 
(\ref{eq:spacecen}) to algebraic formulas, by applying the inversion map 
of Lemma~\ref{lem-inv} and the results for spheres and planes.

% Also notice that, in Corollary~\ref{cor-plane}, 
% the plane's normal is the analogue of the circle's center.

% *********************************************************

% COLLECTION OF INVERSION MAPS REPRESENTATION IS JUST AS GOOD.
% Now that we understand the effect of inversion on the center, 
% we can compute the new center directly, rather than in terms of
% a collection of inversion maps.

\subsection{Orientation}

\begin{defn2}
The {\bf orientation} of a plane is its normal vector.
The orientation of a circle is the orientation of the plane that contains it.
\end{defn2}

\begin{theorem}
\label{thm:orient}
Let $S$ be the sphere of inversion with center $c$ and radius $R$.
\begin{enumerate}

\item
The orientation of a plane circle does not change under inversion.
% which explains why the inversion literature does not contain any discussion 
% of orientation

\item
Let $D$ be a sphere with center $d$, such that $c \in D$.
The orientation of $\mbox{inv}_{S}(D)$ is $d-c$.

\item
Let $D$ be a space circle with center $d$, 
% lying in the plane $p$ with equation $f(x,y,z)=0$ and normal $N$.
lying in the plane $p$.
The orientation of $\mbox{inv}_{S}(D)$ is 
\[
\begin{array}{ll}
\rm{the\ orientation\ of\ } \mbox{inv}_{S}($E$)	&	{\rm if\ } c \in E \\
\rm{the\ vector\ from\ the\ center\ of\ } \mbox{inv}_{S}($p$)
\rm{\ to\ the\ center\ of\ } \mbox{inv}_{S}(E)
						&	{\rm otherwise}
\end{array}
\]
% or from the center of $\mbox{inv}_{S}(D)$ to the center of 
% $\mbox{inv}_{S}(p)$.
%
\Comment{
or, equivalently, 
\[
\begin{array}{ll}
	d-c					&	{\rm if\ } c\in E \\ 
	R^2\ \frac{2f(c)(d-c) + \mbox{power}_E(c) N}
	         {2f(c) \mbox{power}_E(c)}
						&	{\rm otherwise}
\end{array}
\]
}
where $E$ is the sphere with the same center and radius as $D$.
\end{enumerate}
\end{theorem}
\ifFull
\else
\newpage
\fi
\Heading{Proof:}
\begin{description}
\item[2.]
See Figure~\ref{fig:plrad}. % and the proof of Theorem~\ref{thm:rad}.
\item[3.]
If $c \in E$, $\mbox{inv}_{S}(E)$ is a plane that contains $\mbox{inv}_{S}(D)$.
If $c \not \in E$, 
recall that $\mbox{inv}_{S}(D)$ is generated by rotating $X$ about
the axis $C_{1}C_{2}$, in Figure~\ref{fig:space-center}.
Thus, $C_{1}C_{2}$ is a normal to the plane that contains $\mbox{inv}_{S}(D)$.
\end{description}
\QED

The inverse of an arbitrary circle in 3-space is now fully understood.

% *********************************************************************

% circle decomposition via inversion of space circle
% \input{circle-decomp}

\section{Circle decomposition of a cyclide}
\label{sec-decomp}

In this section, we consider the circle decomposition of a special ringed
surface, the cyclide.
For other ringed surfaces, we assume that a circle decomposition is given
directly, but this is not necessary for the cyclide.
There are two ways to see that a
cyclide is no more than a collection of circles.
Its lines of curvature, which cover the surface, are all circles.
Alternatively, it is the inverse of a torus, 
which is clearly a collection of circles,
and circles are preserved under inversion.
This observation leads 
immediately to a simple method for the circle decomposition of a cyclide.
The circle decomposition of a torus is easy, 
and we know how a circle maps under inversion.

\begin{defn2}
$C(t) = [(\mbox{c}(t),r(t),o(t)]$, $t \in I \subset \Re$, is
a {\bf circle decomposition} of a surface S if 
\begin{itemize}
\item
	$C(t)$ is the circle with 
	center $\mbox{c}(t)$, radius $r(t)$, and orientation $o(t)$,
\item
	$S = \cup_{t \in I} C(t)$, and 
\item	
	$\mbox{c}(t)$, $r(t)$, and $o(t)$ all change smoothly with $t$.
\end{itemize}
\end{defn2}

\begin{lemma}
A ring torus with interior circles $C_1(c,r_1,o)$ and 
$C_2(c,r_2,o)$, $r_1 > r_2$, has circle decomposition 
$[\mbox{center}(t) = mid(t),
 \mbox{radius}(t) = \frac{r_1 - r_2}{2},
 \mbox{orientation}(t) = o \times (c - mid(t))]$,
where $mid(t)$ is a parameterization of the circle with center $c$,
radius $\frac{r_1+r_2}{2}$, and orientation~$o$.

A self-intersecting or horned torus with interior circles $C_1(c,r_1,o)$ and 
$C_2(c,r_2,o)$, $r_1 > r_2$, has circle decomposition 
$[\mbox{center}(t) = mid(t),
 \mbox{radius}(t) = \frac{r_1 + r_2}{2},
 \mbox{orientation}(t) = o~\times~(c~-~mid(t))]$,
where $mid(t)$ is a parameterization of the circle with center $c$,
radius $\frac{r_1-r_2}{2}$, and orientation~$o$.
\end{lemma}
% simple proof

\begin{lemma}
Let $C$ be a cyclide and let the torus $T = \mbox{inv}_S(C)$ be its inverse,
where inversion is done with respect
to the center of inversion in Corollary~\ref{cor:coi}.
If $c(t)$ is a circle decomposition of $T$,
then $[\mbox{center of inv}_S(c(t)),\mbox{radius of inv}_S(c(t)),
\mbox{orientation of inv}_S(c(t))]$ is a circle decomposition of $C$.
The center, radius, and orientation of the circles of the latter decomposition
can be determined using the results of Section~\ref{sec1}.
\end{lemma}

\begin{rmk}
A self-intersecting or horned cyclide can also be mapped to a cylinder or
cone, which has a simple circle decomposition.
\end{rmk}

\begin{rmk}
The analogous problem for ruled surfaces is to find a line decomposition.
In \cite{jj91ru}, the present author shows how to achieve this line 
decomposition, by finding generators and a directrix curve on the 
ruled surface.
\end{rmk}

\begin{rmk}
The circle decomposition of a quadric surface is discussed in \cite{JS90u}.
Recall that the centers lie on a line, and the radius function is quadratic.
% NEED TO KNOW RADIUS FUNCTION: don't know for any nontrivial quadric yet?
\end{rmk}

% **************************************************************

The above method is simple and, since it uses inversion, 
it is a natural method for our intersection algorithm.
However, there are other methods for decomposing a cyclide into circles.
% It does a good job of yielding direct information about the center, 
% radius, and orientation of the circles in the decomposition.
% I think Cayley just gives circles, which has the disadvantage
% of not being finitely representable (since there are an infinite number of
% circles).  Our method gives radius function, orientation function, curve
% of centers, which is compact representation of the infinite number of circles
%
An excellent method is due to Cayley \cite{CAY96}:
if $C_1$ and $C_2$ are the two interior circles of a cyclide
and $P$ is one of their centers of similitude,
% (a point that divides the line
% between the centers of C1 and C2 in the ratio of their radii)
% p. 27-28, Coolidge, or p. 282 of CDH89a for definition
then the circles of the cyclide rotate about the center of similitude.
That is, a circle of the cyclide lies in a plane through $P$ (and
perpendicular to the plane of the interior circles),
and its diameter is defined by the intersections of this plane
with the two interior circles.
This method is also reported in Chandru et. al. \cite[p. 282]{CDH89a}
and Coolidge \cite[p. 270]{Coo71}.

Forsyth \cite{F12} gives formulae for the radius and center of the circles on a
cyclide. 
% center is the conic (a cos \theta, b sin \theta)
However, the formulae are for the normal form (\ref{eqcy}) 
and involve trigonometric functions.

\Comment{
Ignorable:
	(2) Maxwell's construction
	(3) anticonics of cyclide
}

\Comment{
STEINER METHOD (NOT PRESENTED HERE, BUT A POTENTIAL METHOD: MAY PRESENT 
		ELSEWHERE)
	Not worthy of attention given the equally good (probably superior)
		method of Coolidge (also outlined in Dutta, On the Geom), 
		using centers of similitude.
}

% *********************************************************************

% circle decomposition via Steiner: a more direct method 
	% perhaps not as useful in our method of intersection

% *********************************************************************

\Comment{
\section{Intersection}
 
The intersection of a circle with a quadric surface is a degree 4 computation.
It can be computed from the following system of equations:
\[
		f(P) = 0		quadric (deg 2)
\]
\[
		dist(P,c) = r		sphere  (deg 2)
\]
\[
		g(P) = 0		plane   (deg 1)
\]
The solution of this system (e.g., by resultants) is a degree 4 computation.
}

% *********************************************************************

% discuss robustness of inversion (since we use quite a few inversion maps
% to compute intersections: if information is lost with each inversion,
% this is a disadvantage.
% % \input{robust}

% *********************************************************************

% advantages of circle decomposition (including offset)
% \input{advantages}

\section{Advantages of a circle decomposition}
\label{advs}

We have already seen that the representation of a surface by circles
is a great advantage in intersection.
However, there are other reasons that a circle decomposition is advantageous.
\begin{itemize}
\item
	$(\mbox{circle}(s))(t)$ is a very good parameterization.
	%
	% This is the 
	% parameterization where $s$ is the parameter on each circle,
	% and $t$ varies over all the circles.
	% That is, $(\rm{circle}(s))(k)$ is the $k^{th}$ circle.
	%
	It is easy to define geometrically meaningful subsets of 
	the surface by a range on $t$, and it is always of degree 2 in $s$.
\item
	For the cyclide, the circle decomposition has the advantage
	of revealing the lines of curvature.
	These lines of curvature are important, for example,
	in defining principal patches from the cyclide
	\cite{DEP84,MAR82,SHAR85}.
		% a parameterization by lines of curvature (which is 
		% often wanted: e.g., p. 103, Martin's thesis)
\item
	It is simple to render the surface, since the circle is a basic
	graphical element.
\item
	Rotation and translation of the surface is simple:
	it reduces to rotation and translation of the directrix curve
	(Definition~\ref{ringed}), which may be simpler
	than rotation and translation of the entire surface.
	The radius and orientation of each circle does not change.
	Moreover, there is no danger of introducing errors into these 
	radii and orientations due to floating point error 
	during the manipulation.
\item
	The surface has the representation of a generalized cylinder,
	which is a preferred representation for computer vision applications.
\Comment{
THIS IS INDEPENDENT OF A CIRCLE DECOMPOSITION:
\item
	Point membership classification of a cyclide $C$ is simple:
	invert the point and classify it against the torus 
	$T = \mbox{inv}_S(C)$, which is an easy test.
% I am not implying that point membership classification is not simple for
% other representations as well: e.g., two circles rep leads easily to 
% implicit representation, which is well known to be good for PMC
}
\item
	Offset computation of the cyclide is simple (Section~\ref{offset}).
\end{itemize}	

% medial axis of cyclide?

\subsection{The offset of a cyclide}
\label{offset}

An important application of the circular decomposition of a cyclide
is in offset computation.
In general, offsets are difficult to compute \cite{FAR86,FN90b,FN90a},
but the offset of a cyclide is simple, as we shall see.
Not only is the offset of a cyclide simple to compute, but
it is still a cyclide.
This is interesting, because the offset of a surface is usually a 
more complicated surface.

The offset of a surface is very important in machining the surface,
where it indicates the surface that the center of the spherical ball cutter
should follow.
It also has applications in motion planning (where an offset surface
may indicate an error buffer about an obstacle, or how to grow the obstacle
by a circular robot), encoding of tolerance, and as a general tool for
growth/shrinkage of an object.

\begin{defn2}
The {\bf $d$-offset} of a surface $(x(s,t),y(s,t),z(s,t))$ is the surface
\[ 
	(x(s,t),y(s,t),z(s,t)) \pm d N(s,t)
\]
where $N(s,t)$ is the unit normal
to the surface at the point $(x(s,t),y(s,t),z(s,t))$.
\end{defn2}

The use of a circle decomposition is important to the 
simple computation of the offset.

% We need only change the radius of each circle of its circular decomposition
% by the same amount.

\begin{theorem}
\label{thm:offset}
The $d$-offset of the cyclide 
with the circle decomposition
\[ ({\rm center}(t), {\rm radius}(t), {\rm orientation}(t)) \]
is the cyclide with the circle decomposition
\[ ({\rm center}(t), {\rm radius}(t) \pm d, {\rm orientation}(t)) \]
\end{theorem}

This result has been established many times before,
but since we have an interesting and short alternative to the other proofs
in terms of circle decompositions, we will give it.
For other discussions of the offset of cyclides, 
the reader should refer to the clear presentation by Pratt \cite{P89}, 
% p. 414
or Martin \cite{MAR82} or Dutta \cite{Dutta89}.
% Martin, e.g., bottom of p. 103
% Dutta et. al. also mention (but don't prove): P7 of p. 289, CDH89
% DUTTA THESIS, P. 53;
% but the resultant usefulness of the circle decomposition
% has not been observed.

To establish the result, 
we need to show that offsetting along the surface normal is equivalent
to offsetting along the normal to the circles of the circular decomposition.
Note that it is not generally true that the surface normal is equivalent 
to the normal to a curve on the surface
(Figure~\ref{normal}).

\figg{normal}{Surface normals and curve normals are not equivalent 
		[29]}{2in}
% Fig 4.13, p. 102, Millman and Parker

\begin{lemma}
\cite{MP77}
% [p. 110]
Let $\psi(s)$ be a unit speed curve on a surface.
% DOES SPEED OF CURVE AFFECT NORMAL? IF NOT, REMOVE UNIT SPEED CRITERION.
The normal to $\psi(s)$ is everywhere normal to the surface if and only if
$\psi(s)$ is a geodesic.
% N = normal to curve and n = normal to surface (p. 102)
\end{lemma}

\begin{lemma}
\label{lem:grau}
\cite{GRAU47}
% [p. 151]
% problem #7
A line of curvature on an ordinary\footnote{A surface is ordinary if it
	is not a developable surface or a plane.}
surface whose Gauss map is a great circle, or a portion thereof, is a geodesic.
\end{lemma}

% Ordinary surface: p. 96 of Graustein.  [Cyclides are ordinary, because
% they are neither developable nor planes.]

% Gauss map [do we need to define?]
% spherical representation = Gauss map, see p. 96 of Graustein

The circles on the cyclide satisfy this criterion.
Thus, they are geodesics and their normals are equivalent to the surface
normals.
This establishes Theorem~\ref{thm:offset}.
% This allows us to offset along the radii of the circles instead,
% and establishes the desired result.

% ***********************

%      V: another approach would be to show that INVERSION PRESERVES GEODESICS
% 	since the circles of a torus or cylinder are certainly geodesics,
% 	because (for torus) they are meridians of surfaces of revolution,
%	which are always geodesic (p. 110, PROP 5.5, 
%	Millman and Parker, Elts of 
%	Differential Geometry); and it is clear for cylinder.
%	This would be a nice property to prove, independent of this problem
%		[Know normal to circles of torus are normals of surface
%		Key problem: does inversion preserve this property?]
%	Property to test: Does inversion always preserve geodesics?



% *********************************************************************

% volume of cyclide?  LEAVE TO DIFFERENT PAPER
% % \input{volume}

% *********************************************************************

% NEED IMPLEMENTATION TO VERIFY MY FINDINGS BEFORE PUBLICATION

%\input{implementation}
%
% this paper is the foundation of a solid modeler including all quadrics,
% all cyclides, as well as all ruled surfaces (based on a related paper
% on ruled surfaces)
%
% already some implementation by Dutta, using circles in planes of symmetry
% see Dutta, `On the Geom of cyclides', Appendix, p. 290

% *********************************************************************

% \input{conclusion}

\section{Conclusions}
\label{sec:conc}

In this paper, we have developed a method for the exact intersection of any
ringed surface with any surface in the present vocabulary (quadrics and
cyclides).
This is a first step towards expanding the vocabulary 
of a solid modeler to include all ringed surfaces.
The intersection algorithm is not only directly applicable 
to modeling with these surfaces,
but it reveals that exact intersection of complex surfaces is feasible,
and that the decomposition of surfaces into simpler components
(in this case, circles) can be powerful.
We suspect that alternatives such as this to 
traditional methods (e.g., elimination methods or implicit/parametric
combinations) will prove very useful in handling other complicated
intersections.

The two key techniques of this paper are circle decomposition and inversion.
The circle decomposition immediately simplifies an apparently degree $4n$
computation (where $n$ is the degree of the ringed surface),
ringed $\cap$ cyclide, 
to a degree 8 computation of ${\rm circle}(t) \cap {\rm cyclide}$.
The inversion map induces another simplification,
transforming a cyclide to a torus (which is an easier surface to work with)
and the circle to a line (which is a curve of lower degree).
This leads to a degree 4 computation of the intersection.

We end by proposing some directions for future work.
First, the extension of other algorithms to all ringed surfaces is needed,
so that we can fully incorporate this class into the vocabulary.
Second, since we have seen the benefit of the description of surfaces 
as circles or lines sweeping along simple curves (ringed and ruled surfaces),
this approach might be extended further, looking at other simple curves
sweeping along simple curves.
The work on generalized cyclides (e.g., Degen \cite{Degen90}) 
is an example of work in this direction.
% e.g., unions of conics, even unions of circles$+$lines = parabolic cyclides
Finally, the intersection of ringed surface with ringed surface, 
and ruled surface with ruled surface is an important problem that we are 
presently working on.




% *********************************************************************

\section{Acknowledgements}

Discussions with Chris Hoffmann were very helpful.
I thank C.-K. Shene for an improved proof of Theorem~\ref{thm:rad},
and for useful discussions.

\bibliographystyle{plain}
\bibliography{/users/cogito/jj/bib/modeling}

\end{document}
