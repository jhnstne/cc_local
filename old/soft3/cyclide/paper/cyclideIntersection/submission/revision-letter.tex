\documentstyle[11pt]{bletter}
\signature{Prof. John K. Johnstone\\jj@cs.jhu.edu}
\begin{document}
\begin{letter}
{Prof. Gerald Farin\\
Executive Editor, CAGD\\
Dept. of Computer Science\\
Arizona State University\\
Tempe, AZ 85287-5406
}

\opening{Dear Prof. Farin:}

I enclose the revised version of the paper `A New Intersection Algorithm using
Circle Decomposition'.  I have implemented the suggestions of the referees,
which were very useful. A description of these changes follows.

\noindent In response to the comments of Referee 1.
\begin{itemize}
\item
	The title has been changed.
\item
	Section 6, starting with the paragraph `We can now restrict our
	attention to the intersection of a ringed surface with a ring
	cyclide', has been altered and augmented,
	in the interests of clarification.
	A fuller definition of the two cyclides $c_1(t)$ and $c_2(t)$
	is included.
	A discussion of the tracking of the center of inversion
	as it moves through the torus is provided just above Remark 6.2
	and in step (4).
%	Of course, a given interval $(t_i,t_{i+1})$ is inside the torus if
%	and only if one of one of its points is.
	A remark at the end of Section 6 about the roots of the original
	circle/cyclide intersection has been added.
\item
	The step-by-step description of the algorithm has been clarified
	and streamlined.
\item
	A discussion of circle-to-line inversion has been added in 
	Section 10 (Section 10.4).
\end{itemize}

\noindent In response to the comments of Referee 2,
\begin{itemize}
\item
	The introduction has been changed in two places to 
	reflect that a parameterization is found (Comment 1).
\item
	References to the fact that a quadric is a ringed surface
	have been added in the last paragraph of Section 2
	(Comment 2).
\item
	The definition of ringed surface has been changed to reflect that
	the circle can change its radius as it sweeps (Comment 3).
\item
	A reference to Properties (1)-(5) of Section 5 has been provided
	(Comment 4).
\item
	I do not understand Comment 5.
	The notation $c$ has not been used previously in this section
	(it was used to denote a center of inversion in Section 5, which
	is consistent); and $k$ is already used to denote the 
	constant parameter value that defines a point on each circle,
	$(\mbox{circle}(t))(k)$, for the second center of inversion.
\item
	The two typos (Comments 6 and 7) have been corrected.
\item
	A definition of center of similitude has been added in Section 11
	(Comment 8).
\item
	Point 4 of Section 12 has been corrected (Comment 9).
\item
	Examples of desired extensions have been given in the Conclusions
	(Comment 10).
\end{itemize}

\noindent Some other small changes have been made.
\begin{itemize}
% \item 
% 	added adjective `finite' to Sabin remark in Section 3
\item
	A sentence has been added to the very end of the abstract.
\item
	The cyclide pictures in Figure~1  have been replaced by our own
	images.  In the final version of the paper, I will provide you with
	photographic slides to insert here.
	However, I have inserted black and white postscript versions
	of these images for this copy.
\item
	Some references on cyclides have been added to the penultimate
	paragraph of Section 3.
\item
	Remark 6.3 about numeric computation 
	has been added at the end of Section 6.
\item
	The first two paragraphs of Section 7 have been rearranged for clarity.
	No material has been added.
\end{itemize}

I hope that this answers all of the comments of the referees.
Please contact me at (410) 516-5560 if you have any questions.

\closing{Sincerely,}
\end{letter}
\end{document}
