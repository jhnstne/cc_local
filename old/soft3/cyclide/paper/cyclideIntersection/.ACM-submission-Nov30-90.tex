% Showing how a Dupin cyclide is actually a circle sweeping along a conic
\newif\ifFull
\Fullfalse
\documentstyle[12pt]{article} 
\input{macros}
\input{ruledmacros}
\input{pageformat-double}
\newtheorem{defn2}{Definition}
\begin{document}
\bibliographystyle{plain}

\title{Sweeping a circle into a cyclide}
\author{John K. Johnstone\thanks{Department of Computer Science, The Johns 
	Hopkins University, Baltimore, Maryland 21218.
	This work is supported by National Science Foundation grant
	IRI-8910366.}}
\date{\today}

\maketitle

% \tableofcontents

\SingleSpace

\Comment{
NORMAL FORMS
	R = radius of circle of inversion
	(a,b,c) = center of a generic circle of cylinder, torus, cone
}

\begin{abstract}
In this paper, we study the inversion map and the Dupin cyclide.
We thoroughly examine the inversion of a circle, which is a circle,
and develop results about the mapping of 
the center, radius, and orientation of a circle in 3-space
under inversion.
This examination reveals a striking symmetry between these maps.
We also show that, in special cases, there is a strong relationship
between the inversion of one circle and the inversion of a collection
of circles.

These results are then used to analyze the Dupin cyclide.
We develop a constructive method for representing
a cyclide as a circle sweeping along a conic.
This decomposition can be useful in the manipulation of cyclides.
\end{abstract}

% Keywords: inversion, Dupin cyclide, circle.

\section{Introduction}

Inversion is an important map, and one of its key properties is that it
maps circles to circles.
A Dupin cyclide is a quartic surface that can be generated by circles.
In this paper, by examining the inversion of a circle, we discover
a way of representing the cyclide as the locus of 
a circle sweeping along a conic
(i.e., a circle whose center sweeps along a conic).
This representation can be useful in the treatment
of cyclides in solid modeling.

Since a circle is defined by its center, radius, and orientation,
the mapping of a circle by inversion is defined by the mapping of its
center, radius, and orientation.
These maps are partially understood, but we examine them in detail to
fully reveal their structure.
% SPACE CIRCLES
% NON-TRIVIAL EXTENSION TO SPACE CIRCLES, WHICH ARE THE DIFFICULT CASE
%	(LIKE SPACE CURVES VS. PLANE CURVES)
Classical results are extended from circles in two dimensions to all circles.
Our work also reveals a strong relationship between the 
mapping of centers, the mapping of radii, the mapping of orientations,
and the inversion map.
We explore this relationship further and discover that, whereas inversion
maps circles to circles, the mapping of centers maps circles to conics.
This result turns out to be strongly related to cyclides, and yields
the above decomposition of a cyclide.

The fact that a cyclide can be generated by sweeping 
a circle along a conic is not new.
For example, Coolidge states it without proof \cite[p. 267]{Coo71}.
The contribution of this paper is to construct this representation:
given a cyclide, we determine the conic and how
the radius and orientation of the circle changes as it sweeps along
the conic.
We also refine the result, showing that most cyclides can be generated
by sweeping a circle along an ellipse.


% RULED SURFACE, GENERALIZED CYLINDERS, RIBBONS

The next section discusses the fundamentals of the inversion map.
Section~\ref{sec-circleinverse} is our investigation of the inverse of a 
circle.  We show how the center, radius, and orientation of a circle 
change under inversion.
Section~\ref{sec-Dupin} applies these results to the Dupin cyclide,
yielding our method for decomposing the cyclide into a circle
sweeping along a conic,
and we finish with some conclusions in Section~\ref{sec-con}.

% ********************************************

\section{Inversion}
We begin with a definition of the inversion map.
%
\begin{definition}
Let $S$ be a sphere of radius $r$ centered at $c$.
The {\em inverse} of a point $p \in \Re^{3}$ with respect to the sphere $S$,
$\mbox{inv}_{S}(p)$, is the point $p' \in \ray{cp}$ such that
$\mbox{dist}(c,p)\ \mbox{dist}(c,p') = r^{2}$.

One may also define an inversion map restricted to a plane.
Let $C$ be a circle of radius $r$ centered at $c$ and lying in the plane $P$.
The {\em inverse} of a point $p \in P$ with respect to the circle $C$, 
$\mbox{inv}_{C}(p)$, is the point $p' \in \ray{cp}$ such that
$\mbox{dist}(c,p)\ \mbox{dist}(c,p') = r^{2}$.

In both cases, $c$ is called the {\em center of inversion}
and $r$ is called the {\em radius of inversion}.
\end{definition}

\figg{}{Inversion with respect to a circle}{2in}
% picture including $c, p', p$
 
The inversion map satisfies $\mbox{inv}(\mbox{inv}(x)) = x$.
We shall often make use of this property.
Inversion is also conformal: the angle between two intersecting 
curves is preserved under inversion.
% \cite{Davis, p. 214}

\begin{lemma}
\label{lem-inv}
The inverse of a point $P = (a,b,c)$ with respect to a sphere
of radius $R$ centered at the origin is 
$\frac{R^{2}}{a^{2} + b^{2} + c^{2}} (a,b,c)$.
\end{lemma}
\Heading{Proof:}
The inverse $P'$ of $P$ lies on the line between $P$ and the 
center of inversion.
Thus, $P' = kP$.
By the law of inversion, $\|kP\| \|P\| = R^{2}$ or
$k = \frac{R^{2}}{\|P\|^{2}}$.
\QED

We will be working with circles in the context of inversion.
One can identify two types of circle: circles that lie in a plane
that contains the center of inversion, which we call {\em plane circles},
and circles whose plane does not contain the center of inversion,
which we call {\em space circles}.
The inversion literature restricts to plane circles.
It is easier to develop results for plane circles,
since one can restrict to the plane of the circle and treat
the problem as a two-dimensional one, using inversion with respect to a circle.
However, space circles are just as important as plane circles, and even more
common.
Therefore, we will develop our theory for all circles.
When we deal with space circles, the sphere with the same center and radius
often plays an important role.

One of the most fundamental properties of inversion is that it maps
circles to circles (or, in degenerate cases, lines).

\begin{lemma}
\label{lem:inversion}
\cite{D49,Coo71}
% \cite[p. 210]{D49}: nice proof for plane circles.
% \cite[p. 26, 228]{Coolidge} % statement for plane circles (p. 26)
			    % statement for spheres and space circles (p. 228)
Let $S$ be the sphere of inversion centered at the origin.
The inverse of a circle $C$, $0 \not\in C$, is a circle.
The inverse of a circle $C$, $0 \in C$,     is a line.\footnote{This is 
	the key to Peaucellier's linkage for drawing a straight line.}

The inverse of a sphere $C$, $0 \not\in C$, is a sphere.
The inverse of a sphere $C$, $0 \in C$, is a plane.
\end{lemma}
\Comment{
Proof for space circles:
Any circle is the intersection of a sphere and a plane.
If the circle contains the center of inversion, then 
the sphere and plane both invert to a plane, which intersect in a line.
Otherwise, the sphere and plane both invert to a sphere,
which intersect in a circle.
}

\begin{corollary}
Let $S$ be the sphere of inversion, centered at the origin.
The inverse of a line $L$, $0 \not\in L$, is a circle $C$ that contains the 
origin.
The inverse of a line $L$, $0 \in L$, is the line $L$.
The inverse of a plane $L$, $0 \not\in L$, is a sphere $C$ that contains the
origin.
The inverse of a plane $L$, $0 \in L$, is the plane $L$.
\end{corollary}
\Heading{Proof:}
$\mbox{inv}(\mbox{inv}(p)) = p$.
\QED

We want to compute the inverse of a circle.
Three pieces of information are necessary to fully define 
a circle in three dimensions: its center, its radius, and its
orientation in space.
In the next section, we examine the effect of inversion on each of these
elements.

% ********************************

\section{The inverse of a circle}
\label{sec-circleinverse}

\subsection{Center of inverse circle}

Given a circle, 
we are interested in the center of the inverse circle.
Since points further away from the sphere of inversion are
stretched more, this center is certainly not the inverse 
of the original center.
The following theorems show how to find it by two applications of the
inversion map.

% TWO-DIMENSIONAL RESULT WAS KNOWN.
% EXTENSION TO SPHERES AND 3D CIRCLES IS NEW

\begin{theorem}
\label{thm-twoappl1}
\cite{Coo71,D49}
%
% The reader is referred to \cite[p. 216]{D49} for a proof of the plane circle
% case.
%
Let $C$ be the sphere of inversion with center $c$.
Let $D$ be a plane circle\footnote{Recall that a plane circle is a
	circle whose plane contains the center of inversion.}
or a sphere, $c \not \in D$.
The center of $\mbox{inv}_{C}(D)$ is $ \mbox{inv}_{C} ( \mbox{inv}_{D} (c))$.
\end{theorem}

% INCLUDE PROOF OF SPHERE CASE (SINCE COOLIDGE HAS NO PROOF)?
%
\Comment{
\Heading{Proof:}
%
% This result is known for plane circles and spheres.
% We must extend it to space circles.
% For plane circles, the result is the following:
% if $C$ and $D$ are two circles in the same plane, then the center of 
% $\mbox{inv}_{C}(D)$ is $\mbox{inv}_{C}(\mbox{inv}_{D}(c))$.
%
Since the result for spheres is stated without proof in Coolidge \cite{Coo71},
we provide a proof.
% p. 231
Let $D$ be a sphere.
Consider any plane $p$ through the centers of both spheres $C$ and $D$.
This plane divides both spheres in half.
That is, it is a plane of symmetry of $D$.
It is clear that points on one side of the plane
stay on that side of the plane under inversion.
Therefore the plane remains a plane of symmetry for the inverse of $D$.
% 
\Comment{
(This can be seen by considering the image of diametral pairs.
If $p$ isn't a plane of symmetry, 
then, on the inverse sphere, there exist two points that define a diameter
(i.e., a diametral pair of points)
that lie strictly on one side of the plane.
But a diametral pair on inv($D$) is the inverse of a diametral pair
on $D$ (because they are extremal ELABORATE), and all diametral pairs 
of the original sphere either lie on opposite
sides of the plane or they both lie on the plane.
This leads to a contradiction.)
}
%
This means that the center of $\mbox{inv}(D)$ is the center of 
$\mbox{inv}(D) \cap p$.
Our problem now reduces to a two-dimensional problem on the plane $p$,
using the two circles $D \cap p$ and $\mbox{inv}_{C}(D) \cap p$.
\QED
}

We need to extend this result to space circles.

\begin{theorem}
\label{thm-twoappl2}
Let $C$ be the sphere of inversion with center $c$.
Let $D$ be a space circle and let $E$ be the sphere with the same center 
and radius as $D$, $c \not \in E$.
The center of $\mbox{inv}_{C}(D)$ is $ \mbox{inv}_{C} ( \mbox{inv}_{E} (c)) $.
\end{theorem}
\Heading{Proof:}
Let $D$ be a space circle.
$D$ is a great circle of $E$.
We wish to show that inv(D) is still a great circle of inv(E).
If $p$ is the plane containing $D$, then $D = E \cap p$.
Thus, $\mbox{inv}(D) = \mbox{inv}(E) \cap \mbox{inv}(p)$.
Since the center of inversion does not lie in $p$ ($D$ is a space circle), 
both $\mbox{inv}(E)$ and $\mbox{inv}(p)$ are spheres.
Points on one side of the plane $p$ are inverted to the inside of the sphere 
$\mbox{inv}(p)$; points on the other side to the outside.
Thus, given two diametral points of the sphere $E$, 
one is mapped to the inside of $\mbox{inv}(p)$ and the other to the outside
(or both are mapped onto $\mbox{inv}(p)$).
$\mbox{inv}(p)$ must intersect $\mbox{inv}(E)$ in a
great circle, otherwise some diametral pair would lie on the same
side of $\mbox{inv}(p)$.
Thus, $\mbox{inv}(D)$ is a great circle of $\mbox{inv}(E)$.
The center of the sphere $\mbox{inv}(E)$ is 
$ \mbox{inv}_{C} ( \mbox{inv}_{E} (c)) $, which is also the center of 
$\mbox{inv}(D)$.
\QED

We can state a similar result for lines and planes.
Inversion in a circle is replaced by reflection in a line.

% LINE CASE ISN'T ACTUALLY NEEDED, ONLY PLANE

\begin{theorem}
\label{lem:cenplane}
\cite{Coo71}
% [p. 30, 231]
Let $C$ be the sphere of inversion with center $c$.
Let $D$ be a line or plane, $c \not \in D$.
The center of $\mbox{inv}_{C}(D)$ is 
$\mbox{inv}_{C}(\mbox{reflection}_{D}(c))$.
\end{theorem}
%
% The result for lines in 2D (i.e., line in same plane as circle $C$)
% is stated in Coolidge \cite[p. 30]{Coo71}.
% BUT NO PROOF IN COOLIDGE
% Same proof idea as that used in the proof of 
% Lemma~\ref{lem-twoappl} in Davis \cite{D49}.
% Let $A$ be the center of the inverse of $D$.
% We shall prove the equivalent result:
% $c = \mbox{reflection}_{D}(\mbox{inv}_{C}(A))$.
% VERY CLOSE TO A PROOF, BUT CAN'T SEEM TO PROVE THAT CENTER OF inv(d) lies
% on the line cA.

% **************************************************************

\subsection{Formula for center of inverse of circle}

The following result shows that the center of the inverse circle 
bears a close relationship to the inverse of the center of the circle.
It applies to both plane and space circles.
Recall that the inverse of the point $(a,b,c)$ in a sphere
of radius $R$ centered at the origin is 
$\frac{R^{2}}{a^{2} + b^{2} + c^{2}} (a,b,c)$ 
(Lemma~\ref{lem-inv}).

% THIS RESULT IS NEW  (ESPECIALLY RELATIONSHIP TO POWER)

\begin{lemma}
\label{lem-center}
Let $C$ be the sphere of inversion of radius $R$ centered at the origin.
Let $D$ be a circle % in 2D or 3D
with center $(a,b,c) \neq 0$ and radius $r$,
and let $E$ be the sphere with the same center and radius as $D$,
$0 \not \in E$.
$D$ is inverted into a circle with center
$\frac{R^{2}}{|a^{2} + b^{2} + c^{2} - r^{2}|} (a,b,c)$.
A circle centered at the origin is inverted into a circle centered at the 
origin.
The same results hold for a sphere $D$.
\end{lemma}
\Heading{Proof:}
By Theorems~\ref{thm-twoappl1} and \ref{thm-twoappl2}, the new center is 
$ \mbox{inv}_{C} ( \mbox{inv}_{E} (0)) $.
Consider $\mbox{inv}_{E}(0)$.
This inverse lies on the line between $0$ and $(a,b,c)$.
In particular, it is $k * (a,b,c)$ such that 
$\mbox{dist}((a,b,c), k(a,b,c)) * \mbox{dist}((a,b,c),0) = r^{2}$.
That is, 
\[ (1-k)\ \|(a,b,c)\|^{2} = r^{2}  \]
\[ k = \left\{ \begin{array}{ll}
	\frac{a^{2} + b^{2} + c^{2} - r^{2}}{a^{2} + b^{2} + c^{2}}
	& \mbox{if $k > 0$} \\
	\frac{r^{2} - (a^{2} + b^{2} + c^{2})}{a^{2} + b^{2} + c^{2}}
	& \mbox{if $k < 0$} 
	\end{array} \right. \]
If the origin lies outside $E$, then the inverse of the origin is
inside $E$ and $k \in [0,1]$, otherwise $k < 0$.
That is, if $r < \|(a,b,c)\|$, then $k >0$, and 
if $r > \|(a,b,c)\|$, then $k < 0$.
Therefore, 
$k = \frac{|a^{2} + b^{2} + c^{2} - r^{2}|}{a^{2} + b^{2} + c^{2}}$
and $P := \mbox{inv}_{E}(0) = 
(\frac{|a^{2}+b^{2}+c^{2}-r^{2}|}{a^{2} + b^{2} + c^{2}})*(a,b,c)$.
By Lemma~\ref{lem-inv}, $\mbox{inv}_{C}(P) = \frac{R^{2}}{\|P\|^{2}} P$.
Since 
$\|P\|^{2} = \frac{|a^{2}+b^{2}+c^{2}-r^{2}|^{2}}{a^{2}+b^{2}+c^{2}}$,
$\mbox{inv}_{C}(P) = \frac{R^{2}}{|a^{2}+b^{2}+c^{2}-r^{2}|} (a,b,c)$.
\QED

It is interesting to note that the new center of a circle does not depend 
upon the orientation of the circle (i.e., the plane that the circle lies in).
The relationship between the new center of a circle and the inverse of a point
(Lemma~\ref{lem-inv}) is now clear: 
only one term has been added to the denominator.
The new center of a circle also has a connection to the power of a point.
%
\begin{definition}
The {\em power of a point} $p$ with respect to a sphere $S$ is usually defined
as the product of distances of $p$ from any pair of points $U$ and $V$
on the sphere such that $U$ and $V$ lie on a line through $p$ 
(Figure~\ref{fig-power1}).
Since this product is the same for all choices of $U$ and $V$, 
this is well defined.
%
% WHY????????????????????????????????
% 
We prefer to give the following equivalent definition:
$\mbox{power}_{S}(p)$ is the product of the distances of $p$ from the
closest point and furthest point of the sphere.
(Here we are using the line through $p$ and the center of the sphere.)
% OLD DEFINITION
% Let $p \in \Re^{3}$ and let $S$ be a sphere.
% If $U$ and $V$ are two points of the sphere on a line through $p$
% (Figure~\ref{fig-power1}), then the {\em power} of $p$ with respect to $S$,
% $\mbox{power}_{S}(p)$,
% is the product $\mbox{dist}(p,U) \mbox{dist}(p,V)$.
% This is well defined because the value of $\mbox{dist}(p,U) \mbox{dist}(p,V)$
% is the same for all choices of $U$ and $V$.
\end{definition}

% DISCUSS CLASSICAL USES OF POWER.
% (WHY IS POWER A CLASSICAL VALUE?)

\begin{example}
\label{eg:power}
Let $p$ be the origin and let $S$ be the sphere of radius $r$ centered at
$(a,b,c)$.
If $p$ lies outside the sphere ($\| (a,b,c) \| > r$),
the closest and furthest points of the sphere 
are at distance $\|(a,b,c)\| - r$ and $\|(a,b,c)\| + r$ 
from $p$, and the power of $p$ with respect to $S$ is 
$a^{2} + b^{2} + c^{2} - r^{2}$.
If $p$ lies inside the sphere ($\| (a,b,c) \| < r$),
then these two points are at distance $\|(a,b,c)\| + r$ and 
$(-1)(r - \|(a,b,c)\|)$, and again the power is 
$a^{2} + b^{2} + c^{2} - r^{2}$.
\end{example}

\figg{fig-power1}{The power of $p$ with respect to the circle is 
		$\mbox{dist}(p,U)\mbox{dist}(p,V)$}{2in}

The power of a point can be thought of as a distance measure of the point
from the sphere: using both the closest and furthest points of the sphere
rather than just the closest point.
If only the closest point is used, then the inside of the sphere cannot 
be distinguished from the outside.
However, the power of a point is positive when the point lies outside the 
sphere and negative when the point lies inside.

% PROVE THAT POWER IS TRULY A DISTANCE METRIC (E.G., SATISFIES TRIANGLE 
% INEQUALITY)

Lemma~\ref{lem-center} shows that the power of a point is related to the
center of an inverse circle.
In particular, the old center $(a,b,c)$ inverts to the new center 
$\frac{R^{2}}{|\mbox{power}_{E}(0)|} (a,b,c)$.
We will see a similar relationship
when we investigate the radius of an inverse circle.
%
%
\Comment{
Also note that $a^{2}+b^{2}+c^{2}-r^{2} = (\|(a,b,c)\| -r)(\|(a,b,c)\| + r)$,
the product of the distance of two points on the circle on the line between
the center of the circle and the center of inversion.
}
%
\begin{definition}
The mapping $f(c)$ between the center $c$ of a circle and the center $f(c)$ of
its inverse will be called {\em central inversion}.
Since this mapping also depends on the radius of the circle, it is not 
strictly a point to point mapping.
We assume that there is
an underlying radius function that associates a radius with each 
center.
\end{definition}

\subsection{Formula for center of inverse of plane}

% PUT AFTER FORMULA FOR CENTER OF CIRCLE AND SPHERE.
% (A) OFFERS COMPLETENESS, 
% (B) SHOWS SIMILARITY OF RESULT FOR DEGENERATE CASE,
% AND (C) THIS RESULT IS NEEDED IN THE SECTION ON ORIENTATION

We now develop the analogous result for planes.

\Comment{
\begin{lemma}
\label{lem:plane}
\cite{SAL1927}
The normal vector of the plane $Ax+By+Cz+D=0$ is $(A,B,C)$.
% p. 513, Thomas and Finney, Calculus and Anal. Geom, 5th edition
The distance of the plane $Ax+By+Cz+D=0$ from the origin is 
$\frac{|D|}{\|(A,B,C)\|}$.
% p. 23, Salmon, Anal. Geom of 3D
\end{lemma}
\Comment{
Proof: First fact is well known. For second, let closest point be k(a,b,c).
Plug this into the equation of the plane and solve for k.  
Distance is |k| * \|(a,b,c)\|.
}
}

\begin{lemma}
\label{cor-plane}
Let $S$ be the sphere of inversion of radius $R$ centered at the origin.
The plane $Ax+By+Cz+D=0$ ($D \neq 0$) inverts into a sphere with center
$\frac{-R^{2}}{2D} (A,B,C)$.
\end{lemma}
\Heading{Proof:}
We wish to determine the reflection of the origin in 
the plane $Ax+By+Cz+D=0$ (Theorem~\ref{lem:cenplane}).
Recall the following fact about planes: 
% The following lemma identifies the relationship between the algebra
% and geometry of the plane.
the unit normal vector of the plane $Ax+By+Cz+D=0$
is $\frac{(A,B,C)}{\|(A,B,C)\|}$
and its distance from the origin is $\frac{|D|}{\|(A,B,C)\|}$.
% \cite{SAL1927}
Therefore, the reflection of the origin is 
$\pm \frac{2D}{\|(A,B,C)\|^{2}} (A,B,C)$.
We must decide which sign to choose.
Let $f(x,y,z) = Ax+By+Cz+D$.
Since the origin and its reflection lie on opposite sides of the plane,
the sign of $f(0) = D$ must be the opposite of the sign of 
$f(\pm \frac{2D}{\|(A,B,C)\|^{2}} (A,B,C)) = \pm D (2\|(A,B,C)\| + 1)$.
Therefore, we should choose the negative sign.
Using a straightforward application of the inversion map (Lemma~\ref{lem-inv}),
the inverse of this reflection is $\frac{-R^{2}}{2D} (A,B,C)$.
\QED

% CHECK THE LAST STATEMENT OF THE ABOVE PROOF ***************************

% *************************************************************

\subsection{Radius of inverse circle}

A circle is defined by its center, radius, and orientation.
The previous section considered the center of the inverse circle.
In this section, we consider its radius.
Note the similarity of the following result with the center of the new circle 
(Lemma~\ref{lem-center}).

% NO PROOF IN COOLIDGE

% USE OUR PROOF (BECAUSE COOLIDGE IS JUST A STATEMENT WITHOUT PROOF)?
% ONLY IN JOURNAL VERSION

\begin{lemma}
\label{lem-radius1}
\cite{Coo71}  % no proof
% [p. 30,231]
% sphere: \cite[p. 231]{Coo71}
Let $S$ be the sphere of inversion of radius $R$ centered at the origin.
Let $D$ be a plane circle or a sphere of radius $r$ and center $(a,b,c)$.
The radius of $\mbox{inv}_{S}(D)$ is 
$\frac{R^{2}}{|\mbox{power}_{D}(0)|} r$ or
$\frac{R^{2}}{|a^{2} + b^{2} + c^{2} - r^{2}|} r$.
\end{lemma}
% In particular, 
% let $D$ be a circle whose plane contains the origin, with 
% center $(a,b,c)$ and radius $r$.
% The radius of $\mbox{inv}_{C}(D)$ is
% $\frac{R^{2}r}{|a^{2}+b^{2}+c^{2}-r^{2}|}$.
% 
\Comment{
\Heading{Proof:}
IF I INCLUDE THIS PROOF, MAKE IT MORE CONSISTENT WITH PROOF OF SPACE CIRCLE
BELOW.
Consider the line $L$ connecting the center of inversion and the center of the
circle.
It intersects the circle in two points, $A$ and $B$, which form a diameter
of the circle.
The inverses of $A$ and $B$ are also a diameter of the inverse circle,
since the center of the inverse circle also lies on $L$ (Figure~\ref{fig-bar}).
Therefore, we want to compute the distance between $\mbox{inv}(A)$ and
$\mbox{inv}(B)$.
$A$ is a distance of $r$ from the center of the circle:
\[ A = (a,b,c) - r \frac{(a,b,c)}{\|(a,b,c)\|} 
     = (1 - \frac{r}{\sqrt{a^{2}+b^{2}+c^{2}}})*(a,b,c) \]
\[ B = (1 + \frac{r}{\sqrt{a^{2}+b^{2}+c^{2}}})*(a,b,c) \]
Applying Lemma~\ref{lem-inv} to get the inverses of $A$ and $B$,
and computing half of the distance between these two inverses,
we find that the radius of the inverse circle is
$\frac{R^{2}r}{|a^{2}+b^{2}+c^{2}-r^{2}|}$.
\QED

\figg{fig-bar}{}{1in}
}

In order to extend this result to space circles, we must generalize
our definition of power from spheres to circles.
Because of the way that we have defined power, this is very easy.
%
\begin{definition}
The {\em power of a point $p$ with respect to a circle $C$} is the product of 
the distances of $p$ from the closest and furthest points of $C$.
We still interpret the power of a point as a distance measure, now from 
a circle.
\end{definition}

With this generalization of power, the radius of a space circle 
is changed in the same way as a plane circle or sphere by inversion.

\begin{lemma}
\label{lem-radius2}
Let $S$ be the sphere of inversion of radius $R$ centered at the origin.
Let $D$ be a space circle of radius $r$.
The radius of $\mbox{inv}_{S}(D)$ is $\frac{R^{2}}{|\mbox{power}_{D}(0)|} r$.
\end{lemma}
\Heading{Proof:}
For simplicity, we can assume without loss of generality that $D$ lies in 
a horizontal plane $z = c$, by rotation about the origin.
We can also assume that its center lies on the $y=0$ plane,
by rotation about the $z$-axis.
Let $D$'s center be $(a,0,c)$.

THIS IS NOT TRUE.
(BEWARE OF THESE WEAK ARGUMENTS.)
If $\seg{AB}$ is a diameter of $D$, 
then $\seg{\mbox{inv}(A)\mbox{inv}(B)}$ 
is a diameter of $\mbox{inv}_{S}(D)$.
To see this, consider the plane through $\seg{AB}$ and the origin.
This is a plane of symmetry for the circle $D$.
It is also a plane of symmetry for the inverse circle, because
inversion maps points on one side of the plane to the same side of the plane.
Therefore, $\seg{\mbox{inv}(A)\mbox{inv}(B)}$, which lies on the plane, 
is a diameter of the inverse circle.

Consider the diameter of $D$ defined by $A=(a-r,0,c)$ and $B=(a+r,0,c)$.
The radius of $\mbox{inv}(D)$ is 
$\frac{1}{2} \|\mbox{inv}(A) - \mbox{inv}(B)\|$. 
Using Lemma~\ref{lem-inv}, this value is 
$\frac{R^{2}}{\|(a-r,0,c)\|\|(a+r,0,c)\|} r
= \frac{R^{2}}{|\mbox{power}_{D}(0)|} r$.
% SHOW PARTS OF DERIVATION.
% IN PARTICULAR, CHECK USE OF (A-R,0,C) RATHER THAN (A-R,B,C).
% I'M SURE IT IS OK: DIFFERENCE IS THAT BEFORE WE WERE ASSUMING CENTER
% OF CIRCLE WAS (A,B,C) NOT (A,0,C)
% THIS DOESN'T YIELD POWER THAT WE WANT.
%
% The new radius is $\frac{1}{2} \|\mbox{inv}(A) - \mbox{inv}(B)\|
% = \frac{R^{2}r}{\sqrt{(a^{2}+b^{2}+c^{2}+r^{2}+2ar)
% 		       (a^{2}+b^{2}+c^{2}+r^{2}-2ar)}}$.
% Derived by myself and checked by Mathematica.
%
\Comment{
In order to establish that 
$\|(a-r,b,c)\|\|(a+r,b,c)\| = \mbox{power}_{D}(0)$, 
it is easiest to use an alternative definition of power.
Recall that the power of a point $p$ with respect to a sphere
could also be defined as the product of the distances of $p$ from 
any pair of points $U$ and $V$ on the sphere, 
where $U$ and $V$ lie on a line through $p$.
Our generalization of the power of a point $p$ with respect to a circle $D$
can also be defined as the product of the distances of $p$ from
any pair of points $U$ and $V$ on the circle $D$, where $U$ and $V$ 
lie on a plane through $p$ and $p'$, the projection of $p$ on $C$'s plane.
If the plane through $p$, $p'$ and the center of the sphere is chosen, 
then $U$ and $V$ are the closest and furthest points of the circle,
and we have the definition in Definition~\ref{def-gpower}.
Moreover, the above definition is well defined because the 
product is independent of $U$ and $V$, by a direct generalization of
the spherical case.
}
\QED

By interpreting power as a distance metric, we can now see a beautiful
symmetry developing between the effect of inversion on a center, radius, 
and point.
If inversion in the sphere of radius $R$ centered at the origin is used,
the center $x$ of the circle $X$ is mapped to the new center 
$\frac{R^{2}}{|\mbox{dist}(O,X)|} x$,
the radius $x$ of the circle $X$ is mapped to the new radius
$\frac{R^{2}}{|\mbox{dist}(O,X)|} x$,
and the point $X$ in the direction of the unit vector $x$ is mapped to
the new point $\frac{R^{2}}{\mbox{dist}(O,X)} x$.
A related structure is also visible in the mapping of the orientation of
a circle, as described in the next section.

% It is interesting that the radius of the new circle does not depend on 
% the orientation of the original circle.

% *************************************************************

\subsection{Orientation of inverse circle}

The orientation of a circle is determined by the plane that contains it.
Since the orientation of a plane circle does not change under inversion, 
the inversion literature does not contain any discussion of orientation.
The following lemma shows that it is sufficient to determine a sphere
that contains the circle.
%
\begin{definition}
The {\em orientation} of a circle is the normal of the plane that contains
the circle.
\end{definition}

\begin{lemma}
\label{lem:sphere}
The orientation of a circle $C$ that lies on a sphere $S$ is $\ray{cs}$,
where $c$ and $s$ are the centers of $C$ and $S$, respectively.
\end{lemma}
\Heading{Proof:}
Simple.
\Comment{
Assume without loss of generality that $C$ lies in a horizontal plane.
We must show that $\ray{cs}$ is vertical.
The sphere is symmetric with respect to all planes through its center.
Thus, all vertical planes through $s$ contain $c$. 
In particular, the vertical line that is the intersection of these planes
also contains $c$.
That is, $\ray{cs}$ is vertical.
}
\QED

% THIS RESULT IS NEW, SINCE ORIENTATION OF PLANE CIRCLE DOES NOT CHANGE

\begin{lemma}
\label{thm:orient}
Let $S$ be the sphere of inversion of radius $R$ centered at the origin.
A circle of orientation $(A,B,C)$, $||A,B,C|| = 1$,
that lies in the plane $Ax+By+Cz+D=0$, $D \neq 0$,
is inverted into a circle of orientation $\frac{R^{2}}{2D} (A,B,C) + c'$,
where $c'$ is the center of the new circle.
The orientation of a plane circle does not change.
\end{lemma}
\Heading{Proof:}
Let $F$ be the circle and let $P$ be the plane $Ax+By+Cz+D=0$.
$\mbox{inv}(F)$ lies on the sphere $\mbox{inv}(P)$.
The center of $\mbox{inv}(P)$ is $\frac{-R^{2}}{2D} (A,B,C)$ 
(Lemma~\ref{cor-plane}).
Therefore, the orientation of $\mbox{inv}(F)$ is
$\frac{-R^{2}}{2D}(A,B,C) - c'$ (Lemma~\ref{lem:sphere}).
\QED

Note that $D$ is the distance of the circle's plane from the origin.

% *************************************************************

\subsection{Central inversion of a circle}

We have shown that the inversion and central inversion 
mappings are algebraically similar.
(Recall that central inversion is the mapping between the center of a circle
and the center of its inverse.)
One might conclude that the image of a curve under one mapping would be
similar to its image under the other mapping.
The following theorem compares the inverse of a plane circle with its central 
inverse, and shows that they are indeed similar: 
in special cases, the central inverse is an ellipse rather than a circle.

Recall the classical method for classifying conics.

\begin{lemma}
\label{lem-disc}
\cite[p. 140]{SALconic}
The {\em discriminant} of a conic $ax^{2} + bxz + cz^{2} + dx + ez + f$ is
$b^{2} - 4ac$.
% B^{2} - 4AC is definitely correct (I've checked)
% IF CORRECT TERM IS NOT DISCRIMINANT, 
% CHANGE THIS TERM WHEREVER IT OCCURS IN THE PAPER
A conic is an ellipse if $b^{2} - 4ac < 0$, 
a parabola if $b^{2} - 4ac = 0$, and a hyperbola if $b^{2} - 4ac > 0$.
\end{lemma}

% NOTE: DON'T WANT TO PROVE THE FOLLOWING FOR SPACE CIRCLE,
% BECAUSE YOU GET THREE EQUATIONS AND YOU WOULD NEED TO TAKE TWO RESULTANTS
% TO ELIMINATE TWO VARIABLES FROM THE THREE EQUATIONS
% TO SEE THIS: 
\Comment{
Let $D$ be a space circle.
By rotation, we can assume without loss of generality that 
$D$ lies in the $z=c$ plane.
In particular, let $D$ be the circle of radius
$r$ centered at $(a,b,c)$, which has the parameterization 
$(r\frac{1-t^{2}}{1+t^{2}} + a, r \frac{2t}{1+t^{2}} + b, c)$, $t \in \Re$
\cite{}.

By Lemma~\ref{lem-center} (after some simplification),
the central inversion of $D$ is
\[ 
\scriptsize{(x,y,z) = 
\frac{R^{2}}{(a^{2} + b^{2} + c^{2} + r^{2} - 2ar - k^{2})t^{2} + 4brt + 
	     (a^{2} + b^{2} + c^{2} + r^{2} + 2ar - k^{2})} 
	(r(1-t^{2}) + a(1+t^{2}), 2rt + b(1 + t^{2}), c(1+t^{2}))}
\]
Rearranging into three equations in $t$, we have
\[ [x(a^{2} + b^{2} + c^{2} + r^{2} - 2ar - k^{2}) - R^{2}(a-r)]t^{2} 
   + 4brxt 
   + (a^{2} + b^{2} + c^{2} + r^{2} + 2ar - k^{2})x - R^{2}(a + r) = 0 \]

\[ [y(a^{2} + b^{2} + c^{2} + r^{2} - 2ar - k^{2}) - R^{2}b]t^{2} 
   + (4bry - 2R^{2}r) t
   + (a^{2} + b^{2} + c^{2} + r^{2} + 2ar - k^{2})y - R^{2}b = 0 \]

\[ [z(a^{2} + b^{2} + c^{2} + r^{2} - 2ar - k^{2}) - R^{2}c]t^{2} 
   + 4brzt
   + (a^{2} + b^{2} + c^{2} + r^{2} + 2ar - k^{2})z - R^{2}c = 0 \]

Need to take Sylvester resultant twice and so it will probably be a quartic
curve.
}


\begin{theorem}
\label{thm-conic}
Let $S$ be the sphere of inversion of radius $R$ centered at the origin.
Let D be a line or a plane circle such that the origin
lies outside $D$ (if $D$ is a circle) and $0 \not \in D$ (if $D$ is a line).
% This is needed for the circle: since we use the fact that
% $k < \mbox{dist}(0,D) = \sqrt{a^{2} + b^{2}} - r$
The central inversion of $D$, with respect to a constant radius function
$k < \mbox{dist}(0,D)$, is an ellipse.\footnote{In this case, the value of 
	$\mbox{dist}(0,D)$ is simply 
	the distance from the origin to the closest point of $D$.
	We are not using the more sophisticated power distance metric, 
	which distinguishes between points inside and outside of $D$.}
\end{theorem}
\Heading{Proof:}
Let $D$ be a line.
We can assume without loss of generality that $D$
is vertical and lies in the $y=0$ plane, 
by rotating about the origin.
Let $D = (a,0,t)$, $t \in \Re$.
% which do not affect the shape
% 	Translate the center of inversion to the origin.
% 	Consider the plane containing the origin and the cylinder's axis.
% 	Rotate this plane, while maintaining contact with the origin,
% 	until it is vertical.
% 	Rotate vertical line about z-axis until it lies in the y=0 plane.

By Lemma~\ref{lem-center}, the central inversion of $D$ is
\begin{equation}
\label{eq1}
	(x,y,z) = \frac{R^{2}}{|a^{2} + t^{2} - k^{2}|} (a,0,t) 
		= \frac{R^{2}}{a^{2} + t^{2} - k^{2}} (a,0,t) 
\end{equation}
since $k < \mbox{dist}(0,D) = a$.
(Notice that without a restriction on the value of $k$, 
the absolute value could not be removed and calculations would be
more difficult.)
We shall implicitize this parameterization in order to determine its type.
Rearranging Equation~(\ref{eq1}) into two equations in $t$, we have
\[ xt^{2} + (a^{2}x - k^{2}x - R^{2}a) = 0 \]
\[ zt^{2} - R^{2}t + (a^{2} - k^{2})z = 0  \]
The implicit equation of this curve is found by eliminating $t$ from the
two equations, using the resultant.
% (The resultant $f(x,z)$ of the two equations is zero if and only if
% the two equations have the same $t$ root for the given values 
% of $x$ and $z$.)
The (Sylvester) resultant of the two equations with respect to $t$ is
\[ \left| \begin{array}{cccc}
	x & 0 & a^{2}x - k^{2}x - R^{2}a & 0 \\
	0 & x & 0 & a^{2}x - k^{2}x - R^{2}a \\
	z & -R^{2} & (a^{2} - k^{2})z & 0 \\
	0 & z & -R^{2} & (a^{2} - k^{2})z \\
	\end{array} \right| = 0 \]
Although it appears that this determinant is an equation of degree
four, it is actually of degree two because of cancellation:
\begin{equation}
\label{eq:lineconic}
a^{2} z^{2} + (a^{2} - k^{2}) x^{2} - R^{2} a x = 0 
\end{equation}
% To simplify, use symbolic constants in place of the two nontrivial 
% exps in the matrix.
% *****************
%
(\ref{eq:lineconic}) represents an ellipse, because
its discriminant is $- 4(a^{2} - k^{2})a^{2}$, which is 
negative since $k < \mbox{dist}(0,D) = a$.

Let $D$ be a plane circle.
By rotation, we can assume without loss of generality that 
$D$ lies in the $z=0$ plane.
In particular, let $D$ be the circle of radius
$r$ centered at $(a,b,0)$, which has the parameterization 
$(r\frac{1-t^{2}}{1+t^{2}} + a, r \frac{2t}{1+t^{2}} + b, 0)$, $t \in \Re$.

By Lemma~\ref{lem-center} (after some simplification),
the central inversion of $D$ is
\[ 
\scriptsize{(x,y,z) = 
\frac{R^{2}}{(a^{2} + b^{2} + r^{2} - 2ar - k^{2})t^{2} + 4brt + 
	     (a^{2} + b^{2} + r^{2} + 2ar - k^{2})} 
	(r(1-t^{2}) + a(1+t^{2}), 2rt + b(1 + t^{2}), 0)}
\]
Rearranging into two equations in $t$, we have
\[ [x(a^{2} + b^{2} + r^{2} - 2ar - k^{2}) - R^{2}(a-r)]t^{2} 
   + 4brxt 
   + (a^{2} + b^{2} + r^{2} + 2ar - k^{2})x - R^{2}(a + r) = 0 \]

\[ [y(a^{2} + b^{2} + r^{2} - 2ar - k^{2}) - R^{2}b]t^{2} 
   + (4bry - 2R^{2}r) t
   + (a^{2} + b^{2} + r^{2} + 2ar - k^{2})y - R^{2}b = 0 \]

The Sylvester resultant of these two equations with respect to $t$
is a conic equation: 
% Source: See Mathematica file `inverse-of-torus-centers'.
\[ [4a^{2}k^{2} - (a^{2} + b^{2} + k^{2} - r^{2})^{2}]\ x^{2} + \ \ 
[2R^{2}a(a^{2} + b^{2} - k^{2} - r^{2})]\ x + \ \ 
[8abk^{2}]\ xy + 
\]
\[
[4b^{2}k^{2} - (a^{2}+b^{2}+k^{2}-r^{2})^{2}]\ y^{2} + \ \ 
[2R^{2}b(a^{2} + b^{2} - r^{2} - k^{2})]\ y - \ \ 
[R^{4}(a^{2} + b^{2} - r^{2})]
\]
%
%
% COMPUTATION OF RESULTANT
% Let $\alpha = x(a^{2} + b^{2} + r^2 - 2ar - k^2) - R^{2}(a-r)$;
% $\beta = (a^{2} + b^{2} + r^2 + 2ar - k^2)x - R^{2}(a + r)$; 
% $\gamma = y(a^{2} + b^{2} + r^2 - 2ar - k^{2}) - R^{2}b$;
% and $\delta = (a^{2} + b^{2} + r^2 + 2ar - k^2)y - R^{2}b$.
%
% the (Sylvester) resultant of the two equations with respect to $t$ is
% \[ \left| \begin{array}{cccc}
% 	\alpha & 4brx & \beta & 0 \\
%	0 & \alpha & 4brx & \beta \\
%	\gamma & 4bry - 2R^{2}r & \delta & 0\\
%	0 & \gamma & 4bry - 2R^{2}r & \delta
%	\end{array} \right| = 0 \]
%
Thus, the central inversion of $D$ is a conic.
The discriminant of this conic is 
% $-4(a^{2} + b^{2} + k^{2} - r^{2})^{2} 
%   (a^{2} + b^{2} - k^{2} - r^{2} - 2kr)
%   (a^{2} + b^{2} - k^{2} - r^{2} + 2kr)$
\[ -4(a^{2} + b^{2} + k^{2} - r^{2})^{2} 
   ((a^{2} + b^{2} - k^{2} - r^{2})^{2} - 4k^{2}r^{2}) \]
% Source: See Mathematica file `inverse-of-torus-centers'.
Since $k < \mbox{dist}(0,D) = \sqrt{a^{2} + b^{2}} - r$ by assumption,
$a^{2} + b^{2} > (k+r)^{2}$.
Therefore, $(a^{2} + b^{2} + k^{2} - r^{2}) > 0$ and the sign
of the discriminant depends on the relationship between 
$(a^{2} + b^{2} - k^{2} - r^{2})^{2}$ and $4k^{2}r^{2}$,
or $(a^{2} + b^{2} - k^{2} - r^{2})$ and $2kr$,
or $(a^{2} + b^{2}) - (k + r)^{2}$ and $0$.
We conclude that the discriminant is negative and the conic is an ellipse.
\QED

\Comment{
PROBABLY ELIMINATE THIS COROLLARY (OR PUT IN FINAL SECTION ON COMPUTATION
OF CONIC OF CYCLIDE).

For future use, we need to extract the concrete relationship between 
the $D$ and its central inverse developed in the above proof.

\begin{corollary}
Let $C$ be as in Theorem~\ref{thm-conic}, with center at the origin.
Let the radius function be the constant $k$.
The central inverse of the line $(a,0,t)$ is the conic 
\[ a^{2} z^{2} + (a^{2} - k^{2}) x^{2} - R^{2} a x = 0 \]
The central inverse of the circle 
$(r\frac{1-t^{2}}{1+t^{2}} + a, r \frac{2t}{1+t^{2}} + b, 0)$
is the conic 
\[ [-(a^{2} + b^{2} - 2ak + k^{2} - r^{2})
     (a^{2} + b^{2} + 2ak + k^{2} - r^{2})]\ x^{2} + \ \ 
[2R^{2}a(a^{2} + b^{2} - k^{2} - r^{2})]\ x + \ \ 
[8abk^{2}]\ xy - \ \ \ \ 
\]
\[
[(a^{2}+b^{2}+k^{2}-r^{2}-2bk)(a^{2}+b^{2}+k^{2}-r^{2}+2bk)]\ y^{2} + \ \ 
[2R^{2}b(a^{2} + b^{2} - r^{2} - k^{2})]\ y - \ \ 
[R^{4}(a^{2} + b^{2} - r^{2})]
\]
\end{corollary}
}

\Comment{
\begin{theorem}
Let $S$ be the sphere of inversion of radius $R$ centered at the origin.
Let D be a line or a plane circle, $0 \not \in D$.
Let $f$ be the furthest point of $D$ from the origin.
The central inversion of $D$, with respect to a constant radius function
$k > \mbox{dist}(0,f)$, is an ?????.
\end{theorem}
\Heading{Proof:}
USE OTHER FORMULA FOR CENTRAL INVERSION.
THIS IS THE ONLY EXTENSION OF THE RESULTS THAT IS EASY.
\QED
}

\Comment{
\begin{lemma}
MAKE THE SAME ASSUMPTIONS AS THEOREM~\ref{thm-conic}.
The central inversion of $D$ is an ellipse (resp.,
parabola, hyperbola)
if and only if the distance of $D$ from the center of inversion $c$
is greater than (resp., equal to, less than) the constant radius.
(ellipse if and only if center of inversion lies outside CYLINDER or TORUS.)
NOT QUITE TRUE IN EITHER CASE. SEE PROOF (e.g., parabola if line contains
center of inversion).
\end{lemma}
\Heading{Proof:}
What type of conic is it?
Its discriminant is $- 4(a^{2} - k^{2})a^{2}$.
Its sign depends on the sign of $a^{2} - k^{2}$.
If the line intersects the center of inversion ($a = 0$),
then the conic is a parabola.
Otherwise, if $a$ (the distance of the line $D$ from the center of inversion) 
is greater than $k$ (the constant radius function), then the conic
is an ellipse, and so on.

CIRCLE:
The discriminant of this conic is 
% $-4(a^{2} + b^{2} + k^{2} - r^{2})^{2} 
%   (a^{2} + b^{2} - k^{2} - r^{2} - 2kr)
%   (a^{2} + b^{2} - k^{2} - r^{2} + 2kr)$
$-4(a^{2} + b^{2} + k^{2} - r^{2})^{2} 
   ((a^{2} + b^{2} - k^{2} - r^{2})^{2} - 4k^{2}r^{2})$.
% Source: See Mathematica file `inverse-of-torus-centers'.
If $(a^{2} + b^{2} + k^{2} - r^{2}) = 0$, then the conic is a parabola.
That is, if $a^{2} + b^{2}$, the square of the 
distance of the center of the circle from the center of inversion, 
is equal to $r^{2} - k^{2}$, then the conic is a parabola.
% r+k is the distance of the center of the torus to the farthest point
% of the torus, r-k the distance to the closest point of the torus
% and r^{2} - k^{2} = (r+k)(r-k).
Otherwise, the sign depends on the relationship between 
$(a^{2} + b^{2} - k^{2} - r^{2})^{2}$ and $4k^{2}r^{2}$,
or $(a^{2} + b^{2} - k^{2} - r^{2})$ and $2kr$,
or $(a^{2} + b^{2}) - (k + r)^{2}$ and $0$,
or $\sqrt{a^{2} + b^{2}}$ and $k + r$.
$\sqrt{a^{2} + b^{2}}$ is the distance of the center of the circle
from the center of inversion,
while $k+r$ is the ``width of the torus, from its center to its outside''.
(STATE LEMMA FOR TORUS, RATHER THAN ABSTRACTLY FOR CIRCLE?
OR MAKE CRITERION FOR TYPE OF CONIC MORE ABSTRACT (THAT IS, NO NATURAL
INTERPRETATION OF $K+R$)?)
Thus, the sign depends on the relationship
between the distance to the center of inversion and the width of the torus:
if the center of inversion lies outside (resp., on, inside) the torus,
then the conic is a ellipse (resp., parabola, hyperbola).
\QED
}

% ************************************************************

\section{Dupin cyclides}
\label{sec-Dupin}

% INTRO TO CYCLIDES

Dupin cyclides are a very interesting subclass of quartic surfaces.
The Dupin cyclide is most easily understood as a generalization
of the torus.
%\footnote{There
%  	is some similarity between the generalization of quadrics
%	to superquadrics and the generalization of tori to cyclides.}
% 
% picture of squashed, horn, double-horn from Hilbert
%
Although originally defined as the envelope of spheres tangent to 
three given spheres, it can also be defined as a special class of 
quartic surfaces.
%   \cite{try to avoid using Forsythe; DePont; want different
%	primary reference}.
The Dupin cyclide is particularly interesting because it is the only
surface whose lines of curvature are all circles.
% \footnote{It is also
% 	the only surface whose two evolutes are both curves,
%	rather than surfaces \cite{Hilbert}.}
% Hilbert: Geometry and the Imagination
We note that a more general definition of cyclide has been introduced
(a cubic containing the circle at infinity or a quartic that contains
the circle at infinity as a double line)
%  \cite{Forsythe: try to avoid;
% also Gerd Fischer I think})
but we are interested in Dupin's original definition of cyclide.
In this paper, cyclide refers to Dupin cyclides.

There is much work using cyclides in the solid modeling literature
\cite{CDH89,DEP84,MAR82,PRA89,SHAR85}.
For example, because of the circular lines of curvature property, 
Martin and other researchers at Cambridge University have used Dupin cyclides
as the primary surface in a surface modeler 
based around principal patches 
(patches whose boundaries are lines of curvature) \cite{DEP84,MAR82,SHAR85}.
Chandru et. al. \cite{CDH89} and Pratt \cite{PRA89} 
have used them for blending.
% They are also useful because they contain the widely used torus as 
% a special case, and they are closed under offset.


% ********************************************************

A torus can be generated by sweeping a circle about a circle, 
and a quadric surface can be generated by sweeping a circle along a line
\cite{JS90a}.
We want to show how a cyclide can be generated by sweeping a circle.
The cyclide can certainly be generated by circles, 
since its lines of curvature are circular.
The following fact is very helpful.

\begin{lemma}
\label{invert-cyclide}
\cite{DEP84,H52}
% De Pont pp. 36-37
Any cyclide can be inverted into a torus, circular cylinder, or cone.
\end{lemma}

The structure of our method is shown in Figure~\ref{fig:box}
(which ignores degenerate cases).
% 
\begin{figure}
\label{fig:box}
\[
\begin{array}{ccc}
\mbox{torus}	&   \leftarrow   & \mbox{cyclide} \\
\downarrow \\
\mbox{circle}	&   \rightarrow   & \mbox{conic}
\end{array}
\]
\caption{Decomposition of a cyclide to a conic}
\end{figure}
%
The cyclide is inverted into a torus, 
which is easily expressed as a circle sweeping about a circle $D$.
By inverting back, the cyclide is expressed as a circle sweeping
about a conic, where the conic is the central inverse of $D$.
The radius and orientation of the sweeping circle are easily 
determined from the radius and orientation of the sweeping circle for
the torus, using Lemmas~\ref{lem-radius1}, \ref{lem-radius2} and 
\ref{thm:orient}.

We need to show how to invert a cyclide into a torus.
Since we are only interested in finding the circle $D$, 
not the entire torus, it is easy to invert the cyclide once we
know the center of inversion.
Theorem~\ref{lem:center} shows how to choose the center of inversion.

We also need to show that we can compute the central inverse of the circle
$D$, using Theorem~\ref{thm-conic}.
That is, we need to show that $D$ is a plane circle and that the center of
inversion lies outside the torus (to fulfill the `$k < \mbox{dist}(0,D)$'
criterion).
Theorem~\ref{lem:center} establishes these facts.

\begin{theorem}
\label{lem:center}
Any cyclide can be inverted into a torus, circular cylinder, or cone $T$
using a center of inversion that is
\begin{enumerate}
\item
	outside $T$
\item
	coplanar with $D$
\end{enumerate}
where $T$ is generated by sweeping a circle 
of constant radius about the circle $D$.
\end{theorem}
\Heading{Proof:}
In order to prove this result, we need to develop some theory of cyclides.
A Dupin cyclide can be represented by the implicit equation
\[ 
(x^{2} + y^{2} + z^{2} - u^{2} + b^{2})^{2} = 4(ax-cu)^{2} + 4b^{2}y^{2}
\] 
where $a,b,c,u > 0$ and $a^{2} = b^{2} + c^{2}$.
We can classify the cyclide into double-horned ($u < c$), 
single-horned ($u=c$), normal ($c < u \leq a$) and self-intersecting
($u > a$) cyclides.

The proof for horned cyclides is simple.
One of the horns is chosen as the center of inversion.
The cyclide inverts into a cone (if it is double-horned) or a cylinder
(if it is single-horned), because all circles of the cyclide pass through
the horn and thus invert to lines \cite{DEP84}.
Since all circles of the cyclide lie on the same side of the horn,
all circles invert to lines on the same side of the horn and, in particular,
the horn lies outside $T$.
Also, the horn trivially lies in the same plane as the line of centers of the
cone or cylinder.

Consider the remaining cyclides ($u > c$).
De Pont \cite[pp. 36-37]{DEP84} shows that if the point 
$(\frac{ua + b\sqrt{u^{2} - c^{2}}}{c},0,0)$ is used as the center of 
inversion, then the cyclide inverts into a torus.
% 
% The torus will be self-intersecting for the self-intersecting cyclide.
%
\Comment{
% Hilbert has the result that center of inversion can be chosen in the plane 
% of the curve of centers. 
% See pp. 36-37 of dePont.  Curve of centers of the torus lies in the
% z=0 plane.  (The two planes of symmetry are the y=0 and z=0 planes (p. 32))
Inv$_{\mbox{point of one of horns}}$ (twohorned cyclide) = cone 
(vertex is other horn)

Inv$_{\mbox{point of only horn}}$ (onehorned cyclide) = circular cylinder

Inv$_{\mbox{special point in plane of curve of centers}}$ 
	(squashed torus cyclide) = torus

Inv$_{\mbox{self-intersecting point}}$ (self-intersecting cyclide) = cone

Inv (limit case of squashed torus: touching) = torus with zero radius hole
}
%
We will show that all of the cyclide lies to the same side of the center of
inversion.
Consider the plane of symmetry $z=0$, which intersects the cyclide in 
two circles.
With some computation, it can be shown that these two circles are 
$(x+c)^{2} + y^{2} = (a+u)^{2}$ and $(x-c)^{2} + y^{2} = (a-u)^{2}$.
The center of inversion lies in the plane $z=0$.
We want to show that it lies to the right of both the circles, or
$\frac{ua + b\sqrt{u^{2} - c^{2}}}{c} > c + (a-u)$.
This is true because:
\[ ua + b\sqrt{u^{2} - c^{2}} - c^{2} 
 > ca + b\sqrt{u^{2} - c^{2}} - cu
 > ca - cu = c(a-u) \]
Thus, all of the cyclide lies to the same side of the center of inversion.
Thus, all of the inverse of the cyclide also lies to the same side of the
center of inversion.
In particular, the center of inversion lies outside $T$.

Since $z=0$ is a plane of symmetry for the cyclide
and the center of inversion lies in this plane, 
$z=0$ remains a plane of symmetry for the inverse of the cyclide $T$.
In particular, $z=0$ contains the centers of the circles that sweep out $T$.
That is, $D$ lies in $z=0$ and is a plane circle.
\QED

We have almost established that a cyclide is a circle sweeping along a conic.
Consider Figure~3.
% HAD TO FORCE 3 BECAUSE IT INSISTED ON OUTPUTTING 4.
Theorem~\ref{lem:center} establishes that we can perform the last step: the
map from the circle to conic.
The conic is the central inverse of the circle, and it can be determined
using Theorem~\ref{thm-conic}.
We have only ignored one case.
Suppose the cyclide is inverted to a cone, which is a circle sweeping
along a line $L$ with varying radius.
Since we cannot apply Theorem~\ref{thm-conic} to find the central inversion
of $L$, we need the following result.

\begin{lemma}
Let $S$ be the sphere of inversion of radius $R$ centered at the origin.
Let $C$ be a cone, $0 \not \in C$, $0 \not \in \mbox{inside}(C)$.
% such that the origin lies strictly outside of the cone.
The centers of the circles of $C$ lie on a line $L(t)$
and the radius of the circle at $L(t)$ is $|kt|$, for some constant $k$.
The central inversion of the line $L(t)$
with respect to the radius function $r(t) = |kt|$ is a conic.
\end{lemma}
\Heading{Proof:}
Assume without loss of generality that the center of inversion is the origin
and $L(t)$ is $(a,0,t)$.
The central inversion of $L(t)$ is
\[
	(x,y,z) = \frac{R^{2}}{a^{2} + t^{2} - k^{2}t^{2}} (a,0,t) 
\]
Rearranging into two equations in $t$, we have
\[ (1-k^{2})xt^{2} + (a^{2}x - R^{2}a) = 0 \]
\[ (1-k^{2})zt^{2} - R^{2}t  + a^{2}z  = 0  \]
The Sylvester resultant of these two equations with respect to $t$ is
\[ \left| \begin{array}{cccc}
	(1-k^{2})x & 0 & a^{2}x - R^{2}a & 0 \\
	0 & (1-k^{2})x & 0 & a^{2}x - R^{2}a \\
	(1-k^{2})z & -R^{2} & a^{2}z & 0 \\
	0 & (1-k^{2})z & -R^{2} & a^{2}z \\
	\end{array} \right| = 0 \]
\[ (k^{2} - 1)(R^{2}x - ax^{2} + a(k^{2} - 1)z^{2})  = 0 \]
Thus, the central inversion of $D$ is a conic.
Since the discriminant of this conic is $4a^{2}(k^{2} - 1)$,
the type of the conic depends on $k$. % the slope of the cone
For example, if the cone's angle of inclination is exactly 45 degrees,
then the conic is a parabola.
\QED

\ifFull{
ONLY FOR JOURNAL VERSION
Although: 

\begin{lemma}
The central inversion of a line
with respect to a fully general linear radius function is a quartic.
\end{lemma}
\Heading{Proof:}
If $k$ is set to $mt+n$ in (\ref{eq1}), the central inverse that results is
a quartic curve.
% Source: Mathematica file `linear-radius-function'
% The two equations are xt^{2} + a^{2}x - (mt+n)^{2} x - R^{2}a = 0
% and 			zt^{2} - R^{2}t + (a^{2} - (mt+n)^{2})z = 0.
% We compute the Sylvester resultant of them with respect to t.
\QED
\fi

% ****************************************

\begin{theorem}
Any cyclide can be generated by sweeping a circle along a conic.
Indeed, other than double-horned cyclides, any cyclide can be generated by 
sweeping a circle along an ellipse.
\end{theorem}
\Heading{Proof:}
A cyclide can be inverted to a torus or a cylinder
(not a cone) if it is not a double-horned cyclide.
(See the proof of Theorem~\ref{lem:center}.)
\QED

% BEGINNING OF BIG COMMENT:  SOME OF THIS MATERIAL WOULD MAKE A GOOD EXAMPLE
% 	FOR THE JOURNAL PAPER

\Comment{
\begin{example}
Example of (a) finding torus from cyclide (once you know center of inversion)
and (b) finding central inverse of circle of torus (use central inversion of
parameterization of circle).
\end{example}


% ********************************************************

\section{From a cyclide to a circle}

Here is the structure of the cyclide->conic decomposition:
\[ 
\mbox{cyclide} \rightarrow \mbox{inverse of cyclide} = 
\mbox{torus/cylinder/cone} \rightarrow \mbox{circle/line (curve of centers)}
\rightarrow \mbox{conic} \]
Earlier sections have shown how to perform the last step of central inversion
from a circle to a conic.
The other important step, the intermediate translation of a cyclide to a circle
(where the circle represents a torus) is described in this section, using
the inversion of a cyclide to a torus that has been studied by others.

We want to invert the cyclide into a torus, cylinder, or cone.
Finding the inverse by applying the inversion mapping is difficult.
A key observation is that to fully define the torus, it is sufficient 
to look at a plane of symmetry.
For example, the major diameter, minor diameter, and center of a torus
in normal form are fully defined by the two circles in the plane $y=0$ 
(Figure~\ref{fig:circles}).
Consider the $y=0$ section of the cyclide, which consists of two circles.
It is possible to choose a center of inversion in the $y=0$ plane such that 
the resulting inverse of the cyclide is a torus (Theorem~\ref{thm:depont}).
The inverse of the two circles in the $y=0$ plane is a pair of circles that
define the torus (Theorem~\ref{thm:twocircles}).

This allows us to change the structure of the decomposition:
%
\[ \mbox{cyclide} \rightarrow \mbox{plane of symmetry of cyclide}
\rightarrow \mbox{circle/line} \rightarrow \mbox{conic} \]
%
That is, we aren't really interested in the torus, cylinder, or cone:
we are interested in its curve of centers.
Therefore, we will not find the inverse of the cyclide,
we shall find the inverse of a plane of symmetry of the cyclide
and determine the curve of centers directly from this.

\figg{fig:circles}{(a) Two circles of cyclide and (b) 
		       two circles of torus}{1in}


GIVE IDEA OF METHOD FOR EACH TYPE OF CYCLIDE.
(1) `squashed torus cyclide': find two circles on cyclide, invert based on 
DePont's center of inversion, use results on center and radius of inverse 
of a circle to determine two circles on torus; these define torus.

(2) `horned cyclide': find point and circle on plane of symmetry of cyclide,
invert circle w.r.t. horn, the new circle defines cylinder.

(3) ...

% ************************************

EXAMPLE OF THIS METHOD ON SQUASHED TORUS CYCLIDE (USING 1ST NORMAL FORM FOR 
CYCLIDE):

\begin{theorem}
\label{thm:depont}
\cite{DePont}
If the center of inversion is 
$(\frac{\mu a + b\sqrt{\mu^{2} - c^{2}}}{c},0,0)$,
then the cyclide (in normal form) 
\[ 
(x^{2} + y^{2} + z^{2} - \mu^{2} + b^{2})^{2} = 4(ax - c\mu)^{2} + 4b^{2}y^{2}
\]
is mapped to a torus by inversion.
In particular, this inversion maps the two circles of the cyclide in the
$y=0$ plane into two circles of equal radius in the $y=0$ plane.
\end{theorem}

We can apply our results on the center and radius of the inverse of a circle
to determine the size and position of the two circles of equal radius that
define the torus.

\begin{theorem}
\label{thm:twocircles}
If the center of inversion is ---, then the cyclide (in normal form) ---
is mapped to a torus with 
major diameter ---, 
minor diameter ---,
and center ---.
\end{theorem}



\begin{theorem}
The cyclide 
\[ 
(x^{2} + y^{2} + z^{2} - \mu^{2} + b^{2})^{2} = 4(ax - c\mu)^{2} + 4b^{2}y^{2}
\]
is the inverse of the torus
with center $(-ka-\frac{kb\mu}{\sqrt{\mu^{2} - c^{2}}},0,0)$,
major radius $\frac{kbc}{\sqrt{\mu^{2} - c^{2}}}$
(where $k = \frac{R^{2}c}{2b(b\mu + a\sqrt{\mu^{2} - c^{2}})}$),
minor radius $2b^{2}\mu + 2ab\sqrt{\mu^{2} - c^{2}}$,
and the curve of centers of the torus lies in the $z = 0$ plane.
Note: major radius is $\frac{bR^{2}c^{2}}
			    {\mbox{minor radius}\sqrt{\mu^{2}-c^{2}}}$.
\end{theorem}
}
% END OF BIG COMMENT

% ************************************************

% DISCUSSION OF USES OF REPRESENTATION IN ALGORITHMS

% Line intersection?

% Plane intersection?

\section{Conclusions}
\label{sec-con}

We have given a constructive method for expressing a cyclide
as a circle sweeping along a conic.
A cyclide can be viewed as a swept surface,
like a ruled surface, except that a line sweeps out a ruled surface
while a circle sweeps out a cyclide.

This type of decomposition is convenient, essentially because 
the treatment of curves, especially simple curves like lines and circles,
is much easier than the treatment of surfaces.
For example, Levin reduces the intersection of quadrics to the intersection
of a parameterized line with a quadric, 
by finding a ruled surface in the pencil of two quadric surfaces \cite{LEVI76}.
In the same way, the intersection of a cyclide
with another surface can now be reduced 
to the intersection of a circle with that surface.
As another example, the display of a cyclide now reduces to the 
display of a circle.
% [Essentially, the $t^{th}$ point of the intersection curve
% is the intersection of the $t^{th}$ circle and the surface.]
% ELABORATE

We have also thoroughly examined the inversion of a circle, yielding
computational results and interesting relationships between the inversion
and central inversion of a point.


\Comment{
THIS ISN'T A REALISTIC APPLICATION
Dupin cyclides are a very useful class of quartic surfaces 
(see Section~\ref{sec-Dupin}).
As such, the Dupin cyclide is a good example of a complex surface that we would
like to add to the vocabulary of a solid modeler.
This paper suggests a way in which we can introduce Dupin cyclides
in terms of curves and surfaces that are already in the vocabulary:
circles and conics.
This is a form of bootstrapping.
We believe that this is a good way of introducing complicated surfaces 
into the working language of solid modeling.
}

\Comment{
This paper shows that one can use a known, simple surface
to determine unknown facts about a more complex surface.
In this case, the known surface is a cylinder, cone, or torus,
the more complex surface is a Dupin cyclide, and inversion is the tool.
This is one way to introduce complicated surfaces
into the working language of solid modeling.

*We shall use the idea of computing properties of a complex surface through
mapping and computing properties of a simple surface, then mapping back
to the original surface.*
A Dupin cyclide will be mapped to a simple surface by inversion,
this simple surface will be expressed as a product,
and this product will then be inverted back to determine the cyclide's product.
}

\bibliography{/users1/jj/bib/modeling}

\end{document}

