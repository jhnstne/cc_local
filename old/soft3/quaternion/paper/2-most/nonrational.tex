\documentstyle[11pt]{article}
\newif\ifFull
\Fullfalse
\input{header}
\SingleSpace

\setlength{\oddsidemargin}{0pt}
\setlength{\topmargin}{0pt}
\setlength{\headsep}{3em}
\setlength{\textheight}{8.75in}
\setlength{\textwidth}{6.5in}
\setlength{\columnsep}{5mm}		% width of gutter between columns

\markright{Nonrational; parameterizations; quadratic spherical maps: \today \hfill}
\pagestyle{myheadings}

% -----------------------------------------------------------------------------

\title{Nonrational maps to the sphere;\\parameterizations of the sphere; \\quadratic spherical maps}
\author{J.K. Johnstone}

\begin{document}
\maketitle

\begin{abstract}
We review the classical {\em nonrational} maps from the sphere,
parameterizations of the sphere in arbitrary dimensions, 
and quadratic spherical maps.
\end{abstract}

\noindent Keywords: Maps to the sphere, parameterizations of the sphere, 
quadratic spherical maps.

\clearpage

\section{Nonrational maps to and from the sphere}

% All of the following are taken from Kreyszig, p. 204ff.

In 3-space, stereographic projection is a map from \Sn{2}\ to $\Re^2$.
Maps from the 2-sphere to the plane have been thoroughly studied
because of the importance of this problem to cartography,
and there are many other well-known examples \cite{kreyszig59}.
These can be decomposed into conformal maps (which preserve angles between 
intersecting curves on the surface) and equiareal maps (which preserve 
the area of regions on the surface).
Each could be inverted to yield a map to \Sn{2}.
All of these maps use the well-known parameterization of \Sn{2}:
\[
	(\cos u \cos v,\ \cos u \sin v,\ \sin u) \hspace{1in} 
	u \in [-\frac{\pi}{2}, \frac{\pi}{2}),\ v \in [0,2\pi)
\]

The conformal maps are stereographic projection and Mercator projection.
We have already discussed stereographic projection.
Note that there is a variant of stereographic projection where
the projection plane $z=0$ is replaced by the tangent plane opposite the pole
$z=-1$ \cite{kreyszig59}.
{\bf Mercator projection} is defined by:
\[
	(\cos u \cos v,\ \cos u \sin v,\ \sin u) \mapsto
	(v,\ \log \tan (\frac{u}{2} + \frac{\pi}{4}))
\]
where $u \in [-\frac{\pi}{2}, \frac{\pi}{2})$ and $v \in [0,2\pi)$.
Like stereographic projection,
Mercator projection neglects to map the north pole of the sphere 
($u = \frac{\pi}{2}$).
Stereographic projection is the only conformal map of \Sn{2}\ to a plane
that preserves circles (i.e., maps circles to circles or straight lines).
Like any conformal map of \Sn{2}\ to the plane,
these projections map loxodromes (curves of 
constant direction on the sphere) to straight lines.

\begin{rmk}
The stereographic projection was known to (and probably discovered by)
Hipparch in 160 B.C. \cite{kreyszig59}. % p. 205
The Mercator projection was invented by Mercator in 1569
and a mathematical formula was first given by Halley in 1695. % p. 207
\end{rmk}

The equiareal maps are the Lambert projection, mapping of Sanson,
and mapping of Bonne.
In all of the following maps, 
$u \in [-\frac{\pi}{2}, \frac{\pi}{2}]$ and $v \in (-\pi,\pi]$.
The {\bf Lambert projection} is:
\[
	(\cos u \cos v, \cos u \sin v, \sin u) \mapsto
	(v, \sin u)
\]
The {\bf mapping of Sanson} is:
\[
	(\cos u \cos v, \cos u \sin v, \sin u) \mapsto
	(v \cos u, u)
\]
The {\bf mapping of Bonne} is:
\[
	(\cos u \cos v, \cos u \sin v, \sin u) \mapsto
	r = \frac{\pi}{2} - u, \alpha = \frac{v \cos u}{\frac{\pi}{2} - u}
\]
where $(r,\alpha)$ are polar coordinates.
The Lambert projection has a geometric interpretation:
consider the cylinder $C$ tangent to the sphere along the equator $z=0$.
The Lambert projection is realized by casting rays from $C$'s axis
onto $C$ along lines parallel to the equatorial plane $z=0$,
and then unrolling the cylinder by cutting along the line associated
with the points $v = \pi$ on the sphere.

A disadvantage of all of these maps, except stereographic projection,
is their nonrationality.

\section{Parameterization of \Sn{n}}

In dealing with \Sn{n}, a parameterization is helpful.
Parameterizations of \Sn{1}\ and \Sn{2}\ are well known.
The two popular parameterizations for \Sn{1}\ are
\[
	(\cos t, \sin t) \hspace{1in}	t \in [0,2\pi]
\]
and 
\[
	(\frac{1-t^2}{1+t^2}, \frac{2t}{1+t^2}) \hspace{1in}
		 t \in (-\infty,\infty)
\]
The former is appreciated for being arc-length, 
and the latter for being rational.
A popular parameterization for \Sn{2}\ is
\[
	(\cos u \cos v, \cos u \sin v, \sin u) \hspace{.5in} 
	u \in [-\frac{\pi}{2}, \frac{\pi}{2}],\ v \in [0,2\pi]
\]
As observed by Barr (Superquadrics paper, 1981), this parameterization can be built
from the \Sn{1}\ parameterization using spherical product.
%
\begin{defn2}
The {\bf spherical product} of two plane curves 
$C_1(t_1) = (x_1(t_1), y_1(t_1))$, $t_1 \in I_1$ and 
$C_2(t_2) = (x_2(t_2), y_2(t_2)),\ t_2 \in I_2$ is the surface
\[
S(t_1,t_2) = (x_1(t_1) x_2(t_2), x_1(t_1) y_2(t_2), y_1(t_1)),\ 
t_1 \in I_1,\ t_2 \in I_2
\]
{\bf Spherical product must be a mathematical construct: look in Barr's references
such as Wallace's topology.}
\end{defn2}
%
Thus, \Sn{2}'s parameterization $(\cos u \cos v, \cos u \sin v, \sin u),\ 
u \in [-\frac{\pi}{2}, \frac{\pi}{2}],\ v \in [0,2\pi]$
is the spherical product of the semicircle
$(\cos u, \sin u),\ u \in [-\frac{\pi}{2}, \frac{\pi}{2}]$ and the circle
$(\cos v, \sin v),\ v \in [0,2\pi]$ (Figure~\ref{fig:sphericalproduct}).
The spherical product can be interpreted geometrically as 
sweeping $C_2$ along the $x_3$-axis 
at a speed controlled by the second component of $C_1$ 
and a scale controlled by the first component of $C_1$ (Figure~\ref{fig:sphericalproduct}).

\begin{figure}
\label{fig:sphericalproduct}
\vspace{2in}
\caption{Spherical product}
\end{figure}

We would like to point out that, using a generalized form of spherical
product, a parameterization of \Sn{n}\ 
can always be developed from a parameterization of \Sn{n-1}.
The spherical product is a natural mechanism for defining \Sn{n}
since \Sn{n}\ can be interpreted geometrically as a sweeping of \Sn{n-1}.

PUNCH LINE: \Sn{n} = spherical-product (half-circle, \Sn{n-1}).

For example, a parameterization of \Sn{3}, in terms of \Sn{2},
is 
\[
	(\cos u \cos v \cos w, \cos u \cos v \sin w, \cos u \sin v, \sin u).
\]

\begin{rmk}
Stereographic injection induces the standard parameterization on \Sn{n},
with pole at $(1,0,\ldots,0)$:
\begin{equation}
\label{eq:param}
\frac{1}{t_1^2 + \cdots + t_n^2 + 1} 
	(t_1^2 + \cdots + t_n^2 - 1, 2t_1, \ldots, 2t_n)
	\hspace{.5in}
	t_i \in (-\infty,\infty)
\end{equation}
\end{rmk}

\section{A proof of the necessary condition using stereographic injection is flawed}

\begin{lemma}[Ammei]	% c. 1817
\label{lem:ammei}
\begin{equation}
\label{eq:ammei}
	(a_1^2 + \cdots + a_n^2)^2 = 
	(a_1^2 + \cdots + a_{n-1}^2 - a_n^2)^2 + (2a_1a_n)^2 + \ldots + 
	(2a_{n-1}a_n)^2
\end{equation}
where $a_1,\ldots,a_n$ are elements of any commutative ring, $n \geq 2$.
\end{lemma}

\begin{defn2}
{\rm 
{\bf Stereographic injection} is the map 
$\sigma^{-1}_{(1,0,\ldots,0)}: x_{1}=0 \rightarrow \Sn{n} - (1,0,\ldots,0)$:
\begin{equation}
\label{eq:injection}
(0,x_1,\ldots,x_n) \mapsto
	\frac{1}{x_1^2 + \cdots + x_n^2 + 1} 
	(x_1^2 + \cdots + x_n^2 - 1, 2x_1, \ldots, 2x_n)
\end{equation}
}
\end{defn2}

\begin{theorem}
\label{thm:necessaryn}
Let $n \geq 2$ and $D$ be the integers or the polynomials 
over $\Re[x_1,\ldots,x_n]$.
A Pythagorean $(n+1)$-tuple over $D$ can be expressed in the form
\begin{equation}
\label{eq:pyth2}
	\alpha (a_1^2 + \ldots + a_{n-1}^2 - a_n^2,
		\ 2a_1a_n,\ldots,\ 2a_{n-1}a_n,
		\ a_1^2 + \ldots + a_n^2)
\end{equation}
for some $a_1,\ldots,a_n,\frac{1}{\alpha} \in D$.
\end{theorem}

There is a similarity between stereographic injection (\ref{eq:injection})
and (\ref{eq:ammei}). % (see Example~\ref{eg:rel})
It is tempting to use this relationship to 
attempt a proof of Theorem~\ref{thm:necessaryn} for integers.
However, this approach is flawed.
Consider a Pythagorean $(n+1)$-tuple $(x_1,\ldots,x_{n+1})$ over the
integers ${\cal Z}$.
The associated point 
$(\frac{x_1}{x_{n+1}},\ldots,\frac{x_n}{x_{n+1}})$ lies on the sphere \Sn{n-1}\ 
and can be represented by the parameterization of \Sn{n-1}\ 
(see (\ref{eq:param})):
$(\frac{x_1}{x_{n+1}},\ldots,\frac{x_n}{x_{n+1}}) = 
\frac{1}{t_1^2 + \ldots + t_{n_1}^2 + 1} 
	(t_1^2 + \ldots + t_{n-1}^2 - 1, 2t_1, \ldots, 2t_{n-1})$
or $(x_1,\ldots,x_{n+1}) = 
(t_1^2 + \ldots + t_{n-1}^2 - 1, 2t_1, \ldots, 2t_{n-1}, 
t_1^2 + \ldots + t_{n_1}^2 + 1)$.
This is in the desired format (\ref{eq:pyth2}) (setting $a_n=1$) except
for one very important detail:
$t_1,\ldots,t_{n-1} \in \Re$, not ${\cal Z}$ as required by 
Theorem~\ref{thm:necessaryn}.

In fact, it is enough to show that $t_1,\ldots,t_{n-1} \in {\cal Q}$\
(where ${\cal Q}$ is the rational numbers),
but this is not possible either.
The fact that 
$(\frac{x_1}{x_{n+1}},\ldots,\frac{x_n}{x_{n+1}}) \in {\cal Q}^4$
is not enough to establish this,
since it merely says that the unit vector in the direction 
$v = (t_1^2 + \cdots + t_{n-1}^2 - 1, 2t_1, \ldots, 2t_{n-1})$ is rational,
which says nothing about the original vector $v$.
Thus, this proof technique, although tantalizing, is flawed.

This shows that Theorem~\ref{thm:necessaryn} actually contains a much 
stronger result than we might at first think.
The strength of the result lies in the fact that $a_i \in D$,
the same ring as the domain of the Pythagorean tuple.

\section{Quadratic spherical maps}
% All of this taken from Ono, Chapter 5.

Quadratic spherical maps are a class of maps from the $m$-sphere to the
$n$-sphere % that have received some attention
(see \cite{ono94}).	% Ono, Chapter 5; Ebbinghaus, p. 236
% especially in the context
% of Hopf maps, which are quadratic spherical maps with some additional properties

\begin{defn2}
A map $f:\Re^m \rightarrow \Re^n$ is {\bf quadratic} if 
\begin{quote}
1) $f(kx) = k^2 f(x)$ whenever $k \in \Re,\ x \in \Re^m$, and \\
2) the map $(x,y) \mapsto \frac{1}{2} [f(x+y) - f(x) - f(y)],\ x,y\in\Re^m$
is bilinear.
	% restricting K to \Re, X to \Re^m, and Y to \Re^n in Ono, p. 165
\end{quote}
\end{defn2}

\begin{defn2}
A quadratic map $f:\Re^m \rightarrow \Re^n$ is {\bf spherical} 
if $\| f(x) \| = \| x \|^2$ for all $x \in \Re^m$.
	% restricting quadratic forms q_x and q_y to squared Euclidean norms
	% x_1^2 + ... + x_m^2 (or x_n^2)
	% which strictly translates into \| f(x) \|^2 = \| x \|^4,
	% which is simplified to above form
\end{defn2}

If $f:\Re^{m} \rightarrow \Re^{n}$ is a quadratic spherical map, 
then $f:\Sn{m-1} \rightarrow \Sn{n-1}$.
Thus, quadratic spherical maps send the unit sphere into the unit sphere,
and hence their name.

Two other classical quadratic spherical maps are 
$f:\Sn{3} \rightarrow \Sn{2}$ defined by:	% \cite[pp. 171,174]{ono94}
\begin{equation}
\label{eqn:classicalHopf}
	(x_1,x_2,x_3,x_4) \mapsto 
	(x_1^2 + x_2^2 - x_3^2 - x_4^2,\ 
	 2(x_1 x_3 + x_2 x_4),\ 
	 2(x_1 x_4 + x_2 x_3))
\end{equation}
and $f:\Sn{n} \rightarrow \Sn{n}$ defined by: % \cite[p. 193]{ono94}
\begin{equation}
\label{eqn:HopfM}
	(w,x_1,\ldots,x_{n}) \mapsto
	(w^2 - x_1^2 - \cdots - x_{n}^2,\ 
	 2wx_1, \ldots,\ 2wx_{n})
\end{equation}
%
% All of these maps are also Hopf maps.
% (\ref{eqn:classicalHopf}) is called the classical Hopf map.
% Notice the similarity between (\ref{eqn:HopfM})
% and the projection version of stereographic injection 
% (\ref{eqn:stereoinjection}).

Recall a normal form for rational maps to \Sn{2}.

\begin{theorem}
\label{thm:n3}
A map is a rational map of $\Re^3$ to \Sn{2}\ if and only if
it is of the form:
\[
(x_{\pi(1)},x_{\pi(2)},x_{\pi(3)}) \mapsto (\frac{a_1^2 + a_2^2 - a_3^2 - a_4^2}{a_1^2 + a_2^2 + a_3^2 + a_4^2},
		       \frac{2a_1a_3 + 2a_2a_4}{a_1^2 + a_2^2 + a_3^2 + a_4^2},
		       \frac{2a_1a_4 - 2a_2a_3}{a_1^2 + a_2^2 + a_3^2 + a_4^2})
\]
where $a_1,a_2,a_3,a_4 \in \Re[x_1,x_2,x_3]$
and $\pi : \{1,2,3\} \rightarrow \{1,2,3\}$ is a permutation.
\end{theorem}

\begin{example}
The normal form of Theorem~\ref{thm:n3} is somewhat reminiscent of the quadratic
spherical map (\ref{eqn:classicalHopf}).
However, notice the difference in dimension of the domain (3 in Theorem~\ref{thm:n3}
and 4 in (\ref{eqn:classicalHopf}))
and the difference in sign of the last component.
\end{example}

\bibliographystyle{plain}
\begin{thebibliography}{Johnstone \& Williams 95}

\bibitem[Kreyszig 59]{kreyszig59}
Kreyszig, E. (1959) Differential Geometry.
Dover (New York).

\bibitem[Ono 94]{ono94}
Ono, T. (1994) Variations on a Theme of Euler.
Plenum Press (New York).

\end{thebibliography}

\end{document}
