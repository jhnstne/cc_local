\documentstyle[11pt]{article}
\newif\ifFull
\Fullfalse
\input{header}
\SingleSpace

\setlength{\oddsidemargin}{0pt}
\setlength{\topmargin}{-.25in}	% technically should be 0pt for 1in
\setlength{\headsep}{3em}
\setlength{\textheight}{8.75in}
\setlength{\textwidth}{6.5in}
\setlength{\columnsep}{5mm}		% width of gutter between columns

% \markright{The most natural map: \today \hfill}
% \pagestyle{myheadings}

% -----------------------------------------------------------------------------

\title{The most natural map to a sphere}
\author{J.K. Johnstone\thanks{Geometric Modeling Lab, 125 Campbell Hall, 
	Department of Computer and Information Sciences, 
	UAB, Birmingham, AL 35294.}}

\begin{document}
\maketitle

\begin{abstract}
Using a characterization of rational maps
from $\Re^{n+1}$ to the sphere \Sn{n},
it can be argued that the most natural map to the sphere is
\[ (x_1,\ldots,x_{n+1}) \mapsto 
	(\frac{x_1^2 + \cdots + x_{n}^2 - x_{n+1}^2}{x_1^2 + \cdots + x_{n+1}^2},
	 \frac{2x_1x_{n+1}}{x_1^2 + \cdots + x_{n+1}^2},\ldots,
	 \frac{2x_{n}x_{n+1}}{x_1^2 + \cdots + x_{n+1}^2}).
\]
We explore the properties of this map,
and establish that it is a powerful generalization
of the inverse of stereographic projection.
We develop a tool for visualizing a map, called a map texture,
that allows us to better compare two maps to a surface.
\end{abstract}

\noindent Keywords: rational maps to the sphere, stereographic projection,
	map visualization.

\clearpage

\section{Introduction}

In a recent paper, we developed 
the following characterization of the rational maps of $\Re^{n+1}$ to \Sn{n}
(the unit sphere $x_1^2 + \ldots + x_{n+1}^2 - 1 = 0$ in $\Re^{n+1}$)
and used it in the design of rational quaternion splines
\cite{jj+jimbo99}.

\begin{theorem}
\label{thm:nn}
A map is a rational map of $\Re^{n+1}$ to \Sn{n}, $n \geq 1$, if and only if
it is of the form:
\begin{equation}
\label{eq:normalform}
	(x_{1},\ldots,x_{n+1}) \mapsto 
	\pi(\frac{a_1^2 + \cdots + a_{n}^2 - a_{n+1}^2}{a_1^2 + \cdots + a_{n+1}^2},
	 \frac{2a_1a_{n+1}}{a_1^2 + \cdots + a_{n+1}^2},
	 \ldots,
	 \frac{2a_{n}a_{n+1}}{a_1^2 + \cdots + a_{n+1}^2})
\end{equation}
where $a_1,\ldots,a_{n+1} \in \Re[x_1,\ldots,x_{n+1}]$,
the ring of polynomials in $x_1,\ldots,x_{n+1}$ over the field of real numbers,
and $\pi$ is a permutation.
\end{theorem}

We would like to study the `most natural' map from this collection,
that generated by the `most natural' choice for $\pi$,
the identity permutation, and the `most natural' choice for $a_i$,
$(a_1,\ldots,a_{n+1}) = (x_1,\ldots,x_{n+1})$.
This map is 
\begin{equation}
\label{eq:Mmap}
	M(x_1,\ldots,x_{n+1}) = 
	(\frac{x_1^2 + \cdots + x_{n}^2 - x_{n+1}^2}{x_1^2 + \cdots + x_{n+1}^2},
	 \frac{2x_1x_{n+1}}{x_1^2 + \cdots + x_{n+1}^2},\ldots,
	 \frac{2x_{n}x_{n+1}}{x_1^2 + \cdots + x_{n+1}^2}).
\end{equation}
which we call the {\bf most natural} rational map of $\Re^{n+1}$ to \Sn{n}.
In this paper, we will show that $M$ is a powerful extension
of stereographic injection, the inverse of stereographic projection.
Section~\ref{sec:stereo} reviews stereographic projection and stereographic
injection.
We compare $M$ and stereographic injection 
in Sections~\ref{sec:M} and \ref{sec:maptexture}.
Section~\ref{sec:maptexture} introduces the map texture, 
a technique for the visualization of maps to a surface, 
to aid in this comparison.
We end with some conclusions in Section~\ref{sec:conclusions}.

\section{Stereographic projection}
\label{sec:stereo}

The most well-known map involving the sphere is stereographic projection, 
the map from \Sn{n}\ to the hyperplane $x_{n+1}=0$ 
in which a point of \Sn{n}\ is perspectively projected from 
the north pole of \Sn{n}\ to $x_{n+1}=0$  % \cite{thorpe79}. % or kreyszig59
(Figure~\ref{fig:stereo}).\footnote{Stereographic projection has been
	used at least since Hipparch in 160 B.C. \cite{kreyszig59}.}

\begin{lemma}
Stereographic projection is the map from 
$\Sn{n} - (0,\ldots,0,1)$ to $x_{n+1}=0$ defined by:
\[ (x_1,\ldots,x_{n+1}) \mapsto \frac{1}{1-x_{n+1}} (x_1,\ldots,x_n,0) \]
\end{lemma}
\prf
The projector line $tp + (1-t)q$ through $p = (x_1,\ldots,x_{n+1})$ 
and $q = (0,\ldots,0,1)$ intersects $x_{n+1}=0$ 
when $tx_{n+1} + (1-t) = 0$ or $t = \frac{1}{1-x_{n+1}}$.
\QED

\begin{figure}[ht]
\vspace{3in}
\special{psfile=/usr/people/jj/modelTR/2-most/img/fig:stereo.ps hoffset=40}
% /ca/jj/Orientation/S3curve/paper/img/fig:stereo.ps}
\caption{Stereographic projection in 3-space}
\label{fig:stereo}
\end{figure}

Since stereographic projection is one-to-one and onto,
it has a well-defined inverse.
This is a rational map to the sphere.
%
\begin{lemma}
The inverse of stereographic projection is the map from the hyperplane
$x_{n+1}=0$ to $\Sn{n} - (0,\ldots,0,1)$ defined by:
\[ (x_1,\ldots,x_n,0) \mapsto
	\frac{1}{x_1^2 + \cdots + x_n^2 + 1} 
	(2x_1, \ldots, 2x_n, x_1^2 + \cdots + x_n^2 - 1)
\]	% see thorpe79, p. 125
\end{lemma}
\prf
The projector line through $(r,0)$ and $q = (0,\ldots,0,1)$, $t(r,0) + (1-t)q$,
intersects \Sn{n} at $t=0$ and $t=\frac{2}{\|r\|^2 + 1}$.
\QED

Viewing $x_{n+1}=0$ as an embedding of $n$-space in $\Re^{n+1}$, 
this inverse can be interpreted as a rational map from $\Re^{n}$ to 
$\Sn{n} - (0,\ldots,0,1)$.\footnote{One can interpret the 
	line at infinity of the $\Re^n$ embedding
	as mapping to the missing pole $(0,\ldots,0,1)$.}
By rotation, we can easily generalize to a rational map from $\Re^n$
to $\Sn{n} - \{q\}$, where $q$ is an arbitrary point of \Sn{n}
	% by choosing a different center of
	% projection on \Sn{n}, say $q$, and projecting onto the associated hyperplane:
	% the hyperplane through the origin and parallel to $q$'s tangent plane.
	% This is simply a rotation of the conventional stereographic projection map.
whose choice dictates the hyperplane embedding of $\Re^n$ in $\Re^{n+1}$.
For example, the following version of the inverse map, which we shall use in
this paper and call stereographic injection, results from 
moving the pole to $(1,0,\ldots,0)$ and the projection plane to $x_1=0$.

\begin{defn2}
{\rm 
{\bf Stereographic injection} is the map 
$\sigma^{-1}_{(1,0,\ldots,0)}: x_{1}=0 \rightarrow \Sn{n} - (1,0,\ldots,0)$:
\begin{equation}
\label{eq:injection}
(0,x_1,\ldots,x_n) \mapsto
	\frac{1}{x_1^2 + \cdots + x_n^2 + 1} 
	(x_1^2 + \cdots + x_n^2 - 1, 2x_1, \ldots, 2x_n)
\end{equation}
or $\sigma^{-1}: \Re^n \rightarrow \Sn{n} - (1,0,\ldots,0)$:
\begin{equation}
\label{eq:si}
(x_1,\ldots,x_n) \mapsto
	\frac{1}{x_1^2 + \cdots + x_n^2 + 1} 
	(x_1^2 + \cdots + x_n^2 - 1, 2x_1, \ldots, 2x_n)
\end{equation}
}
\end{defn2}

\Comment{
The standard rational parameterization of \Sn{n}\ 
can be developed from stereographic injection.

Since stereographic injection is a one-to-one, 
(almost) onto map from an $n$-dimensional
hyperplane to the $n$-dimensional sphere, 
it induces a parameterization of the sphere.
For example, in the case of \Sn{3}, this parameterization is:
\begin{equation}
	S(t_1,t_2,t_3) = 
	\frac{1}{t_1^2 + t_2^2 + t_3^2 + 1} 
	(2t_1, 2t_2, 2t_3, t_1^2 + t_2^2 + t_3^2 - 1) \ \ \ \ t_i \in (-\infty,\infty)
\end{equation}
% For the circle \Sn{1}, this parameterization is a variant of the
% classical $(\frac{1-t^2}{1+t^2}, \frac{2t}{1+t^2})$.
}

Stereographic injection is very similar to the rational map of $\Re^{n+1}$ to
\Sn{n}\ in Theorem~\ref{thm:nn} generated by the choice 
$(a_1,\ldots,a_{n+1}) = (x_1,\ldots,x_{n},1)$ and the identity permutation:
%
\begin{equation}
\label{eq:rat1}
(x_1,\ldots,x_{n+1}) \mapsto 
	\frac{1}{x_1^2 + \cdots + x_{n}^2 + 1}
	(x_1^2 + \cdots + x_{n}^2 - 1,
	 2x_1, \ldots, 2x_{n})
\end{equation}
%
Since every hyperplane $x_{n+1}=k$ of (\ref{eq:rat1}) gets mapped like 
(\ref{eq:si}), this is a many-to-one map with many embedded copies of
stereographic injection.
(\ref{eq:rat1}) will be called {\bf embedded stereographic injection}.

\section{The most natural map}
\label{sec:M}

We would like to compare the most natural map $M$ with stereographic injection.
$M$ works over a larger domain than stereographic injection ($\Re^{n+1}$
rather than $\Re^n$).
Therefore, the most we can expect is that $M$ contains some embedding of
stereographic injection.
It does: the restriction of $M$ to the hyperplane $x_{n+1}=1$
is equivalent to stereographic injection.
However, $x_{n+1}=1$ is the only hyperplane restriction that is equivalent to 
stereographic injection.
For example, the restriction of $M$ to the hyperplane $x_{n+1}=5$ is
\[
	(x_{1},\ldots,x_n) \rightarrow
	\frac{1}{x_1^2 + \cdots + x_n^2 + 25}
	(x_1^2 + \cdots + x_{n}^2 - 25, 10x_1, \ldots, 10x_n).
\]
Thus, $M$ can be characterized as an extension of stereographic injection:
its restriction to the hyperplane $x_{n+1}=1$ is equivalent
	% the hyperplane restriction $M_{x_{n+1}=1}$ is equivalent to $\sigma^{-1}$
while its behaviour on other hyperplanes of $\Re^{n+1}$ is different
from $\sigma^{-1}$.

A comparison of $M$ and stereographic injection in projective space initially 
seems to reveal a deeper relationship between the two maps,
but this is misleading.
Translating $M$ to projective space (with projective coordinate $x_{n+2}$), 
we have
% where little changes since the terms are already homogeneous of degree 2:
\begin{equation}
\label{eq:projM}
(x_1,\ldots,x_{n+1},x_{n+2}) \mapsto 
	(x_1^2 + \cdots + x_{n}^2 - x_{n+1}^2,
	 2x_1x_{n+1}, \ldots, 2x_nx_{n+1}, x_1^2 + \cdots + x_{n+1}^2)
\end{equation}
Similarly, translating stereographic injection to projective space 
(with projective coordinate $x_{n+1}$), we have
\begin{equation}
\label{eq:projsi}
	(x_1,\ldots,x_{n+1}) \mapsto
	(x_1^2 + \cdots + x_n^2 - x_{n+1}^2,\ 2x_{1}x_{n+1}, \ldots,\ 2x_{n}x_{n+1},\ 
	x_1^2 + \cdots + x_{n+1}^2)
\end{equation}
	% These look virtually identical!
These seem to have the same relationship as embedded stereographic
injection and stereographic injection,
as every hyperplane $x_{n+2}=k$ of (\ref{eq:projM}) gets mapped 
like stereographic injection (\ref{eq:projsi}).
However, we are now dealing with the very different case of 
hyperplanes in the projective coordinate $x_{n+2}$.
Even more importantly, $x_{n+1}$ is a typical % affine?
coordinate in (\ref{eq:projM}) but the added projective coordinate in (\ref{eq:projsi}).
The interchange of projective and other coordinates is a significant
change to a map. %, unlike the minor effect of interchanging two affine coordinates.
For example, the two points $(0,0,1)$ and $(0,1,0)$ in projective 2-space
are identical except for an interchange of coordinates,
yet $(0,0,1)$ is the origin while $(0,1,0)$ is a point at infinity
along the $y$-axis.
Thus, $M$ is indeed significantly different from stereographic injection.

\section{A map texture}
\label{sec:maptexture}

As an aid to understanding the difference between $M$ and stereographic 
injection, we would like to visualize these two maps.
In this section, we introduce a map visualization tool called the map texture,
an experiment in the use of visualization for map comparison.
It may also offer an alternative method for developing interesting texture
maps on surfaces.

Consider the visualization of an injective % that is, 1-1
map from a surface $A$ in $\Re^3$ to itself, $f: A \rightarrow A$.
An injective map $g:A \rightarrow \Re^3$ from a 2-surface to 3-space
can usually be visualized
by displaying the image of $A$, which is another surface in $\Re^3$.
However, this is not effective for the map $f$, since the image of $A$
is simply $A$ again.
We not only need to display the image of $A$, but colorcode it in some way.
We propose to visualize $f:A \rightarrow A$ by colorcoding each point $P \in A$
by the (normalized) distance this point moves under the map:
\[
\mbox{greyscale}(P) = \frac{\mbox{dist}(P,f(P))}{\mbox{max}_{\footnotesize{P \in A}} 
				\{ \mbox{dist}(P,f(P))\}}.
\]
We call this colorcoded image a {\bf map texture}.
See Figure~\ref{fig:textureperfect}.
Distance is measured on the sphere: $\mbox{dist}(P,f(P)) = \cos^{-1}(P \cdot f(P))$.
Map textures can also be used to gain insight into
the behaviour of maps $f:\Re^{3} \rightarrow A$, by visualizing
the restriction of $f$ to the surface $A$.

\ifFull
\begin{example}
Figure~\ref{fig:injection} shows the map texture of the inversion map.
Inversion of a point in the unit circle centered at the origin
is the map \cite{jjnewintalg}
\[
(x_1,x_2) \rightarrow \frac{1}{x_1^2 + x_2^2}(x_1,x_2)
\]
To visualize this map, we embed it in $\Re^3$ as a map from a hyperplane $A$ to $A$:
\[
(x_1,x_2,0) \rightarrow \frac{1}{x_1^2 + x_2^2}(x_1,x_2,0)
\] 
\end{example}
\begin{example}
Vector map in Lang.
\end{example}
\begin{example}
Dietz map (Example 5.3).
\end{example}
\fi

We wish to compare the map textures of $M$ and stereographic injection.
We consider the 3-space versions of these maps:
$M: \Re^3 \rightarrow \Sn{2}$ and $\sigma^{-1}:\Re^2 \rightarrow \Sn{2}$.
Since $\sigma^{-1}$ is a map from only a hyperplane of $\Re^3$ to the sphere,
we cannot visualize it directly.
However, we can visualize the strongly related map,
embedded stereographic injection (\ref{eq:rat1}).
Figures~\ref{fig:texturestereo} and \ref{fig:textureperfect} show the 
map textures of $M$ and embedded stereographic injection, respectively.
These reveal some interesting regions of smaller movement in $M$
(smaller distances are blacker)
and the essentially higher complexity of $M$.

\begin{figure}
\vspace{5in}
\special{psfile=/usr/people/jj/modelTR/2-most/img/texture-stereo.ps hoffset=100}
\caption{Map texture of embedded stereographic injection}
\label{fig:texturestereo}
% tops texture-stereo.rgb -m 8.5 4 > texture-stereo.ps
\end{figure}

\section{Conclusions}
\label{sec:conclusions}

We have studied the map $M$, 
the `most natural' rational map from $\Re^{n+1}$ to \Sn{n}.
	% based on the normal form (\ref{eq:normalform}).
The development and understanding of superior, and possibly optimal,
maps to a surface is important to Riemannian modeling, 
the design of models in Riemannian spaces such as surfaces.
We have shown that $M$ is a natural extension of the inverse map of
stereographic projection, from a hyperplane to the entire space,
exhibiting more complex behaviour.
The added power of $M$ is vividly revealed when the two maps are used
in an application, such as the design of quaternion splines.
Quaternion splines generated using $M$ are of much higher quality
than those generated using stereographic injection \cite{jj+jimbo99}.
We have also exhibited a map visualization tool, the map texture,
which can be used to better understand a map to a surface.
The map texture could also be used as a artistic mechanism for the 
creation of texture maps on a surface, using maps to that surface.

% put here so that it doesn't get placed alone on a page (as it does if we put above immediately after 1st map texture)
\begin{figure}
\vspace{5in}
\special{psfile=/usr/people/jj/modelTR/2-most/img/texture-natural.ps hoffset=100}
\caption{Map texture of the most natural map $M$}
\label{fig:textureperfect}
% tops texture-natural.rgb -m 8.5 4 > texture-natural.ps
\end{figure}

\section{Acknowledgements}

I appreciate discussions with Ken Sloan about the map texture.

\bibliographystyle{plain}
\begin{thebibliography}{Johnstone \& Williams 99}

\bibitem[Johnstone \& Williams 99]{jj+jimbo99}
Johnstone, J. and J. Williams (1999)
Rational Quaternion Splines of Arbitrary Continuity.
Technical Report, CIS Dept., UAB.

\bibitem[Kreyszig 59]{kreyszig59}
Kreyszig, E. (1959) Differential Geometry.
Dover (New York).

\bibitem[Thorpe 79]{thorpe79}
Thorpe, J.A. (1979) Elementary Topics in Differential Geometry.
Springer-Verlag (New York), p. 124.

\end{thebibliography}

\end{document}

\Comment{
Mathematicians have long been interested in maps {\em from} the sphere
(which can be inverted into maps to the sphere) because of problems
in cartography.
However, since properties such as conformality (preservation of angles between intersecting
curves) and equiareality (preservation of area) were more important than
rationality (the ability to express the map in terms of quotients of polynomials)
in the design of these `cartographic' maps, most of these maps
are not rational (e.g., the Mercator projection, Lambert projection,
mapping of Sanson, and mapping of Bonne \cite{kreyszig59}).
A rational map to the sphere is important in the design of curves on the
sphere, since it generates rational curves (such as rational B-splines).
}
