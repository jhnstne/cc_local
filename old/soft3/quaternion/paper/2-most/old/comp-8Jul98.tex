\documentstyle[11pt]{article}
\newif\ifFull
\Fullfalse
\input{header}
\SingleSpace

\setlength{\oddsidemargin}{0pt}
\setlength{\topmargin}{-.25in}	% technically should be 0pt for 1in
\setlength{\headsep}{3em}
\setlength{\textheight}{8.75in}
\setlength{\textwidth}{6.5in}
\setlength{\columnsep}{5mm}		% width of gutter between columns

\markright{The most natural map: \today \hfill}
\pagestyle{myheadings}

% -----------------------------------------------------------------------------

\title{The most natural map to a sphere}
\author{J.K. Johnstone}

\begin{document}
\maketitle

\begin{abstract}
We have recently developed a characterization of all rational maps
from $\Re^{n+1}$ to the sphere \Sn{n}.
Using this characterization, 
the most natural map to the sphere is
\[ (x_1,\ldots,x_{n+1}) \mapsto 
	(\frac{x_1^2 + \cdots + x_{n}^2 - x_{n+1}^2}{x_1^2 + \cdots + x_{n+1}^2},
	 \frac{2x_1x_{n+1}}{x_1^2 + \cdots + x_{n+1}^2},\ldots,
	 \frac{2x_{n}x_{n+1}}{x_1^2 + \cdots + x_{n+1}^2}).
\]
This map initially appears similar to stereographic projection,
but this is misleading.
We explore the properties of this map, especially its essential difference
from stereographic projection.
We develop a tool for visualizing a map, called a map texture,
that allows us to better compare two maps to a surface.
\end{abstract}

\noindent Keywords: Map visualization, map texture,
		rational maps to the sphere, stereographic projection.

\clearpage

\section{Introduction}

In a recent paper, we developed 
a full characterization of the rational maps of $\Re^{n+1}$ to the 
unit sphere in $\Re^{n+1}$, \Sn{n}.

\begin{theorem}
\label{thm:nn}
A map is a rational map of $\Re^{n+1}$ to \Sn{n}, $n \geq 1$, if and only if
it is of the form:
\begin{equation}
\label{eq:normalform}
	(x_{\pi(1)},\ldots,x_{\pi(n+1)}) \mapsto 
	(\frac{a_1^2 + \cdots + a_{n}^2 - a_{n+1}^2}{a_1^2 + \cdots + a_{n+1}^2},
	 \frac{2a_1a_{n+1}}{a_1^2 + \cdots + a_{n+1}^2},
	 \ldots,
	 \frac{2a_{n}a_{n+1}}{a_1^2 + \cdots + a_{n+1}^2})
\end{equation}
where $a_1,\ldots,a_{n+1} \in \Re[x_1,\ldots,x_{n+1}]$
and $\pi: \{1,\ldots,n+1\} \rightarrow \{1,\ldots,n+1\}$ is a permutation.
\end{theorem}

The most natural choice for the polynomials $a_1,\ldots,a_{n+1}$
is $(a_1,\ldots,a_{n+1}) = (x_1,\ldots,x_{n+1})$
and the most natural choice for the permutation is the identity permutation,
leading to the map
\begin{equation}
\label{eq:Mmap}
	M(x_1,\ldots,x_{n+1}) = 
	(\frac{x_1^2 + \cdots + x_{n}^2 - x_{n+1}^2}{x_1^2 + \cdots + x_{n+1}^2},
	 \frac{2x_1x_{n+1}}{x_1^2 + \cdots + x_{n+1}^2},\ldots,
	 \frac{2x_{n}x_{n+1}}{x_1^2 + \cdots + x_{n+1}^2})
\end{equation}
which we consider the most natural rational map of $\Re^{n+1}$ to \Sn{n}.
In this paper, we will analyze this map $M$, especially its apparent
relationship to stereographic projection.

In Section~\ref{sec:stereo}, we review stereographic projection and its inverse,
which we call stereographic injection.
A related map is presented in Section~\ref{sec1}.
A comparison of $M$ and stereographic projection is given in Section~\ref{sec:M}.
To further this comparison, 
we introduce the map texture in Section~\ref{sec:maptexture}, 
a technique for the visualization of maps to a surface.
We end with some conclusions.

% ------

\section{Stereographic projection and injection}
\label{sec:stereo}

The most well-known map involving the sphere is
stereographic projection (Figure~\ref{fig:stereo}).
It leads to a rational map from $\Re^n$, a hyperplane in $\Re^{n+1}$,
to \Sn{n}, the unit sphere in $\Re^{n+1}$.

\begin{defn2}
{\bf Stereographic projection} is the map from \Sn{n}\ 
to the hyperplane $x_{n+1}=0$,
in which a point of \Sn{n}\ is perspectively projected from 
the north pole of \Sn{n}\ to $x_{n+1}=0$ \cite{thorpe79}.	% or kreyszig59
It is fruitful to view the range of stereographic projection, $x_{n+1}=0$, 
as an embedding of $n$-space in $\Re^{n+1}$, so that stereographic projection
is a map from \Sn{n}\ to $\Re^n$.\footnote{Although the image of the north 
	pole is undefined, the north pole may be interpreted
	as mapping to the line at infinity of the $\Re^n$ embedding.}
\end{defn2}

\begin{lemma}
Stereographic projection is the map from 
$\Sn{n} - (0,\ldots,0,1)$ to $x_{n+1}=0$ defined by:
\[ (x_1,\ldots,x_{n+1}) \mapsto \frac{1}{1-x_{n+1}} (x_1,\ldots,x_n,0) \]
\end{lemma}
\prf
The projector line $tp + (1-t)q$ through $p = (x_1,\ldots,x_{n+1})$ 
and $q = (0,\ldots,0,1)$ intersects $x_{n+1}=0$ 
when $tx_{n+1} + (1-t) = 0$ or $t = \frac{1}{1-x_{n+1}}$.
\QED

\begin{figure}[ht]
\vspace{3in}
\special{psfile=/usr/people/jj/modelTR/3-spline/img/fig:stereo.ps}
% /ca/jj/Orientation/S3curve/paper/img/fig:stereo.ps}
\caption{Stereographic projection in 3-space}
\label{fig:stereo}
\end{figure}


Since stereographic projection is one-to-one and onto,
it has a well-defined inverse.
This is a map to the sphere.
%
\begin{lemma}
The inverse of stereographic projection is the map from the hyperplane
$x_{n+1}=0$ to $\Sn{n} - (0,\ldots,0,1)$ defined by:
\[ (x_1,\ldots,x_n,0) \mapsto
	\frac{1}{x_1^2 + \cdots + x_n^2 + 1} 
	(2x_1, \ldots, 2x_n, x_1^2 + \cdots + x_n^2 - 1)
\]	% see thorpe79, p. 125
\end{lemma}
\prf
The projector line through $(r,0)$ and $q = (0,\ldots,0,1)$, $t(r,0) + (1-t)q$,
intersects \Sn{n} at $t=0$ and $t=\frac{2}{\|r\|^2 + 1}$.
\QED

This inverse is a map to \Sn{n}\ without the north pole.
We can easily generalize it to a map to \Sn{n}\ 
without the arbitrary point $q \in \Sn{n}$, through rotation.
% by choosing a different center of
% projection on \Sn{n}, say $q$, and projecting onto the associated hyperplane:
% the hyperplane through the origin and parallel to $q$'s tangent plane.
% This is simply a rotation of the conventional stereographic projection map.
Thus, stereographic projection and its inverse are 
maps $\sigma_q: \Sn{n} - \{q\} \rightarrow \Re^n$
and $\sigma_q^{-1}: \Re^n \rightarrow \Sn{n} - \{q\}$,
where the choice of $q \in \Sn{n}$ dictates the hyperplane
embedding of $\Re^n$ in $\Re^{n+1}$.
For example, moving the pole to $(1,0,\ldots,0)$ and the projection
plane to $x_1=0$ yields our preferred version of the map.

\begin{defn2}
{\rm 
{\bf Stereographic injection} is the map 
$\sigma^{-1}_{(1,0,\ldots,0)}: x_{1}=0 \rightarrow \Sn{n} - (1,0,\ldots,0)$:
\begin{equation}
\label{eq:injection}
(0,x_1,\ldots,x_n) \mapsto
	\frac{1}{x_1^2 + \cdots + x_n^2 + 1} 
	(x_1^2 + \cdots + x_n^2 - 1, 2x_1, \ldots, 2x_n)
\end{equation}
}
\end{defn2}

\section{Embedded stereographic injection}
\label{sec1}

Consider the general form for rational maps of $\Re^{n+1}$ to the sphere
(Theorem~\ref{thm:nn}).
With $(a_1,\ldots,a_{n+1}) = (x_1,\ldots,x_{n},1)$ and the identity permutation,
the rational map to the sphere $\Re^{n+1} \rightarrow \Sn{n}$ is
\begin{equation}
\label{eq:rat1}
(x_1,\ldots,x_{n+1}) \mapsto 
	\frac{1}{x_1^2 + \cdots + x_{n}^2 + 1}
	(x_1^2 + \cdots + x_{n}^2 - 1,
	 2x_1, \ldots, 2x_{n})
\end{equation}
This is very similar to stereographic injection 
$\Re^n \rightarrow \Sn{n}$:
\begin{equation}
\label{eq:si}
(x_1,\ldots,x_n) \mapsto
	\frac{1}{x_1^2 + \cdots + x_n^2 + 1} 
	(x_1^2 + \cdots + x_n^2 - 1, 2x_1, \ldots, 2x_n)
\end{equation}
Every hyperplane $x_{n+1}=k$ of (\ref{eq:rat1}) gets mapped like stereographic injection.
Thus, this is a many-to-one map, with many embedded copies of stereographic injection.
Consequently, we call the map (\ref{eq:rat1}) 
{\bf embedded stereographic injection}.

\section{The most natural map $M$}
\label{sec:M}

Now consider $M$:
\begin{equation}
\label{eq:M}
	(x_{1},\ldots,x_{n+1}) \rightarrow
	\frac{1}{x_1^2 + \cdots + x_{n+1}^2}
	(x_1^2 + \cdots + x_{n}^2 - x_{n+1}^2,
	 2x_1x_{n+1}, \ldots, 2x_{n}x_{n+1})
\end{equation}
%
Clearly, $M: \Re^{n+1} \rightarrow \Sn{n}$ maps the entire space to the sphere,
while stereographic injection $\sigma^{-1}:\Re^n \rightarrow \Sn{n}$ only maps
a hyperplane to the sphere.
$M$ does contain an embedding of stereographic injection:
the restriction of $M$ to the hyperplane $x_{n+1}=1$
is stereographic injection.
However, this is the only hyperplane that is equivalent
to stereographic injection, in contrast to embedded stereographic injection
(\ref{eq:rat1}).
%  where the behaviour of the map on every $x_{n+1}=k$ hyperplane 
% is equivalent to stereographic injection.
Thus, $M$ can be characterized as a superset of stereographic injection 
$\sigma^{-1}$,
where the hyperplane restriction $M_{x_{n+1}=1}$ is equivalent to $\sigma^{-1}$
and the rest of the map $M$ is different than stereographic injection.
For example, the restriction of $M$ to the hyperplane $x_{n+1}=5$ is
\[
	(x_{1},\ldots,x_n) \rightarrow
	\frac{1}{x_1^2 + \cdots + x_n^2 + 25}
	(x_1^2 + \cdots + x_{n}^2 - 25, 10x_1, \ldots, 10x_n).
\]

A comparison of $M$ and stereographic injection in projective space seems
to reveal a deeper relationship between the two maps,
but this is also misleading, 
as one would expect from the previous discussion.
Translating $M$ to projective space (with projective coordinate $x_{n+2}$), 
we have
% where little changes since the terms are already homogeneous of degree 2:
\begin{equation}
\label{eq:projM}
(x_1,\ldots,x_{n+1},x_{n+2}) \mapsto 
	(x_1^2 + \cdots + x_{n}^2 - x_{n+1}^2,
	 2x_1x_{n+1}, \ldots, 2x_nx_{n+1}, x_1^2 + \cdots + x_{n+1}^2)
\end{equation}
Compare this to stereographic injection in projective space (with projective
coordinate $x_{n+1}$)
\begin{equation}
\label{eq:projsi}
	(x_1,\ldots,x_{n+1}) \mapsto
	(x_1^2 + \cdots + x_n^2 - x_{n+1}^2,\ 2x_{1}x_{n+1}, \ldots,\ 2x_{n}x_{n+1},\ 
	x_1^2 + \cdots + x_{n+1}^2)
\end{equation}
These look virtually identical!
Indeed, they seem to have the same relationship as embedded stereographic
injection and stereographic injection,
as every `hyperplane' $x_{n+2}=k$ gets mapped 
like stereographic injection.
However, we are now dealing with the very different case of 
hyperplanes in the projective coordinate $x_{n+2}$.
Even more importantly, $x_{n+1}$ is a typical affine coordinate 
in (\ref{eq:projM}),
while $x_{n+1}$ is the projective coordinate in (\ref{eq:projsi}).
The interchange of projective and affine coordinates is a significant
change to a map, unlike the minor effect of interchanging two
affine coordinates.
We again conclude that $M$ is substantially different from stereographic
injection.

\section{A map texture}
\label{sec:maptexture}

As an aid to establishing the difference between $M$ and stereographic injection,
we would like to visualize these two maps.
In this section, we introduce a map visualization tool called the map texture.
Since we have already established the fundamental difference
between the maps, this is largely an experiment or proof-of-concept
in the use of visualization for map comparison.
It may also offer an alternative method for developing interesting texture
maps on surfaces.

Consider the visualization of an injective map $f: A \rightarrow A$,
where $A$ is a surface in $\Re^3$.
An injective map $f:A \rightarrow \Re^3$ can usually be visualized by
displaying the image of $A$, which is another surface in $\Re^3$.
However, this is not effective for the map $f$, since the image of $A$
is simply $A$ again.
We not only need to display the image of $A$, but colorcode it in some way.
We propose to visualize $f:A \rightarrow A$ by colorcoding each point $P \in A$
by:
\[
\mbox{greyscale}(P) = \frac{\mbox{dist}(P,f(P))}{\mbox{max}_{\footnotesize{P \in A}} 
				\{ \mbox{dist}(P,f(P))\}}
\]
and call this a {\bf map texture}.
Map textures can also be used to gain insight into
the behaviour of maps $f:\Re^{3} \rightarrow A$, by visualizing
the restriction of $f$ to the surface $A$.

% \begin{example}
% Figure~\ref{fig:injection} shows the map texture of the inversion map.
% Inversion of a point in the unit circle centered at the origin
% is the map \cite{jjnewintalg}
% \[
% 	(x_1,x_2) \rightarrow \frac{1}{x_1^2 + x_2^2}(x_1,x_2)
% \]
% To visualize this map, we embed it in $\Re^3$ as a map from a hyperplane $A$ to $A$:
% \[
% 	(x_1,x_2,0) \rightarrow \frac{1}{x_1^2 + x_2^2}(x_1,x_2,0)
% \] 
% \end{example}
%
% \begin{example}
%  Vector map in Lang.
% \end{example}
%
% \begin{example}
% Dietz map (Example 5.3).
% \end{example}

Let us now compare the map textures of $M$ and stereographic injection.
We consider the 3-space versions of the maps:
$M: \Re^3 \rightarrow \Sn{2}$ and 
$\sigma^{-1}_{(1,0,0)}:\Re^2 \rightarrow \Sn{2}$.
We cannot visualize stereographic injection directly,
since it is a map from only a hyperplane of $\Re^3$ to the sphere.
However, we can visualize the related map of Section~\ref{sec1},
embedded stereographic injection (\ref{eq:rat1}),
whose strong relationship with stereographic injection
will reveal differences between stereographic injection and $M$.
Figures~\ref{fig:textureperfect} and \ref{fig:texturestereo} show the 
map textures of the most natural map to the sphere, $M$, and
embedded stereographic injection, respectively.
These reveal some interesting `poles' of higher movement in $M$,
and the essentially higher complexity of $M$.
An interactive visualization of the map texture gives an even better sense 
of the map behaviour.
This is available at www.cis.uab.edu/~jj/javaapplet.

\section{Conclusions} 

$M$, the most natural rational map from $\Re^{n+1}$ to \Sn{n}\ 
based on the normal form (\ref{eq:normalform}),
is a much more complex map than stereographic injection,
despite the similarity of the formulae (\ref{eq:Mmap}) and (\ref{eq:injection}).
The difference between the two maps
is also vividly revealed when the two maps are used
in an application, such as the design of curves on \Sn{n}.
See \cite{jj+jimbo98}.

% \cite{dietz93} shows the relationship of the rational map -- to 
% stereographic projection and hyperbolic projection.

The map texture shows some promise as a tool for evaluating a map to a surface.
It is also a mechanism for creating texture maps.

\section{Acknowledgements}

I appreciate discussions with Ken Sloan about the map texture.

\bibliographystyle{plain}
\begin{thebibliography}{Ebbinghaus 90}

\bibitem[Johnstone 98b]{jj+jimbo98}
Johnstone, J. and J. Williams (1995)
Rational Quaternion Splines,
Technical Report 98-03, CIS Dept., UAB.

\bibitem[Kreyszig 59]{kreyszig59}
Kreyszig, E. (1959) Differential Geometry.
Dover (New York).

\bibitem[Thorpe 79]{thorpe79}
Thorpe, J.A. (1979) Elementary Topics in Differential Geometry.
Springer-Verlag (New York), p. 124.

\end{thebibliography}

\clearpage

\begin{figure}
\vspace{3.5in}
\special{psfile=/usr/people/jj/modelTR/2-most/img/texture-perfect.ps
% Orientation/S3curve/paper/img/texture-perfect.ps
	hoffset = 100}
\caption{Map texture of the most natural map}
\label{fig:textureperfect}
\end{figure}

\begin{figure}
\vspace{3.5in}
\special{psfile=/usr/people/jj/modelTR/2-most/img/texture-stereo.ps
% /ca/jj/Orientation/S3curve/paper/img/texture-stereo.ps
	hoffset = 100}
\caption{Map texture of embedded stereographic injection}
\label{fig:texturestereo}
\end{figure}

Another version of stereographic projection definition from quaternion splines paper:

\subsection{A rational parameterization of \Sn{3}}

The standard rational parameterization of \Sn{n}\ 
can be developed from stereographic projection,
the well-known conformal map of a sphere to a plane (Figure~\ref{fig:stereo}). 
% initially used in cartography.

\begin{defn2}
{\bf Stereographic projection} in $(n+1)$-space is the map from \Sn{n}\ 
to the hyperplane $x_{n+1}=0$,
in which a point of \Sn{n}\ is perspectively projected from 
the north pole of \Sn{n}\ to $x_{n+1}=0$. % \cite{thorpe79} or kreyszig63
That is, stereographic projection is the map:
\[ \Sn{n} - (0,\ldots,0,1) \rightarrow x_{n+1}=0 \]
\[ (x_1,\ldots,x_{n+1}) \mapsto \frac{1}{1-x_{n+1}} (x_1,\ldots,x_n,0) \]
since the projector line $tp + (1-t)q$ through $p = (x_1,\ldots,x_{n+1})$ 
and $q = (0,\ldots,0,1)$ intersects $x_{n+1}=0$ 
when $tx_{n+1} + (1-t) = 0$ or $t = \frac{1}{1-x_{n+1}}$.
\end{defn2}

It is fruitful to view the range of stereographic projection, $x_{n+1}=0$, 
as an embedding of $n$-space in $\Re^{n+1}$, so that stereographic projection
is a map from \Sn{n}\ to $\Re^n$.
(Although the image of the north pole is undefined, 
the north pole may be interpreted
as mapping to the line at infinity of the $\Re^n$ embedding.)

Since stereographic projection is one-to-one and onto,
it has a well-defined inverse.
This is a map to the sphere.
%
\begin{lemma}
The inverse of stereographic projection is the map from the hyperplane
$x_{n+1}=0$ to $\Sn{n} - (0,\ldots,0,1)$ defined by:
\[ (x_1,\ldots,x_n,0) \mapsto
	\frac{1}{x_1^2 + \cdots + x_n^2 + 1} 
	(2x_1, \ldots, 2x_n, x_1^2 + \cdots + x_n^2 - 1)
\]	% see thorpe79, p. 125
\end{lemma}
\prf
The projector line through $(r,0)$ and $q = (0,\ldots,0,1)$, $t(r,0) + (1-t)q$,
intersects \Sn{n} at $t=0$ and $t=\frac{2}{\|r\|^2 + 1}$.
\QED

Since the inverse of stereographic projection is a one-to-one, 
(almost) onto map from an $n$-dimensional
hyperplane to the $n$-dimensional sphere, 
it induces a parameterization of the sphere.
For example, in the case of \Sn{3}, this parameterization is:
\begin{equation}
	S(t_1,t_2,t_3) = 
	\frac{1}{t_1^2 + t_2^2 + t_3^2 + 1} 
	(2t_1, 2t_2, 2t_3, t_1^2 + t_2^2 + t_3^2 - 1) \ \ \ \ t_i \in (-\infty,\infty)
\end{equation}
For the circle \Sn{1}, this parameterization is a variant of the
classical $(\frac{1-t^2}{1+t^2}, \frac{2t}{1+t^2})$.



\end{document}


