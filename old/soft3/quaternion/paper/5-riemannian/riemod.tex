% paper on general Riemannian modeling

\section{Introduction}
\label{sec:intro}

The design of curves on surfaces is an important problem,
with applications ranging from the trimming of surfaces in solid modeling,
to the design of splines on the quaternion sphere for orientation control
in computer animation, to the design of shortest paths on configuration
space obstacles in robot motion planning.
%  \footnote{Shortest paths amongst obstacles must inherently hug the obstacles.}
The design of a curve on a surface is more challenging than the 
conventional design of a curve in Euclidean space.
First, a surface constraint is added to the more typical constraints
such as interpolation of data points and preservation of smoothness criteria.
Second, surfaces are Riemannian spaces and 
Riemannian geometry is significantly different from Euclidean geometry.
% \cite{kreyszig63}.
For example, in moving from Euclidean to Riemannian geometry,
straight lines are replaced by geodesics.
% (\eg, a helix on a cylinder).

In this paper, we shall present a new general approach to the design of curves
on surfaces and discuss its implementation and advantages.
A single case study of this theory will be presented in detail:
the design of curves on \Sn{3}.

Since the modeling of curves in Euclidean space is much better understood, 
a promising approach to the
design of a curve on a surface is to somehow reduce the problem to the 
design of a curve in Euclidean space, not restricted to any surface.
% `Doctor, it hurts when I do this!'  `Well, then don't do it.'
% This reduction of an unknown problem to a known problem is a time-honored
% classical mathematical technique.
This allows the Riemannian problem to be solved as an extension
of the well-studied Euclidean problem, rather than as an entirely new
problem domain.
An approach for achieving this reduction to an Euclidean problem is as follows.

Suppose that we are designing a curve on the hypersurface\footnote{A
	hypersurface in $n$-space is a manifold of dimension $n-1$.}
$S$ in $n$-space.
Let $f:\Re^m \rightarrow S$ be a map to the surface.
Then any curve $C$ in $m$-space yields a curve $f(C)$ on the surface.
(See Figure~\ref{fig:reduce}.)
Consequently, we can now concentrate on designing the curve $C$, 
an Euclidean problem.
Although this is the simplest statement of the approach (the approach becomes
much more interesting when we add constraints on the type of map $f$ and 
the type of curve on the surface),
it reveals the essence of the approach most clearly:
the problem of constraining the curve to the surface is reduced to the
problem of developing a map to the surface.
The design of a curve on a surface through the design of a curve in $m$-space
and a map from $m$-space to the surface will be called the {\bf Euclidean-space
	% image-space, surface-map
approach} to curve design on a surface.
The resulting curves will be called {\bf Euclidean curves} on the surface.

\begin{figure}
\vspace{2.5in}
\special{psfile=/usr/people/jj/modelTR/3-spline/img/fig1.ps
	 hoffset=100}
\caption{The Euclidean-space approach to curve design on a surface}
% file: fig1.showcase
% tops fig1.rgb -m 6.5 1.5 > fig1.ps
% a simple picture of S, C, f(C), and f (as an arrow) for a generic surface S
\label{fig:reduce}
\end{figure}

Not surprisingly, the Euclidean-space approach has been used before.
% \cite{dietz93,wang94}.
% Given its simplicity and elegance, this is not surprising.
The classical solution to the design of trim curves on a surface 
is an example:
the trim curve is designed in the parameter space of the surface \cite{foley96}.
The surface parameterization is used as the map to the surface, % ($m=n-1$), 
which has the advantage that no map to the surface needs to be
developed.
However, a different map to the surface using the full power of
the Euclidean-space approach can improve the quality of the curve.
One can use added control in both steps of the Euclidean approach to optimize
the curve: the map can be directly controlled and, by choosing the
correct map, the design of the curve in Euclidean space is given more
flexibility (see below), which can be used to incorporate design optimizations.
Another reason to look beyond the trim-curve solution is the design of 
interpolating curves on a surface.
We shall now explore the rational interpolating curve on a surface,
which is the main focus of the paper.

	% Interpolating curves are perhaps the most important class of curves
	% in practical applications of curve design.
The interpolation of data points is necessary in many applications:
for example, in the design of quaternion 
splines for computer animation, where the data points are the orientations of 
the known keyframes; or in the design of paths on configuration-space
obstacles for robot motion, where the data points are the source, destination,
and rendezvous points of the path.
The design of rational curves is also highly desirable:
the rational curve is the most efficient of curves, because of the efficiency
of polynomial computations and the elegant theory of rational Bezier
and B-spline curves.
As a result, the rational curve is the defacto standard in modeling systems, 
with a large, established suite of algorithms for its manipulation.
Thus, the design of a rational curve will allow the curve to be incorporated 
immediately into existing software, and subsequent computations with the curve
will be simple and efficient.
Finally, we note that we are interested in curve design on
hypersurfaces in arbitrary dimensions.
For applications in animation and robotics, 
curves on surfaces in 4-space and 6-space are of just as much
interest as curves on surfaces in 3-space.

Consider the design of a rational
curve on the hypersurface $S$ that interpolates the set of points 
$\{p_i\}_{i=1,\ldots,k} \subset S$, using the Euclidean-space approach.
The algorithm is as follows.
Notice that (1a-b) are preprocessing steps.
%
\begin{description}
\item[(1a)] Design a rational map to the surface, $f:\Re^m \rightarrow S$ (preferably $m=n$).
\item[(1b)] Compute the inverse map $f^{-1}:S \rightarrow \Re^m$.
\item[(2a)] Map the data points to Euclidean space, by computing $\{f^{-1}(p_i)\}_{i=1,\ldots,k}$.
\item[(2b)] Design a rational curve $C$ in Euclidean space, interpolating $\{f^{-1}(p_i)\}_{i=1,\ldots,k}$.
\item[(2c)] Map the curve back to the surface, by computing $f(C)$.
\end{description}

We prefer maps to the surface of the form $f:\Re^n \rightarrow S$ (i.e., $m=n$).
Since $\Re^n$ is one dimension larger than $S$,
if $m=n$ then $f^{-1}(p_i)$, the inverse image of a point on $S$, is a curve.
This leads to a useful flexibility in the interpolation step (2b),
since one is interpolating curves, not points.
This can be used to design better curves on the surface.
This is a further disadvantage of the use of the surface
parameterization $f:\Re^{n-1} \rightarrow S$ as the map to the surface,
which yields points $f^{-1}(p_i)$ in Euclidean space.

Note that the inverse map $f^{-1}$ need not be rational.
Only the map $f$ and the curve $C$ must be rational.
Later in the paper, we shall consider the addition of derivative data
for the curve to interpolate.
% We postpone this discussion in order to avoid overcomplicating the issue
% at the outset.

As a case study of the proposed method,
we will consider in detail the design of rational interpolating curves 
on the 3-sphere \Sn{3}, also called quaternion splines.
We will show that this design is efficient and yields curves of superior
quality.
Quaternion splines arise in computer animation in the control of
orientation: the points on \Sn{3}\ are unit quaternions representing
discrete orientations of an object in a series of keyframes,
and the problem is to design a smooth motion of the object through
these keyframes,
which reduces to the design of a curve on \Sn{3}
interpolating the quaternions.
The application of quaternion 
splines is not restricted to computer animation, although they were
first developed in this context.  
Any application involving motion control,
such as robot motion or spacecraft control, 
can make good use of quaternion splines for orientation control.

A major advantage of the Euclidean-space approach is that it is easily
implemented and integrated into existing modelers,
since the heart of the algorithm is the design
of a curve in Euclidean space (step 2b).
The resulting curve can also be easily integrated into existing
geometric models, since it is a true rational curve.
Previous quaternion splines, for example, have required entirely new
weaponry (such as slerping or constrained optimization)
for the design of the curve;
and the resulting curve was highly nonrational so it could not be
directly handled by a NURBS modeling system.

%
% Advantages of native-space approach: move to Rational quaternion splines paper.
%
{\bf From `characterization paper':}

The native-space approach allows some added freedom in the design of the curve.
In general, the inverse image $f^{-1}(p)$ of a point on $A$ is a curve,
while $\rho^{-1}(p)$ is typically a point.
Consequently, the design of a curve on $A$ interpolating the points $\{p_i\}$
with the native-space approach
only requires the design of a curve interpolating the curves
$\{f^{-1}(p_i)\}$, as opposed to a curve interpolating
the points $\{\rho^{-1}(p_i)$.
The added freedom in the native-space approach can allow the design
of more optimal curves.
This is illustrated in \cite{dietz93}. (Where is the optimality?)
A second advantage of the native-space approach is more freedom
in the choice of map back to the surface.
Not being limited to the surface parameterization,
the native-space approach can use any rational map of $\Re^n$ to the surface.
This freedom can lead to superior curves on the surface, as illustrated
in \cite{jj+jimbo99}.


Another paper (on general Riemannian modeling):
The design of interpolating curves 
(curves that interpolate a set of data points) 
is a fundamental problem in geometric modeling.
A general approach to the design of rational interpolating curves on a surface
is proposed.
The curves constructed by the new method are of superior quality
and the method is highly efficient.
The curve's rationality allows smooth integration into existing
geometric models.
The design process relies on classical techniques for curve
interpolation in Euclidean space (unconstrained to a surface).
Thus, little new software needs to be written, and the design is simple
and efficient.
The default choice is to design curves with $C^2$ continuity,
but continuity can be controlled and is not bounded.
{\bf Check that \Sn{3}\ curve are $C^2$ continuous.}
The method is coordinate-frame invariant.
There is flexibility in the design of the curve that is lacking from 
earlier techniques.
The essence of the method is to map the data points to Euclidean space,
% choose a collection of points from the resulting collection of 
% data curves in Euclidean space, 
design the curve through selected points in unconstrained Euclidean space, 
and map the curve back to the surface.


