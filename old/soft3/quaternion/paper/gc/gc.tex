\documentstyle{article} 

\newif\ifFull
\Fullfalse

\makeatletter
\def\@maketitle{\newpage
 \null
 \vskip 2em                   % Vertical space above title.
 \begin{center}
       {\Large\bf \@title \par}  % Title set in \Large size. 
       \vskip .5em               % Vertical space after title.
       {\lineskip .5em           %  each author set in a tabular environment
        \begin{tabular}[t]{c}\@author 
        \end{tabular}\par}                   
  \end{center}
 \par
 \vskip .5em}                 % Vertical space after author
\makeatother

% default values are 
% \parskip=0pt plus1pt
% \parindent=20pt
\parskip=.2in
\parindent=0pt

\newcommand{\SingleSpace}{\edef\baselinestretch{0.9}\Large\normalsize}
\newcommand{\DoubleSpace}{\edef\baselinestretch{1.4}\Large\normalsize}
\newcommand{\Comment}[1]{\relax}  % makes a "comment" (not expanded)
\newcommand{\Heading}[1]{\par\noindent{\bf#1}\nobreak}
\newcommand{\Tail}[1]{\nobreak\par\noindent{\bf#1}}
\newcommand{\QED}{\vrule height 1.4ex width 1.0ex depth -.1ex\ } % square box
\newcommand{\arc}[1]{\mbox{$\stackrel{\frown}{#1}$}}
\newcommand{\lyne}[1]{\mbox{$\stackrel{\leftrightarrow}{#1}$}}
\newcommand{\ray}[1]{\mbox{$\vec{#1}$}}          
\newcommand{\seg}[1]{\mbox{$\overline{#1}$}}
\newcommand{\tab}{\hspace*{.2in}}
\newcommand{\se}{\mbox{$_{\epsilon}$}}  % subscript epsilon
\newcommand{\ie}{\mbox{i.e.}}
\newcommand{\eg}{\mbox{e.\ g.\ }}
\newcommand{\figg}[3]{\begin{figure}[htbp]\vspace{#3}\caption{#2}\label{#1}\end{figure}}
\newcommand{\be}{\begin{equation}}
\newcommand{\ee}{\end{equation}}
\newcommand{\prf}{\noindent{{\bf Proof}:\ \ \ }}
\newcommand{\choice}[2]{\mbox{\footnotesize{$\left( \begin{array}{c} #1 \\ #2 \end{array} \right)$}}}      
\newcommand{\scriptchoice}[2]{\mbox{\scriptsize{$\left( \begin{array}{c} #1 \\ #2 \end{array} \right)$}}}
\newcommand{\tinychoice}[2]{\mbox{\tiny{$\left( \begin{array}{c} #1 \\ #2 \end{array} \right)$}}}
\newcommand{\ddt}{\frac{\partial}{\partial t}}

\newtheorem{rmk}{Remark}[section]
\newtheorem{example}{Example}[section]
\newtheorem{conjecture}{Conjecture}[section]
\newtheorem{claim}{Claim}[section]
\newtheorem{notation}{Notation}[section]
\newtheorem{lemma}{Lemma}[section]
\newtheorem{theorem}{Theorem}[section]
\newtheorem{corollary}{Corollary}[section]
\newtheorem{defn2}{Definition}

% \font\timesr10
% \newfont{\timesroman}{timesr10}
% \timesroman

\setlength{\oddsidemargin}{0pt}
\setlength{\topmargin}{-.25in}	% technically should be 0pt for 1in
\setlength{\headsep}{0pt}
\setlength{\textheight}{8.75in}
\setlength{\textwidth}{6.5in}
\setlength{\columnsep}{5mm}		% width of gutter between columns

\title{Rational tangent-following frames}
\author{John K. Johnstone\\
	Department of Computer and Information Sciences\\
	The University of Alabama at Birmingham\\
	Birmingham, Alabama, USA 35294\\
	johnstone@cis.uab.edu\\
	205-934-2213
	\and 
	James P. Williams\\
	Department of Computer Science\\
	The Johns Hopkins University\\
	Baltimore, Maryland, USA 21218\\
	jimbo@cs.jhu.edu\\
	410-516-7784}

\begin{document}

\maketitle

\begin{abstract}
\end{abstract}

\section{Introduction}

The moving trihedron or Frenet frame at the point $r(u_0)$ is the frame
defined by the tangent, principal normal and binormal vectors at $r(u_0)$
\cite{kreyszig}.   % p. 36
The principal normal and binormal vectors are undefined
at points where the curvature is 0 (equivalently, $r''(u)=0$),
since the principal normal and binormal are defined in terms of the
osculating plane, the plane spanned by $r'(u)$ and $r''(u)$.
Therefore, the Frenet frame breaks down on linear components of the spine
and at inflection points.
Moreover, the Frenet frame cannot be easily patched at the inflection point
(by taking limits) since the binormal flips 180 degrees at this point.
{\em Problems with positive torsion: see Tang}.
{\em Problems with instability: see O'Donnell, page with Figure 5)}.
{\em Check our method on 1) linear components 2) inflection points
3) twists and 4) torsion}.
{\em Implement EGC to compare.}

\section{}

A quaternion can be interpreted algebraically as a 4-tuple,
or geometrically as a point on the quaternion sphere $S^3$.
Algebraically, quaternion multiplication \cite{hamilton69} is
\begin{equation}
\label{eq:qmult-alg}
	qr = (q_1 r_1 - (q_2,q_3,q_4) \cdot (r_2,r_3,r_4),
	      q_1 (r_2,r_3,r_4) + r_1 (q_2,q_3,q_4) + 
	      (q_2,q_3,q_4) \times (r_2,r_3,r_4))
% Shoemake, p. 246
\end{equation}
where $q=(q_1,q_2,q_3,q_4)$ and $r=(r_1,r_2,r_3,r_4)$.
The following lemma shows that, geometrically, quaternion multiplication
is a form of rotation on the quaternion sphere.

\begin{lemma}
\label{lem:multisrotation}
Quaternion multiplication is equivalent to rotation
on the quaternion sphere.
In particular, quaternion multiplication by a constant quaternion $q$
\[
  P(r) = qr
\]
where $qr$ is the quaternion multiplication of (\ref{eq:qmult-alg}),
is equivalent to constant rotation of a point on the quaternion sphere
\[
  P(r) = M_q r
\]
where $M_q$ is the rotation matrix
\begin{equation}
M_q = 
	\left( \begin{array}{cccc}
	q_1 & -q_2 & -q_3 & -q_4 \\
	q_2 & q_1  & -q_4 & q_3  \\
	q_3 & q_4  & q_1  & -q_2 \\
	q_4 & -q_3 & q_2  & q_1
	\end{array} \right)
\end{equation}
and $q = (q_1,q_2,q_3,q_4)$.
\end{lemma}
\prf
\[
M_q r = 
	\left( \begin{array}{cccc}
	q_1 & -q_2 & -q_3 & -q_4 \\
	q_2 & q_1  & -q_4 & q_3  \\
	q_3 & q_4  & q_1  & -q_2 \\
	q_4 & -q_3 & q_2  & q_1
	\end{array} \right)
	\left( \begin{array}{c} r_1 \\ r_2 \\ r_3 \\ r_4 \end{array} \right)
      = \left( \begin{array}{c}
	q_1 r_1 - q_2 r_2 - q_3 r_3 - q_4 r_4 \\
	q_1 r_2 + q_2 r_1 + q_3 r_4 - q_4 r_3 \\
	q_1 r_3 + q_3 r_1 + q_4 r_2 - q_2 r_4 \\
	q_1 r_4 + q_4 r_1 + q_2 r_3 - q_3 r_2
	\end{array} \right)
      = qr
\]
using (\ref{eq:qmult-alg}).
$M_q$ is a rotation matrix in 4-space, since its rows and columns are all unit
vectors and all rows are mutually orthogonal, as are all columns.
\QED

\begin{rmk}
$M_q$ is not the rotation matrix equivalent to the quaternion $q$.
\end{rmk}

\section{}

A frame can be represented by a point on the quaternion sphere.

\begin{theorem}
Let $r_(u_0)$ be a point of a curve $r(u)$.
The set of all tangent-following frames at $r(u_0)$
can be represented by a great circle on the quaternion sphere.
(In general, the set of all frames with a fixed coordinate axis pointing
along a fixed vector can be represented by a great circle on the quaternion sphere.)
\end{theorem}
\prf
There are many tangent-following frames at a point $r(u_0)$ of a curve $r(u)$,
because there is a degree of freedom in rotating the world $x$-axis
into the tangent $r'(u_0)$.
In particular, after any specific rotation $M$ of the world $x$-axis
to the tangent, the entire frame can rotate by an arbitrary amount
about this tangent.
This degree of freedom can be captured as a quaternion $q_1(\theta)$,
as follows.
For easier calculation, we place the degree of freedom at the beginning,
by first rotating an arbitrary amount about the $x$-axis,
followed by the rotation $M$ of the $x$-axis to the tangent.
As a quaternion, a rotation by the angle $\theta$ about the $x$-axis 
is $q_1(\theta) = (\cos \frac{\theta}{2}, \sin \frac{\theta}{2} (1,0,0))$,
$\theta \in [0,2\pi)$.
The rotation $M$ to the tangent can be calculated exactly (see Remark~\ref{}
following this proof), but for the purposes of this proof it is sufficient
that it is a fixed quaternion $q_2$.
Thus, a tangent-following frame at $r(u_0)$ can be represented
by the quaternion $q_2 q_1(\theta)$ for some $\theta \in [0,2\pi)$.
$q_1(\theta)$, $\theta \in [0,2\pi)$, is clearly a great circle
on the quaternion sphere.
By Lemma~\ref{lem:multisrotation}, 
$q_2 q_1(\theta)$ is a rotation of this great circle.
Thus, the set of all tangent-following frames at a fixed point $r(u_0)$
can be represented by the set of quaternions on a great circle
of the quaternion sphere.
\QED

In order to find the specific great circle $q_2 q_1(\theta)$ associated
with the tangent-following frames at $r(u_0)$,
we need to find $q_2$, the quaternion that rotates the $x$-axis into $r'(u_0)$.
Since the $i^{th}$ column of a rotation matrix is the image of the $i^{th}$
coordinate axis under the rotation \cite{foleyVanDam},
a rotation matrix that sends the world $x$-axis into $r'(u)$ is
%
\begin{equation}
M = ([r'(u)]_{\mbox{unit}} [r'(u) \times A]_{\mbox{unit}} 
	[r'(u) \times (r'(u) \times A)]_{\mbox{unit}} )
\end{equation}
%
where $V_{\mbox{unit}}$ is the unit vector $\frac{V}{\|V\|}$
and
\[
	A = \left\{ \begin{array}{ll}
		(1,0,0)	& \mbox{if} r'(u) \neq (1,0,0) \\
		(0,1,0) & \mbox{otherwise}
		\end{array} \right.
\]
Thus, one choice for $q_2$ is the quaternion associated with this matrix:
%
\begin{equation}
	q_2 = f_{\mbox{matrix} \rightarrow \mbox{quaternion}} 
	(([r'(u)]_{\mbox{unit}} [r'(u) \times A]_{\mbox{unit}} 
	 [r'(u) \times (r'(u) \times A)]_{\mbox{unit}} ))
\end{equation}
%
An algorithm for translating from quaternion to rotation matrix is available
in [Shoemake? \cite{} (but need to reverse the signs! get a better reference
such as Kane Spacecraft Dynamics)].
This circle is well defined by two of its non-antipodal points,
such as $q_2 q_1(0) = q_2 (1,0,0,0)$ and $q_2 q_1(\frac{\pi}{2})
= q_2 (0,1,0,0)$;
and indeed we never need to calculate more than these two points.

\section{}

Given a point $P$ on $S^3$ and a great circle $C$ on $S^3$ that does not contain
$P$ (Figure~\ref{}), the point $Q \in C$ closest to $P$ can be computed
as follows.

\bibliographystyle{plain}
\begin{thebibliography}{99}

\bibitem{hamilton69}
Hamilton, W. (1969)
Elements of Quaternions.
Chelsea (New York).

\end{thebibliography}

\end{document}
