\documentstyle[12pt]{article} 

\makeatletter
\def\@maketitle{\newpage
\null
 %\vskip 2em                   % Vertical space above title.
\begin{center}
      {\Large\bf \@title \par}  % Title set in \Large size. 
      \vskip .5em               % Vertical space after title.
      {\lineskip .5em           %  each author set in a tabular environment
\begin{tabular}[t]{c}\@author 
\end{tabular}\par}                   
\end{center}
\par
\vskip .5em}                 % Vertical space after author
\makeatother

\parskip=8pt plus1pt
\parindent=0pt

\newcommand{\DoubleSpace}{\edef\baselinestretch{1.4}\Large\normalsize}

\DoubleSpace

\setlength{\oddsidemargin}{0pt}
\setlength{\evensidemargin}{0pt}
\setlength{\headsep}{0pt}
\setlength{\topmargin}{0pt}
\setlength{\textheight}{8.75in}
\setlength{\textwidth}{6.5in}

\title{Exact Bezier Representations of Swept Spheres along Arbitrary Paths
	\thanks{This work supported by National Science Foundation 
	grant CCR9213918.}}
\author{John K. Johnstone 
	and James P. Williams\thanks{Department of Computer Science,
	The Johns Hopkins University,
	Baltimore, Maryland 21218 USA.}}

\begin{document}

\maketitle

Swept surfaces abound in the products that we wish to model in CAD/CAM
and geometric modeling in general,
from the motion of manufacturing tools such as ball cutters, lathes and robot 
arms to the blending and joining of surfaces.
In this paper, we are interested in the Bezier representation of swept surfaces.
This has generally been done for restricted
types of sweeping (e.g., rotational) or for approximations of the sweep.
We address the problem of arbitrary sweeps and exact representations
for the interesting problem of sweeping spheres.
This representation is exact in the sense that the swept surface that 
we create is consistent with all of the finite data we give it (see below).

We are motivated by the following scenario:
a sweeping sphere (or more general object) regularly outputs its position,
its instantaneous direction of movement, 
and size.\footnote{The object is allowed to scale during the sweep.
	If appropriate, the orientation of the object is also
	regularly output, but for spheres this is of course irrelevant.}
Thus, we have a finite number of positions, tangents, and sizes of 
the sweeping sphere.
We wish to create a Bezier representation for a swept surface
that interpolates this data, and is consistent with the fact
that the sweeping object is always a sphere.
%In particular, we will create a Bezier curve $c(t)$ that interpolates
%the position (center of the sphere), a Bezier function $r(t)$
%that interpolates the size (radius), impose a condition
%that guarantees a sweeping sphere, and create a surface from this data.

A sphere sweeping along a curve $c(t)$ called the directrix curve
can be represented by a circle sweeping along $c(t)$ such
that the plane of the circle is always normal to the directrix
curve (or equivalently the normal of the circle's plane is
always the tangent of the directrix curve).
Therefore, we represent a sweeping sphere by a sweeping circle,
which leads more simply to a parameterization without needing to compute
envelopes of spheres.
A sweeping circle is specified by a center, orientation, and radius
function, $c(t)$, $o(t)$, $r(t)$; and 
we have the added restriction that $o(t) = c'(t)$
since we wish to represent sweeping spheres.
The parameterization of the swept surface becomes
$C(t)*O(t)*R(t)*\mbox{circ}(s)$,
where $\mbox{circ}(s)$ is a piecewise Bezier parameterization 
of the unit circle at the origin in the $xy$-plane,
$C(t)$ is the matrix for translation by $c(t)$,
$O(t)$ is the matrix for rotation by $o(t)$,
and $R(t)$ is the matrix for scaling by $r(t)$.
We want this parameterization to be rational so that we can translate
it to a Bezier representation of the surface.
The difficulty lies with the orientation/rotation component.
By the nature of rotation matrices,
$O(t)$ will not be rational unless the length of the vector 
$o(t)$ can be rationally parameterized
(without loss of generality, unit length).
Since $o(t) = c'(t)$, this restricts the directrix curve $c(t)$ to
be a curve with tangents of unit (or rationally parameterized) length.
Curves of this form have been studied by Farouki, and were
dubbed Pythagorean hodograph curves.\footnote{This restriction on the
	form of the directrix curve shows why we cannot 
	consider the representation of a swept sphere
	where the position/size information is 
	given exactly at {\em all} points, rather than only at a finite
	number of points.  The directrix curve would not
	be a Pythagorean hodograph curve in general.}

We give a construction of quintic space Pythagorean hodograph splines,
which serve as the directrix curves of the sweeping spheres.
This extends Farouki's work, which had been predominantly on plane curves,
to space curves.
Also, by splining the Pythagorean hodograph curves 
together we avoid problems arising from the restricted freedom
of a single Pythagorean hodograph curve.

Since functions $o(t)$ of unit-length vectors
are equivalent to curves on the unit sphere,
this paper is related to recent work by Dietz, Hoschek and J\"uttler
that shows how to create a curve on the sphere that interpolates
a finite number of points.
However, their method cannot be applied to our problem
because of our added restriction that $o(t)$ agree with the tangent of $c(t)$.
Also, our method is closed-form, unlike theirs.
Nevertheless, their method can be used to generate valid orientation
functions that can be used with arbitrary center and radius functions
to generate a more general type of rational swept surface generated
by circles.

In summary, we generate Bezier representations
of arbitrary sweeps of spheres through the development of 
interpolating splines that consistently have tangents of polynomial length,
which are used as the directrix curves of the sweeps.

\end{document}
