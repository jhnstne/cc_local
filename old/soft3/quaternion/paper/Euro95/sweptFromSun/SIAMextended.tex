\documentstyle[l-extabs]{report} 
\begin{document}
 
\title{An exact sweep of a curve}

Consider a fixed curve sweeping through 3-space and suppose
that we want a rational Bezier model of the surface that it sweeps out.
We show how to build a rational swept surface that
interpolates $n$ frames of a fixed rational space curve 
({\em i.e.}, $n$ position/orientation pairs) 
and, more importantly, 
maintains the shape of the curve at all intermediate points of the sweep.
Thus, the surface models 
an exact sweep of the curve, consistent with the given data.
For added flexibility, we also allow the sweeping curve to change its
size through the sweep.
This swept surface models sweeping bodies that do not change shape,
such as in robotics and CAD/CAM,
and the frames can be generated automatically by the moving body.
For design, it encompasses the surface of revolution 
and extruded surface, and extends them to arbitrary sweeps of a curve.

We assume discrete data on the position and orientation
of the sweeping body for reasons both of application and design.
For the modeling of a physically sweeping body ({\em e.g.}, a robot), 
it is simple to gather discrete position/orientation data, 
but much more difficult to gather it continuously.
In design, it is usually appropriate to design the swept surface
by specifying only a finite number of frames.

The major challenge is to rationally preserve the shape of the curve 
through the sweep.
The interpolation of curves that can vary in shape during the sweep
can already be solved using the traditional tensor product surface.
And the shape of the curve can be preserved through the sweep
using Frenet frames if the surface need not be rational.
However, we want a rational model, for compatibility with 
present modeling systems.
We build a rational tensor product Bezier surface, complete with a full
definition of its control mesh.

There is a rich literature on swept surfaces, but none
that yields a rational parametric surface
representing an exact sweep for our large class of surfaces.
Several papers use the Frenet frame of differential geometry
to control the sweep of a curve ({\em e.g.}, Pegna 1988,
Bronsvoort and Klok 1985).
The path of the sweep is specified by a space curve $d(t)$,
which defines the position of the reference point on the
sweep curve, and the sweep curve's orientation is defined by
the Frenet frame of the position curve $d$.
The rotation defined by the Frenet frame is not rational,
and thus neither is the swept surface.  
The envelope of swept surfaces ({\em i.e.}, the implicit equation of the surface
swept out by a sweeping surface) has been the subject of many papers
({\em e.g.}, Martin and Stephenson 1990).
Coquillart and Bloomenthal and Riesenfeld 
construct swept surfaces by approximating offset curves
of the axis of the sweep, in the latter paper to define translational sweeps,
surfaces of revolution, and approximations of rotation-minimizing sweeps.
(The rotation-minimizing frame 
is an attempt to remove some of the problems of the Frenet frame.)
The important special class of canal surfaces, 
surfaces swept by a sphere, are studied by Rossignac and Requicha.
Farouki shows that rational canal surfaces can be constructed by sweeping
a sphere along a special Pythagorean hodograph space curve, and gives the
rational parameterization.
The ray tracing of swept surfaces (as opposed to their Bezier surface 
definition) has been studied by van Wijk (both with the sweep of a sphere
of varying radius and the rotation, translation,
and translation with scaling of a planar cubic spline)
and by Bronsvoort and Klok (for generalized cylinders).

Although our method directly models the sweep of a curve,
the sweep of a surface or a solid can be modeled indirectly
by using curve frames that represent the intersection of two 
instantaneously close frames of the surface.
Although this will not construct an 
exact model of an actual sweep of the surface,
it is likely to be a good approximation since the 
envelope of a surface is computed by such intersections
of neighbouring positions of the surface.

A curve is defined by its position, orientation, and scale.
We isolate each of these three components and independently 
design each of them to interpolate the frames.
Rational interpolation of position and scale is straightforward, 
but rational orientation interpolation is challenging.
The orientation of a curve is represented by a unit quaternion.
This reduces the problem to interpolation of points on the unit
sphere in 4-space by a curve constrained to this sphere.
This can be done using optimization but this will not create
a rational curve.
We show how to translate to an equivalent problem of interpolation
of points in 4-space without any constraints.

Once the three components (position, orientation, and scale) of the sweeping
curve are defined, they must be fused together into a surface representation.
The quaternion curve is translated to a rotation matrix 
and the control mesh is built by analyzing the
path of each control point of the curve.

In the presentation, 
we will show how to interpolate orientations constrained to the quaternion sphere
and how to explicitly build the control mesh of the surface.
We will also show pictures of surfaces generated by our method.

\end{document}

