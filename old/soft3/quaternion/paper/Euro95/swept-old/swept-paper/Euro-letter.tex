\documentstyle[11pt]{letter}
\address{\ \ }
\signature{Prof. John K. Johnstone}
\begin{document}
\begin{letter}
{Ms. Fran Melvin\\
Blackwell Publishers Limited\\
108 Cowley Road\\
Oxford OX4 1JF, UK
}

\opening{Dear Ms. Melvin:}

I enclose the camera-ready copy of the paper 
`A rational model of the surface swept by a curve'
by John K. Johnstone and James P. Williams,
for publication in the special Eurographics '95 issue of
Computer Graphics Forum.
(Please note that this paper was previously titled 
`A rational model of a sweeping curve'.  This was changed
at the advice of a referee.)
I also enclose two photocopies, the offprint order form,
and the copyright transfer form.

There are three figures in my paper (on the 10th and 11th pages).  
The paper includes postscript images of these figures, which
are good quality images.
However, I have also included laser copies of these pictures.
If you feel that the laser copies will reproduce more accurately,
please use them in place of the present images.

I have revised the paper in response to the comments of the
three referees.
The following changes have been made:
\begin{itemize}
\item
	Referee 1:
	The title has been changed and the list of references
	has been expanded (with the two noted articles).
	The clarity of the figures has been improved,
	and their size has been increased to as large an extent
	as the 12-page limit allowed.
	The discussion of speed of rotation in the
	first paragraph of Section 4 has been clarified.
	Reference to SO(3) has been replaced by reference
	to rotation matrices (the subclass of interest here).
\item
	Referee 2:
	A reference to Frenet frames has been added to the
	sentence `It is simple to preserve ...' at the top
	of page 2, to clarify.  
	This topic is also addressed in Section 2.

	In the first paragraph of Section 3, we are not
	stating that the surface passes through all of the control
	points, rather the control points of a column define the
	path of a control point of the sweeping curve.
\item
	Referee 3:
	We have added an explanatory footnote to step 3 of the
	algorithm of Section 4, explaining why we do not force
	the algorithm to interpolate directly in Bezier form:
	we wish to make clear that the designer has the freedom
	to choose any interpolation method, rather than
	unnecessarily restricting to Bezier immediately.

	Formula 3 is one of the tools we use to realize a rational
	map to the unit sphere in 4-space.  We have not ascribed
	this result to ourselves: it is a number theoretical result
	of Euler.

	We have elaborated, at the bottom of page 4, on our use
	of the term Gauss map, adding the words `of a surface'.
	The Gauss map, which sends a point of a surface to the point
	on the unit sphere associated with its unit normal,
	is certainly an example of a nonrational map to the sphere.
\end{itemize}

This addresses all of the comments of the referees.
Please do not hesitate to call me at (205) 934-2213,
fax at (205) 934-5473 or
email me at \\
johnstone@cis.uab.edu if you have any questions.
	
\closing{Sincerely,}
\cc {Frits H. Post, Martin G\"{o}bel}
\end{letter}
\end{document}
