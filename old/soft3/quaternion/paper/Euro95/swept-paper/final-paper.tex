\documentstyle[times]{article} 

\newif\ifFull
\Fullfalse

\makeatletter
\def\@maketitle{\newpage
\null
% \vskip 2em                   % Vertical space above title.
\begin{center}
        {\Large\bf \@title \par}  % Title set in \Large size. 
        \vskip .5em               % Vertical space after title.
       {\lineskip .5em           %  each author set in a tabular environment
        \begin{tabular}[t]{c}\@author 
        \end{tabular}\par}                   
\end{center}
\par
\vskip .5em}                 % Vertical space after author
\makeatother

% non-indented paragraphs with xtra space
% set the indentation to 0, and increase the paragraph spacing:
%% \parskip=8pt plus1pt                             
%% \parindent=0pt
% default values are 
% \parskip=0pt plus1pt
% \parindent=20pt
% for plain tex.

\newenvironment{summary}[1]{\if@twocolumn
\section*{#1} \else
\begin{center}
{\bf #1\vspace{-.5em}\vspace{0pt}} 
\end{center}
% \quotation
\fi}{\if@twocolumn\fi}
% {\if@twocolumn\else\endquotation\fi}

\renewenvironment{abstract}{\begin{summary}{Abstract}}{\end{summary}}

\newcommand{\SingleSpace}{\edef\baselinestretch{0.9}\Large\normalsize}
\newcommand{\DoubleSpace}{\edef\baselinestretch{1.4}\Large\normalsize}
\newcommand{\Comment}[1]{\relax}  % makes a "comment" (not expanded)
\newcommand{\Heading}[1]{\par\noindent{\bf#1}\nobreak}
\newcommand{\Tail}[1]{\nobreak\par\noindent{\bf#1}}
\newcommand{\QED}{\vrule height 1.4ex width 1.0ex depth -.1ex\ } % square box
\newcommand{\arc}[1]{\mbox{$\stackrel{\frown}{#1}$}}
\newcommand{\lyne}[1]{\mbox{$\stackrel{\leftrightarrow}{#1}$}}
\newcommand{\ray}[1]{\mbox{$\vec{#1}$}}          
\newcommand{\seg}[1]{\mbox{$\overline{#1}$}}
\newcommand{\tab}{\hspace*{.2in}}
\newcommand{\se}{\mbox{$_{\epsilon}$}}  % subscript epsilon
\newcommand{\ie}{\mbox{i.e.}}
\newcommand{\eg}{\mbox{e.\ g.\ }}
\newcommand{\figg}[3]{\begin{figure}[htbp]\vspace{#3}\caption{#2}\label{#1}\end{figure}}
\newcommand{\be}{\begin{equation}}
\newcommand{\ee}{\end{equation}}
\newcommand{\prf}{\noindent{{\bf Proof} :\ }}
\newcommand{\choice}[2]{\left( \begin{array}{c} \mbox{\footnotesize{$#1$}} \\ \mbox{\footnotesize{$#2$}} \end{array} \right)}      
\newcommand{\ddt}{\frac{\partial}{\partial t}}

\newtheorem{rmk}{Remark}[section]
\newtheorem{example}{Example}[section]
\newtheorem{conjecture}{Conjecture}[section]
\newtheorem{claim}{Claim}[section]
\newtheorem{notation}{Notation}[section]
\newtheorem{lemma}{Lemma}[section]
\newtheorem{theorem}{Theorem}[section]
\newtheorem{corollary}{Corollary}[section]
\newtheorem{defn2}{Definition}

% \ifFull                                          
% \SingleSpace
% \else
% \DoubleSpace
% \fi

\pagestyle{empty}

\setlength{\oddsidemargin}{2mm}
\setlength{\evensidemargin}{2mm}
\setlength{\headsep}{0pt}
\setlength{\topmargin}{0pt}
\setlength{\textheight}{216mm}
\setlength{\textwidth}{156mm}

\title{A rational model of the surface swept by a curve
% A rational model of a sweeping curve
        \thanks{This work was partially supported by 
	National Science Foundation grant CCR-9213918.}}
\author{John K. Johnstone\thanks{Dept. of Computer and Information Sciences,
	University of Alabama at Birmingham, 115A Campbell Hall,
	1300 University Boulevard, Birmingham, AL 35294 USA, 
	johnstone@cis.uab.edu.}
	\and James P. Williams\thanks{Dept. of Computer Science,
 	Johns Hopkins University, Baltimore, MD 21218 USA.}}

\begin{document}
\maketitle
\thispagestyle{empty}
% *********************************************************************
\begin{abstract}
This paper shows how to construct a rational Bezier model of a swept
surface that interpolates
$N$ frames ({\em i.e.}, $N$ position/orientation pairs)
of a fixed rational space curve $c(s)$ 
and maintains the shape of the curve at all intermediate points 
of the sweep.
Thus, the surface models
an exact sweep of the curve, consistent with the given data.
The primary novelty of the method is that 
this exact modeling of the sweep is achieved
without sacrificing a rational representation for the surface.
Through a simple extension, we also allow the sweeping curve to change
its size through the sweep.
The position, orientation, and size of the sweeping curve can change with
arbitrary continuity (we use $C^2$ continuity in this paper).
Our interpolation between frames has the classical properties
of Bezier interpolation, such as the convex hull property and linear precision.

This swept surface is a useful primitive for geometric design.
It encompasses the surface of revolution and extruded surface, but extends
them to arbitrary sweeps.
It is a useful modeling primitive for robotics and CAD/CAM, using frames generated
automatically by a moving robot or tool.
% A video is being prepared and will be presented at the conference if
% the paper is accepted.
\end{abstract}

\vspace{.1in}

\noindent {\bf Keywords}: geometric modeling, swept surface, Bezier surface,
	  interpolation, quaternion.

\section{Introduction}

Consider a curve sweeping through 3-space,
periodically outputting its position and orientation.
This paper shows how to construct a rational Bezier model of the swept
surface generated by this curve.
In particular, a rational swept surface $sweep(s,t)$ is constructed that 
(1) interpolates
$N$ frames of a fixed rational space curve $c(s)$ ({\em i.e.}, $N$ position/orientation pairs) 
and, more importantly, (2) maintains the shape of the curve at all intermediate points 
of the sweep.
Thus, the surface models 
an exact sweep of the curve, consistent with the given data.
For added flexibility, we also allow the sweeping curve to change its
size through the sweep.

This swept surface is a useful modeler for applications (such as
robotics and CAD/CAM) that involve sweeping bodies that do not change shape 
(such as robot arms and machining tools).
The frames can be generated automatically by the moving body.
Our swept surface is also a useful primitive for geometric design.
It encompasses the surface of revolution and extruded surface
(Section~\ref{sec:eg}), but extends them to arbitrary sweeps.

We assume discrete data on the position and orientation
of the sweeping body for reasons both of application and design.
For the modeling of a physically sweeping body ({\em e.g.}, a robot), 
it is simple to gather discrete position/orientation data, 
but much more difficult to gather it continuously.
In design also, it is usually appropriate to design the swept surface
by specifying only a finite number of frames.

Although our method directly models the sweep of a curve,
the sweep of a surface or a solid can be modeled indirectly
by using curve frames that represent the intersection of two 
"instantaneously close" frames of the surface,
{\em i.e.}, $C_i = S(t_i) \cap S(t_i+\epsilon)$.
Although this will not construct an 
exact model of an actual sweep of the surface,
it is likely to be a good approximation since the 
envelope of a surface is computed by such intersections
of neighbouring positions of the surface.

The major challenge is to preserve the shape of the curve through the sweep.
After all, the interpolation of curves that can vary in shape during the sweep
can already be solved using the general tensor product surface.
It is simple to preserve the shape of the curve through the sweep
if the surface need not be rational, using Frenet frames,
but we want a rational model.
% define a Hermite interpolating curve of position and use the Frenet frame
This is crucial for compatibility, 
since most present modeling systems work with rational Bezier or
B-spline curves and surfaces.
We build a rational tensor product Bezier surface, complete with a full
definition of its control mesh.

A curve is defined by its position, orientation, and scale.
The key idea to our modeling of a sweeping curve is to isolate each
of these three components and independently design each of them 
to interpolate the frames.
Interpolation of position and scale is straightforward.
However, orientation interpolation is challenging.

\begin{defn2}
{\em
A {\bf frame} of a curve $C$ is a pair ($P$,$O$) or a triple 
($P$,$O$,$S$)
defining the position $P$ (of the reference vertex), 
orientation $O$, and possibly scale $S$ of the curve.
Let $c(s)$ be a parameterization of $C$ in canonical position,
with its reference point at the origin.
If $P$, $O$, and $S$ are represented by translation, rotation, and scaling 
matrices, respectively, the frame $(P,O,S)$ represents the curve 
$P \cdot O \cdot S \cdot c(s)$.
}
\end{defn2}

Here is an outline of our method.
Let $c(s)$ be the sweep curve (in canonical position) and 
let $(P_i,O_i,S_i), i = 1,\ldots,N$ be its $N$ frames.

\begin{enumerate}
\item
	{\bf (Position)} Construct a rational curve
	interpolating the positions $P_i$;
	in this paper, the position curve is 
	a $C^2$-continuous cubic Bezier spline.
\item
	{\bf (Scale)} Construct a rational function
	interpolating the scales $S_i$; in this paper, 
	the scale function is $C^2$-continuous cubic Bezier.
\item
	{\bf (Orientation)}
\begin{description}
\item[(a)]
	Represent each orientation $O_i$ by a unit quaternion
	(a point on the unit 4-sphere).
	Construct a Bezier spline on the unit 4-sphere
	that interpolates the orientations $O_i$.
	In this paper, this curve is a $C^2$-continuous sextic Bezier spline.
\item[(b)]
	Translate the orientation function to rotation matrix form, $O(t)$.
\end{description}
\item 
	{\bf (Parameterization)}
	The parameterization of the swept surface is
\begin{equation}
\label{eqn:matrix}
	sweep(s,t) = P(t)O(t)S(t)c(s)
\end{equation}
	where $P(t)$ (resp., $S(t)$) is the translation (resp., scaling)
	matrix associated with the position curve (resp., scaling function).
\item
	{\bf (Control mesh)}
	Compute each column of the Bezier control mesh of $sweep(s,t)$,
	using the sweep of each control point of $c(s)$.
\end{enumerate}

Section~\ref{sec:orient-express} covers Step~3,
the construction of the orientation component of the sweep.
This method for rational interpolation of orientation
was developed originally in Johnstone and Williams \cite{jjjimbo94a}, 
where it was used to rationally interpolate keyframes for an animation.
Step 3(b), the translation from a quaternion representation of the
orientation function to a rotation matrix representation, is covered
in Section~\ref{sec:q2rot}.
Section~\ref{sec:theory} develops the 
theory of constructing a Bezier control mesh of the surface generated
by a sweeping curve, which is needed for Step 5.
The actual construction of the control mesh is presented in 
Section~\ref{sec:mesh}.
Examples of our swept surface are given in Section~\ref{sec:eg}
and we end with some conclusions.
In the next section, we begin with a review of related work on swept surfaces.

% ALTERNATIVE OPENING 1:
% The classical problem of point-interpolation is to design a curve
% that interpolates a finite set of points.
% This may be viewed as the path generated by sweeping a point so that
% it interpolates the given finite set.
% This problem can be lifted from point-interpolation to
% curve-interpolation: design a surface that interpolates a finite set of curves
% (all of the same shape).
% This has important applications to swept surfaces.

\section{Related work}
\label{sec:related}

There is a rich literature on swept surfaces.
Several papers use the Frenet frame of differential geometry
to control the sweep of a curve ({\em e.g.}, Pegna \cite{Pegna88},
Bronsvoort and Klok \cite{bronsklok85}).
The path of the sweep is specified by a space curve $d(t)$,
which defines the position of the reference point on the
sweep curve, and the sweep curve's orientation is defined by
the Frenet frame of the position curve $d$.
Unfortunately, the rotation defined by the Frenet frame is not rational,
and thus neither is the swept surface.  % see Pegna, p. 47
The envelope of swept surfaces ({\em i.e.}, the implicit equation of the surface
swept out by a sweeping surface) has been the subject of many papers
({\em e.g.}, Martin \cite{martinsteph90}, Wang and Wang \cite{WangWang86}).
\nocite{Flaquer92}
% Flaquer et. al. \cite{Flaquer92}).
% removed for lack of space
% Ganter \cite{ganter?whereisthis}
Coquillart \cite{coquillart87} and Bloomenthal and Riesenfeld 
\cite{bloom91} construct swept surfaces by approximating offset curves
of the axis of the sweep, in the latter paper to define translational sweeps,
surfaces of revolution, and approximations of rotation-minimizing 
sweeps.\footnote{The rotation-minimizing frame 
	% (see Klok \cite{klok86})
        is an attempt to remove some of the problems of the Frenet frame.}
Canal surfaces are studied by Rossignac and Requicha 
\cite{rossrequicha84,rossignac85}.
Farouki shows that rational canal surfaces can be constructed by sweeping
a sphere along a special Pythagorean hodograph space curve, and gives the
rational parameterization \cite{farouki94}.
% only works for rotationally symmetric circle sweep curve
The ray tracing of swept surfaces (as opposed to their Bezier surface 
definition) has been studied by van Wijk (both with the sweep of a sphere
of varying radius \cite{vanwijk84euro} and the rotation, translation,
and translation with scaling of a planar cubic spline \cite{vanwijk84tog})
and by Bronsvoort and Klok \cite{bronsklok85} (for generalized cylinders).
Other papers that deal with generalized cylinders 
include Bronsvoort and Waarts \cite{bronswaart92} and Park and Kim 
\cite{parkkim93}.

\section{Sweeping a curve}
\label{sec:theory}

If a polynomial Bezier curve, $\sum_{i=0}^{m} \vec{b_i} B_{i}^{m}(u)$,
is swept through space so that its $i$th control point follows the path 
$\vec{b_i} = \sum_{j=0}^{n} \vec{b_{ij}} B_{j}^{n}(v)$,
the surface that is swept out by the curve
can be expressed as the tensor product surface 
$\sum_{i=0}^{m} \sum_{j=0}^{n} \vec{b_{ij}} B_{i}^{m}(u) B_{j}^{n}(v)$.
This is the classical interpretation of the $i^{th}$ column of a tensor
product Bezier surface, $\{\vec{b_{ij}}: j=0,\ldots,n\}$, 
as the control polygon of the curve swept out by the $i^{th}$ control-point
(Farin \cite{farin93}).
We will use this observation to model our swept surface:
we shall record the path of each control point of the
sweep curve, and then find the control polygon
of this path, which becomes a column of the tensor product control mesh.

Since our application involves rational curves sweeping along rational paths,
we first need to generalize the above argument.
When a rational Bezier curve,
$\frac{\sum_{i=0}^{m} w_i \vec{b_i} B_i^{m}(u)}{\sum_{i=0}^{m} w_i B_i^{m}(u)}$,
is swept through space so that its $i$th control point follows
a polynomial Bezier path, 
$\vec{b_{i}}(v) = \sum_{j=0}^{n} \vec{b_{ij}} B_{j}^{n}(v)$,
then the resulting rational tensor product surface is
% \frac{\sum_{i=0}^{m} \sum_{j=0}^n w_i b_{ij} B_i^{m}(u) B_j^n(v)}
% {\sum_{i=0}^{m} w_i B_i^{m}(u)}
(using the convex hull property of Bernstein polynomials)
\[
\frac{\sum_{i=0}^{m} \sum_{j=0}^n w_i \vec{b_{ij}} B_i^{m}(u) B_j^n(v)}
{\sum_{i=0}^{m} \sum_{j=0}^n  w_i B_i^{m}(u) B_j^n(v)}
\]
That is, the sweep $\vec{b_{i}}(v)$ of the curve's $i^{th}$ control point still
determines the $i^{th}$ column of the mesh,
with the weights of this column identical to the weight of $\vec{b_i}$.

Finally, we can generalize to the sweep of a rational Bezier
curve such that each control point follows a {\em rational} path,
as long as the weights of the rational paths are the same for each path.
That is, if a rational Bezier curve,
$\frac{\sum_{i=0}^{m} w_i \vec{b_i} B_i^{m}(u)}{\sum_{i=0}^{m} w_i B_i^{m}(u)}$,
is swept through space so that its $i$th control point follows
a rational Bezier path, 
$\vec{b_i(v)} = \frac{\sum_{j=0}^{n} w_j \vec{b_{ij}} B_{j}^{n}(v)}
{\sum_{j=0}^{n} w_j B_{j}^{n}(v)}$
(notice that the weight $w_j$ does not depend on $i$),
then the resulting rational tensor product surface is
\begin{equation}
\label{eq:ratrat}
\frac{\sum_{i=0}^{m} \sum_{j=0}^n w_i w_j \vec{b_{ij}} B_i^{m}(u) B_j^n(v)}
{\sum_{i=0}^{m} \sum_{j=0}^n w_i w_j B_i^{m}(u) B_j^n(v)}
\end{equation}
% (since the interior denominator $\sum_{j=0}^{n} w_j B_{j}^{n}(v)$ can be pulled out
% of the $i$-sum and thus into the denominator).
The sweep of the curve's $i^{th}$ control point again determines the
$i^{th}$ column of the mesh,
but now the weights are the product of the weights of the path and sweep curve.

Thus, we have reduced our problem to finding a path for each control point of
the sweeping curve since, from this information, 
we can construct the surface's control mesh.

\section{Rational interpolation of orientation}
\label{sec:orient-express}

In this section, we address the interpolation of the $N$ orientations
of the sweeping curve.
Although we will eventually want to represent the orientation 
by a rotation matrix, we initially 
represent orientation by a unit quaternion.
The unit quaternion
$(\cos \frac{\theta}{2}, \vec{v} \sin \frac{\theta}{2})$, \ $\|\vec{v}\| = 1$
corresponds to a rotation of $\theta$ about the axis $\vec{v}$ 
\cite{hoschek+lasser93}.
Since a single rotation about an axis is sufficient
to represent an arbitrary orientation of a solid object,
unit quaternions are a representation for rigid body 
orientation.\footnote{Although the orientation of some special curves
     can be represented more simply ({\em e.g.}, a circle by a normal 
     vector), the orientation of a general space curve has the same degrees
     of freedom as a rigid body.}
Unlike Euler angles, quaternions have a unique representation 
for each orientation, do not experience gimbal lock, and can be
combined easily.
Unlike both Euler angles and rotation matrices,
the quaternion has a concise representation (4 numbers) with a natural
geometric analogue (through identification of the set of unit quaternions 
with the unit sphere $S^3$ in 4-space).
It is this latter property that is of the utmost importance for 
our solution of the
interpolation problem: orientation interpolation can be posed as
a problem of interpolating points on a 4-sphere.
In the rest of this section, a unit quaternion will be identified
with a point on $S^3$, the unit sphere in 4-space, which will be
called the {\em quaternion sphere}.
We also note here that the unit quaternion allows control and prediction
of the speed of rotation of the sweeping curve, since the amount of
rotation is directly encoded by distance on the sphere.\footnote{That is,
	the distance metric of the sphere $S^3$ is equivalent to the
	angular metric of rotation matrices \cite{misner73}.}

We now present an outline of our method for orientation interpolation.
The major idea is to design an interpolating curve freely in 4-space
using traditional techniques and then map back onto the quaternion sphere
(using a map $M$ from 4-space onto the sphere).
This approach is reminiscent of a technique of Dietz, Hoschek, and J\"{u}ttler 
\cite{Dietz93} for designing rational curves on the unit sphere in 3-space,
with interesting departures.
The input is a set of $n$ orientations of a solid
represented as unit quaternions $q_1,\ldots,q_n$.

\begin{enumerate}
\item
	Construct a rational map $M$ from 4-space onto the unit sphere in 
	4-space.
\item
	({\bf Map quaternions into 4-space})
	Map the quaternions $q_i$ by $M^{-1}$ into 4-space.
\item
	({\bf Interpolate in 4-space})
	Interpolate the points $M^{-1}(q_i)$ in 4-space
	by an arbitrary\footnote{Any method of interpolation 
	can be used in this step, as long as it can be translated to a
	a Bezier spline.  We need to translate to a Bezier spline so that
	the method of step (5) can be applied.}
	polynomial curve $C(t)$, say a cubic B-spline.
\item
	({\bf Translate to Bezier spline})
	Translate $C(t)$ to the equivalent cubic Bezier spline 
	$C_{\mbox{bez}}(t)$.
\item
	({\bf Map back onto the sphere})
	Map $C_{\mbox{bez}}(t)$ back to the sphere using $M$, 
	one Bezier segment at a time, yielding a  
	Bezier spline $D_{\mbox{bez}}(t)$. 
\end{enumerate}

$D_{\mbox{bez}}(t)$ is the desired orientation curve, a Bezier spline
that interpolates the quaternions $q_i$ and defines a valid orientation
at every time $t$.
In our examples, since we use a cubic B-spline in step (3),
$D_{\mbox{bez}}(t)$ is $C_2$-continuous.
Since steps (3) and (4) are well understood (Farin \cite{farin93}),
we need only show how to construct $M$, its inverse $M^{-1}$,
and the map of a Bezier segment by $M$.

We want a rational\footnote{A map $\alpha(t) = (x_1(t),\ldots,x_k(t))$
	is rational if each component $x_i(t)$ can be expressed
	as the quotient of two polynomials in $t$.}
map from 4-space onto the unit sphere $S^3$ in
4-space.
The challenge is to make the map rational,
since it is simple to find non-rational maps onto the sphere 
({\em e.g.}, normalization, Gauss map of a surface).
A formula from number theory yields a solution
\cite{Dickson52}[p. 318].

\begin{lemma}[Euler, Aida]
\begin{equation}
\label{eqn:aida}
(a^2 + b^2 + c^2 - d^2)^2 + (2ad)^2 + (2bd)^2 + (2cd)^2 = 
(a^2 + b^2 + c^2 + d^2)^2
\end{equation}
\end{lemma}

It is most natural to express the following map in projective 4-space, $P^4$.
$(x_1,x_2,x_3,x_4,x_5)$ in projective 4-space is equivalent to
$(\frac{x_1}{x_5},\frac{x_2}{x_5},\frac{x_3}{x_5},\frac{x_4}{x_5})$ 
in affine 4-space.

\begin{corollary}
The map 
$M:P^4 \rightarrow S^3 \subset P^4$:
\begin{equation}
\label{eq:M}
	M(x_1,x_2,x_3,x_4,1) = \mbox{\footnotesize{$\left( \begin{array}{c}
		x_1^2+x_2^2+x_3^2-x_4^2 \\
		2x_1x_4 \\
		2x_2x_4 \\
		2x_3x_4 \\
		x_1^2+x_2^2+x_3^2+x_4^2
		\end{array} \right)$}}
\end{equation}
sends a point in projective 4-space onto the unit sphere in projective
4-space.
\end{corollary}

\begin{lemma}
\label{lem:invM}
The map $M^{-1}:S^3 \rightarrow P^4$ is defined by
\begin{equation}
\label{eq:invM}
M^{-1}(x_1,x_2,x_3,x_4,x_5)=
\left\{ \begin{array}{ll}
(x_2,x_3,x_4,x_5-x_1,+2\sqrt{\frac{x_5-x_1}{2}}) 
	& \mbox{if } (x_1,x_2,x_3,x_4,x_5) \neq (1,0,0,0,1) \\
	& \mbox{(equivalently, if } x_1 \neq x_5 \mbox{)} \\
\mbox{the hyperplane } x_4 = 0 
	& \mbox{otherwise}
\end{array} \right.
\end{equation}
where $(x_1,x_2,x_3,x_4,x_5)$ lies on the unit sphere $S^3$.
\end{lemma}

Note that $M$ is surjective ({\em i.e.}, it maps onto the entire sphere)
since every point of the 4-sphere has an inverse image.
When the quaternion $q_i = (q_{i1},q_{i2},q_{i3},q_{i4})$ is mapped off
of the sphere using $M^{-1}$,
we apply the map $M^{-1}(q_{i1},q_{i2},q_{i3},q_{i4},1)$.
That is, we simply set the homogeneous coordinate to 1.
This maps the quaternion to a unique point.
The inverse of the projective point 
$\{(kq_{i1},kq_{i2},kq_{i3},kq_{i4},k)\}_{k \in \Re}$
is a line,
but we choose a unique preimage in order to apply point interpolation.

To take advantage of the Bezier representation,
the Bezier spline in 4-space (see step 3 above)
must be mapped back to the sphere
as a Bezier curve, not simply a rational curve.
The image of each cubic Bezier segment of the spline is a sextic Bezier segment.
The following lemma reveals the Bezier structure of these sextic
image segments on the sphere.
Notice that the structure of the map $M$ is preserved in the control
points (compare (\ref{eq:control-pts}) and (\ref{eq:M})).

\begin{lemma}
\label{sextic}
The image of a polynomial cubic Bezier segment $c(t)$ in 4-space 
under the map $M$
is a rational Bezier segment of degree 6 with control points $c_k$
and weights $w_k$ ($k = 0, \ldots, 6$):
\begin{equation}
\label{eq:control-pts}
c_k = \sum_{\begin{array}{c} 0 \leq i \leq 3 \\ 
			     0 \leq j \leq 3 \\ 
			     i+j=k
			     \end{array}} 
        \frac{\choice{3}{i} * \choice{3}{j}}{\choice{6}{k}}
	\left( \begin{array}{c}
            (b_{i1} b_{j1} + b_{i2} b_{j2} + b_{i3} b_{j3} - b_{i4} b_{j4}) / w_k \\
            2b_{i1} b_{j4} / w_k \\
            2b_{i2} b_{j4} / w_k \\
            2b_{i3} b_{j4} / w_k
	\end{array} \right)
\end{equation}
\begin{equation}
w_k = \sum_{\begin{array}{c} 0 \leq i \leq 3 \\ 
			     0 \leq j \leq 3 \\ 
			     i+j=k
			     \end{array}}
        \frac{\choice{3}{i} * \choice{3}{j}}{\choice{6}{k}}
	(b_{i1} b_{j1} + b_{i2} b_{j2} + b_{i3} b_{j3} + b_{i4} b_{j4})
\end{equation}
where $b_i = (b_{i1},b_{i2},b_{i3},b_{i4})$ are the control points of $c(t)$
($i=0,1,2,3$).
\end{lemma}
\prf
This can be established by an application of the product rule
of Bernstein polynomials (Farin \cite{farin93})
and letting $i+j=k$.
A full proof is in Johnstone and Williams \cite{jjjimbo94a}.
\QED

In Johnstone and Williams \cite{jjjimbo94a}, we show that the above
method generates good orientation functions, with all of the classical
advantages of Bezier curves.
In summary, given a finite number of orientations, we have a method
for defining a rational orientation function that interpolates them well.


\section{Weaponry}
\label{sec:weapon}

The following lemma allows the product of two Bezier functions to be
expressed as a single Bezier function of higher degree, and will be
used in future sections.
It is derived using the product rule of Bernstein polynomials \cite{farin93}.

\begin{lemma}
\label{lem:product}
\begin{equation}
\label{eq:m_n}
\sum_{i=0}^{m} B_i^m(t) b_i \ \ \sum_{j=0}^{n} B_j^n(t) c_j
= \sum_{k=0}^{m+n} B_k^{m+n}(t) d_{ij}
\end{equation}
where
\begin{equation}
\label{eq:dij}
d_{ij} := \sum_{\begin{array}{c} 0 \leq i \leq m \\ 
			     0 \leq j \leq n \\ 
			     i+j=k
			     \end{array}}
	\frac{\choice{m}{i} \choice{n}{j}}{\choice{m+n}{k}}  b_i c_j
\end{equation}
for $b_i,c_j \in \Re$.
\end{lemma}
% \prf
%% Apply the product rule of Bernstein polynomials \cite{farin93} and
%% let $k=i+j$.
% $(\sum_{i=0}^{m} B_i^m(t) b_i) \ \ (\sum_{j=0}^{n} B_j^n(t) c_j)
% = \sum_{i=0}^{m} \sum_{j=0}^{n} B_i^m(t) B_j^n(t) b_i c_j$.
% By the product rule of Bernstein polynomials \cite{farin93},
% this is equal to
% $\sum_{i=0}^{m} \sum_{j=0}^{n} 
% 	\frac{\choice{m}{i} \choice{n}{j}}{\choice{m+n}{i+j}}
%	B_{i+j}^{m+n} b_i c_j$.
%Let $k=i+j$.
%%\QED

Henceforth, we will use the shorthand notation 
$\sum_{\{ijk\}}^{m+n} b_i c_j$ for the new control points $d_{ij}$
in (\ref{eq:dij}).
Notice that $k$ is a free variable in this expression.
Formula (\ref{eq:m_n}) becomes
\begin{equation}
\label{eq:shorthand}
\sum_{i=0}^{m} B_i^m(t) b_i \ \ \sum_{j=0}^{n} B_j^n(t) c_j
= \sum_{k=0}^{m+n} B_k^{m+n}(t) \sum_{\{ijk\}}^{m+n} b_i c_j
\end{equation}

In future sections,
we will work in projective space for cleanliness.
Rational Bezier curves and surfaces will also be represented in projective
space.
For example, the rational Bezier curve 
$\sum_{i=0}^{n} B_i^n(t)$ $(b_{i,1},\ldots,b_{i,d+1})$ in projective space is
equivalent, in affine space, to the rational curve \\
$\frac{\sum_{i=0}^{n} b_{i,d+1} B_i^n(t) 
	(\frac{b_{i,1}}{b_{i,d+1}},\ldots,\frac{b_{i,d}}{b_{i,d+1}})}
      {\sum_{i=0}^{n} b_{i,d+1} B_i^n(t)}$
with weights $b_{i,d+1}$.

\section{From quaternion to rotation matrix}
\label{sec:q2rot}

In Section~\ref{sec:orient-express}, we expressed the orientation function in terms
of quaternions.
However, we eventually need to represent it in terms of a rotation matrix,
in order to apply it in the 
formula (\ref{eqn:matrix}).\footnote{Theoretically, we could use
	quaternion multiplication rather than matrix multiplication, but
	the rotation matrix simplifies our development.}
Therefore, we need a mechanism to translate from quaternions to
rotation matrices.
The following lemma is the projective version of a 
formula in Shoemake \cite{shoemake85}.


\begin{lemma}
\label{lem:qtorot}
The rotation matrix that represents the same orientation as the 
% unit quaternion $(x_1,x_2,x_3,x_4)$ is:
% \[
% 	\left( \begin{array}{ccc}
% 	1 - 2(x_3^2 + x_4^2)	& 2(x_2x_3 + x_1x_4)	& 2(x_2x_4 - x_1x_3) \\
%	2(x_2x_3 - x_1x_4)	& 1-2(x_2^2 + x_4^2)	& 2(x_3x_4 + x_1x_2) \\
%	2(x_2x_4 + x_1x_3)	& 2(x_3x_4 - x_1x_2)	& 1-2(x_2^2 + x_3^2)
%	\end{array} \right)
% \]
unit quaternion \\$(x_1,\ldots,x_5)$ is
\begin{equation}
	\frac{1}{x_5^2}
	\left( \begin{array}{ccc}
	x_5^2 - 2(x_3^2 + x_4^2)& 2(x_2x_3 - x_1x_4)	& 2(x_2x_4 + x_1x_3) \\
	2(x_2x_3 + x_1x_4)	& x_5^2-2(x_2^2 + x_4^2)& 2(x_3x_4 - x_1x_2) \\
	2(x_2x_4 - x_1x_3)	& 2(x_3x_4 + x_1x_2)	& x_5^2-2(x_2^2 + x_3^2)
	\end{array} \right)
\end{equation}
\end{lemma}
%%\prf
%% This is a projective version of the formula in Shoemake \cite{shoemake85}.
% except that Shoemake in SIGGRAPH gets the angle reversed, so off-diagonal
% elements are slightly wrong (this can be explained by his using 
% q^{-1}.v.q as the rotation for quaternions, rather than the correct q.v.q^{-1}.
% We substitute $(x1/x5,...,x4/x5)$ into Shoemake's formula.
%%\QED

\begin{lemma}[Orientation function]
\label{lem:Mij}
Let $q(t) = \sum_{k=0}^6 B_k^6(t) (q_{k1},\ldots,q_{k5})$
be a segment of the orientation function, expressed as a quaternion.
The same segment of the orientation function, expressed as a parameterized
rotation matrix, is:
\begin{equation}
\label{eq:Ot}
O(t) = \frac{\sum_{k=0}^{12} B_k^{12}(t) \sum_{\{ijk\}}^{6+6}\ \ M_{ij}}
	    {\sum_{k=0}^{12} B_k^{12}(t) \sum_{\{ijk\}}^{6+6} \ \ q_{i5}q_{j5}}
\end{equation}
where
\[
	M_{ij} = \left( \begin{array}{ccc}
	q_{i5}q_{j5} - 2(q_{i3}q_{j3} + q_{i4}q_{j4}) &
	2(q_{i2}q_{j3} + q_{i1}q_{j4}) &
	2(q_{i2}q_{j4} - q_{i1}q_{j3}) \\
	2(q_{i2}q_{j3} - q_{i1}q_{j4}) &
	q_{i5}q_{j5} - 2(q_{i2}q_{j2} + q_{i4}q_{j4}) &
	2(q_{i3}q_{j4} + q_{i1}q_{j2}) \\
	2(q_{i2}q_{j4} + q_{i1}q_{j3}) &
	2(q_{i3}q_{j4} - q_{i1}q_{j2}) &
	q_{i5}q_{j5} - 2(q_{i2}q_{j2} + q_{i3}q_{j3}) 
	\end{array} \right)
\]
\end{lemma}
\prf
Apply Lemma~\ref{lem:qtorot} and then Lemma~\ref{lem:product}.
% Lemma~\ref{lem:qtorot} can be used to translate $q(t)$ 
% into a rotation matrix.
% Consider a typical entry of the rotation matrix equivalent 
% to the quaternion $q(t) = \sum_{k=0}^6 B_k^6(t) (q_{k1},\ldots,q_{k5})$,
% say the top left entry:
% \[
% \frac{	(\sum_{k=0}^6 B_k^6(t) q_{k5})^2 - 
% 	2(\sum_{k=0}^6 B_k^6(t) q_{k3})^2 - 2(\sum_{k=0}^6 B_k^6(t) q_{k4})^2}
%      {  (\sum_{k=0}^6 B_k^6(t) q_{k5})^2 }
% \]
% By Lemma~\ref{lem:product} (in the simplified form of (\ref{eq:shorthand})), 
% nthis is equivalent to
% \[
% \frac{	\sum_{k=0}^{12} B_k^{12}(t) \sum_{\{ijk\}}^{6+6}
% 	\ \ (q_{i5} q_{j5} - 2q_{i3}q_{j3} - 2q_{i4}q_{j4}) }
%      {  \sum_{k=0}^{12} B_k^{12}(t) \sum_{\{ijk\}}^{6+6} \ \ q_{i5} q_{j5}} 
% \]
% Expanding each entry in this way leads to (\ref{eq:Ot}).
\QED

\section{Control mesh of the swept surface}
\label{sec:mesh}

We now want to construct the tensor product Bezier control mesh 
for our swept surface.
The following theorem shows how each patch is created.
We postpone the incorporation of curve scaling until the next theorem.

\begin{theorem}
\label{thm:ten1}
Let
\[
c(s) = \frac{\sum_{i=0}^n w_i^{c} b_i^{c} B_i^n(s)}
	    {\sum_{i=0}^n w_i^{c} B_i^n(s)}
\]
be a segment of the rational Bezier spline of degree $n$
representing the sweeping curve, in orientation $(1,0,0,0)$
with its reference point at the origin.
\begin{itemize}
\item
Let $\{(P_i,O_i)\}$, $i=1,\ldots,N$, be $N$ frames of $c(s)$.
\item
Let $P(t)$ be a cubic Bezier spline interpolating the points $P_i$
(representing the position component of the sweep).
\item
Let 
\[
O(t) = \frac{\sum_{j=0}^{12} B_j^{12}(t) \sum_{\{klj\}}^{6+6}\ \ M_{kl}}
	    {\sum_{j=0}^{12} B_j^{12}(t) \sum_{\{klj\}}^{6+6} \ \ q_{k5}q_{l5}}
\]
as in (\ref{eq:Ot}),
be a segment of the rotation matrix Bezier function
interpolating the orientations $O_i$ 
(representing the orientation component of the sweep).
\item
Let $P^{+}(t) = \sum_{j=0}^{12} p_j^{+} B_j^{12}(t)$
be the degree-elevated Bezier segment equivalent to $P(t)$, 
computed using classical degree elevation (Farin \cite{farin93}).
% \footnote{If $\sum_{i=0}^n b_i B_i^{n}(t) = \sum_{i=0}^{n+1} b_i^{(1)} 
% 								B_i^{n+1}(t)$,
% 	then $b_i^{(1)} = \frac{i}{n+1} b_{i-1} + (1-\frac{i}{n+1})b_i$.}
\end{itemize}

We wish to construct a rational swept surface $SWEEP(s,t)$ satisfying 
the following two conditions:
\begin{enumerate}
\item $SWEEP(s,t)$ interpolates each of the $N$ frames
\item each isoparametric curve $\{SWEEP(s,k) : s \in \Re \}$
	is a rigid transformation of $c(s)$
\end{enumerate}
The patch of this swept surface corresponding to the sweep of the segment
$c(s)$ along the segment $P^{+}(t)$ of the position curve, controlled 
by orientation $O(t)$, is the tensor product Bezier surface with control points
(for $i=0,\ldots,n$, $j=0,\ldots,12$):
\[
b_{ij} = p_j^{+} + \frac{\sum_{\{klj\}}^{6+6} M_{kl} b_i^{c}}{w_j^{path}}
\]
and weights $w_{ij} = w_i^{c}  w_j^{path}$,
where $w_j^{path} = \sum_{\{klj\}}^{6+6} q_{k5} q_{l5}$.
\end{theorem}
\prf
As discussed in Section~\ref{sec:theory},
the $i^{th}$ column of the swept surface's control mesh is defined by
the path of the $i^{th}$ control point of the sweep curve $c(t)$, $b_i^{c}$, 
during the sweep.
This path is $P(t) + O(t)b_i^{c}$, because the position of the sweep
curve's reference point at time $t$ is $P(t)$ and the $i^{th}$ control
point is at position $b_i^c$ in relation to the reference point, rotated
by the orientation at time $t$.
Since matrix multiplication is distributive,
\[
	O(t)b_i^{c} = 
\frac{\sum_{j=0}^{12} B_j^{12}(t) \sum_{\{klj\}}^{6+6}
	\ \ (M_{kl} b_i^{c}) }
     {\sum_{j=0}^{12} B_j^{12}(t) \sum_{\{klj\}}^{6+6} 
	\ \ q_{k5}q_{l5}}
\]
Moving to projective space for clarity, 
$O(t)b_i^{c}$ is the rational curve
\[
	\sum_{j=0}^{12} B_j^{12}(t) 
	\left(	\begin{array}{c}
	\sum_{\{klj\}}^{6+6} M_{kl} b_i^{c} \\ w_j^{path}
	\end{array} \right)
\]
where $w_j^{path} = \sum_{\{klj\}}^{6+6} q_{k5} q_{l5}$.
Adding the position component yields
\[
P(t) + O(t)b_i^{c} = 
	\sum_{j=0}^{12} B_j^{12}(t) 
	\left(	\begin{array}{c}
	w_j^{path} p_j^{+} + \sum_{\{klj\}}^{6+6} M_{kl} b_i^{c} \\ 
	w_j^{path}
	\end{array} \right)
\]
which is the rational curve followed by the $i^{th}$ control point
of $c(t)$ during the sweep.
This rational curve has control points $b_{ij}$ (as defined above)
and weights $w_j^{path}$.
Notice that the weights of the $i^{th}$ rational path 
(the path followed by $b_i^c$) are 
independent of $i$, as required by (\ref{eq:ratrat}).
Using (\ref{eq:ratrat}), 
the $i^{th}$ column of the swept surface's control mesh consists
of the control points $b_{ij}$
and weights $w_i^{c} w_j^{path}$ ($j=0,\ldots,12$).
\QED

Using this result, one can easily construct the entire surface
(for the entire sweep curve over the entire sweep path).

\begin{corollary}
The resulting swept surface is a (12,n) tensor product Bezier surface,
with columns of degree 12 and rows of degree $n$
(where $n$ is the degree of the sweeping curve).
\end{corollary}

We now show how to extend this result so that the sweep curve
is allowed to change its size during the sweep.
Recall that the scaling component is easily computed by interpolating the
$N$ scale factors by a $C^2$-continuous cubic Bezier spline.
% We choose to change the curve size with $C^{2}$ continuity, just like the
% position and orientation.
% Since this creates a Bezier surface of higher degree (15 vs. 12),
% we isolate this result from Theorem~\ref{}.

\begin{theorem}
\label{thm:ten2}
Let $c(t)$, $P(t)$, and $O(t)$ be as in Theorem~\ref{thm:ten1},
representing segments of the sweeping curve and its position and orientation components.
\begin{itemize}
\item
Let $\{(P_i,O_i,S_i)\}$, $i=1,\ldots,N$, be $N$ frames of $c(t)$.
\item
Let $S(t) = \sum_{m=0}^{3} B_m^3(t) s_m$, $s_m \in \Re$
be the segment of the scaling component of the sweep (interpolating the scales
$S_i$) that corresponds to $P(t)$ and $O(t)$.
\item
Let $P^{+}(t) = \sum_{j=0}^{15} p_j^{+} B_j^{15}(t)$
be the degree-elevated version of $P(t)$ (but now raised to degree 15).
\end{itemize}

We wish to construct a rational swept surface $SWEEP(s,t)$ satisfying 
the following two conditions:
\begin{enumerate}
\item $SWEEP(s,t)$ interpolates each of the $N$ frames, and
\item  each isoparametric curve $\{SWEEP(s,k) : s \in \Re \}$
	is an affine transformation of $c(s)$
\end{enumerate}
The patch of this swept surface corresponding to the sweep of the segment
$c(s)$ along the segment $P^{+}(t)$ of the position curve, controlled 
by orientation $O(t)$ and scale $S(t)$, 
is the tensor product Bezier surface with control points
(for $i=0,\ldots,n$, $j=0,\ldots,12$):
\[
b_{ij} = p_j^{+} + \frac{\sum_{\{hmj\}}^{12+3} \ \ 
		(\sum_{\{klh\}}^{6+6} M_{kl} b_i^c) \ \ s_m}{w^{(3)}_j}
\]
and weights $w_{ij} = w_i^c  w^{(3)}_j$,
where $w^{(3)}_j$ is computed using degree elevation from 
$w_j^{path} = \sum_{\{klj\}}^{6+6} q_{k5} q_{l5}$:
$\sum_{j=0}^{12} w_j^{path} B_j^{12}(t) = 
   \sum_{j=0}^{15} w_j^{(3)}  B_j^{15}(t)$.
% \begin{equation}
% \label{eq:w2}
% w^{(2)}_j = \frac{j(j-1)w_{j-2}^{path} + 2j(14-j)w_{j-1}^{path} 
% 		+ (14-j)(13-j)w_j^{path}}{13(14)}
% \end{equation}
\end{theorem}
\prf
The path of the $i^{th}$ control point of the sweep curve $c(t)$,
$b_i^c$, is $P(t) + O(t)S(t)b_i^c$,
which defines the $i^{th}$ column of the surface's control mesh.
We have seen that 
\[
O(t) b_i^c = 	\sum_{h=0}^{12} B_h^{12}(t) 
	\left(	\begin{array}{c}
	\sum_{\{klh\}}^{6+6} M_{kl} b_i^c \\ w_h^{path}
	\end{array} \right)
\]
where $w_h^{path} = \sum_{\{klh\}}^{6+6} q_{k5} q_{l5}$.
Thus,
\[
O(t)S(t)b_i^c = 
	\sum_{h=0}^{12} B_h^{12}(t) 
	\left(	\begin{array}{c}
	\sum_{\{klh\}}^{6+6} M_{kl} [(\sum_{m=0}^3 B_m^3(t) s_m) b_i^c] \\
	w_h^{path}
	\end{array} \right)
\]
and, since $m$ is independent of the other variables, 
we can pull out terms involving $m$ and apply Lemma~\ref{lem:product}:
%\[
% = \frac{(\sum_{h=0}^{12} B_h^{12}(t) \ \ \sum_{\{klh\}}^{6+6} M_{kl} b_i^c) 
%	\ \ (\sum_{m=0}^3 B_m^3(t) \ \ s_m)}
%     {\sum_{h=0}^{12} B_h^{12}(t) w_h^{path}}
%\]
\[
= \frac{\sum_{j=0}^{15} B_j^{15}(t) \ \ \sum_{\{hmj\}}^{12+3} \ \ 
		(\sum_{\{klh\}}^{6+6} M_{kl} b_i^c) \ \ s_m}
     {\sum_{j=0}^{12} B_j^{12}(t) w_j^{path}}
\]
The denominator can be degree-elevated to degree 15 using 
traditional degree elevation,
%\[
%O(t)S(t)b_i^c = \sum_{j=0}^{15} B_j^{15}(t) 
%	\left(	\begin{array}{c}
%	\sum_{\{hmj\}}^{12+3} \ \ 
%		(\sum_{\{klh\}}^{6+6} M_{kl} b_i^c) \ \ s_m
%	\\ w^{(3)}_j
%	\end{array} \right)
%\]
and adding the position component yields
\[
P(t) + O(t)S(t)b_i^c = 
\sum_{j=0}^{15} B_j^{15}(t) 
	\left(	\begin{array}{c}
	w^{(3)}_j p_j^{+} + 
	\sum_{\{hmj\}}^{12+3} \ \ 
		(\sum_{\{klh\}}^{6+6} M_{kl} b_i^c) \ \ s_m
	\\ w^{(3)}_j
	\end{array} \right)
\]
This is a rational curve with control points $b_{ij}$
(as defined above) and weights $w^{(3)}_j$.
Since the weights are independent of $i$, we can apply (\ref{eq:ratrat})
to find that the $i^{th}$ column of the swept surface's control mesh
consists of the control points $b_{ij}$ and weights 
$w_i^c  w^{(3)}_j$ ($j=0,\ldots,12$).
\QED

We have enforced $C^2$-continuity on the position, orientation,
and scale of the sweeping curve, and consequently the degree of the
swept surface is $(12,n)$ or $(15,n)$.
The degree of the swept surface can be lowered by lowering this continuity:
for example, if the orientation component
in Theorem~\ref{thm:ten1} is given $C^1$ rather than $C^2$ continuity, 
the degree of the swept surface drops from $(12,n)$ to $(8,n)$.
In general, the surface degree is $(4(i+1)+j+1,n)$ where the orientation
component has $C^i$ continuity and the scaling component $C^j$ continuity.
To see this, note that we begin with an interpolating curve of degree $i+1$
in quaternion 4-space; this is doubled in degree both when we map to the 
quaternion sphere using $M$ and when we translate to a rotation matrix
using Lemma~\ref{lem:qtorot}.

\section{Examples}
\label{sec:eg}

We have implemented the swept surface primitive developed in this paper
and in this section we shall give three examples of surfaces modeled with it.
For each example, we provide a figure of the
swept surface (shown to interpolate the input)
and a series of isoparametric curves of the swept surface,
each of which preserves the shape of the curve.
The input frames of the sweeping curve are drawn in thicker lines in the 
right figure.

Figure~\ref{fig:coke} shows a surface of revolution, but as generated
by our method.
% Figure~\ref{fig:cokewarp} illustrates how our method can relax the
% restrictions on this surface.
A surface of revolution can be represented exactly with our method
by using a circle 
centered at the origin
as the sweep curve $c(t)$, and frames of various sizes with positions
$(0,0,k_i)$ and orientations $(1,0,0,0)$.
An extruded  or translational surface is also easily achieved, due to
the linear precision of Bezier surfaces, using two curve frames
with the same orientation and scale.

Figure~\ref{fig:canoe} is suggestive of the design
possibilities of the method (a canoe).
None of these curves have the same orientation.
Figure~\ref{fig:robot} suggests how the surface could be used ({\em e.g.}, for 
interference detection) in an environment
where an object, such as a robot arm, is truly sweeping through space
and outputting its position and orientation.

\begin{figure}
\vspace{2.75in}
\special{psfile=/rb/jj/Research/swept/Euro1.ps
	 hscale=70 vscale=70 voffset=-320}
\caption{Conventional revolute surfaces can be built}
% Coke bottle
% curve.circle, POS.bottle
\label{fig:coke}
\end{figure}

% \begin{figure}
% \vspace{2in}
% \caption{Removing the shackles of the surface of revolution
% 	(a) Position only (b) Orientation only}
% % curve.circle, POS.warpbottle1 or POS.warpbottle2
% \label{fig:cokewarp}
% \end{figure}

\begin{figure}
\vspace{2.75in}
\special{psfile=/rb/jj/Research/swept/Euro2.ps
	 hscale=70 vscale=70 voffset=-320}
\caption{But also more general sweeps}
% curve.canoe, POS.canoe
\label{fig:canoe}
\end{figure}

\begin{figure}
\vspace{2.75in}
\special{psfile=/rb/jj/Research/swept/Euro3.ps
	 hscale=70 vscale=70 voffset=-320}
\caption{Envelope of a sweeping object}
% curve.robot, POS.snake
\label{fig:robot}
\end{figure}

% Canal surface (to see how well we can simulate a swept surface).
% Curve = solid($t$) $\cap$ solid($t+\epsilon$) [simulating envelope]
% Span bridge, wave  LATER
% Extruded surface (star shape or octagonal shape). NOT REALLY NEEDED.

Note that the interpolation of each component (position, orientation,
and scale) is by an interpolating 
rational Bezier function, which has all of the properties of Bezier
functions such as variation-diminution, the convex hull property,
and linear precision.

\section{Conclusions}

We have shown how to build an exact rational model of a sweeping curve
that interpolates a finite number of keyframes.
This model ensures that there is no deformation of the curve as it sweeps.
Since classical point interpolation can be viewed as the path generated
by sweeping a point so that it interpolates a finite set of points,
our method is analogous to point interpolation, with the point replaced
by a rigid curve.
By fully defining its rational Bezier control mesh, we ensure that
this new swept surface primitive may be easily incorporated into
current geometric modeling systems.

% \section{FUTURE}

% Another approach to reduce the degree of the Bezier swept surface
% is to design a triangular Bezier surface, since triangular surfaces
% can represent the same surface in lower degree than tensor product
% surfaces (REFERENCE?).
% We are presently working on this approach.

% {\bf Degree reduce} the orientation curve to degree 3 using ideas of Forrest (except
% for a rational curve instead of a polynomial curve).
% (Check if this degree reduced orientation curve still
% has $C^2$ continuity.)

% We have not attempted to model the sweep of a solid exactly however:
% for example, in order to model the sweep of a sphere (a canal surface) 
% by the sweep of a circle, we need to add the restriction that
% the circle always be orthogonal to the position curve (traveled by
% the center of the circle). 
% This is tantamount to adding a dependence between the orientation function
% and the position function.
% The present paper explores the independent definition of position,
% orientation, and scaling of the curve.

% as in Dietz et. al.,
% use linear system solution of curve interpolating $n$ preimage lines
% in 4-space rather than interpolating $n$ preimage points

% Note that the only way that the curve can change as it sweeps
% is through change of size (e.g., ellipse with major radius 1 to 
% ellipse with major radius 2),
% it cannot change type (e.g., from ellipse to parabola).
% But general t.p. surfaces model sweep of varying curves.

% How should we control the speed of knots?
% At present, the interpolation in 4-space quaternion curve defines the
% knots and thus the knots of the orientation function, and the other
% components.

% Our surface is appropriate for the modeling of sweeping in
% CAD/CAM, robotics, and blending, but is not appropriate, for example,
% for the modeling of a surface reconstruction of several MRI slice
% contours (taking an example from biomedical visualization), 
% where the shape of the curve changes not only in scale but in overall
% geometry throughout the sweep.

% The definition of each of the components is completely
% independent of the other components.
% (This latter can be viewed as an advantage or a disadvantage.)

\bibliographystyle{plain}
\bibliography{/rb/jj/bib/modeling}
\ifFull
\begin{thebibliography}{Shoemake 85}

\bibitem{bloom91}
Bloomenthal, M. and R. Riesenfeld (1991)
Approximation of sweep surfaces by tensor product NURBS.
SPIE Proceedings,
% : Curves and Surfaces in Computer Vision and Graphics II,
Vol. 1610, 132--144.

\bibitem{bronsklok85}
Bronsvoort, W. and F. Klok (1985)
Ray tracing generalized cylinders.
ACM Transactions on Graphics, 4(4), October, 291--303.

\bibitem{bronswaart92}
Bronsvoort, W.F. and Waarts, J.J. (1992)
A method for converting the surface of a generalized cylinder into
a B-spline surface.
Computers \& Graphics 16(2), 175--178.

\bibitem{coquillart87}
Coquillart, S. (1987)
A control-point-based sweeping technique.
IEEE Computer Graphics and Applications, November, 36--45.

\bibitem{dickson52}
Dickson, L.E. (1952) History of the theory of numbers: Volume II,
Diophantine analysis.  Chelsea (New York).

\bibitem{Dietz93}
Dietz, R., J. Hoschek and B. J\"{u}ttler (1993)
An algebraic approach to curves and surfaces on the sphere and on other
quadrics.  Computer Aided Geometric Design 10, 211--229.

\bibitem{farin93}
Farin, G. (1993) Curves and surfaces for computer aided geometric design.
Academic Press (New York), third edition.

\bibitem{farouki94}
Farouki, R. (1994)
Pythagorean-hodograph space curves.
Advances in Computational Mathematics, 2, 41--66.

% removed for lack of space
% \bibitem{Flaquer92}
% Flaquer, J., G. Garate and M. Pargada (1992)
% Envelopes of moving quadric surfaces.
% Computer Aided Geometric Design, 9, 299--312.

% removed for lack of space
% \bibitem{hoschek+lasser93}
% Hoschek, J. and D. Lasser (1993)
% Fundamentals of computer aided geometric design.
% A.K. Peters (Wellesley, Massachusetts).

\bibitem{jjjimbo94a}
Johnstone, J.K. and J.P. Williams (1994) Rational control of orientation
for animation, Proceedings of Graphics Interface '95, Quebec City.

\bibitem{martinsteph90}
Martin, R. and P. Stephenson (1990)
Sweeping of three-dimensional objects.
CAD, 22, 223--234.

\bibitem{misner73}
Misner, C., K. Thorne and J. Wheeler (1973)
Gravitation.  W.H. Freeman (San Francisco).

\bibitem{parkkim93}
Park, E.-J. and Kim, M.-S. (1993)
Modeling generalized cylinders with variable radius offset space curves.
Proc. of Pacific Graphics '93, Vol. 1, S. Shin and T. Kunii (eds),
World Scientific, 20--34.

\bibitem{Pegna88}
Pegna, J. (1988) Variable sweep geometric modeling.
Ph.D. thesis, Dept. of Mechanical Engineering, Stanford University.

\bibitem{rossignac85}
Rossignac, J. (1985)
Blending and offsetting solid models.
Ph.D thesis, University of Rochester.

\bibitem{rossrequicha84}
Rossignac, J. and A. Requicha (1984)
Constant-radius blending in solid modeling.
Computers in Mechanical Engineering, July, 65--73.

% removed for lack of space
% \bibitem{seidel90}
% Seidel, H.P. (1990) Quaternionen in Computergraphik und Robotik.
% Informationstechnik 32, 266--275.

\bibitem{shoemake85}
Shoemake, K. (1985) Animating rotation with quaternion curves.
SIGGRAPH '85, San Francisco, 19(3), 245--254.

\bibitem{vanwijk84euro}
van Wijk, J. (1984)
Ray tracing objects defined by sweeping a sphere.
EUROGRAPHICS '84, 73--82.

\bibitem{vanwijk84tog}
van Wijk, J. (1984)
Ray tracing objects defined by sweeping planar cubic splines.
ACM Transactions on Graphics, 3(3), July, 223--237.

\bibitem{wangwang86}
Wang, W. and K. Wang (1986)
Geometric modeling for swept volume of moving solids.
IEEE Computer Graphics and Applications, December, 8--17.

\end{thebibliography}
\fi


% The new swept surface can perhaps be best understood by comparing it
% with the conventional revolute and extruded surface.

% We call our method the {\em keyframe swept surface method}.

% Degree elevation is needed in the paper.
% The following lemma is easily derived from Farin \cite{farin93}.

% \begin{lemma}
% \label{lem:deg-elevate}
% $\sum_{i=0}^{n} b_i B_i^n(t) = \sum_{i=0}^{n+2} b_i^{(2)} B_i^{n+2}(t)$,
% where\footnote{Although undefined $b_{-2}$ and $b_{-1}$ appear in this equation
%	(if $i=0,1$), their definition is unnecessary since they
%	are always zeroed out.}
% \[
% b_i^{(2)} = \frac{i(i-1)b_{i-2} + 2i(n+2-i)b_{i-1} + (n+2-i)(n+1-i)b_i}{(n+1)(n+2)}
% \]
%\end{lemma}
% \prf
% Using the formula for degree elevation
% ($b_i^{(1)} = \frac{i}{n+1} b_{i-1} + (1-\frac{i}{n+1})b_i$),
% \[
% b_i^{(2)} = \frac{i}{n+2} b_{i-1}^{(1)} + (1-\frac{i}{n+2})b_i^{(1)}
% \]
% \[
% = \frac{i}{n+2}(\frac{i-1}{n+1} b_{i-2} + (1-\frac{i-1}{n+1})b_{i-1})
% + (1-\frac{i}{n+2})(\frac{i}{n+1} b_{i-1} + (1-\frac{i}{n+1})b_i)
% \]
% \[
% = \frac{i(i-1)}{(n+1)(n+2)} b_{i-2}
% + [(1-\frac{i-1}{n+1})(\frac{i}{n+2}) + (1-\frac{i}{n+2})(\frac{i}{n+1})]b_{i-1}
% + (1-\frac{i}{n+2})(1-\frac{i}{n+1})b_i
% \]
% \[
% = \frac{i(i-1)}{(n+1)(n+2)} b_{i-2}
% + \frac{2(n+2-i)}{(n+1)(n+2)} i b_{i-1}
% + \frac{(n+2-i)(n+1-i)}{(n+1)(n+2)} b_i
% \]
% \[
% = \frac{i(i-1)b_{i-2} + 2i(n+2-i)b_{i-1} + (n+2-i)(n+1-i)b_i}{(n+1)(n+2)}
% \]
% \QED

\end{document}
