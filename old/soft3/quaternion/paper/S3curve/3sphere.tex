\documentstyle[times]{article} 

\newif\ifFull
\Fullfalse

\makeatletter
\def\@maketitle{\newpage
 \null
 \vskip 2em                   % Vertical space above title.
 \begin{center}
       {\Large\bf \@title \par}  % Title set in \Large size. 
       \vskip .5em               % Vertical space after title.
       {\lineskip .5em           %  each author set in a tabular environment
        \begin{tabular}[t]{c}\@author 
        \end{tabular}\par}                   
  \end{center}
 \par
 \vskip .5em}                 % Vertical space after author
\makeatother

% default values are 
% \parskip=0pt plus1pt
% \parindent=20pt
\parskip=.2in
\parindent=0pt

\newcommand{\SingleSpace}{\edef\baselinestretch{0.9}\Large\normalsize}
\newcommand{\DoubleSpace}{\edef\baselinestretch{1.4}\Large\normalsize}
\newcommand{\Comment}[1]{\relax}  % makes a "comment" (not expanded)
\newcommand{\Heading}[1]{\par\noindent{\bf#1}\nobreak}
\newcommand{\Tail}[1]{\nobreak\par\noindent{\bf#1}}
\newcommand{\QED}{\vrule height 1.4ex width 1.0ex depth -.1ex\ } % square box
\newcommand{\arc}[1]{\mbox{$\stackrel{\frown}{#1}$}}
\newcommand{\lyne}[1]{\mbox{$\stackrel{\leftrightarrow}{#1}$}}
\newcommand{\ray}[1]{\mbox{$\vec{#1}$}}          
\newcommand{\seg}[1]{\mbox{$\overline{#1}$}}
\newcommand{\tab}{\hspace*{.2in}}
\newcommand{\se}{\mbox{$_{\epsilon}$}}  % subscript epsilon
\newcommand{\ie}{\mbox{i.e.}}
\newcommand{\eg}{\mbox{e.\ g.\ }}
\newcommand{\figg}[3]{\begin{figure}[htbp]\vspace{#3}\caption{#2}\label{#1}\end{figure}}
\newcommand{\be}{\begin{equation}}
\newcommand{\ee}{\end{equation}}
\newcommand{\prf}{\noindent{{\bf Proof}:\ \ \ }}
\newcommand{\choice}[2]{\mbox{\footnotesize{$\left( \begin{array}{c} #1 \\ #2 \end{array} \right)$}}}      
\newcommand{\scriptchoice}[2]{\mbox{\scriptsize{$\left( \begin{array}{c} #1 \\ #2 \end{array} \right)$}}}
\newcommand{\tinychoice}[2]{\mbox{\tiny{$\left( \begin{array}{c} #1 \\ #2 \end{array} \right)$}}}
\newcommand{\ddt}{\frac{\partial}{\partial t}}

\newtheorem{rmk}{Remark}[section]
\newtheorem{example}{Example}[section]
\newtheorem{conjecture}{Conjecture}[section]
\newtheorem{claim}{Claim}[section]
\newtheorem{notation}{Notation}[section]
\newtheorem{lemma}{Lemma}[section]
\newtheorem{theorem}{Theorem}[section]
\newtheorem{corollary}{Corollary}[section]
\newtheorem{defn2}{Definition}

% \font\timesr10
% \newfont{\timesroman}{timesr10}
% \timesroman

\DoubleSpace

\setlength{\oddsidemargin}{0pt}
\setlength{\topmargin}{-.25in}	% technically should be 0pt for 1in
\setlength{\headsep}{0pt}
\setlength{\textheight}{8.75in}
\setlength{\textwidth}{6.5in}
\setlength{\columnsep}{5mm}		% width of gutter between columns

\title{Rational interpolating curves on the 3-sphere}
% (ignoring the context of `pt on 3-sphere = unit quaternion = orientation')
	% CAGD is best since this is a followup to Dietz et. al. in CAGD
	% and since this deals with design of curves on surfaces, a CAGD topic
\author{John K. Johnstone\\
	Department of Computer and Information Sciences\\
	The University of Alabama at Birmingham\\
	Birmingham, Alabama, USA 35294\\
	johnstone@cis.uab.edu\\
	205-934-2213
	\and 
	James P. Williams\\
	Department of Computer Science\\
	The Johns Hopkins University\\
	Baltimore, Maryland, USA 21218\\
	jimbo@cs.jhu.edu\\
	410-516-7784}

\begin{document}

\maketitle

\begin{abstract}
The design of interpolating curves on the 3-sphere, an important problem
in animation, has been approached in the literature using nonrational
curves or rational curves considerably limited in degree and continuity,
both requiring the implementation of a new curve system.
In this paper, we develop a general method for the design of any rational
interpolating curve on the 3-sphere,
% of arbitrary even degree and continuity
which relies on the existing machinery of rational interpolating curves 
and only requires the implementation of a rational map and its inverse.
We prove a completeness result establishing that any rational curve on the
3-sphere can be built with this method.
Using a technique originated by Dietz, Hoschek and J\"{u}ttler for the design
of curves on quadrics in 3-space, we design curves in an image space rather than
the 3-sphere, effectively removing the constraint to the 3-sphere, and
then map these curves back to the 3-sphere.
The use of Euler's Four Squares Theorem is explored for the generation
of rational maps to the sphere.
We elaborate on the example of the design of a $C^2$ continuous sextic rational
Bezier interpolating spline on the 3-sphere using this method.
\end{abstract}

Keywords: Rational curves, interpolation, curves on surfaces, curve design,
	  3-sphere, quaternion, Pythagorean tuple, Euler's Four Squares Theorem.

\section{Introduction}

The design of curves on surfaces has received considerable study
and the design of curves on quadrics \cite{boehmHansford91,dietz93}, 
especially the sphere \cite{hoschekSeemann92}, has been of particular interest.
% Curves on surfaces can be used, for example, in trimming surfaces (?)
% and in motion planning on configuration space obstacles.
In this paper, we consider the design of curves on the 3-sphere $S^3$ in 4-space.
This problem is of interest as a higher-dimensional variant of the traditional
problem.  
% For example, configuration space obstacles are usually higher-dimensional
% surfaces \cite{}.
It is also of considerable practical interest because
a curve on the 3-sphere can define the orientation of a moving object
(such as an object in a computer animation \cite{shoemake85}, a camera, 
a swept surface \cite{jjjw95b}, a docking spacecraft \cite{junkins86}, 
or a robotic device),
through identification of a point on the 3-sphere with a quaternion.

Our problem is to construct a rational curve on the 3-sphere
that interpolates a finite set of points.
We insist upon a rational curve in the interests of computational efficiency,
compatibility with rational curve modeling systems,
and for simple manipulation of the resulting curve.
The curve must not only interpolate the points, but is also constrained
to lie on the surface.
This makes the design of curves on surfaces a more challenging version
of point interpolation.
However, our approach is very similar to traditional point interpolation,
much more so than previous solutions to the problem.

A preliminary version of this paper is Johnstone and Williams \cite{jjjw95}.
The following notation is used in this paper.
%
\begin{itemize}
\item
$\Re[t]$ is the ring of polynomials over the real numbers $\Re$.
\item
$P_A^{n}$ is projective $n$-space over $A$.
$P_A^{n} = \{(x_1,\ldots,x_{n+1}) \neq 0 : x_i \in A, 1 \leq i \leq n+1\}$.\\
$(x_1,\ldots,x_{n+1}) \in P_A^{n}$
is equivalent to the point $(\frac{x_1}{x_{n+1}}, \ldots, \frac{x_n}{x_{n+1}})$
in affine $n$-space.\\
We will be primarily interested in $P_{\Re}^4$, projective 4-space over the reals.
\item
$S^i$ is the $i$-sphere, the unit sphere in ($i+1$)-space.
\end{itemize}

\section{Related work}

There is a substantial literature on the design of interpolating curves on $S^3$
(often called quaternion curves),
because of the applications of these curves in computer animation.
All of the previous methods require the introduction of a new curve type
and most yield nonrational curves.

% don't mention: not a candidate since interactive: essentially `build it yourself'
% simply normalizing the curve \cite{nielson93}
% can use tension and nu-spline to remove kinks but only interactively

The most popular method is to use spherical linear interpolation 
(also called slerping) 
\cite{shoemake85,duff85,pletinckx89,schlag91,nielson92,nielson93,kim95,nam95}.
The spherical linear interpolation between the points $P_0$ and $P_1$ on $S^3$ is:
\[ P(t) = \frac{\sin((1-t)\theta) P_0 + \sin(t \theta) P_1}{\sin \theta}
\]
where $\cos \theta = P_0 \cdot P_1$, which is the great circle between $P_0$ and $P_1$.
A spherical analog to the de Casteljau algorithm based upon this
spherical linear interpolation is used to build curves,
mimicking Bezier \cite{shoemake85,kim95}, B-spline 
\cite{duff85,nielson92,nielson93,kim95}, Hermite \cite{kim95,nam95},
cardinal spline \cite{pletinckx89}, and Catmull-Rom \cite{schlag91} curves.
Since slerping is a nonrational operation,
all of the methods based on slerping generate nonrational curves.
Most of these curves are defined only by a geometric construction, and have
no closed form algebraic definition.
Computation of derivatives of these curves is complicated,
as is the imposition of $C^2$ continuity, and
Kim et. al. \cite{kim95} go to considerable trouble to solve these two problems.
The use of conventional Bezier curves completely removes 
these problems with continuity or derivative calculation.

The work of Wang \cite{wang93, wang95} and Nielson \cite{nielson93}
is closest in spirit to this paper, since it designs rational curves on $S^3$.
However, these curves are limited in scope.
Wang \cite{wang93,wang95} (using spherical biarcs) 
and Nielson \cite{nielson93} develop quadratic curves
with $G^1$ continuity, and Wang \cite{wang95} develops sextic curves
with $C^1$ continuity.
% $C^2$ continuity is desirable, especially for animation.
In Wang \cite{wang93,wang95}, the input points must be augmented with 
tangents, since the problem is posed as a Hermite interpolation problem.
Both Wang and Nielson's methods also involve heuristic, input-dependent choices 
% (such as the choice of a spherical biarc from
% a one-parameter family of valid spherical biarcs, or a center of projection)
that can be difficult to make.
Neither of the methods has any completeness result about the
coverage of rational curves by their method.
Our method subsumes the work of Wang and Nielson, 
by creating rational curves of arbitrary
even degree (all rational curves on $S^3$ have even degree \cite{wang95}) 
and arbitrary continuity, using a single data-independent map that is provably complete.

The above methods require implementation of new curves,
such as slerped curves or spherical biarcs.
In our method, no new curves must be implemented, just two maps:
a map to the 3-sphere and its inverse.
Consequently, the implementation of our method and its incorporation
into any modeling software is much simpler.

Barr et.\ al.\ \cite{barr92} design interpolating curves on $S^3$ 
through constrained optimization: the constraint to the 3-sphere 
is realized through its introduction into an optimization problem.
This yields a nonrational curve again, indeed with no closed-form expression
and approximate.
The numerical optimization is also expensive.
The method does have the advantage of allowing extra constraints 
to be incorporated into the optimization (they choose to design a curve
that minimizes tangential acceleration).

\section{A new approach}
\label{sec:general}

All of the previous methods are restricted and dictated to by the constraint 
to the 3-sphere,
either in the primitives they use for interpolation (biarcs or slerped curves)
or in the method (optimization).
The key to our method is to remove this constraint, by mapping to an image space.

We are motivated by a paper of Dietz, Hoschek and J\"{u}ttler \cite{dietz93}
on the design of rational interpolating curves on the 2-sphere $S^2$ 
% (which can be extended to rational interpolating curves on other quadrics) 
\cite{dietz93}.
Dietz et.\ al.\ design curves constrained to a surface in $n$-space by 
designing a map $M$ from an arbitrary point of $n$-space to the surface.
This allows the free design of a curve (unconstrained to the surface)
followed by a mapping of the curve to the surface using $M$.
In this way, the two steps in the design of curves on surfaces,
interpolation and constraint to the surface, are separated:
the surface constraint is applied independently using $M$, 
after the interpolation.
This is an ingenious divide-and-conquer strategy: other methods
struggle with the two problems of constrained interpolation together.

A formal statement of this approach is as follows.
Also see Figures~\ref{fig1}-\ref{fig6} for the application of this
approach to our problem.
Let $S$ be the surface in $n$-space (to which the curve must be constrained), 
let $\{P_i\}$ be the given points on $S$ (which must be interpolated), 
and let $M$ be a map from $n$-space to $S$.

\vspace{.2in}

\centerline{\underline{The image space construction method for curves on surfaces}}
\begin{enumerate}
\item Map the points $\{P_i\}$ to the image space, $\{M^{-1}(P_i)\}$.
\item Design an interpolating curve $C(t)$ in the image space.
\item Map $C(t)$ back to $S$ using $M$, yielding an interpolating curve on 
	the surface, $M(C(t))$.
\end{enumerate}

There are two main steps to the design of a curve on a surface using
this method: the construction of the map $M$ from $n$-space to 
the surface and its inverse, and the definition of the image of a {\em curve} 
under $M$,
since it is impossible to apply the map to each of the infinite
points of the curve in step 3 of the method.

We apply this approach to our problem of the design of rational 
curves on the 3-sphere.
The design of a rational map to the 3-sphere and its inverse
is covered in Sections~\ref{sec:M} and \ref{sec:invM}, respectively.
The image of a curve is defined in Section~\ref{sec:curveimage}.
In Section~\ref{sec:complete}, we establish a completeness result,
that all rational curves
on the 3-sphere can be constructed using our method.
Examples of the method (to augment Figures~\ref{fig1}-\ref{fig6})
are presented in Section~\ref{sec:eg} and we close the paper
with some remarks in Section~\ref{sec:conclude}.
In the next section, we begin by looking at related material 
on Pythagorean $i$-tuples and Euler's Four Squares Theorem.

\twocolumn

\begin{figure}[h]
\vspace{2.5in}
\special{psfile=/rb/jj/Research/S3curve/img/fig1.ps}
\caption{Input points}
\label{fig1}
\end{figure}
% figures 1-6: GLUTS3curve < data/S3.in
% tops fig1.rgb -eps -m 8.5 2 > fig1.ps

\begin{figure}[h]
\vspace{2.5in}
\special{psfile=/rb/jj/Research/S3curve/img/fig2.ps}
\caption{Input and image points}
\label{fig2}
\end{figure}

\begin{figure}[h]
\vspace{2.5in}
\special{psfile=/rb/jj/Research/S3curve/img/fig3.ps}
\caption{Interpolating curve in image space}
\label{fig3}
\end{figure}

\begin{figure}[h]
\vspace{2.5in}
\special{psfile=/rb/jj/Research/S3curve/img/fig4.ps}
\caption{Interpolating curves on 3-sphere and in image space}
\label{fig4}
\end{figure}

\begin{figure}[h]
\vspace{2.5in}
\special{psfile=/rb/jj/Research/S3curve/img/fig5.ps}
\caption{Curves and control polygons}
\label{fig5}
\end{figure}

\begin{figure}[h]
\vspace{2.5in}
\special{psfile=/rb/jj/Research/S3curve/img/fig6.ps}
\caption{A rational solution}
\label{fig6}
\end{figure}

\onecolumn

\section{Euler's Four Squares Theorem}

\begin{defn2}
A {\em Pythagorean $i$-tuple} over F, $i \geq 3$, is an $i$-tuple 
$(x_1,\ldots,x_i)$, $x_i \in F$, that satisfies \\
$x_1^2 + \ldots + x_{i-1}^2 = x_{i}^2$.
\end{defn2}

The study of curves on spheres is deeply related to the study of Pythagorean 
$i$-tuples, since a point \\
$x = (x_1,\ldots,x_{i+2}) \in P^{i+1}$ lies on the sphere $S^i$ if and only if
\[
     \|x\| = (\frac{x_1}{x_{i+2}})^2 + \ldots + (\frac{x_{i+1}}{x_{i+2}})^2 = 1
\mbox{\ \ \  or \ \ \ }  
      x_1^2 + \ldots + x_{i+1}^2 = x_{i+2}^2
\]
\begin{description}
\item[(Rule 1)]
A point $x \in P_{\Re}^{i+1}$ lies on $S^i$ if and only if
$x$ is a Pythagorean ($i+2$)-tuple over $\Re$.
\item[(Rule 2)]
A rational curve 
$x(t) = (x_1(t),\ldots,x_{i+2}(t)) \in P_{\Re[t]}^{i+1}$ lies on $S^i$
if and only if $x(t)$ is a Pythagorean $(i+2)$-tuple over $\Re[t]$.
\end{description}

Thus, the study of curves on the 3-sphere leads to the study of Pythagorean
5-tuples over $\Re[t]$.
Since Pythagorean $i$-tuples involve sums of squares, we are
interested in number-theoretical formulae concerning sums of squares.
This section presents results concerning one such formula,
Euler's Four Squares Theorem.
This is a fundamental formula from number theory
on the sum of four squares \cite{dickson52, ebbinghaus90}.

\begin{theorem}[Euler's Four Squares Theorem, 1748]

% p. 277 of Dickson, Vol. 2
\[
(a^2 + b^2 + c^2 + d^2) (a'^2 + b'^2 + c'^2 + d'^2) = \hfill
\]
\begin{equation}
\label{eqn:euler1}
(aa' + bb' + cc' + dd')^2 +
(ab' - ba' \pm cd' \mp dc')^2 +
(ac' \mp bd' - ca' \pm db')^2 +
(ad' \pm bc' \mp cb' - da')^2
\end{equation}
or
%
%
% p. 209, Ebbinghaus
\[
(a^2 + b^2 + c^2 + d^2) (a'^2 + b'^2 + c'^2 + d'^2) = \hfill
\]
\begin{equation}
\label{eqn:euler2}
(aa' - bb' - cc' - dd')^2 +
(ab' + ba' + cd' - dc')^2 +
(ac' - bd' + ca' + db')^2 +
(ad' + bc' - cb' + da')^2
\end{equation}
where $a,b,c,d,a',b',c',d'$ are elements of any commutative ring 
(such as the integers, the rationals, the reals, or 
polynomials over the integers, rationals, or reals).
% p. 210, Ebbinghaus for any commutative ring
\end{theorem}

\begin{corollary}
\label{cor:specialEuler}
Two special cases of Euler's Four Squares Theorem are: 
\begin{equation}
\label{eq:euler1}
(a^2 + b^2 + c^2 + d^2)^2 = 
(2ad-2bc)^2 + (2ac+2bd)^2 + (a^2 + b^2 - c^2 - d^2)^2
\end{equation}
%
% dickson52, p. 318
\begin{equation}
\label{eq:aida}
(a^2 + b^2 + c^2 + d^2)^2 = 
(a^2 + b^2 + c^2 - d^2)^2 + (2ad)^2 + (2bd)^2 + (2cd)^2
\end{equation}
\end{corollary}
\prf
Let $a'=a, b'=b, c'=-c, d'=-d$ in (\ref{eqn:euler1}) for (\ref{eq:euler1}),
and $a'=-a, b'=b, c'=c, d'=d$ in (\ref{eqn:euler1}) for (\ref{eq:aida}).
(\ref{eq:aida}) was independently established by Aida \cite{dickson52}. % p. 318
% (c. 1810) 
\QED

Both of these formulae can be interpreted as Pythagorean tuples.
Indeed, we can view these formulae as recipes for generating
Pythagorean 4- and 5-tuples.
In the next section, we will use these formulae to build maps to the 2-
and 3-sphere.

\begin{rmk}
Euler's Four Squares Theorem is a generalization of 
Diophantus' Two Squares Theorem \cite{dickson52}:
% Dickson, p. 225; Ebbinghaus, p. 75
\begin{equation}
(a^2 + b^2)(a'^2 + b'^2) = (aa' - bb')^2 + (ab' + ba')^2
\end{equation}
where $a,b,a',b'$ are again elements of any commutative ring.
Moreover, just as the Two Squares Theorem encodes the product rule 
of complex numbers:
\[
| a+ib |^2 | a'+ib' |^2 = | (a+ib)(a'+ib') |^2 
\]
% or \[ |A| |B| = |AB| \ \ \ (A,B \in {\cal C})\] 
the Four Squares Theorem, as expressed in (\ref{eqn:euler2}),
encodes the product formula of quaternions
(although it predates the discovery of quaternions 
by Hamilton in 1843 by a century!):

\begin{equation}
\begin{array}{lll}
(a,b,c,d) * (a',b',c',d') & = & (a + bi + cj + dk) * (a' + b'i + c'j + d'k) \\
& = & (aa' - bb' - cc' - dd') + (ab' + ba' + cd' - dc')i \\
& & + (ac' - bd' + ca' + db')j + (ad' + bc' - cb' + da')k
\end{array}
\end{equation}

This relationship between the Four Squares Theorem and quaternions is
serendipitous, since the Four Squares Theorem will be
used to design curves on the 3-sphere, whose main application is the
control of orientation via quaternions in computer animation.
\end{rmk}

\section{A rational map to the 3-sphere}
\label{sec:M}

We have noted the equivalence of points on the 3-sphere and Pythagorean
5-tuples.
Consequently, we can use the Pythagorean 5-tuple `generator' formula
of Corollary~\ref{cor:specialEuler} to build a map to the 3-sphere.

\begin{lemma}
\label{lem:M}
The map
\begin{equation}
\label{eq:M}
	M(a,b,c,d,1) = \left( \begin{array}{c}
		a^2+b^2+c^2-d^2 \\
		2ad \\
		2bd \\
		2cd \\
		a^2+b^2+c^2+d^2
		\end{array} \right)
\end{equation}
is a rational map from projective 4-space $P^4_{\Re} \setminus \{0\}$ 
onto the 3-sphere $S^3 \subset P^4_{\Re}$.
\end{lemma}
\prf
$M(a,b,c,d,1)$ is a point on the 3-sphere $S^3$,
using (\ref{eq:aida}) and Rule 1.
The image of the origin is undefined because 
the point $(0,0,0,0,0)$ is undefined in projective 4-space.
We will show below that $M$ has an inverse map that is well defined over the
entire sphere (Lemma~\ref{lem:invM}), 
which establishes that $M$ is onto the sphere.
\QED

For historical purposes, we observe that the map to the 2-sphere
used in Dietz \cite{dietz93} can also be derived from 
Corollary~\ref{cor:specialEuler}, using the other formula, as follows
(although they derive it from a result of Lebesgue).

\begin{lemma}
\label{lem:delta}
The map 
\begin{equation}
\label{eq:delta}
\delta(a,b,c,d) = \left( \begin{array}{c}
		2ad - 2bc \\
		2ac + 2bd \\
		a^2 + b^2 - c^2 - d^2 \\
		a^2 + b^2 + c^2 + d^2
		\end{array} \right)
\end{equation}
is a rational map from projective 3-space $P^3_{\Re} \setminus \{0\}$
onto the 2-sphere $S^2 \subset P^3_{\Re}$.
\end{lemma}
\prf
$\delta(a,b,c,d)$ is a point on the 2-sphere $S^2$,
using (\ref{eq:euler1}) and Rule 1.
The image of the origin is undefined because 
the point $(0,0,0,0)$ is undefined in projective 3-space.
Finally, let $s = (s_1,s_2,s_3,s_4)$ be a point of $S^2$.
If $s$ is not the north pole ($s \neq (0,0,1,1)$),
$\delta(s_1,s_2,0,s_4-s_3)=s$
otherwise $\delta(a,b,0,0)=s$ ($a$ and $b$ arbitrary except $ab \neq 0$).
Thus, $\delta$ is onto the entire 2-sphere.
Notice that the inverse image of the north pole, $(a,b,0,0)$, 
is a point at infinity in the $z=0$ plane, just as in stereographic projection.
\QED

A classical map to the sphere is stereographic projection \cite{levinson70},
% p. 29,43 Levinson
which maps the $x_{n+1} = 0$ plane to the $n$-sphere in $n+1$-space.
In 4-space, this is a map to the 3-sphere:
%
\begin{equation}
\label{eq:stereo}
	P(a,b,c,0) = \frac{1}{a^2+b^2+c^2+1}(2a,2b,2c,a^2+b^2+c^2-1)
\end{equation}
%
Unfortunately, $(0,0,0,1)$ is a pole of stereographic projection (it is only attained in the limit as $(a,b,c,0)$
goes to infinity), which causes problems in using this map for the design of
curves on $S^3$, as noted by Wang \cite{wang95}.
Both (\ref{eq:M}) and (\ref{eq:delta}) are related to stereographic 
projection---$\delta$ is the composition of stereographic projection with 
hyperbolic projection \cite{dietz93}, and $M$ is visibly similar---,
but $M$ does not suffer from this problem with poles.
The point that would be expected to be a pole of $M$ in analogy to stereographic
projection is the point $(1,0,0,0) \in S^3$.
This point is still special, but in the opposite way to a pole:
the entire $x_4=0$ hyperplane maps to $(1,0,0,0)$ under $M$, so it is 
over-represented rather than under-represented.
See Lemma~\ref{lem:invM} and Remark~\ref{rmk:1000}.

We end this section with a lemma on sphere parameterization,
which is closely related to maps to the sphere.
This lemma is used in the next section and also clarifies stereographic projection.

\begin{lemma}
\label{lem:sphereparam}
The $n$-sphere $S^n$ has a parameterization
\begin{equation}
\label{eq:param1}
	(\frac{t_1^2+\ldots+t_n^2-1}{t_1^2+\ldots+t_n^2+1},
	 \frac{2t_1}{t_1^2+\ldots+t_n^2+1},
	 \ldots, 
	 \frac{2t_n}{t_1^2+\ldots+t_n^2+1})
\end{equation}
% or
% \begin{equation}
% \label{eq:param2}
% 	(\frac{1-t_1^2-\ldots-t_n^2}{t_1^2+\ldots+t_n^2+1},
% 	 \frac{2t_1}{t_1^2+\ldots+t_n^2+1},
% 	 \ldots, 
% 	 \frac{2t_n}{t_1^2+\ldots+t_n^2+1})
% \end{equation}
$t_i\in(-\infty,\infty)$.
$(1,0,\ldots,0)$ is the only pole of this parameterization.
% All points of (\ref{eq:param2}) except $(-1,0,\ldots,0)$ 
% have finite parameter values.
\end{lemma}
\prf
A sphere can be parameterized by inheriting the natural 
parameterization of a hyperplane.
% eg., \cite{sommerville34}
Let $P$ be a point of the sphere and $H$ be a hyperplane, $P \not \in H$.
Any line through $P$ intersects the sphere in exactly one other point $s$ 
(counting properly, by Bezout's Theorem)
and the hyperplane $H$ in one point $h$.
Thus, lines through $P$ establish a 1-1 correspondence between points
of the sphere and points of the hyperplane, and it is well defined to
assign the parameters of $h$ to $s$.
In effect, the parameters of the hyperplane are splatted onto the quadric.
Stereographic projection is a special case of this general idea.

The obvious choices for $P$ and $H$, the two degrees of freedom, 
are $P=\pm e_i$ and $H: x_i=0$.
For (\ref{eq:param1}), let $P=(1,0,\ldots,0)$ and $H$ be $x_1=0$.
The line through $P$ and $(0,t_1,t_2,\ldots,t_n)$ on $H$
is \mbox{$(1-\alpha, \alpha t_1, \alpha t_2, \ldots, \alpha t_n)$}, 
which intersects the sphere $S_n$ when
\[
	(1-\alpha)^2 + (\alpha t_1)^2 + \ldots + (\alpha t_n)^2 - 1
	= \alpha^2 (1 + t_1^2 + \ldots + t_n^2) - 2 \alpha = 0
\] 
or $\alpha=0,\frac{2}{1 + t_1^2 + \ldots + t_n^2}$.
These two intersections are the points $P$ and (\ref{eq:param1}).
% For (\ref{eq:param2}), let $P=(-1,0,\ldots,0)$ and $H$ be $x_1=0$.
For stereographic projection, $P = (0,\ldots,0,1)$ and $H$ is $x_n=0$.
% which is (\ref{eq:param1}) with $x_1$ and $x_n$ interchanged.
\QED

% For example, the unit circle has the parameterization 
% \[ (x,y) = (\frac{1-t^2}{1+t^2}, \frac{2t}{1+t^2}) \]
% and $S^3$ has the parameterization 
% \[
% (\frac{t_1^2+t_2^2+t_3^2-1}{t_1^2+t_2^2+t_3^2+1},
% 		     \frac{2t_1}{t_1^2+t_2^2+t_3^2+1},
% 		     \frac{2t_2}{t_1^2+t_2^2+t_3^2+1},
%		     \frac{2t_3}{t_1^2+t_2^2+t_3^2+1})
% \]

% \begin{rmk}
% To get a map from 4-space to the unit sphere,
% we are looking for a formula of the form $A^2+B^2+C^2+D^2=E^2$
% % (a Pythagorean 5-tuple)
% where $A$, $B$, $C$, $D$ and $E$ are functions of at most four variables
% and at least one is a function of exactly four variables.
% Then a point in 4-space can be
% mapped to $(A,B,C,D,E)$ where $\|(A,B,C,D,E)\| = 1$.
% % (\ref{eq:aida}) is the only one we have been able to find.
% \end{rmk}

% Every positive integer is the sum of four squares,
% and coincidentally, this fact can be proved using quaternions (Hardy,
% An introduction to the theory of numbers, Oxford (1960), 4th edition).

\section{Completeness of $M$}
\label{sec:complete}

$M$ is an excellent choice for our purpose---the design of curves on
the 3-sphere---since it is not only onto all {\em points} on the 3-sphere 
(Lemma~\ref{lem:M}),
it is also onto all {\em curves} on the 3-sphere.
That is, any rational curve on $S^3$
can be generated as the image of some polynomial curve under $M$.
We call a map with this property {\em complete}.
The completeness property guarantees that the design of curves on the 3-sphere 
using $M$ is not restricted.

We prove this result by proving the equivalent statement for Pythagorean
5-tuples over the ring of real polynomials (Rule 2).
We begin with a weaker statement: a characterization over the integers
rather than the polynomial ring.
Although this is not the desired result,
the proof is illuminating, as it clarifies the relationship between
the map $M$ and the sphere parameterization (\ref{eq:param1}).

\begin{theorem}
\label{thm:ntuples}
Any Pythagorean $n$-tuple over {\cal Z}
\[
	x_1^2 + \ldots + x_{n-1}^2 = x_n^2
\]
can be generated by 
\[
	x_1 = \alpha (a_1^2+\ldots+a_{n-2}^2-d^2)
\]
\[
	x_2 = \alpha (2a_1d)
\]
\[
	\ldots
\]
\[
	x_{n-1} = \alpha (2a_{n-2}d)
\]
\[
	x_n = \alpha (a_1^2+\ldots+a_{n-2}^2+d^2)
\]
where $a_1,\ldots,a_{n-2},d \in {\cal Z}$ and $\alpha \in {\cal Q}$.
\end{theorem}
\prf
% See p. 88 of Ebbinghaus as motivation.
Let $(x_1,\ldots,x_n)$ be a Pythagorean $n$-tuple over {\cal Z}.
Then $(\frac{x_1}{x_n},\ldots,\frac{x_{n-1}}{x_n}) \in {\cal Q}^{n-1}$ 
is a point on the ($n-2$)-sphere, by Rule 1.
Assume $(x_1,\ldots,x_n) \neq (1,0,\ldots,0)$
(otherwise let $d=0, \alpha = x_1, a_1=1, \mbox{ and } a_i = 0$ for $i\neq 1$).
Since it is a rational point, $(\frac{x_1}{x_n},\ldots,\frac{x_{n-1}}{x_n})$
has a rational parameter value in the sphere parameterization (\ref{eq:param1}),
say $(t_1,\ldots,t_n) = (\frac{a_1}{d},\ldots,\frac{a_{n-2}}{d})$
where $a_1,\ldots,a_{n-2},d \in {\cal Z}$.
That is,
\[
 (\frac{x_1}{x_n},\ldots,\frac{x_{n-1}}{x_n})
 = (\frac{(\frac{a_1}{d})^2 + \ldots + (\frac{a_{n-2}}{d})^2 - 1}
         {(\frac{a_1}{d})^2 + \ldots + (\frac{a_{n-2}}{d})^2 + 1},
    \frac{\frac{2a_1}{d}}
         {(\frac{a_1}{d})^2 + \ldots + (\frac{a_{n-2}}{d})^2 + 1},
    \ldots,
    \frac{\frac{2a_{n-2}}{d}}
         {(\frac{a_1}{d})^2 + \ldots + (\frac{a_{n-2}}{d})^2 + 1},
\]
or
\[
(x_1,\ldots,x_{n-1}) = 
(\frac{x_n}{a_1^2 + \ldots + a_{n-2}^2 + d^2})
(a_1^2 + \ldots + a_{n-2}^2 - d^2, 2a_1 d, \ldots, 2a_{n-2} d).
\]
Let $\alpha = \frac{x_n}{a_1^2 + \ldots + a_{n-2}^2 + d^2}$.
\QED

\begin{corollary}
Any Pythagorean 5-tuple $(x_1,\ldots,x_5)$ over {\cal Z} can be generated
by $(x_1,\ldots,x_5) = \alpha M(a,b,c,d)$,
where $a,b,c,d \in {\cal Z}$ and $\alpha \in {\cal Q}$.
\end{corollary}


% \subsection{Characterizations of Pythagorean tuples over polynomial rings}

Unfortunately, the proof technique of Theorem~\ref{thm:ntuples} 
does not generalize to polynomial rings,
since the use of rational numbers no longer applies, as we are working
with polynomials with real coefficients.
We use a different proof technique in the following theorem, motivated
by Kubota \cite{kubota72}.

% \begin{theorem}[Kubota 1972]
% \label{thm:kubota}
% Any Pythagorean triple over $D$
% \[
% 	x_1^2 + x_2^2 = x_3^2
% \]
% can be generated by
% \[
% 	x_1 = \alpha (a^2 - b^2)
% \]
% \[
% 	x_2 = \alpha (2ab)
% \]
% \[
% 	x_3 = \alpha (a^2 + b^2)
% \]
% where $\alpha, a,b \in D$ and $D$ is 
% any unique factorization domain of characteristic $d\neq2$
% such that the element 2 is prime or invertible (such as $\Re$ or $\Re[t]$).
% \end{theorem}

\begin{theorem}
\label{thm:complete}
Any Pythagorean $n$-tuple over $\Re[t]$
\[
	x_1^2 + \ldots + x_{n-1}^2 = x_n^2
\]
can be generated by 
\begin{equation}
\label{eq:pythn2}
	x_1 = \alpha (a_1^2+\ldots+a_{n-2}^2-d^2)
\end{equation}
\[
	x_2 = \alpha (2a_1d)
\]
\[
	\ldots
\]
\[
	x_{n-1} = \alpha (2a_{n-2}d)
\]
\[
	x_n = \alpha (a_1^2+\ldots+a_{n-2}^2+d^2)
\]
where $a_1,\ldots,a_{n-2},d,\frac{1}{\alpha} \in \Re[t]$.
% Alternate version:
% where $\alpha,d \in \Re[t]$ 
% and $a_1,\ldots,a_{n-2} \in \frac{\Re[t]}{\Re[t]}$.
% {\bf (We would prefer all in $\Re[t]$ but cannot yet use Kubota's 
% technique---or any other--- to establish this for this more general case.)}
\end{theorem}
\prf
Kubota \cite{kubota72} proved a version of this result for $n=3$ 
(and also extended $\Re[t]$ to more general unique factorization domains).
We use a proof technique motivated by Kubota.
% (We relax the restriction on $\alpha$ to make $a_i$ polynomial).
Let $(x_1,\ldots,x_n)$ be a Pythagorean $n$-tuple over $\Re[t]$.
Assume $x_n - x_1 \neq 0$ (otherwise the result is trivial).
Factor $x_n - x_1$ into a square component and a square-free component:
$x_n - x_1 = gh^2$ where $g,h \in \Re[t]$ and $g$ and $h$ are both square-free.
This factorization is always possible.
Let 
\[
\begin{array}{lll}
\hspace{2in} & a_i = hx_{i+1} & \hspace{1in} i=1,\ldots,n-2 \\
& d = h(x_n - x_1) \\
& \alpha = \frac{g}{2(x_n - x_1)^2}
\end{array}
\]
% Alternate version: 
% Let 
% \[
% \begin{array}{lll}
% \hspace{2in} & a_i = \frac{hx_{i+1}}{x_n - x_1} & \hspace{1in} i=1,\ldots,n-2 \\
% & d = h \\
% & \alpha = \frac{g}{2} 
% \end{array}
% \]
Then $a_1,\ldots,a_{n-2},d,\alpha$ generate the Pythagorean $n$-tuple
$(x_1,\ldots,x_n)$ as in (\ref{eq:pythn2}).
For example,
\[
\alpha (a_1^2 + \ldots + a_{n-2}^2 - d^2)
= \frac{g}{2} 
  (\frac{h^2(x_2^2 + \ldots + x_{n-1}^2) - h^2(x_n - x_1)^2}{(x_n-x_1)^2})
\]
and applying $x_1^2 + \ldots + x_{n-1}^2 = x_n^2$ and $x_n - x_1 = gh^2$,
\[
= \frac{1}{2} (\frac{-2x_1^2 + 2x_1x_n}{gh^2})
= \frac{x_1(x_n - x_1)}{x_n - x_1} = x_1.
\]
Or, as another example,
\[
\alpha (2a_1 d) = \frac{g}{2(x_n - x_1)^2} (2h^2 x_2) (x_n - x_1)
		= \frac{gh^2 x_2}{x_n - x_1} = x_2.
\]
Finally, $\alpha = \frac{g}{2(x_n - x_1)^2} = \frac{g}{2(gh^2)^2} 
= \frac{1}{2gh^4}$, so $\frac{1}{\alpha} \in \Re[t]$.
\QED

\begin{corollary}
\label{thm:quintuplecondition}
Any Pythagorean quintuple over $\Re[t]$
can be generated by $\alpha M(a,b,c,d)$
for some $a,b,c,d,\frac{1}{\alpha} \in \Re[t]$.
%\[
%	x_1 = \alpha (a^2+b^2+c^2-d^2)
%\]
%\[
%	x_2 = \alpha (2ad)
%\]
%\[
%	x_3 = \alpha (2bd)
%\]
%\[
%	x_4 = \alpha (2cd)
%\]
%\[
%	x_5 = \alpha (a^2+b^2+c^2+d^2)
%\]
\end{corollary}

This finally yields the desired result.

\begin{corollary}
Any rational curve $C$ on the 3-sphere can be generated as the
image under $M$ of some polynomial curve $D$ in 4-space: $C = M(D)$.
\end{corollary}
\prf
Let $C = (x_1(t),\ldots,x_5(t))$ be a rational curve on $S^3$.
Since $(x_1,\ldots,x_5)$ is a Pythagorean quintuple,
$(x_1,\ldots,x_5) = \alpha M(a,b,c,d)$ 
for some $a,b,c,d,\frac{1}{\alpha} \in \Re[t]$,
or $\frac{1}{\alpha}(x_1,\ldots,x_5) = M(a,b,c,d)$.
But $\frac{1}{\alpha}(x_1,\ldots,x_5)$ is the same rational curve 
as $(x_1,\ldots,x_5)$, since points in projective space are invariant
under multiplication by a constant.
% since $(x_1,x_2,x_3,x_4,x_5) = 
% (\frac{x_1}{x_5},\ldots,\frac{x_4}{x_5}) = 
% (\frac{kx_1}{kx_5},\ldots,\frac{kx_4}{kx_5})$ (expressed first in projective
% space and then in affine space) for any $k \in \Re[t]$.
Thus, $C = M(a,b,c,d) = M(D)$ for some polynomial curve $D$ in 4-space.
\QED

Dietz et. al. \cite{dietz93} establish an analogous but slightly weaker
result for curves on the 2-sphere:
any rational curve on the 2-sphere can be generated as
the image under $\delta$ of some {\em rational} curve in 3-space.
It is important that we need only images of {\em polynomial} curves for
completeness, since we will design polynomial curves in the image space.

% \begin{theorem}[Dietz et. al. 1993]
% Any Pythagorean quadruple over $\Re[t]$
% \[
% 	x_1^2 + x_2^2 + x_3^2 = x_4^2
% \]
% can be generated by 
% \[
% 	x_1 = \alpha (2ad - 2bc)
% \]
% \[
% 	x_2 = \alpha (2ac + 2bd)
% \]
% \[
% 	x_3 = \alpha (a^2 + b^2 - c^2 - d^2)
% \]
% \[
% 	x_4 = \pm \alpha (a^2 + b^2 + c^2 + d^2)
% \]
% where $\alpha,a,b,c,d \in \Re[t]$.
% \end{theorem}

% \begin{corollary}[Dietz et. al. 1993]
% "Any irreducible rational Bezier curve of degree $2n$ on the [2-sphere]
% can be obtained as the image of a rational Bezier curve of degree $n$
% under $\delta$" \cite[p. 216]{dietz93}.
% \end{corollary}

\section{The inverse map}
\label{sec:invM}

The inverse map $M^{-1}$ is also needed,
to map input points from the 3-sphere to image space (Figure~\ref{fig2}).
Note that this map need not be rational, since it is only applied to
a finite number of points.
Lemma~\ref{lem:invM} defines the inverse image of a point in projective 4-space,
and Corollary~\ref{cor:inverseline}
defines the inverse image of a point in affine 4-space.

\begin{lemma}
\label{lem:invM}
The inverse of $M$, $M^{-1}:S^3 \rightarrow P^4$, is
\begin{equation}
\label{eq:invM}
M^{-1} (x_1,x_2,x_3,x_4,x_5) = 
\left\{ \begin{array}{ll}
\mbox{\footnotesize{$(x_2,x_3,x_4,x_5-x_1,\sqrt[+]{2(x_5-x_1)})$}}
	& \mbox{if } x_1 \neq x_5 \\
\mbox{the hyperplane } x_4 = 0 
	& \mbox{if } x_1 = x_5
\end{array} \right.
\end{equation}
where $x_5 > 0$.
\end{lemma}
\prf
%
% DIRECT, LONG PROOF:
% Let $X = M(1,p,q,r,s)$, $X := (x_0,x_1,x_2,x_3,x_4)$.
% Then $M^{-1}(X) = (1,p,q,r,s)$.
% Suppose $s \neq 0$.
% From $X = (p^2+q^2+r^2+s^2, p^2+q^2+r^2-s^2, 2ps, 2qs, 2rs)$,
% $p = \frac{x_2}{2s}$, $q = \frac{x_3}{2s}$, $r = \frac{x_4}{2s}$.
% Substituting into $x_0$ and $x_1$, we get 
% $ 4s^4 - (4x_0)s^2 + (x_2^2 + x_3^2 + x_4^2) = 0
% 4s^4 + (4x_1)s^2 - (x_2^2 + x_3^2 + x_4^2)$,
% or $4s^2(2s^2+ x_1 - x_0) = 0$.
% Thus, $s=0$ or $s = \pm \sqrt{\frac{x_0 -x_1}{2}}$.
%
If $x_1 \neq x_5$, $M^{-1}(M(p,q,r,s,1)) = 
 M^{-1}(p^2+q^2+r^2-s^2,2ps,2qs,2rs,p^2+q^2+r^2+s^2) = 
 (2ps,2qs,2rs,2s^2,2s) = (p,q,r,s,1)$.
%
$(x_1,x_2,x_3,x_4,x_5) \in S^3$ and $x_1 = x_5$ implies that
$(x_1,x_2,x_3,x_4,x_5) = (1,0,0,0,1)$.
Since $M(x_1,x_2,x_3,0,1) = (1,0,0,0,1)$, the inverse image
of $(1,0,0,0,1)$ contains the entire hyperplane $x_4=0$.
It is easy to see that it only contains this hyperplane.
%
Finally, notice that there are no problems with square roots of negative
numbers in (\ref{eq:invM}), since $x_5-x_1 \geq 0$ whenever $x_5 > 0$ and
$(x_1,x_2,x_3,x_4,x_5) \in S^3$.
\QED

\begin{corollary}
\label{cor:inverseline}
The inverse image of the affine point $(x_1,x_2,x_3,x_4) \in S^3 \subset \Re^4$,
$(x_1,x_2,x_3,x_4) \neq (1,0,0,0)$,
is the ray from the origin through $(x_2,x_3,x_4,1-x_1)$.
The inverse image of $(1,0,0,0)$ is still the hyperplane $x_4=0$.
\end{corollary}
\prf
The affine point $(x_1,x_2,x_3,x_4) \in S^3$ is equivalent to the points
$\{ (kx_1,kx_2,kx_3,kx_4,k): k \neq 0 \}$ in projective space,
and we restrict to $k>0$.
$M^{-1}(kx_{1},kx_{2},kx_{3},kx_{4},k) =$ 
$(kx_{2},kx_{3},kx_{4},k(1-x_{1}),\sqrt{2(1-x_{1})} \sqrt{k})$, 
or in affine space 
$\frac{\sqrt{k}}{\sqrt{2(1-x_{1})}} (x_{2},x_{3},x_{4},1-x_{1})$,
so $M^{-1}\{(kx_{1},kx_{2},kx_{3},kx_{4},k): k > 0 \}$
is the ray from the origin through $(x_{2},x_{3},x_{4},1-x_{1})$.
The affine point $(1,0,0,0)$ is equivalent to the points
$(x_1,x_2,x_3,x_4,x_5)$ with $x_1=x_5 \neq 0$ in projective space.
\QED

The input points on the 3-sphere are therefore mapped to lines in image space.
We could interpolate these lines with a curve.
However, we prefer to use point interpolation rather than line interpolation,
since this allows
any number of simple, classical point interpolation methods to be inherited
from 3-space to 4-space.
These point interpolation methods are inexpensive, well understood
by designers, and already available in systems.
Put another way, the interpolation of lines rather than points offers
more degrees of freedom to the interpolation, but more difficulty.
For the present problem, there is sufficient freedom in the interpolation of points
to generate a good curve, without the added expense of line interpolation.

% also lose automatic C^2 continuity of image curve with lines

In the 2-sphere case, Dietz et. al. \cite{dietz93}
use inverse image lines, which they interpolate by a single Bezier curve
by solving a linear system.
The degree of the Bezier curve is high: at least $d-1$ where $d$ is
the number of input points.
The added freedom of line interpolation has intriguing
potential for the design of curves with more optimality criteria, 
such as curves on the 3-sphere with minimal tangential acceleration \cite{barr92}.
% angular velocity?
It may simplify this optimization in image space to add the freedom
of interpolating a curve through lines rather than points.
This is a topic for future study.

To enable point interpolation, we need to
choose a single point on the inverse image
line $M^{-1}(X)$ to act as the inverse image of $X$.
We make the natural choice:
\[
M^{-1} (x_1,x_2,x_3,x_4) := M^{-1} (x_1,x_2,x_3,x_4,1) 
\]
Serendipitously,  
$M^{-1} (x_1,x_2,x_3,x_4)$ now lies on the sphere $S^3$!
($\frac{x_2^2+x_3^2+x_4^2 + (1-x_1)^2}{2(1-x_1)} = \frac{2-2x_1}{2-2x_1} = 1$.)
This is an added advantage,
since the interpolating curve in image space will lie close
to the 3-sphere and thus avoid the origin,
which it must do since $M$ is undefined at the origin.\footnote{The 
	interpolating curve could still approach the origin if
	consecutive $M^{-1}(P_i)$ lie on opposite sides of the sphere,
	but it is implicitly understood that consecutive input points $P_i$,
	and thus consecutive $M^{-1}(P_i)$ using Lemma~\ref{lem:invM},
	lie relatively close together on $S^3$ and, in particular, 
	are far from antipodal.
	This is certainly the case when the points are
	quaternions, since antipodal quaternions represent the same
	orientation.}
It is also useful for interactive design and manipulation of the curve.
Since the inverse image is still a point on the 3-sphere,
one may ask what we have accomplished by mapping to image space.
The benefit is that we can now freely interpolate these new points,
rather than restricting the interpolation to the sphere.

It still remains to define the inverse image of the special point $(1,0,0,0)$.
We choose to define $M^{-1}(1,0,0,0)$ so that it too lies on the 3-sphere,
and we choose the first three coordinates equal since $M^{-1}_{y \neq 0}(x,y,y,y)$
is a point whose first three coordinates are equal:
\[ M^{-1}(1,0,0,0) := (\frac{\sqrt{3}}{3},\frac{\sqrt{3}}{3},\frac{\sqrt{3}}{3}, 0)
\]

We now know how to map input points on the 3-sphere to points in 
image space using $M^{-1}$.
This is the first step of our algorithm (Section~\ref{sec:general}).
The second step is to interpolate these points in image space by a curve
(Figure~\ref{fig3}).
This can be done using any number of classical methods.
The most natural choice is probably a cubic B-spline interpolating curve \cite{farin93},
and this is the curve that we use in all of our examples.

\begin{rmk}
\label{rmk:1000}
Since all points on the $x_4=0$ plane map under $M$ to the same point
on the 3-sphere, $(1,0,0,0)$, it is important that our curve in image space
not have components on the $x_4=0$ plane.
This is true since an interpolating curve will only have a component in a plane
$P$ if at least two consecutive data points lie in $P$;
but the only $X \in S^3$ with $M^{-1}(X)$ in $x_4=0$ is $(1,0,0,0)$.
\end{rmk}

% Does restricting to point on image line affect completeness result? No.
% Incidentally, by collapsing lines to points in the image space,
% we have not affected the completeness result: since every point on a line
% of Corollary~\ref{cor:inverseline} maps to the same point on $S^3$,
% choosing one specific point on the line is immaterial to the final curve
% on $S^3$.

\section{The image of a curve}
\label{sec:curveimage}

The final step is to map the interpolating curve in image space 
back to an interpolating curve on the 3-sphere using $M$ (Figure~\ref{fig4}).
The following theorem shows how this mapping is done, segment by segment,
for the important case of a cubic polynomial curve in image space,
expressed as a Bezier curve.

\begin{theorem}
\label{sextic}
Let $c(t)$ be a polynomial cubic Bezier curve in 4-space with
control points $b_i = (b_{i1},b_{i2},b_{i3},b_{i4})$, ($i=0,\ldots,3$).
The image of $c(t)$ under $M$ is a rational sextic Bezier curve with 
control points $c_k$ ($k = 0, \ldots, 6$):
\begin{equation}
\label{eq:control-pts}
c_k = \frac{1}{w_k} 
      \sum_{\begin{array}{c} \mbox{\footnotesize{$0 \leq i \leq 3$}} \\ 
			     \mbox{\footnotesize{$0 \leq j \leq 3$}} \\ 
			     \mbox{\footnotesize{$i+j=k$}}
			     \end{array}} 
        \frac{\choice{3}{i} * \choice{3}{j}}{\choice{6}{k}}
	\left( \begin{array}{c}
            b_{i1} b_{j1} + b_{i2} b_{j2} + b_{i3} b_{j3} - b_{i4} b_{j4} \\
            2b_{i1} b_{j4} \\
            2b_{i2} b_{j4} \\
            2b_{i3} b_{j4} 
	\end{array} \right)
\end{equation}
and weights $w_k$:
\begin{equation}
\label{eq:weights}
w_k = \sum_{\begin{array}{c} \mbox{\footnotesize{$0 \leq i \leq 3$}} \\ 
			     \mbox{\footnotesize{$0 \leq j \leq 3$}} \\ 
			     \mbox{\footnotesize{$i+j=k$}}
			     \end{array}}
        \frac{\choice{3}{i} * \choice{3}{j}}{\choice{6}{k}}
	(b_{i1} b_{j1} + b_{i2} b_{j2} + b_{i3} b_{j3} + b_{i4} b_{j4})
\end{equation}
\end{theorem}
\prf
Let $M(c(t)) = M(\sum_{i=0}^3 B_i^3(t) b_{i}) 
:= (m_1(t),m_2(t),m_3(t),m_4(t),m_5(t))$, 
where $B_i^n(t)$ is the $i^{\mbox{th}}$ Bernstein polynomial of degree $n$.
Consider the coordinate
\[ m_5(t) =  [\sum_{i=0}^3 B_i^3(t) b_{i1}]^2 + 
	\ldots + [\sum_{i=0}^3 B_i^3(t) b_{i4}]^2
     =   \sum_{i=0}^3 \sum_{j=0}^3 
	\frac{\choice{3}{i} * \choice{3}{j}}{\choice{6}{i+j}}
       B^6_{i+j}(t) (b_{i1} b_{j1} + \ldots + b_{i4} b_{j4})
\]
by the product rule of Bernstein polynomials \cite{farin93}.
Letting $k=i+j$, this becomes
\[ \sum_{k=0}^6 B_k^6(t) 
	\sum_{\begin{array}{c}  \mbox{\footnotesize{$0 \leq i \leq 3$}} \\ 
			     \mbox{\footnotesize{$0 \leq j \leq 3$}} \\ 
			     \mbox{\footnotesize{$i+j=k$}}
			     \end{array}} 
	\frac{\scriptchoice{3}{i} * \scriptchoice{3}{j}}{\scriptchoice{6}{k}}
	(b_{i1} b_{j1} + \ldots + b_{i4} b_{j4}) \]
Computing the other coordinates analogously yields
\[ M(c(t)) = 
   \sum_{k=0}^6 B_k^6(t)
	\sum \frac{\choice{3}{i} * \choice{3}{j}}{\choice{6}{k}}
	\left( \begin{array}{c}
            b_{i1} b_{j1} + b_{i2} b_{j2} + b_{i3} b_{j3} - b_{i4} b_{j4} \\
            2b_{i1} b_{j4} \\
            2b_{i2} b_{j4} \\
            2b_{i3} b_{j4} \\
            b_{i1} b_{j1} + b_{i2} b_{j2} + b_{i3} b_{j3} + b_{i4} b_{j4}
	\end{array} \right) \]
% \[         = \sum_{k=0}^6 B_k^6(t) 
%	\sum_{\begin{array}{c} 0 \leq i \leq 3 \\ 
%			     0 \leq j \leq 3 \\ 
%			     i+j=k
%			     \end{array}} 
%       \frac{\choice{3}{i} * \choice{3}{j}}{\choice{6}{k}}
%	\left( \begin{array}{c}
%           b_{i1} b_{j1} + b_{i2} b_{j2} + b_{i3} b_{j3} + b_{i4} b_{j4} \\
%            w_k (\frac{b_{i1} b_{j1} + b_{i2} b_{j2} + b_{i3} b_{j3} - b_{i4} b_{j4}}{w_k}) \\
%            w_k (\frac{2b_{i1} b_{j4}}{w_k}) \\
%            w_k (\frac{2b_{i2} b_{j4}}{w_k}) \\
%            w_k (\frac{2b_{i3} b_{j4}}{w_k})
%	\end{array} \right) \]
%
which is a sextic rational Bezier curve with 
control points (\ref{eq:control-pts}) and weights (\ref{eq:weights}).
\QED

Notice how the map $M$ is hidden in the formula for the control points 
and weights (compare (\ref{eq:control-pts})-(\ref{eq:weights}) and (\ref{eq:M})).

Since $M$ is a rational map,
the curve on the 3-sphere inherits the continuity of the curve in image space.
Consequently, since it is simple to define interpolating curves
in image space with $C^2$ continuity, or curves of higher degree with
even higher continuity, 
the curve on $S^3$ can easily have $C^2$ continuity
and indeed arbitrary continuity.
Other interpolation methods (e.g., slerping methods, spherical biarc methods)
have considerable difficulty with continuity.

\section{Examples}
\label{sec:eg}

We present more examples of the method in Figures~\ref{fig7}-\ref{fig8}.
The image curve is again a cubic B-spline interpolating curve.
Notice how well-behaved the curves are, inheriting the quality
of the image curves.
We have experimented with many other examples, with excellent results.

To visualize this four-dimensional problem in 3-space,
we use a standard trick.
We use input points with $x_3=0$, which can be visualized by removing
the $x_3$ component.
Fortunately, points on $S^3$ with $x_3=0$ map under $M^{-1}$ to points in 4-space with $x_2=0$,
and points in 4-space with $x_2=0$ map under $M$ to points on $S^3$ with $x_3=0$.
Thus, we can visualize points in the original space (on $S^3$) by removing
the $x_3$ component, and points in image space by removing the $x_2$ 
component.\footnote{The special case
$M^{-1}(1,0,0,0) = (\frac{\sqrt{3}}{3}, \frac{\sqrt{3}}{3}, \frac{\sqrt{3}}{3}, 0)$
is a problem for this visualization technique, 
since $x_2 \neq 0$ for this point in image space.
We use $M^{-1}(1,0,0,0) = (\frac{\sqrt{2}}{2}, 0, \frac{\sqrt{2}}{2}, 0)$
for purposes of visualization.}
Of course, all computations are still performed in 4-space and projection to 3-space
is only used for visualization.

\begin{figure}
\vspace{2.4in}
\special{psfile=/rb/jj/Research/S3curve/img/fig7.ps hoffset=150}
\caption{Another S3 curve}
\label{fig7}
\end{figure}
% GLUTS3curve -i < data/input3

\begin{figure}
\vspace{2.4in}
\special{psfile=/rb/jj/Research/S3curve/img/fig8.ps hoffset=150}
\caption{And another}
\label{fig8}
\end{figure}
% GLUTS3curve -i < data/input6

% \begin{figure}[h]
% \vspace{2.4in}
% \special{psfile=/rb/jj/Research/S3curve/img/fig9.ps}
% \caption{Yet another}
% \label{fig9}
% \end{figure}
% GLUTS3curve < data/wavyinput

\section{Conclusions}
\label{sec:conclude}

This paper has explored a new approach to the design of curves
on the 3-sphere that is capable of building any rational curve on the 3-sphere.
Using Euler's Four Squares Theorem, a rational map to the 3-sphere was developed 
that is onto all points and all curves on the 3-sphere.
The inverse of this map was used to map points on $S^3$ into an image space,
where the design of an interpolating curve is no longer constrained to
the 3-sphere and can thus be approached using known classical techniques.
This polynomial curve in image space is then mapped back to the sphere using the 
rational map to the 3-sphere.

Our curve inherits all of the favourable properties of rational curves,
as well as many favourable properties of the specific curve or curve type
(e.g., Bezier) in image space, such as continuity and Bezier structure.
For example, if Bezier curves are designed as in Section~\ref{sec:curveimage},
the curve on the 3-sphere enjoys all of the advantages of Bezier curves, such as
efficient computation, subdivision, local control, affine invariance, 
variation diminution, as well as a predictable behaviour and ease of 
implementation because of the rich understanding of Bezier curves.
Since the curve has a complete analytic description, it allows
simple manipulation and complete control.
This construction answers many of the challenges for future work
outlined by Shoemake in his paper \cite{shoemake85}.

% A topic for future study is the design of curves with more explicit optimality
% criteria, such as the minimal tangential acceleration of Barr et.\ al.\ \cite{barr92},
% using our method.
% Since the curve is designed in an image space,
% this must be approached differently.

\parindent=-20mm

\section{Acknowledgements}

This work was supported in part
by the National Science Foundation under grant CCR-9213918.

%\bibliographystyle{unsrt}
\bibliographystyle{plain}
\begin{thebibliography}{99}

\bibitem{barr92}
Barr, A.H., B. Currin, S. Gabriel and J.F. Hughes (1992)
Smooth interpolation of orientations with angular velocity
constraints using quaternions.  SIGGRAPH '92, Chicago, 26(2), 313--320.

\bibitem{boehmHansford91}
Boehm, W. and D. Hansford (1991)
Bezier Patches on Quadrics.
In {\em NURBS for Curve and Surface Design}, edited by G. Farin,
SIAM, Philadelphia, 1--14.

\bibitem{dickson52}
Dickson, L.E. (1952) History of the theory of numbers: Volume II,
Diophantine analysis.  Chelsea (New York).

\bibitem{dietz93}
Dietz, R., J. Hoschek and B. J\"{u}ttler (1993)
An algebraic approach to curves and surfaces on the sphere and on other
quadrics.  Computer Aided Geometric Design 10, 211--229.

\bibitem{duff85}
Duff, T. (1985)
Quaternion splines for animating orientation.
1985 Monterey Computer Graphics Workshop, 54--62.

\bibitem{ebbinghaus90}
Ebbinghaus, H.-D., H. Hermes, F. Hirzebruch, M. Koecher, K. Mainzer,
J. Neukirch, A. Prestel and R. Remmert (1990)
Numbers.
Springer-Verlag (New York).

\bibitem{farin93}
Farin, G. (1993) Curves and surfaces for computer aided geometric design.
Academic Press (New York), third edition.

\bibitem{hoschekSeemann92}
Hoschek, J. and G. Seemann (1992)
Spherical splines.
Mathematical Modeling and Numerical Analysis, 26(1), 1--22.

\bibitem{jjjw95}
Johnstone, J. and J. Williams (1995)
Rational Control of Orientation for Animation.
{\em Graphics Interface '95}, Quebec City, 179--186.

\bibitem{jjjw95b}
Johnstone, J.K. and J. Williams (1995)
A Rational Model of the Surface Swept by a Curve.
{\em Computer Graphics Forum}, 14(3), Proceedings of Eurographics '95,
Maastricht, 77--88.

\bibitem{junkins86}
Junkins, J. and J. Turner (1986)
Optimal Spacecraft Rotational Maneuvers.
Elsevier, New York.

\bibitem{kim95}
Kim, M.-J., M.-S. Kim and S. Shin (1995)
A general construction scheme for unit quaternion curves with simple
high order derivatives.
SIGGRAPH '95, Los Angeles, 369--376. 

\bibitem{kubota72}
Kubota, K.K. (1972) Pythagorean triples in unique factorization domains.
American Mathematical Monthly 79, 503--505.

\bibitem{levinson70}
Levinson, N. and R.M. Redheffer (1970)
Complex Variables.
Holden-Day (San Francisco).

\bibitem{nam95}
Nam, K.-W. and M.-S. Kim (1995)
Hermite interpolation of solid orientations based on a smooth blending
of two great circular arcs on SO(3).
Proc. of CG International '95.

\bibitem{nielson92}
Nielson, G. and R. Heiland (1992)
Animated rotations using quaternions and splines on a 4D sphere.
Programming and Computer Software, 145--154.

\bibitem{nielson93}
Nielson, G. (1993)
Smooth interpolation of orientations.
In Models and Techniques in Computer Animation, Springer-Verlag (New York),
75--93.

\bibitem{pletinckx89}
Pletinckx, D. (1989) 
Quaternion calculus as a basic tool in computer graphics.
The Visual Computer 5, 2--13.

\bibitem{schlag91}
Schlag, J. (1991) Using geometric constructions to interpolate
orientation with quaternions.  In Graphics Gems II, Academic Press (New York),
377--380.

\bibitem{shoemake85}
Shoemake, K. (1985) Animating rotation with quaternion curves.
SIGGRAPH '85, San Francisco, 19(3), 245--254.

\bibitem{wang93}
Wang, W. and B. Joe (1993)
Orientation interpolation in quaternion space using spherical biarcs.
{\em Graphics Interface '93}, 24--32.

\bibitem{wang95}
Wang, W. (1995) 
Rational spherical curves.  Technical Report, Dept. of Computer Science,
University of Hong Kong.

\end{thebibliography}

\end{document}

% Problems of Shoemake solved by our method:
% (1) trigonometric slerping (Slerp(p,q,u) = $(p:q)_u$) replaced by fully polynomial
% interpolation (added speed, robustness, compatibility);
% (2) can add new points on interpolating curve trivially since curve is Bezier,
% whereas not with Shoemake: "there are simple algorithms for adding new sequence
% points to ordinary splines without altering the original curve; they do not
% work for this interpolant" (p. 251);
% (3) answers Shoemake's query "is there is some related interpolant [to Shoemake's
% spherical `Bezier' curves] that is well-characterized?":
% our curves are true Bezier curves in 4-space not pseudo-Bezier curves;
% (4) Cannot achieve  Shoemake's goal of a spherical curve parameterized by
% arc length (p. 251) using arguments of Farouki, but we will try to control
% the speed by knot sequences (and possibly by PH-curve techniques?)

% Conclusion: an improvement of Shoemake's spherical interpolation technique
% (moreover addressing all of the issues that Shoemake left open for improvement).
% and without resorting to costly and nonpolynomial optimization techniques.

Future work:

1) How can we use the added freedom of line interpolation to design
	curves with more optimal behaviour on the sphere?

2) Can our method be extended to interpolating curves on quadrics in 4-space, using
ideas analogous to pp. 215,228 of Dietz et. al.?

\bibitem{boehmHansford91}
Boehm, W. and D. Hansford (1991)
Bezier Patches on Quadrics.
In {\em NURBS for Curve and Surface Design}, edited by G. Farin,
SIAM, Philadelphia, 1--14.

\bibitem{dietz93}
Dietz, R., J. Hoschek and B. J\"{u}ttler (1993)
An algebraic approach to curves and surfaces on the sphere and on other
quadrics.  Computer Aided Geometric Design 10, 211--229.

\bibitem{hoschekSeemann92}
Hoschek, J. and G. Seemann (1992)
Spherical splines.
Mathematical Modeling and Numerical Analysis, 26(1), 1--22.

\bibitem{shoemake85}
Shoemake, K. (1985) Animating rotation with quaternion curves.
SIGGRAPH '85, San Francisco, 19(3), 245--254.

\bibitem{jjjw95b}
Johnstone, J.K. and J. Williams (1995)
A Rational Model of the Surface Swept by a Curve.
{\em Computer Graphics Forum}, 14(3), Proceedings of Eurographics '95,
Maastricht, 77--88.

\bibitem{junkins86}
Junkins, J. and J. Turner (1986)
Optimal Spacecraft Rotational Maneuvers.
Elsevier, New York.

\bibitem{jjjw95}
Johnstone, J. and J. Williams (1995)
Rational Control of Orientation for Animation.
{\em Graphics Interface '95}, Quebec City, 179--186.

\bibitem{duff85}
Duff, T. (1985)
Quaternion splines for animating orientation.
1985 Monterey Computer Graphics Workshop, 54--62.

\bibitem{pletinckx89}
Pletinckx, D. (1989) 
Quaternion calculus as a basic tool in computer graphics.
The Visual Computer 5, 2--13.

\bibitem{schlag91}
Schlag, J. (1991) Using geometric constructions to interpolate
orientation with quaternions.  In Graphics Gems II, Academic Press (New York),
377--380.

\bibitem{nielson92}
Nielson, G. and R. Heiland (1992)
Animated rotations using quaternions and splines on a 4D sphere.
Programming and Computer Software, 145--154.

\bibitem{nielson93}
Nielson, G. (1993)
Smooth interpolation of orientations.
In Models and Techniques in Computer Animation, Springer-Verlag (New York),
75--93.

\bibitem{kim95}
Kim, M.-J., M.-S. Kim and S. Shin (1995)
A general construction scheme for unit quaternion curves with simple
high order derivatives.
SIGGRAPH '95, Los Angeles, 369--376. 

\bibitem{nam95}
Nam, K.-W. and M.-S. Kim (1995)
Hermite interpolation of solid orientations based on a smooth blending
of two great circular arcs on SO(3).
Proc. of CG International '95.

\bibitem{wang93}
Wang, W. and B. Joe (1993)
Orientation interpolation in quaternion space using spherical biarcs.
{\em Graphics Interface '93}, 24--32.

\bibitem{wang95}
Wang, W. (1995) 
Rational spherical curves.  Technical Report, Dept. of Computer Science,
University of Hong Kong.

\bibitem{barr92}
Barr, A.H., B. Currin, S. Gabriel and J.F. Hughes (1992)
Smooth interpolation of orientations with angular velocity
constraints using quaternions.  SIGGRAPH '92, Chicago, 26(2), 313--320.

\bibitem{dickson52}
Dickson, L.E. (1952) History of the theory of numbers: Volume II,
Diophantine analysis.  Chelsea (New York).

\bibitem{ebbinghaus90}
Ebbinghaus, H.-D., H. Hermes, F. Hirzebruch, M. Koecher, K. Mainzer,
J. Neukirch, A. Prestel and R. Remmert (1990)
Numbers.
Springer-Verlag (New York).

\bibitem{levinson70}
Levinson, N. and R.M. Redheffer (1970)
Complex Variables.
Holden-Day (San Francisco).

\bibitem{kubota72}
Kubota, K.K. (1972) Pythagorean triples in unique factorization domains.
American Mathematical Monthly 79, 503--505.

\bibitem{farin93}
Farin, G. (1993) Curves and surfaces for computer aided geometric design.
Academic Press (New York), third edition.

