A calculus of sweeping
----------------------

(1) A Bezier representation of ringed surfaces

A ringed surface is defined by a (center, radius, orientation) triple:
(c(t), r(t), o(t)).  It is the surface constructed by sweeping a circle
such that, at time t, its center lies at c(t), its radius is r(t), and
the normal of its plane is o(t).
Let c(t) and o(t) be Bezier curves, and r(t) be a Bezier function.
A parameterization of this ringed surface is $M_3(t) M_2(t) M_1(t) C(s)$, 
where C(s) is a rational parameterization of the unit circle centered at
the origin in the xy-plane, M_1(t) is the transformation matrix for
translation by c(t), M_2(t) is the transformation matrix for scaling
by r(t), and M_3(t) is the transformation matrix for rotation of the
z-axis to o(t).
If each of these elements is rational, then the entire
parameterization is rational, and we can determine its control
points through analysis of products of Bernstein polynomials or blossoming
or some other method.

The problem arises with M_3(t): it will not be rational unless o(t)
is a unit vector for all t, or at least the length of o(t) is rational
for all t (which enforces o(t) to remain constant at some rational number,
since rational numbers are sparse in the reals).
Therefore, the Bezier curve o(t) must lie on a sphere.
There are methods for constructing a Bezier curve that interpolates a 
finite collection of points on a sphere [Dietz-Hoschek]: this is equivalent
to being able to specify a finite number of orientations of the circles
on the ringed surface.  However, we would optimally like to have a 
method of controlling the orientation of every circle on the ringed 
surface: otherwise it is impossible to exactly model known surfaces
(such as the cyclide or the quadric) or to exactly define a blending 
surface as a sweeping marble tangent to two surfaces.
Consequently, we would like to have a method for constructing spherical
Bezier curves with more control: by taking any space curve
and projecting it onto the unit sphere, using the origin as center of 
projection.  Undoubtedly the problem is that these images of arbitrary 
curves are not rational and thus cannot be represented as Bezier curves.
It would be interesting to establish the conditions on a space curve that 
guarantee its projection is rational.  (One would suspect that the 
orientation curve of the cyclide satisfies these conditions, since the 
cyclide has an exact representation as a biquadratic ringed surface.)
Note that it is certainly worthwhile to pursue the Dietz-Hoschek channel
as well, since the input to many ringed surfaces will be exactly a finite
number of circles, rather than a continuous stream of circles, in which 
case the Dietz-Hoschek method works wonderfully.

Concise statement of challenge: 
             (1) Determine the projection map of a point 
onto a sphere, with center of projection at the origin.
             (2) Apply this projection map to an entire space curve.
             (3) Establish when this projection is rational, and thus 
which class of space curves can be successfully transformed into 
spherical Bezier curves.
             (4) Check that the cyclide's orientation curve o(t) satisfies 
these criteria.
