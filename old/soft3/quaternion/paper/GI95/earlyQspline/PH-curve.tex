\documentstyle[12pt]{article} 
\input{macros}
\SingleSpace
\setlength{\oddsidemargin}{0pt}
\setlength{\evensidemargin}{0pt}
\setlength{\headsep}{0pt}
\setlength{\topmargin}{0pt}
\setlength{\textheight}{9in}
\setlength{\textwidth}{6.5in}
\setlength{\parindent}{0in}
\setlength{\parskip}{.1in}

\newtheorem{rmk}{Remark}[section]
\newtheorem{dfn}{Definition}[section]
\newtheorem{prop}{Property}[section]
\newtheorem{propn}{Proposition}[section]

% -------------------------------------------------------------

\begin{document}

\section{Plane curve definitions}

\begin{dfn}
\label{def-plane}
[Original definition \cite{FS90}] \\
A {\bf plane polynomial} curve $c(t) = (x(t),y(t))$
is a {\em Pythagorean-hodograph curve}
if the components of its hodograph $c'(t) = (x'(t),y'(t))$
are members of a Pythagorean triple $(x'(t),y'(t),\sigma(t))$,
{\em i.e.}, $x'^2(t) + y'^2(t) = \sigma^2(t)$ for some
polynomial $\sigma(t) \in R[t]$.
\end{dfn}

The following two equivalent definitions of a 
Pythagorean-hodograph curve are more suggestive for our 
applications.

\begin{dfn}
\label{def:alt}
[Alternate definitions]
\begin{description}
\item[(a)]
A polynomial curve $c(t)$
is a {\em Pythagorean-hodograph curve} if
the length $\sigma(t) = \| c'(t) \|$ of its tangent
is a polynomial in $t$.

\item[(b)]
A polynomial curve $c(t)$
is a {\em Pythagorean-hodograph curve} if
the distance $\| c'(t) \|$ of its hodograph from the origin
is a polynomial in $t$.
\end{description}
\end{dfn}

\begin{dfn}
Since the term `Pythagorean-hodograph curve' is unwieldy,
and since we will appeal more to the polynomial tangent-length
and polynomial-distance properties of these curves from
Definition~\ref{def:alt}
than to the Pythagorean triple property of Definition~\ref{def-plane},
we will use the term {\bf Farouki curve} for the curve
of Definitions~\ref{def-plane} and \ref{def:alt}.
This also has the attraction of consistency with other terminology
within CAGD, such as Bezier curves, Coons patches, and Hermite
interpolants.
\end{dfn}

\section{Applications}

Definition~\ref{def:alt}(a) immediately suggests an attraction of these curves:
they can be easily measured.

\begin{propn}
\label{arclength}
Consider the arc length $L(t)$ of a segment $\{c(t):t\in [0,t]\}$ of a Farouki 
curve $c(t)$.
Since $\| c'(t) \|$ is the integrand of the arc length formula:
\[ \label{eq1}
L(t) = \int_0^t \| c'(t) \| dt  = \int_0^t \mbox{length of tangent} 
\]
and $\| c'(t) \|$ is a polynomial in $t$ for Farouki curves,
the arc length $L(t)$ of Farouki curves is also a polynomial in $t$.
Thus, it can be easily computed.
We call this polynomial arc-length property 
the {\em mensuration property} of Farouki curves.
\end{propn}

For general polynomial curves, arc length must be 
computed numerically using quadrature algorithms (e.g., Simpson's
algorithm), which is less efficient and less robust.

%\begin{example}
%\end{example}

\begin{rmk}
\label{mens-destroy}
The mensuration property of Farouki curves (Proposition~\ref{arclength})
is lost even if the restriction $\| c'(t) \| \in R[t]$ 
($\| c'(t) \|$ is a polynomial) is relaxed to 
$\| c'(t) \| \in R(t)$ ($\| c'(t) \|$ is a rational
polynomial):
the integral of a rational polynomial
is not an (integral or rational) polynomial 
({\em e.g.}, $\int \frac{1}{t} = ln |x|$).
\end{rmk}

We shall be interested in exploring Farouki curves
because of another advantageous property, related to sweeping.

\begin{rmk}
Suppose that we sweep a plane object along a curve $c(t)$,
and wish to orient the plane of the object always perpendicular
to the tangent $c'(t)$ of the curve (for example, in our sweeping of a
circle to mimic the sweeping of a sphere).
The parameterization of the resulting swept surface is
\[ X(s,t) = \mbox{Trans}(c(t)) * \mbox{Rot}(z \rightarrow c'(t)) * \mbox{obj}(s)
\]
where Trans$(c(t))$ is the matrix for translation by $c(t)$,
Rot$(z \rightarrow c'(t))$ is the matrix for rotation of the $z$-axis
into $c'(t)$, and obj$(s)$ is a parameterization of the object 
in canonical position (centered at the origin in the $xy$-plane).
We want this parameterization to be rational, so that it is tractable
and can be represented in Bezier form.
If the plane object and $c(t)$ are both rational, then 
$\mbox{Trans}(c(t))$ and obj$(s)$ will be rational.
The challenge lies in making the rotation matrix rational.

Since the $z$-axis is rotated into the third column of the rotation matrix,
and the columns of the rotation matrix are orthonormal,
\begin{equation}
\label{eq:rot}
\mbox{Rot}(z \rightarrow c'(t)) = 
\left( \begin{array}{ccc} a & \frac{a \times c'(t)}{\|a \times c'(t)\|}
			    & \frac{c'(t)}{\|c'(t)\|} \end{array} \right)
\end{equation}
where $a$ is a unit vector orthogonal to $c'(t)$ (say 
$\frac{c'(t) \times A}{\|c'(t) \times A\|}$ for some vector A).
Note: need to choose A appropriately (e.g., to make $x$-axis of object 
plane's coordinate system sweep smoothly through space).

(Note: we must also be concerned about the twist of the object's plane
about its normal $c'(t)$, especially for nonsymmetric objects $obj(s)$.
THIS IS CONTROLLED BY THE CHOICE OF $A$ ABOVE.
It might seem that we could ignore this twist for circles, since
they are symmetric about the origin, but even here the isoparametric
lines will be strange unless we solve this problem, which is undesirable
and inelegant.  Investigate Frenet frames (to control the twist) to
solve this problem.)

If $c(t)$ is a Farouki curve,
then both $c'(t)$ and $\| c'(t) \|$ will be polynomial,
and the rotation matrix will be rational.
IF $\|c'(t)\|$ IS POLYNOMIAL, WE MUST SHOW THAT THIS GUARANTEES
THAT $\|c'(t) \times A\|$ and $\|c'(t) \times a\|$ ARE ALSO POLYNOMIAL
OR RATIONAL.  THIS IS IMMEDIATELY CLEAR.

More generally, if we sweep an arbitrary object along a curve $c(t)$
such that the orientation of the object is controlled by the
Frenet frame of the curve ({\em i.e.}, the local coordinate system 
of the object at position $c(t_0)$ is equivalent to the Frenet frame 
coordinate system of $c(t)$ at $c(t_0)$, defined by tangent, normal,
and binormal),
then I hypothesize that the rotation of the local coordinate system 
into its position
at any point $c(t_0)$ of the sweep will involve rational polynomials.

The property that sweeping a rational curve along a Farouki curve
(with its orientation controlled by the Farouki curve)
yields a rational swept surface will be called the {\em sweeping
property} of Farouki curves.
\end{rmk}

\clearpage

\begin{rmk}
Farouki and Sakkalis \cite{FS90} are also motivated by an offset property:
the offset of a plane polynomial Pythagorean-hodograph curve is
always rational, unlike general plane polynomial curves.
We will not pursue this offset property further.
\end{rmk}

\begin{rmk}
\label{stillrat}
Note that the rotation matrix (\ref{eq:rot}) is still rational if both
$c'(t)$ and $\| c'(t) \|$ are rational.
Therefore, as far as the sweeping property is concerned,
it is possible to relax the definition of Farouki curves 
(to rational $\| c'(t) \|$).
We will explore this extension below in the context of rational curves.
It is meaningless to explore it now for polynomial curves,
since `$c(t)$ is polynomial' implies $\| c'(t) \|^2$ is also polynomial,
so $\| c'(t) \|$ cannot be rational.
\end{rmk}

We wish to sweep objects along space curves, not plane curves,
and so we want to define space Farouki curves.
Our definitions of Farouki curves extend directly to space curves.
The only change is from `Pythagorean triple' to `Pythagorean quadruple'
in Definition~\ref{def-plane}.

% ------------------------------------------

\begin{dfn}
\label{def-space}
A {\bf space polynomial} curve $c(t) = (x(t),y(t),z(t))$
is a {\em Farouki curve} if the following three equivalent conditions hold:
\begin{description}
\item[(a)]
	the components of its hodograph $c'(t) = (x'(t),y'(t),z'(t))$
	are members of a Pythagorean quadruple $(x'(t),y'(t),z'(t),\sigma(t))$,
	{\em i.e.}, $x'^2(t) + y'^2(t) + z'^2(t) = \sigma^2(t)$ for some
	polynomial $\sigma(t) \in R[t]$.
\item[(b)]
	the length $\sigma(t) = \| c'(t) \|$ of its tangent
	is a polynomial in $t$.
\item[(c)]
	the distance $\| c'(t) \|$ of its hodograph from the origin
	is a polynomial in $t$.
\end{description}
\end{dfn}

The mensuration and sweeping properties extend straightforwardly to
space curves.

\begin{rmk}
A spherical curve (a curve that lies on a fixed sphere)
{\em that is also polynomial}
is a special case of a hodograph with the properties of 
Definition~\ref{def-space}(c).
Thus, the integral of such a curve is a space polynomial Farouki curve.
\end{rmk}

The more interesting extension is to rational curves.
We want to define the rational version of the Farouki curve so that it
preserves the mensuration and sweeping properties.

\begin{dfn}
A {\bf plane rational} curve $c(t) = (x(t),y(t),w(t))$
is a Farouki curve if the following three equivalent conditions hold:
\begin{description}
\item[(a)]
	the first two components of its hodograph (in projective space)\\
	$c'(t) = (x'w - xw', y'w - yw', w^2)$
	are members of a Pythagorean triple \\
	$(x'w - xw', y'w - yw', w^2 \sigma)$
	for some polynomial $\sigma(t) \in R[t]$.
\item[(b)]
	the length $\| c'(t) \|$ of its tangent is a polynomial in $t$.
\item[(c)]
	the distance $\| c'(t) \|$ of its hodograph from the origin
	is a polynomial in $t$.
\end{description}
\end{dfn}
\Heading{Proof:}
$\| c'(t) \| = \frac{\sqrt{(x'w - xw')^2 + (y'w - yw')^2}}{w^2}
	     = \frac{\sqrt{(w^2 \sigma)^2}}{w^2} = \sigma$.
\QED

Note that the third member of the Pythagorean triple must contain
a $w^2$ factor.
This preserves the polynomial property of $\|c'(t)\|$,
and thus the mensuration and sweeping properties.

\begin{rmk}
We can guarantee that the third member of the Pythagorean triple 
contains a $w^2$ factor simply by assuring that, in Kubota's scheme,
either ${\cal W}(t)$ contains a $w^2$ factor or both ${\cal U}(t)$
and ${\cal V}(t)$ contain a $w$ factor.
% $\mu,\nu,\omega$.
\end{rmk}

There is another possible extension of the Farouki curve definition 
to rational curves, which we call an r-Farouki curve to distinguish
it and highlight the {\bf r}ational nature of the tangent length
$\| c'(t) \|$.

\begin{dfn}
\label{r-plane-rat}
A {\bf plane rational} curve $c(t) = (x(t),y(t),w(t))$
is an {\bf r-Farouki} curve if the following three equivalent conditions hold:
\begin{description}
\item[(a)]
	the first two components of its hodograph (in projective space)\\
	$c'(t) = (x'w - xw', y'w - yw', w^2)$
	are members of a Pythagorean triple \\
	$(x'w - xw', y'w - yw', \sigma)$
	for some polynomial $\sigma(t) \in R[t]$.
\item[(b)]
	the length $\| c'(t) \|$ of its tangent is a rational polynomial 
	$\frac{\sigma(t)}{w^2(t)}$ for some polynomial $\sigma(t) \in R[t]$.
\item[(c)]
	the distance $\| c'(t) \|$ of its hodograph from the origin
	is a rational polynomial $\frac{\sigma(t)}{w^2(t)}$.
\end{description}
\end{dfn}

This is the definition for rational curves adopted by Farouki and Sakkalis.
It is more natural for property (a) but loses the polynomial property
for (b) and (c).
Therefore, these r-Farouki curves lose the mensuration property (see 
Remark~\ref{mens-destroy} above).
However, the sweeping property is preserved: recall that the rotation matrix
(\ref{eq:rot}) remains rational if both $c'(t)$ and $\|c'(t)\|$ are rational
(Remark~\ref{stillrat}).

It is tempting to extend the definition still further,
and allow the tangent length $\| c'(t) \|$ to be an arbitrary rational
polynomial $\frac{\sigma_1(t)}{\sigma_2(t)}$.
After all, the sweeping property is preserved as long as $\| c'(t) \|$
is rational.
However, this is an equivalent class of curves, as shown by 
the following remark.

\begin{rmk}
If $\| c'(t) \|$ is rational,
then its denominator {\em must} be the square of the rational component
of the curve $c(t)$:
	$\| c'(t) \| = \frac{\sigma_1(t)}{\sigma_2(t)}$
	$\Rightarrow \| c'(t) \|^2 = \frac{\sigma_1^2(t)}{\sigma_2^2(t)}$
	$\Rightarrow (\frac{x(t)}{w(t)})'^2 + (\frac{y(t)}{w(t)})'^2 
				   = \frac{\sigma_1^2(t)}{\sigma_2^2(t)}$
	$\Rightarrow \frac{(x'w - xw')^2 + (y'w - yw')^2}{w^4}
				   = \frac{\sigma_1^2(t)}{\sigma_2^2(t)}$
	$\Rightarrow \sigma_2 = w^2$,
	since $x$, $y$, $w$, $\sigma_1$ and $\sigma_2$ are all polynomial.
\end{rmk}

Thus, the following definition of r-Farouki curve is equivalent
to Definition~\ref{r-plane-rat} (and perhaps more intuitively clean).

\begin{dfn}
\label{r-plane-rat-alt}
[Alternate definition] \\
A plane rational curve $c(t) = (x(t),y(t),w(t))$
is an {\em r-Farouki} curve if the following (equivalent) conditions hold:
\begin{description}
\item[(a)]
	the first two components of its hodograph (in projective space)\\
	$c'(t) = (x'w - xw', y'w - yw', w^2)$
	are members of a Pythagorean triple \\
	$(x'w - xw', y'w - yw', \sigma)$
	for some polynomial $\sigma(t) \in R[t]$.
\item[(b)]
	the length $\| c'(t) \|$ of its tangent is a rational polynomial 
	$\frac{\sigma_1(t)}{\sigma_2(t)}$ for some polynomials
	$\sigma_1(t),\sigma_2(t) \in R[t]$.
\item[(c)]
	the distance $\| c'(t) \|$ of its hodograph from the origin
	is a rational polynomial $\frac{\sigma_1(t)}{\sigma_2(t)}$. 
\end{description}
\end{dfn}

The extension of these definitions to space curves is trivial.

\begin{dfn}
A {\bf space rational} curve $c(t) = (x(t),y(t),z(t),w(t))$
is a Farouki curve if the following three equivalent conditions hold:
\begin{description}
\item[(a)]
	the first three components of its hodograph (in projective space)\\
	$c'(t) = (x'w - xw', y'w - yw', z'w - zw', w^2)$
	are members of a Pythagorean quadruple \\
	$(x'w - xw', y'w - yw', z'w - zw', w^2 \sigma)$
	for some polynomial $\sigma(t) \in R[t]$.
\item[(b)]
	the length $\| c'(t) \|$ of its tangent is a polynomial in $t$.
\item[(c)]
	the distance $\| c'(t) \|$ of its hodograph from the origin
	is a polynomial in $t$.
\end{description}
\end{dfn}

\begin{dfn}
A {\bf space rational} curve $c(t) = (x(t),y(t),z(t),w(t))$
is an {\bf r-Farouki} curve if the following three equivalent conditions hold:
\begin{description}
\item[(a)]
	the first three components of its hodograph (in projective space)\\
	$c'(t) = (x'w - xw', y'w - yw', z'w - zw', w^2)$
	are members of a Pythagorean triple \\
	$(x'w - xw', y'w - yw', z'w - zw', \sigma)$
	for some polynomial $\sigma(t) \in R[t]$.
\item[(b)]
	the length $\| c'(t) \|$ of its tangent 
	(or the distance $\| c'(t) \|$ of its hodograph from the origin)
	is a rational polynomial 
	$\frac{\sigma(t)}{w^2(t)}$ for some polynomial $\sigma(t) \in R[t]$.
\item[(c)]
	the length $\| c'(t) \|$ of its tangent 
	(or the distance $\| c'(t) \|$ of its hodograph from the origin)
	is a rational polynomial 
	$\frac{\sigma_1(t)}{\sigma_2(t)}$ for some polynomials 
	$\sigma_1(t),\sigma_2(t) \in R[t]$.
\end{description}
\end{dfn}

\clearpage

\section{Kubota results}

The following two results on Pythagorean triples and quadruples
by Kubota \cite{} add useful structure to our Farouki curves.

\begin{lemma}
\label{kubota1}
The components of the hodograph of a plane polynomial Farouki
curve must be in the following form:
\begin{eqnarray*}
	x'(t) & = & {\cal W}(t) [{\cal U}^2(t) - {\cal V}^2(t)] \\
	y'(t) & = & 2{\cal W}(t){\cal U}(t){\cal V}(t)
\end{eqnarray*}
where ${\cal U}(t), {\cal V}(t), {\cal W}(t) \in R[t]$.
The resulting third member
of the Pythagorean triple, the tangent length $\|c'(t)\|$, is 
\[
	\| c'(t) \| = {\cal W}(t) [{\cal U}^2(t) + {\cal V}^2(t)]
\]
\end{lemma}

Without loss of generality ({\em i.e.}, by removing Farouki curves
that are points and lines),
we can assume ${\cal U}(t),{\cal V}(t),{\cal W}(t) \not\equiv 0$,
either ${\cal U}(t)$ or ${\cal V}(t)$ is not a constant, and
GCD({\cal U}(t),{\cal V}(t))=1 \cite{FS90}.

\begin{lemma}
The components of the hodograph of a space polynomial Farouki
curve must be in the following form:
\begin{eqnarray*}
	x'(t) & = &   \\
	y'(t) & = &   \\
	z'(t) & = &   \\
\end{eqnarray*}
\end{lemma}


\section{Future}

\begin{rmk}
An expression similar to (\ref{eq1}) 
appears in the formula for the surface area A
of a surface patch $\{X(s,t):s\in [a,b], t \in [c,d] \}$:
\[ A = \int_a^b \int_c^d \| \frac{dX}{ds} \times \frac{dX}{dt} \| ds dt 
     = \int \int \mbox{length of normal} \]
Thus, surfaces $X(s,t)$
whose area is a polynomial in $s$ and $t$ 
might be developed in an analogous fashion to PH-curves.
\end{rmk}

\begin{rmk}
$\ X'(t) \|$ also appears in the curvature formula:
$\mbox{curvature} = \kappa = \frac{\| X' \times X'' \|}{\| X' \|^3}$
	% Kreyszig, p. 35
Moreover, $\| X' \times X'' \| = |x'y'' - x''y'|$,
which is also polynomial.
Thus, it would appear that the curvature of PH-curves is also
easily computable.
\end{rmk}


\section{Appendix: Properties of plane polynomial Farouki curves}

This appendix lists a number of properties of plane 
polynomial Farouki curves, especially cubics.
All of these properties are from Farouki and Sakkalis \cite{FS90}.

\begin{prop} 
\label{prop:f1}
Plane polynomial Farouki curves of degree $n$ have at most
$n+3$ degrees of freedom: {\em $n-2$ affect the shape
of the curve} and 5 only affect the curve through rigid motion
or reparameterization (which leave the shape unchanged).
In contrast, general plane polynomial curves of degree $n$
have at most $2n+2$ degrees of freedom, {\em $2n-3$ of which 
affect the shape}.
\end{prop}

\begin{prop}
\label{prop:f2}
Real irregular points of plane polynomial Farouki curves 
are found exactly at the real roots of $w(t)$ 
(where $w(t)$ is as in Lemma~\ref{kubota1}).
(An irregular point is a point c(t) where $x'(t) = y'(t) = 0$, such as at cusps.)
Therefore, a plane polynomial Farouki curve will be guaranteed
not to have any real irregular points if we choose $w(t) = 1$.
\end{prop}

\subsection{Cubics}

\begin{prop}
\label{prop:f3}
A plane Bezier cubic curve is Farouki if and only if its
control polygon satisfies:
\begin{eqnarray*}
L_2 & = & \sqrt{L_1 L_3} \\
\theta_1 & = & \theta_2
\end{eqnarray*}
where $L_1$, $L_2$, $L_3$ are the lengths of the edges of the control polygon
and $\theta_i = \angle (L_i,L_{i+1})$.
\end{prop}

\begin{prop}
\label{prop:f4}
Plane polynomial Farouki curves have no real inflection points.
\end{prop}

\begin{prop}
\label{prop:f5}
Every plane polynomial cubic Farouki curve has a crunode (a self-intersection),
and the two parameter values associated with this crunode can be calculated
exactly in terms of the polynomials $u(t)$ and $v(t)$ of Lemma~\ref{kubota1}
(and thus it is easy to avoid the crunode on a plane cubic Farouki curve 
by eliminating one of these values from the parameter range).
\end{prop}

\begin{prop}
\label{prop:f6}
Plane cubic Farouki curves are equivalent to Tschirnhausen's (or l'Hopital's)
cubic.
This cubic is also called the trisectrix of Catalan.
\end{prop}

\clearpage

\subsection{Quintics}

\begin{prop} 
(FS90)
\\
\begin{itemize}
\item
``Pythagorean-hodograph cubics cannot interpolate with curvature continuity
discrete data whose `shape' implies inflections.''
\item
``The non-cuspidal quintics are the lowest-order Pythagorean-hodograph
curves that exhibit real inflections.''
\end{itemize}
\end{prop}

\subsection{Hermite interpolation by Pythagorean-hodograph quintics}

Farouki gives formulae for the control points of a PH-quintic
that interpolates two points and two tangents:
the collection (24), (26) and (28) in Section 4 of \cite{PHpractical}.
Given the two points and tangents, there are four choices for the
quintic, one clearly superior to the others.

\bibliographystyle{alpha}
\bibliography{/users/cogito/jj/bib/modeling}

\end{document}