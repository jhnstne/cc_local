\documentstyle[times]{article} 

\newif\ifFull
\Fullfalse

\makeatletter
\def\@maketitle{\newpage
 \null
 \vskip 2em                   % Vertical space above title.
 \begin{center}
       {\Large\bf \@title \par}  % Title set in \Large size. 
       \vskip .5em               % Vertical space after title.
       {\lineskip .5em           %  each author set in a tabular environment
        \begin{tabular}[t]{c}\@author 
        \end{tabular}\par}                   
  \end{center}
 \par
 \vskip .5em}                 % Vertical space after author
\makeatother

% default values are 
% \parskip=0pt plus1pt
% \parindent=20pt

\newcommand{\SingleSpace}{\edef\baselinestretch{0.9}\Large\normalsize}
\newcommand{\DoubleSpace}{\edef\baselinestretch{1.4}\Large\normalsize}
\newcommand{\Comment}[1]{\relax}  % makes a "comment" (not expanded)
\newcommand{\Heading}[1]{\par\noindent{\bf#1}\nobreak}
\newcommand{\Tail}[1]{\nobreak\par\noindent{\bf#1}}
\newcommand{\QED}{\vrule height 1.4ex width 1.0ex depth -.1ex\ \\[.5in]} % square box
\newcommand{\arc}[1]{\mbox{$\stackrel{\frown}{#1}$}}
\newcommand{\lyne}[1]{\mbox{$\stackrel{\leftrightarrow}{#1}$}}
\newcommand{\ray}[1]{\mbox{$\vec{#1}$}}          
\newcommand{\seg}[1]{\mbox{$\overline{#1}$}}
\newcommand{\tab}{\hspace*{.2in}}
\newcommand{\se}{\mbox{$_{\epsilon}$}}  % subscript epsilon
\newcommand{\ie}{\mbox{i.e.}}
\newcommand{\eg}{\mbox{e.\ g.\ }}
\newcommand{\figg}[3]{\begin{figure}[htbp]\vspace{#3}\caption{#2}\label{#1}\end{figure}}
\newcommand{\be}{\begin{equation}}
\newcommand{\ee}{\end{equation}}
\newcommand{\prf}{\noindent{{\bf Proof}:\ \ \ }}
\newcommand{\choice}[2]{\mbox{\footnotesize{$\left( \begin{array}{c} #1 \\ #2 \end{array} \right)$}}}      
\newcommand{\scriptchoice}[2]{\mbox{\scriptsize{$\left( \begin{array}{c} #1 \\ #2 \end{array} \right)$}}}
\newcommand{\tinychoice}[2]{\mbox{\tiny{$\left( \begin{array}{c} #1 \\ #2 \end{array} \right)$}}}
\newcommand{\ddt}{\frac{\partial}{\partial t}}

\newtheorem{rmk}{Remark}[section]
\newtheorem{example}{Example}[section]
\newtheorem{conjecture}{Conjecture}[section]
\newtheorem{claim}{Claim}[section]
\newtheorem{notation}{Notation}[section]
\newtheorem{lemma}{Lemma}[section]
\newtheorem{theorem}{Theorem}[section]
\newtheorem{corollary}{Corollary}[section]
\newtheorem{defn2}{Definition}

% \font\timesr10
% \newfont{\timesroman}{timesr10}
% \timesroman

\setlength{\oddsidemargin}{0pt}
\setlength{\topmargin}{-.25in}	% technically should be 0pt for 1in
\setlength{\headsep}{0pt}
\setlength{\textheight}{8.75in}
\setlength{\textwidth}{6.5in}
\setlength{\columnsep}{5mm}		% width of gutter between columns

\title{Attitude control with rational orientation curves in keyframe animation
        \thanks{The work of both authors was partially supported 
	by the National Science Foundation under grant CCR-9213918.}\\
	(To be submitted to IEEE Computer Graphics and Applications.)}
\author{John K. Johnstone\\
	Department of Computer and Information Sciences\\
	University of Alabama at Birmingham\\
	Birmingham, Alabama, USA 35294\\
	johnstone@cis.uab.edu\\
	Tel. 205-934-2213
	\and 
	James P. Williams\\
	Department of Computer Science\\
	Johns Hopkins University\\
	Baltimore, Maryland, USA 21218\\
	jimbo@cs.jhu.edu\\
	Tel. 410-516-5298}

\begin{document}

\maketitle
\tableofcontents

% \begin{abstract}
% \end{abstract}

\section{Introduction}

Many problem domains, such as computer animation, simulation for CAD/CAM, 
and robot motion planning, require manipulation of the motion of objects.
The motion of a rigid object is fully defined by its change of position
and orientation.
This paper considers the control of the orientation of an object,
a problem that is less understood than the associated problem
of position control.
Since an orientation can be represented by a point (in higher-dimensional
space), an object's orientation during a motion can be represented by
an orientation curve (in higher-dimensional space).
Our problem is the following: given two orientations, what is
a natural and computationally tractable orientation curve between them?
In general, what is a natural and computationally tractable orientation curve
interpolating $n$ orientations?
Thus, the assumed paradigm is that the object's orientation is controlled
by first specifying a small, finite number of orientations during the motion,
from which a full description of the orientation during the motion (an
orientation curve) is generated.
Our measure of the success of this motion generation is the
efficiency of the generation, the computational tractability of the
resulting specification of the orientation, and the naturalness
of the generated motion.

Note that $(1,0,0,0,1)$ is the identity quaternion, representing rotation by
0 degrees about an arbitrary axis.
We have plenty of freedom in our choice of its inverse image.
(three degrees of freedom, in particular).
We can use this freedom to optimize the interpolation in 4-space.
Moreover, one would predict that the identity quaternion would 
be the most frequently used quaternion, representing the object
in its canonical orientation (e.g., a cube aligned wih the coordinate
axes), so we can take advantage of this freedom often.

Coordination of orientation and position maps.

\clearpage

Smooth interpolation of three-dimensional object orientation,
starting from $n$ keyframe orientations, is used in computer animation
to model moving solids, cameras, and lights.
Shoemake clarified the superiority of unit quaternions as the
representation of orientation in this setting \cite{shoemake85},
thus casting the problem as one of interpolation of $n$ points on
the quaternion sphere (the unit sphere in 4-space).
Subsequently, many papers have been written solving the problem of
constructing good interpolating curves on the quaternion sphere
\cite{shoemake85,duff85,gabriel85,pletinckx89,schlag91,barr92},
for orientation interpolation.
However, all of these methods have constructed non-rational curves
(using {\em slerping}, a trigonometric function, and/or
constrained optimization).
They have also lacked strong interactive control over the curve
({\em e.g.}, subdivision, local control, efficient redesign).

This paper shows how to construct
a rational Bezier interpolating curve on the
quaternion sphere, for orientation interpolation.
Since this curve is a true Bezier spline (not an imitation of a Bezier
curve as in Shoemake and others),
it enjoys all of the advantages of Bezier curves, such as efficient
computation, subdivision, local control, affine invariance, 
variation diminution, as well as a predictable behaviour and ease of 
implementation because of the rich understanding of Bezier curves.
Since the curve has a complete analytic description, it allows
simple manipulation and complete control.
This construction answers many of the challenges for future work
outlined by Shoemake in his paper.

Our method does not attempt to design the curve directly on the
quaternion sphere as in other methods (which must apply restrictive
constraints to stay on the sphere).
Instead, the curve is initially designed freely in 4-space
(using traditional interpolation techniques)
and is then mapped to the sphere by a special rational map.


Related work is discussed in Section~\ref{sec:related}.
Section~\ref{sec:quaternion} reviews the theory of quaternions.
Sections~\ref{sec:method}-\ref{sec:cusps} are the heart of the paper:
Section~\ref{sec:method} presents an outline of the new method,
the map onto the quaternion sphere is developed in Section~\ref{sec:onto},
its inverse in Section~\ref{sec:invM}, and the map of a single cubic Bezier
segment onto the sphere in Section~\ref{sec:sextic}.
Ways to control the speed of rotation are presented in Section~\ref{sec:speed},
and Section~\ref{sec:cusps} discusses cusps.
Examples of curves and animations generated by the method 
are presented in Section~\ref{sec:eg},
and we close with some final thoughts in Section~\ref{sec:finito}.

\section{Related work}
\label{sec:related}

Rather than discussing the approach of each of the other papers
on orientation interpolation through curves on the quaternion sphere,
it is enough to discuss a common tool of the methods: {\em slerping}.
{\em Slerping} refers to spherical linear interpolation
\cite{shoemake85}:
$\mbox{Slerp}(q_1,q_2;u) 
:= \frac{\sin (1-u) \theta}{\sin \theta} q_1 + 
   \frac{\sin u \theta}{\sin \theta} q_2$,
where $q_1$ and $q_2$ are unit quaternions and $\theta$
is the angle between these two vectors.
This achieves interpolation along a great arc of the quaternion sphere.
It is clearly a non-rational, trigonometric map.
Various papers have used various spline techniques based around replacement
of linear interpolation by slerping:
Bezier curve (Shoemake \cite{shoemake85}),
B-spline (Duff \cite{duff85}),
cardinal spline (Pletinckx \cite{pletinckx89}),
Catmull-Rom spline (Schlag \cite{schlag91}).

The paper of Barr {\em et.\ al.} \cite{barr92} uses a different technique:
constrained optimization to minimize
tangential acceleration of the spherical curve.
(It also uses slerping for interpolation.)
It is also notable for its excellent motivation of the
design of splines on non-Euclidean curved manifolds.

% Kajiya: involves solving a differential equation

Our paper is strongly motivated by % related to
a paper of Dietz, Hoschek, and J\"{u}ttler
\cite{dietz93} on the construction of interpolating curves on quadrics 
(including the sphere).
Like the present paper, Dietz {\em et.\ al.} map points from the sphere to
3-space, find an interpolating curve in 3-space, and map this curve back
to the sphere.
The major differences arise from the differences between 3-space and
4-space, and our particular attention to the use of the curves in
animation (which lead to our analysis of cusps and speed control).
Their map onto the sphere in 3-space is quite different than our map onto the
sphere in 4-space.
Also, we look at a single point of the map's inverse image rather than the
entire line,
% we look at the map's inverse image 
% in affine space rather than projective space, 
% thus mapping points to points rather than lines,
which allows classical point interpolation methods to be applied in 4-space,
rather than Dietz {\em et.\ al.}'s system of equations approach to the interpolation 
of a curve through lines in 3-space.

\section{Quaternions and the quaternion sphere}
\label{sec:quaternion}

The theory of quaternions is well documented, such as in Shoemake
\cite{shoemake85} which also contains an excellent motivation of their
advantages for representation of orientation.
% (see also Seidel \cite{seidel90}).
The relevant facts about quaternions for this paper are as follows.
%
A quaternion is a 4-vector $(x_1,x_2,x_3,x_4) = x_1 + \mbox{$x_2*i$}~
+ x_3*j +
x_4*k$, a 4-dimensional analogue of complex numbers,\footnote{$i$, $j$,
	and $k$ each act very much like the imaginary number $i$:
	$i^2 = j^2 = k^2 = ijk = -1$.}
invented by Hamilton.
A unit quaternion 
\[ (x_1,x_2,x_3,x_4) 
= (\cos \frac{\theta}{2}, v \sin \frac{\theta}{2}), \ \ \|v\| = 1
\]
corresponds to a rotation of $\theta$ about the axis $v$.\footnote{Two
	quaternions $[s_1,v_1]$ and $[s_2,v_2]$ (where $v_i$ are 3-vectors)
	are multiplied by the formula: 
	$[s_1,v_1] * [s_2,v_2] = [(s_1*s_2 - v_1 \cdot v_2,
		s_1*v_2 + s_2*v_1 + v_1 \times v_2]$.
	Representing a vector $w$ as the quaternion $[0,w]$,
	the result of rotating $w$ by the quaternion $q = [s,v]$ is
	$q^{-1} [0,w] q$, where $q^{-1} = ([s,-v])/(s^2 + v \cdot v)$.
	This is the same result as rotating $w$ about $v$ by an angle 
	$2 cos^{-1}(s)$.}

Since a single rotation about an axis is sufficient
to represent an arbitrary orientation of a solid object,
unit quaternions are a representation for rigid body orientation.
The other primary choices are the rotation matrix and Euler angles.
Quaternions are the most elegant representation, at least for animation.
% for the following reasons.
Unlike Euler angles, quaternions have a unique representation 
for each orientation, do not experience gimbal lock, and can be
combined easily.
Unlike both Euler angles and rotation matrices,
the quaternion has a concise representation (4 numbers) with a natural
geometric analogue (through identification of the set of unit quaternions 
with the unit sphere $S^3$ in 4-space) which is highly useful for interpolation.

A major advantage of the unit quaternion is that we can 
control and predict the speed of rotation of the tumbling body,
since the metrics of the sphere $S^3$ and rotation (the angular metric of
SO(3)) are equivalent.
That is, distance on the sphere is speed of rotation ({\em e.g.},
a constant speed path on the sphere yields a constant speed rotation
of the object).
We will explore this control in Section~\ref{sec:speed}.

% spherical metric of $S^3$ = angular metric of SO(3) (rotation matrices).
% whereas constant speed on line, for e.g., yields speedup of
% rotation in middle: cannot control rotation speed.

In the rest of the paper, a unit quaternion will be identified
with a point on $S^3$, the unit sphere in 4-space, which will be
called the {\em quaternion sphere}.

\section{Curves on the 3-sphere}

\section{Control of speed}
\label{sec:speed}

To control the speed of rotation of the tumbling body,
we would like to control our speed along the spherical curve
(see Section~\ref{sec:quaternion}).
With our rational Bezier spherical curve, we cannot solve
Shoemake's open challenge of designing a spherical curve
parameterized by arc length \cite{shoemake85},
which would generate perfectly regular changes of orientation,
since no rational curve can be parameterized by a rational function
of its arc length (Farouki and Sakkalis \cite{farouki91}).
However, we have a great deal of control over our speed on the
spherical curve: through the knot sequence.
Moreover, this control is very simple and intuitive.

\subsection{Choice of knot sequence}

Since the spherical curve that we create using Lemma~\ref{sextic}
is a Bezier spline, we can use its knot sequence to control
the speed along the curve.
For example, we might like to approximate an arc-length parameterization
(since it is impossible to attain exactly)
for a uniform change of orientation.
Chord-length parameterizations assign knot intervals proportional
to the Euclidean distance between data points.
Since our points and curve lie on a sphere, it is more appropriate
to use a non-Euclidean metric measuring distance on the sphere:
\[
\mbox{dist}(A,B) = \theta r = \theta = cos^{-1}(A \cdot B)
\]
(where $\theta$ is the angle subtended by the points A and B on the unit sphere).
Non-Euclidean variants of other parameterizations, such as centripetal,
could also be used.

In many cases, perfectly regular tumbling will not be our goal,
and manipulation of the knot sequence can just as easily achieve
other effects.
The above non-Euclidean metric will still be useful in these contexts.

This control of the speed along the curve via the knot sequence 
is another benefit of using a Bezier representation of the spherical curve.

\subsection{Choice of frames based on arc length}

Another simple mechanism for effective speed control in the animation
is to choose intermediate frames based on arc-length.
Suppose that, for example, we wish to simulate constant speed along
the orientation curve ({\em i.e.}, arc length parameterization).
Based on computation of arc length, the orientation of intermediate frames
can be chosen at approximately equal spacing along the orientation curve.
The length of a segment should be computed by subdivision
(until an acceptable degree of accuracy is achieved)
rather than by exact computation via the integral 
$\int_a^b \|c'(t)\| dt$ (for the segment $c(t)$).
This method is possible since we have a closed form Bezier representation
of the spherical curve.  Other methods, which do not have
direct representation of the entire curve cannot perform
this type of arc-length based choice of intermediate frames.

\section{Avoidance of cusps}
\label{sec:cusps}

% DOES C^2 CONTINUOUS IMPLY NO CUSPS?
% NO, BECAUSE YOU COULD HAVE COINCIDENT CONTROL POINTS
% IN GENERAL, THERE IS NO CONTRADICTION BETWEEN CUSPS
% AND C^I CONTINUITY SINCE CUSPS SIMPLY MEAN THAT THE
% DERIVATIVE BECOMES ZERO, NOT DISCONTINUOUS.
% THUS, WE CANNOT USE C^I CONTINUITY TO RULE OUT CUSPS.

A cusp in a spherical curve is associated with an abrupt change
of orientation, undesirable in an animation.
All of the spherical curves created by other methods can contain cusps
(or what Shoemake calls `kinks'), and so can ours.
However, the next lemma shows that our spherical curves rarely have cusps.
Since it is simple to design our original 4-space curve without cusps
(a nonplanar cubic Bezier curve cannot contain a cusp \cite{farin93}),
	% proof: the control polygon of the hodograph of a nonplanar cubic Bezier
	% will consist of 3 points, not all coplanar with the origin.
	% So the plane of the conic hodograph does not contain the origin,
	% implying that it does not intersect the origin (since a conic is 
	% always a plane curve).  Thus, the cubic Bezier does not have a cusp,
	% since the condition for a cusp is that the hodograph passes through
	% the origin.
the only cusps are those introduced by the map $M$.
The following lemma shows that $M$ only introduces cusps in unusual
situations, which should be easy to avoid during design
(especially since design is interactive with our method).

\begin{lemma}
M introduces a cusp in the curve $C(t)$ at $t=t_0$
if one of the following conditions holds:
\begin{itemize}
\item
	$C(t_0) = (0,0,0,0)$
\item
	$C'(t_0) = (0,0,0,0)$
\item
	$C(t_0) \cdot C'(t_0) = 0$ and $C_4(t_0) = C'_4(t_0) = 0$
\item
	$\ddt(M_1(C(t_0))) = kM_1(C(t_0))$, where $M_1$ is defined below
	and $k = \frac{\ddt(M_1(C)) \cdot M_1(C)}{M_1(C) \cdot M_1(C)}$.
\end{itemize}
\end{lemma}
\prf
$M$ can be expressed as the composition of two maps (in affine space):
\[
M_1: (p,q,r,s) \rightarrow (p^2 + q^2 + r^2 -s^2 , 2ps,2qs,2rs)
\]
and
\[
M_2: V \rightarrow V/\|V\|
\]
The problem now reduces to determining when these two maps introduce cusps.
We will show that $M_1(C(t_0))$ is a cusp when
$C(t_0)$ or $C'(t_0)$ is the origin or $C(t_0) \cdot C'(t_0) = 0$;
while $M_2(C(t_0))$ is a cusp when $C'(t_0) = kC(t_0),\ k \in \Re$.

Consider the map $M_1$.
A curve $C(t)$ has a cusp at $t=t_0$ if $\ddt(C(t_0)) = (0,0,0,0)$.
% \cite{farin93}.
Let $C(t) = (p(t),q(t),r(t),s(t))$.
Suppose that $M_1(C(t_0))$ is a cusp.
\begin{eqnarray}
\label{eq:d1}
\ddt(M_1(p,q,r,s)) & = & 
\mbox{\footnotesize{$\left( \begin{array}{c}
	2pp' + 2qq' + 2rr'  - 2ss' \\
	2p's + 2ps' \\
	2q's + 2qs' \\
        2r's + 2rs' 
\end{array} \right)$}} \nonumber \\
& = & (0,0,0,0)
\end{eqnarray}
If $s \neq 0$,
$       p' = -p(s'/s)$,
$       q' = -q(s'/s)$,
$       r' = -r(s'/s)$,
and substituting into $2pp' + 2qq' + 2rr'  - 2ss' = 0$ yields
$        \frac{-2s'}{s}(p^2+q^2+r^2+s^2) = 0 $.
If $s'\neq 0$, this reduces to
$ p^2 + q^2 + r^2 + s^2 = 0$
or $(p,q,r,s)$ is the origin.

If $s = 0$ and $s' \neq 0$,
(\ref{eq:d1}) again reduces to $(p,q,r,s) = (0,0,0,0)$.
If $s = s' = 0$,
(\ref{eq:d1}) reduces to $pp' + qq' + rr' = 0$ 
(or $(p,q,r,s) \cdot (p',q',r',s') = 0$).
If $s \neq 0$ and $s' = 0$,
(\ref{eq:d1}) reduces to $(p',q',r',s') = (0,0,0,0)$.

Next consider $M_2$.
Suppose that $M_2(C(t_0))$ is a cusp.
We assume that $C(t_0) \neq (0,0,0,0)$.
\[ 
\ddt(M_2(C(t_0))) 
= \ddt(\frac{C(t_0)}{\|C(t_0)\|})
= \ddt(\frac{C(t_0)}{\sqrt{C(t_0) \cdot C(t_0)}})
\]
\[
= \frac{\|C\| C' - (\frac{C \cdot C'}{\sqrt{C \cdot C}}) C}{C \cdot C}
= (0,0,0,0)
\]
Multiplying by $\|C(t_0)\|^3$,
\[
\|C\|^2 C' - (C \cdot C') C = (0,0,0,0)
\]
\[
C' = kC
\]
where $k = \frac{C \cdot C'}{C \cdot C}$.
In other words,
the map $M_2$ only introduces cusps into the curve $C(t)$ when
$C'(t_0) = kC(t_0)$.

Note that $M_2$ preserves cusps: that is, if $C(t_0)$ is a cusp,
then $M_2(C(t_0))$ is also a cusp.
% (if $C'(t)=0$ then $C'(t) = kC(t)$).
\QED

Thus, $M$ introduces a cusp in $C(t)$
when the curve $C(t)$ passes through
the origin, or its hodograph $C'(t)$ passes through the origin.
$M$ also introduces a cusp if $C(t_0)$ and $C'(t_0)$, 
the vectors to the curve and hodograph {\em at the same parameter value
$t_0$}, are orthogonal as well as lie in the hyperplane $x_4=0$.
Finally, a cusp can be introduced if the vectors to the
curve $M_1(C(t))$ and its tangent, at the same parameter value $t_0$, 
are multiples in the special ratio $k$ of the theorem.
These are all pathological occurences which most curves will not
contain, and moreover they are unstable conditions which can be
removed by manipulation of the original 4-space curve.

\section{Natural physical movement}

\section{Physical optimizations: angular velocity, tangential acceleration,
	spacecraft docking}

Examine how tangent lengths (and thus angular velocity) are changed
in image of cubic Bezier segment: difficult formula for tangent
of spherical curve.

\section{}

Use texture map study to explore how $M$ and $M^{-1}$ preserve or warp
distance between points on sphere and in image space.
The lack of poles of these maps should imply that there are no radical
changes in distance, but limits on the change would be interesting.

\section{Examples}
\label{sec:eg}

The Bezier nature of our spherical curves predicts good quality curves
({\em e.g.}, variation-diminishing).
Moreover, our spherical curve will be $C^2$ continuous, since it is the image
under a rational map of a $C^2$-continuous cubic B-spline.
This quality is supported by our practical experience.
The curves that we have generated are well-behaved.

We present an example of a tumbling maple leaf.
Figure~\ref{fig:input} shows the input to our animation problem:
$n$ orientations of $n$ keyframes of a solid.
The orientations are shown on the left as red unit quaternions
on the quaternion sphere, with the associated keyframes on the right.
Figure~\ref{fig:output} shows the interpolating rational Bezier curve
on the quaternion sphere as determined by our method (on the left)
and the animation corresponding to this spherical curve (on the right).
In this static picture, we only show a few of the intermediate frames
generated by the spherical curve.
The control polygon of the Bezier curve is also drawn, in black.
We visualize the curves in 4-space by using quaternions with $x_3=0$,
thus allowing projection onto the 3-dimensional hyperplane 
$x_3=0$.\footnote{Quaternions with $x_3=0$ are mapped by $M^{-1}$ to points
	in 4-space with $x_2=0$, so the entire interpolation in 4-space
	lies in the plane $x_2=0$ and maps back to a spherical curve
	entirely in the hyperplane $x_3=0$.}
This is purely for purposes of visualization: all computations
are in 4-space.
% a point on the
% quaternion sphere with $x_3=0$ is mapped to a point in 4-space with
% $x_2=0$, thus allowing a projection to 3-space.
% stereographic projection may be worth a try

The construction of the orientation curve is illustrated in
Figure~\ref{fig:method}.
The input quaternions are drawn in red.
They are mapped by $M^{-1}$ to the blue points, which are interpolated
freely in 4-space by the blue Bezier curve (with its control polygon).
Finally, the blue space curve is mapped back onto the sphere by $M$
to the red spherical curve, which interpolates the original quaternions.

Notice that the spherical curve only controls the orientation of the frames.
The position of the tumbling leaf in
each frame is controlled by a second interpolating curve.
It is impossible to visualize the change in orientation unless
the object also moves through 3-space.

Finally, we remind the reader that the representation of an orientation
by a quaternion is not unique: antipodal quaternions represent the same
orientation. 
In constructing the quaternion representation of the
series of keyframe orientations, 
it is wise to use this degree of freedom and 
choose whichever quaternion lies closest (on the sphere)
to the previous quaternion, so that the animation does not perform
undesired flips.
For example, if two consecutive identical orientations are 
represented by two antipodal quaternions, 
the object will perform a full rotation rather than remaining stationary.

% This non-Euclidean chord-length parameterization is used in our examples.

\section{Conclusions}
\label{sec:finito}

We have developed a way of controlling orientation rationally,
by developing rational Bezier curves that interpolate quaternion orientations.
Control of orientation by a rational Bezier spline is more efficient
and more amenable to manipulation ({\em e.g.}, alteration of the curve, control
of speed) than control by a non-rational curve 
for which no direct, closed-form representation is known.

By constructing a rational map from 4-space to the quaternion sphere,
and the inverse map from the sphere to 4-space,
we reduce orientation interpolation on the sphere 
to point interpolation in 4-space.
The Bezier structure of the curve in 4-space is mapped to the sphere.
Point interpolation in 4-space can be performed using any
classical method.
We have chosen a cubic B-spline,
since it is perhaps the most widely understood method,
but other methods could certainly be used.
Since it applies traditional techniques of point interpolation,
our method can be viewed as an extension of the interpolation of position
(in 3-space) to the interpolation of orientation (in 4-space).


% Problems of Shoemake solved by our method:
% (1) trigonometric slerping (Slerp(p,q,u) = $(p:q)_u$) replaced by fully polynomial
% interpolation (added speed, robustness, compatibility);
% (2) can add new points on interpolating curve trivially since curve is Bezier,
% whereas not with Shoemake: "there are simple algorithms for adding new sequence
% points to ordinary splines without altering the original curve; they do not
% work for this interpolant" (p. 251);
% (3) answers Shoemake's query "is there is some related interpolant [to Shoemake's
% spherical `Bezier' curves] that is well-characterized?":
% our curves are true Bezier curves in 4-space not pseudo-Bezier curves;
% (4) Cannot achieve  Shoemake's goal of a spherical curve parameterized by
% arc length (p. 251) using arguments of Farouki, but we will try to control
% the speed by knot sequences (and possibly by PH-curve techniques?)

% Conclusion: an improvement of Shoemake's spherical interpolation technique
% (moreover addressing all of the issues that Shoemake left open for improvement).
% and without resorting to costly and nonpolynomial optimization techniques.

\parindent=-20mm

\bibliographystyle{alpha}
\begin{thebibliography}{Shoemake 85}

\bibitem{barr92}
Barr, A.H., B. Currin, S. Gabriel and J.F. Hughes (1992)
Smooth interpolation of orientations with angular velocity
constraints using quaternions.  SIGGRAPH '92, Chicago, 26(2), 313--320.

\bibitem{dickson52}
Dickson, L.E. (1952) History of the theory of numbers: Volume II,
Diophantine analysis.  Chelsea (New York).

\bibitem{dietz93}
Dietz, R., J. Hoschek and B. J\"{u}ttler (1993)
An algebraic approach to curves and surfaces on the sphere and on other
quadrics.  Computer Aided Geometric Design 10, 211--229.

\bibitem{ebbinghaus90}
Ebbinghaus, H.-D., H. Hermes, F. Hirzebruch, M. Koecher, K. Mainzer,
J. Neukirch, A. Prestel and R. Remmert (1990)
Numbers.
Springer-Verlag (New York).

\bibitem{farin93}
Farin, G. (1993) Curves and surfaces for computer aided geometric design.
Academic Press (New York), third edition.

\bibitem{farouki91}
Farouki, R.T. and T. Sakkalis (1991) Real rational curves are not
"unit speed".  Computer Aided Geometric Design 8, 151--157.

\bibitem{gabriel85}
Gabriel, S. and J. Kajiya (1985) Spline interpolation in curved space.
In SIGGRAPH '85 course notes `State of the art image synthesis'.

\bibitem{kubota72}
Kubota, K.K. (1972) Pythagorean triples in unique factorization domains.
American Mathematical Monthly 79, 503--505.

\bibitem{misner73}	% for equivalence of S^3 and SO(3) metrics (p. 725),
				% more direct quote would be Weyl or Coxeter
			% for term `spin matrix' (1136), 
			% Royal Canal anecdote (1135),
			% discussion of S^3 (unit sphere in 4-space) (723ff)
Misner, C.W., K.S. Thorne and J.A. Wheeler (1973)
Gravitation.  W.H. Freeman (San Francisco).

\bibitem{pletinckx89}
Pletinckx, D. (1989) Quaternion calculus as a basic tool in computer graphics.
The Visual Computer 5, 2--13.

\bibitem{schlag91}
Schlag, J. (1991) Using geometric constructions to interpolate
orientation with quaternions.  In Graphics Gems II, Academic Press (New York),
377--380.

%\bibitem{seidel90}
%Seidel, H.P. (1990) Quaternionen in Computergraphik und Robotik.
%Informationstechnik 32, 266--275.

\bibitem{shoemake85}
Shoemake, K. (1985) Animating rotation with quaternion curves.
SIGGRAPH '85, San Francisco, 19(3), 245--254.

\end{thebibliography}

\clearpage

\begin{figure}
%\special{psfile=/rb/jj/Research/quaternion/figs/curve1.ps}
%  	hscale=75 vscale=75 voffset=-200
\vspace{2.5in}
\special{psfile=/rb/jj/Research/animation/GI-fig1.ps
	 hscale=70 vscale=70}	% voffset=-320
\caption{Input to the animation problem}
\label{fig:input}
% pos.in-evenspeed/quaternion.in5; animation
\end{figure}

\begin{figure*}
%\special{psfile=/rb/jj/Research/quaternion/figs/curve2.ps}
\vspace{2in}
\special{psfile=/rb/jj/Research/animation/GI-fig2.ps
	 hscale=70 vscale=70}	% voffset=-320
\caption{Orientation curve and animation}
\label{fig:output}
% pos.in-evenspeed/quaternion.in5; animation -i -m
\end{figure*}

\begin{figure}
%\special{psfile=/rb/jj/Research/quaternion/figs/curve3.ps}
\vspace{2.5in}
\special{psfile=/rb/jj/Research/animation/GI-fig3.ps
	 hscale=70 vscale=70}	% voffset=-320
\caption{Mapping on and off the quaternion sphere, using $M$}
% Quaternions (red), their inverse images (blue), curve on sphere (red), and
%	space curve (blue), with control polygons}
\label{fig:method}
% pos.in-evenspeed/quaternion.in5; animation -S -m
\end{figure}

\ifFull %%%%%%%%%%%%%%%%%%%%%%%%%%%%%
\section{Appendix}

\begin{lemma}
\label{lem:product}
\begin{equation}
\sum_{i=0}^{3} B_i^3(t) b_i \ \ \sum_{j=0}^{3} B_j^3(t) c_j
= \sum_{k=0}^{6} B_k^6(t) [\sum_{\begin{array}{c} 0 \leq i \leq 3 \\ 
			     0 \leq j \leq 3 \\ 
			     i+j=k
			     \end{array}}
	\frac{\choice{3}{i} * \choice{3}{j}}{\choice{6}{k}}  b_i c_j ]
\end{equation}
for $b_i,c_j \in \Re$.
\end{lemma}
\prf
$\sum_{i=0}^{3} B_i^3(t) b_i \ \ \sum_{j=0}^{3} B_j^3(t) c_j
= \sum_{i=0}^{3} \sum_{j=0}^{3} B_i^3(t) B_j^3(t) b_i c_j
= \sum_{i=0}^{3} \sum_{j=0}^{3} 
	\frac{\choice{3}{i}*\choice{3}{j}}{\choice{6}{i+j}}
	B_{i+j}^6 b_i c_j$
by the product rule of Bernstein polynomials \cite{farin93}.
Now simply let $k=i+j$.
\QED
\fi %%%%%%%%%%%%%%%%%%%%%%%%%%%%%%%%%%%%

% \ifFull
% \subsection{Routines}
% \begin{itemize}
% \item
% 	DISPLAY-N-FRAMES (n,orientation-curve,translation-curve).
% 	Given the interpolation curve for orientation and translation,
% 	choose n t-values (intelligently) to display n well-distributed
% 	frames of the animation.
% \item
%	float ARCLENGTH(Beziercurve,t1,t2)
%	Compute arclength from t1 to t2 on a Bezier segment.
%	Used in DISPLAY-N-FRAMES.
%\end{itemize}
%\fi

\end{document}
