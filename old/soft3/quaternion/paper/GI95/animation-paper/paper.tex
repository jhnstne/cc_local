% header directly from /rb/jj/Research/.cyclide/paper.tex

\documentstyle[12pt]{article} 

\newif\ifFull
\Fulltrue

\makeatletter
\def\@maketitle{\newpage
 \null
 %\vskip 2em                   % Vertical space above title.
 \begin{center}
       {\Large\bf \@title \par}  % Title set in \Large size. 
       \vskip .5em               % Vertical space after title.
       {\lineskip .5em           %  each author set in a tabular environment
        \begin{tabular}[t]{c}\@author 
        \end{tabular}\par}                   
  \end{center}
 \par
 \vskip .5em}                 % Vertical space after author
\makeatother

% non-indented paragraphs with xtra space
% set the indentation to 0, and increase the paragraph spacing:
\parskip=8pt plus1pt                             
\parindent=0pt
% default values are 
% \parskip=0pt plus1pt
% \parindent=20pt
% for plain tex.

\newenvironment{summary}[1]{\if@twocolumn
\section*{#1} \else
\begin{center}
{\bf #1\vspace{-.5em}\vspace{0pt}} 
\end{center}
\quotation
\fi}{\if@twocolumn\else\endquotation\fi}

\renewenvironment{abstract}{\begin{summary}{Abstract}}{\end{summary}}

\newcommand{\SingleSpace}{\edef\baselinestretch{0.9}\Large\normalsize}
\newcommand{\DoubleSpace}{\edef\baselinestretch{1.4}\Large\normalsize}
\newcommand{\Comment}[1]{\relax}  % makes a "comment" (not expanded)
\newcommand{\Heading}[1]{\par\noindent{\bf#1}\nobreak}
\newcommand{\Tail}[1]{\nobreak\par\noindent{\bf#1}}
\newcommand{\QED}{\vrule height 1.4ex width 1.0ex depth -.1ex\ } % square box
\newcommand{\arc}[1]{\mbox{$\stackrel{\frown}{#1}$}}
\newcommand{\lyne}[1]{\mbox{$\stackrel{\leftrightarrow}{#1}$}}
\newcommand{\ray}[1]{\mbox{$\vec{#1}$}}          
\newcommand{\seg}[1]{\mbox{$\overline{#1}$}}
\newcommand{\tab}{\hspace*{.2in}}
\newcommand{\se}{\mbox{$_{\epsilon}$}}  % subscript epsilon
\newcommand{\ie}{\mbox{i.e.}}
\newcommand{\eg}{\mbox{e.\ g.\ }}
\newcommand{\figg}[3]{\begin{figure}[htbp]\vspace{#3}\caption{#2}\label{#1}\end{figure}}
\newcommand{\be}{\begin{equation}}
\newcommand{\ee}{\end{equation}}
\newcommand{\prf}{\noindent{{\bf Proof} :\ }}
\newcommand{\choice}[2]{\left( \begin{array}{c} \mbox{\footnotesize{#1}} \\ \mbox{\footnotesize{#2}} \end{array} \right)}      
\newcommand{\ddt}{\frac{\partial}{\partial t}}

\newtheorem{rmk}{Remark}[section]
\newtheorem{example}{Example}[section]
\newtheorem{conjecture}{Conjecture}[section]
\newtheorem{claim}{Claim}[section]
\newtheorem{notation}{Notation}[section]
\newtheorem{lemma}{Lemma}[section]
\newtheorem{theorem}{Theorem}[section]
\newtheorem{corollary}{Corollary}[section]
\newtheorem{defn2}{Definition}

\ifFull                                          
\SingleSpace
\else
\DoubleSpace
\fi

\setlength{\oddsidemargin}{0pt}
\setlength{\evensidemargin}{0pt}
\setlength{\headsep}{0pt}
\setlength{\topmargin}{0pt}
\setlength{\textheight}{8.75in}
\setlength{\textwidth}{6.5in}

% \input{clock}

\setlength{\headsep}{.2in}
% \setclock              

\title{Rational control of orientation for animation
        \thanks{This work supported by National Science Foundation grant
        CCR-9213918.}}                           
\author{John K. Johnstone\thanks{Dept. of Computer and Information 
	Sciences,
	University of Alabama at Birmingham,
	125 Campbell Hall, 1300 University Boulevard,
	Birmingham, Alabama  35294-1170 USA, johnstone@cis.uab.edu.}
	\and James P. Williams\thanks{Dept. of Computer Science,
	Johns Hopkins University, Baltimore, Maryland 21218 USA.}}
% \date{Version: \today \clock}

\begin{document}

% HEADER ON EACH PAGE
% \markright{\fbox{{\bf Drawing}} \hrulefill 
%          \fbox{{\bf Johnstone, \today}} \hrulefill}
% \pagestyle{myheadings}

\maketitle

% *********************************************************************

\begin{abstract}
The smooth interpolation of keyframes of a rigid body, in particular 
their orientations, is an important problem in animation.
Using the quaternion as a representation for orientation,
several papers have solved this problem through the generation of
smooth curves on the quaternion sphere.
However, none of these methods have constructed rational curves.

This paper develops a method for generating true {\em rational Bezier} curves
on the quaternion sphere that interpolate a given set of orientations.
The control of orientation by a rational Bezier curve has all of the
typical advantages of Bezier curves, such as efficient computation, 
subdivision, and variation diminution.
We also discuss control of the speed of rotation, and cusp avoidance,
both of which are simpler with our method.

This paper can be viewed as an extension of the classical work on 
interpolation of points (i.e., position) to the interpolation of
orientations.

{\bf Note}: A video is under preparation.  If accepted, it will be
	presented at the conference.

% Rather than attempting to directly design a curve constrained to the sphere,
% a curve is designed in 4-space and then mapped to the sphere.

% The disadvantage is also that we map off of the sphere, since 
% it is no longer possible to directly optimize the curve, such as
% by minimizing acceleration (using cubic splines or optimization)
% or by measuring arclength (through design of PH curves).
\end{abstract}

%%%%%%%%%%%%%%%%%%%%%%%%%%%%%%%%%%%%%%%%%%%%%%%%%%%%%%%%%%%%%%%%%
\Comment{
The orientation of a sweeping object can be represented by
a curve on the 4-dimensional quaternion sphere.

We can define the orientation by manipulating orientation curves
directly (e.g., by moving control points) or by specifying orientations
for the orientation curve to interpolate.
The problem with the former is that general orientation curves are
not rational, so we must define a means of defining rational orientation
curves directly.
We therefore use the second method.

(Could we manipulate the orientation curve on the sphere directly
by moving control points and automatically moving neighbouring
control points to maintain the curve as the image of some space curve
under M?)

Example of sweeping object in solid modeling context for which we would
want to define the orientation.
(Animation has the natural key-frame example, which may or may not
transfer to solid modeling applications.)

This work can be interpreted as the development of methods for
the interpolation of orientations rather than the classical
interpolation of points.
Many desirable criteria for point interpolation translate over
to orientation interpolation, such as continuity conditions.
(Challenge: point interpolation is the domain of CAGD, not
solid modeling.  However, swept surfaces are certainly solids.
May want point location and Boolean operations on these swept
solids too.)

We will certainly want translation mechanisms from the quaternion curve
back to at least parametric representations and preferably implicit
representations of the swept surface, which indeed could be the main topic
of the paper.


-------------------------------

The modeling of a sweep requires the representation of the position
of the object and the orientation of the object at all times.
For simple sweeps such as the ruled surface (the sweep of a line),
the orientation can be represented simply by a vector
(the direction of the line at that time).
[The position is represented by a curve c(t) and the orientation can be
represented either by an explicit vector v(t) at each point c(t)
or, as is commonly done (and called lofting), by a second curve d(t) that 
implicitly defines a vector d(t)-c(t) for c(t).]
However, the orientation of an object generally requires a more complex
representation: by a rotation matrix, a set of Euler angles, or a quaternion.
If the object is planar and rotationally symmetric (i.e., a circle),
its orientation can again be represented by a single vector, the normal
of its plane.

Representation of (single) orientation:
	line		vector
	circle		unit vector (plane's normal; 
				     unit so that rotation matrix is rational)
			actually vector of rational length is enough
	plane curve	vector (plane's normal) + angle of rotation = quaternion
				     i.e., non-rotationally plane curves = plane curves - circle)
	space curve	?
	rigid solid	Euler angles; (?) vector and angle of rotation about this vector = quaternion

Thus, full representation of orientation across sweep:

	line		space curve (either representing endpoint of vector from
				origin or from point c(t))
	circle		3-d spherical curve
			actually, hodograph of Pythagorean hodograph curve is enough
	plane curve	4-d quaternion curve


Ruled surface previous work (e.g., my previous work)
Ringed surface/canal surface previous work (e.g, Farouki and Sakkalis)
}
%%%%%%%%%%%%%%%%%%%%%%%%%%%%%%%%%%%%%%%%%%%%%%%%%%%%%%%%%%%%%%%%%%%%%%%
% *********************************************************************

\clearpage

\section{Introduction}

Smooth interpolation of three-dimensional object orientation,
starting from $n$ keyframe orientations, is used in computer animation
to model moving solids, cameras, and lights.
Shoemake clarified the superiority of unit quaternions as the
representation of orientation in this setting \cite{shoemake85},
thus casting the problem as one of interpolation of $n$ points on
the quaternion sphere (the unit sphere in 4-space).
Subsequently, many papers have been written solving the problem of
constructing good interpolating curves on the quaternion sphere
\cite{shoemake85,duff85,gabriel85,pletinckx89,schlag91,barr92},
for orientation interpolation.
However, all of these methods have constructed non-rational curves
(using {\em slerping}, a trigonometric function, and/or
constrained optimization).
They have also lacked strong interactive control over the curve
(e.g., subdivision, local control, efficient redesign).

This paper shows how to construct
a {\em rational Bezier} interpolating curve on the
quaternion sphere, for orientation interpolation.
Since this curve is a true Bezier spline (not an imitation of a Bezier
curve as in Shoemake and others),
it enjoys all of the advantages of Bezier curves, such as efficient
computation, subdivision, local control, affine invariance, 
variation diminution, as well as a predictable behaviour and ease of 
implementation because of the rich understanding of Bezier curves.
Since the curve has a complete analytic description, it allows
simple manipulation and complete control.
This construction answers many of the challenges for future work
outlined by Shoemake in his paper.

Our method does not attempt to design the curve directly on the
quaternion sphere as in other methods (which must apply restrictive
constraints to stay on the sphere).
Instead, the curve is initially designed freely in 4-space
(using traditional interpolation techniques)
and is then mapped to the sphere by a special rational map.


Related work is discussed in Section~\ref{sec:related}.
Section~\ref{sec:quaternion} reviews the theory of quaternions.
Sections~\ref{sec:method}-\ref{sec:cusps} are the heart of the paper:
Section~\ref{sec:method} presents an outline of the new method,
the map onto the quaternion sphere is developed in Section~\ref{sec:onto},
its inverse in Section~\ref{sec:invM}, and the map of a single cubic Bezier
segment onto the sphere in Section~\ref{sec:sextic}.
Ways to control the speed of rotation are presented in Section~\ref{sec:speed},
and Section~\ref{sec:cusps} discusses cusps.
Examples of curves and animations generated by the method 
are presented in Section~\ref{sec:eg},
and we close with some final thoughts in Section~\ref{sec:finito}.

\section{Related work}
\label{sec:related}

Rather than discussing the approach of each of the other papers
on orientation interpolation through curves on the quaternion sphere,
it is enough to discuss a common tool of the methods: {\em slerping}.
{\em Slerping} refers to spherical linear interpolation
\cite{shoemake85}:
$\mbox{Slerp}(q_1,q_2;u) 
:= \frac{\sin (1-u) \theta}{\sin \theta} q_1 + 
   \frac{\sin u \theta}{\sin \theta} q_2$,
where $q_1$ and $q_2$ are unit quaternions and $\theta$
is the angle between these two vectors.
This achieves interpolation along a great arc of the quaternion sphere.
It is clearly a non-rational, trigonometric map.
Various papers have used various spline techniques based around replacement
of linear interpolation by slerping:
Bezier curve (Shoemake \cite{shoemake85}),
B-spline (Duff \cite{duff85}),
cardinal spline (Pletinckx \cite{pletinckx89}),
Catmull-Rom spline (Schlag \cite{schlag91}).

The paper of Barr et. al. \cite{barr92} uses a different technique:
constrained optimization to minimize
tangential acceleration of the spherical curve.
(It also uses slerping for interpolation.)
It is also notable for its excellent motivation of the
design of splines on non-Euclidean curved manifolds.

% Kajiya: involves solving a differential equation

Our paper is strongly motivated by % related to
a paper of Dietz, Hoschek, and J\"{u}ttler
\cite{dietz93} on the construction of interpolating curves on quadrics 
(including the sphere).
Like the present paper, Dietz et. al. map points from the sphere to
3-space, find an interpolating curve in 3-space, and map this curve back
to the sphere.
The major differences arise from the differences between 3-space and
4-space, and our particular attention to the use of the curves in
animation (which lead to our analysis of cusps and speed control).
Their map onto the sphere in 3-space is quite different than our map onto the
sphere in 4-space.
Also, we look at a single point of the map's inverse image rather than the
entire line,
% we look at the map's inverse image 
% in affine space rather than projective space, 
% thus mapping points to points rather than lines,
which allows classical point interpolation methods to be applied in 4-space,
rather than Dietz et. al.'s system of equations approach to the interpolation 
of a curve through lines in 3-space.

\section{Quaternions and the quaternion sphere}
\label{sec:quaternion}

The theory of quaternions is well documented, such as in Shoemake
\cite{shoemake85} which also contains an excellent motivation of their
advantages for representation of orientation (see also Seidel
\cite{seidel90}).
The relevant facts about quaternions for this paper are as follows.
%
A quaternion is a 4-vector $(x_1,x_2,x_3,x_4) = x_1 + x_2*i + x_3*j +
x_4*k$, a 4-dimensional analogue of complex numbers,\footnote{$i$, $j$,
	and $k$ each act very much like the imaginary number $i$:
	$i^2 = j^2 = k^2 = ijk = -1$.}
invented by Hamilton.
A unit quaternion 
\[ (x_1,x_2,x_3,x_4) 
= (\cos \frac{\theta}{2}, v \sin \frac{\theta}{2}), \ \ \|v\| = 1
\]
corresponds to a rotation of $\theta$ about the axis $v$.\footnote{Two
	quaternions $[s_1,v_1]$ and $[s_2,v_2]$ (where $v_i$ are 3-vectors)
	are multiplied by the formula: 
	$[s_1,v_1] * [s_2,v_2] = [(s_1*s_2 - v_1 \cdot v_2,
		s_1*v_2 + s_2*v_1 + v_1 \times v_2]$.
	Representing a vector $w$ as the quaternion $[0,w]$,
	the result of rotating $w$ by the quaternion $q = [s,v]$ is
	$q^{-1} [0,w] q$, where $q^{-1} = ([s,-v])/(s^2 + v \cdot v)$.
	This is the same result as rotating $w$ about $v$ by an angle 
	$2 cos^{-1}(s)$.}

Since a single rotation about an axis is sufficient
to represent an arbitrary orientation of a solid object,
unit quaternions are a representation for rigid body orientation.
The other primary choices are the rotation matrix and Euler angles.
Quaternions are the most elegant representation, at least for animation.
% for the following reasons.
Unlike Euler angles, quaternions have a unique representation 
for each orientation, do not experience gimbal lock, and can be
combined easily.
Unlike both Euler angles and rotation matrices,
the quaternion has a concise representation (4 numbers) with a natural
geometric analogue (through identification of the set of unit quaternions 
with the unit sphere $S^3$ in 4-space) which is highly useful for interpolation.

A major advantage of the unit quaternion is that we can 
control and predict the speed of rotation of the tumbling body,
since the metrics of the sphere $S^3$ and rotation (the angular metric of
SO(3)) are equivalent.
That is, distance on the sphere is speed of rotation (e.g.,
a constant speed path on the sphere yields a constant speed rotation
of the object).
We will explore this control in Section~\ref{sec:speed}.

% spherical metric of $S^3$ = angular metric of SO(3) (rotation matrices).
% whereas constant speed on line, for e.g., yields speedup of
% rotation in middle: cannot control rotation speed.

In the rest of the paper, a unit quaternion will be identified
with a point on $S^3$, the unit sphere in 4-space, which will be
called the {\em quaternion sphere}.

\section{Our method of orientation interpolation}
\label{sec:method}

We now present an outline of our method.
The major idea is to design the interpolating curve freely in 4-space
using traditional techniques and then map back onto the sphere,
(using a map $M$ from 4-space onto the sphere).
The input is a set of $n$ orientations of a solid
represented as unit quaternions $q_1,\ldots,q_n$.

\begin{enumerate}
\item
	If necessary, translate the orientations to unit quaternions.
\item
	({\bf Map quaternions off of the sphere})
	Map the quaternions $q_i$ by $M^{-1}$,
	off of the sphere into 4-space.
\item
	({\bf Interpolate in 4-space})
	Interpolate the points $M^{-1}(q_i)$ in 4-space
	by a polynomial curve $C(t)$, for example a cubic B-spline.
\item
	({\bf Translate to Bezier spline})
	Translate $C(t)$ to the equivalent cubic Bezier spline 
	$C_{\mbox{bez}}(t)$.
\item
	({\bf Map back onto the sphere})
	Map $C_{\mbox{bez}}(t)$ back to the sphere using $M$, 
	one Bezier segment at a time, yielding a  
	Bezier spline $D_{\mbox{bez}}(t)$. 
\end{enumerate}

$D_{\mbox{bez}}(t)$ is the desired spherical curve, a Bezier spline
that interpolates the quaternions $q_i$.
In our examples, since we use a cubic B-spline in step (3),
$D_{\mbox{bez}}(t)$ is $C_2$-continuous.
Traditional techniques are used for steps (3) and (4),
the interpolation in 4-space
and the translation from polynomial curve 
to Bezier spline (see Farin \cite{farin93}).
Step 1 is also well understood (see Shoemake \cite{shoemake85}).
Steps 2 and 5 are the only steps that require elaboration.
Step 2 is discussed in Section~\ref{sec:invM} and 
step 5 in Section~\ref{sec:sextic}.

Notice that any method of interpolation can be used in step (3),
as long as it can be translated to a Bezier spline.
We need to translate to a Bezier spline so that 
the method of step (5) can be applied.

\section{Onto the sphere}
\label{sec:onto}

We want a rational\footnote{A map $\alpha(t) = (x_1(t),\ldots,x_k(t))$
	is rational if each component $x_i(t)$ can be expressed
	as the quotient of two polynomials.}
map from 4-space onto the unit sphere $S^3$ in
4-space.
The challenge is to make the map rational,
since it is simple to find non-rational maps onto the sphere 
(e.g., Gauss map, normalization).
% Gauss map is not rational since taking unit tangent (to map to unit sphere)
% involves a square root.
A formula from number theory yields a solution
\cite{dickson52}[p. 318].
%\footnote{This is a special case of
%	Euler's famous formula, % p. 277, dickson52
%	as well as a special case of a formula of Aida.}

\begin{lemma}[Euler, Aida]
\begin{equation}
\label{eqn:aida}
(a^2 + b^2 + c^2 - d^2)^2 + (2ad)^2 + (2bd)^2 + (2cd)^2 = 
(a^2 + b^2 + c^2 + d^2)^2
\end{equation}
\end{lemma}

\ifFull
\begin{rmk}
This is a special form of Euler's formula \cite{Dickson52}[p. 277],
$(a^2 + b^2 + c^2 + d^2)(p^2 + q^2 + r^2 + s^2) = 
  (ap+bq+cr+ds)^2 + 
  (aq-bp \pm cs \mp dr)^2 + 
  (ar \mp bs - cp \pm dq)^2 + 
  (as \pm br \mp cq - dp)^2$.
 Simply let $p=-a, q=b, r=c, s=d$.
\end{rmk}
\fi

\begin{notation}
It is most natural to express the following map in projective 4-space, $P^4$.
$(x_1,x_2,x_3,x_4,x_5)$ in projective 4-space is equivalent to
$(\frac{x_1}{x_5},\frac{x_2}{x_5},\frac{x_3}{x_5},\frac{x_4}{x_5})$ 
in affine 4-space.
\end{notation}

\begin{corollary}
The map $M:P^4 \rightarrow S^3 \subset P^4$:
\begin{equation}
\label{eq:M}
	M(x_1,x_2,x_3,x_4,1) = \left( \begin{array}{c}
		x_1^2+x_2^2+x_3^2-x_4^2 \\
		2x_1x_4 \\
		2x_2x_4 \\
		2x_3x_4 \\
		x_1^2+x_2^2+x_3^2+x_4^2
		\end{array} \right)
\end{equation}
sends any point in projective 4-space onto the unit sphere in projective
4-space.
That is, $\| M(x_1,x_2,x_3,x_4,1) \| = 1$.
\end{corollary}

\begin{rmk}
To get a map from 4-space to the unit sphere,
we are looking for a formula of the form $A^2+B^2+C^2+D^2=E^2$
% (a Pythagorean 5-tuple)
where $A$, $B$, $C$, $D$ and $E$ are functions of at most four variables
and at least one is a function of exactly four variables.
Then a point in 4-space can be
mapped to $(A,B,C,D,E)$ where $\|(A,B,C,D,E)\| = 1$.
% (\ref{eqn:aida}) is the only one we have been able to find.
\end{rmk}

% Every positive integer is the sum of four squares,
% and coincidentally, this fact can be proved using quaternions (Hardy,
% An introduction to the theory of numbers, Oxford (1960), 4th edition).

The analogous map in 3-space is Lebesgue's
$ {\cal M}(x_1,x_2,x_3,x_4) = 
(2(x_1x_2-x_3x_4),2(x_2x_4-x_1x_3),
x_2^2 + x_3^2 - x_1^2 - x_4^2,\pm(x_1^2 + x_2^2 + x_3^2 + x_4^2))$
\cite{dickson52,dietz93}, which is used by Dietz et. al. 
\cite{dietz93}.\footnote{These maps onto the sphere are 
	instances of Pythagorean n-tuples.  An interesting work
	on the use of Pythagorean triples for curve design
	is the work of Farouki and Sakkalis on Pythagorean hodograph curves
	\cite{farouki90}.}
This is an elegant map since it is necessary and sufficient: 
every rational curve $c(t)$
on the unit sphere in 3-space has the form
${\cal M}(x_1(t),x_2(t),x_3(t),x_4(t))$,
for some choice of $x_1(t)$, $x_2(t)$, $x_3(t)$, $x_4(t)$.
We are not as lucky in 4-space with our map $M$.
% : we do not know that
% every rational spherical curve has the form
% $(x_1^2+x_2^2+x_3^2+x_4^2,x_1^2+x_2^2+x_3^2-x_4^2,2x_1x_4,2x_2x_4,2x_3x_4)$.
%% In other words, for our problem, 
%% the image of the class of all rational interpolating curves
%% of $M^{-1}(\{q_i\})$ (where $q_i$ are quaternions)
%% will not cover the class of all rational interpolating curves of $\{q_i\}$
%% on the sphere.
But the fact that $M$ does not map onto the set of all rational 
interpolating curves is not important: we need only insure that it
creates a good interpolating curve for animation.

% (Note: every sum of three squares is a sum of four squares:
% see Dickson, p. 300.)

\section{Back to 4-space}
\label{sec:invM}

The inverse map $M^{-1}$ is needed
to map the quaternions $q_i$ to $M^{-1}(q_i)$ as input to the
interpolation in 4-space.
If we find a curve interpolating $M^{-1}(q_i)$,
then its image under $M$ will be a curve interpolating $q_i$.

\begin{lemma}
\label{lem:invM}
The map $M^{-1}:S^3 \rightarrow P^4$ is defined by
\begin{equation}
\label{eq:invM}
M^{-1}(x_1,x_2,x_3,x_4,x_5)=
\left\{ \begin{array}{ll}
(x_2,x_3,x_4,x_5-x_1,+2\sqrt{\frac{x_5-x_1}{2}}) 
	& \mbox{if } (x_1,x_2,x_3,x_4,x_5) \neq (1,0,0,0,1) \\
	& \mbox{(equivalently, if } x_1 \neq x_5 \mbox{)} \\
\mbox{the hyperplane } x_4 = 0 
	& \mbox{otherwise}
\end{array} \right.
\end{equation}
where $(x_1,x_2,x_3,x_4,x_5)$ lies on the unit sphere $S^3$
and only the positive square root is used.
\end{lemma}
\prf
\ifFull
Let $X = M(1,p,q,r,s)$, $X := (x_0,x_1,x_2,x_3,x_4)$.
Then $M^{-1}(X) = (1,p,q,r,s)$.
Suppose $s \neq 0$.
From $X = (p^2+q^2+r^2+s^2, p^2+q^2+r^2-s^2, 2ps, 2qs, 2rs)$,
$p = \frac{x_2}{2s}$, $q = \frac{x_3}{2s}$, $r = \frac{x_4}{2s}$.
Substituting into $x_0$ and $x_1$, we get 
$ 4s^4 - (4x_0)s^2 + (x_2^2 + x_3^2 + x_4^2) = 0
4s^4 + (4x_1)s^2 - (x_2^2 + x_3^2 + x_4^2)$,
or $4s^2(2s^2+ x_1 - x_0) = 0$.
Thus, $s=0$ or $s = \pm \sqrt{\frac{x_0 -x_1}{2}}$.
COMPLETE THIS PROOF.  JUST ONE MORE SENTENCE.
\fi
If $x_1 \neq x_5$,
$M^{-1}(M(p,q,r,s,1)) = 
 M^{-1}(p^2+q^2+r^2-s^2,2ps,2qs,2rs,p^2+q^2+r^2+s^2) = 
 (2ps,2qs,2rs,2s^2,2s) = 
 (p,q,r,s,1)$.

Since $M(x_1,x_2,x_3,0,x_5) = (1,0,0,0,1)$, the inverse image
of $(1,0,0,0,1)$ is the entire hyperplane $x_4=0$.
\QED

\begin{corollary}
M is surjective.
\end{corollary}
\prf
Every point of the unit sphere $S^3$ has an inverse image.
\QED

Note that $(1,0,0,0,1)$ is the identity quaternion, representing rotation by
0 degrees about an arbitrary axis.
% We have plenty of freedom in our choice of its inverse image.
% (three degrees of freedom, in particular).
% We can use this freedom to optimize the interpolation in 4-space.
% Moreover, one would predict that the identity quaternion would 
% be the most frequently used quaternion, representing the object
% in its canonical orientation (e.g., a cube aligned wih the coordinate
% axes), so we can take advantage of this freedom often.


When the quaternion $q_i = (q_{i1},q_{i2},q_{i3},q_{i4})$ is mapped off
of the sphere using $M^{-1}$,
we apply the map $M^{-1}(q_{i1},q_{i2},q_{i3},q_{i4},1)$.
That is, we simply set the homogeneous coordinate to 1.
This maps the quaternion to a unique point.
The inverse of the projective point 
$\{(kq_{i1},kq_{i2},kq_{i3},kq_{i4},k)\}_{k \in \Re}$
is a line,\footnote{$M^{-1}(kq_{i1},kq_{i2},kq_{i3},kq_{i4},k) = 
	(kq_{i2},kq_{i3},kq_{i4},k(1-q_{i1}),2\sqrt{\frac{1-q_{i1}}{2}} \sqrt{k})$, 
	or in affine space \\
	$\frac{\sqrt{k}}{\sqrt{2(1-q_{i1})}} (q_{i2},q_{i3},q_{i4},1-q_{i1})$,
	so $M^{-1}\{(kq_{i1},kq_{i2},kq_{i3},kq_{i4},k)\}_{k \in \Re}$
	is the line from the origin through $(q_{i2},q_{i3},q_{i4},1-q_{i1})$ 
	in affine space.}
but we choose a unique preimage in order to apply point interpolation.

\section{The image of a cubic Bezier segment}
\label{sec:sextic}

To take advantage of the Bezier representation,
the Bezier spline in 4-space must be mapped back to the sphere
as a Bezier curve, not simply a rational curve.
It turns out that the image of each cubic Bezier segment of the spline
is a sextic Bezier segment.
The following lemma reveals the Bezier structure of these sextic
image segments on the sphere.
Notice that the structure of the map $M$ is preserved in the control
points (compare (\ref{eq:control-pts}) and (\ref{eq:M})).

\begin{lemma}
\label{sextic}
The image of a polynomial cubic Bezier segment $c(t)$ in 4-space 
under the map $M$
is a rational Bezier segment of degree 6 with control points $c_k$
and weights $w_k$ ($k = 0, \ldots, 6$):
\begin{equation}
\label{eq:control-pts}
c_k = \sum_{\begin{array}{c} 0 \leq i \leq 3 \\ 
			     0 \leq j \leq 3 \\ 
			     i+j=k
			     \end{array}} 
        \frac{\choice{3}{i} * \choice{3}{j}}{\choice{6}{k}}
	\left( \begin{array}{c}
            (b_{i1} b_{j1} + b_{i2} b_{j2} + b_{i3} b_{j3} - b_{i4} b_{j4}) / w_k \\
            2b_{i1} b_{j4} / w_k \\
            2b_{i2} b_{j4} / w_k \\
            2b_{i3} b_{j4} / w_k
	\end{array} \right)
\end{equation}
\begin{equation}
w_k = \sum_{\begin{array}{c} 0 \leq i \leq 3 \\ 
			     0 \leq j \leq 3 \\ 
			     i+j=k
			     \end{array}}
        \frac{\choice{3}{i} * \choice{3}{j}}{\choice{6}{k}}
	(b_{i1} b_{j1} + b_{i2} b_{j2} + b_{i3} b_{j3} + b_{i4} b_{j4})
\end{equation}
where $b_i = (b_{i1},b_{i2},b_{i3},b_{i4})$ are the control points of $c(t)$
($i=0,1,2,3$).
\end{lemma}
\prf
Let $M(c(t)) = M(\sum_{i=0}^3 B_i^3(t) b_{i}) 
:= (m_1(t),m_2(t),m_3(t),m_4(t),m_5(t))$ 
(where $B_i^n(t)$ is the $i^{\mbox{th}}$ Bernstein polynomial of degree
$n$).
Consider the coordinate 
\[ m_5(t) =  [\sum_{i=0}^3 B_i^3(t) b_{i1}]^2 + 
	[\sum_{i=0}^3 B_i^3(t) b_{i2}]^2 +
	[\sum_{i=0}^3 B_i^3(t) b_{i3}]^2 +
	[\sum_{i=0}^3 B_i^3(t) b_{i4}]^2
\]
\[     =   \sum_{i=0}^3 \sum_{j=0}^3 
	\frac{\choice{3}{i} * \choice{3}{j}}{\choice{6}{i+j}}
       B^6_{i+j} (b_{i1} b_{j1} + b_{i2} b_{j2} + b_{i3} b_{j3} + b_{i4} b_{j4})
\]
by the product rule of Bernstein polynomials \cite{farin93}.
Letting $k=i+j$, 
\[ m_5(t) =   \sum_{k=0}^6 B_k^6(t) 
	\sum_{\begin{array}{c} 0 \leq i \leq 3 \\ 
			     0 \leq j \leq 3 \\ 
			     i+j=k
			     \end{array}} 
        \frac{\choice{3}{i} * \choice{3}{j}}{\choice{6}{k}} 
	(b_{i1} b_{j1} + b_{i2} b_{j2} + b_{i3} b_{j3} + b_{i4} b_{j4}) \]
Computing the other coordinates analogously yields
\[ M(c(t)) = \sum_{k=0}^6 B_k^6(t) 
	\sum_{\begin{array}{c} 0 \leq i \leq 3 \\ 
			     0 \leq j \leq 3 \\ 
			     i+j=k
			     \end{array}} 
        \frac{\choice{3}{i} * \choice{3}{j}}{\choice{6}{k}}
	\left( \begin{array}{c}
            b_{i1} b_{j1} + b_{i2} b_{j2} + b_{i3} b_{j3} - b_{i4} b_{j4} \\
            2b_{i1} b_{j4} \\
            2b_{i2} b_{j4} \\
            2b_{i3} b_{j4} \\
            b_{i1} b_{j1} + b_{i2} b_{j2} + b_{i3} b_{j3} + b_{i4} b_{j4}
	\end{array} \right) \]
% \[         = \sum_{k=0}^6 B_k^6(t) 
%	\sum_{\begin{array}{c} 0 \leq i \leq 3 \\ 
%			     0 \leq j \leq 3 \\ 
%			     i+j=k
%			     \end{array}} 
%       \frac{\choice{3}{i} * \choice{3}{j}}{\choice{6}{k}}
%	\left( \begin{array}{c}
%           b_{i1} b_{j1} + b_{i2} b_{j2} + b_{i3} b_{j3} + b_{i4} b_{j4} \\
%            w_k (\frac{b_{i1} b_{j1} + b_{i2} b_{j2} + b_{i3} b_{j3} - b_{i4} b_{j4}}{w_k}) \\
%            w_k (\frac{2b_{i1} b_{j4}}{w_k}) \\
%            w_k (\frac{2b_{i2} b_{j4}}{w_k}) \\
%            w_k (\frac{2b_{i3} b_{j4}}{w_k})
%	\end{array} \right) \]
%
which is a sextic rational Bezier curve with 
control points (\ref{eq:control-pts}) and weights $w_k$.
\QED

Figures~\ref{fig:curve1}-\ref{fig:curve3} show the control polygons of 
the rational Bezier curves on the sphere.  
Notice that the control polygon does not in general
lie on the sphere, only the associated Bezier curve.

\section{Control of speed}
\label{sec:speed}

To control the speed of rotation of the tumbling body,
we would like to control our speed along the spherical curve
(see Section~\ref{sec:quaternion}).
With our rational Bezier spherical curve, we cannot solve
Shoemake's open challenge of designing a spherical curve
parameterized by arc length \cite{shoemake85},
which would generate perfectly regular changes of orientation,
since no rational curve can be parameterized by a rational function
of its arc length (Farouki and Sakkalis \cite{farouki91}).
However, we have a great deal of control over our speed on the
spherical curve: through the knot sequence.
Moreover, this control is very simple and intuitive.

\subsection{Choice of knot sequence}

Since the spherical curve that we create using Lemma~\ref{sextic}
is a Bezier spline, we can use its knot sequence to control
the speed along the curve.
For example, we might like to approximate an arc-length parameterization
(since it is impossible to attain exactly)
for a uniform change of orientation.
Chord-length parameterizations assign knot intervals proportional
to the Euclidean distance between data points.
Since our points and curve lie on a sphere, it is more appropriate
to use a non-Euclidean metric measuring distance on the sphere:
\[
\mbox{dist}(A,B) = \theta r = \theta = cos^{-1}(A \cdot B)
\]
(where $\theta$ is the angle subtended by the points A and B on the unit sphere).
Non-Euclidean variants of other parameterizations, such as centripetal,
could also be used.

In many cases, perfectly regular tumbling will not be our goal,
and manipulation of the knot sequence can just as easily achieve
other effects.
The above non-Euclidean metric will still be useful in these contexts.

This control of the speed along the curve via the knot sequence 
is another benefit of using a Bezier representation of the spherical curve.

\subsection{Choice of frames based on arc length}

Another simple mechanism for effective speed control in the animation
is to choose intermediate frames based on arc-length.
Suppose that, for example, we wish to simulate constant speed along
the orientation curve (i.e., arc length parameterization).
Based on computation of arc length, the orientation of intermediate frames
can be chosen at approximately equal spacing along the orientation curve.
The length of a segment should be computed by subdivision
(until an acceptable degree of accuracy is achieved)
rather than by exact computation via the integral 
$\int_a^b \|c'(t)\| dt$ (for the segment $c(t)$).
This method is possible since we have a closed form Bezier representation
of the spherical curve.  Other methods, which do not have
direct representation of the entire curve cannot perform
this type of arc-length based choice of intermediate frames.

\section{Cusps}
\label{sec:cusps}

% DOES C^2 CONTINUOUS IMPLY NO CUSPS?
% NO, BECAUSE YOU COULD HAVE COINCIDENT CONTROL POINTS
% IN GENERAL, THERE IS NO CONTRADICTION BETWEEN CUSPS
% AND C^I CONTINUITY SINCE CUSPS SIMPLY MEAN THAT THE
% DERIVATIVE BECOMES ZERO, NOT DISCONTINUOUS.
% THUS, WE CANNOT USE C^I CONTINUITY TO RULE OUT CUSPS.

A cusp in a spherical curve is associated with an abrupt change
of orientation, undesirable in an animation.
All of the spherical curves created by other methods can contain cusps
(or what Shoemake calls `kinks'), and so can ours.
However, the next lemma shows that our spherical curves rarely have cusps.
Since it is simple to design our original 4-space curve without cusps
(a nonplanar cubic Bezier curve cannot contain a cusp \cite{farin93}),
	% proof: the control polygon of the hodograph of a nonplanar cubic Bezier
	% will consist of 3 points, not all coplanar with the origin.
	% So the plane of the conic hodograph does not contain the origin,
	% implying that it does not intersect the origin (since a conic is 
	% always a plane curve).  Thus, the cubic Bezier does not have a cusp,
	% since the condition for a cusp is that the hodograph passes through
	% the origin.
the only cusps are those introduced by the map $M$.
The following lemma shows that $M$ only introduces cusps in unusual
situations, which should be easy to avoid during design
(especially since design is interactive with our method).

\begin{lemma}
M introduces a cusp in the curve $C(t)$ at $t=t_0$
if one of the following conditions holds:
\begin{itemize}
\item
	$C(t_0) = (0,0,0,0)$
\item
	$C'(t_0) = (0,0,0,0)$
\item
	$C(t_0) \cdot C'(t_0) = 0$ and $C_4(t_0) = C'_4(t_0) = 0$
\item
	$\ddt(M_1(C(t_0))) = kM_1(C(t_0))$, where $M_1$ is defined below
	and $k = \frac{C \cdot C'}{C \cdot C}$.
\end{itemize}
\end{lemma}
\prf
M can be expressed as the composition of two maps (in affine space):
\[
M_1: (p,q,r,s) \rightarrow (p^2 + q^2 + r^2 -s^2 , 2ps,2qs,2rs)
\]
and
\[
M_2: V \rightarrow V/\|V\|
\]
The problem now reduces to determining when these two maps introduce cusps.
We will show that $M_1(C(t_0))$ is a cusp when
$C(t_0)$ or $C'(t_0)$ is the origin or $C(t_0) \cdot C'(t_0) = 0$;
while $M_2(C(t_0))$ is a cusp when $C'(t_0) = kC(t_0),\ k \in \Re$.

Consider the map $M_1$.
A curve $C(t)$ has a cusp at $t=t_0$ if $\ddt(C(t_0)) = (0,0,0,0)$.
% \cite{farin93}.
Let $C(t) = (p(t),q(t),r(t),s(t))$.
Suppose that $M_1(C(t_0))$ is a cusp.
\begin{equation}
\label{eq:d1}
\ddt(M_1(p,q,r,s)) = 
\left( \begin{array}{c}
	2pp' + 2qq' + 2rr'  - 2ss' \\
	2p's + 2ps' \\
	2q's + 2qs' \\
        2r's + 2rs' 
\end{array} \right)
	   = (0,0,0,0)
\end{equation}
If $s \neq 0$,
$       p' = -p(s'/s)$,
$       q' = -q(s'/s)$,
$       r' = -r(s'/s)$,
and substituting into $2pp' + 2qq' + 2rr'  - 2ss' = 0$ yields
$        \frac{-2s'}{s}(p^2+q^2+r^2+s^2) = 0 $.
If $s'\neq 0$, this reduces to
$ p^2 + q^2 + r^2 + s^2 = 0$
or $(p,q,r,s)$ is the origin.

If $s = 0$ and $s' \neq 0$,
(\ref{eq:d1}) again reduces to $(p,q,r,s) = (0,0,0,0)$.
If $s = s' = 0$,
(\ref{eq:d1}) reduces to $pp' + qq' + rr' = 0$ 
(or $(p,q,r,s) \cdot (p',q',r',s') = 0$).
If $s \neq 0$ and $s' = 0$,
(\ref{eq:d1}) reduces to $(p',q',r',s') = (0,0,0,0)$.

Next consider $M_2$.
Suppose that $M_2(C(t_0))$ is a cusp.
We assume that $C(t_0) \neq (0,0,0,0)$.
\[ 
\ddt(M_2(C(t_0))) 
= \ddt(\frac{C(t_0)}{\|C(t_0)\|})
= \ddt(\frac{C(t_0)}{\sqrt{C(t_0) \cdot C(t_0)}})
\]
\[
= \frac{\|C\| C' - (\frac{C \cdot C'}{\sqrt{C \cdot C}}) C}{C \cdot C}
= (0,0,0,0)
\]
Multiplying by $\|C(t_0)\|^3$,
\[
\|C\|^2 C' - (C \cdot C') C = (0,0,0,0)
\]
\[
C' = kC
\]
where $k = \frac{C \cdot C'}{C \cdot C}$.
In other words,
the map $M_2$ only introduces cusps into the curve $C(t)$ when
$C'(t_0) = kC(t_0)$.

Note that $M_2$ preserves cusps: that is, if $C(t_0)$ is a cusp,
then $M_2(C(t_0))$ is also a cusp.
% (if $C'(t)=0$ then $C'(t) = kC(t)$).
\QED

Thus, $M$ introduces a cusp in $C(t)$
when the curve $C(t)$ passes through
the origin, or its hodograph $C'(t)$ passes through the origin.
$M$ also introduces a cusp if $C(t_0)$ and $C'(t_0)$, 
the vectors to the curve and hodograph {\em at the same parameter value
$t_0$}, are orthogonal as well as lie in the hyperplane $x_4=0$.
Finally, a cusp can be introduced if the vectors to the
curve $M_1(C(t))$ and its tangent, at the same parameter value $t_0$, 
are multiples in the special ratio $k$ of the theorem.
These are all pathological occurences which most curves will not
contain, and moreover they are unstable conditions which can be
removed by manipulation of the original 4-space curve.

\section{Examples}
\label{sec:eg}

The Bezier nature of our spherical curves predicts good quality curves
(e.g., variation-diminishing).
Moreover, our spherical curve will be $C^2$ continuous, since it is the image
under a rational map of a $C^2$-continuous cubic B-spline.
This quality is supported by our practical experience
(Figures~\ref{fig:curve1}-\ref{fig:curve3}).
The curves that we have generated are well-behaved.

We present three examples of a tumbling brick.
In the left picture, the rational Bezier curve on the quaternion sphere
(as determined by our method) is drawn.
It interpolates the indicated quaternions.
The control polygon of the Bezier curve is also drawn.
We visualize the curves in 4-space by using quaternions with $x_3=0$,
thus allowing projection onto the 3-dimension hyperplane 
$x_3=0$.\footnote{Quaternions with $x_3=0$ are mapped by $M^{-1}$ to points
	in 4-space with $x_2=0$, so the entire interpolation in 4-space
	lies in the plane $x_2=0$ and maps back to a spherical curve
	entirely in the hyperplane $x_3=0$.}
This is purely for purposes of visualization: all computations
are in 4-space.
% a point on the
% quaternion sphere with $x_3=0$ is mapped to a point in 4-space with
% $x_2=0$, thus allowing a projection to 3-space.
% stereographic projection may be worth a try

In the right picture, the animation corresponding to this spherical
curve is drawn.
Keyframes (in black) are given as input
and intermediate frames (in light grey) 
are generated using the spherical curve in the left picture.
The orientation of the keyframes is the same as the quaternions
interpolated at left.
In these examples, 5 intermediate frames are created between 
each pair of keyframes.

Notice that the spherical curve only controls the orientation of the frames.
The position of the tumbling brick in
each frame is controlled by a second interpolating curve.
(It is impossible to visualize the change in orientation unless
the brick also moves through 3-space.)

% This non-Euclidean chord-length parameterization is used in our examples.

\section{Conclusions}
\label{sec:finito}

We have developed a way of controlling orientation rationally,
by developing rational Bezier curves that interpolate quaternion orientations.
Control of orientation by a rational Bezier spline is more efficient
and more amenable to manipulation (e.g., alteration of the curve, control
of speed) than control by a non-rational curve 
for which no direct, closed-form representation is known.

By constructing a rational map from 4-space to the quaternion sphere,
and the inverse map from the sphere to 4-space,
we reduce orientation interpolation on the sphere 
to point interpolation in 4-space.
The Bezier structure of the curve in 4-space is mapped to the sphere.
Point interpolation in 4-space can be performed using any
classical method.
We have chosen a cubic B-spline,
since it is perhaps the most widely understood method,
but other methods could certainly be used.
Since it applies traditional techniques of point interpolation,
our method can be viewed as an extension of the interpolation of position
(in 3-space) to the interpolation of orientation (in 4-space).


% Problems of Shoemake solved by our method:
% (1) trigonometric slerping (Slerp(p,q,u) = $(p:q)_u$) replaced by fully polynomial
% interpolation (added speed, robustness, compatibility);
% (2) can add new points on interpolating curve trivially since curve is Bezier,
% whereas not with Shoemake: "there are simple algorithms for adding new sequence
% points to ordinary splines without altering the original curve; they do not
% work for this interpolant" (p. 251);
% (3) answers Shoemake's query "is there is some related interpolant [to Shoemake's
% spherical `Bezier' curves] that is well-characterized?":
% our curves are true Bezier curves in 4-space not pseudo-Bezier curves;
% (4) Cannot achieve  Shoemake's goal of a spherical curve parameterized by
% arc length (p. 251) using arguments of Farouki, but we will try to control
% the speed by knot sequences (and possibly by PH-curve techniques?)

% Conclusion: an improvement of Shoemake's spherical interpolation technique
% (moreover addressing all of the issues that Shoemake left open for improvement).
% and without resorting to costly and nonpolynomial optimization techniques.


\bibliographystyle{alpha}
\begin{thebibliography}{Shoemake 85}

\bibitem{barr92}
Barr, A.H., B. Currin, S. Gabriel and J.F. Hughes (1992)
Smooth interpolation of orientations with angular velocity
constraints using quaternions.  SIGGRAPH '92, Chicago, 26(2), 313--320.

\bibitem{dickson52}
Dickson, L.E. (1952) History of the theory of numbers: Volume II,
Diophantine analysis.  Chelsea (New York).

\bibitem{dietz93}
Dietz, R., J. Hoschek and B. J\"{u}ttler (1993)
An algebraic approach to curves and surfaces on the sphere and on other
quadrics.  Computer Aided Geometric Design 10, 211--229.

\bibitem{duff85}
Duff, T. (1985) Quaternion splines for animating orientation.
Proceedings of the Monterey Computer Graphics Workshop,
54--62.

\bibitem{farin93}
Farin, G. (1993) Curves and surfaces for computer aided geometric design.
Academic Press (New York), third edition.

\bibitem{farouki90}
Farouki, R.T. and T. Sakkalis (1990) Pythagorean hodographs.
IBM Journal of Research and Development 34, 736--752.

\bibitem{farouki91}
Farouki, R.T. and T. Sakkalis (1991) Real rational curves are not
"unit speed".  Computer Aided Geometric Design 8, 151--157.

\bibitem{gabriel85}
Gabriel, S. and J. Kajiya (1985) Spline interpolation in curved space.
In SIGGRAPH '85 course notes `State of the art image synthesis'.

\bibitem{kubota72}
Kubota, K.K. (1972) Pythagorean triples in unique factorization domains.
American Mathematical Monthly 79, 503--505.

\bibitem{misner73}	% for equivalence of S^3 and SO(3) metrics (p. 725),
				% more direct quote would be Weyl or Coxeter
			% for term `spin matrix' (1136), 
			% Royal Canal anecdote (1135),
			% discussion of S^3 (unit sphere in 4-space) (723ff)
Misner, C.W., K.S. Thorne and J.A. Wheeler (1973)
Gravitation.  W.H. Freeman (San Francisco).

\bibitem{pletinckx89}
Pletinckx, D. (1989) Quaternion calculus as a basic tool in computer graphics.
The Visual Computer 5, 2--13.

\bibitem{schlag91}
Schlag, J. (1991) Using geometric constructions to interpolate
orientation with quaternions.  In Graphics Gems II, Academic Press (New York),
377--380.

\bibitem{seidel90}
Seidel, H.P. (1990) Quaternionen in Computergraphik und Robotik.
Informationstechnik 32, 266--275.

\bibitem{shoemake85}
Shoemake, K. (1985) Animating rotation with quaternion curves.
SIGGRAPH '85, San Francisco, 19(3), 245--254.

\end{thebibliography}

%\bibliography{/rb/jj/bib/modeling}

% \clearpage
% 
% \begin{figure}
% \label{fig:method}
% \vspace{3in}
% \caption{FIGURE OF SPHERE, CURVE ON SPHERE, CURVE IN SPACE, AND MAPS M and $M^{-1}$
% (AS ARROWS) BETWEEN CURVES.}
% \end{figure}

% \clearpage

\begin{figure}[h]
%\special{psfile=/rb/jj/Research/quaternion/figs/curve1.ps}
%  	hscale=75 vscale=75 voffset=-200
\vspace{5.5in}
\caption{(a) Rational Bezier spherical curve
(b) Intermediate frames (grey) generated by this spherical curve
	(input keyframes in black)}
\label{fig:curve1}
\end{figure}

\clearpage

\begin{figure}
%\special{psfile=/rb/jj/Research/quaternion/figs/curve2.ps}
\vspace{6in}
\caption{(a) Rational Bezier spherical curve 
(b) Intermediate frames (grey) generated by this spherical curve
	(input keyframes in black)}
\label{fig:curve2}
\end{figure}

\clearpage

\begin{figure}
%\special{psfile=/rb/jj/Research/quaternion/figs/curve3.ps}
\vspace{6in}
\caption{(a) Rational Bezier spherical curve 
(b) Intermediate frames (grey) generated by this spherical curve
	(input keyframes in black)}
\label{fig:curve3}
\end{figure}

\clearpage

\ifFull %%%%%%%%%%%%%%%%%%%%%%%%%%%%%
\section{Appendix}

\begin{lemma}
\label{lem:product}
\begin{equation}
\sum_{i=0}^{3} B_i^3(t) b_i \ \ \sum_{j=0}^{3} B_j^3(t) c_j
= \sum_{k=0}^{6} B_k^6(t) [\sum_{\begin{array}{c} 0 \leq i \leq 3 \\ 
			     0 \leq j \leq 3 \\ 
			     i+j=k
			     \end{array}}
	\frac{\choice{3}{i} * \choice{3}{j}}{\choice{6}{k}}  b_i c_j ]
\end{equation}
for $b_i,c_j \in \Re$.
\end{lemma}
\prf
$\sum_{i=0}^{3} B_i^3(t) b_i \ \ \sum_{j=0}^{3} B_j^3(t) c_j
= \sum_{i=0}^{3} \sum_{j=0}^{3} B_i^3(t) B_j^3(t) b_i c_j
= \sum_{i=0}^{3} \sum_{j=0}^{3} 
	\frac{\choice{3}{i}*\choice{3}{j}}{\choice{6}{i+j}}
	B_{i+j}^6 b_i c_j$
by the product rule of Bernstein polynomials \cite{farin93}.
Now simply let $k=i+j$.
\QED
\fi %%%%%%%%%%%%%%%%%%%%%%%%%%%%%%%%%%%%

\clearpage

\ifFull
\subsection{Routines}
\begin{itemize}
\item
	Draw a banana/cyclide in a specified orientation and position.\\
	void DRAW-CYCLIDE(vector,angle,pos-of-ref-vertex).
	(Cube is too symmetric.)
\item
	DISPLAY-N-FRAMES (n,orientation-curve,translation-curve).
	Given the interpolation curve for orientation and translation,
	choose n t-values (intelligently) to display n well-distributed
	frames of the animation.
\item
	float ARCLENGTH(Beziercurve,t1,t2)
	Compute arclength from t1 to t2 on a Bezier segment.
	Used in DISPLAY-N-FRAMES.
\end{itemize}
\fi

\end{document}
