Rational quaternion splines with derivative information
-------------------------------------------------------

DO AN ANALYSIS OF PATCHING TWO OF THESE SPLINES TOGETHER SMOOTHLY
IN A SEPARATE PAPER.  OVERLY COMPLICATES THIS PAPER
AND IS UNNECESSARY UNLESS YOU HAVE HUGE NUMBER OF POINTS
OR WANT TO INTERPOLATE DERIVATIVE DATA

We have shown that other choices of map {\em can} beat the surface
parameterization.
But what is the {\em best} map for a given surface and type of curve (e.g., interpolating)?

Note: must take care with choice of knot sequence across spline
in this multi-curve version.
Centripetal knot vector worked best in the past.
Also in the past, choice of knot vector had a large influence on the 
quality of the quaternion spline (when two or more curves).

\section{Interpolating derivative information}
\label{sec:deriv}

NO, SKIP IN THIS PAPER.

It is desirable for the design of an interpolating curve to allow the 
addition of derivative as well as position data.
For example, it may be useful to enforce end tangents or curvatures,
and derivative information may even be known at intermediate points.

% added control, easy splicing

% The inverse of a point on \Sn{3}\ is needed to interpolate points.
% We may also want the inverse of a derivative so that we can 
% interpolate derivative information on \Sn{3}.
% Fortunately, this is computed using straightforward differentiation.

% \begin{corollary}
% \label{cor:higherderiv}
% Let $C(t) = (x_1(t), x_2(t), x_3(t), x_4(t)) \subset \Sn{3}$
	% don't use C, that is reserved for the curve in Euclidean space (see 2B)
% be a curve on \Sn{3}\ not containing the pole $(1,0,0,0)$.
% \[
% 	M^{-1}(C(t)) = \alpha(t)(x_2(t), x_3(t), x_4(t), 1-x_1(t))
% \]
% \[
% 	\frac{\partial}{\partial t}M^{-1}(C(t)) = 
% 	\alpha(t)(x'_2(t), x'_3(t), x'_4(t), -x'_1(t)) +
% 	\alpha'(t) (x_2(t), x_3(t), x_4(t), 1-x_1(t))
% \]
% \end{corollary}

% This corollary shows that it is simple to design curves on \Sn{3}\ that
% interpolate derivative data as well as points.
% The inverse images of the derivatives are simply added as constraints
% to the interpolation of step 2A.

\section{Designing the curve in Euclidean space} 

PREAMBLE SAME AS RATIONAL QUATERNION SPLINE PAPER.

Therefore, we would prefer the curve in Euclidean space to move
efficiently between the lines $\{M^{-1}(p_i)\}$.
It would seem that an optimal choice of points	
$\{q_i\}$ would be a set of closest points:
	
\begin{itemize}
\item On the first inverse line, let $q_1$ be the intersection
of the line $M^{-1}(p_1)$ and \Sn{3}.
\item On subsequent lines, let $q_i$ be the closest point to $q_{i-1}$.
\end{itemize}

Unfortunately, a closer look at this solution reveals some problems.
The first problem is that the points $\{q_i\}$ begin to fall
into the origin, continually attracted inwards by their 
shortness criterion (Figure~\ref{fig:spiral}).
% See data66-1.
% About 15 points seems to be the maximum before spiraling reaches origin.
%
% This problem is annoying, but it can be solved by decomposing
% the data set into several parts.
It is also very difficult or even impossible to add derivative data to 
a quaternion spline, as we eventually want to do in Section~\ref{sec:deriv},
with this choice of $q_i$.
This is because the position of $q_i$ depends on the position
of both $p_{i-1}$ and $p_i$, which complicates the mapping of tangents
from \Sn{3}\ to Euclidean space.
For these reasons, we do not use the `closest point' criterion for $\{q_i\}$.

\begin{figure}
\vspace{2in}
\special{psfile=/usr/people/jj/modelTR/3b-splineWithDerivData/img/spiralIn.ps
	 hoffset=150}
\caption{$q_i$ spiraling into the origin when closest is chosen}
% file: s3spline -c < data30-1
% tops spiralIn.rgb -m 6.5 1.5 > spiralIn.ps
\label{fig:spiral}
\end{figure}

\Comment{
By choosing $q_i$ based on the `closest' criterion,
the position of $q_i$ depends on the position of a {\em neighbour} $q_{i-1}$,
and thus on the position of not only the associated data point $p_i$ on the
surface, but also the position of a neighbouring data point $p_{i-1}$.
We would prefer a choice of $q_i$ that only depends on the corresponding
data point $p_i$ on the surface.
(With our eventual choice of the intersection with \Sn{3},
$q_i$ is dependent only on $p_i$, not any of its neighbours.)
The neighbour-dependency of closest-$\{q_i\}$
causes substantial difficulties when translating tangents in Riemannian space
to tangents in Euclidean space: 
we basically need knowledge of a neighbourhood of $p_i$ in Riemannian space
to build the tangent at $q_i$ in Euclidean space.
That is, we need a curve on the surface in order to build the tangents
in Euclidean space!
This is clearly backward.
There is some possibility of solving this apparently self-referential problem,
% recursive definition?
but it involves a sophisticated solution and is a source for future work.
%
% For clarity, consider a new interpretation of our problem:
% notice that by Theorem~\ref{thm:inverse},
% \[
%	q_i = \alpha_i * \mbox{defining point}(p_i),\ \ \alpha_i \in \Re, \ \alpha_i \neq 0.
% \]
% so the construction of $q_i$ can be reduced to the construction of $\alpha_i$.
%
% soln: build first curve, say on points 1-15;
% then build tangent at 15 as follows: build optimal rotation for 13-30
% (that is add in last couple of points of previous subcurve);
% rotate first curve using this 13-30 rotation: the neighbourhood of the
% curve around the endpoint 15 will be safely away from pole;
% compute M^{-1} (first curve) and measure tangent at endpoint 15 off this curve.
% This reduces the problem to finding M^{-1} (curve).
% This reduces to finding $\alpha(t)$ for the inverse curve.
% The set of inverse lines forms a ruled surface.
% We are interested in a line of regression of this ruled surface,
% which follows the closest point at each stage (I believe).
% The computation of this line of regression is equivalent to computing
% the closest point on every line.
% Once \alpha(t) is known, we could determine the Bezier curve rep of the
% curve alpha(t) (x2(t),x3(t),x4(t),1-x1(t))$, or we can simply 
% differentiate $\alpha(t)(x2(t),x3(t),x4(t),1-x1(t))$ and evaluate at endpoint.
}

A good alternative to the `closest point' is the `point on \Sn{3}'.
That is, suppose that we choose $q_i$ to be the intersection of
$M^{-1}(p_i)$ with \Sn{3} with $t>0$.
Then $q_i$ is a good approximation to the closest point to $q_{i-1}$.
% being almost always close to $q_{i-1}$,
This is especially true when the quaternions are not widely spaced, which
is the common scenario in motion control.
This is formalized in Lemma~\ref{lem:goodapprox} below.

\begin{lemma}
If $p = (x_{1}, x_{2}, x_{3}, x_{4}) \in \Sn{3}$, the intersection
of $M^{-1}(p)$ with \Sn{3} (with $t>0$) is
\[
%	M^{-1}(p) \cap \Sn{3} 
\frac{(x_{2}, x_{3}, x_{4}, 1 - x_{1})}{\sqrt{2-2x_{1}}}
\]
\end{lemma}
\prf
The defining point of $M^{-1}(p)$ is $(x_{2}, x_{3}, x_{4}, 1 - x_{1})$
whose length is 
$\sqrt{x_{2}^2 + x_{3}^2 + x_{4}^2 + (1 - x_{1})^2}
= \sqrt{2-2x_{1}}$.
\QED

\begin{lemma}
\label{lem:goodapprox}
Let 
\[ p=(p_1,p_2,p_3,p_4)
\]
and 
\[
P=(P_1 = p_1+\epsilon,P_2,P_3,P_4)
\]
be two unit quaternions at distance $\theta$ on \Sn{3}.
% where $\theta$ is small.\footnote{$\theta$ should certainly be less than $\frac{\pi}{4}$.}
Let 
\[ 
q = M^{-1}(p) \cap \Sn{3}
\]
and 
\[
Q = M^{-1}(P) \cap \Sn{3}
\]
be their inverse images on \Sn{3} (both using $t>0$),
and let $Q'$ be the closest point to $q$ on $M^{-1}(P)$.
Then
\begin{itemize}
\item	the distance between $q$ and $Q$ is $\cos^{-1}(\mbox{foo})$, and
\item	the distance between $Q$ and $Q'$ is $1 - \mbox{foo}$
\end{itemize}
where
\[
	\mbox{foo} = \frac{(\cos \theta + 1) - 2p_1 - \epsilon}
{2\sqrt{(1-p_1-\frac{\epsilon}{2})^2 - \frac{\epsilon^2}{4}}}
	\approx \frac{(\cos \theta + 1) - 2p_1 - \epsilon}
			  {2 - 2p_1 - \epsilon}
\]
If $\theta$ is small, both of these distances are also small.
% , since foo is close to 1.
\end{lemma}
\prf
The angle $\theta$ between $p$ and $P$ (which measures their distance on \Sn{3})
satisfies 
\[	
	\cos \theta = p_1P_1 + p_2P_2 + p_3P_3 + p_4P_4.
\]
Similarly, the angle $\alpha$ between $q$ and $Q$ satisfies
\[
	\cos \alpha = \frac{(p_2,p_3,p_4,1-p_1)}{\sqrt{2-2p_1}}
	\cdot
	\frac{(P_2,P_3,P_4,1-P_1)}{\sqrt{2-2P_1}}
	= \frac{\cos \theta + 1 - p_1 - P_1}
	       {2\sqrt{1-p_1-P_1+p_1P_1}}
\]
\[
	= \frac{\cos \theta + 1 - 2p_1 - \epsilon}
	       {2\sqrt{1-2p_1-\epsilon+p_1^2 + \epsilon p_1}}
	= \frac{\cos \theta + 1 - 2p_1 - \epsilon}
	       {2\sqrt{(1-p_1-\frac{\epsilon}{2})^2 - \frac{\epsilon^2}{4}}}
\]
This is the expression called foo in the statement of the lemma.
Now suppose that $\theta$ is small.
Then $\epsilon$ is small and $\frac{\epsilon^2}{4}$ is very small.
If we ignore it, we can simplify the above expression to
\[
	\cos \alpha 
% = \frac{(\cos \theta + 1) - 2p_1 - \epsilon}
% {2\sqrt{(1-p_1-\frac{\epsilon}{2})^2}}
	       	    = \frac{(\cos \theta + 1) - 2p_1 - \epsilon}
	       {2 - 2p_1 - \epsilon}
\]
Since $\cos \theta + 1$ is close to 2, 
\[
	\cos \alpha \approx 
	\frac{2 - 2p_1 - \epsilon}{2 - 2p_1 - \epsilon} = 1
\]
This shows that the distance $\alpha$ between $q$ and $Q$ is
small, since $\cos \alpha \approx 1$ implies $\alpha \approx 0$.
Thus, quaternions that are close together are mapped to
inverse images in Euclidean space that are close together.

Figure~\ref{fig:QQdist} shows that the 
distance between $Q$ and $Q'$ is $1-\cos \alpha$.
This distance is also small because $\cos \alpha \approx 1$.
\QED

\begin{figure}
\vspace{2.5in}
\special{psfile=/usr/people/jj/modelTR/3b-splineWithDerivData/img/dist1-cosAlpha.ps
	 hoffset=150}
\caption{The distance between $Q$ and $Q'$ is $1-\cos\alpha$}
% file: dist1-cosAlpha.showcase
% tops dist1-cosAlpha.rgb -m 6.5 1.5 > dist1-cosAlpha.ps
\label{fig:QQdist}
\end{figure}

We have established that the choice of $q_i$ on \Sn{3}\ is a good
approximation to the closest point on the inverse line
and will satisfy our goal of moving efficiently between the inverse lines.
Consequently, the version of step (2b) that we implement is:
%
\begin{description}
\item[(2b)]
	Design a rational curve $C$ in Euclidean space,
	interpolating $\{M^{-1}(p_i) \cap \Sn{3}\}_{i=1,\ldots,k}$.
\end{description}

The observant reader will have noticed that, by mapping to Euclidean
space in this way, we have replaced the
interpolation of points on \Sn{3} (quaternions) by the interpolation
of points on \Sn{3} (their images in Euclidean space)!
However, we have actually made progress:
the design of the interpolating curve is no longer constrained to
a surface.

Open problems:
\begin{itemize}
\item
	Determining tangent in Euclidean space when using closest point on inverse line.
		(Involves computing $\alpha(t)$.)
		[See comments embedded in paper above.]
\item
	Compare eager and lazy approaches to input subdivision,
	   as well as maximal/optimal sizes for the subset in eager evaluation
\item	
	$C^2$ continuous splines, by passing 2nd derivative back too
		(still easy to set 2nd derivative, filling the 2nd dof of the cubic spline)
\end{itemize}		

A topic for future study is the choice of an optimal value
for the subset size $k$ in lazy subdivision 
and the tradeoff between lazy and eager subdivision.




Explicitly minimizing covariant acceleration:
Covariant acceleration is zero at a point of a curve on \Sn{3}\ if the curve's
second derivative vector is orthogonal to the sphere.
We can try to enforce this restriction at data points,
since the desired second derivative vector for the quaternion spline
can be mapped to a desired second derivative vector in inverse space,
where design is carried out.
To interpolate a second derivative vector at every data point will require
a quintic Hermite curve in inverse space, which will map to a degree 10
curve on \Sn{3}.
One can keep a cubic curve in inverse space by simply interpolating
second derivative vectors at the endpoints, since there are two free
end conditions in a cubic B-spline.

