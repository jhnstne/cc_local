\documentclass[11pt]{article} 
\usepackage{times}
\usepackage[pdftex]{graphicx}
\makeatletter
\def\@maketitle{\newpage
 \null
 \vskip 2em                   % Vertical space above title.
 \begin{center}
       {\Large\bf \@title \par}  % Title set in \Large size. 
       \vskip .5em               % Vertical space after title.
       {\lineskip .5em           %  each author set in a tabular environment
        \begin{tabular}[t]{c}\@author 
        \end{tabular}\par}                   
  \end{center}
 \par
 \vskip .5em}                 % Vertical space after author
\makeatother

% default values are 
% \parskip=0pt plus1pt
% \parindent=20pt

\newcommand{\SingleSpace}{\edef\baselinestretch{0.9}\Large\normalsize}
\newcommand{\DoubleSpace}{\edef\baselinestretch{1.4}\Large\normalsize}
\newcommand{\Comment}[1]{\relax}  % makes a "comment" (not expanded)
\newcommand{\Heading}[1]{\par\noindent{\bf#1}\nobreak}
\newcommand{\Tail}[1]{\nobreak\par\noindent{\bf#1}}
\newcommand{\QED}{\vrule height 1.4ex width 1.0ex depth -.1ex\ \vspace{.3in}} % square box
\newcommand{\arc}[1]{\mbox{$\stackrel{\frown}{#1}$}}
\newcommand{\lyne}[1]{\mbox{$\stackrel{\leftrightarrow}{#1}$}}
\newcommand{\ray}[1]{\mbox{$\vec{#1}$}}          
\newcommand{\seg}[1]{\mbox{$\overline{#1}$}}
\newcommand{\tab}{\hspace*{.2in}}
\newcommand{\se}{\mbox{$_{\epsilon}$}}  % subscript epsilon
\newcommand{\ie}{\mbox{i.e.}}
\newcommand{\eg}{\mbox{e.\ g.\ }}
\newcommand{\figg}[3]{\begin{figure}[htbp]\vspace{#3}\caption{#2}\label{#1}\end{figure}}
\newcommand{\be}{\begin{equation}}
\newcommand{\ee}{\end{equation}}
\newcommand{\prf}{\noindent{{\bf Proof}:\ \ \ }}
\newcommand{\choice}[2]{\mbox{\footnotesize{$\left( \begin{array}{c} #1 \\ #2 \end{array} \right)$}}}      
\newcommand{\scriptchoice}[2]{\mbox{\scriptsize{$\left( \begin{array}{c} #1 \\ #2 \end{array} \right)$}}}
\newcommand{\tinychoice}[2]{\mbox{\tiny{$\left( \begin{array}{c} #1 \\ #2 \end{array} \right)$}}}
\newcommand{\ddt}{\frac{\partial}{\partial t}}
\newcommand{\Sn}[1]{\mbox{{\bf S}$^{#1}$}}
\newcommand{\calP}[1]{\mbox{{\bf {\cal P}}$^{#1}$}}

\newtheorem{theorem}{Theorem}	
\newtheorem{rmk}[theorem]{Remark}
\newtheorem{example}[theorem]{Example}
\newtheorem{conjecture}[theorem]{Conjecture}
\newtheorem{claim}[theorem]{Claim}
\newtheorem{notation}[theorem]{Notation}
\newtheorem{lemma}[theorem]{Lemma}
\newtheorem{corollary}[theorem]{Corollary}
\newtheorem{defn2}[theorem]{Definition}
\newtheorem{observation}[theorem]{Observation}

% \font\timesr10
% \newfont{\timesroman}{timesr10}
% \timesroman


\setlength{\oddsidemargin}{0pt}
\setlength{\topmargin}{0in}
\setlength{\textheight}{8.6in}
\setlength{\textwidth}{6.875in}
\setlength{\columnsep}{5mm}
% \markright{Orientation control (\today) \hfill}
% \pagestyle{myheadings}

% -----------------------------------------------------------------------------
\title{Tech report 1: Quaternion splines for rational motion design}
% Johnstone and Williams
\begin{document}
\maketitle

% next paper: ACTIVE ORIENTATION OBSTACLES
% design of quaternion splines that avoid active orientation obstacles
% (and the definition of these orientation obstacles using quaternion splines);
% an active orientation obstacle is an obstacle based on the movement of the robot,
% not on the environment, such as not spilling milk;
% 
% elaboration on this theme: PASSIVE ORIENTATION OBSTACLES AND INTELLIGENT PLANNING
% generation of collision-free motions, using
% passive orientation obstacles defined by the environment; this is harder
% and would solve a problem mentioned way back in Lozano-Perez
%
% another future paper: DUMB MOTION PLANNING (IS THIS ENOUGH TO MOTIVATE RATIONALITY?)
% collision detection using the rational swept volumes generated by the rational motions
% (this would be a dumb replacement of the smart motion planner proposed above:
% dumb in the sense that a motion is simply generated and tested, then regenerated
% based on the found collision and retested, perhaps ad infinitum)

The design of quaternion splines for motion design may be approached through
a two-step process, as follows.
Let $f$ be a map from Euclidean space (typically 3-space or 4-space) to $S^3$,
and let $Q$ be a set of quaternions.

\begin{itemize}
% \item Map the quaternions $Q$ under $f^{-1}$.
\item Build a curve C that interpolates the points $f^{-1}(Q)$.
\item Map the curve C under $f$, yielding a quaternion spline $f(C)$ 
      that interpolates $Q$, embedded in $S^3$.
\end{itemize}

The advantage of this approach is that it isolates the two challenges
of quaternion spline design, point interpolation and constraint to the 3-sphere.
The first step addresses the interpolation, while the second step addresses the constraint
to the 3-sphere.
Since the interpolation does not need to worry about the surface constraint, it can use
classical interpolation in free Euclidean space.
The map to the 3-sphere handles the surface constraint.
An added benefit is that, if the map $f$ is rational, the quaternion spline is rational.

In an earlier paper, we solved one central challenge of this approach, 
the design of a rational map to the sphere.
In particular, a characterization of all rational maps from Euclidean $m$-space 
to the n-sphere was developed.
This companion paper continues on with the study of motion design,
and makes concrete the entire algorithm for the construction of rational 
quaternion splines.
In particular, the following problems are solved:
the choice of a particular rational map $f$ to \Sn{3},
the construction of the inverse $f^{-1}$ of this map,
techniques to map the quaternions $Q$ away from the poles of $f^{-1}$ (inherent to any
$f^{-1}$) before mapping,
% (which leads to the division of the large interpolation problem
% into several sub-interpolation problems),
the image of derivatives under $f^{-1}$ (demanded by the solution to pole avoidance),
% see p. 37 forward of hermiteQ.pdf
and the image of a Bezier curve under $f$.

\end{document}
