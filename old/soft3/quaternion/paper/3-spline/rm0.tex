\documentclass[11pt]{article} 
\usepackage{times}
\usepackage[pdftex]{graphicx}
\makeatletter
\def\@maketitle{\newpage
 \null
 \vskip 2em                   % Vertical space above title.
 \begin{center}
       {\Large\bf \@title \par}  % Title set in \Large size. 
       \vskip .5em               % Vertical space after title.
       {\lineskip .5em           %  each author set in a tabular environment
        \begin{tabular}[t]{c}\@author 
        \end{tabular}\par}                   
  \end{center}
 \par
 \vskip .5em}                 % Vertical space after author
\makeatother

% default values are 
% \parskip=0pt plus1pt
% \parindent=20pt

\newcommand{\SingleSpace}{\edef\baselinestretch{0.9}\Large\normalsize}
\newcommand{\DoubleSpace}{\edef\baselinestretch{1.4}\Large\normalsize}
\newcommand{\Comment}[1]{\relax}  % makes a "comment" (not expanded)
\newcommand{\Heading}[1]{\par\noindent{\bf#1}\nobreak}
\newcommand{\Tail}[1]{\nobreak\par\noindent{\bf#1}}
\newcommand{\QED}{\vrule height 1.4ex width 1.0ex depth -.1ex\ \vspace{.3in}} % square box
\newcommand{\arc}[1]{\mbox{$\stackrel{\frown}{#1}$}}
\newcommand{\lyne}[1]{\mbox{$\stackrel{\leftrightarrow}{#1}$}}
\newcommand{\ray}[1]{\mbox{$\vec{#1}$}}          
\newcommand{\seg}[1]{\mbox{$\overline{#1}$}}
\newcommand{\tab}{\hspace*{.2in}}
\newcommand{\se}{\mbox{$_{\epsilon}$}}  % subscript epsilon
\newcommand{\ie}{\mbox{i.e.}}
\newcommand{\eg}{\mbox{e.\ g.\ }}
\newcommand{\figg}[3]{\begin{figure}[htbp]\vspace{#3}\caption{#2}\label{#1}\end{figure}}
\newcommand{\be}{\begin{equation}}
\newcommand{\ee}{\end{equation}}
\newcommand{\prf}{\noindent{{\bf Proof}:\ \ \ }}
\newcommand{\choice}[2]{\mbox{\footnotesize{$\left( \begin{array}{c} #1 \\ #2 \end{array} \right)$}}}      
\newcommand{\scriptchoice}[2]{\mbox{\scriptsize{$\left( \begin{array}{c} #1 \\ #2 \end{array} \right)$}}}
\newcommand{\tinychoice}[2]{\mbox{\tiny{$\left( \begin{array}{c} #1 \\ #2 \end{array} \right)$}}}
\newcommand{\ddt}{\frac{\partial}{\partial t}}
\newcommand{\Sn}[1]{\mbox{{\bf S}$^{#1}$}}
\newcommand{\calP}[1]{\mbox{{\bf {\cal P}}$^{#1}$}}

\newtheorem{theorem}{Theorem}	
\newtheorem{rmk}[theorem]{Remark}
\newtheorem{example}[theorem]{Example}
\newtheorem{conjecture}[theorem]{Conjecture}
\newtheorem{claim}[theorem]{Claim}
\newtheorem{notation}[theorem]{Notation}
\newtheorem{lemma}[theorem]{Lemma}
\newtheorem{corollary}[theorem]{Corollary}
\newtheorem{defn2}[theorem]{Definition}
\newtheorem{observation}[theorem]{Observation}

% \font\timesr10
% \newfont{\timesroman}{timesr10}
% \timesroman


\setlength{\oddsidemargin}{0pt}
\setlength{\topmargin}{0in}
\setlength{\textheight}{8.6in}
\setlength{\textwidth}{6.875in}
\setlength{\columnsep}{5mm}
% \markright{Orientation control (\today) \hfill}
% \pagestyle{myheadings}

% -----------------------------------------------------------------------------
\title{Tech report 0: Rational maps to the [quaternion] sphere for motion\\January-February 2005}
% Johnstone and Williams
% ask Ian Knowles for a suggestion of a mathematical journal
\begin{document}
\maketitle

Keywords: sphere, Pythagorean tuples, Euler's four squares theorem, (quaternion spline?).

This report studies rational maps from Euclidean space to the sphere,
in arbitrary dimensions,
including a complete % an exhaustive (but hopefully not exhausting)
characterization of these maps, a complete characterization of the 
related Pythagorean tuples, examples of these maps and a consideration
of classical maps to the sphere, 
% an analysis of the inverse maps % from the sphere to Euclidean space 
% (which are typically not rational), 
and an analysis of the
% inevitable 
poles of the inverse maps.
% [We could include the image of a rational Bezier curve under these maps, 
% if more is needed.  But I think that this is more natural in the paper on quaternion 
% splines, which also addresses avoidance of the poles, divide and conquer, and hence
% derivatives. Delay discussion of stereographic projection to this second paper too,
% since only its inverse is a map to the sphere and we heavily use s.p. in building'
% rational quaternion splines]
%
In particular, this paper studies rational maps from $\Re^m$ to \Sn{n}.

\begin{defn2}
\Sn{n}\ is the unit sphere in $(n+1)$-space $x_1^2 + \ldots + x_{n+1}^2 - 1 = 0$,
a manifold of dimension $n$.
\end{defn2}

\begin{defn2}
A map $(x_1,\ldots,x_n) \mapsto (f_1 (x_1,\ldots,x_n),\ldots,f_m (x_1,\ldots,x_n))$
is {\bf rational} if 
the $f_i$ are all rational polynomials in $x_1,\ldots,x_n$, $i=1,\ldots,m$.
A {\bf rational polynomial} is a quotient of polynomials.
\end{defn2}

The study of rational maps to the sphere is motivated by a problem in motion design.
% Consider motion, % of a rigid object, or of a rigid component of a nonrigid object,
% which is change in position and orientation.
A common task in motion design is the interpolation of a collection of quaternions
by a quaternion spline, which then represents the orientation component of a motion.
The unit quaternion is a preferred representation for orientation
in computer graphics, physics \cite{misner}, and robotics.
Since unit quaternions are points on the unit sphere $S^3$ in 4-space, 
the orientation component of a motion may be captured by a curve on $S^3$, called
a quaternion spline.
The design of a quaternion spline is made particularly challenging because 
the curve is constrained to remain on the sphere $S^3$.
This awkward constrained interpolation problem may be reduced to 
a classical free interpolation problem through the use of a map from Euclidean space 
to the sphere, as follows.
Let $f$ be a map from Euclidean space (typically 3-space or 4-space) to $S^3$,
and let $Q$ be a set of quaternions.

\begin{itemize}
% \item Map the quaternions $Q$ under $f^{-1}$.
\item Build a curve C that interpolates the points $f^{-1}(Q)$, a classical
      interpolation problem.
\item Map the curve C under $f$, yielding a quaternion spline $f(C)$ 
      that interpolates $Q$, embedded in $S^3$.
\end{itemize}

In this solution, the two challenges of point interpolation and
constraint to the sphere are separated,
unlike other solutions that handle both challenges together \cite{animationtext}.
The first step concentrates solely on point interpolation,
while the constraint to the sphere is handled by the spherical map.
Another advantage of this solution is its potential for rationality.
The quaternion spline's constraint to $S^3$ has typically led to nonrational solutions.
If $f$ is a rational map to the sphere
and the interpolating curve $C$ is rational (as it will be in the 
classical construction \cite{farin}),
the quaternion spline $f(C)$ will also be rational, with all of the 
attendant advantages of rational curves.
% (based on constrained optimization, trigonometric slerping, or Lie exponential maps.)
Rational quaternion splines are particularly attractive for analysis of the 
resulting motion, such as collision detection.
A third advantage is that the above construction of rational quaternion splines 
is more efficient than some other nonrational solutions \cite{barr92,barrafter92}.

[{\em Figure of a quaternion spline and its associated motion.
Figure of algorithm's solution, step by step (perhaps using stereographic projection).}]

\clearpage

\section{Rational maps to the sphere and Pythagorean tuples}
\label{sec:ratpy}

The key observation in the characterization of rational maps to the sphere is their
equivalence to Pythagorean tuples.
A Pythagorean tuple is a tuple that satisfies a sum of squares relationship,
generalizing the sum of squares relationship between the sides of a right triangle
in the Pythagorean Theorem.

\begin{defn2}
A {\bf Pythagorean tuple over $K$} % n+1-tuple is clear from context
is a tuple $(f_1,\ldots,f_{n+1}) \in K^{n+1}$
where $f_1^2 + \ldots + f_n^2 = f_{n+1}^2$.
\end{defn2}

% n --> m
Consider a rational map to the sphere $f: \Re^m \rightarrow \Sn{n-1}$:
\[
	f(x_1,\ldots,x_m) =
	(\frac{f_1(x_1,\ldots,x_m)}{f_{n+1}(x_1,\ldots,x_m)}, \ldots,
	 \frac{f_n(x_1,\ldots,x_m)}{f_{n+1}(x_1,\ldots,x_m)})
\]
where $f_1,\ldots,f_{n+1}$ are polynomials.
Since $f$ maps to the unit sphere at the origin,
\begin{eqnarray}
       (\frac{f_1}{f_{n+1}})^2 + \cdots + (\frac{f_n}{f_{n+1}})^2 & = & 1 \\
	f_1^2 + \cdots + f_n^2 & = & f_{n+1}^2
\end{eqnarray}
and the components of the rational map define a Pythagorean $(n+1)$-tuple
over the ring of polynomials $\Re[x_1,\ldots,x_m]$.
In short, rational maps to \Sn{n-1} are identified with Pythagorean $(n+1)$-tuples.

\Comment{
\begin{defn2}
$\Re[x_1,\ldots,x_n]$ is the ring of polynomials in the $n$ variables
$x_1,\ldots,x_n$ with real coefficients.
\end{defn2}
}

% This reduces the study of rational maps from $\Re^{n}$ to \Sn{m-1}\ 
% to the study of Pythagorean $(m+1)$-tuples over polynomials $\Re[x_1,\ldots,x_n]$.

% \begin{lemma}
% \label{lem:iffpyth}
% A rational map $f: \Re^n \rightarrow \Re^m$ defined by 
% $f(x_1,\ldots,x_n) = (\frac{f_1(x_1,\ldots,x_n)}{f_{m+1}(x_1,\ldots,x_n)}, \ldots,
%	 \frac{f_m(x_1,\ldots,x_n)}{f_{m+1}(x_1,\ldots,x_n)})$,
% where $f_1,\ldots,f_{m+1} \in \Re[x_1,\ldots,x_n]$, is a rational map from $\Re^n$
% to the sphere \Sn{m-1}\ if and only if $(f_1,\ldots,f_{m+1})$ is a Pythagorean
% $(m+1)$-tuple over polynomials $\Re[x_1,\ldots,x_n]$.
% \end{lemma}

% notice that m+1-tuples map to m-1-sphere

Pythagorean tuples and sums of squares have been heavily studied in number theory.
An examination of this literature will lead us to a characterization of rational maps
to the sphere.
We begin with a review of Pythagorean triples and Pythagorean quadruples,
which are well known, and the related maps to \Sn{1} and \Sn{2}.
We then consider Euler's fundamental work on the sum of four squares,
which can be related to Pythagorean quintuples.
Although this work does not yield a full characterization of Pythagorean quintuples,
it is sufficient to characterize rational maps to \Sn{3}.
Finally, we develop a characterization of rational maps to \Sn{n}
by generalizing the technique for Pythagorean quintuples, 
using Ammei's generalization of Euler's result.

% The rest of the paper is structured as follows.

% Since there is flexibility in which Euclidean space is involved, we consider a general
% Euclidean space.

\section{Previous work on maps to the sphere}

Maps from the 2-sphere to the plane, whose inverses are maps to the 2-sphere,
have long been of interest because of their application to cartography.
There are many well-known examples, such as the mapping of Sanson,
the mapping of Bonne, and the Lambert projection \cite{kreyszig59}.
The most famous is the Mercator projection, which maps the point
$(\cos u \cos v,\ \cos u \sin v,\ \sin u)$ on the 2-sphere 
to the point $(v,\ \log \tan (\frac{u}{2} + \frac{\pi}{4}))$ on the plane.
However, none of these classical maps are rational,
since properties such as conformality (preservation of angles between 
intersecting curves) and equiareality (preservation of area) are more important than
rationality (the ability to express the map in terms of quotients of polynomials)
in the design of cartographic maps.
Stereographic projection is the rare exception: a rational 
map from the sphere to Euclidean space, whose inverse is also rational.
We shall fully explore this inverse map in another paper \cite{jj}.

% stereographic projection: no, see above

Rational parameterizations of the sphere are rational maps to the sphere.
% since the parameterization of a surface is a map of the parameter space 
% (an Euclidean space) to the surface.
For example, the classical rational parameterization of \Sn{2}, constructed using the
spherical product \cite{barr81} of a semi-circle and a circle,
% $(a(u),b(u)) \times (c_1(v_1,...,v_{n-1}),\ldots,c_n(v_1,\ldots,v_{n-1}) 
% = (a c_1, \ldots, a c_{n-1}, b c_n)$.
% The spherical product is a general scheme for sweeping one curve along another,
% while modulating its size.
is the following rational map of 2-space to the sphere:\footnote{The 
  restriction of $u$ to $[-1,1]$
  can be removed without problem: the map then double covers the sphere.}
\begin{equation}
\label{eq:s2param}
S(u,v) = (\frac{1-u^2}{1+u^2}(\frac{1-v^2}{1+v^2}),\  
          \frac{1-u^2}{1+u^2}(\frac{2v}{1+v^2}),\  \frac{2u}{1+u^2}),\ 
u \in [-1,1],\ v \in (-\infty,+\infty)
\end{equation}
% Rational parameterizations are of interest for the development of rational Bezier or
% NURBS representations of the sphere.

{\bf 
Reduction to Pythagorean tuples (done).
Euler characterization of Pythagorean tuples (done).
Full characterization (done).
Visualization (removed).
Poles (done).
}

\section{Pythagorean tuples}

\subsection{Pythagorean triples}

It is a classical result for K = integers, 
generalized to K = polynomials by Kubota \cite{kubota72},
that a triple over K is a Pythagorean triple if and only if
it is of the form 
$\alpha(a_1^2 - a_2^2, 2a_1a_2, a_1^2 + a_2^2)$
for some $a_1,a_2,\alpha \in K$.\footnote{Whenever
	we give a normal form for a Pythagorean $(n+1)$-tuple,
	it is understood that the first $n$ elements are free to undergo
	a permutation.}
  % p. 89, Ebbinghaus, Numbers
Farouki and Sakkalis \cite{farouki90} have used
Kubota's result to define Pythagorean hodograph curves, 
which have attractive arc length and offset properties.
The hodographs of these curves are components of a Pythagorean triple
of polynomials.
  % which among other things have polynomial arc length (in the curve parameter).
To illustrate the association between Pythagorean triples and rational maps to \Sn{1},
note that the standard rational parameterization for \Sn{1} (a rational map to \Sn{1})
may be developed from the normal form for Pythagorean triples 
using $(a_1,a_2,\alpha) = (1,t,1)$:
$(\frac{1-t^2}{1+t^2}, \frac{2t}{1+t^2})$.

\subsection{Pythagorean quadruples}

Catalan \cite{catalan85} established the normal form for Pythagorean quadruples over
K = integers, generalized by Dietz et. al. \cite{dietz93} to K = polynomials in 
$\Re[x_1]$, $\Re[x_1,x_2]$ or $\Re[x_1,x_2,x_3]/<x_1+x_2+x_3-1>$:
a quadruple over $K$ is a Pythagorean quadruple if and only if it is of the form
\begin{equation}
\label{eq:quad}
\alpha (a_1^2+a_2^2-a_3^2-a_4^2,\ 2a_1a_3+2a_2a_4,\ 2a_1a_4-2a_2a_3,
\ a_1^2+a_2^2+a_3^2+a_4^2)
\end{equation}
for some $a_1,a_2,a_3,a_4,\alpha \in K$.
The rational map to \Sn{2}\ 
resulting from the choice $(a_1,a_2,a_3,a_4) = (x_1,x_2,x_3,1)$ 
% (with a permutation (1,2,3) to (3,2,1)) 
was used by Dietz et. al. \cite{dietz93}.
The standard rational parameterization (\ref{eq:s2param}) of \Sn{2}
may be developed from the normal form
for Pythagorean quadruples using $(a_1,a_2,a_3,a_4,\alpha) = (1,uv,u,v,1)$.
% and a permutation.
% $\frac{1}{1+u^2+v^2+u^2v^2}(1-u^2-v^2+u^2v^2, 2v - 2u^2v,2u+2uv^2)$.

\subsection{Pythagorean quintuples}

We are not aware of any specific study of Pythagorean quintuples.
However, there is an extensive study of the sum of four squares in the 
number theory literature, which may be related to Pythagorean quintuples.
Moreover, this development can be generalized to arbitrary Pythagorean tuples.

The study of four squares in the eighteenth century
was driven by the elusive search for a proof that
every positive integer is the sum of the squares of four integers.
This famous result was apparently known to Diophantus in the third century,
since he assumes it implicitly in his writings \cite{dickson52}, % p. 275
and Fermat % (in characteristic fashion!) 
claimed that he had a proof.\footnote{"I can
	not give the proof here, which depends upon numerous and abstruse
	mysteries of numbers; for I intend to devote an entire book
	to this subject" \cite[p. 6]{dickson52}.}
However, the first published proof was by Lagrange in 1770 \cite{herstein75}, % p. 375
over a century later,
during which time many mathematicians worked on the problem.
One of the mathematicians who was intensely interested in this problem was Euler, 
and he established the following crucial result in 1748 \cite{weil84}.   
        % p. 226 of Weil
        % p. 209 of Ebbing, p. 373 of Herstein herstein75
	% p. 210, Ebbinghaus for any commutative ring and Gauss' proof
	% another bulky version in \cite[p. 277]{dickson52}.

\begin{theorem}[Euler's Four Squares Theorem]
\label{thm:euler}
\[
\begin{array}{ll}
& (a_1^2 + a_2^2 + a_3^2 + a_4^2) 
(\hat{a}^2_1 + \hat{a}^2_2 + \hat{a}^2_3 + \hat{a}^2_4) = \\
& (a_1 \hat{a}_1 + a_2\hat{a}_2 + a_3\hat{a}_3 + a_4\hat{a}_4)^2 +
  (a_1 \hat{a}_2 - a_2\hat{a}_1 - a_3\hat{a}_4 + a_4\hat{a}_3)^2 + \\
& (a_1\hat{a}_3 + a_2\hat{a}_4 - a_3\hat{a}_1 - a_4\hat{a}_2)^2 +
  (a_1\hat{a}_4 - a_2\hat{a}_3 + a_3\hat{a}_2 - a_4\hat{a}_1)^2
\end{array}
\]
for any integers $a_1,a_2,a_3,a_4,\hat{a}_1,\hat{a}_2,\hat{a}_3,\hat{a}_4$.
%(but it can easily be generalized to polynomials, or any elements of a commutative ring).
% \footnote{The original statement was for integers, but it easily generalizes.}
% see p. 210 of Ebbinghaus
\end{theorem}
%
\Comment{
% version in Herstein (called Lagrange's identity) and Ebbinghaus
% the version directly attributed to Euler in Dickson and Weil is cleaner, so we use it
USING THE DICKSON FORM:
\begin{theorem}[Euler's Four Squares Theorem]
\label{thm:euler}
\[
\begin{array}{ll}
& (a_1^2 + a_2^2 + a_3^2 + a_4^2) 
(\hat{a}^2_1 + \hat{a}^2_2 + \hat{a}^2_3 + \hat{a}^2_4) = \\
& (a_1 \hat{a}_1 - a_2\hat{a}_2 - a_3\hat{a}_3 - a_4\hat{a}_4)^2 +
   (a_1\hat{a}_2 + a_2\hat{a}_1 + a_3\hat{a}_4 - a_4\hat{a}_3)^2 + \\
& (a_1\hat{a}_3 - a_2\hat{a}_4 + a_3\hat{a}_1 + a_4\hat{a}_2)^2 +
   (a_1\hat{a}_4 + a_2\hat{a}_3 - a_3\hat{a}_2 + a_4\hat{a}_1)^2
\end{array}
\]
for integers $a_1,a_2,a_3,a_4,\hat{a}_1,\hat{a}_2,\hat{a}_3,\hat{a}_4$.
% \footnote{The original statement was for integers, but it easily generalizes.}
% see p. 210 of Ebbinghaus
\end{theorem}
}
%
This result, which is also called Lagrange's identity \cite{herstein75}, % p. 373
establishes that the product of two sums of four squares is another sum of four squares,
reducing the problem of showing that every integer is the sum of four squares to 
the simpler problem of showing that every prime is the sum of four squares.
  % since every integer may be expressed as the product of primes.
However, for our purposes, 
Euler's Four Squares Theorem is a Pythagorean quintuple generator.

\begin{corollary}
\label{cor:euler}
\begin{equation}
\label{eq:aida}
(a_1^2 + a_2^2 + a_3^2 - a_4^2)^2 + (2a_1a_4)^2 + (2a_2a_4)^2 + (2a_3a_4)^2
= (a_1^2 + a_2^2 + a_3^2 + a_4^2)^2
\end{equation}
for any rational numbers or rational polynomials $a_1,a_2,a_3,a_4$.
  % where $a_1,a_2,a_3,a_4$ are elements of any commutative ring.
\end{corollary}
\prf
(\ref{eq:aida}) is generated by substituting 
$(\hat{a}_1,\hat{a}_2,\hat{a}_3,\hat{a}_4) = (a_1,a_2,a_3,-a_4)$.
This is the next obvious choice after 
$(\hat{a}_1,\hat{a}_2,\hat{a}_3,\hat{a}_4) = (a_1,a_2,a_3,a_4)$,
which yields a tautology.
This proves the result for integers.
The Four Squares Theorem may easily be generalized to polynomials, 
or any elements of a commutative ring (see Definition~ref{defn:commring} below), 
such as the rationals and rational polynomials.
% It can be further generalized to rational numbers or rational polynomials
% by placing these numbers or polynomials over a least common multiple,
% which then factors out.
\QED

\Comment{
    \section{A Characterization of Rational Maps of $\Re^4$ to \Sn{3}}

    Euler's Four Squares Theorem may be used to develop
    a weak\footnote{It is weak in the sense of its relaxed restriction on $\alpha$,
    as discussed below.} 
    normal form for Pythagorean quintuples.

    % VERSION FOR DICKSON FORM OF 4-SQUARE THEOREM  
    We get the following formula by letting 
    $(\hat{a}_1,\hat{a}_2,\hat{a}_3,\hat{a}_4) = (a_1,-a_2,-a_3,a_4)$
    in the Four Squares Theorem:
    % \label{lem:corEuler4square}
    \begin{equation}
    \label{eq:aida}
    (a_1^2 + a_2^2 + a_3^2 - a_4^2)^2 + (2a_1a_4)^2 + (2a_2a_4)^2 + (2a_3a_4)^2
    = (a_1^2 + a_2^2 + a_3^2 + a_4^2)^2
    \end{equation}
    % where $a_1,a_2,a_3,a_4$ are elements of any commutative ring.
}

This leads to a normal form for Pythagorean quintuples.

\begin{theorem}
\label{thm:quint}
A quintuple over K is a Pythagorean quintuple if and only it is of the form
\begin{equation}
\label{eq:pythquint}
	\alpha (a_1^2 + a_2^2 + a_3^2 - a_4^2,
		\ 2a_1a_4,\ 2a_2a_4,\ 2a_3a_4,
		\ a_1^2 + a_2^2 + a_3^2 + a_4^2)
\end{equation}
for some $a_1,a_2,a_3,a_4,\alpha\neq 0 \in K$, 
where $K$ is the field of rational numbers
or rational polynomials over $\Re[x_1,x_2,\ldots,x_n]$.
\end{theorem}
\prf
(If:)
Suppose that $Q = \alpha (a_1^2 + a_2^2 + a_3^2 - a_4^2,
		\ 2a_1a_4,\ 2a_2a_4,\ 2a_3a_4,
		\ a_1^2 + a_2^2 + a_3^2 + a_4^2)$ is a quintuple over $K$.
By Corollary~\ref{cor:euler}, $\frac{1}{\alpha} Q$ is a Pythagorean quintuple.
But any nonzero multiple of a Pythagorean tuple is also a Pythagorean tuple,
so $Q$ is a Pythagorean quintuple.

(Only if:) 
Let $Q = (p_1,p_2,p_3,p_4,p_5)$ be a Pythagorean quintuple over $K$.
% and let $f(a_1,a_2,a_3,a_4,\alpha)$ be a shorthand for the normal form
% $\frac{1}{\alpha} (a_1^2 + a_2^2 + a_3^2 - a_4^2,
%		 \ 2a_1a_4,\ 2a_2a_4,\ 2a_3a_4,
%		 \ a_1^2 + a_2^2 + a_3^2 + a_4^2)$.
If $p_1 = p_5$, then $Q = (p_1,0,0,0,p_1)$
and the substitution $(a_1,a_2,a_3,a_4,\alpha) = (p_1,0,0,0,\frac{1}{p_1})$ into the
normal form yields $Q$.
If $p_1 \neq p_5$, then the substitution 
$(a_1,a_2,a_3,a_4,\alpha) = 
 (\frac{p_2}{2(p_5-p_1)},
  \frac{p_3}{2(p_5-p_1)},
  \frac{p_4}{2(p_5-p_1)},
  \frac{1}{2}, 2(p_5-p_1))$ into the normal form yields $Q$.
To see this, note that
\[
  \alpha(a_1^2 + a_2^2 + a_3^2 - a_4^2)
= 2(p_5-p_1) \frac{p_2^2 + p_3^2 + p_4^2 - (p_5 - p_1)^2}{(2(p_5-p_1))^2}
= \frac{(p_5^2 - p_1^2) - (p_5 - p_1)^2}{2(p_5 - p_1)}
= \frac{-2p_1^2 + 2p_1p_5}{2(p_5 - p_1)} 
= p_1
\]
where we use the Pythagorean property in the second equation.
Secondly, note that
\[
\alpha (2a_i a_4) = \frac{2p_{i+1}(p_5 - p_1)}{2(p_5 - p_1)} = p_{i+1}
\]
for $i=1,2,3$.
Finally, note that 
\[
\alpha(a_1^2 + a_2^2 + a_3^2 + a_4^2)
= \frac{p_2^2 + p_3^2 + p_4^2 + (p_5 - p_1)^2}{2(p_5-p_1)} 
= \frac{(p_1^2 + p_2^2 + p_3^2 + p_4^2) + p_5^2 -2p_1p_5}{2(p_5-p_1)} 
= \frac{2p_5^2 - 2p_1p_5}{2(p_5-p_1)} 
= p_5
\]
\QED

\Comment{
  OTHER VERSION:
  This formula is very close to a normal form for Pythagorean quintuples.

  \begin{theorem}
    \label{thm:necessary4}
    A quintuple over K is a Pythagorean quintuple if and only it is of the form
    \begin{equation}
      \label{eq:pythquint}
      \frac{1}{\alpha} (a_1^2 + a_2^2 + a_3^2 - a_4^2,
      \ 2a_1a_4,\ 2a_2a_4,\ 2a_3a_4,
      \ a_1^2 + a_2^2 + a_3^2 + a_4^2)
    \end{equation}
    for some $a_1,a_2,a_3,a_4,\alpha\neq 0 \in K$, 
    where $K$ is the ring of integers
    or polynomials over $\Re[x_1,x_2,\ldots,x_n]$.
    % NB: proof not valid for comm. ring, since comm. ring may not have unit elt 1, 
    % which is needed immediately
  \end{theorem}
  \prf
  (If:)
   Suppose that $Q = \frac{1}{\alpha} (a_1^2 + a_2^2 + a_3^2 - a_4^2,
		\ 2a_1a_4,\ 2a_2a_4,\ 2a_3a_4,
		\ a_1^2 + a_2^2 + a_3^2 + a_4^2)$ is a quintuple over $K$.
   By Corollary~\ref{cor:euler}, $\alpha Q$ is a Pythagorean quintuple.
   But any nonzero multiple of a Pythagorean tuple is also a Pythagorean tuple,
   so $Q$ is a Pythagorean quintuple.

   (Only if:) 
   Let $Q = (p_1,p_2,p_3,p_4,p_5)$ be a Pythagorean quintuple over $K$.
   % and let $f(a_1,a_2,a_3,a_4,\alpha)$ be a shorthand for the normal form
   % $\frac{1}{\alpha} (a_1^2 + a_2^2 + a_3^2 - a_4^2,  
   %		 \ 2a_1a_4,\ 2a_2a_4,\ 2a_3a_4,
   %		 \ a_1^2 + a_2^2 + a_3^2 + a_4^2)$.
   If $p_1 = p_5$, then $Q = (p_1,0,0,0,p_1)$
   and the substitution $(a_1,a_2,a_3,a_4,\alpha) = (p_1,0,0,0,p_1)$ into the
   normal form yields $Q$.
   If $p_1 \neq p_5$, then the substitution 
   $(a_1,a_2,a_3,a_4,\alpha) = (p_2,p_3,p_4,p_5 - p_1, 2(p_5 - p_1))$ into the normal
   form yields $Q$.
   To see this, note that
   \[
   \frac{a_1^2 + a_2^2 + a_3^2 - a_4^2}{\alpha}
   = \frac{p_2^2 + p_3^2 + p_4^2 - (p_5 - p_1)^2}{2(p_5-p_1)}
   = \frac{(p_5^2 - p_1^2) - (p_5 - p_1)^2}{2(p_5 - p_1)}
   = \frac{-2p_1^2 + 2p_1p_5}{2(p_5 - p_1)} 
   = p_1
   \]
   where we use the Pythagorean property in the second equation.
   Secondly, note that
   \[
   \frac{2a_i a_4}{\alpha} = \frac{2p_{i+1}(p_5 - p_1)}{2(p_5 - p_1)} = p_{i+1}
   \]
   for $i=1,2,3$.
   Finally, note that 
   \[
   \frac{a_1^2 + a_2^2 + a_3^2 + a_4^2}{\alpha}
   = \frac{p_2^2 + p_3^2 + p_4^2 + (p_5 - p_1)^2}{2(p_5-p_1)} 
   = \frac{(p_1^2 + p_2^2 + p_3^2 + p_4^2) + p_5^2 -2p_1p_5}{2(p_5-p_1)} 
   = \frac{2p_5^2 - 2p_1p_5}{2(p_5-p_1)} 
   = p_5
   \]
   \QED

   Unfortunately, 
   (\ref{eq:pythquint}) is not a sufficient condition for Pythagorean quintuples
   over $\Re[x_1,x_2,\ldots,x_n]$, since
   a quintuple in the form (\ref{eq:pythquint}) is not necessarily
   even a quintuple over $\Re[x_1,x_2,x_3,x_4]$.
   For example, $(a_1,a_2,a_3,a_4,\alpha) = (1,1,1,1,x_1)$ yields the
   quintuple $(\frac{2}{x_1}, \frac{2}{x_1}, \frac{2}{x_1}, \frac{2}{x_1}, \frac{4}{x_1})$,
   which is not ---.
   However, it successfully yields a characterization
   of the rational maps of $\Re^4$ to \Sn{3}.\footnote{The weaker restriction on $\alpha$
     suffices because of the cancellation of $\alpha$ as we translate from Pythagorean
     tuples to rational maps, as illustrated in the proof.}

   It suffices to establish a weaker version,
   with a relaxed restriction on $\alpha$,
   since our goal is not a characterization of Pythagorean tuples {\em per se},
   but a characterization of rational maps to the sphere.
} % \Comment

This leads directly to a characterization of rational maps to \Sn{3}, through the
identification of Pythagorean $n$-tuples and rational maps to the sphere \Sn{n-2}
(Section~\ref{sec:ratpy}).

\begin{corollary}
\label{thm:map4}
The map $f: \Re^m \rightarrow \Sn{3}$ to the 3-sphere is rational if and only if
it is of the following form (up to coordinate permutation):
\begin{equation}
\label{eq:re4s3}
% \footnotesize{(x_{\pi(1)},x_{\pi(2)},x_{\pi(3)},x_{\pi(4)})} \mapsto 
f(x_1,\ldots,x_m) = 
\footnotesize{(\frac{a_1^2 + a_2^2 + a_3^2 - a_4^2}{a_1^2 + a_2^2 + a_3^2 + a_4^2},
	 \frac{2a_1a_4}{a_1^2 + a_2^2 + a_3^2 + a_4^2},
	 \frac{2a_2a_4}{a_1^2 + a_2^2 + a_3^2 + a_4^2},
	 \frac{2a_3a_4}{a_1^2 + a_2^2 + a_3^2 + a_4^2})}
\end{equation}
where $a_1,a_2,a_3,a_4 \in \Re[x_1,\ldots,x_m]$.
% and $\pi : \{1,2,3,4\} \rightarrow \{1,2,3,4\}$ is a permutation.
\end{corollary}
\Comment{
\prf
(Only if): Consider a rational map
of $\Re^m$ to \Sn{3}, $f(x_1,\ldots,x_m) \mapsto 
(\frac{f_1}{f_5},\frac{f_2}{f_5},\frac{f_3}{f_5},\frac{f_4}{f_5})$,
where $f_1,\ldots,f_5 \in \Re[x_1,\ldots,x_m]$.
Then $(f_1,\ldots,f_5)$ is a Pythagorean quintuple and 
can be expressed as some permutation of
$\alpha (a_1^2 + a_2^2 + a_3^2 - a_4^2,
2a_1a_4,2a_2a_4,2a_3a_4,$ $a_1^2 + a_2^2 + a_3^2 + a_4^2)$
for some rational polynomials $a_1,a_2,a_3,a_4,\alpha \neq 0$.
}
% Thus, $(x_{\pi(1)},x_{\pi(2)},x_{\pi(3)},x_{\pi(4)}) \mapsto$
% \[
%	(\frac{\alpha(a_1^2 + a_2^2 + a_3^2 - a_4^2)}{\alpha(a_1^2 + a_2^2 + a_3^2 + a_4^2)},
%	 \frac{\alpha(2a_1a_4)}{\alpha(a_1^2 + a_2^2 + a_3^2 + a_4^2)},
%	 \frac{\alpha(2a_2a_4)}{\alpha(a_1^2 + a_2^2 + a_3^2 + a_4^2)},
%	 \frac{\alpha(2a_3a_4)}{\alpha(a_1^2 + a_2^2 + a_3^2 + a_4^2)})
% \]
% \[
% = 	(\frac{a_1^2 + a_2^2 + a_3^2 - a_4^2}{a_1^2 + a_2^2 + a_3^2 + a_4^2},
%	 \frac{2a_1a_4}{a_1^2 + a_2^2 + a_3^2 + a_4^2},
%	 \frac{2a_2a_4}{a_1^2 + a_2^2 + a_3^2 + a_4^2},
%	 \frac{2a_3a_4}{a_1^2 + a_2^2 + a_3^2 + a_4^2})
% \]
% Notice that it is legal to cancel the $\alpha$, since
% $\alpha = \frac{1}{\beta}$ for some $\beta \in \Re[x_1,x_2,x_3,x_4]$
% which implies that $\alpha$ is never zero.\\
% (If): Consider a map of the form (\ref{eq:re4s3}).
% This is a rational map of $\Re^4$ to \Sn{3}\ by (\ref{eq:suff4})
% and Lemma~\ref{lem:iffpyth}.
% \QED

\subsection{Pythagorean n-tuples}

Circa 1817, Ammei \cite{dickson52} % p. 318
generalized Corollary~\ref{cor:euler}.
We can use Ammei's result 
to generalize the characterization of Pythagorean tuples and rational maps to the sphere.

\begin{defn2}
\label{defn:commring}
A {\bf commutative ring} is a ring in which multiplication is commutative.
\end{defn2}
%
Examples of commutative rings are the integers, the rationals, the reals,
and polynomials in $n$ variables over the integers, rationals, or reals.

\begin{lemma}[Ammei]	% c. 1817
\label{lem:ammei}
\begin{equation}
\label{eq:ammei}
(a_1^2 + \cdots + a_{n-1}^2 - a_n^2)^2 + (2a_1a_n)^2 + \ldots + (2a_{n-1}a_n)^2 
= (a_1^2 + \cdots + a_n^2)^2
\end{equation}
where $a_1,\ldots,a_n$ are elements of any commutative ring, $n \geq 2$.
\end{lemma}

\begin{theorem}
\label{thm:general}
An $n+1$-tuple over K, $n \geq 2$, is a Pythagorean $n+1$-tuple if and only if 
it is of the form 
\begin{equation}
\label{eq:pyth2}
	\alpha (a_1^2 + \ldots + a_{n-1}^2 - a_n^2,
		\ 2a_1a_n,\ldots,\ 2a_{n-1}a_n,
		\ a_1^2 + \ldots + a_n^2)
\end{equation}
for some $a_1,\ldots,a_n,\alpha \neq 0 \in K$,
where $K$ is any commutative ring, such as the field of rational numbers
or rational polynomials.
\end{theorem}
\prf
This proof is completely analogous to the proof of Theorem~\ref{thm:quint},
replacing quintuple by $n+1$-tuple and 5 by $n+1$.
\QED

\begin{corollary}
\label{cor:ratmap}
The map $f: \Re^m \rightarrow \Sn{n-1}$ to the $n-1$-sphere is rational if and only if
it is of the following form (up to coordinate permutation):
\begin{equation}
\label{eq:re4s3}
% \footnotesize{(x_{\pi(1)},x_{\pi(2)},x_{\pi(3)},x_{\pi(4)})} \mapsto 
f(x_1,\ldots,x_m) = 
\footnotesize{(\frac{a_1^2 + \cdots + a_{n-1}^2 - a_n^2}{a_1^2 + \cdots + a_n^2},
	 \frac{2a_1a_n}{a_1^2 + \cdots + a_n^2},
	 \ldots, 
	 \frac{2a_{n-1}a_n}{a_1^2 + \cdots + a_n^2})}
\end{equation}
where $a_1,\ldots,a_n \in \Re[x_1,\ldots,x_m]$.
% and $\pi : \{1,2,3,4\} \rightarrow \{1,2,3,4\}$ is a permutation.
\end{corollary}

\Comment{
This provides a sufficient condition for Pythagorean $(n+1)$-tuples.
We can again establish a necessary condition with a weaker restriction on
$\alpha$.

\begin{theorem}
\label{thm:necessaryn}
Let $n \geq 2$ and $D$ be the integers or the polynomials 
over $\Re[x_1,\ldots,x_n]$.
A Pythagorean $(n+1)$-tuple over $D$ can be expressed in the form
\begin{equation}
\label{eq:pyth2}
	\alpha (a_1^2 + \ldots + a_{n-1}^2 - a_n^2,
		\ 2a_1a_n,\ldots,\ 2a_{n-1}a_n,
		\ a_1^2 + \ldots + a_n^2)
\end{equation}
for some $a_1,\ldots,a_n,\frac{1}{\alpha} \in D$.
\end{theorem}
\prf
This proof is analogous to that of Theorem~\ref{thm:necessary4}.
Let $(p_1,\ldots,p_{n+1})$ be a Pythagorean $(n+1)$-tuple over $D$.
If $p_1 = p_{n+1}$, let 
$(a_1,\ldots,a_n,\alpha) = (p_1,0,\ldots,0,\frac{1}{p_1})$.
Assume without loss of generality that $p_1 \neq p_{n+1}$.
Let 
\[
(a_1,\ldots,a_n,\alpha) = (p_2,\ldots,p_n,p_{n+1}-p_1,
	\frac{1}{2(p_{n+1}-p_1)})
\]
Then $a_1,\ldots,a_n,\alpha$ generate the Pythagorean $(n+1)$-tuple
$(p_1,\ldots,p_{n+1})$ as in (\ref{eq:pyth2}).
In particular,
\[
\alpha (a_1^2 + \ldots + a_{n-1}^2 - a_n^2)
= \frac{p_2^2 + \cdots + p_n^2 - (p_{n+1}-p_1)^2}{2(p_{n+1}-p_1)}
\]
and applying $p_1^2 + \ldots + p_n^2 = p_{n+1}^2$,
\[
= \frac{-2p_1^2 + 2p_1p_{n+1}}{2(p_{n+1} - p_1)}
= p_1
\]
And
\[
\alpha (2a_i a_n) 
= \frac{2p_{i+1}(p_{n+1}-p_1)}{2(p_{n+1}-p_1)}
= p_{i+1}
\]
for $i=1,\ldots,n-1$.
Finally, 
\[ 
\alpha (a_1^2 + \ldots + a_n^2)
= \frac{p_2^2 + \cdots + p_n^2 + (p_{n+1}-p_1)^2}{2(p_{n+1}-p_1)}
= p_{n+1}.
\]
\QED

This leads to our most general result: 
a full characterization of the rational maps of $\Re^n$ to \Sn{n-1}.

\begin{theorem}
\label{thm:nn}
A map is a rational map of $\Re^n$ to \Sn{n-1}, $n \geq 2$, if and only if
it is of the form:
\[
	(x_{\pi(1)},\ldots,x_{\pi(n)}) \mapsto 
	(\frac{a_1^2 + \cdots + a_{n-1}^2 - a_n^2}{a_1^2 + \cdots + a_n^2},
	 \frac{2a_1a_n}{a_1^2 + \cdots + a_n^2},
	 \ldots,
	 \frac{2a_{n-1}a_n}{a_1^2 + \cdots + a_n^2})
\]
where $a_1,\ldots,a_n \in \Re[x_1,\ldots,x_n]$
and $\pi: \{1,\ldots,n\} \rightarrow \{1,\ldots,n\}$ is a permutation.
\end{theorem}
\prf
Identical to Theorem~\ref{thm:map4}'s proof.
\QED
}

Theorem~\ref{thm:general} applies to all $n \geq 2$ and thus represents 
a characterization of the Pythagorean triples, quadruples, and quintuples
that have already been considered.
It agrees with our previous characterizations for $n=2,4$ (Pythagorean triples
and quintuples),
but it yields a different characterization for $n=3$, as can be seen 
by comparing with (\ref{eq:quad}).
This is not a contradiction: any Pythagorean quadruple
may be expressed either in the form of (\ref{eq:quad})
or in the form of Theorem~\ref{thm:general} with $n=3$.
%
It turns out that if we set 
$(a_1,a_2,a_3,a_4) = (\hat{a}_2, \hat{a}_1, 0, \hat{a}_3)$
in the characterization (\ref{eq:quad}),
the characterization of Theorem~\ref{thm:general} for $n=3$ is produced.
Thus, the characterization of Theorem~\ref{thm:general} is stronger than
(\ref{eq:quad}): not only can a Pythagorean quadruple
be put in the form of (\ref{eq:quad}), it can be put in the form
of (\ref{eq:quad}) with $a_3=0$.

\Comment{
We now have a characterization of all rational maps from $\Re^n$ to \Sn{n-1}.
The related findings on Pythagorean $(n+1)$-tuples are reviewed 
in Table~\ref{tab:pyth}.
The second column of this table gives the normal form for Pythagorean
$(n+1)$-tuples for various $n$.
The third column indicates when this normal form was shown to be a sufficient 
condition for an $(n+1)$-tuple to be a Pythagorean $(n+1)$-tuple 
over the integers.
The fourth column indicates when this normal form was shown to be a necessary 
condition for an $(n+1)$-tuple to be a Pythagorean $(n+1)$-tuple 
over the integers.
The fifth column indicates when this normal form was shown to be a necessary
condition for an $(n+1)$-tuple to be a Pythagorean $(n+1)$-tuple 
over polynomial rings.
The results from this paper are not full proofs (and are thus marked
by stars), since $\frac{1}{\alpha} \in D$, not $\alpha \in D$.

\begin{table}[h]
\label{tab:pyth}
\begin{tabular}{|c|c|c|c|c|}
\hline
$n$ & normal form for Pythagorean $(n+1)$-tuples & suff {\cal Z} & nec {\cal Z} & 
\footnotesize{nec $\Re[x_1,\ldots,x_n]$} \\
\hline
$2$ & \tiny{$\alpha(a_1^2 - a_2^2, 2a_1a_2, a_1^2 + a_2^2)$} &
\footnotesize{classical} & 
\footnotesize{classical} & 
\footnotesize{Kubota 1972} \\ 
\hline
$3$ & \tiny{$\alpha (a_1^2+a_2^2-a_3^2-a_4^2, 2(a_1a_3+a_2a_4), 
    2(a_1a_4-a_2a_3), a_1^2+a_2^2+a_3^2+a_4^2)$} &
\footnotesize{Euler 1748} & 
\footnotesize{Catalan 1885} & 
\footnotesize{Dietz 1993} \\ 
\hline
$4$ & \tiny{$\alpha (a_1^2 + a_2^2 + a_3^2 - a_4^2,
		2a_1a_4, 2a_2a_4, 2a_3a_4,
		a_1^2 + a_2^2 + a_3^2 + a_4^2)$} &
\footnotesize{Euler 1748} & 
\footnotesize{$^*$Johnstone 1998} & 
\footnotesize{$^*$Johnstone 1998} \\
\hline
$n$ & \tiny{$\alpha (a_1^2 + \ldots + a_{n-1}^2 - a_n^2,
	2a_1a_n, \ldots, 2a_{n-1}a_n,
	a_1^2 + \ldots + a_n^2)$} & 
\footnotesize{Ammei 1817} & 
\footnotesize{$^*$Johnstone 1998} & 
\footnotesize{$^*$Johnstone 1998} \\ \hline
\end{tabular}
\end{table}
}

\section{Examples}
\label{sec:eg}

Consider some examples of rational maps to the sphere created by the normal form
of Corollary~\ref{cor:ratmap}.
The natural choice of polynomials $(a_1,\ldots,a_n) = (x_1,\ldots,x_n)$ yields a map from
$n$-space to \Sn{n-1}, which we call Euler's map to the sphere:
\begin{equation}
\label{eq:euler}
  M(x_1,\ldots,x_n) = \frac{1}{x_1^2 + \cdots + x_n^2} 
  (x_1^2 + \cdots + x_{n-1}^2 - x_n^2, 2x_1x_n, \ldots, 2x_{n-1}x_n)
\end{equation}
%
%
The map $f: \Re^n \rightarrow \Sn{n}$ defined by the assignment 
$(a_1,\ldots,a_n) = (x_1,\ldots,x_{n-1},1)$ and a permutation:
\[
f(x_1,\ldots,x_n) = \frac{1}{x_1^2 + \cdots + x_{n-1}^2 + 1}
   (2x_1, \ldots, 2x_{n-1}, x_1^2 + \cdots + x_{n-1}^2 - 1)
\]
yields the inverse of the stereographic projection map.
%
Some maps require sophisticated choices for the polynomials in the normal form.
For example, the standard rational parameterization of the 2-sphere \Sn{2}, 
as shown in (\ref{eq:s2param}),
is equivalent to the assignment 
$(a_1,a_2,a_3) = (\frac{2v(1-u^2)}{4u^2+4v^2},
                  \frac{2u(1+v^2)}{4u^2+4v^2},
                  \frac{1}{2})$.
% \alpha = 4u^2+4v^2
These difficult choices for substitution into the normal form may be developed using
the proof of Theorem~\ref{thm:quint} and its generalization.
The following lemma elaborates.

\begin{lemma}
Given a Pythagorean $(n+1)$-tuple $(p_1,\ldots,p_{n+1})$,
the assignment to the normal form of Theorem~\ref{thm:general} that realizes this tuple 
is $(a_1,\ldots,a_n,\alpha) = 
    (\frac{p_2}{2(p_{n+1}-p_1)},\ldots,
     \frac{p_n}{2(p_{n+1}-p_1)},
     \frac{1}{2},
                2(p_{n+1}-p_1) )$
if $p_1 \neq p_{n+1}$,
and $(a_1,\ldots,a_n,\alpha) = (p_1,0,\ldots,0,\frac{1}{p_1})$ if $p_1 = p_{n+1}$.
\end{lemma}
\prf
See the proof of Theorem~\ref{thm:quint}.
\QED

For the rational parameterization of the sphere, the associated Pythagorean tuple is
$(p_1,p_2,p_3,p_4) = (1-u^2-v^2+u^2v^2,\ 2v(1-u^2),\ 2u(1+v^2),\ 1+u^2+v^2+u^2v^2)$.
Notice that the parameter range of the sphere parameterization is not indicated
by the normal form, so the map will be many-to-one.
Examination of the assignment of parameter intervals to the normal form is a topic
for future study.

% \section{Inverse maps}
% Development of general inverse will be very difficult.
% See comment in section on development of inverse of M.
% Practically, we should do it for specific maps only.
% It suffices to establish the poles, which can be done without the inverse maps.

\section{Poles}
\label{sec:pole}

For a rational map from $\Re^m$ to \Sn{n}, 
we expect a manifold of dimension $m-n$ to map to each point of the sphere,
since the Euclidean space simply has more points.
For example, with a map from $\Re^n$ to \Sn{n-1}, the preimage of a typical point
of the sphere is a line.
However, for some anomalous points of the sphere, the dimension of the preimages
is larger than expected, resulting in a poorly defined inverse at this point.
For example, with the Euler map (\ref{eq:euler}) from $\Re^n$ to \Sn{n-1},
the entire hyperplane $x_n=0$ maps to $(1,0,\ldots,0)$:
the preimage of $(1,0,\ldots,0)$ is a hyperplane of dimension $n-1$ rather than a line
of dimension one.
This is not a vagary of the Euler map: it is a general property of these
rational maps to the sphere.
If we consider the normal form (\ref{cor:ratmap}) for rational maps to the sphere,
the entire hypersurface $a_n(x_1,\ldots,x_m)=0$ will map to the same point,
$(1,0,\ldots,0)$.
This is a manifold of dimension $m-1$ rather than the expected manifold of dimension
$m-n+1$.
These points lead to poles in the associated inverse map that maps from the sphere
to Euclidean space.

Points of the sphere that have more preimages than they should are present even in maps
built from the normal form with $a_n = k \neq 0$.
For example, consider the map (\ref{eq:s2param})
associated with the rational parameterization of \Sn{2},
where the normal form for this rational map has $a_n = \frac{1}{2}$
(see Section~\ref{sec:eg}).
This map has an anomaly at $(0,0,1)$:
the entire hyperplane $u=1$ maps to (0,0,1),\footnote{This 
  is the part of the parameterization associated with the circle of radius 0 in 
  the sweeping of a circle using the spherical product.  
  Alternatively, it is the degenerate line of latitude.
  There is another pole at the other end of the sphere.}
a manifold of dimension one rather than the expected manifold of dimension zero 
for this map of $\Re^2$ to \Sn{2}.

The accumulation of preimages at certain points is not surprising.
It is an artifact of wrapping a Euclidean space about the sphere.
It is well known that mappings of the sphere to Euclidean space have these poles,
which have long been the bane of cartographers as they map the spherical earth to
the flat page.

\Comment{
\section{Visualization}
\label{sec:viz}

The rational maps to the sphere generated by our characterization may be compared
using a visualization technique that measures how they spread Euclidean space
across the sphere.
This is most easily done for \Sn{2}, so we will concentrate the visualization there.
However, the visualization of higher dimensions \cite{brown mathematician}
could be used to consider other rational maps.

There are two techniques.
In the first, we sample some region of Euclidean m-space uniformly 
and visualize the number of samples that are mapped to each area of \Sn{n}
(the more samples, the lighter).
Recall that typically many points of Euclidean space will map to a given point of 
the sphere (Section~\ref{sec:pole}).
This technique is particularly useful if it is known how an application restricts 
the domain,
which may be used to choose the region of Euclidean space that is considered.
We do this for 3-space and 2-space (the hyperplane $z=0$ in 3-space).
Some applications are only interested in mapping the n-sphere back to itself: 
the domain is restricted to \Sn{n}.
(For example, the application of these maps to quaternion splines is sometimes
restricted to maps of \Sn{3} to \Sn{3}.)
In the second visualization, we sample \Sn{n} and visualize the distance that 
each sample moves.
This is a local measure of the map, but still evocative of distinctions between
the maps.

RESURRECT MAP TEXTURES AND IMPORT PROSE, IMAGES AND SOFTWARE.

The maps that we consider are the following classical products of the characterization:
1) Euler spherical map
2) inverse of stereographic projection
3) standard rational parameterization of sphere
4) fancier one
5) Dietz map?
}

\section{Conclusions}

START HERE

An essential observation in the characterization of rational maps to the sphere 
is the equivalence between rational maps to the sphere and Pythagorean tuples.

Pythagorean quintuples are not as well understood as Pythagorean triples
and quadruples, and our study leads to some new characterizations.
% (Theorems~\ref{thm:necessary4} and \ref{thm:map4}).

\begin{rmk}
Euler's Four Squares Theorem also anticipates the normal form
for Pythagorean quadruples (Section~{sec:prevwork}):
\begin{equation}
\label{eq:euler1}
(a_1^2 + a_2^2 + a_3^2 + a_4^2)^2 = 
(a_1^2 + a_2^2 - a_3^2 - a_4^2)^2 + (2a_1a_3+2a_2a_4)^2 + (2a_1a_4-2a_2a_3)^2
\end{equation}
using $(a_1,a_2,a_3,a_4) = (\hat{a}_1,-\hat{a}_2,\hat{a}_3,\hat{a}_4)$ in 
Theorem~\ref{thm:euler}.
\end{rmk}

Euler's 4 squares theorem anticipates the quaternion ({\bf elaborate})...,
which is the subject of the natural follow-on to this paper, 
to the application of rational maps in the design of quaternion splines
for motion control.

\begin{rmk}
Euler's Four Squares Theorem anticipates
the quaternion (a century in advance of its invention by Hamilton),
in the sense that it encodes the product formula for quaternions,
perhaps the most defining characteristic of the quaternion:
% (compare Lemma~\ref{lem:euler}):
\[
\begin{array}{ll}
& (a_1 + a_2 i + a_3 j + a_4 k) 
(\hat{a}_1 + \hat{a}_2 i + \hat{a}_3 j + \hat{a}_4 k) = \\
& (a_1 \hat{a}_1 - a_2\hat{a}_2 - a_3\hat{a}_3 - a_4\hat{a}_4) +
   (a_1\hat{a}_2 + a_2\hat{a}_1 + a_3\hat{a}_4 - a_4\hat{a}_3) i + \\
&  (a_1\hat{a}_3 - a_2\hat{a}_4 + a_3\hat{a}_1 + a_4\hat{a}_2) j +
   (a_1\hat{a}_4 + a_2\hat{a}_3 - a_3\hat{a}_2 + a_4\hat{a}_1) k
\end{array}
\]
Thus, it is quite natural to use Euler's Four Squares Theorem
to characterize Pythagorean tuples (through generalization) and rational maps, 
which are then used to build quaternion splines.
\end{rmk}

\bibliographystyle{plain}
\begin{thebibliography}{99}

\bibitem{barr81}
Barr, A. (1981) Superquadrics and Angle-Preserving Transformations.
IEEE Computer Graphics and Applications.

\bibitem{barr92}
Barr, A. and B. Currin and S. Gabriel and J. Hughes (1992)
Smooth Interpolation of Orientations with Angular Velocity Constraints
using Quaternions.
SIGGRAPH '92, 313--320.

\bibitem{catalan85}
Catalan, E. (1885) Bull. Acad. Roy. Belgique 3(9), p. 531.
Referenced in Dickson, L.E. (1952) History of the Theory of Numbers: Volume II,
Diophantine Analysis.  Chelsea (New York), p. 269.

\bibitem{dickson52}
Dickson, L.E. (1952) History of the Theory of Numbers: Volume II,
Diophantine Analysis.  Chelsea (New York).

\bibitem{dietz93}
Dietz, R. and J. Hoschek and B. Juttler (1993)
An algebraic approach to curves and surfaces on the sphere and
on other quadrics.
Computer Aided Geometric Design 10, 211-229.

\bibitem{euler48}
Fuss, P., editor (1843) Corresp. Math. et Phys.,
`Correspondance entre Leonhard Euler et C. Goldbach 1729-1763',
St. Petersburg, Vol. 1, p. 452.  
Referenced in Dickson, L.E. (1952) History of the Theory of Numbers: Volume II,
Diophantine Analysis.  Chelsea (New York), p. 277.
% see p. 209, Ebbinghaus, Numbers

\bibitem{farouki90}
Farouki, R. and T. Sakkalis (1990)
Pythagorean Hodographs.
IBM J. Res. Develop. 34, 736--752.

\bibitem{herstein75}
Herstein, I. (1975) Topics in Algebra.
2nd edition, John Wiley (New York).

\bibitem{kreyszig59}
Kreyszig, E. (1959) Differential Geometry.
Dover (New York).

\bibitem{kubota72}
Kubota, K. (1972) Pythagorean triples in unique factorization domains.
American Mathematical Monthly 79, 503--505.

\bibitem{weil84}
Weil, A. (1984) Number Theory.
Birkh\"auser (Boston).

\end{thebibliography}

In technical report version, we could add my proof of Kubota's result, 
as in Appendix of characterization.ps.

\clearpage

We can characterize ALL of the rational maps to the sphere from first principles.
(BUT FROM WHAT N-SPACE? WE HAVE CHARACTERIZED N-SPACE BUT WHAT ABOUT N-1-SPACE 
FOR STEREOGRAPHIC PROJECTION?)

See 'A Characterization of Rational Maps of Rn to $S^{n-1}$ for Riemannian Modeling'
and its file characterization.tex in 1-char.  (print it out, and the short version)

See 'The most natural map to the sphere' for stereographic projection.

See nonrational.ps for nonrational maps to the sphere and 
related quadratic spherical maps.

% from hermiteQ.tex
{\bf Exposure sheet}

Maps to the sphere.
\[
	M(x_1,x_2,x_3,x_4) =
	\frac{1}{x_1^2 + x_2^2 + x_3^2 + x_4^2}
	(x_1^2 + x_2^2 + x_3^2 - x_4^2, 2x_1x_4, 2x_2x_4, 2x_3x_4)
\]
\[
SP^{-1}(x_1,x_2,x_3) = \frac{1}{x_1^2 + x_2^2 + x_3^2 + 1} 
                       (2x_1, 2x_2, 2x_3, x_1^2+x_2^2+x_3^2 - 1)
\]

Maps away from the sphere.
\[
M^{-1}|_{S3}(x_1,x_2,x_3,x_4) = \frac{1}{\sqrt{2(1-x_1)}}(x_2,x_3,x_4,1-x_1)
\]
\[
SP(x_1,x_2,x_3,x_4) = \frac{1}{1-x_4} (x_1,x_2,x_3)
\]

\section{Spherical introduction}

In the classic 'Flatland',
Edwin Abbott Abbott (sic!) related the ominous appearance of The Sphere in the
cosy confines of two-space.
This was Abbott's way to introduce the concept of higher dimensions.
We will discuss the sphere one dimension higher, the 3-sphere in 4-space.
Cartographers have long wrestled with the problem of mapping the spherical earth
to a flat page.
{\bf The sphere has long been a Platonic ideal for shape.
Indeed, all of the Platonic solids\footnote{The Platonic solids are the tetrahedron,
  cube, octahedron, dodecahedron and icosahedron.}
aspire to the sphere, with differing levels of success.\footnote{Geodesation can
  help them to achieve this aspiration.}
}
We are interested in maps to and from the sphere, 
for reasons both mundane (their application to the design of quaternion splines)
and aesthetic.
One of the most burning questions is: why must they always have poles?

A surface's parameterization may be interpreted as a map to the surface,
since it maps some parameter space to the surface.
For example, the parameterization of a 2-manifold S is a map from $\Re^2$ to S.
The inverse of this parameterization yields a map from the surface.

\subsection{Sphere}

In the case of the sphere, the classical parameterization is
\[
   () x () = 
\]
which can be viewed as a spherical product of a circle and half-circle \cite{barr}.
Unfortunately, this map is not rational.
A rational parameterization is 
\[
   () x () = 
\]
Both of these parameterizations have poles, since the circle shrinks to a point
at the endpoint of the half-circle.
This is reflected in the earth's lines of latitude: the 90th line of latitude at the
North (or South) Pole is degenerate.
Other choices of map to the sphere, such as the Euler map or 
stereographic projection, suffer from poles, as one maps away from the sphere.
This suggests that poles may be an inherent property of maps from the sphere,
certainly of rational maps.
But are they?

\section{Preamble to 'Rational maps to the sphere' for Orientation control paper}

A key component of the proposed solution is a rational map to the sphere
and its companion map off of the sphere.
The latter map, from $S^3$ to some Euclidean $n$-space, frees the quaternions from
their surface constraint, while the former map reimposes this constraint.
Since the map from the sphere is applied to a finite pointset, while the map
to the sphere is applied to a polynomial curve, rationality is only necessary for the
map to the sphere, to preserve rationality of the curve, yielding a rational
quaternion spline.
If the user is willing to settle for nonrational curves,
a nonrational map could be used,
but rational Bezier curves offer many advantages, including simplicity,
efficiency, and pleasant properties such as the variation-diminishing and convex-hull
properties.

A map between $S^3$ and $\Re^3$ is more appropriate than a map between $S^3$ and $\Re^4$
for the inverse map.
The inverse will be unique, while in the latter case the map from $S^3$ to $\Re^4$ is
inherently one-to-many, since we are moving from a 3-manifold to a 4-manifold.

\section{Rational maps to the sphere}
% $S^3$ maps, to and fro

[Note: The only published paper is GI and Eurographics.
We have moved beyond these papers by at least the 30\% guideline.
This material is valid for a journal publication, but regardless we want a resolution
of this issue in a technical report.
Additional material: stereographic projection, image of rational Bezier,
generalized construction, visualization.]

In this paper, we consider rational maps to the 3-sphere and their inverses.
Since the inverse maps often (always?) have poles, we explore this issue as well.
Derivatives are computed when the maps are applied to a parametric curve on 
the 3-sphere.
[It is hard to motivate the interest in this derivative result, so this should
be delayed to another paper.]
The image of a Bezier curve under these maps, expressed as a rational Bezier curve,
is computed, which allows any Bezier curve to be projected onto the sphere.
(Can we compute the image of a rational Bezier curve too? This would be additional.)
We discuss the generalization to construction of maps to other surfaces.

\subsection{Stereographic projection}
\label{sec:stereo}

The most well-known map involving the sphere is stereographic projection.
It has been used at least since Hipparch in 160 B.C. \cite{kreyszig59}. % p. 205
% probably discovered by Hipparch too
Stereographic projection maps from \Sn{n}\ to the hyperplane $x_{n+1}=0$ in $(n+1)$-space.
A point of \Sn{n}\ is perspectively projected from 
the north pole of \Sn{n}\ to $x_{n+1}=0$  % \cite{thorpe79}. % or kreyszig59
(Figure~\ref{fig:stereo}).
% Note that there is a variant of stereographic projection where
% the projection plane $z=0$ is replaced by the tangent plane opposite the pole
% $z=-1$ \cite{kreyszig59}.
The image of the north pole $(0,\ldots,0,1)$ is undefined.\footnote{It is well defined
  in projective space, where the pole maps to the line at infinity of the hyperplane.}
Stereographic projection is a map away from the sphere, 
but its inverse is a rational map to the sphere.

\begin{figure}[ht]
\begin{center}
\includegraphics[scale=.25]{img/fig:stereo.jpg}
\end{center}
\caption{Stereographic projection in 3-space}
\label{fig:stereo}
\end{figure}

\begin{lemma}
Stereographic projection 
$f_{SP}: \Sn{n} - (0,\ldots,0,1) \rightarrow x_{n+1}=0 \subset \Re^{n+1}$ 
from the sphere
and its inverse $f_{SP}^{-1}: x_{n+1}=0 \rightarrow \Sn{n} - (0,\ldots,0,1)$ 
to the sphere
are defined by:
\begin{eqnarray*}
f_{SP}(x_1,\ldots,x_{n+1}) & = & \frac{1}{1-x_{n+1}} (x_1,\ldots,x_n,0) \\
f_{SP}^{-1}(x_1,\ldots,x_n,0) & = &
	\frac{1}{x_1^2 + \cdots + x_n^2 + 1} 
	(2x_1, \ldots, 2x_n, x_1^2 + \cdots + x_n^2 - 1)
\end{eqnarray*}	% see thorpe79, p. 125
\end{lemma}
\vspace{.1in}
\prf
The projector line $(1-t)p + tq$ through $p = (0,\ldots,0,1)$ and 
$q = (x_1,\ldots,x_{n+1})$ intersects $x_{n+1}=0$ 
when $1-t + tx_{n+1} = 0$ or $t = \frac{1}{1-x_{n+1}}$.
This defines the forward map.
Since stereographic projection is one-to-one and onto,
it has an inverse that is also one-to-one.
The projector line $(1-t)p + t(r,0)$ through $p = (0,\ldots,0,1)$ and $(r,0)$
intersects \Sn{n} when $\|(1-t)(0,\ldots,0,1) + t(r,0)\|_2^2 = 1$,
or $(1-t)^2 + t^2\|r\|_2^2 = 1$, or $t^2(\|r\|^2 + 1) = 2t$, or 
$t=0$ and $t=\frac{2}{\|r\|^2 + 1}$.
The second root yields the inverse point.
\QED

Some other properties of the stereographic projection are interesting, although
more relevant to cartography than to our study.
It is a conformal map and, like any conformal map of \Sn{2}\ to the plane,
it maps loxodromes (curves of constant direction on the sphere) to straight lines.
It is the only conformal map from $S^2$ to a plane that preserves
circles (circles are mapped to circles or straight lines).

\begin{lemma}
Let $C(t) = (x_1(t),x_2(t),x_3(t),x_4(t))$ be a parametric curve that lies on $S^3$.
The first derivative of the stereographic projection of the curve $C(t)$ is 
\[
(\frac{x^{'2}_1(1-x_4) + x_1x'_1x'_4}{(1-x_4)^2},\ 
                \frac{x^{'2}_2(1-x_4) + x_2x'_2x'_4}{(1-x_4)^2},\  
                  \frac{x^{'2}_3(1-x_4) + x_3x'_3x'_4}{(1-x_4)^2}, 0)
\]
\end{lemma}
\prf
\begin{eqnarray*}
(f_{SP}(C(t))' & = & f_{SP}'(C(t)) \cdot C'(t) \\
           & = & (\frac{x'_1(1-x_4) + x_1x'_4}{(1-x_4)^2},\ 
                  \frac{x'_2(1-x_4) + x_2x'_4}{(1-x_4)^2},\  
                  \frac{x'_3(1-x_4) + x_3x'_4}{(1-x_4)^2}, 0) 
		  \cdot (x'_1, x'_2, x'_3, x'_4) \\
           & = & (\frac{x^{'2}_1(1-x_4) + x_1x'_1x'_4}{(1-x_4)^2},\ 
                  \frac{x^{'2}_2(1-x_4) + x_2x'_2x'_4}{(1-x_4)^2},\  
                  \frac{x^{'2}_3(1-x_4) + x_3x'_3x'_4}{(1-x_4)^2}, 0)
\end{eqnarray*}
\QED

Higher derivatives are found simply by repeating this process.

Since $x_{n+1}=0$ is an embedding of $n$-space in $\Re^{n+1}$,
stereographic projection and its inverse may be interpreted as maps between 
$\Sn{n}$ and $\Re^n$.

DEFINE BEZIER CURVE, BERNSTEIN POLYNOMIALS, PROJECTIVE SPACE.
REVIEW RATIONAL BEZIER CURVE AND BEZIER CURVE IN PROJECTIVE SPACE,
PRODUCT RULE FOR BERNSTEIN POLYNOMIALS.

HOW ABOUT ARBITRARY BEZIER CURVE?

\begin{theorem}
\label{thm:imagecurve}
The image under inverse stereographic projection 
of a cubic Bezier curve with control points $b_i = (b_{i1},\ldots,b_{i4})$
is a sextic rational Bezier curve 
with control points $\breve{b}_i = $ 
and weights $w_i = $.
This maps an arbitrary cubic curve to a curve on \Sn{3}.
\end{theorem}
\prf
Let $c(t) = \sum_{i=0}^3 b_i B_i^3(t)$ be the cubic Bezier curve.
Since the image of $c(t)$ is a rational Bezier curve,
it is easier to work in projective space.
% When $c(t)$ is expressed in projective space, it becomes
% \[
% \sum_{i=0}^3 (b_{i1},b_{i2},b_{i3},b_{i4}, 1) B_i^3(t)
% \]
In projective space, inverse perspective projection becomes
\[
f^{-1}_{SP}(x_1,x_2,x_3,0,1) = (2x_1, 2x_2, 2x_3,\ x_1^2 + x_2^2 + x_3^2 - 1, 
                                \ x_1^2 + x_2^2 + x_3^2 + 1)
\]
The fourth coordinate of the image curve may be expanded using the product rule
for Bernstein polynomials:
\begin{eqnarray*}
& [\sum_{i=0}^3 B_i^3(t) b_{i1}]^2 + 
[\sum_{i=0}^3 B_i^3(t) b_{i2}]^2 + 
[\sum_{i=0}^3 B_i^3(t) b_{i3}]^2 - 1 \\
= & \sum_{i=0}^3 \sum_{j=0}^3 
	\frac{\choice{3}{i} \choice{3}{j}}{\choice{6}{i+j}}
       B^6_{i+j}(t) (b_{i1} b_{j1} + b_{i2} b_{j2} + b_{i3} b_{j3} - 1) 
\end{eqnarray*}
Letting $k=i+j$, this becomes
\[ \sum_{k=0}^6 B_k^6(t) 
	\sum_{\begin{array}{c}  \mbox{\footnotesize{$0 \leq i \leq 3$}} \\ 
	                        \mbox{\footnotesize{$0 \leq j \leq 3$}} \\ 
			        \mbox{\footnotesize{$i+j=k$}}
			     \end{array}} 
	\frac{\scriptchoice{3}{i} \scriptchoice{3}{j}}{\scriptchoice{6}{k}}
	(b_{i1} b_{j1} + b_{i2} b_{j2} + b_{i3} b_{j3} - 1) 
\]
The fifth coordinate may be expanded analogously, yielding the weights
\[
	\sum_{\begin{array}{c}  \mbox{\footnotesize{$0 \leq i \leq 3$}} \\ 
	                        \mbox{\footnotesize{$0 \leq j \leq 3$}} \\ 
			        \mbox{\footnotesize{$i+j=k$}}
			     \end{array}} 
	\frac{\scriptchoice{3}{i} \scriptchoice{3}{j}}{\scriptchoice{6}{k}}
	(b_{i1} b_{j1} + b_{i2} b_{j2} + b_{i3} b_{j3} + 1) 
\]
The first, second, and third coordinates must be degree elevated,
from degree 3 to 6.
START HERE BY ADDRESSING DEGREE ELEVATION
\QED

% image under stereographic projection
\Comment{
\prf
Let $c(t) = \sum_{i=0}^3 b_i B_i^3(t)$ be the cubic Bezier curve.
Since the image of $c(t)$ is a rational Bezier curve,
it is easier to work in projective space.
% When $c(t)$ is expressed in projective space, it becomes
% \[
% \sum_{i=0}^3 (b_{i1},b_{i2},b_{i3},b_{i4}, 1) B_i^3(t)
% \]
Perspective projection becomes
\[
f_{SP}(x_1,x_2,x_3,x_4,1) = (x_1,x_2,x_3,0,1-x_4)
\]
% Let the image curve in projective space be $f_{SP}(c(t)) = (f_1(t),\ldots,f_5(t))$.
Note that 
\[
1 - \sum_{i=0}^3 b_{i4} B_i^3(t) = \sum_{i=0}^3 B_i^3(t) - \sum_{i=0}^3 b_{i4} B_i^3(t)
= \sum_{i=0}^3 (1-b_{i4})B_i^3(t)
\]
using the partition of unity property of Bernstein polynomials.
Therefore, 
\[
f_{SP}(c(t)) = \sum_{i=0}^3 (b_{i1}, b_{i2}, b_{i3}, 0, 1-b_{i4}) B_i^3(t)
\]
This is a cubic rational Bezier curve with weights $w_i = 1-b_{i4}$
and control points $\frac{1}{w_i} (b_{i1}, b_{i2}, b_{i3}, 0)$.
\QED
}

[Be careful using stereographic projection, since the conclusions of comp.tex are that
stereographic projection leads to an inferior quaternion spline.
Doesn't this imply that we have implemented the map of a Bezier curve under
stereographic projection too?]

CREATE BIB FILE.

\subsection{Development of Euler's map}

Develop from first principles using 4 squares theorem.
See ratqspline.tex and characterization.tex(?).
Show that all maps to sphere are versions of 4 squares (but then what about stereographic
projection?).

\subsection{Euler's map}

\begin{defn2}
The Euler map $M: \Re^4 - \{0\} \rightarrow \Sn{3}$ to the sphere is defined by:
\begin{equation}
\label{eqM}
	M(x_1,x_2,x_3,x_4) =
	\frac{1}{x_1^2 + x_2^2 + x_3^2 + x_4^2}
	(x_1^2 + x_2^2 + x_3^2 - x_4^2, 2x_1x_4, 2x_2x_4, 2x_3x_4)
\end{equation}
%
If we let 
${\cal M}(p,q): = (p_1q_1 + p_2q_2 + p_3q_3 - p_4q_4, 2p_1q_4, 2p_2q_4, 2p_3q_4)$,
the Euler map may be re-expressed as:
\begin{equation}
\label{eq:calM}
M(P) = \frac{{\cal M}(P,P)}{P \cdot P}
\end{equation}
which is valuable in Theorem~\ref{thm:imagecurve} on the image of a Bezier curve
under the Euler map.
\end{defn2}

The Euler map has both bilinear and quadratic characteristics.
In Euclidean space, every coordinate of ${\cal M}$ is a bilinear map,
while in projective space, every coordinate of $M$ is a quadratic form.

Since the Euler map transforms a 4-space into a 3-manifold, losing a dimension,
the inverse of a point is a 1-manifold.

\begin{lemma}
\label{lem:inverse}
The inverse Euler map $M^{-1}: \Sn{3} - (1,0,0,0) \rightarrow \Re^4$ from the sphere
is defined by:
\begin{equation}
M^{-1}(x_1,x_2,x_3,x_4) = (1-t)(0,0,0,0) + t(x_2,x_3,x_4,1-x_1), t \neq 0
\end{equation}
In other words, the image of a point is a line through the origin.
The map has a special behaviour at the pole $(1,0,0,0)$:
\begin{equation}
M^{-1}(1,0,0,0) = \mbox{ the hyperplane $x_4=0$ without the origin}
\end{equation}
\end{lemma}
\prf
\QED

It is often desirable for the inverse of a point to be a unique point.
In this case, it is possible to choose a particular point on the inverse line.
A natural choice is to choose the point on \Sn{3}, since every inverse line must intersect
\Sn{3} and the choice of \Sn{3} adds a certain symmetry.
This leads to the following restricted inverse map.

\begin{lemma}
The restricted inverse Euler map $M^{-1}_{S3}: S^3 - \{(1,0,0,0)\} \rightarrow S^3$,
defined by $M^{-1}_{S3}(P) = M^{-1}(P) \cap S^3$, has the form
\begin{equation}
(x_1,x_2,x_3,x_4) \mapsto \frac{1}{\sqrt{2-2x_1}}(x_2,x_3,x_4,1-x_1)
\end{equation}
\end{lemma}
\prf
An intersection of the inverse line with $S^3$
is the point $\frac{(x_2,x_3,x_4,1-x_1)}{\|x_2,x_3,x_4,1-x_1\|_2}$.
\[
\|(x_2,x_3,x_4,1-x_1)\|_2^2 = x_2^2 + x_3^2 + x_4^2 + (1-x_1)^2
= x_1^2 + x_2^2 + x_3^2 + x_4^2 + 1 - 2x_1
\]
Since $(x_1,x_2,x_3,x_4) \in S^3$,
\[
\|(x_2,x_3,x_4,1-x_1)\|_2^2 = 2 - 2x_1
\]
\QED

% It is a map from $S^3$ to $S^3$.
% This is not a rational map, but it need not be.

\begin{lemma}
Let $C(t) = (x_1(t),x_2(t),x_3(t),x_4(t))$ be a parametric curve that lies on \Sn{3}.
The restricted inverse Euler map of $C(t)$ is another curve on \Sn{3}.
The first derivative of this curve $M^{-1}|_{S3}(C(t))$ is:
\[
(\frac{x'_2}{(2-2x_1)^{1/2}} + \frac{x'_1x_2}{(2-2x_1)^{3/2}}, 
 \frac{x'_3}{(2-2x_1)^{1/2}} + \frac{x'_1x_3}{(2-2x_1)^{3/2}},
 \frac{x'_4}{(2-2x_1)^{1/2}} + \frac{x'_1x_4}{(2-2x_1)^{3/2}},
 \frac{-x'_1}{2\sqrt{2-2x_1}})
\]
\end{lemma}
\prf
\begin{eqnarray*}
(M^{-1}(C(t))' & = & (M^{-1})'(C(t)) \cdot C'(t)
\end{eqnarray*}
Consider the first coordinate of $(M^{-1})'(C(t))$.
The second and third coordinates are computed analogously.
Recall that the restricted inverse of the Euler map is
\[
M^{-1}|_{S3}(C(t)) = \frac{(x_2,x_3,x_4,1-x_1)}{\sqrt{2-2x_1}}
\]
If we let $g(t) = (2-2x_1(t))^{\frac{1}{2}}$, then
\[
g'(t) = \frac{-x'_1}{\sqrt{2-2x_1}}
\]
and it is clear that the first coordinate of $(M^{-1})'(C(t))$ is:
\[
\frac{x'_2 \sqrt{2-2x_1} + \frac{x'_1 x_2}{\sqrt{2-2x_1}}}{2-2x_1}
= \frac{x'_2}{(2-2x_1)^{1/2}} + \frac{x'_1x_2}{(2-2x_1)^{3/2}}
% = \frac{x'_2(2-2x_1) + x'_1x_2}{(2-2x_1)^{3/2}}
% = \frac{x'_1x_2 - 2x_1x'_2 + 2x'_2}{(2-2x_1)^{3/2}}
\]
The fourth coordinate of $M^{-1}(C(t))$ is 
$\frac{1-x_1}{\sqrt{2-2x_1}} = (\frac{1-x_1}{2})^{1/2}$
and its derivative is:
\[
\frac{1}{2} (\frac{1-x_1}{2})^{-1/2}(\frac{-x'_1}{2})
= \frac{-x'_1}{2\sqrt{2-2x_1}}
\]
This yields the stated derivative.
% Collecting, the derivative of the inverse Euler map is:
% \[
% \frac{1}{\sqrt{2-2x_1}} 
% (\frac{x'_1x_2 - 2x_1x'_2 + 2x'_2}{2-2x_1},
%  \frac{x'_1x_3 - 2x_1x'_3 + 2x'_3}{2-2x_1},
%  \frac{x'_1x_4 - 2x_1x'_4 + 2x'_4}{2-2x_1},
% -\frac{x'_1}{2})
% \]
\QED

Higher derivatives are found simply by repeating this process. 
DO SO IN MAPLE.

\begin{theorem}
\label{thm:imagecurve}
The image of the cubic Bezier curve $\sum_{i=0}^3 b_i B_i^3(t)$ under M 
is the sextic rational Bezier curve $\frac{\sum_{i=0}^6 w_i \breve{b}_i B_i^6(t)}
                                          {\sum_{i=0}^6 w_i B_i^6(t)}$ 
with control points:
\begin{eqnarray}
\breve{b}_0 & = & \frac{{\cal M}(b_0,b_0)}{b_0 \cdot b_0} \nonumber \\
\breve{b}_1 & = & \frac{{\cal M}(b_0,b_1) + {\cal M}(b_1,b_0)}{2b_0\cdot b_1} \nonumber \\
\breve{b}_2 & = & \frac{{\cal M}(b_0,b_2) + 3{\cal M}(b_1,b_1) + {\cal M}(b_2,b_0)}{2b_0 \cdot b_2 + 3b_1 \cdot b_1} \nonumber \\
\breve{b}_3 & = & \frac{{\cal M}(b_0,b_3) + 9{\cal M}(b_1,b_2) + 9{\cal M}(b_2,b_1) + {\cal M}(b_3,b_0)}{2b_0 \cdot b_3 + 18 b_1 \cdot b_2} \nonumber \\
\breve{b}_4 & = & \frac{{\cal M}(b_1,b_3) + 3{\cal M}(b_2,b_2) + {\cal M}(b_3,b_1)}{2b_1 \cdot b_3 + 3 b_2 \cdot b_2} \nonumber \\
\breve{b}_5 & = & \frac{{\cal M}(b_2,b_3) + {\cal M}(b_3,b_2)}{2b_2 \cdot b_3} \nonumber \\
\breve{b}_6 & = & \frac{{\cal M}(b_3,b_3)}{b_3 \cdot b_3} \nonumber
\end{eqnarray}
and weights:
\begin{eqnarray}
w_0 & = & b_0 \cdot b_0 \nonumber \\
w_1 & = & b_0 \cdot b_1 \nonumber \\
w_2 & = & \frac{2b_0 \cdot b_2 + 3b_1 \cdot b_1}{5} \nonumber \\
w_3 & = & \frac{b_0 \cdot b_3 + 9b_1 \cdot b_2}{10} \nonumber \\
w_4 & = & \frac{2b_3 \cdot b_1 + 3b_2 \cdot b_2}{5} \nonumber \\
w_5 & = & b_3 \cdot b_2 \nonumber \\
w_6 & = & b_3 \cdot b_3 \nonumber
\end{eqnarray}
Notice that ${\cal M}$ is the bilinear map of (\ref{eq:calM}).
\end{theorem}
\prf See the appendix.
\QED

\subsection{Appendix: proof of Euler map's image of a Bezier curve}
\label{sec:appendix}

\subsubsection{Bezier derivatives}

If $b(t)$ is a rational Bezier curve of degree $n$ 
with control points $\{b_i\}_{i=0}^n$ and 
weights $\{w_i\}_{i=0}^n$ defined over $t \in [0,1]$, then 
\begin{eqnarray}
\label{eqn:firstderiv}
b'(0) & = & n \frac{w_1}{w_0} (b_1 - b_0) \\
\label{eqn:secondderiv}
b''(0) & = & \frac{n-1}{n} \frac{w_0 w_2}{w_1^2} 
             \frac{\triangle b_0 \times \triangle b_1}{\| \triangle b_0\|^3}
\end{eqnarray}
where $\triangle b_i = b_{i+1} - b_i$.
See Farin \cite{farin02} and Hoschek/Lasser \cite{hoschekLasser0?}. 
% p. 244 of HL has conditions for curvature continuity
Notice the relationship of second derivative to curvature $\kappa$:
\[
    \kappa = \|b''\|
\]
if $b$ is arc-length parameterized, and in general,
\[
    \kappa = \frac{\| b' \times b'' \|}{\| b' \|^3}
\]
$C^2$ continuity guarantees curvature continuity.
See Farin \cite{farin02} or a differential geometry text.
% the formula for curvature of a rational Bezier curve is 
% k = \frac{n-1}{n} \frac{w_0 w_2}{w_1^2} 
%     \frac{\mbox{dist}(b_0,\lyne{b_1 - b_0}}{\mbox{dist}(b_0,b_1)^2}

The product rule of Bernstein polynomials \cite{farin02} states:
\[
B_i^m(t) B_j^n(t) = \frac{\choice{m}{i} \choice{n}{j}}{\choice{m+n}{i+j}} B_{i+j}^{m+n}(t)
\]

\subsubsection{Proof of Theorem~\ref{thm:imagecurve}}

\prf
Let $c(t)$ be the cubic Bezier curve.
Since the image of $c(t)$ is a rational Bezier curve, it is easier 
to work in projective space.
In projective space, the map $M$ becomes
\[
	 (x_1,x_2,x_3,x_4,x_5) \rightarrow
	 (x_1^2 + x_2^2 + x_3^2 - x_4^2,\ 
	 2x_1 x_4,\ 2x_2 x_4,\ 2x_3 x_4,\ 
	 x_1^2 + x_2^2 + x_3^2 + x_4^2).
\]
Let the image curve be expressed in projective space: 
$M(c(t)) = (m_1(t),m_2(t),m_3(t),m_4(t),m_5(t))$.
Each coordinate $m_i(t)$ may be simplified using the product rule 
of Bernstein polynomials \cite{farin02};
consider $m_5(t)$ as an example.
\begin{eqnarray}
m_5(t) & = & [\sum_{i=0}^3 B_i^3(t) b_{i1}]^2 + 
	\ldots + [\sum_{i=0}^3 B_i^3(t) b_{i4}]^2 \nonumber \\
     & = &  \sum_{i=0}^3 \sum_{j=0}^3 
	\frac{\choice{3}{i} \choice{3}{j}}{\choice{6}{i+j}}
       B^6_{i+j}(t) (b_{i1} b_{j1} + \ldots + b_{i4} b_{j4}) \nonumber
\end{eqnarray}
Letting $k=i+j$, 
\[ m_5(t) = \sum_{k=0}^6 B_k^6(t) 
	\sum_{\begin{array}{c}  \mbox{\footnotesize{$0 \leq i \leq 3$}} \\ 
			     \mbox{\footnotesize{$0 \leq j \leq 3$}} \\ 
			     \mbox{\footnotesize{$i+j=k$}}
			     \end{array}} 
	\frac{\scriptchoice{3}{i} \scriptchoice{3}{j}}{\scriptchoice{6}{k}}
	(b_{i1} b_{j1} + \ldots + b_{i4} b_{j4}) \]
The other coordinates can be computed analogously to yield:
\[ M(c(t)) = 
   \sum_{k=0}^6 B_k^6(t)
	\sum_{\begin{array}{c}  \mbox{\footnotesize{$0 \leq i \leq 3$}} \\ 
			     \mbox{\footnotesize{$0 \leq j \leq 3$}} \\ 
			     \mbox{\footnotesize{$i+j=k$}}
			     \end{array}} 
	\frac{\choice{3}{i} \choice{3}{j}}{\choice{6}{k}}
	\left( \begin{array}{c}
            b_{i1} b_{j1} + b_{i2} b_{j2} + b_{i3} b_{j3} - b_{i4} b_{j4} \\
            2b_{i1} b_{j4} \\
            2b_{i2} b_{j4} \\
            2b_{i3} b_{j4} \\
            b_{i1} b_{j1} + b_{i2} b_{j2} + b_{i3} b_{j3} + b_{i4} b_{j4}
	\end{array} \right) \]
This is a sextic rational Bezier curve with weights 
\begin{eqnarray}
\label{eq:weights}
w_k & = & \sum_{\begin{array}{c} \mbox{\footnotesize{$0 \leq i \leq 3$}} \\ 
			     \mbox{\footnotesize{$0 \leq j \leq 3$}} \\ 
			     \mbox{\footnotesize{$i+j=k$}}
			     \end{array}}
        \frac{\choice{3}{i} \choice{3}{j}}{\choice{6}{k}}
	\ (b_{i1} b_{j1} + b_{i2} b_{j2} + b_{i3} b_{j3} + b_{i4} b_{j4}) \nonumber \\
& = & \sum_{\begin{array}{c} \mbox{\footnotesize{$0 \leq i \leq 3$}} \\ 
			     \mbox{\footnotesize{$0 \leq j \leq 3$}} \\ 
			     \mbox{\footnotesize{$i+j=k$}}
			     \end{array}}
        \frac{\choice{3}{i} \choice{3}{j}}{\choice{6}{k}}
	\ b_i \cdot b_j
\end{eqnarray}
and control points 
\begin{eqnarray}
\label{eq:control-pts}
\breve{b}_k & = & \frac{1}{w_k} 
      \sum_{\begin{array}{c} \mbox{\footnotesize{$0 \leq i \leq 3$}} \\ 
			     \mbox{\footnotesize{$0 \leq j \leq 3$}} \\ 
			     \mbox{\footnotesize{$i+j=k$}}
			     \end{array}} 
        \frac{\choice{3}{i} \choice{3}{j}}{\choice{6}{k}}
	\left( \begin{array}{c}
            b_{i1} b_{j1} + b_{i2} b_{j2} + b_{i3} b_{j3} - b_{i4} b_{j4} \\
            2b_{i1} b_{j4} \\
            2b_{i2} b_{j4} \\
            2b_{i3} b_{j4} 
	\end{array} \right) \nonumber \\
& = & \frac{1}{w_k} 
      \sum_{\begin{array}{c} \mbox{\footnotesize{$0 \leq i \leq 3$}} \\ 
			     \mbox{\footnotesize{$0 \leq j \leq 3$}} \\ 
			     \mbox{\footnotesize{$i+j=k$}}
			     \end{array}} 
        \frac{\choice{3}{i} \choice{3}{j}}{\choice{6}{k}}
{\cal M}(b_i, b_j)
\end{eqnarray}
for $k = 0, \ldots, 6$.

By evaluating (\ref{eq:weights}), $w_0$ through $w_3$ can be simplified to
\begin{eqnarray}
(w_0,w_1,w_2,w_3) & = & 
% (b_0 \cdot b_0, 
% \frac{3b_0 \cdot b_1 + 3b_1 \cdot b_0}{6}, 
% \frac{3b_0 \cdot b_2 + 9b_1 \cdot b_1 + 3b_2 \cdot b_0}{15}, 
% \frac{b_0 \cdot b_3 + 9b_1 \cdot b_2 + 9b_2 \cdot b_1 + b3 \cdot b_0}{20}) \nonumber \\
% & = &
(b_0 \cdot b_0, b_0 \cdot b_1, \frac{2b_0 \cdot b_2 + 3b_1 \cdot b_1}{5}, 
\frac{b_0 \cdot b_3 + 9b_1 \cdot b_2}{10}) \nonumber 
\end{eqnarray}

Similarly, by evaluating (\ref{eq:control-pts}), $\breve{b}_0$ through $\breve{b}_3$ 
can be simplified to 
\begin{eqnarray}
\breve{b}_0 & = & % \frac{{\cal M}(b_0,b_0)}{w_0} =
                    \frac{{\cal M}(b_0,b_0)}{b_0 \cdot b_0} \nonumber \\
\breve{b}_1 & = & % \frac{{\cal M}(b_0,b_1) + {\cal M}(b_1,b_0)}{2w_1} =
                    \frac{{\cal M}(b_0,b_1) + {\cal M}(b_1,b_0)}{2b_0\cdot b_1} \nonumber \\
\breve{b}_2 & = & % \frac{{\cal M}(b_0,b_2) + 3{\cal M}(b_1,b_1) + {\cal M}(b_2,b_0)}{5w_2} = 
                    \frac{{\cal M}(b_0,b_2) + 3{\cal M}(b_1,b_1) + {\cal M}(b_2,b_0)}
	                 {2b_0 \cdot b_2 + 3b_1 \cdot b_1} \nonumber \\
\breve{b}_3 & = & % \frac{{\cal M}(b_0,b_3) + 9{\cal M}(b_1,b_2) + 9{\cal M}(b_2,b_1) + {\cal M}(b_3,b_0)}{20w_3} =
                    \frac{{\cal M}(b_0,b_3) + 9{\cal M}(b_1,b_2) + 9{\cal M}(b_2,b_1) + {\cal M}(b_3,b_0)}{2b_0 \cdot b_3 + 18b_1 \cdot b_2} \nonumber
\end{eqnarray}

$\breve{b}_4$ through $\breve{b}_6$ and $w_4$ through $w_6$ can then be determined 
through the symmetry of Bezier curves: 
since the geometry of the cubic Bezier curve does not change 
under reversal of the control points, neither does the algebra of its image.
The leverage of symmetry also generates more efficient formulae.\footnote{For example,
  $w_6$ and $\breve{b}_6$ are both expressed originally as sums of 7 terms.
  Using symmetry, they are now expressed by one term.}
\begin{eqnarray}
w_4 & = & \frac{2b_3 \cdot b_1 + 3b_2 \cdot b_2}{5} \nonumber \\
w_5 & = & b_3 \cdot b_2 \nonumber \\
w_6 & = & b_3 \cdot b_3 \nonumber \\
\breve{b}_4 & = & \frac{{\cal M}(b_3,b_1) + 3{\cal M}(b_2,b_2) + {\cal M}(b_1,b_3)}
	                 {2b_3 \cdot b_1 + 3b_2 \cdot b_2} \nonumber \\
\breve{b}_5 & = & \frac{{\cal M}(b_3,b_2) + {\cal M}(b_2,b_3)}{2b_3\cdot b_2} \nonumber \\
\breve{b}_6 & = & \frac{{\cal M}(b_3,b_3)}{b_3 \cdot b_3} \nonumber
\end{eqnarray}
\QED

\section{Poles}

The special behaviour of $M^{-1}$ at the pole $(1,0,0,0)$
is not unusual with maps from the sphere.
Consider the classical problem in cartography: how to portray the earth
on a 2-dimensional page.
A conformal map from $S^3$ to $\Re^2$ is desired.
Also consider the latitude/longitude parameterization of the sphere, with its two poles.
A parameterization is a map between the surface and $\Re^2$.

\subsection{Postamble to rational maps to the sphere}

The inverse image $M^{-1}(P)$ is a line or plane.
By perturbing the data away from the pole (Section~\ref{sec:perturb}), 
we may assume that $P \neq (1,0,0,0)$ and $M^{-1}(P)$ is a line.
Therefore, the interpolation of $\{M^{-1}(P_i)\}$ is the design of a curve through a
collection of lines.
Rather than directly attacking this non-traditional problem (see the future problems in
Section~\ref{sec:future}), we will reduce it to a traditional point
interpolation problem by choosing a point on each line.
Which point should be chosen on each line?
One solution is to always choose $M^{-1}(P)$ on $S^3$.
This solution (seems to?) promote short curves, a good thing, 
and avoids collapse into the origin, a bad thing.
Note that this is a well-defined choice, since $M^{-1}(P)$ is a line through the origin.
Another advantage of this choice is that it yields a simple inverse.

% From 'A rational quaternion spline of arbitrary continuity'

\section{A map to and from \Sn{3}}
\label{sec:map}

The first step in our algorithm for the design of a rational quaternion
spline is to design a good rational map from 4-space to the surface \Sn{3}.
Recall that we prefer a map with domain $\Re^4$, so that the data points will be
mapped to one-dimensional curves in Euclidean space and the 
curve design in Euclidean space will enjoy more flexibility.

Our approach is as follows.
We shall first develop a normal form for all rational maps of $\Re^4$ to \Sn{3}.
We will then choose a particular map.
Finally, we will compute the inverse of this map.

\subsection{Rational maps of $\Re^4$ to \Sn{3}}

Consider a rational map from $\Re^4$ to \Sn{3}:
\[
	(x_1,\ldots,x_4) \mapsto
	(\frac{f_1(x_1,\ldots,x_4)}{f_{5}(x_1,\ldots,x_4)}, \ldots,
	 \frac{f_4(x_1,\ldots,x_4)}{f_{5}(x_1,\ldots,x_4)})
\]
where $f_1,\ldots,f_{5}$ are polynomials.
Since the image lies on \Sn{3}, $f_1^2 + \cdots + f_4^2 = f_{5}^2$
and $(f_1,f_2,f_3,f_4,f_5)$ is a Pythagorean quintuple.\footnote{This term derives
	from the Pythagorean Theorem on right triangles,
	which involves Pythagorean triples.}
\begin{defn2}
$(a_1,\ldots,a_{5}) \in K^{n+1}$
is a {\bf Pythagorean quintuple over $K$} 
if $a_1^2 + \ldots + a_4^2 = a_5^2$.
\end{defn2}
%
Thus, the study of rational maps from $\Re^{4}$ to \Sn{3}\ 
is equivalent to the study of Pythagorean quintuples of polynomials.
Pythagorean quintuples involve the sum of four squares.
In the number theory literature, there is an extensive study of the
sum of four squares (driven by the search for a proof that
every positive integer is the sum of the squares of four integers \cite{dickson52}).
An important result was developed by Euler,\footnote{This result shows that the product of a sum of four squares and a sum of four
	squares is another sum of four squares.
	This reduces the problem of showing that every integer is the sum
	of four squares to the simpler problem
	of showing that every prime is the sum of four squares,
	since every integer can be expressed as the product of primes.}
which we can use to build Pythagorean quintuples, 
and then a characterization of maps to \Sn{3}.

\begin{lemma}[Euler's Four Squares Theorem \cite{herstein75}]
\label{lem:euler}
% p. 373 of Herstein
\[
\begin{array}{ll}
& (a_1^2 + a_2^2 + a_3^2 + a_4^2) 
(\hat{a}^2_1 + \hat{a}^2_2 + \hat{a}^2_3 + \hat{a}^2_4) = \\
& (a_1 \hat{a}_1 - a_2\hat{a}_2 - a_3\hat{a}_3 - a_4\hat{a}_4)^2 +
   (a_1\hat{a}_2 + a_2\hat{a}_1 + a_3\hat{a}_4 - a_4\hat{a}_3)^2 + \\
& (a_1\hat{a}_3 - a_2\hat{a}_4 + a_3\hat{a}_1 + a_4\hat{a}_2)^2 +
   (a_1\hat{a}_4 + a_2\hat{a}_3 - a_3\hat{a}_2 + a_4\hat{a}_1)^2
\end{array}
\]
where $a_1,a_2,a_3,a_4,\hat{a}_1,\hat{a}_2,\hat{a}_3,\hat{a}_4$ are elements of a
commutative ring.
% \footnote{The original statement was for integers, but it easily generalizes.}
% see p. 210 of Ebbinghaus
\end{lemma}

\begin{corollary}
\label{lem:suff4}
$(a_1^2 + a_2^2 + a_3^2 - a_4^2,\ 2a_1a_4,\ 2a_2a_4,\ 2a_3a_4,\ 
 a_1^2 + a_2^2 + a_3^2 + a_4^2)$
is a Pythagorean quintuple for any polynomials $a_1,a_2,a_3,a_4$.
\end{corollary}
\prf
Let $(\hat{a}_1,\hat{a}_2,\hat{a}_3,\hat{a}_4) = (a_1,-a_2,-a_3,a_4)$. 
\QED

\noindent The following lemma establishes a weaker version of the necessary condition
associated with Corollary~\ref{lem:suff4}.
This in turn leads to the desired necessary and sufficient condition for rational maps
of $\Re^4$ to \Sn{3}.

\begin{lemma}
\label{thm:necessary4}
A quintuple of polynomials is Pythagorean only if it can be expressed in the form
\begin{equation}
\label{eq:pyth}
	\alpha (a_1^2 + a_2^2 + a_3^2 - a_4^2,
		\ 2a_1a_4,\ 2a_2a_4,\ 2a_3a_4,
		\ a_1^2 + a_2^2 + a_3^2 + a_4^2)
\end{equation}
for some polynomials $a_1,a_2,a_3,a_4,\frac{1}{\alpha}$.
\end{lemma}
\prf
Let $(p_1,p_2,p_3,p_4,p_5)$ be a Pythagorean quintuple of polynomials.
If $p_1 = p_5$, let\\
$(a_1,a_2,a_3,a_4,\alpha) = (p_1,0,0,0,\frac{1}{p_1})$.
Thus, we may assume without loss of generality that $p_1 \neq p_5$.
Let $(a_1,a_2,a_3,a_4,\alpha) = (p_2,p_3,p_4,p_5-p_1,\frac{1}{2(p_5 - p_1)})$.
Then $a_1,a_2,a_3,a_4,\alpha$ generate the Pythagorean quintuple
$(p_1,\ldots,p_5)$ as in (\ref{eq:pyth}).
In particular,
\[
\alpha (a_1^2 + a_2^2 + a_3^2 - a_4^2)
= \frac{p_2^2 + p_3^2 + p_4^2 - (p_5 - p_1)^2}{2(p_5-p_1)}
\]
and applying $p_1^2 + p_2^2 + p_3^2 + p_4^2 = p_5^2$,
\[
\alpha (a_1^2 + a_2^2 + a_3^2 - a_4^2) = \frac{-2p_1^2 + 2p_1p_5}{2(p_5 - p_1)} = p_1.
\]
Also, 
\[
\alpha (2a_i a_4) = \frac{2p_{i+1}(p_5 - p_1)}{2(p_5 - p_1)} = p_{i+1}
\]
for $i=1,2,3$.
Finally, 
\[
\alpha(a_1^2 + a_2^2 + a_3^2 + a_4^2) 
= \frac{p_2^2 + p_3^2 + p_4^2 + (p_5 - p_1)^2}{2(p_5-p_1)} = p_5
\]
Thus, $(p_1,p_2,p_3,p_4,p_5) = \alpha (a_1^2 + a_2^2 + a_3^2 - a_4^2,
		\ 2a_1a_4,\ 2a_2a_4,\ 2a_3a_4,
		\ a_1^2 + a_2^2 + a_3^2 + a_4^2)$.
\QED

\begin{corollary}
\label{thm:map4}
A map is a rational map of $\Re^4$ to \Sn{3}\ if and only if
it is of the form:
\begin{equation}
\label{eq:re4s3}
\small{(x_1,x_2,x_3,x_4)} \mapsto 
\small{\frac{1}{a_1^2 + a_2^2 + a_3^2 + a_4^2},
	(a_1^2 + a_2^2 + a_3^2 - a_4^2, 2a_1a_4, 2a_2a_4, 2a_3a_4)}
\end{equation}
where $a_1,a_2,a_3,a_4$ are polynomials over $x_1,x_2,x_3,x_4$
(or some permutation of this form).\footnote{This result generalizes
	to rational maps from $\Re^n$ to \Sn{n-1}, $n \geq 2$,
	using an identical proof.
	That is, a map is a rational map of $\Re^n$ to \Sn{n-1}, $n \geq 2$, 
	if and only if it is of the form:
\[
	(x_1,\ldots,x_n) \mapsto
	(\frac{a_1^2 + \cdots + a_{n-1}^2 - a_n^2}{a_1^2 + \cdots + a_n^2},
	 \frac{2a_1a_n}{a_1^2 + \cdots + a_n^2},
	 \ldots,
	 \frac{2a_{n-1}a_n}{a_1^2 + \cdots + a_n^2})
\]
where $a_1,\ldots,a_n$ are polynomials over $x_1,\ldots,x_n$.}
\end{corollary}
\prf
Consider a rational map
of $\Re^4$ to \Sn{3}, $(x_1,x_2,x_3,x_4) \mapsto 
(\frac{f_1}{f_5},\frac{f_2}{f_5},\frac{f_3}{f_5},\frac{f_4}{f_5})$,
where $f_1,\ldots,f_5$ are polynomials over $x_1,\ldots,x_4$.
Then $(f_1,\ldots,f_5)$ is a Pythagorean quintuple
so it can be expressed in the normal form
$(f_1,\ldots,f_5) = \alpha (a_1^2 + a_2^2 + a_3^2 - a_4^2,
2a_1a_4,2a_2a_4,2a_3a_4,$ $a_1^2 + a_2^2 + a_3^2 + a_4^2)$
for some polynomials $a_1,a_2,a_3,a_4,\frac{1}{\alpha}$
by Lemma~\ref{thm:necessary4}.
Thus, the map is of the form (\ref{eq:re4s3}), since the leading $\alpha$
cancels when expressed in 
$(\frac{f_1}{f_5},\frac{f_2}{f_5},\frac{f_3}{f_5},\frac{f_4}{f_5})$.
Notice that it is legal to cancel the $\alpha$, since
$\alpha = \frac{1}{\beta}$ for some polynomial $\beta$,
which implies that $\alpha$ is never zero.

Now consider a map of the form (\ref{eq:re4s3}).
This is a rational map of $\Re^4$ to \Sn{3}\ by Corollary~\ref{lem:suff4}.
\QED

\noindent The most natural choice for the polynomials $a_i$ 
in Corollary~\ref{thm:map4} is $a_i = x_i$ and the identity permutation.
%
\begin{defn2}
The {\bf most natural} map to \Sn{3}\ is 
$M: \Re^4 - \{0\} \rightarrow \Sn{3}$ defined by:
\begin{equation}
\label{eqM}
\footnotesize{
	M(x_1,x_2,x_3,x_4) =
	\frac{1}{x_1^2 + x_2^2 + x_3^2 + x_4^2}
	(x_1^2 + x_2^2 + x_3^2 - x_4^2, 2x_1x_4, 2x_2x_4, 2x_3x_4)
	 }
\end{equation}
\end{defn2}
%
We fully analyze this map in \cite{jj98b}.
For example, we show that it is a powerful extension of
the inverse map of stereographic projection.
(We also further explain the title `most natural'.)
We shall use the map $M$ as our rational map to \Sn{3}.

START HERE

\subsection{The inverse map}

To map the quaternions to Euclidean space,
the inverse of our map to \Sn{3}\ is needed (step 1b of the algorithm
in Section~\ref{subsec:design}).
Since the manifold $\Re^4$ is one dimension larger than \Sn{3}, 
one would expect the preimage of a point of \Sn{3}\ to be 
a curve in $\Re^4$.
This is indeed the case.
In fact, the inverse of a point is simply a line.
The proof of the following theorem is simplified by working in projective space.
%
\begin{defn2}
\label{defn:projspace}
% We shall make use of projective space in some of the proofs in this paper.
Real {\bf projective $n$-space} $P^n$ is the space 
$\{ (x_1,x_2,\ldots,x_{n+1}) : x_i \in \Re, \mbox{not all zero} \}$
under the equivalence relation 
\begin{equation}
\label{eq:projequivalence}
(x_1,\ldots,x_{n+1}) = k(x_1,\ldots,x_{n+1}),\ \ k \neq 0 \in \Re.
\end{equation}
The point $(x_1,\ldots,x_{n+1})$ in projective $n$-space, $x_{n+1} \neq 0$,
is equivalent to the point $(\frac{x_1}{x_{n+1}},\ldots,\frac{x_n}{x_{n+1}})$
in $n$-space.
The point $(x_1,\ldots,x_n,0)$ in projective $n$-space represents the point
at infinity in the direction $(x_1,\ldots,x_n)$.
To translate from $n$-space to projective $n$-space, the point 
$(x_1,\ldots,x_n)$ is typically transformed into the point $(x_1,\ldots,x_n,1)$.
See \cite{harris92} for more details on projective space.
\end{defn2}

\begin{theorem}
\label{thm:inverse}
$M^{-1}: S^3 \rightarrow \Re^4$ is defined as follows:
\[ 
\footnotesize{
	M^{-1}(x_1,x_2,x_3,x_4) =
	\left\{ 
	\begin{tabular}{ll}
		$t(x_2,x_3,x_4,1-x_1),\ t \in \Re,\ t \neq 0 \hspace{.2in}$
		  & $\mbox{if } (x_1,x_2,x_3,x_4) \neq (1,0,0,0)$\\
		$H$ & $\mbox{if } (x_1,x_2,x_3,x_4) = (1,0,0,0)$
	\end{tabular}
	\right.
}
\]
where $H$ is the hyperplane $x_4 = 0$ minus the origin.
That is, the preimage of $(x_1,x_2,x_3,x_4) \neq (1,0,0,0)$ on \Sn{3}
is a line through the origin, minus the origin.
\end{theorem}
\prf 
We work in projective space, where the map $M$ becomes
\begin{equation}
\label{eq:proj}
	 (x_1,x_2,x_3,x_4,x_5) \rightarrow
	 (x_1^2 + x_2^2 + x_3^2 - x_4^2,\ 
	 2x_1 x_4,\ 2x_2 x_4,\ 2x_3 x_4,\ 
	 x_1^2 + x_2^2 + x_3^2 + x_4^2).
\end{equation}
Let $p = (p_1,p_2,p_3,p_4,1) \in S^3 \subset P^4$.
We want to determine the conditions on $q = (q_1,q_2,q_3,q_4,q_5)$
so that $M(q_1,q_2,q_3,q_4,q_5) = p$.
Suppose that $M(q_1,q_2,q_3,q_4,q_5) = p$.
Using (\ref{eq:projequivalence}), we have
\begin{equation}
\label{eq1}
	q_1^2 + q_2^2 + q_3^2 - q_4^2 = kp_1
\end{equation}
\begin{equation}
\label{eq2}
	2q_1q_4	= kp_2
\end{equation}
\begin{equation}
\label{eq3}
	2q_2q_4	= kp_3
\end{equation}
\begin{equation}
\label{eq4}
	2q_3q_4	= kp_4
\end{equation}
\begin{equation}
\label{eq5}
	q_1^2 + q_2^2 + q_3^2 + q_4^2 = k
\end{equation}
for some $k \neq 0$.
$q_5$ is arbitrary, since it does not appear in these equations.
Subtracting (\ref{eq1}) from (\ref{eq5}), 
we have $2q_4^2 = k(1 - p_1)$
or 
\begin{equation}
\label{eq:q4}
q_4 = \pm \sqrt{\frac{k(1-p_1)}{2}}
\end{equation}
%
{\bf Case 1:\ }
Suppose $p_1 = 1$. Then $p = (1,0,0,0,1)$ since $p \in \Sn{3}$.
$q_4 = 0$ by (\ref{eq:q4})
and $M(q) = (q_1^2 + q_2^2 + q_3^2, 0, 0, 0, q_1^2 + q_2^2 + q_3^2)
       = (1,0,0,0,1)$ for any values of $q_1,q_2,q_3$, not all zero.
That is, $M(q) = p$ if and only if $q \in H$ where $H$ is 
the hyperplane $x_4 = 0$ minus the origin.
Equivalently, $M^{-1}(p) = H$.\\
%
{\bf Case 2:\ }
Suppose $p_1 \neq 1$.
Then $p_1 < 1$ since $p \in \Sn{3}$. 
From (\ref{eq:q4}), $q_4 \neq 0$ and $q_4$ is a real number.
From (\ref{eq2}-\ref{eq4}), $q_i = \frac{kp_{i+1}}{2q_4}$ for $i=1,2,3$:
\[
	q = (\frac{kp_2}{2q_4}, \frac{kp_3}{2q_4}, \frac{kp_4}{2q_4}, q_4, q_5)
\]
Using (\ref{eq:projequivalence}), 
\[
	q = \frac{2q_4}{k} q
	  = (p_2, p_3, p_4, \frac{2q_4^2}{k},\frac{2q_4q_5}{k})
\]
and then (\ref{eq:q4}),
\[
	q = (p_2,p_3,p_4,1-p_1, \pm \sqrt{\frac{2(1-p_1)}{k}} q_5)
\]
Since $q_5$ and $k$ are arbitrary,
\[
	q = (p_2,p_3,p_4,1-p_1, k') \hspace{1in} k' \in \Re
\]
We have shown that $M(q) = p$ only if 
$q = (p_2,p_3,p_4,1 - p_1,k')$.

On the other hand, if $q = (p_2,p_3,p_4,1 - p_1,k')$, $k' \in \Re$
and $p_1 \neq 1$,
then 
\[
\scriptsize{M(q) = (p_2^2 + p_3^2 + p_4^2 - (1 - p_1)^2,\ 
	2p_2 (1 - p_1),\ 2p_3 (1 - p_1),\ 2p_4 (1 - p_1),\ 
	p_2^2 + p_3^2 + p_4^2 + (1 - p_1)^2).}
\]
Using $p_1^2 + p_2^2 + p_3^2 + p_4^2 = 1$ ($p \in \Sn{3}$),
\[
M(q) = (2p_1(1 - p_1),\ 2p_2 (1 - p_1),\ 2p_3 (1 - p_1),\ 2p_4 (1 - p_1),\ 
2(1 - p_1))
\]
or $M(q) = (p_1,p_2,p_3,p_4,1)$ using (\ref{eq:projequivalence}).
Thus, $M(q) = p$ if and only if 
$q = (p_2,p_3,p_4,1 - p_1,k')$, $k' \in \Re$.
Translating back from projective space,
$M(q) = p$ if and only if 
$q = t(p_2,p_3,p_4, 1-p_1)$, $t \neq 0 \in \Re$.
Equivalently, $M^{-1}(p) = t(p_2,p_3,p_4, 1-p_1)$, $t \neq 0 \in \Re$.
It is understandable that the preimage does not contain the origin, 
since $M$ is undefined there.
\QED
% $q \in M^{-1}(p)$ if and only if $M(q) = p$.

\begin{defn2}
\label{defn:pole}
The special point $(1,0,0,0)$ is called the {\bf pole} of the map $M^{-1}$.
$(x_2,x_3,x_4,1-x_1)$ is called the {\bf defining point} of the preimage 
$M^{-1}(x_1,x_2,x_3,x_4)$.
\end{defn2}

\noindent Notice the beautiful simplicity of the defining point.

\begin{rmk}
Although the proof technique of Theorem~\ref{thm:inverse}
works for any individual map,
we cannot apply it to the abstract general map (\ref{eq:re4s3})
to reveal the general inverse map.
The above proof requires working in projective space, to take 
advantage of the extra equation it offers;
this would require getting inside the arbitrary polynomials
$a_i$ of (\ref{eq:re4s3}) to translate them into projective space,
an unfeasible task
since these polynomials are of unknown and varying degree.
\end{rmk}

% Example of another map: using $a_1=x_1 - 1$, $a_2 = x_2^2$, $a_3=0$, 
% and $a_4=x_4$,
% let $M(q_1,q_2,q_3,q_4,q_5) = (q_1^2 - 2q_1 + 1 + q_2^4 - q_4^2,
% 				2q_1q_4 - 2q_4, 2q_2^2 q_4, 0, 
% 				q_1^2 - 2q_1 + 1 + q_2^4 + q_4^2)$.

\end{document}
