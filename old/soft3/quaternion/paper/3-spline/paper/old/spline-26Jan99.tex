\documentstyle[12pt,titlepage]{article}
\newif\ifFull
\Fullfalse
\input{header}
\SingleSpace

\setlength{\oddsidemargin}{0pt}
\setlength{\topmargin}{0in}	% should be 0pt for 1in
% \setlength{\headsep}{.5in}
\setlength{\textheight}{8.5in}
\setlength{\textwidth}{6.5in}
\setlength{\columnsep}{5mm}	% width of gutter between columns
% \markright{Quaternion splines: \today \hfill}
% \pagestyle{myheadings}
% -----------------------------------------------------------------------------

\title{A Simple but Effective Rational Quaternion Spline}
% Rational Quaternion Splines as Euclidean Curves}
% \title{Euclidean curves on a surface}
\author{J.K. Johnstone \and J.P. Williams}

\begin{document}
\maketitle

\begin{abstract}
The main battering ram for new breakthrough should be the rationality
of the quaternion spline and its good quality (better than rational trim curve
or nonrational Lie methods or rational biarc).
So present the paper as a new high-quality rational quaternion spline.
We can talk about extensions to other surfaces in the conclusions, but
it is not yet a general method for design of curves on surfaces,
so save that argument for when it blossoms beyond S3.
The argument of mapping to Euclidean space still holds and can motivate
the method (paragraph 3 on).
But start with the rational quaternion spline problem and proceed
to the Euclidean space approach rather than the other way around.
This also deflects criticism as extension of Dietz.
It is not a massive change from the approach of Dietz of curves on
surfaces (although we choose points in Euclidean space, rather than 
curves directly from curves, which changes many things)
and the general approach is not the main point anyway.

Get optimal curve by using 1) closest image point in Euclidean space
2) centripetal knot sequence w.r.t. Euclidean curve.
Get bi-invariance by using 3) optimal perturbation of best-fitting
plane.

The design of interpolating curves 
(curves that interpolate a set of data points) 
is a fundamental problem in geometric modeling.
A new approach to the design of rational interpolating curves on a surface
is proposed, and applied to the design of rational quaternion splines.
The curves constructed by the new method are of superior quality
and the method is highly efficient.
The curve's rationality allows smooth integration into existing
geometric models.
The design process relies on classical techniques for curve
interpolation in Euclidean space (unconstrained to a surface).
Thus, little new software needs to be written, and the design is simple
and efficient.
The default choice is to design curves with $C^2$ continuity,
but continuity can be controlled and is not bounded.
{\bf Check that \Sn{3}\ curve are $C^2$ continuous.}
The method is coordinate-frame invariant.
There is flexibility in the design of the curve that is lacking from 
earlier techniques.
The essence of the method is to map the data points to Euclidean space,
% choose a collection of points from the resulting collection of 
% data curves in Euclidean space, 
design the curve through selected points in unconstrained Euclidean space, 
and map the curve back to the surface.
\end{abstract}	

% \tableofcontents

\section{Introduction}
\label{sec:intro}

The design of curves on surfaces is an important problem,
with applications ranging from the trimming of surfaces in solid modeling,
to the design of splines on the quaternion sphere for orientation control
in computer animation, to the design of shortest paths on configuration
space obstacles in robot motion planning.
%  \footnote{Shortest paths amongst obstacles must inherently hug the obstacles.}
The design of a curve on a surface is more challenging than the 
conventional design of a curve in Euclidean space.
First, a surface constraint is added to the more typical constraints
such as interpolation of data points and preservation of smoothness criteria.
Second, surfaces are Riemannian spaces and 
Riemannian geometry is significantly different from Euclidean geometry.
% \cite{kreyszig63}.
For example, in moving from Euclidean to Riemannian geometry,
straight lines are replaced by geodesics.
% (\eg, a helix on a cylinder).

In this paper, we shall present a new general approach to the design of curves
on surfaces and discuss its implementation and advantages.
A single case study of this theory will be presented in detail:
the design of curves on \Sn{3}.

Since the modeling of curves in Euclidean space is much better understood, 
a promising approach to the
design of a curve on a surface is to somehow reduce the problem to the 
design of a curve in Euclidean space, not restricted to any surface.
% `Doctor, it hurts when I do this!'  `Well, then don't do it.'
% This reduction of an unknown problem to a known problem is a time-honored
% classical mathematical technique.
This allows the Riemannian problem to be solved as an extension
of the well-studied Euclidean problem, rather than as an entirely new
problem domain.
An approach for achieving this reduction to an Euclidean problem is as follows.

Suppose that we are designing a curve on the hypersurface\footnote{A
	hypersurface in $n$-space is a manifold of dimension $n-1$.}
$S$ in $n$-space.
Let $f:\Re^m \rightarrow S$ be a map to the surface.
Then any curve $C$ in $m$-space yields a curve $f(C)$ on the surface.
(See Figure~\ref{fig:reduce}.)
Consequently, we can now concentrate on designing the curve $C$, 
an Euclidean problem.
Although this is the simplest statement of the approach (the approach becomes
much more interesting when we add constraints on the type of map $f$ and 
the type of curve on the surface),
it reveals the essence of the approach most clearly:
the problem of constraining the curve to the surface is reduced to the
problem of developing a map to the surface.
The design of a curve on a surface through the design of a curve in $m$-space
and a map from $m$-space to the surface will be called the {\bf Euclidean-space
	% image-space, surface-map
approach} to curve design on a surface.
The resulting curves will be called {\bf Euclidean curves} on the surface.

\begin{figure}
\vspace{2.5in}
\special{psfile=/usr/people/jj/modelTR/3-spline/img/fig1.ps
	 hoffset=100}
\caption{The Euclidean-space approach to curve design on a surface}
% file: fig1.showcase
% tops fig1.rgb -m 6.5 1.5 > fig1.ps
% a simple picture of S, C, f(C), and f (as an arrow) for a generic surface S
\label{fig:reduce}
\end{figure}

Not surprisingly, the Euclidean-space approach has been used before.
% \cite{dietz93,wang94}.
% Given its simplicity and elegance, this is not surprising.
The classical solution to the design of trim curves on a surface 
is an example:
the trim curve is designed in the parameter space of the surface \cite{foley96}.
The surface parameterization is used as the map to the surface, % ($m=n-1$), 
which has the advantage that no map to the surface needs to be
developed.
However, a different map to the surface using the full power of
the Euclidean-space approach can improve the quality of the curve.
One can use added control in both steps of the Euclidean approach to optimize
the curve: the map can be directly controlled and, by choosing the
correct map, the design of the curve in Euclidean space is given more
flexibility (see below), which can be used to incorporate design optimizations.
Another reason to look beyond the trim-curve solution is the design of 
interpolating curves on a surface.
We shall now explore the rational interpolating curve on a surface,
which is the main focus of the paper.

	% Interpolating curves are perhaps the most important class of curves
	% in practical applications of curve design.
The interpolation of data points is necessary in many applications:
for example, in the design of quaternion 
splines for computer animation, where the data points are the orientations of 
the known keyframes; or in the design of paths on configuration-space
obstacles for robot motion, where the data points are the source, destination,
and rendezvous points of the path.
The design of rational curves is also highly desirable:
the rational curve is the most efficient of curves, because of the efficiency
of polynomial computations and the elegant theory of rational Bezier
and B-spline curves.
As a result, the rational curve is the defacto standard in modeling systems, 
with a large, established suite of algorithms for its manipulation.
Thus, the design of a rational curve will allow the curve to be incorporated 
immediately into existing software, and subsequent computations with the curve
will be simple and efficient.
Finally, we note that we are interested in curve design on
hypersurfaces in arbitrary dimensions.
For applications in animation and robotics, 
curves on surfaces in 4-space and 6-space are of just as much
interest as curves on surfaces in 3-space.

Consider the design of a rational
curve on the hypersurface $S$ that interpolates the set of points 
$\{p_i\}_{i=1,\ldots,k} \subset S$, using the Euclidean-space approach.
The algorithm is as follows.
Notice that (1a-b) are preprocessing steps.
%
\begin{description}
\item[(1a)] Design a rational map to the surface, $f:\Re^m \rightarrow S$ (preferably $m=n$).
\item[(1b)] Compute the inverse map $f^{-1}:S \rightarrow \Re^m$.
\item[(2a)] Map the data points to Euclidean space, by computing $\{f^{-1}(p_i)\}_{i=1,\ldots,k}$.
\item[(2b)] Design a rational curve $C$ in Euclidean space, interpolating $\{f^{-1}(p_i)\}_{i=1,\ldots,k}$.
\item[(2c)] Map the curve back to the surface, by computing $f(C)$.
\end{description}

We prefer maps to the surface of the form $f:\Re^n \rightarrow S$ (i.e., $m=n$).
Since $\Re^n$ is one dimension larger than $S$,
if $m=n$ then $f^{-1}(p_i)$, the inverse image of a point on $S$, is a curve.
This leads to a useful flexibility in the interpolation step (2b),
since one is interpolating curves, not points.
This can be used to design better curves on the surface.
This is a further disadvantage of the use of the surface
parameterization $f:\Re^{n-1} \rightarrow S$ as the map to the surface,
which yields points $f^{-1}(p_i)$ in Euclidean space.

Note that the inverse map $f^{-1}$ need not be rational.
Only the map $f$ and the curve $C$ must be rational.
Later in the paper, we shall consider the addition of derivative data
for the curve to interpolate.
% We postpone this discussion in order to avoid overcomplicating the issue
% at the outset.

As a case study of the proposed method,
we will consider in detail the design of rational interpolating curves 
on the 3-sphere \Sn{3}, also called quaternion splines.
We will show that this design is efficient and yields curves of superior
quality.
Quaternion splines arise in computer animation in the control of
orientation: the points on \Sn{3}\ are unit quaternions representing
discrete orientations of an object in a series of keyframes,
and the problem is to design a smooth motion of the object through
these keyframes,
which reduces to the design of a curve on \Sn{3}
interpolating the quaternions.
The application of quaternion 
splines is not restricted to computer animation, although they were
first developed in this context.  
Any application involving motion control,
such as robot motion or spacecraft control, 
can make good use of quaternion splines for orientation control.

A major advantage of the Euclidean-space approach is that it is easily
implemented and integrated into existing modelers,
since the heart of the algorithm is the design
of a curve in Euclidean space (step 2b).
The resulting curve can also be easily integrated into existing
geometric models, since it is a true rational curve.
Previous quaternion splines, for example, have required entirely new
weaponry (such as slerping or constrained optimization)
for the design of the curve;
and the resulting curve was highly nonrational so it could not be
directly handled by a NURBS modeling system.

The rest of the paper will be structured as follows.
We present some basic terminology in Section~\ref{sec:defn}.
Section~\ref{sec:prevwork} reviews previous work in curve design on surfaces
in general, and quaternion spline design in particular.
The heart of the paper is Sections~\ref{sec:map}-\ref{sec:avoid},
in which we present our implementation of the Euclidean-space approach
for quaternion splines.
We show how to design a good map to the surface, and its inverse,
in Section~\ref{sec:map}.
This section is largely based on a separate paper \cite{jj98a}.
The design of the curve in Euclidean space and its mapping back 
to Riemannian space are discussed in Sections~\ref{sec:eucdesign} and
\ref{sec:curveimage}, respectively.
We discuss a region of instability for the inverse in Section~\ref{sec:ill},
and show how to avoid this region using principal component analysis 
(Section~\ref{sec:best}) or subdivision (Section~\ref{sec:divandconq}),
leading to a final revised version of the 
algorithm for curve design in Section~\ref{sec:restatement}.
We need to measure the quality of a curve in order to compare methods.
In Section~\ref{sec:cov}, we discuss quality measures for a curve
and in Section~\ref{sec:comparison} we present several examples
of quaternion splines and use the quality measures
to argue the superiority of the proposed approach, including
the advantage of its map flexibility and interpolation flexibility.
We end with some conclusions and ideas for future work.
The appendix contains the proof of a theorem from Section~\ref{sec:map}.

% from `A characterization of maps' paper:
% A discussion of the inverse map is postponed to the 
% `Rational quaternion splines' paper.
%
% The inverse map $f^{-1}$ is also of interest, because 
% of its use in the design of interpolating curves on the surface.
% $f(C)$ can easily be made to interpolate a set of points $p_i$ 
% on $A$ by designing $C$ to interpolate the curves $f^{-1}(p_i)$.
% Derivative information can be interpolated in a similar fashion.


%
% Advantages of native-space approach: move to Rational quaternion splines paper.
%
% The native-space approach allows some added freedom in the design of the curve.
% In general, the inverse image $f^{-1}(p)$ of a point on $A$ is a curve,
% while $\rho^{-1}(p)$ is typically a point.
% Consequently, the design of a curve on $A$ interpolating the points $\{p_i\}$
% with the native-space approach
% only requires the design of a curve interpolating the curves
% $\{f^{-1}(p_i)\}$, as opposed to a curve interpolating
% the points $\{\rho^{-1}(p_i)$.
% The added freedom in the native-space approach can allow the design
% of more optimal curves.
% This is illustrated in \cite{dietz93}. (Where is the optimality?)
% A second advantage of the native-space approach is more freedom
% in the choice of map back to the surface.
% Not being limited to the surface parameterization,
% the native-space approach can use any rational map of $\Re^n$ to the surface.
% This freedom can lead to superior curves on the surface, as illustrated
% in \cite{jj+jimbo99}.



% ****************************************************************************
% ****************************************************************************
\section{Definitions}
\label{sec:defn}

\begin{defn2}
\Sn{n}\ is the unit sphere in $(n+1)$-space $x_1^2 + \ldots + x_{n+1}^2 - 1 = 0$.
\Sn{n}\ is called the {\bf unit $n$-sphere}.
\end{defn2}
%
The superscript of \Sn\ refers to the dimension of the manifold, which is one less
than the dimension of its resident Euclidean space:
\Sn{1}\ is a circle, \Sn{2}\ is a sphere, and \Sn{3}\ is a hypersphere
in 4-space.
Since \Sn{3}\ can be identified with the set of unit quaternions,
we shall sometimes refer to it as the {\bf quaternion sphere}.

\begin{defn2}
A {\bf rational polynomial} is a quotient of polynomials.
A map 
$(x_1,\ldots,x_n) \mapsto (f_1 (x_1,\ldots,x_n),\ldots,f_m (x_1,\ldots,x_n))$
is {\bf rational} if 
its components $f_i$ are all rational polynomials.
\end{defn2}

\begin{defn2}
$\Re[x_1,\ldots,x_n]$ is the ring of polynomials in the $n$ variables
$x_1,\ldots,x_n$ with real coefficients.
\end{defn2}

We shall make use of projective space in some of the proofs in this paper.

\begin{defn2}
\label{defn:projspace}
Real {\bf projective $n$-space} $P^n$ is the space of $(n+1)$-tuples
$\{ (x_1,x_2,\ldots,x_{n+1}) : x_i \in \Re, i=1,\ldots,n, \mbox{not all zero} \}$
under the equivalence relation $(x_1,\ldots,x_{n+1}) = k(x_1,\ldots,x_{n+1})$
if $k \neq 0 \in \Re$.
The point $(x_1,\ldots,x_{n+1})$ in projective $n$-space, $x_{n+1} \neq 0$,
is equivalent to the point $(\frac{x_1}{x_{n+1}},\ldots,\frac{x_n}{x_{n+1}})$
in $n$-space.
The point $(x_1,\ldots,x_n,0)$ in projective $n$-space represents the point
at infinity in the direction $(x_1,\ldots,x_n)$.
To translate from $n$-space to projective $n$-space, the point 
$(x_1,\ldots,x_n)$ is typically transformed into the point $(x_1,\ldots,x_n,1)$.
See \cite{harris92} for more details on projective space.
\end{defn2}

\section{Related work}
\label{sec:prevwork}

two main differences from Hoschek: 
	4d version and application to quaternion spline, 
	point interpolation in Euclidean space (choice of point on inverse image)

There is a rich literature on quaternion splines.
\cite{shoemake85} introduced quaternion splines as a solution
for keyframe animation in 1985, using slerping (spherical linear
interpolation) between points on \Sn{3}.
\cite{barr92} used constrained optimization to develop optimal
quaternion splines, introducing low covariant acceleration
as a desirable property of a quaternion spline,
and refining this approach for added efficiency in \cite{rama97}.
\cite{park97} uses Lie algebra to design the quaternion spline.
Park and Ravani work on the SO(3) manifold, rather than \Sn{3}\ manifold.
\cite{kim95} also uses Lie algebra and its exponential map.
\cite{wang93} builds quaternion splines from biarcs 
(great circles of the sphere).

Our work on quaternion splines differs from the earlier work
in two important ways: it designs rational curves\footnote{The
	only rational method is \cite{wang93}, which is quite restricted:
	designing curves from segments of great circles
	with limited continuity.}
and it is very efficient as it avoids expensive tools,
such as the constrained optimization in \cite{barr92}.

Since $M$ is a rational map,
the curve on the 3-sphere inherits the continuity of the curve in image space.
Since it is simple to define interpolating curves
in image space with $C^2$ continuity, or curves of higher degree with
even higher continuity, 
the curve on $S^3$ can easily have $C^2$ continuity
and indeed arbitrary continuity.
Other interpolation methods (e.g., slerping methods, spherical biarc methods)
have considerable difficulty with continuity.

\cite{dietz93} and \cite{wang94} design rational curves on quadrics,
especially the sphere (\Sn{2}).
They both use a variant of the Euclidean-space approach.
\marginpar{ELABORATE}

\cite{park95}

from 3sphere.tex:

The most popular method is to use spherical linear interpolation 
(also called slerping) 
\cite{shoemake85,duff85,pletinckx89,schlag91,nielson92,nielson93,kim95,nam95}.
The spherical linear interpolation between the points $P_0$ and $P_1$ on $S^3$ is:
\[ P(t) = \frac{\sin((1-t)\theta) P_0 + \sin(t \theta) P_1}{\sin \theta}
\]
where $\cos \theta = P_0 \cdot P_1$, which is the great circle between $P_0$ and $P_1$.
A spherical analog to the de Casteljau algorithm based upon this
spherical linear interpolation is used to build curves,
mimicking Bezier \cite{shoemake85,kim95}, B-spline 
\cite{duff85,nielson92,nielson93,kim95}, Hermite \cite{kim95,nam95},
cardinal spline \cite{pletinckx89}, and Catmull-Rom \cite{schlag91} curves.
Since slerping is a nonrational operation,
all of the methods based on slerping generate nonrational curves.
Most of these curves are defined only by a geometric construction, and have
no closed form algebraic definition.
Computation of derivatives of these curves is complicated,
as is the imposition of $C^2$ continuity, and
Kim et. al. \cite{kim95} go to considerable trouble to solve these two problems.
The use of conventional Bezier curves completely removes 
these problems with continuity or derivative calculation.

The work of Wang \cite{wang93, wang94} and Nielson \cite{nielson93}
is closest in spirit to this paper, since it designs rational curves on $S^3$.
However, these curves are limited in scope.
Wang \cite{wang93,wang94} (using spherical biarcs) 
and Nielson \cite{nielson93} develop quadratic curves
with $G^1$ continuity, and Wang \cite{wang94} develops sextic curves
with $C^1$ continuity.
% $C^2$ continuity is desirable, especially for animation.
In Wang \cite{wang93,wang94}, the input points must be augmented with 
tangents, since the problem is posed as a Hermite interpolation problem.
Both Wang and Nielson's methods also involve heuristic, input-dependent choices 
% (such as the choice of a spherical biarc from
% a one-parameter family of valid spherical biarcs, or a center of projection)
that can be difficult to make.
Neither of the methods has any completeness result about the
coverage of rational curves by their method.
Our method subsumes the work of Wang and Nielson, 
by creating rational curves of arbitrary
even degree (all rational curves on $S^3$ have even degree \cite{wang94}) 
and arbitrary continuity, using a single data-independent map that is provably complete.

The above methods require implementation of new curves,
such as slerped curves or spherical biarcs.
In our method, no new curves must be implemented, just two maps:
a map to the 3-sphere and its inverse.
Consequently, the implementation of our method and its incorporation
into any modeling software is much simpler.

Barr et.\ al.\ \cite{barr92} design interpolating curves on $S^3$ 
through constrained optimization: the constraint to the 3-sphere 
is realized through its introduction into an optimization problem.
This yields a nonrational curve again, indeed with no closed-form expression
and approximate.
The numerical optimization is also expensive.
The method does have the advantage of allowing extra constraints 
to be incorporated into the optimization (they choose to design a curve
that minimizes tangential acceleration).


\section{A map to and from the surface}
% \section{A rational map to the surface}
\label{sec:map}

The first step in our algorithm is to design a good rational map to the 
surface \Sn{3}.
We prefer a map with domain $\Re^4$, so that the data points will be
mapped to one-dimensional curves in Riemannian space and the 
curve design in Riemannian space will enjoy considerable flexibility.
In \cite{jj98a}, we develop the following characterization of rational
maps of $\Re^4$ to \Sn{3}.

\begin{theorem}
\label{thm:map4}
A map is a rational map of $\Re^4$ to \Sn{3}\ if and only if
it is of the form:
\begin{equation}
\label{eq:re4s3}
\footnotesize{(x_{\pi(1)},x_{\pi(2)},x_{\pi(3)},x_{\pi(4)})} \mapsto 
\footnotesize{
	\frac{1}{a_1^2 + a_2^2 + a_3^2 + a_4^2},
	(a_1^2 + a_2^2 + a_3^2 - a_4^2, 2a_1a_4, 2a_2a_4, 2a_3a_4)
	}
\end{equation}
where $a_1,a_2,a_3,a_4 \in \Re[x_1,x_2,x_3,x_4]$
and $\pi : \{1,2,3,4\} \rightarrow \{1,2,3,4\}$ is a permutation.
\end{theorem}

\noindent The most natural choice for the polynomials $a_i$ and the permutation $\pi$ 
in Theorem~\ref{thm:map4} is $a_i = x_i$ and the identity permutation.
This leads to a particular rational map to \Sn{3}\ that we call the natural map.
%
\begin{defn2}
The {\bf natural} map to \Sn{3}\ is 
$M: \Re^4 - \{0\} \rightarrow \Sn{3}$ defined by:
\begin{equation}
\label{eqM}
\footnotesize{
	M(x_1,x_2,x_3,x_4) =
	\frac{1}{x_1^2 + x_2^2 + x_3^2 + x_4^2}
	(x_1^2 + x_2^2 + x_3^2 - x_4^2, 2x_1x_4, 2x_2x_4, 2x_3x_4)
	 }
\end{equation}
\end{defn2}
%
We shall use the natural map $M$ as our map to \Sn{3}.
We fully analyze this map in \cite{jj98b}.
It can be viewed as a generalization of the inverse of stereographic projection.
% establishing that $M$ is related to stereographic
% projection but significantly more sophisticated.
% In particular, the restriction of $M$ to the hyperplane $x_4=1$ 
% is the inverse of stereographic projection, while the restriction of $M$
% to all the other hyperplanes is unrelated to stereographic projection.

To map the data points from Riemannian space to Euclidean space,
the inverse of the map to the surface is needed.
Since the manifold $\Re^4$ is one dimension larger than \Sn{3}, one would expect
several points of $\Re^4$ to map to the same point of \Sn{3}\ under $M$.
That is, in general, the preimage of a point of \Sn{3}\ should be a curve in $\Re^4$.
This is indeed the case.

\begin{theorem}
\label{thm:inverse}
The inverse of the natural map to \Sn{3}, $M^{-1}: S^3 \rightarrow \Re^4$, is
\[ 
\footnotesize{
	M^{-1}(x_1,x_2,x_3,x_4) =
	\left\{ 
	\begin{tabular}{ll}
		$\alpha(x_2,x_3,x_4,1-x_1),\ \alpha \in \Re,\ \alpha \neq 0 \hspace{.2in}$
		  & $\mbox{if } (x_1,x_2,x_3,x_4) \neq (1,0,0,0)$\\
		$H$ & $\mbox{if } (x_1,x_2,x_3,x_4) = (1,0,0,0)$
	\end{tabular}
	\right.
}
\]
where $H$ is the hyperplane $x_4 = 0$ minus the origin.
That is, the preimage of $(x_1,x_2,x_3,x_4) \neq (1,0,0,0)$ on \Sn{3}
is a line through the origin, minus the origin.
\end{theorem}
\prf 
We will work in projective space, where the map $M$ becomes\footnote{Notice 
    that all $x_5$ terms cancel in $M(\frac{x_1}{x_5},\ldots,\frac{x_4}{x_5})$,
    since the numerator and denominator of the terms are both homogeneous of 
    degree two.}
\begin{equation}
\label{eq:proj}
	 (x_1,x_2,x_3,x_4,x_5) \rightarrow
	 (x_1^2 + x_2^2 + x_3^2 - x_4^2,\ 
	 2x_1 x_4,\ 2x_2 x_4,\ 2x_3 x_4,\ 
	 x_1^2 + x_2^2 + x_3^2 + x_4^2).
\end{equation}
% M_proj (x1,x2,x3,x4,x5) = M(x1/x5,...,x4/x5) = --- in R4 = --- in P4
Let $p = (x_1,x_2,x_3,x_4,x_5) \in S^3 \subset P^4$.
Suppose $M(q_1,q_2,q_3,q_4,q_5) = p$.
We wish to determine the necessary form of $q = (q_1,q_2,q_3,q_4,q_5)$.

By (\ref{eq:proj}) and Definition~\ref{defn:projspace}, we have
\begin{equation}
\label{eq1}
	q_1^2 + q_2^2 + q_3^2 - q_4^2 = kx_1
\end{equation}
\begin{equation}
\label{eq2}
	2q_1q_4	= kx_2
\end{equation}
\begin{equation}
\label{eq3}
	2q_2q_4	= kx_3
\end{equation}
\begin{equation}
\label{eq4}
	2q_3q_4	= kx_4
\end{equation}
\begin{equation}
\label{eq5}
	q_1^2 + q_2^2 + q_3^2 + q_4^2 = kx_5
\end{equation}
for some $k \neq 0$.
$q_5$ is arbitrary, since it does not appear in these equations.
Subtracting (\ref{eq1}) from (\ref{eq5}), 
we have $2q_4^2 = k(x_5 - x_1)$
or 
\begin{equation}
\label{eq:q4}
q_4 = \pm \sqrt{\frac{k(x_5-x_1)}{2}}
\end{equation}
%
{\bf Case 1:\ }
Suppose $x_1 = x_5$. That is, $p = (1,0,0,0,1)$ since $p \in \Sn{3}$.
$q_4 = 0$ by (\ref{eq:q4})
and $M(q) = (q_1^2 + q_2^2 + q_3^2, 0, 0, 0, q_1^2 + q_2^2 + q_3^2)
       = (1,0,0,0,1)$ for any values of $q_1,q_2,q_3$, not all zero.
That is, $M(q) = p$ if and only if $q \in H$ where $H$ is 
the hyperplane $x_4 = 0$ minus the origin.
Equivalently, $M^{-1}(p) = H$.\\
%
{\bf Case 2:\ }
Suppose $x_1 \neq x_5$.
This implies $x_5 > x_1$ since $p \in \Sn{3}$. 
Thus, $q_4 \neq 0$ and $q_4$ is a real number.
From (\ref{eq2}), $q_1 = \frac{kx_2}{2q_4}$.
Similarly, $q_2 = \frac{kx_3}{2q_4}$ and $q_3 = \frac{kx_4}{2q_4}$.
Thus, 
\[
	q = (\frac{kx_2}{2q_4}, \frac{kx_3}{2q_4}, \frac{kx_4}{2q_4}, q_4, q_5)
\]
Using (\ref{defn:projspace}), 
\[
	q = \frac{2q_4}{k} q
	  = (x_2, x_3, x_4, \frac{2q_4^2}{k},\frac{2q_4q_5}{k})
\]
and using (\ref{eq:q4}),
\[
	q = (x_2,x_3,x_4,x_5-x_1, \pm \sqrt{\frac{2(x_5-x_1)}{k}} q_5)
\]
Since $q_5$ is arbitrary (as well as $k$),
\[
	q = (x_2,x_3,x_4,x_5-x_1, k') \hspace{1in} k' \in \Re
\]
We conclude that $M(q) = p$ only if 
$q = (x_2,x_3,x_4,x_5 - x_1,k')$.
On the other hand, if $q = (x_2,x_3,x_4,x_5 - x_1,k')$, $k' \in \Re$
and $x_1 \neq x_5$,
then 
\begin{equation}
\label{eq:Mq}
\tiny{
M(q) = (x_2^2 + x_3^2 + x_4^2 - (x_5 - x_1)^2,\ 
	2x_2 (x_5 - x_1),\ 2x_3 (x_5 - x_1),\ 2x_4 (x_5 - x_1),\ 
	x_2^2 + x_3^2 + x_4^2 + (x_5 - x_1)^2).
}	
\end{equation}	
Since $p \in \Sn{3}$, $x_1^2 + x_2^2 + x_3^2 + x_4^2 = x_5^2$,
which we can use to reduce (\ref{eq:Mq}) to
\[
M(q) = (2x_1(x_5 - x_1),\ 2x_2 (x_5 - x_1),\ 2x_3 (x_5 - x_1),\ 2x_4 (x_5 - x_1),\ 
2x_5(x_5 - x_1))
\]
or $p = (x_1,x_2,x_3,x_4,x_5)$ using (\ref{defn:projspace}).

Thus, $M(q) = p$ if and only if 
$q = (x_2,x_3,x_4,x_5 - x_1,k')$, $k' \in \Re$.
Translating back from projective space (see Definition~\ref{defn:projspace}),
$M(q) = p$ if and only if 
$q = t(x_2,x_3,x_4, 1-x_1)$, $t \neq 0 \in \Re$.

The preimage does not contain the origin, since the natural
map is undefined there.
\QED

% $q \in M^{-1}(p)$ if and only if $M(q) = p$.

\begin{rmk}
Although the proof technique of Theorem~\ref{thm:inverse}
works for any individual map,
we unfortunately cannot apply it to the abstract general map (\ref{eq:re4s3})
to reveal the general inverse map.
The above proof requires working in projective space, to take 
advantage of the extra equation it offers;
this would require getting inside the arbitrary polynomials
$a_i$ of (\ref{eq:re4s3}) to translate them into projective space,
an unfeasible task
since these polynomials are of unknown and varying degree.
\end{rmk}

% Example of another map: using $a_1=x_1 - 1$, $a_2 = x_2^2$, $a_3=0$, 
% and $a_4=x_4$,
% let $M(q_1,q_2,q_3,q_4,q_5) = (q_1^2 - 2q_1 + 1 + q_2^4 - q_4^2,
% 				2q_1q_4 - 2q_4, 2q_2^2 q_4, 0, 
% 				q_1^2 - 2q_1 + 1 + q_2^4 + q_4^2)$.

%
\begin{defn2}
\label{defn:pole}
The special point $(1,0,0,0)$ is called the {\bf pole} of the map $M^{-1}$.\\
$(x_2,x_3,x_4,1-x_1)$ is called the {\bf defining point} of the preimage 
$M^{-1}(x_1,x_2,x_3,x_4)$.
\end{defn2}

% The inverse of a point on \Sn{3}\ is needed to interpolate points.
% We may also want the inverse of a derivative so that we can 
% interpolate derivative information on \Sn{3}.
% Fortunately, this is computed using straightforward differentiation.

% \begin{corollary}
% \label{cor:higherderiv}
% Let $C(t) = (x_1(t), x_2(t), x_3(t), x_4(t)) \subset \Sn{3}$
	% don't use C, that is reserved for the curve in Euclidean space (see 2B)
% be a curve on \Sn{3}\ not containing the pole $(1,0,0,0)$.
% \[
% 	M^{-1}(C(t)) = \alpha(t)(x_2(t), x_3(t), x_4(t), 1-x_1(t))
% \]
% \[
% 	\frac{\partial}{\partial t}M^{-1}(C(t)) = 
% 	\alpha(t)(x'_2(t), x'_3(t), x'_4(t), -x'_1(t)) +
% 	\alpha'(t) (x_2(t), x_3(t), x_4(t), 1-x_1(t))
% \]
% \end{corollary}

% This corollary shows that it is simple to design curves on \Sn{3}\ that
% interpolate derivative data as well as points.
% The inverse images of the derivatives are simply added as constraints
% to the interpolation of step 2A.

\section{Designing the curve in Euclidean space}
\label{sec:eucdesign}

Once the data points have been mapped into Euclidean space,
the next step is to design the curve in Euclidean space.
This involves interpolating a curve through the 
inverse image lines $M^{-1}(p_i)$.
We will reduce this to point interpolation, 
by choosing one point per line
and applying conventional interpolation through a set of points.
Each point should be chosen wisely, taking full advantage of our
flexibility to choose any point on the line.
A more precise statement of step (2b) of our algorithm
for curve design on a surface is now:
%
\begin{description}
\item[(i)] 
	Intelligently choose $k$ points
	$\{q_i\}$ on the $k$ lines $\{M^{-1}(p_i)\}$.
\item[(ii)]
	Design a rational curve $C$ in Euclidean space,
	interpolating $\{q_i\}_{i=1,\ldots,k}$, using classical techniques.
\end{description}

Consider the choice of points $\{q_i\}$.
Shorter curves on \Sn{3}\ are
desirable, since they are associated with low covariant
acceleration (see Section~\ref{sec:cov}).
Moreover, unnecessary movement in Euclidean space
may be reflected in unnecessary movement in Riemannian space.
(The connection between movement in $\Re^4$ and movement on \Sn{3}\ 
is not tightly coupled, as witnessed by the extreme case of motion along
an inverse line in $\Re^4$ which causes no motion on \Sn{3}.
However, one can certainly
say that the avoidance of motion in Euclidean space leads to the avoidance
of motion in Riemannian space.)
Therefore, we would prefer the curve in Euclidean space to move
efficiently between the lines $\{M^{-1}(p_i)\}$.
It would seem that an optimal choice of points	
$\{q_i\}$ would be a set of closest points:
	
\begin{itemize}
\item On the first inverse line, let $q_1$ be the intersection
of the line $M^{-1}(p_1)$ and \Sn{3}.
\item On subsequent lines, let $q_i$ be the closest point to $q_{i-1}$.
\end{itemize}

Unfortunately, a closer look at this solution reveals some problems.
The first problem is that the points $\{q_i\}$ begin to fall
into the origin, continually attracted inwards by their shortness criterion.
This problem is annoying, but it can be solved by decomposing
the data set into several parts (see Section~\ref{sec:divandconq}).
A more intractable problem is associated with the addition of derivative
data to the curve design problem.

It is desirable for the design of an interpolating curve to allow the 
addition of derivative as well as position data.
For example, it may be useful to enforce end tangents or curvatures,
and derivative information may even be known at intermediate points.
Moreover, later developments in our study of interpolating curves on \Sn{3}\ will
reveal that interpolation of derivative information is not only a desirable 
luxury but, in some unusual cases, a necessity for the correct 
interpolation of position data alone (Section~\ref{sec:avoid}).
It is this interpolation of derivative data that causes particular problems
with the choice of closest points $\{q_i\}$.

By choosing $q_i$ based on the `closest' criterion,
the position of $q_i$ depends on the position of a {\em neighbour} $q_{i-1}$,
and thus on the position of not only the associated data point $p_i$ on the
surface, but also the position of a neighbouring data point $p_{i-1}$.
We would prefer a choice of $q_i$ that only depends on the corresponding
data point $p_i$ on the surface.
The neighbour-dependency of closest-$\{q_i\}$
causes substantial difficulties when translating tangents in Riemannian space
to tangents in Euclidean space: 
we basically need knowledge of a neighbourhood of $p_i$ in Riemannian space
to build the tangent at $q_i$ in Euclidean space.
That is, we need a curve on the surface in order to build the tangents
in Euclidean space!
This is clearly backward.
There is some possibility of solving this apparently self-referential problem,
% recursive definition?
but it involves a sophisticated solution and is a source for future work.
%
% soln: build first curve, say on points 1-15;
% then build tangent at 15 as follows: build optimal rotation for 13-30
% (that is add in last couple of points of previous subcurve);
% rotate first curve using this 13-30 rotation: the neighbourhood of the
% curve around the endpoint 15 will be safely away from pole;
% compute M^{-1} (first curve) and measure tangent at endpoint 15 off this curve.
% This reduces the problem to finding M^{-1} (curve).
% This reduces to finding $\alpha(t)$ for the inverse curve.
% The set of inverse lines forms a ruled surface.
% We are interested in a line of regression of this ruled surface,
% which follows the closest point at each stage (I believe).
% The computation of this line of regression is equivalent to computing
% the closest point on every line.
% Once \alpha(t) is known, we could determine the Bezier curve rep of the
% curve alpha(t) (x2(t),x3(t),x4(t),1-x1(t))$, or we can simply 
% differentiate $\alpha(t)(x2(t),x3(t),x4(t),1-x1(t))$ and evaluate at endpoint.
%
% WARNING: Choice of closest point is susceptible to spiraling of the inverse point into
% the origin.  See data66-1.
% Another motivation to decompose input into subsets.
% About 15 points seems to be the maximum before spiraling reaches origin.
%

% For clarity, consider a new interpretation of our problem:
% notice that by Theorem~\ref{thm:inverse},
% \[
%	q_i = \alpha_i * \mbox{defining point}(p_i),\ \ \alpha_i \in \Re, \ \alpha_i \neq 0.
% \]
% so the construction of $q_i$ can be reduced to the construction of $\alpha_i$.

A good alternative to the `closest point' is the `point on \Sn{3}'.
That is, suppose that we choose $q_i$ to be the intersection of
$M^{-1}(p_i)$ with \Sn{3}.
Then, $q_i$ is almost always close to $q_{i-1}$
and is a good approximation to the closest point to $q_{i-1}$
(especially when the data points on \Sn{3}\ are not widely spaced, which
is the common scenario in motion control).
\marginpar{{\bf FORMALIZE.}}
Moreover, $q_i$ is dependent only on $p_i$, not any of its neighbours.

\begin{lemma}
If $p_i = (x_{1i}, x_{2i}, x_{3i}, x_{4i}) \in \Sn{3}$, 
\[
	M^{-1} \cap \Sn{3} 
	= \frac{(x_{2i}, x_{3i}, x_{4i}, 1 - x_{1i})}{\sqrt{2-2x_{1i}}}
\]
\end{lemma}
\prf
$\| (x_{2i}, x_{3i}, x_{4i}, 1 - x_{1i}) \|
 = \sqrt{x_{2i}^2 + x_{3i}^2 + x_{4i}^2 + (1 - x_{1i})^2}
 = \sqrt{2-2x_{1i}}$
since $p_i \in \Sn{3}$.
\QED

\noindent This allows a further refinement of step (2b), the design of the
curve in Euclidean space:
%
\begin{description}
\item[(2b)]
	Design a rational curve $C$ in Euclidean space,
	interpolating $\{M^{-1}(p_i) \cap \Sn{3}\}_{i=1,\ldots,k}$.
\end{description}

Ironically, this refinement implies
that the curve design in Euclidean space is also interpolating points 
on the surface \Sn{3}\ (although different points than the data points):
the major difference is that the design is no longer constrained
to the surface.

Clearly, one could imagine the choice of different $q_i$ to satisfy other
optimality criteria.
A full tapping of the inherent flexibility in the choice of $q_i$
and a full understanding of the association between curves in Euclidean
and Riemannian space is a fascinating topic for further study.

\section{Mapping back to Riemannian space}
% \subsection{Mapping the curve back to the surface}
\label{sec:curveimage}

After the curve is designed in $\Re^4$, it must be mapped back to a curve
on \Sn{3}\ using $M$.
The following theorem shows how this mapping is done, segment by segment,
for the important case of a cubic Bezier curve in $\Re^4$.
We concentrate on the cubic Bezier curve since classical design of an 
interpolating curve in $\Re^4$ will generate
a cubic polynomial curve (see Chapter 9 of \cite{farin97}).
The image of other curves under $M$ can be computed similarly.

\begin{theorem}
\label{sextic}
Let $c(t)$ be a cubic Bezier curve in 4-space with
control points $b_i = (b_{i1},b_{i2},b_{i3},b_{i4})$, $i=0,\ldots,3$.
The image of $c(t)$ under $M$ is a rational sextic Bezier curve with 
weights $w_k$ ($k = 0, \ldots, 6$):
\begin{equation}
\label{eq:weights}
w_k = \sum_{\begin{array}{c} \mbox{\footnotesize{$0 \leq i \leq 3$}} \\ 
			     \mbox{\footnotesize{$0 \leq j \leq 3$}} \\ 
			     \mbox{\footnotesize{$i+j=k$}}
			     \end{array}}
        \frac{\choice{3}{i} * \choice{3}{j}}{\choice{6}{k}}
	(b_{i1} b_{j1} + b_{i2} b_{j2} + b_{i3} b_{j3} + b_{i4} b_{j4})
\end{equation}
and control points $c_k$:
\begin{equation}
\label{eq:control-pts}
c_k = \frac{1}{w_k} 
      \sum_{\begin{array}{c} \mbox{\footnotesize{$0 \leq i \leq 3$}} \\ 
			     \mbox{\footnotesize{$0 \leq j \leq 3$}} \\ 
			     \mbox{\footnotesize{$i+j=k$}}
			     \end{array}} 
        \frac{\choice{3}{i} * \choice{3}{j}}{\choice{6}{k}}
	\left( \begin{array}{c}
            b_{i1} b_{j1} + b_{i2} b_{j2} + b_{i3} b_{j3} - b_{i4} b_{j4} \\
            2b_{i1} b_{j4} \\
            2b_{i2} b_{j4} \\
            2b_{i3} b_{j4} 
	\end{array} \right)
\end{equation}
\end{theorem}
\prf
This is a simple application of the product rule of Bernstein polynomials
\cite{farin97}.
We will work in projective space, just as in the proof of 
Theorem~\ref{thm:inverse}.\\
Let $M(c(t)) = M(\sum_{i=0}^3 B_i^3(t) b_{i}) 
:= (m_1(t),m_2(t),m_3(t),m_4(t),m_5(t))$, 
where $B_i^n(t)$ is the $i^{th}$ Bernstein polynomial of degree $n$.
The coordinate $m_5$ can be simplified using the product rule 
of Bernstein polynomials, as follows:
\[ m_5(t) =  [\sum_{i=0}^3 B_i^3(t) b_{i1}]^2 + 
	\ldots + [\sum_{i=0}^3 B_i^3(t) b_{i4}]^2
     =   \sum_{i=0}^3 \sum_{j=0}^3 
	\frac{\choice{3}{i} * \choice{3}{j}}{\choice{6}{i+j}}
       B^6_{i+j}(t) (b_{i1} b_{j1} + \ldots + b_{i4} b_{j4})
\]
Letting $k=i+j$, this becomes
\[ \sum_{k=0}^6 B_k^6(t) 
	\sum_{\begin{array}{c}  \mbox{\footnotesize{$0 \leq i \leq 3$}} \\ 
			     \mbox{\footnotesize{$0 \leq j \leq 3$}} \\ 
			     \mbox{\footnotesize{$i+j=k$}}
			     \end{array}} 
	\frac{\scriptchoice{3}{i} * \scriptchoice{3}{j}}{\scriptchoice{6}{k}}
	(b_{i1} b_{j1} + \ldots + b_{i4} b_{j4}) \]
Computing the other coordinates analogously yields
\[ M(c(t)) = 
   \sum_{k=0}^6 B_k^6(t)
	\sum \frac{\choice{3}{i} * \choice{3}{j}}{\choice{6}{k}}
	\left( \begin{array}{c}
            b_{i1} b_{j1} + b_{i2} b_{j2} + b_{i3} b_{j3} - b_{i4} b_{j4} \\
            2b_{i1} b_{j4} \\
            2b_{i2} b_{j4} \\
            2b_{i3} b_{j4} \\
            b_{i1} b_{j1} + b_{i2} b_{j2} + b_{i3} b_{j3} + b_{i4} b_{j4}
	\end{array} \right) \]
% \[         = \sum_{k=0}^6 B_k^6(t) 
%	\sum_{\begin{array}{c} 0 \leq i \leq 3 \\ 
%			     0 \leq j \leq 3 \\ 
%			     i+j=k
%			     \end{array}} 
%       \frac{\choice{3}{i} * \choice{3}{j}}{\choice{6}{k}}
%	\left( \begin{array}{c}
%           b_{i1} b_{j1} + b_{i2} b_{j2} + b_{i3} b_{j3} + b_{i4} b_{j4} \\
%            w_k (\frac{b_{i1} b_{j1} + b_{i2} b_{j2} + b_{i3} b_{j3} - b_{i4} b_{j4}}{w_k}) \\
%            w_k (\frac{2b_{i1} b_{j4}}{w_k}) \\
%            w_k (\frac{2b_{i2} b_{j4}}{w_k}) \\
%            w_k (\frac{2b_{i3} b_{j4}}{w_k})
%	\end{array} \right) \]
%
which is a sextic rational Bezier curve with weights (\ref{eq:weights}) and 
control points (\ref{eq:control-pts}).
\QED

Notice how the map $M$ reveals itself in the formula for the control points
and weights, its numerator in the control points and its denominator
in the weights.
The remaining components of the formulae (\ref{eq:weights}-\ref{eq:control-pts})
are a reflection of the product rule for Bernstein polynomials.

\section{Avoiding the pole}
\label{sec:avoid}

The design of our curve is essentially complete.
However, we must refine one of the steps in the interests of stability.
The mapping of the data into Euclidean space (step 2a) is unstable
if any of the data is close to the pole $(1,0,0,0)$.
In this section, we shall show how the data points $\{p_i\}_{i=1,\ldots,k}$
can be moved away from the pole for robust curve design.

\subsection{Ill-conditioning near the pole}
\label{sec:ill}

$M^{-1}$, the map to Euclidean space, is not well-behaved near 
its pole $(1,0,0,0)$.
Recall that the image of a point under $M^{-1}$ is a line through the
origin and the defining point of $M^{-1}(p)$ (see Theorem~\ref{thm:inverse} 
and Definition~\ref{defn:pole}).
As a point $p$ on \Sn{3}\ approaches the pole $(1,0,0,0)$,
the defining point of $M^{-1}(p)$ approaches the origin:
\[ 
\mbox{lim}_{(x_1,x_2,x_3,x_4) \rightarrow (1,0,0,0)} (x_2,x_3,x_4,1-x_1)
= (0,0,0,0) 
\]
Consequently, the defining point becomes ill-conditioned:
small motions of $p$ can cause large motions of $M^{-1}(p)$.
This is analogous to the solution of a linear
system when the condition number of the matrix becomes large,
and the linear system becomes very sensitive to perturbation of the matrix.

\begin{example}
Suppose a data point $p$ on \Sn{3}\ is at distance $d$ from the pole,
and thus the defining point of $M^{-1}(p)$ is at distance $d$ from the
origin.
Then a motion of $d+\epsilon$ of the data point (corresponding to a motion
of the defining point directly towards the origin) can yield
a 90-degree change of orientation of the line $M^{-1}(p)$.
\end{example}

In other words, near the pole on \Sn{3}, there is little correspondence 
between position of $M^{-1}(p)$ in Euclidean space 
and position of $p$ on \Sn{3}.
This leads to curves on \Sn{3}\ that jump wildly about the pole 
(Figure~\ref{fig:wild}).
We conclude that we want to move data points away from the pole,
an issue addressed in the next section.
We end this section with an empirical definition of how close to the pole
is `too close'.

\begin{figure}
\vspace{2in}
\caption{Unstable behaviour near the pole}
% example of bad behaviour at pole if no rotation applied away from pole
% data/...
\label{fig:wild}
\end{figure}

\begin{defn2}
\label{defn:far}
A point $p \in \Sn{3}$ is {\bf too close to the pole}
if the angle formed by the vectors $p$ and $(1,0,0,0)$ is smaller
than 30 degrees ($\frac{\pi}{6}$ radians). 
This is an empirical definition, arising from experimental evidence
that points closer to the pole lead to undesirable behaviour in the curve.
% 30 = PI/6; PI/7 is definitely too small (lowerbound), as witnessed by data5-1
% using TOOCLOSEDIST = PI/7;
% we have not found an example that is not treated well by PI/6.
\end{defn2}

\begin{example}
data5-1 using $\pi/7$ vs. $\pi/6$.
\end{example}

\Comment{
Another way of seeing this ill-conditioning is by observing the behaviour
of $M$ near the origin.
$M$ is undefined at the origin, due to the division by zero.
However, $M$ is also badly behaved {\bf near} the origin.
For example, consider approaching near the origin along the line $(0,0,t,.01)$.
In moving the short distance from $(0,0,.01,.01)$ to $(0,0,0,.01)$, 
the image under $M$ moves all the way from $(0,0,0,1)$ to $(-1,0,0,0)$.
That is, a move of distance $.01$ in $\Re^4$ maps to a move of angular distance
$\frac{\pi}{2}$ on \Sn{3}.
The portion of a curve in $\Re^4$ that is near the origin
will be stretched as it is mapped under $M$.
This is not desirable, as shown in Figure~\ref{fig:tooCloseToPole}.

This is not quite as strong or direct an argument for ill-conditioning as the previous
argument, since there is nothing forcing the user to use preimage points 
near the origin.
However, notice that if you do not stay close to the origin, 
then the large changes in the line with small changes on \Sn{3} mean that 
small distances on \Sn{3} will be filled by long curves (and thus not optimal
geodesics) since the preimage curve will be necessarily long.
}

\subsection{Choosing a rotation}

Random choice

\subsection{The best rotation}
\label{sec:best}

We shall use a single rotation to move the data points far away from the 
pole $(1,0,0,0)$.
Once the curve on \Sn{3}\ has been robustly designed, it will be rotated 
back on \Sn{3}\ by the same amount, 
yielding the desired curve.\footnote{Rotation of a rational 
	Bezier curve can be achieved simply by
	rotation of its control polygon.
	This is another illustration of an advantage of rational curves.}

First, does such a rotation exist?
Recal that a data point is considered too close to the pole if it is
within an angular distance of 30 degrees from the pole.
Our problem reduces to finding a point $p$ on \Sn{3}\ such that no data
point is within 30 degrees of $p$:
then a rotation of $p$ to the pole will move all data points a safe
distance from the pole.
It is theoretically possible that the data points are so densely packed
on the sphere that no such empty region surrounding a point $p$ exists,
but the requisite number of data points is huge.
We need a data point covering every empty region of the sphere.
Consider the maximal region of \Sn{3}\ that a single data point can
`cover' (forcing the desired point $p$ to look elsewhere).
A data point covers \Sn{3}\ in a region 30 degrees wide about it,
which has the following surface area.
On \Sn{2}, this region is 
\[
  \int_0^{2\pi} \int_0^{\frac{\pi}{18}}  \sin \phi d\phi d\theta
  = .0000289
\]
The entire surface area of \Sn{2}\ is $4\pi$ or $---$,
so the fraction of \Sn{2}\ covered by each data point is at most
$\frac{.0000289}{4\pi}$,
or more than $431,073$ data points are needed to densely pack \Sn{2}\
so that no rotation can successfully move all data points safely away
from the pole.
Even more data points are needed to densely pack \Sn{3} (ANALYZE).
Therefore, the above approach will work for data sets of size 
$431,072$ or less.

There are various ways to find a candidate empty region of \Sn{3}\
to rotate to the pole.
Since there are so many empty regions in \Sn{3} (as revealed by the
above analysis), a random choice is a simple and effective way to 
choose an empty region.
That is, we randomly choose a point on \Sn{3}, test if it is
sufficiently far away from all data points (if not randomly choose
another, and so on), and rotate this point to the pole.
The results in Section~\ref{sec:results} support the effectiveness
of this random choice of empty region.

CAN PROBABLY REMOVE REST OF THIS SECTION.
COMPUTATION OF COVARIANCE MATRIX IS OVERKILL, UNLESS RANDOM
CHOICE SHOWS IT TO BE NECESSARY.
MAKE SURE RANDOM CHOICE IS IN 4-space, NOT OUR 3-space USED
FOR VISUALIZATION.

Section~\ref{sec:divandconq} will deal with the case where
this rotation does not successfully move all of the points far enough
from the pole.

What single rotation will move the entire set of data points far from the pole?
It makes sense to find a good representative for all the data
points and rotate it farthest away from the pole.
We choose the best-fitting plane.
The added benefit of this choice is that the best-fitting plane
can be computed efficiently using the covariance matrix.

\begin{defn2}
The {\bf best-fitting plane} of a set of points $\{r_i\}_{i=1,\ldots,k}$
is the plane that minimizes $\sum_{i=1}^k \mbox{dist}^2 (r_i,\mbox{plane})$.
The dispersion (or {\bf variance}) of a set of scalars
$\{x_i\}_{i=1,\ldots,k}$ is $\sum_{i=1}^k \frac{(x_i -
\bar{x})^2}{k-1}$, where $\bar{x} = \sum_{i=1}^k x_i / k$ is the mean.
This is a measure of the spread of the sample.
The dispersion of a set of points in the direction $v$ is the dispersion
of the orthogonal projection of the points onto $v$.
The {\bf covariance matrix} of a set of points $\{r_i\}_{i=1,\ldots,k}$ 
is $\sum_{i=1}^k \frac{(r_i - \bar{r})(r_i - \bar{r})^T}{k-1}$,
where $\bar{r}$ is the sample mean of the points.
\end{defn2}

The normal of the best-fitting plane of the data points $\{p_i\}_{i=1,\ldots,k}$
is the direction of minimum dispersion of $\{p_i\}$.
This is the direction that the plane can best afford to ignore,
and hence the normal direction.
The covariance matrix is a generalization of dispersion to many dimensions.
Not surprisingly, it can be used to compute properties of dispersion.
In particular, the direction of minimum dispersion of 
the point set $\{p_i\}$ is the minimum 
eigenvector (say $v_{min}$) of the covariance matrix of $\{p_i\}$ \cite{ballard82}.
Hence, the best-fitting plane is the plane through the sample mean
with normal $v_{min}$.

To maximize the distance of the best-fitting plane from the pole,
its normal should be oriented towards the pole (Figure~\ref{fig:avoid}).
We need to decide whether to rotate $+v_{min}$ or $-v_{min}$ to the pole
$(1,0,0,0)$, choosing the rotation that moves the best-fitting plane
farthest away from the pole.
Under either rotation, the best-fitting plane 
will be rotated to a plane $x_1 = c$ for some $c \in \Re$.
Since the sample mean lies on the best-fitting plane,
we can compute its image under the two rotations
(or just the $x_1$-coordinate of its image) 
and choose the rotation that yields the smaller $x_1$-coordinate.

\begin{example}
\label{eg:resolve}
The choice of $\pm v_{min}$ is an important one.
Consider a set of data points on the plane $x_1=.9$.
The best-fitting plane is $x_1=.9$
and the direction of minimum dispersion is $\pm(1,0,0,0)$.
Rotating $(1,0,0,0)$ to the pole does not move the points,
leaving them undesirably close to the pole,
but rotating $(-1,0,0,0)$ to the pole achieves the
goal of flipping the data points to the other side.
\end{example}

\begin{figure}
\vspace{1in}
\caption{Input set (a) before and (b) after rotation}
\label{fig:avoid}
\end{figure}
% figure of an example with best-fitting plane shown

The rotation matrix that rotates $v_{min}$
to $(1,0,0,0)$ is computed easily, using the fact that the rows
of a rotation matrix are mapped to the coordinate axes under the
rotation.
Thus, $\frac{v_{min}}{\|v_{min}\|}$ is the first row of the desired rotation matrix,
and the other rows are chosen orthogonal to $v_{min}$ using
cross product.

A COROLLARY OF THE CHOICE OF OPTIMAL ROTATION IS THAT THE METHOD
IS BI-INVARIANT (INDEPENDENT OF COORDINATE FRAME).
Give an argument and an example.

Figure of data5-1, data5-1-rot, data5-1-rot2.

Bi-invariant, since no matter where a point set is rotated,
it will be perturbed to the same optimal (for that point set)
orientation for mapping to Euclidean space.
Claim: no matter how you rotate the input, it is perturbed to the same
place before it is mapped to image space.
The map $M$ is not invariant to rotation, but the overall method is.
BUT WAIT: IS ROTATION OF THE DATA SET ON S3 EQUIVALENT TO APPLYING
A ROTATION TO THE ORIENTATIONS OF THE ACTUAL BODIES.
THAT IS, WE WANT TO BE ABLE TO ROTATE THE COORDINATE FRAME OF THE 
ANIMATION, EQUIVALENT TO ROTATING THE KEYFRAMES BY THE SAME AMOUNT:
HOW DOES THIS AFFECT THE QUATERNIONS?
Answer: it is equivalent to rotating the quaternions (see proof on
separate page): that is, multiplication of a quaternion by another
quaternion is equivalent to rotation of the quaternion on S3.

{\bf You need a tie-breaking scheme for assigning
the eigenvectors to positive and negative coordinate axes.
My present idea is to project the sample points onto
the eigenvector and choose the direction of the extreme
point as the positive axis.
If there is a tie, choose the extreme point with the smaller
index.  (Of course, this makes the method variant under
*reversal* of the data, which is not so nice.)
}

% \subsection{Bi-invariance}
%
% Claim: rotation of input set yields same shape curve, just rotated
%  i.e., invariant under pre-rotation
%  what about post-rotation?
%

\subsection{Sometimes no single rotation will do}
\label{sec:divandconq}

Since the best-fitting plane of $\{p_i\}_{i=1,\ldots,k}$ 
minimizes distance from the point set in the least-squares sense,
there may be outliers that are far from the best-fitting plane.
These outliers could be moved close to the pole by the rotation,
even as the best-fitting plane is moved far from it.
In other words,
if there are enough data points sampled densely across 
the sphere, there may be no single rotation that 
moves all data points sufficiently far from the pole.
For example, consider a sample of points on \Sn{3}\ such that
the maximum angular spacing between any point and its nearest neighbour
is less than 30 degrees.
Using Definition~\ref{defn:far},
there is no single rotation of this point set that moves every point far
from the pole.

\subsection{A restatement of the algorithm}
\label{sec:restatement}

The original algorithm of Section~\ref{sec:intro} 
has been significantly refined: we now restate it in its final form.
Notice that all of step (1) is a preprocessing phase, 
which is done once offline before the design of any curves.
Our design of a rational interpolating curve on \Sn{3}\ is as follows.

\vspace{.2in}

\centerline{{\bf Preprocessing phase}}
\begin{description}
\item[(1A)] Design a rational map to the surface \Sn{3}, $M:\Re^4 \rightarrow \Sn{3}$.
\item[(1B)] Compute the inverse map $M^{-1}: \Sn{3} \rightarrow \Re^4$.
%\item[(1C)] Design ${\cal M}^I$, a specialization of the inverse map
%		from lines to points.
\end{description}

\vspace{.2in}

\centerline{{\bf Design phase}}
\begin{description}
\item[(2A)] Compute a rotation $R$ of the data points $\{p_i\}_{i=1,\ldots,k}$
	away from the pole, and apply it to the points.
\item[(2B)] Map the rotated data points to Euclidean space, by computing 
	$\{M^{-1}(R(q_i)) \ \cap \ \Sn{3}\}_{i=1,\ldots,k}$.
\item[(2C)] Design a rational curve in Euclidean space
	interpolating $\{M^{-1}(R(q_i)) \cap \Sn{3}\}$.
\item[(2D)] Map the curve back to the surface using $M$.
\item[(2E)] Rotate the subcurve back to its original position, using $R^{-1}$.
\end{description}

To consolidate our understanding of the algorithm before moving on to its
analysis, we will now provide some examples.

-EXAMPLES-

Note {\bf Visualization of 4 dimensions}
Visualization mode x3=0 quaternions.  Visualization of 4D in 3D.  
(using input with x3=0; see VISUALIZE).
Points of $x_3=0$ are mapped to points of $x_2=0$ in Euclidean space,
which are then mapped back to points of $x_3=0$ on S3 again.
Therefore, we can visualize in 3d using data samples in $x_3=0$.

Recall that these figures are slightly misleading, since they are
three-dimensional images of a four-dimensional scene.

\section{Results}
\label{sec:results}

\subsection{A quality measure}
\label{sec:cov}

\cite{barr92} proposed that covariant acceleration
be used as a measure of the quality of a quaternion spline:
the smaller the covariant acceleration, the more desirable the curve.
%
\begin{defn2}
Let $a,b \in \Re^n$.
$a \setminus b = a - (\frac{a \cdot b}{b \cdot b}) b$.
The {\bf covariant acceleration} $C$ of a curve segment $c(t)$, $t \in I$, is 
$\int_I \| c''(t) \setminus c(t)\| \ dt$.
\end{defn2}
%
\cite{barr92} provides a good motivation for the avoidance of covariant
acceleration, as follows.
The acceleration of a curve on a surface can be decomposed into
normal and covariant components: normal acceleration is necessary,
as it keeps the curve on the surface, but covariant acceleration
is not strictly necessary.
%
% They built a quaternion spline that minimized covariant acceleration,
% through constrained optimization, as their interpretation 
% of the `optimal' curve.
% Incidentally, they
% used the term `tangential acceleration', but we prefer to use the 
% more classical term, covariant acceleration.
Low covariant acceleration can also be motivated by appealing to the design of 
geodesic curves.
Other factors being equal, we prefer shorter curves.
This is especially true when using the quaternion spline to control
motion.
An object should not experience unnecessary rotation: loop-the-loops
or gyration should not be introduced unless explicitly designed into
the motion.
Since the metric of \Sn{3}\ is the same as the metric of the rotation
matrix group SO(3) \cite{misner73}, arc length on \Sn{3}\ represents the
change in orientation of the object controlled by this curve.
Within the context of other constraints, 
we want to minimize any unnecessary spinning of the object.
This leads to a desire to minimize covariant acceleration since
geodesics, the shortest curves through a set of points,
have zero covariant acceleration \cite{thorpe79}.
(The geodesic itself is not attainable as a rational curve, in general,
and would not necessarily be desirable even if it was,
as the rigidity of perfection necessarily ignores all other
constraints.)

We will use covariant acceleration as a measure of the relative quality
of our curves.
We compute an approximation by discretely sampling the curve:
% and computing the covariant acceleration at each of these samples:
\[ \int_I \| c''(t) \setminus c(t)\| \ dt
\approx \sum_{i=0}^N 
	\| c''(t_0 + i \delta) \setminus c(t_0 + i \delta) \|\  \delta.
\]

\subsection{The comparison}
\label{sec:comparison}

Our first quality comparison explores the advantages of the flexibility 
to choose any point on the lines $M^{-1}(p_i)$.
We compare several strategies:
(1) a choice of the closest point (the first method proposed in
Section~\ref{sec:eucdesign})
(2) the chosen method, where the point is chosen as the intersection with
\Sn{3}, and
(3) a choice of the defining point of $M^{-1}(p_i)$ (which is basically
an arbitrary choice of point on the line in Euclidean space).
We expect these choices to be progressively worse.
Recall that (1) was not used since it has trouble with the definition
of tangents and large data sets.
Therefore, we use small data sets that do not require subdivision
for this comparison.

\begin{table}[h]
\label{tab:cov}
\begin{tabular}{|c|c|c|c|c|}  	\hline
Data set & \multicolumn{3}{c|}{Covariant accel.\ of Euclidean curve} & Time \\ \hline
	 & Closest & On \Sn{3} & Defining point & \\ \hline
data100-1 & 38.3 & 220.3 & &  \\ \hline
data100-2 & 30.8 & 201.2 & 327.5 & \\ \hline
test4-100 & 20.7 & 329.5 & 400.6 & \\ \hline
\end{tabular}
\end{table}

We next compare the method of this paper with trim curves.
Recall that trim curves use the surface parameterization as the map
to the surface, and design the curve in the 
parameter space of the surface (Section~\ref{sec:intro}).
This comparison should capture the advantage of using a well-designed map to the
surface, especially the added power of a map from the domain $\Re^4$.
In order to design trim curves, we need a rational parameterization 
of \Sn{3}\ and we use the standard one:
\begin{equation}
\label{Snparam}
	S(t_1,t_2,t_3) = 
	\frac{1}{t_1^2 + t_2^2 + t_3^2 + 1} 
	(2t_1, 2t_2, 2t_3, t_1^2 + t_2^2 + t_3^2 - 1) \ \ \ \ t_i \in (-\infty,\infty)
\end{equation}

% inverse of this parameterization is stereographic projection

% A fairer comparison of trim curves to Euclidean curves might be to
% use a Bezier representation of the sphere.

The results from some of these tests are presented in Table~\ref{tab:cov},
which shows the consistent improvement of the new approach.
This behaviour was observed on many other trials as well.
The improvement was typically several orders of magnitude.
Record execution times too.

\begin{table}
\label{tab:cov2}
\begin{tabular}{|c|c|c|}  	\hline
Data set & \multicolumn{2}{c|}{Covariant acceleration} \\ \hline
	 & Euclidean curve  & Trim curve \\ \hline
data100-1 & 38.3 & 401.4 \\ \hline
data100-2 & 30.8 & 539.0 \\ \hline
test4-100 & 20.7 & 458.8 \\ \hline
\end{tabular}
\end{table}

COMPARE AGAINST A GOOD NONRATIONAL TECHNIQUE: RAVANI's LIE ALGEBRA
TECHNIQUE.
HOPEFULLY IS COMPARABLE IN QUALITY EVEN TO NONRATIONAL METHODS.
COMPARE TO SPHERICAL BIARC RATIONAL SOLUTION: SHOULD BLOW IT OUT
OF THE WATER.  NO, DON'T DIGNIFY BIARC BY COMPARING IT.

We now compare to the nonrational Ravani's method.
We can only use small inputs, since the intermediate values in the
Lie method (led by u) become progressively larger at an explosive rate,
overflowing after about 15 points.
However, these small inputs are enough to show that the proposed
rational method performs well against the nonrational Lie method.


\begin{table}
\label{tab:cov3}
\begin{tabular}{|c|c|c|}  	\hline
Data set & \multicolumn{2}{c|}{Covariant acceleration} \\ \hline
	 & Euclidean curve & Lie curve \\ \hline
data5-1 & 5.551 & 12.345 \\ \hline
data5-2 & 3.62  & 21.18 \\ \hline
data10-1 & 13.33 & 303.31 \\ \hline
data10-2 & 8.74 & 337.2 \\ \hline
\end{tabular}
\end{table}

{\bf Analysis of perturbation technique: 
1) random choice: how many attempts before success?  quality of curve?
2) optimal choice: how many times does it not find successful perturbation? 
quality of curve?}

{\bf Argue that double covering of \Sn{3}\ by SO(3) doesn't affect
practical modeling of quaternion splines and put this concern to rest
once and for all.  Basic idea: if you change orientation by more than PI
in one step, then it should be arbitrary which way you travel to that
orientation; this large change of orientation should not be allowed
in almost all cases anyway, since it leaves too much to the imagination.} 

{\bf Input conventions}
Sampling criterion on input.  % MAXANGLEDIFF in qspline.c
See 4/21/97 comment in s3spline.c history section for motivation behind
sampling criterion. (Do we need to?)

\section{Conclusions and future work}

The image-space approach to curve design on surfaces offers
added flexibility in the choice of map and the choice of interpolation
over the classical trim-curve approach to curve design on surfaces.
Using the important case of curves on \Sn{3}, 
we have shown that this added flexibility improves curve design.
By choosing the natural map rather than the surface parameterization,
better curves result, so map flexibility is an improvement.
By choosing the closest-point sequence of points on the inverse lines
of the generalization of stereographic projection
rather than the unique surface parameterization inverse,
better curves result, so interpolation flexibility is an improvement.
Much more study can be done on the potential of map and interpolation
flexibility, but the message is clear that the 
image-space approach in its most general form offers a promising
method for the design of curves on surfaces.

Despite the specific advantages of the quaternion spline
over earlier quaternion splines (rationality, generality, and efficiency), 
our main purpose in presenting this material
is an example of the more general Euclidean-space method for designing
curves on surfaces.

% By using rational map to $\Re^4$ rather than parameter space,
%	there are two added degrees of freedom for curve design:
%	interpolating curves rather than points in image space
%	(choose any point along the line = one dof)
%	and choosing *any* rational map to \Sn{3}\ rather than 
%	automatically using the parameterization map.

Mention relationship of quaternions to 4-squares theorem and Hamilton/Gauss.

Open problems: 
	1. We have shown that other choices of map {\em can} beat the surface
	parameterization.
	But what is the {\em best} map for a given surface and type of curve (e.g., interpolating)?
	2. We have shown a good map for the design of curves on \Sn{3}.
	But what about the design of curves on other surfaces?
	3. General problem of designing a curve that interpolates curves, rather than points.
	4. Determining tangent in Euclidean space when using closest point on inverse line.
		(Involves computing $\alpha(t)$.)
		[See comments embedded in paper above.]
	5. Compare eager and lazy approaches to input subdivision,
	   as well as maximal/optimal sizes for the subset in eager evaluation
	6. $C^2$ continuous splines, by passing 2nd derivative back too
		(still easy to set 2nd derivative, filling the 2nd dof of the cubic spline)

In the future, we want to explore the direct interpolation of the set
of inverse image curves by a curve.

A topic for future study is the choice of an optimal value
for the subset size $k$ in lazy subdivision (Section~\ref{sec:divandconq})
and the tradeoff between lazy and eager subdivision.

Other optimization measures (strain energy, etc.)...
Another measure used in spacecraft dynamics is ---.

{\bf Explicitly minimizing covariant acceleration}

Covariant acceleration is zero at a point of a curve on \Sn{3}\ if the curve's
second derivative vector is orthogonal to the sphere.
We can try to enforce this restriction at data points,
since the desired second derivative vector for the quaternion spline
can be mapped to a desired second derivative vector in inverse space,
where design is carried out.
To interpolate a second derivative vector at every data point will require
a quintic Hermite curve in inverse space, which will map to a degree 10
curve on \Sn{3}.
One can keep a cubic curve in inverse space by simply interpolating
second derivative vectors at the endpoints, since there are two free
end conditions in a cubic B-spline.

\bibliographystyle{plain}
\begin{thebibliography}{Lozano-Perez 83}

\bibitem[Ballard 82]{ballard82}
Ballard, D. and C. Brown (1982)
Computer Vision.
Prentice-Hall (Englewood Cliffs, NJ).

\bibitem[Barr 92]{barr92}
Barr, A. and B. Currin and S. Gabriel and J. Hughes (1992)
Smooth Interpolation of Orientations with Angular Velocity Constraints
using Quaternions.
SIGGRAPH '92, 313--320.

\bibitem[Dietz 93]{dietz93}
Dietz, R. and J. Hoschek and B. Juttler (1993)
An algebraic approach to curves and surfaces on the sphere and
on other quadrics.
Computer Aided Geometric Design 10, 211-229.

\bibitem[Duff 85]{duff85}
Duff, T. (1985)
Quaternion splines for animating orientation.
1985 Monterey Computer Graphics Workshop, 54--62.

\bibitem[Farin 97]{farin97}
Farin, G. (1997)
Curves and Surfaces for CAGD: A Practical Guide (4th edition).
Academic Press (New York).

\bibitem[Foley 96]{foley96}
Foley, J. and A. van Dam and S. Feiner and J. Hughes (1996)
Computer Graphics: Principles and Practice (2nd edition in C).
Addison-Wesley (Reading, Massachusetts).

\bibitem[Golub 89]{golubvanloan89}
Golub, G. and C. Van Loan (1989)
Matrix Computations.
2nd edition, Johns Hopkins University Press (Baltimore).

\bibitem[Harris 92]{harris92}
Harris, J. (1992)
Algebraic Geometry: A First Course.
Springer-Verlag (New York).

\bibitem[Hoschek 92]{hoschekSeemann92}
Hoschek, J. and G. Seemann (1992)
Spherical splines.
Mathematical Modeling and Numerical Analysis, 26(1), 1--22.

\bibitem[Johnstone 95]{jjjw95}
Johnstone, J. and J. Williams (1995)
Rational Control of Orientation for Animation.
{\em Graphics Interface '95}, Quebec City, 179--186.

\bibitem[Johnstone 98a]{jj98a}
Johnstone, J.K. (1998)
A Characterization of Rational Maps of $\Re^n$ to \Sn{n-1}.
Manuscript.

\bibitem[Johnstone 98b]{jj98b}
Johnstone, J.K. (1998)
The Most Natural Map to the $n$-sphere.
Manuscript.

\bibitem[Kim 95]{kim95}
Kim, M.-J. and M.-S. Kim and S. Shin (1995)
A General Construction Scheme for Unit Quaternion Curves
with Simple Higher Order Derivatives.
SIGGRAPH '95, 369--376.

\bibitem[Kreyszig 63]{kreyszig63}
Kreyszig, E. (1963)
Differential Geometry.
Dover (New York).

\bibitem[Misner 73]{misner73}
Misner, C. and K. Thorne and J. Wheeler (1973)
Gravitation.
W.H. Freeman (San Francisco).

\bibitem[Nam 95]{nam95}
Nam, K.-W. and M.-S. Kim (1995)
Hermite interpolation of solid orientations based on a smooth blending
of two great circular arcs on SO(3).
Proc. of CG International '95.

\bibitem[Nielson 92]{nielson92}
Nielson, G. and R. Heiland (1992)
Animated rotations using quaternions and splines on a 4D sphere.
Programming and Computer Software, 145--154.

\bibitem[Nielson 93]{nielson93}
Nielson, G. (1993)
Smooth interpolation of orientations.
In Models and Techniques in Computer Animation, Springer-Verlag (New York),
75--93.

\bibitem[Park 95]{park95}
Park, F. and B. Ravani (1995)
Bezier Curves on Riemannian Manifolds and Lie Groups with
Kinematics Applications.
Transactions of the ASME 117, 36--54.

\bibitem[Park 97]{park97}
Park, F. and B. Ravani (1997)
Smooth Invariant Interpolation of Rotations.
ACM Transactions on Graphics 16(3), 277--295.

\bibitem[Pletinckx 89]{pletinckx89}
Pletinckx, D. (1989) 
Quaternion calculus as a basic tool in computer graphics.
The Visual Computer 5, 2--13.

\bibitem[Rama 97]{rama97}
Ramamoorthi, R. and A. Barr (1997)
Fast Construction of Accurate Quaternion Splines.
SIGGRAPH '97, Los Angeles, 287--292.

\bibitem[Schlag 91]{schlag91}
Schlag, J. (1991) Using geometric constructions to interpolate
orientation with quaternions.  In Graphics Gems II, Academic Press (New York),
377--380.

\bibitem[Shoemake 85]{shoemake85}
Shoemake, K. (1985) Animating rotation with quaternion curves.
SIGGRAPH '85, San Francisco, 19(3), 245--254.

\bibitem[Thorpe 79]{thorpe79}
Thorpe, J. (1979)
Elementary Topics in Differential Geometry.
Springer-Verlag (New York).

\bibitem[Wang 93]{wang93}
Wang, W. and B. Joe (1993)
Orientation Interpolation in Quaternion Space using Spherical Biarcs.
Graphics Interface '93, 24--32.

\bibitem[Wang 94]{wang94}
Wang, W. (1994)
Rational Spherical Curves.
Technical Report, Dept. of Computer Science, University of Hong Kong.

% \bibitem[Woo 97]{opengl97}
% Woo, M. and J. Neider and T. Davis (1997)
% OpenGL Programming Guide (2nd edition).
% Addison-Wesley Developers Press (Reading, Mass.),
% pp. 464-8.

\end{thebibliography}

\end{document}





% \section{Motion control}
% \label{sec:motion}
% \subsection{Quaternions and orientation}
% \subsection{Orientation keyframes for animation and robot motion}

% Call the new quaternion splines `Euclidean quaternion splines'.
% Call the general class of new interpolating curves on the surface
% `Euclidean curves on the surface'.

% {\bf Separate paper on applications of S3 splines to orientation control,
% sing examples from graphics and robotics.}


% \section{Curve interpolation}

% \section{Interactive editing of a curve}

