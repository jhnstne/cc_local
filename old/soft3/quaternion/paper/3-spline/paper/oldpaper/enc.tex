\documentstyle[12pt]{article}

\newif\ifFull
\Fullfalse

\makeatletter
\def\@maketitle{\newpage
 \null
 \vskip 2em                   % Vertical space above title.
 \begin{center}
       {\Large\bf \@title \par}  % Title set in \Large size. 
       \vskip .5em               % Vertical space after title.
       {\lineskip .5em           %  each author set in a tabular environment
        \begin{tabular}[t]{c}\@author 
        \end{tabular}\par}                   
  \end{center}
 \par
 \vskip .5em}                 % Vertical space after author
\makeatother

% default values are 
% \parskip=0pt plus1pt
% \parindent=20pt
\parskip=.2in
\parindent=0pt

\newcommand{\SingleSpace}{\edef\baselinestretch{0.9}\Large\normalsize}
\newcommand{\DoubleSpace}{\edef\baselinestretch{1.4}\Large\normalsize}
\newcommand{\Comment}[1]{\relax}  % makes a "comment" (not expanded)
\newcommand{\Heading}[1]{\par\noindent{\bf#1}\nobreak}
\newcommand{\Tail}[1]{\nobreak\par\noindent{\bf#1}}
\newcommand{\QED}{\vrule height 1.4ex width 1.0ex depth -.1ex\ \vspace{.3in}} % square box
\newcommand{\arc}[1]{\mbox{$\stackrel{\frown}{#1}$}}
\newcommand{\lyne}[1]{\mbox{$\stackrel{\leftrightarrow}{#1}$}}
\newcommand{\ray}[1]{\mbox{$\vec{#1}$}}          
\newcommand{\seg}[1]{\mbox{$\overline{#1}$}}
\newcommand{\tab}{\hspace*{.2in}}
\newcommand{\se}{\mbox{$_{\epsilon}$}}  % subscript epsilon
\newcommand{\ie}{\mbox{i.e.}}
\newcommand{\eg}{\mbox{e.\ g.\ }}
\newcommand{\figg}[3]{\begin{figure}[htbp]\vspace{#3}\caption{#2}\label{#1}\end{figure}}
\newcommand{\be}{\begin{equation}}
\newcommand{\ee}{\end{equation}}
\newcommand{\prf}{\vspace{-.3in}\noindent{{\bf Proof}:\ \ \ }}
\newcommand{\choice}[2]{\mbox{\footnotesize{$\left( \begin{array}{c} #1 \\ #2 \end{array} \right)$}}}      
\newcommand{\scriptchoice}[2]{\mbox{\scriptsize{$\left( \begin{array}{c} #1 \\ #2 \end{array} \right)$}}}
\newcommand{\tinychoice}[2]{\mbox{\tiny{$\left( \begin{array}{c} #1 \\ #2 \end{array} \right)$}}}
\newcommand{\ddt}{\frac{\partial}{\partial t}}
\newcommand{\Sn}[1]{\mbox{{\bf S}$^{#1}$}}
\newcommand{\calP}[1]{\mbox{{\bf {\cal P}}$^{#1}$}}

\newtheorem{rmk}{Remark}[section]
\newtheorem{example}{Example}[section]
\newtheorem{conjecture}{Conjecture}[section]
\newtheorem{claim}{Claim}[section]
\newtheorem{notation}{Notation}[section]
\newtheorem{theorem}{Theorem}[section]
\newtheorem{lemma}[theorem]{Lemma}
\newtheorem{corollary}[theorem]{Corollary}
\newtheorem{defn2}{Definition}[section]

\DoubleSpace

\setlength{\oddsidemargin}{0pt}
\setlength{\topmargin}{-.25in}	% technically should be 0pt for 1in
\setlength{\headsep}{0pt}
\setlength{\textheight}{8.75in}
\setlength{\textwidth}{6.5in}
\setlength{\columnsep}{5mm}		% width of gutter between columns

% -----------------------------------------------------------------------------

\title{Encyclopedia Galactica}
% History: begun 8/13/97.

\begin{document}
\maketitle
% ----------------------------------------------------------------------------

\section{Stereographic projection}

Stereographic projection is a map from \Sn{n} to $\Re^n$,
embedded in $\Re^{n+1}$ as a hyperplane.
A point is mapped to its projection from
a center of projection on \Sn{n}, called the pole of the stereographic projection,
to a hyperplane through the origin parallel to the pole's tangent plane (Figure~\ref{fig:stereo}).
Conventionally, the pole is the north pole $(0,\ldots,0,1)$
and consequently the hyperplane is $x_{n+1}=0$.
Stereographic projection $\sigma_q$ with pole $q$ and hyperplane $H$ is
a map
\[	\sigma_q : \Sn{n} - \{ q \} \rightarrow H \equiv \Re^n	\]

\begin{figure}
\vspace{2in}
\caption{Stereographic projection}
\label{fig:stereo}
\end{figure}

The inverse map to stereographic projection is a map onto the sphere
\[	\varphi_H : H \rightarrow \Sn{n} - \{ q \}	\]
Both maps are rational.

\begin{lemma}
Let $q$ be the pole $(0,\ldots,0,1)$ and $H$ the hyperplane $x_{n+1}=0$.
\[
	\sigma_q (x_1,\ldots,x_{n+1}) = 
	\frac{1}{1-x_{n+1}} (x_1, \ldots, x_n, 0)
\]
\[
	\varphi_H (x_1,\ldots,x_n,0) = 
	\frac{1}{x_1^2 + \cdots + x_n^2 + 1} 
	(2x_1, \ldots, 2x_n, x_1^2 + \cdots + x_n^2 - 1)
\]	% see thorpe79, p. 125 for latter formula
\end{lemma}
\prf
The line through $p = (x_1,\ldots,x_{n+1})$ and $q$, $tp + (1-t)q$,
intersects $H$ at $t = \frac{1}{1 - x_{n+1}}$.
The line through $(r,0)$ and $q$, $t(r,0) + (1-t)q$,
intersects \Sn{n} at $t=0,\frac{2}{\|r\|^2 + 1}$.
\QED

Reference: Thorpe (1979).
Use: Rational interpolation on 3-sphere paper (GI '95).

Additional comments: 
	- Circles are preserved.  (mentioned in Boehm/Paluszny paper I reviewed)

% ----------------------------------------------------------------------------

\section{Pythagorean $n$-tuples}

\begin{defn2}
{\rm 
A {\em Pythagorean $n$-tuple} ($n \geq 3$) is an $n$-tuple 
$(x_1,\ldots,x_n)$ that satisfies\\
$x_1^2 + \ldots + x_{n-1}^2 = x_{n}^2$.
If $x_i \in F$, we may refer to a Pythagorean $n$-tuple over $F$.
}
\end{defn2}

% ----------------------------------------------------------------------------

\bibliographystyle{plain}
\begin{thebibliography}{Dickson 52}

\bibitem[Thorpe 79]{thorpe79}
Thorpe, J.A. (1979) Elementary Topics in Differential Geometry.
Springer-Verlag (New York), p. 124.

\end{thebibliography}

\end{document}
