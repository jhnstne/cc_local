\documentstyle[12pt]{article}

\newif\ifFull
\Fullfalse

\makeatletter
\def\@maketitle{\newpage
 \null
 \vskip 2em                   % Vertical space above title.
 \begin{center}
       {\Large\bf \@title \par}  % Title set in \Large size. 
       \vskip .5em               % Vertical space after title.
       {\lineskip .5em           %  each author set in a tabular environment
        \begin{tabular}[t]{c}\@author 
        \end{tabular}\par}                   
  \end{center}
 \par
 \vskip .5em}                 % Vertical space after author
\makeatother

% default values are 
% \parskip=0pt plus1pt
% \parindent=20pt

\newcommand{\SingleSpace}{\edef\baselinestretch{0.9}\Large\normalsize}
\newcommand{\DoubleSpace}{\edef\baselinestretch{1.4}\Large\normalsize}
\newcommand{\Comment}[1]{\relax}  % makes a "comment" (not expanded)
\newcommand{\Heading}[1]{\par\noindent{\bf#1}\nobreak}
\newcommand{\Tail}[1]{\nobreak\par\noindent{\bf#1}}
\newcommand{\QED}{\vrule height 1.4ex width 1.0ex depth -.1ex\ \vspace{.3in}} % square box
\newcommand{\arc}[1]{\mbox{$\stackrel{\frown}{#1}$}}
\newcommand{\lyne}[1]{\mbox{$\stackrel{\leftrightarrow}{#1}$}}
\newcommand{\ray}[1]{\mbox{$\vec{#1}$}}          
\newcommand{\seg}[1]{\mbox{$\overline{#1}$}}
\newcommand{\tab}{\hspace*{.2in}}
\newcommand{\se}{\mbox{$_{\epsilon}$}}  % subscript epsilon
\newcommand{\ie}{\mbox{i.e.}}
\newcommand{\eg}{\mbox{e.\ g.\ }}
\newcommand{\figg}[3]{\begin{figure}[htbp]\vspace{#3}\caption{#2}\label{#1}\end{figure}}
\newcommand{\be}{\begin{equation}}
\newcommand{\ee}{\end{equation}}
\newcommand{\prf}{\vspace{-.3in}\noindent{{\bf Proof}:\ \ \ }}
\newcommand{\choice}[2]{\mbox{\footnotesize{$\left( \begin{array}{c} #1 \\ #2 \end{array} \right)$}}}      
\newcommand{\scriptchoice}[2]{\mbox{\scriptsize{$\left( \begin{array}{c} #1 \\ #2 \end{array} \right)$}}}
\newcommand{\tinychoice}[2]{\mbox{\tiny{$\left( \begin{array}{c} #1 \\ #2 \end{array} \right)$}}}
\newcommand{\ddt}{\frac{\partial}{\partial t}}
\newcommand{\Sn}[1]{\mbox{{\bf S}$^{#1}$}}
\newcommand{\calP}[1]{\mbox{{\bf {\cal P}}$^{#1}$}}

\newtheorem{rmk}{Remark}[section]
\newtheorem{example}{Example}[section]
\newtheorem{conjecture}{Conjecture}[section]
\newtheorem{claim}{Claim}[section]
\newtheorem{notation}{Notation}[section]
\newtheorem{theorem}{Theorem}[section]
\newtheorem{lemma}[theorem]{Lemma}
\newtheorem{corollary}[theorem]{Corollary}
\newtheorem{defn2}{Definition}[section]

% \font\timesr10
% \newfont{\timesroman}{timesr10}
% \timesroman

\SingleSpace

\setlength{\oddsidemargin}{0pt}
\setlength{\topmargin}{-.25in}	% technically should be 0pt for 1in
\setlength{\headsep}{0pt}
\setlength{\textheight}{8.75in}
\setlength{\textwidth}{6.5in}
\setlength{\columnsep}{5mm}		% width of gutter between columns

% -----------------------------------------------------------------------------

\title{Rational quaternion curves}
% Rational interpolation on the 3-sphere
\author{J.K. Johnstone \and J.P. Williams}
% ignore as much as possible the context of `pt on 3-sphere = unit quaternion = orientation'

% History: begun 7/7/97.

\begin{document}
\maketitle
\tableofcontents
\clearpage
\parskip=.2in
\parindent=0pt
\DoubleSpace

\begin{abstract}
The design of quaternion curves (or interpolating curves on the 3-sphere),
an important problem in motion control for animation and robotics,
has been solved in the literature using nonrational curves
or rational curves limited in degree and continuity.
In this paper, we develop a method for the design of quaternion
curves that are rational of arbitrary degree and continuity.
Unlike previous methods, which require the implementation of a new curve
system, the proposed method relies on the existing machinery of 
rational interpolating curves.

The major challenge of the new approach is the design of a rational
map from 4-space to the 3-sphere.
We examine several maps to the 3-sphere, including stereographic projection
and Hopf maps, and compare their quality with the chosen map,
a map motivated by Euler's 4-Squares Theorem.

The curves constructed by the new method are shown to be superior
to those constructed by other rational maps to the 3-sphere,
and superior or significantly more efficient than rational or 
nonrational curves designed by earlier methods.
\end{abstract}

Keywords: curves on surfaces, interpolation, 3-sphere,
	  orientation control, animation.

\clearpage

% ----------------------------------------------------------------------------

\section{Introduction}
\label{sec:intro}

Quaternion curves (curves on $S^3$) are widely used for
the interpolation of keyframe orientations.
Much previous literature, notably Shoemake \cite{}, Barr et al \cite{},
Kim et al \cite{}, and Ramamoorthi and Barr \cite{},
has developed nonrational curves on $S^3$.
Because of the computational efficiency of rational curves
and the fact that rational curves (such as NURBS) are the standard
curve in existing graphics and modeling systems,
the development of rational curves on $S^3$ has certain strong attractions.
Wang \cite{} and Nielson \cite{} constructed rational curves 
of limited degree and continuity.
In this paper, we develop a method for the design of interpolating
curves on $S^3$ (quaternion curves)
that are rational of arbitrary degree and continuity.

Suppose that $f:\Re^4 \rightarrow S^3$ is a rational map to the 3-sphere.
A rational curve on \Sn{3} interpolating the points 
$p_i$, $i=1,\ldots,m$ can be built as follows.
If $C(t)$ is a rational curve interpolating $f^{-1}(p_i)$,
then $f(C(t))$ is a rational curve on \Sn{3} interpolating $p_i$.

The strength of this approach is that, unlike the problem 
of design of rational interpolating curves constrained to surfaces,
the design of rational interpolating curves in $\Re^n$ is well understood.
In the proposed method, in effect the constraint to the 3-sphere is
enforced by the rational map $f$.

The above approach reduces the problem of rational curve design on the 3-sphere
to the design of rational maps to the 3-sphere.
These maps are the focus of most of the rest of this paper,
which is structured as follows.
The previous literature is presented in more depth in Section --.
The obvious choice for a map to the 3-sphere, stereographic projection,
is explored in Section --.
In Section --, we develop a superior map to the 3-sphere,
and compare its quality for 3-sphere curve design with stereographic projection
in Section --.
Other maps to the 3-sphere, Hopf maps, are considered in Section --.
Poles of the superior map to the 3-sphere are introduced in Section --,
as well as their undesirable effect on 3-sphere curves
and a method for their avoidance.
Several examples of 3-sphere curves generated by the proposed method
are presented in Section --.
In Section --, we compare our curves to the nonrational 
slerped curves of Shoemake \cite{},
the nonrational optimized curves of Barr et al \cite{} 
and Ramamoorthi and Barr \cite{},
and the rational biarc curves of Wang \cite{}.
Examples of applications that will benefit from rational 3-sphere curves
are given in Section --.
We conclude with ideas for future work in Section --.

% ----------------------------------------------------------------------------

\section{Basic definitions}

\begin{defn2}
{\rm 
A {\em rational polynomial} is a quotient of polynomials.
A {\em rational curve} is a parametric curve $C(t) = (x_1(t),\ldots,x_n(t)$ 
whose components $x_i(t)$ are all rational polynomials.
A {\em rational map} $h:\Re^n \rightarrow \Re^m$ is a map
$h(x_1,\ldots,x_n) = (H_1 (x_1,\ldots,x_n),\ldots,H_m (x_1,\ldots,x_n))$
whose components $H_i$ are all rational polynomials in $x_1,\ldots,x_n$.
}
\end{defn2}

Since a rational curve can always be expressed using a common denominator,
it can be interpreted as a polynomial curve in projective space:
\[ 
  (\frac{x_1(t)}{x_{n+1}(t)}, \frac{x_2(t)}{x_{n+1}(t)}, \ldots, 
   \frac{x_n(t)}{x_{n+1}(t)})  \rightarrow (x_1(t),\ldots,x_{n+1}(t))
\]
This notational convenience will be used throughout the paper,
for rational curves and for points (see (\ref{eq:M})).

\begin{notation}
{\rm 
Projective $n$-space (over the field of real numbers) will be denoted \calP{n}.
}
\end{notation}

\begin{defn2}
{\rm
The unit {\em $n$-sphere} $\Sn{n} \subset \Re^{n+1}$ 
is the unit sphere in $\Re^{n+1}$, $x_1^2 + \cdots + x_{n+1}^2 = 1$.
\Sn{1} is the unit circle, \Sn{2} is the unit sphere, and \Sn{3} is the
unit hypersphere in 4-space.
}
\end{defn2}

% ----------------------------------------------------------------------------

\section{Previous work}

Problems with optimization (give system to optimize) and slerping.

Disadvantages of unconstrained interpolation: indirect definition of curve;
e.g., more difficult to control secondary effects such as velocity,
since this must be done through the mirror of the map
(like da Vinci's trick of writing so that script is readable indirectly in a mirror,
not directly.)

Wang's approach is basically using our approach and the stereographic projection map.
He uses a different method to avoid the pole of this map
ahan we do (see Section~\ref{}).

curves on quadrics: especially Dietz et al

trimmed surfaces: idea of building curve in parameter plane
is similar to building curve in stereographic projection hyperplane,
since hyperplane is really parameter plane as standard parameterization
of sphere uses stereographic projection `chart' (see Thorpe for chart).

% ----------------------------------------------------------------------------

\section{Stereographic projection}

Stereographic projection is a map from \Sn{n} to $\Re^n$,
embedded in $\Re^{n+1}$ as a hyperplane.
A point is mapped to its projection from a center of projection $q$ on \Sn{n}
to a hyperplane through the origin parallel to q's tangent plane (Figure~\ref{fig:stereo}).
Conventionally, $q$ is the north pole $(0,\ldots,0,1)$
and consequently the hyperplane is $x_{n+1}=0$.
Stereographic projection $\sigma_q$ with pole $q$ and hyperplane $H$ is
a map
\[	\sigma_q : \Sn{n} - \{ q \} \rightarrow H \equiv \Re^n	\]

\begin{figure}
\vspace{2in}
\caption{Stereographic projection}
\label{fig:stereo}
\end{figure}

The inverse map to stereographic projection is a map onto the sphere
\[	\varphi_H : H \rightarrow \Sn{n} - \{ q \}	\]
Both maps are rational.

\begin{lemma}
Let $q$ be $(0,\ldots,0,1)$ and $H$ the hyperplane $x_{n+1}=0$.
\[
	\sigma_q (x_1,\ldots,x_{n+1}) = 
	\frac{1}{1-x_{n+1}} (x_1, \ldots, x_n, 0)
\]
\[
	\varphi_{x_{n+1}=0} (x_1,\ldots,x_n,0) = 
	\frac{1}{x_1^2 + \cdots + x_n^2 + 1} 
	(2x_1, \ldots, 2x_n, x_1^2 + \cdots + x_n^2 - 1)
\]	% see thorpe79, p. 125 for latter formula
Let $q$ be $(1,0,\ldots,0)$ and $H$ the hyperplane $x_1=0$.
\[
	\varphi_{x_1=0} (0,x_1,\ldots,x_n) =
	\frac{1}{x_1^2 + \cdots + x_n^2 + 1} 
	(x_1^2 + \cdots + x_n^2 - 1, 2x_1, \ldots, 2x_n)
\]
\end{lemma}
\prf
The line through $p = (x_1,\ldots,x_{n+1})$ and $q$, $tp + (1-t)q$,
intersects $H$ at $t = \frac{1}{1 - x_{n+1}}$.
The line through $(r,0)$ and $q$, $t(r,0) + (1-t)q$,
intersects \Sn{n} at $t=0,\frac{2}{\|r\|^2 + 1}$.
$\varphi_{x_1=0}$ can be derived from $\varphi_{x_{n+1}=0}$
by permutation of coordinates.
\QED

The map $\varphi_H$ can be used to generate rational curves on $S^3$,
using the approach of Section~\ref{sec:intro}.
(See Section~\ref{} for examples of its use.)
Is there another map to $S^3$ that generates better curves?
It turns out that there is.
We develop this improved map in the following section.

% ----------------------------------------------------------------------------

\section{An Eulerian map to the 3-sphere}

Let us return to first principles.
Consider a rational map $f : \Re^4 \rightarrow \Sn{3}$.
By the definition of rational maps, $f$ can be expressed as follows:
\[	f(x_1,x_2,x_3,x_4) = (\frac{g(x_1,x_2,x_3,x_4)}{k(x_1,x_2,x_3,x_4)}, 
			      \frac{h(x_1,x_2,x_3,x_4)}{k(x_1,x_2,x_3,x_4)},
			      \frac{i(x_1,x_2,x_3,x_4)}{k(x_1,x_2,x_3,x_4)},
			      \frac{j(x_1,x_2,x_3,x_4)}{k(x_1,x_2,x_3,x_4)}) \]
where $g,h,i,j,k$ are polynomials.
Since $f(x_1,x_2,x_3,x_4) \in \Sn{3}$, $\| f(x_1,x_2,x_3,x_4) \| = 1$ or
\begin{equation}
\label{eq:quint}
	g^2 + h^2 + j^2 + k^2 = l^2	
\end{equation}
The nature of the constraint (\ref{eq:quint}) suggests a number-theoretical 
approach to the construction of a map to the 3-sphere.\footnote{Since
	(\ref{eq:quint}) is a Pythagorean quintuple, one may think
	that the theory of Pythagorean tuples can be used.
	However, although there is a construction for all Pythagorean 
	triples \cite{kubota} and quadruples \cite{dietz}, 
	we are not aware of any such result for Pythagorean quintuples.}
The sum of four squares has been studied extensively in number theory,
motivated by the search for a proof that every natural number is the 
sum of the squares of four natural numbers(?) \cite{dickson},
a result originally conjectured by Fermat in 1659 and finally proved
by Lagrange in 1770.
An important result from this literature is Euler's Four-Squares 
Theorem \cite{dickson52}.

\begin{theorem}[Euler, 1748]		% May 4, 1748 for general form of theorem
% {dickson52, p. 277,530}; {ebbinghaus90, Numbers, p. 209} in slightly different form
% p. 210, Ebbinghaus for any commutative ring
% {dickson52, p. 318} for specialized form
% Euler proved 4 squares theorem in attempting to prove that 
% 	every natural # is sum of 4 natural #'s,
%	a result conjectured by Fermat (1659) and later proved by Lagrange (1770)
% specialized formula later independently established by Aida \cite{dickson52}. 
% 	(c. 1810) 

\begin{equation}
\label{eq:euler}
(a^2 + b^2 + c^2 + d^2)^2 = 
(a^2 + b^2 + c^2 - d^2)^2 + (2ad)^2 + (2bd)^2 + (2cd)^2
\end{equation}
where $a,b,c,d$ are elements of any commutative ring.
\end{theorem}
\prf
This is actually a special case of Euler's Four Squares Theorem,
which states that 
\[
(a^2 + b^2 + c^2 + d^2) (a'^2 + b'^2 + c'^2 + d'^2) = \hfill
\]
\[
(aa' + bb' + cc' + dd')^2 +
(ab' - ba' \pm cd' \mp dc')^2 +
(ac' \mp bd' - ca' \pm db')^2 +
(ad' \pm bc' \mp cb' - da')^2
\]
where $a,b,c,d,a',b',c',d'$ are elements of any commutative ring 
(such as the integers, the rationals, the reals, or 
polynomials over the integers, rationals, or reals).
Let $a'=-a, b'=b, c'=c, d'=d$.
\QED

Interpreting this theorem in light of (\ref{eq:quint}),
it reveals a rational map to \Sn{3} (which we shall rename
$M$ to distinguish from the generic map $f$ to \Sn{3}):

\[	g(x_1,x_2,x_3,x_4) = x_1^2 + x_2^2 + x_3^2 - x_4^2	\]
\[	h(x_1,x_2,x_3,x_4) = 2x_1x_4	\]
\[	i(x_1,x_2,x_3,x_4) = 2x_2x_4	\]
\[	j(x_1,x_2,x_3,x_4) = 2x_3x_4	\]
\[	k(x_1,x_2,x_3,x_4) = x_1^2 + x_2^2 + x_3^2 + x_4^2	\]
Consequently,
\begin{equation}
\label{eq:M}
	M(x_1,x_2,x_3,x_4) = \frac{1}{x_1^2 + x_2^2 + x_3^2 + x_4^2}
			     (x_1^2 + x_2^2 + x_3^2 - x_4^2,
			      2x_1x_4, 2x_2x_4, 2x_3x_4)	
\end{equation}

$M$ is a new map to \Sn{3}.
It is undefined at the origin, so $M:\Re^4 - \{0\} \rightarrow \Sn{3}$.
We will show in Section -- that this new map is
preferable to the inverse of the stereographic projection map.

Although we have presented the search for rational
maps to \Sn{3} as just one method of finding rational
curves on \Sn{3}, we can now establish that these two problems 
are actually equivalent.
We have seen that the search for rational maps to \Sn{3} 
can be reduced to the search for Pythagorean quintuples over
the ring of polynomials (\ref{eq:quint}).
The search for quaternion curves can also be reduced
to the search for Pythagorean quintuples over the ring
of polynomials.

\begin{defn2}
{\rm 
A {\em Pythagorean $n$-tuple} ($n \geq 3$) is an $n$-tuple 
$(x_1,\ldots,x_n)$ that satisfies\\
$x_1^2 + \ldots + x_{n-1}^2 = x_{n}^2$.
If $x_i \in F$, we may refer to a Pythagorean $n$-tuple over $F$.
}
\end{defn2}

\begin{lemma}
A rational curve lies on the 3-sphere if and only if its components
form a Pythagorean 5-tuple.
Let $x(t) = (x_1(t),\ldots,x_{n+2}(t)) \subset \Re^{n+1}$
be a rational curve.
$x(t) \subset \Sn{n}$ if and only if $(x_1(t),\ldots,x_{n+2}(t))$ is a Pythagorean
$(n+2)$-tuple over the ring of polynomials $\Re[t]$.
\end{lemma}
\prf
$x(t)$ lies on \Sn{n} if and only if
$(\frac{x_1(t)}{x_{n+2}(t)})^2 + \ldots 
	+ (\frac{x_{n+1}(t)}{x_{n+2}(t)})^2 = 1$
or
$x_1^2(t) + \ldots + x_{n+1}^2(t) = x_{n+2}^2(t)$.
\QED

Thus, the general method of this paper,
expressing a quaternion curve as the image $f(C)$ of a curve $C$
interpolating $f^{-1}(p_i)$, where $f:\Re^4 \rightarrow \Sn{3}$,
is basically the only way to build rational quaternion curves:
any other method can be reduced to our method.

% ----------------------------------------------------------------------------

\section{$M^{-1}$}

Before we go on to analyze the new map, let us develop its inverse $M^{-1}$,
which is needed in our method of Section~\ref{sec:intro}.

% ----------------------------------------------------------------------------

\section{$M$ and stereographic projection}

The new map $M$ is very similar to the inverse of the stereographic projection map,
but the differences are important, leading to significant improvements
in curve design (Section --).
Consider the inverse stereographic projection map $\varphi_{x_1=0}$ in 4-space
(with pole $(1,0,0,0)$):
\[
	\varphi_{x_1=0} (0,x_1,x_2,x_3) =
	\frac{(x_1^2 + x_2^2 + x_3^2 - 1, 2x_1,2x_2,2x_3)}
	     {x_1^2 + x_2^2 + x_3^2 + 1}
\]
In projective space after homogenization,
\[
	\varphi_{x_1=0} (0,x_1,x_2,x_3,w) =
	(x_1^2 + x_2^2 + x_3^2 - w^2,
	 2x_1w,2x_2 w,2x_3 w,
	x_1^2 + x_2^2 + x_3^2 + w^2)
\]
This is identical to $M(x_1,x_2,x_3,w)$.
However, notice that, in $M$, $w$ is functioning in Euclidean space,
whereas in $\varphi_{x_1=0}$ it acts in projective space.
$M$'s abduction into finite Euclidean space of the behaviour of stereographic 
projection in projective space is critical.

The relationship between the two maps can be made even more clear.
The map $\nu: (a,b,c,d) \mapsto (0,\frac{a}{d},\frac{b}{d},\frac{c}{d})$
is a simple affine map, the composition of a scaling, translation, 
and rotation: $(a,b,c,d) \mapsto (\frac{a}{d},\frac{b}{d},\frac{c}{d},1)
\mapsto (\frac{a}{d},\frac{b}{d},\frac{c}{d},0)
\mapsto (0,\frac{a}{d},\frac{b}{d},\frac{c}{d})$.
It is represented by the
following matrix:
\[
	\left( \begin{array}{ccccc} 
	0 & 0 & 0 & 1 & 0 \\
	1 & 0 & 0 & 0 & 0 \\
	0 & 1 & 0 & 0 & 0 \\
	0 & 0 & 1 & 0 & 0 \\
	0 & 0 & 0 & 0 & 1
	\end{array} \right) 
	\left( \begin{array}{ccccc} 
	1 & 0 & 0 & 0 & 0 \\
	0 & 1 & 0 & 0 & 0 \\
	0 & 0 & 1 & 0 & 0 \\
	0 & 0 & 0 & 1 & -1 \\
	0 & 0 & 0 & 0 & 1 
	\end{array} \right)
	\left( \begin{array}{ccccc} 
	\frac{1}{d} & 0 & 0 & 0 & 0 \\
	0 & \frac{1}{d} & 0 & 0 & 0 \\
	0 & 0 & \frac{1}{d} & 0 & 0 \\
	0 & 0 & 0 & \frac{1}{d} & 0 \\
	0 & 0 & 0 & 0 & 1
	\end{array} \right)
	\left( \begin{array}{c}
	a \\ b \\ c \\ d \\ 1
	\end{array} \right)	
	= \left( \begin{array}{c}
	0 \\ \frac{a}{d} \\ \frac{b}{d} \\ \frac{c}{d} \\ 1
	\end{array} \right)
\]
\[
	\left( \begin{array}{ccccc} 
	0 & 0 & 0 & \frac{1}{d} & -1 \\
	\frac{1}{d} & 0 & 0 & 0 & 0 \\
	0 & \frac{1}{d} & 0 & 0 & 0 \\
	0 & 0 & \frac{1}{d} & 0 & 0 \\
	0 & 0 & 0 & 0 & 1
	\end{array} \right)
	\left( \begin{array}{c}
	a \\ b \\ c \\ d \\ 1
	\end{array} \right)
	= \left( \begin{array}{c}
	0 \\ \frac{a}{d} \\ \frac{b}{d} \\ \frac{c}{d} \\ 1
	\end{array} \right)
\]
Thus, $M = \varphi_{x_1=0} \cdot \nu$.

\section{Differences with stereographic projection}

\begin{itemize}
\item
$M$ maps all of $\Re^4$ to the sphere while stereographic projection
maps only a hyperplane.
Thus, interpolation in image space is restricted to a plane in the latter,
yielding a smaller class of curves(?).
\item
Notice that we needed to consider the stereographic projection map
in projective space before it became similar to $M$.
The interaction of projective space and its plane at infinity with the
map $M$ is an important difference from standard stereographic projection.

The $x_4=0$ hyperplane in the $M^{-1}$ image space
is `equivalent' to the plane at infinity of stereographic projection.
That is, $x_4=0$ is mapped to the plane at infinity by $\nu$.
This can also be seen from the equation 
$M(a,b,c,d) = \varphi_{x_1=0} (0,\frac{a}{d},\frac{b}{d},\frac{c}{d})$.
The effect is that the behaviour of the hyperplane at infinity is 
mapped into finite space (brought
into play) in the $M^{-1}$ image space, since it becomes a finite hyperplane,
whereas this behaviour is not felt in the
stereographic projection image space (since data points will not be
mapped to the plane at infinity in stereographic projection image space,
since they avoid the pole).
The use of projective space by $M$ is important.
\end{itemize}

% ----------------------------------------------------------------------------

\section{Quality analysis}

compare curves generated by M and stereographic projection;
also compare nonrational curves generated by Shoemake and Kim [if we can implement in time]

Hypothesis: STEREOG PROJ METHOD IS LIKELY TO BE EQUIVALENT WITH OUR METHOD, 
SINCE AFFINE MAP $\nu$ TO PLANE DOES NOT DO MUCH.
Hypothesis is wrong!

\subsection{Covariant acceleration (Kim claims this is torque: p. 371 top)}

less is better 
	- 0 cov accel = geodesic
		- thus, more efficient change of orientation if less cov accel
	- Barr 92

Note that any rational spherical curve will never be a geodesic, since these must have
constant speed and we have observed that no rational curve can be arc-length parameterized.
Thus, it will always have some nontrivial tangential acceleration.

\subsection{Arc length}

more efficient orientation curve if shorter: less unnecessary change

\subsection{Angular velocity}

\cite[p. 659]{beer77}

preferable not to spin too fast (smaller is better)

maximum and net (sum of squares)

\subsection{Angular acceleration (Torque according to Kim)}

preferable not to change rotation axis too fast (smaller is better)

maximum and net (sum of squares)



Note: not appropriate to consider strain energy for this nonplanar curve.

% ----------------------------------------------------------------------------

\section{Sufficiency of stereographic projection}

Sufficient to generate all rational curves (Wang) by mapping.

\section{Insufficiency of stereographic projection?}

Only maps plane.
In effect, maps $R^n$ to $S^n$ rather than $R^{n+1}$.
Inefficiency of Wang's use of projection and moving pole.
BUT WHY DO WE NEED TO MAP POINTS TO $N+1$ SPACE?
$N$ SPACE MAY BE JUST AS GOOD.

\section{Related work}

The map to the 2-sphere used in Dietz, Hoschek and J\"{u}ttler
\cite{} can also be derived from Euler's Four Squares Theorem.
(They derive their map from a result of Lebesgue instead.)
This time Euler's theorem yields a recipe for generating Pythagorean 4-tuples.

% ----------------------------------------------------------------------------

\section{Completeness}

M is onto.

Related fact: (Wang) Any rational curve on \Sn{3} is the image of some
	rational curve under stereographic projection.
	That is, given any rational curve $c(t)$ on \Sn{3},
	there exists rational curve $d(t)$ such that $c = \varphi (d)$.

\section{Rational maps to the sphere}

We motivated the map $M$ through the generation of Pythagorean quintuples,
which are equivalent to rational curves on \Sn{3}.
$M$ is a rational quadratic map from $\Re^4$ to \Sn{3}.
It turns out that any rational map from $\Re^4$ to \Sn{3} can be used
to generate rational curves on \Sn{3}, by mapping rational curves in $\Re^4$
to the 3-sphere using $M$.
Let us consider this more general problem: 
the construction of rational maps from $\Re^4$ to \Sn{3}.
The lower the degree of these maps, the better.
It is easy to see that there is no rational linear map from $\Re^4$ to \Sn{3}.
$M$ is an example of a rational quadratic map.
Are there others?
We explore the theory of spherical maps \cite{ono94} for other maps.

\section{Spherical maps}

\begin{defn2}
{\rm
A map $f:\Re^m \rightarrow \Re^n$ is {\em quadratic} if 1) $f(kx) = k^2 f(x)$
whenever $k \in \Re, x \in \Re^m$
and 2) the map $(x,y) \mapsto \frac{1}{2} [f(x+y) - f(x) - f(y)], x,y\in\Re^m$
is bilinear.
	% restricting K to \Re, X to \Re^m, and Y to \Re^n in Ono, p. 165
}
\end{defn2}

\begin{defn2}
{\rm
A quadratic map $f:\Re^m \rightarrow \Re^n$ is {\em spherical} 
if $\| f(x) \| = \| x \|^2$ for all $x \in \Re^m$.
Notice that $f(\Sn{m-1}) \subset \Sn{n-1}$.
We shall be interested in the restriction of the map to 
$f:\Sn{m-1} \rightarrow \Sn{n-1}$.
	% restricting quadratic forms q_x and q_y to squared Euclidean norms
	% x_1^2 + ... + x_m^2 (or x_n^2)
	% which strictly translates into \| f(x) \|^2 = \| x \|^4,
	% which is simplified to above form
}
\end{defn2}

Hopf maps are special quadratic spherical maps (p. 172; Theorem 5.5, p. 183, and Theorem 5.7, p. 187).
(Hopf himself studied the map (\ref{eqn:classicalHopf}) below from $\Sn{3}$ to $\Sn{2}$.)

{\em How can a spherical map (and thus a Hopf map) 
be a map from $\Re^{m}$ to \Sn{n-1},
since $f(x) \in \Sn{n-1} \Rightarrow \| f(x) \| = 1 \Rightarrow \|x\|^2 = 1
\Rightarrow \|x\| = 1$.
That is, the only points that are mapped to the sphere are points already
on the sphere!
Does this jive with (\ref{eqn:HopfM} below, which seems to be almost equivalent
to our $M$ and thus a map from $\Re^4$ to \Sn{3}?
}


Spherical maps can be combined with stereographic projection to
construct rational maps from $\Re^4$ to \Sn{3}.
Since stereographic projection is a map $P:\Re^4 \rightarrow \Sn{4}$,
if we can find a rational spherical map $S:\Sn{4} \rightarrow \Sn{3}$,
the composition yields the desired map: $S \circ P: \Re^4 \rightarrow \Sn{3}$.
Unfortunately, there are no nontrivial quadratic spherical maps from $\Re^5$ to
$\Re^4$, and thus none from \Sn{4} to \Sn{3} \cite[p. 196]{ono94}.

Simplest nontrivial spherical map and Hopf map: $f:\Re^2 \rightarrow \Re^2$ 
and $f:S^1 \rightarrow S^1$ is \cite[pp. 169,173]{ono94}:
\[
	f(x_1,x_2) = \left( \begin{array}{c}
	x_1^2 - x_2^2 \\ 2x_1 x_2
		\end{array} \right)
\]

Classical Hopf map and spherical map: $f:\Re^4 \rightarrow \Re^3$
and $f:\Sn{3} \rightarrow \Sn{2}$ is \cite[pp. 171,174]{ono94}
\[
\label{eqn:classicalHopf}
	f(x_1,x_2,x_3,x_4) = \left( \begin{array}{c}
	x_1^2 + x_2^2 - x_3^2 - x_4^2 \\
	2(x_1 x_3 + x_2 x_4) \\ 2(x_1 x_4 + x_2 x_3)
		\end{array} \right)
\]
This is closely related to Dietz et. al.'s map.

A Hopf map (of the first kind) $f:\Re^n \rightarrow \Re^n$ is \cite[p. 193]{ono94}
\[
\label{eqn:HopfM}
	f(x,y_1,\ldots,y_{n-1}) = \left( \begin{array}{c}
		x^2 - y_1^2 - \cdots - y_{n-1}^2 \\
		2xy_1 \\ \vdots \\ 2xy_{n-1}
		\end{array} \right)
\]
In 4-space, this map is very similar to $M$!
(Not in projective space, and opposite sign to first component.)

% ----------------------------------------------------------------------------

\section{Future work}

Does every rational map to $S^3$ involve stereographic projection
in some strong way?
Is there a better rational map to $S^3$ that yields even better curves?

Conclusions:

Have shown that our general approach to the construction
of rational curves, finding rational maps to \Sn{3},
is the only approach.

Our discussion of rational maps to \Sn{3} also applies to
rational maps to \Sn{n}, $n > 3$.
Euler's Four Squares Theorem has an identical version in higher dimensions
(see Aida and Hopf maps),
as of course does stereographic projection.
Thus, rational curves on general \Sn{n} can be built using this paper.
Similar techniques to those used in Dietz \cite{} can be used to build
rational curves on other quadric surfaces as well.

% ----------------------------------------------------------------------------

\bibliographystyle{plain}
\begin{thebibliography}{Dickson 52}

\bibitem[Dickson 52]{dickson52}
Dickson, L.E. (1952) History of the Theory of Numbers: Volume II,
Diophantine Analysis.  Chelsea (New York), pp. 277, 318.

\bibitem[Ono 94]{ono94}
Ono, T. (1994) Variations on a Theme of Euler.
Plenum Press (New York).

\bibitem[Thorpe 79]{thorpe79}
Thorpe, J.A. (1979) Elementary Topics in Differential Geometry.
Springer-Verlag (New York), p. 124.

\end{thebibliography}

\end{document}
