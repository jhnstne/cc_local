\documentstyle[11pt]{article}
\newif\ifFull
\Fullfalse
\input{header}
\newcommand{\add}[1]{#1}		% {{\bf #1}}
\DoubleSpace

\setlength{\oddsidemargin}{0pt}
\setlength{\topmargin}{.5in}	% should be 0pt for 1in
% \setlength{\headsep}{.5in}
\setlength{\textheight}{8.5in}
\setlength{\textwidth}{6.5in}
\setlength{\columnsep}{5mm}	% width of gutter between columns
% \markright{Quaternion splines: \today \hfill}
% \pagestyle{myheadings}
% -----------------------------------------------------------------------------

\title{A rational quaternion spline of arbitrary continuity}
% A Simple but Effective Rational Quaternion Spline
% Rational Quaternion Splines as Euclidean Curves}
% \title{Euclidean curves on a surface}
\author{J.K. Johnstone\thanks{Geometric Modeling Lab, 125 Campbell,
	Computer and Information Sciences, 
	UAB, Birmingham, AL 35294.}\ \ and 
	J.P. Williams\thanks{Imaging and Visualization Group,
	Siemens Corporate Research, Princeton, NJ.}}
	% \\August 1999 (revision of July 1999)}

\begin{document}
\maketitle

\begin{abstract}
Quaternion splines are a classical tool for orientation control in computer
animation and robotics.
In this paper, we design a rational quaternion spline
with many desirable properties:
it is a fully general NURBS curve of arbitrary continuity;
\add{it has a closed form algebraic description, leading to simple 
derivative computation;}
its construction is an efficient generalization of classical interpolation
techniques in Euclidean space, leading to simple implementation and
easy incorporation into existing NURBS-based modelers;
it is coordinate-frame invariant;
and it is of high quality.
% demonstrably better than earlier curves;
% compare to trim, Wang and Joe, and Ravani
% and it can interpolate derivative information.

The vast majority of quaternion splines have been non-rational curves.
There are many advantages to the design of rational quaternion splines,
such as computational efficiency (both in designing the
quaternion spline and in later using the spline) and compatibility 
with existing NURBS technology.
The few other rational quaternion splines have been limited to
$C^1$ continuity and have had other difficulties.
% There has only been one attempt at a rational quaternion spline, by Wang
% and Joe. (also by Nielsen)
% Their construction, however, is not of a general rational curve 
% but a curve composed entirely of spherical biarcs (great circles of a sphere), 
% which imposes considerable restrictions on the resulting curve
% (for example, low continuity)
% and is again incompatible with existing NURBS technology.

We believe that our rational quaternion spline will be of considerable
use in interactive animation, interactive robot motion control, and 
motion analysis.
Its design is also suggestive of a general approach for the design
of rational curves on arbitrary surfaces.
\end{abstract}

\noindent Keywords: quaternion spline, rational curves, interpolation,
	  orientation, animation, motion control.
\clearpage

\section{Introduction}
\label{sec:intro}

\subsection{A rational quaternion spline}

Quaternion splines are a classical tool for orientation control in computer
animation and robotics.
The design of quaternion splines has received a great deal
of attention since Shoemake introduced them in 1985 \cite{shoemake85}.
However, quaternion splines have traditionally been non-rational curves 
\cite{shoemake85,duff85,pletinckx89,schlag91,barr92,nielson92,nielson93,nam95,park95,kim95,rama97}.
The only exceptions are a quaternion spline constructed
from spherical biarcs by Wang and Joe \cite{wang93}
and a quadratic quaternion spline of Nielson \cite{nielson92,nielson93},
both of which have restricted $C^1$ continuity.
In this paper, we design a rational quaternion spline that can have
arbitrary continuity.
It is also of superior quality compared to both rational and nonrational
quaternion splines.

A rational quaternion spline has many advantages.\footnote{A 
	curve is rational if its parameterization can be 
	expressed as a rational map.
	A map $(x_1,\ldots,x_n) \mapsto (f_1 (x_1,\ldots,x_n),\ldots,f_m (x_1,\ldots,x_n))$
	is rational if its components $f_i$ are all rational polynomials
	(quotients of polynomials).}
% due to its efficiency and compatibility.
Rational curves (usually represented as NURBS)
	% \footnote{Any rational curve can be expressed as a NURBS curve.}
are the most efficient of curves,
due to the inherent efficiency of polynomial computations and 
the elegant and powerful theory of rational Bezier and B-spline curves.
As a result of this elegance,
rational Bezier curves and NURBS are the {\em de facto} standard 
in modeling systems, with a large, established suite of algorithms for their 
manipulation.
Thus, the design of a rational quaternion spline will allow the 
curve to be incorporated immediately into existing software and 
geometric models, 
and subsequent computations with the curve will be simple and efficient.
Our algorithm for the quaternion spline's construction is also efficient, 
largely because of the rationality of the computations.
This will allow interactive design for interactive animation, 
interactive robot motion control, and efficient motion analysis.
Since the design of a motion matching a designer's intent 
is as much of an art as a science,
interactive design and refinement is a desirable feature.
\add{Last but not least, a rational quaternion spline has a closed form algebraic representation,
unlike most other quaternion splines.
A closed form makes curve analysis possible, such as derivative computation.
Derivative computation is important in many applications
of quaternion splines to motion control, such as in spacetime optimization
\cite{witkin88} in motion editing \cite{popovic99}.}

\subsection{Design on a surface}
\label{subsec:design}

The main challenge of quaternion spline design is that it involves the design of
a curve {\em on a surface}.
A unit quaternion is a point on the unit sphere \Sn{3}\ in 4-space,
$x_1^2 + x_2^2 + x_3^2 + x_4^2 - 1 = 0$.
	% The superscript 3 refers to the dimension of the manifold.
In particular, the unit quaternion 
$(\cos \frac{\theta}{2}, v \sin \frac{\theta}{2})$, $v \in \Re^3$, $\|v\|=1$,
represents the orientation of an object 
where the object in its canonical orientation 
has been rotated by $\theta$ radians about the axis $v$.
(See Section~\ref{sec:quaternion} for a full review of quaternions.)
If the orientations of a rigid object in a finite series of keyframes are
represented by quaternions, the design of a smooth curve
on \Sn{3}\ interpolating these quaternions represents a smooth
motion of the object through the desired orientations.
This interpolating curve on \Sn{3}\ is called a quaternion spline.
Thus, quaternion spline design is curve design on a surface.

The design of a curve on a surface is more challenging than the conventional
design of a curve in Euclidean space, since surfaces are Riemannian spaces
with a significantly different geometry than Euclidean space.
For example, in moving from Euclidean to Riemannian geometry, 
straight lines are replaced by geodesics.
Indeed, the constraint of the quaternion spline to \Sn{3}\ is the major 
reason for the nonrationality of the existing quaternion spline methods, 
since constraint to a sphere is most easily
done through a nonrational analogue of linear interpolation, slerping
\cite{shoemake85,duff85,pletinckx89,schlag91,nielson92,nielson93,kim95,nam95},
or by inherently nonrational constrained optimization \cite{barr92,rama97}.

Since the modeling of curves in Euclidean space
is much better understood than the modeling of curves in Riemannian space, 
another promising approach to the
design of a quaternion spline is to somehow reduce the problem to the 
design of a curve in Euclidean space, unconstrained to a surface.
% `Doctor, it hurts when I do this!'  `Well, then don't do it.'
% This reduction of an unknown problem to a known problem is a time-honored
% classical mathematical technique.
An approach for achieving this reduction is as follows.
Let $f:\Re^m \rightarrow \Sn{3}$ be a map to \Sn{3}.
Then any curve $C$ in $m$-space yields a curve $f(C)$ on \Sn{3}\ 
(Figure~\ref{fig:reduce}).
We can now concentrate on the Euclidean problem of designing the curve $C$.
The problem of constraining the curve to the surface has been
effectively reduced to the problem of constructing a map to the surface.

\begin{defn2}
The design of a curve on a surface through the design of a curve in $m$-space
and a map from $m$-space to the surface will be called the {\bf Euclidean-space
	% image-space, surface-map
approach} to curve design on the surface.
\end{defn2}
\begin{figure}
\vspace{2.5in}
\special{psfile=/usr/people/jj/modelTR/3-spline/img/fig1white.ps
	 hoffset=100}
\caption{A curve in Euclidean space maps to a curve on the surface}
% file: fig1.showcase
% tops fig1white.rgb -m 6.5 1.5 > fig1white.ps
% a simple picture of S, C, f(C), and f (as an arrow) for a generic surface S
\label{fig:reduce}
\end{figure}

Since we want the quaternion spline to be rational,
the curve $C$ and the map $f$ must be rational.
It is simple to construct a rational curve $C$,
since the classical methods for curve design in Euclidean space yield 
rational curves \cite{farin97}.
The construction of a good rational map to \Sn{3}\ is more challenging.

We are now ready to give our general algorithm for the design of a rational
quaternion spline through the quaternions
$\{p_i\}_{i=1,\ldots,k} \subset \Sn{3}$.
Refer to Figures~\ref{fig:alg1a}-\ref{fig:alg2c}.
%
\begin{description}
\item[(1a)] Design a rational map to \Sn{3}, $f:\Re^4 \rightarrow \Sn{3}$.
\item[(1b)] Compute the inverse map $f^{-1}:\Sn{3} \rightarrow \Re^4$.
\item[(2a)] Compute $\{f^{-1}(p_i)\}_{i=1,\ldots,k}$, 
	    mapping the quaternions to Euclidean space.
\item[(2b)] Design a rational curve $C$ in Euclidean space, interpolating $\{f^{-1}(p_i)\}_{i=1,\ldots,k}$.
\item[(2c)] Compute $f(C)$, mapping the curve back to the surface.
	    This is the desired rational quaternion spline interpolating
	    the quaternions $\{p_i\}_{i=1,\ldots,k}$.
\end{description}

We prefer rational maps to \Sn{3}\ with domain $\Re^4$, the resident
Euclidean space, since then $f^{-1}(p_i)$ is a curve in step 2a
(as opposed to a point if the domain is $\Re^3$ or a surface if $\Re^5$).
This leads to a useful flexibility in the interpolation step 2b,
since we can interpolate curves rather than points,
which can be used to design better curves on the surface.

Note that the inverse map $f^{-1}$ need not be rational.
Also note that (1a-b) are preprocessing steps,
(2b) is a well understood interpolation problem
and (2a) and (2c) are simple operations.
Thus, this algorithm is easily implemented and integrated into
existing modeling/animation systems.
	% Later in the paper, we shall consider the addition of derivative data
	% for the curve to interpolate.
	%% We postpone this discussion in order to avoid overcomplicating the issue
	%% at the outset.

\clearpage
\twocolumn

\begin{figure}
\vspace{2in}
\special{psfile=/usr/people/jj/modelTR/3-spline/img/step1a.ps hoffset=40}
\caption{Step 1a}
\label{fig:alg1a}
% file: ~jj/modelTR/3-spline/img/step1a.showcase
% tops step1a.rgb -m 6.5 1.5 > step1a.ps
% 1a: point in 3-space, sphere, arrow from point to sphere marked 'f'
\end{figure}

\begin{figure}
\vspace{2in}
\special{psfile=/usr/people/jj/modelTR/3-spline/img/step1b.ps hoffset=40}
\caption{Step 1b ($g=f^{-1}$)}
\label{fig:alg1b}
% file: ~jj/modelTR/3-spline/img/step1b.showcase
% tops step1b.rgb -m 6.5 1.5 > step1b.ps
% 1b: point on sphere, sphere, arrow from point to general 3-space marked g
\end{figure}

\begin{figure}
\vspace{2in}
\special{psfile=/usr/people/jj/modelTR/3-spline/img/stepdata.ps hoffset=40}
\caption{Quaternions}
\label{fig:algdata}
% the following all use `s3spline -m 90 < data5-1'
% Data Points on (quaternions on S3)
% tops stepdata.rgb -m 6.5 1.5 > stepdata.ps
\end{figure}

\begin{figure}
\vspace{2in}
\special{psfile=/usr/people/jj/modelTR/3-spline/img/step2a.ps hoffset=40}
\caption{Step 2a}
\label{fig:alg2a}
% 2a: Data Points and Inverse Lines on
% tops step2a.rgb -m 6.5 1.5 > step2a.ps
\end{figure}

\begin{figure}
\vspace{2in}
\special{psfile=/usr/people/jj/modelTR/3-spline/img/step2b.ps hoffset=40}
\caption{Step 2b}
\label{fig:alg2b}
% 2b: Data Points, Inverse Lines and Image Curve on
% tops step2b.rgb   -m 6.5 1.5 > step2b.ps
\end{figure}

\begin{figure}
\vspace{2in}
\special{psfile=/usr/people/jj/modelTR/3-spline/img/step2c.ps hoffset=40}
\caption{Step 2c}
\label{fig:alg2c}
% 2c: Data Points, Image Curve and S3 curve on
% tops step2c.rgb   -m 6.5 1.5 > step2c.ps
\end{figure}

\clearpage
\onecolumn

Since $f$ is a rational map,
the curve on \Sn{3}\ inherits the continuity of the curve $C$ in $\Re^4$.
Since it is simple to define cubic interpolating curves
in Euclidean space with $C^2$ continuity, or curves of higher degree with
even higher continuity, the curve on $S^3$ can easily have $C^2$ continuity
and indeed arbitrary continuity.
Other quaternion splines have considerable difficulty with continuity.

The rest of the paper will be structured as follows.
Section~\ref{sec:prevwork} reviews related work on quaternion splines
and Section~\ref{sec:quaternion} gives the basic theory of quaternions.
In Section~\ref{sec:map}, we build a good rational map to \Sn{3} and its inverse.
In Section~\ref{sec:eucdesign},
we discuss the design of an interpolating curve in Euclidean space,
especially the interpolation of curves as opposed to points. 
In Section~\ref{sec:curveimage}, we show how to map a curve in $\Re^4$
back to a curve on \Sn{3} while preserving the Bezier structure,
for the important case of the cubic Bezier curve.
In Section~\ref{sec:avoid}, we discuss a region of instability 
for the inverse map and show how to avoid this region.
	% In Section~\ref{sec:deriv}, we show how the quaternion spline
	% can be designed to interpolate derivative data as well.
Section~\ref{sec:eg} gives examples of quaternion splines designed using the
new method.	
In Section~\ref{sec:results}, we show how to analyze the quality of a 
quaternion spline, and use this measure to show the superiority
of our method over some other methods.
We end with some conclusions and ideas for future work in 
Section~\ref{sec:conclusions}.
\add{An appendix contains a proof of a technical result from 
Section~\ref{sec:eucdesign}.}

\section{Related work}
\label{sec:prevwork}

There is a rich literature on quaternion splines.
\cite{shoemake85} introduced them as a solution 
for keyframe animation in 1985.
He used spherical linear interpolation (slerping) between points on \Sn{3}.
Many others have also used slerping 
\cite{duff85,pletinckx89,schlag91,nielson92,nielson93,kim95,nam95}.
The spherical linear interpolation between points $P_0$ and $P_1$ on $S^3$ is:
\[ P(t) = \frac{\sin((1-t)\theta) P_0 + \sin(t \theta) P_1}{\sin \theta}
\]
where $\cos \theta = P_0 \cdot P_1$.
This represents an arc of the great circle between $P_0$ and $P_1$.
The de Casteljau algorithm can be used to build traditional Bezier curves.
A spherical analog to the de Casteljau algorithm based upon
spherical linear interpolation is used to build quaternion splines,
mimicking Bezier \cite{shoemake85,kim95}, B-spline 
\cite{duff85,nielson92,nielson93,kim95}, Hermite \cite{kim95,nam95},
cardinal spline \cite{pletinckx89}, and Catmull-Rom \cite{schlag91} curves.
Since slerping is a nonrational operation,
all of the methods based on slerping generate nonrational curves.
Most of these curves are defined only by a geometric construction, and have
no closed form algebraic definition.
Computation of derivatives of these curves is complicated,
as is the imposition of $C^2$ continuity.
Kim et. al. \cite{kim95} provide solutions to the latter two problems,
using Lie algebra and its exponential map.
Our use of conventional Bezier or B-spline curves completely removes 
these problems with continuity or derivative calculation.
\cite{park97} again uses Lie algebra to design the quaternion spline,
working on the SO(3) manifold rather than the \Sn{3}\ manifold.

\cite{barr92} uses constrained optimization to develop optimal
quaternion splines, optimizing the constraint that the curve lie on \Sn{3}.
They also introduce low covariant acceleration
as a desirable property of a quaternion spline,
and incorporate it into their constraints.
This ability to incorporate extra constraints into the optimization
is a nice feature of their algorithm.
Their quaternion spline is nonrational with no closed-form expression,
and their numerical optimization can be expensive.
Their approach is refined for added efficiency in \cite{rama97}.

\cite{wang93, wang94} and \cite{nielson93} are close in spirit 
to this paper, since they design rational quaternion splines.
However, these curves are limited in scope.
\cite{wang93,wang94} and \cite{nielson93} develop quadratic curves
with $G^1$ continuity, and \cite{wang94} develops sextic curves
with $C^1$ continuity.
At least $C^2$ continuity is desirable, especially for animation.
	% In \cite{wang93,wang94}, the input points must be augmented with 
	% tangents, since the problem is posed as a Hermite interpolation problem.
The quaternion splines of \cite{wang93} are built from biarcs 
(great circles of the sphere), and those of \cite{nielson93} from circular arcs.
Both Wang and Nielson's methods also involve heuristic, data-dependent choices
that can be difficult to make,
such as the choice of a spherical biarc from
a one-parameter family of valid spherical biarcs, or a center of projection.
Our method generalizes the work of Wang and Nielson, 
by creating rational curves of arbitrary even degree 
(all rational curves on $S^3$ have even degree \cite{wang94}) 
and arbitrary continuity,
based on traditional NURBS and without any data-dependent choices.

The Euclidean-space approach has been used before
for the design of curves on surfaces \cite{dietz93,wang94}.
	% \cite{dietz93} and \cite{wang94} design rational curves on quadrics
	% in 3-space.
	% They both use a variant of the Euclidean-space approach.
The classical solution to the design of trim curves on a surface 
is another example, where the trim curve is designed in the parameter space 
of the surface and then mapped back to the surface using the
parameterization. % \cite{foley96}
Trim curves are not a good solution to quaternion splines, however.
It turns out that a parameterization of \Sn{3}\ is not a good map
to the surface for quaternion spline design (see Section~\ref{sec:results}).
Its domain is also wrong ($\Re^3$ rather than $\Re^4$) as discussed
in Sections~\ref{sec:intro} and~\ref{sec:eucdesign}.

\section{Quaternions and orientation}
\label{sec:quaternion}

The unit quaternion is a preferred representation for the
orientation of a rigid object in computer animation.
The quaternion was invented by Hamilton in 1843
	% H. Coxeter, Non-Euclidean Geometry, Univ. of Toronto Press, 1961, p. 122
as a 4-dimensional generalization of complex numbers.
It was soon recognized that the quaternion can also be used
to represent an orientation.
To see this, we appeal to a fundamental result of Euler from 1752
\cite{goldstein50} (Figure~\ref{fig:eulerRotation}).	
	% goldstein, p. 118
	% also junkins, optimal spacecraft rotational maneuvers, p. 26
	% Cayley quickly realized the quaternion's use for rotation,
	% establishing a connection to the rotation matrix 
	% (see Coxeter, Non-Euclidean Geometry, p. 122).
\begin{theorem}[Euler]
A rigid body can be moved from an arbitrary initial orientation
to an arbitrary final orientation by a single rotation of the body
about a fixed axis.
\end{theorem}
%
\begin{figure}
\vspace{2.5in}
\special{psfile=/usr/people/jj/modelTR/3-spline/img/EulerRotThm.ps
	 hoffset=150}
\caption{Any change of orientation can be expressed as a rotation about a fixed axis}
% file: EulerRotThm.showcase
% tops EulerRotThm.rgb -m 6.5 1.5 > EulerRotThm.ps
\label{fig:eulerRotation}
\end{figure}
%
This shows that the orientation of a rigid body can be represented by 
a rotation axis $v$, $\|v\|=1$, and the rotation angle $\theta$ about 
this axis required to rotate the body into the given orientation from a
canonical orientation (say, the orientation in which the exact geometry
of the body was originally specified).
$v$ and $\theta$ are encoded in a unit quaternion by
$(\cos \frac{\theta}{2}, v \sin \frac{\theta}{2})$.
Notice that $v$ and $\theta$ can be easily extracted from this encoding.

The quaternion's representation of orientation has two great benefits.
First, since it is a unit vector, it can be interpreted geometrically
as a point on \Sn{3}.
Second, we can measure the amount of rotation using this geometric
interpretation, since the metric on \Sn{3}
is the same as the metric of the rotation group \cite{misner73}.
%
\begin{theorem}
\label{thm:metric}
The metrics on \Sn{3}\ and the rotation group SO(3) are equivalent.
	% That is, arc length on \Sn{3}\ is proportional (equivalent?) to
	% the (Frobenius norm?) of the associated rotation matrix.
\end{theorem}
%
In short, the speed of a body's rotation through 3-space
can be directly measured,
and directly controlled, through the length of the quaternion spline.
This is an important tool for motion control in animation or robotics.
	% For example, constant speed on quaternion spline yields
	% constant speed of rotation of the rigid body.

The rotation of an object by a quaternion need not be 
computed using its literal interpretation as rotation about an axis.
Quaternion algebra offers an easier solution, which we now review.
The quaternion $(a,b,c,d)$ is actually shorthand for $a + bi + cj + dk$,
where $i$, $j$ and $k$ satisfy the relationship 
	% generalizations of the imaginary unit $i$
\begin{equation}
\label{eq:ijk}
	i^2 = j^2 = k^2 = ijk = -1. 
\end{equation}
A quaternion $(a,b,c,d)$ is often expressed as a scalar component
and a vector component, $(s,v)$, where $s=a$ and $v=(b,c,d)$.
Using this notation and applying (\ref{eq:ijk}),
the formula for quaternion multiplication is
\begin{equation}
\label{eq:qmult}
	[s_1,v_1] * [s_2,v_2] =
	[s_1*s_2 - v_1 \cdot v_2,\ 
	 s_1 * v_2 + s_2 * v_1 + v_1 \times v_2]
\end{equation}
A point $p \in \Re^3$ is rotated by the unit quaternion $[s,v]$
to the point $[s,-v] * [0,p] * [s,v]$.
This is a quaternion, but it can be interpreted
as a point in 3-space since its scalar part is necessarily 0.
	% quaternions form a noncommutative division ring (ring whose nonzero elements
	% form a group under multiplication: Hernstein, p. 125

Mathematically, an orientation can be represented by either of two antipodal quaternions
(by flipping the rotation axis $v$).
That is, \Sn{3}\ is a double covering of SO(3).\add{\footnote{A natural
	explanation for this double covering is provided by Dirac's belt trick,
	well developed in \cite{hart94}.}}
However, when using quaternions for motion control, only one of the two
quaternion representations is appropriate in a given context:
the quaternion with the smaller angular gap on \Sn{3} to the previous quaternion.
This is true because a small angular gap between consecutive quaternions
creates less spinning than a large angular gap.
Consider the difference between a small rotation and the complementary
rotation that represents an almost complete pirouette.
If the designer's intent is to include the pirouette, then this should be
made explicit in the motion control by adding intermediate quaternions
that imply this spin.
A motion should not introduce spinning unless it is explicitly designed in.

The quaternion has several advantages over other representations
for orientation, such as the rotation matrix and Euler angles.
Unlike Euler angles, quaternions do not experience gimbal lock,
can be combined easily, and have an effectively unique representation
for each orientation.
Unlike rotation matrices, quaternions have a concise representation,
4 scalars rather than 9.\footnote{A quaternion is still slightly 
	larger than it needs to be since $v$, being a unit vector, 
	is fully determined by only 2 of its elements.
	That is, $v$ and $\theta$ could have been encoded in 
	a quaternion using only 3 scalars.
	However, the redundancy of 4 scalars is necessary for
	the quaternion's geometric interpretation on a unit sphere 
	and the metric equivalence of Theorem~\ref{thm:metric}.}
And unlike both Euler angles and rotation matrices, 
quaternions have a natural geometric interpretation through identification
with \Sn{3}, which is a crucial element in algorithmic development.

\section{A map to and from \Sn{3}}
\label{sec:map}

The first step in our algorithm for the design of a rational quaternion
spline is to design a good rational map from 4-space to the surface \Sn{3}.
Recall that we prefer a map with domain $\Re^4$, so that the data points will be
mapped to one-dimensional curves in Euclidean space and the 
curve design in Euclidean space will enjoy more flexibility.

Our approach is as follows.
We shall first develop a normal form for all rational maps of $\Re^4$ to \Sn{3}.
We will then choose a particular map.
Finally, we will compute the inverse of this map.

\subsection{Rational maps of $\Re^4$ to \Sn{3}}

Consider a rational map from $\Re^4$ to \Sn{3}:
\[
	(x_1,\ldots,x_4) \mapsto
	(\frac{f_1(x_1,\ldots,x_4)}{f_{5}(x_1,\ldots,x_4)}, \ldots,
	 \frac{f_4(x_1,\ldots,x_4)}{f_{5}(x_1,\ldots,x_4)})
\]
where $f_1,\ldots,f_{5}$ are polynomials.
Since the image lies on \Sn{3}, $f_1^2 + \cdots + f_4^2 = f_{5}^2$
and $(f_1,f_2,f_3,f_4,f_5)$ is a Pythagorean quintuple.\footnote{This term derives
	from the Pythagorean Theorem on right triangles,
	which involves Pythagorean triples.}
\begin{defn2}
$(a_1,\ldots,a_{5}) \in K^{n+1}$
is a {\bf Pythagorean quintuple over $K$} 
if $a_1^2 + \ldots + a_4^2 = a_5^2$.
\end{defn2}
%
Thus, the study of rational maps from $\Re^{4}$ to \Sn{3}\ 
is equivalent to the study of Pythagorean quintuples of polynomials.
Pythagorean quintuples involve the sum of four squares.
In the number theory literature, there is an extensive study of the
sum of four squares (driven by the search for a proof that
every positive integer is the sum of the squares of four integers \cite{dickson52}).
An important result was developed by Euler,\footnote{This result shows that the product of a sum of four squares and a sum of four
	squares is another sum of four squares.
	This reduces the problem of showing that every integer is the sum
	of four squares to the simpler problem
	of showing that every prime is the sum of four squares,
	since every integer can be expressed as the product of primes.}
which we can use to build Pythagorean quintuples, 
and then a characterization of maps to \Sn{3}.

\begin{lemma}[Euler's Four Squares Theorem \cite{herstein75}]
\label{lem:euler}
% p. 373 of Herstein
\[
\begin{array}{ll}
& (a_1^2 + a_2^2 + a_3^2 + a_4^2) 
(\hat{a}^2_1 + \hat{a}^2_2 + \hat{a}^2_3 + \hat{a}^2_4) = \\
& (a_1 \hat{a}_1 - a_2\hat{a}_2 - a_3\hat{a}_3 - a_4\hat{a}_4)^2 +
   (a_1\hat{a}_2 + a_2\hat{a}_1 + a_3\hat{a}_4 - a_4\hat{a}_3)^2 + \\
& (a_1\hat{a}_3 - a_2\hat{a}_4 + a_3\hat{a}_1 + a_4\hat{a}_2)^2 +
   (a_1\hat{a}_4 + a_2\hat{a}_3 - a_3\hat{a}_2 + a_4\hat{a}_1)^2
\end{array}
\]
where $a_1,a_2,a_3,a_4,\hat{a}_1,\hat{a}_2,\hat{a}_3,\hat{a}_4$ are elements of a
commutative ring.
% \footnote{The original statement was for integers, but it easily generalizes.}
% see p. 210 of Ebbinghaus
\end{lemma}

\begin{corollary}
\label{lem:suff4}
$(a_1^2 + a_2^2 + a_3^2 - a_4^2,\ 2a_1a_4,\ 2a_2a_4,\ 2a_3a_4,\ 
 a_1^2 + a_2^2 + a_3^2 + a_4^2)$
is a Pythagorean quintuple for any polynomials $a_1,a_2,a_3,a_4$.
\end{corollary}
\prf
Let $(\hat{a}_1,\hat{a}_2,\hat{a}_3,\hat{a}_4) = (a_1,-a_2,-a_3,a_4)$. 
\QED

\noindent The following lemma establishes a weaker version of the necessary condition
associated with Corollary~\ref{lem:suff4}.
This in turn leads to the desired necessary and sufficient condition for rational maps
of $\Re^4$ to \Sn{3}.

\begin{lemma}
\label{thm:necessary4}
A quintuple of polynomials is Pythagorean only if it can be expressed in the form
\begin{equation}
\label{eq:pyth}
	\alpha (a_1^2 + a_2^2 + a_3^2 - a_4^2,
		\ 2a_1a_4,\ 2a_2a_4,\ 2a_3a_4,
		\ a_1^2 + a_2^2 + a_3^2 + a_4^2)
\end{equation}
for some polynomials $a_1,a_2,a_3,a_4,\frac{1}{\alpha}$.
\end{lemma}
\prf
Let $(p_1,p_2,p_3,p_4,p_5)$ be a Pythagorean quintuple of polynomials.
If $p_1 = p_5$, let\\
$(a_1,a_2,a_3,a_4,\alpha) = (p_1,0,0,0,\frac{1}{p_1})$.
Thus, we may assume without loss of generality that $p_1 \neq p_5$.
Let $(a_1,a_2,a_3,a_4,\alpha) = (p_2,p_3,p_4,p_5-p_1,\frac{1}{2(p_5 - p_1)})$.
Then $a_1,a_2,a_3,a_4,\alpha$ generate the Pythagorean quintuple
$(p_1,\ldots,p_5)$ as in (\ref{eq:pyth}).
In particular,
\[
\alpha (a_1^2 + a_2^2 + a_3^2 - a_4^2)
= \frac{p_2^2 + p_3^2 + p_4^2 - (p_5 - p_1)^2}{2(p_5-p_1)}
\]
and applying $p_1^2 + p_2^2 + p_3^2 + p_4^2 = p_5^2$,
\[
\alpha (a_1^2 + a_2^2 + a_3^2 - a_4^2) = \frac{-2p_1^2 + 2p_1p_5}{2(p_5 - p_1)} = p_1.
\]
Also, 
\[
\alpha (2a_i a_4) = \frac{2p_{i+1}(p_5 - p_1)}{2(p_5 - p_1)} = p_{i+1}
\]
for $i=1,2,3$.
Finally, 
\[
\alpha(a_1^2 + a_2^2 + a_3^2 + a_4^2) 
= \frac{p_2^2 + p_3^2 + p_4^2 + (p_5 - p_1)^2}{2(p_5-p_1)} = p_5
\]
Thus, $(p_1,p_2,p_3,p_4,p_5) = \alpha (a_1^2 + a_2^2 + a_3^2 - a_4^2,
		\ 2a_1a_4,\ 2a_2a_4,\ 2a_3a_4,
		\ a_1^2 + a_2^2 + a_3^2 + a_4^2)$.
\QED

\begin{corollary}
\label{thm:map4}
A map is a rational map of $\Re^4$ to \Sn{3}\ if and only if
it is of the form:
\begin{equation}
\label{eq:re4s3}
\small{(x_1,x_2,x_3,x_4)} \mapsto 
\small{\frac{1}{a_1^2 + a_2^2 + a_3^2 + a_4^2},
	(a_1^2 + a_2^2 + a_3^2 - a_4^2, 2a_1a_4, 2a_2a_4, 2a_3a_4)}
\end{equation}
where $a_1,a_2,a_3,a_4$ are polynomials over $x_1,x_2,x_3,x_4$
(or some permutation of this form).\footnote{This result generalizes
	to rational maps from $\Re^n$ to \Sn{n-1}, $n \geq 2$,
	using an identical proof.
	That is, a map is a rational map of $\Re^n$ to \Sn{n-1}, $n \geq 2$, 
	if and only if it is of the form:
\[
	(x_1,\ldots,x_n) \mapsto
	(\frac{a_1^2 + \cdots + a_{n-1}^2 - a_n^2}{a_1^2 + \cdots + a_n^2},
	 \frac{2a_1a_n}{a_1^2 + \cdots + a_n^2},
	 \ldots,
	 \frac{2a_{n-1}a_n}{a_1^2 + \cdots + a_n^2})
\]
where $a_1,\ldots,a_n$ are polynomials over $x_1,\ldots,x_n$.}
\end{corollary}
\prf
Consider a rational map
of $\Re^4$ to \Sn{3}, $(x_1,x_2,x_3,x_4) \mapsto 
(\frac{f_1}{f_5},\frac{f_2}{f_5},\frac{f_3}{f_5},\frac{f_4}{f_5})$,
where $f_1,\ldots,f_5$ are polynomials over $x_1,\ldots,x_4$.
Then $(f_1,\ldots,f_5)$ is a Pythagorean quintuple
so it can be expressed in the normal form
$(f_1,\ldots,f_5) = \alpha (a_1^2 + a_2^2 + a_3^2 - a_4^2,
2a_1a_4,2a_2a_4,2a_3a_4,$ $a_1^2 + a_2^2 + a_3^2 + a_4^2)$
for some polynomials $a_1,a_2,a_3,a_4,\frac{1}{\alpha}$
by Lemma~\ref{thm:necessary4}.
Thus, the map is of the form (\ref{eq:re4s3}), since the leading $\alpha$
cancels when expressed in 
$(\frac{f_1}{f_5},\frac{f_2}{f_5},\frac{f_3}{f_5},\frac{f_4}{f_5})$.
Notice that it is legal to cancel the $\alpha$, since
$\alpha = \frac{1}{\beta}$ for some polynomial $\beta$,
which implies that $\alpha$ is never zero.

Now consider a map of the form (\ref{eq:re4s3}).
This is a rational map of $\Re^4$ to \Sn{3}\ by Corollary~\ref{lem:suff4}.
\QED

\noindent The most natural choice for the polynomials $a_i$ 
in Corollary~\ref{thm:map4} is $a_i = x_i$ and the identity permutation.
%
\begin{defn2}
The {\bf most natural} map to \Sn{3}\ is 
$M: \Re^4 - \{0\} \rightarrow \Sn{3}$ defined by:
\begin{equation}
\label{eqM}
\footnotesize{
	M(x_1,x_2,x_3,x_4) =
	\frac{1}{x_1^2 + x_2^2 + x_3^2 + x_4^2}
	(x_1^2 + x_2^2 + x_3^2 - x_4^2, 2x_1x_4, 2x_2x_4, 2x_3x_4)
	 }
\end{equation}
\end{defn2}
%
We fully analyze this map in \cite{jj98b}.
For example, we show that it is a powerful extension of
the inverse map of stereographic projection.
\add{We also further explain the title `most natural'.}
We shall use the map $M$ as our rational map to \Sn{3}.

\begin{rmk}
Ironically, Euler's Four Squares Theorem, our main tool
in the development of the map $M$ to the quaternion sphere \Sn{3}, 
can be viewed as an anticipation
of quaternions themselves (a century in advance of their invention by Hamilton!)
since it encodes the product formula for quaternions 
(compare Lemma~\ref{lem:euler}):
\[
\begin{array}{ll}
& (a_1 + a_2 i + a_3 j + a_4 k) 
(\hat{a}_1 + \hat{a}_2 i + \hat{a}_3 j + \hat{a}_4 k) = \\
& (a_1 \hat{a}_1 - a_2\hat{a}_2 - a_3\hat{a}_3 - a_4\hat{a}_4) +
   (a_1\hat{a}_2 + a_2\hat{a}_1 + a_3\hat{a}_4 - a_4\hat{a}_3) i + \\
&  (a_1\hat{a}_3 - a_2\hat{a}_4 + a_3\hat{a}_1 + a_4\hat{a}_2) j +
   (a_1\hat{a}_4 + a_2\hat{a}_3 - a_3\hat{a}_2 + a_4\hat{a}_1) k
\end{array}
\]
\end{rmk}

\subsection{The inverse map}

To map the quaternions to Euclidean space,
the inverse of our map to \Sn{3}\ is needed (step 1b of the algorithm
in Section~\ref{subsec:design}).
Since the manifold $\Re^4$ is one dimension larger than \Sn{3}, 
one would expect the preimage of a point of \Sn{3}\ to be 
a curve in $\Re^4$.
This is indeed the case.
In fact, the inverse of a point is simply a line.
The proof of the following theorem is simplified by working in projective space.
%
\begin{defn2}
\label{defn:projspace}
% We shall make use of projective space in some of the proofs in this paper.
Real {\bf projective $n$-space} $P^n$ is the space 
$\{ (x_1,x_2,\ldots,x_{n+1}) : x_i \in \Re, \mbox{not all zero} \}$
under the equivalence relation 
\begin{equation}
\label{eq:projequivalence}
(x_1,\ldots,x_{n+1}) = k(x_1,\ldots,x_{n+1}),\ \ k \neq 0 \in \Re.
\end{equation}
The point $(x_1,\ldots,x_{n+1})$ in projective $n$-space, $x_{n+1} \neq 0$,
is equivalent to the point $(\frac{x_1}{x_{n+1}},\ldots,\frac{x_n}{x_{n+1}})$
in $n$-space.
The point $(x_1,\ldots,x_n,0)$ in projective $n$-space represents the point
at infinity in the direction $(x_1,\ldots,x_n)$.
To translate from $n$-space to projective $n$-space, the point 
$(x_1,\ldots,x_n)$ is typically transformed into the point $(x_1,\ldots,x_n,1)$.
See \cite{harris92} for more details on projective space.
\end{defn2}

\begin{theorem}
\label{thm:inverse}
$M^{-1}: S^3 \rightarrow \Re^4$ is defined as follows:
\[ 
\footnotesize{
	M^{-1}(x_1,x_2,x_3,x_4) =
	\left\{ 
	\begin{tabular}{ll}
		$t(x_2,x_3,x_4,1-x_1),\ t \in \Re,\ t \neq 0 \hspace{.2in}$
		  & $\mbox{if } (x_1,x_2,x_3,x_4) \neq (1,0,0,0)$\\
		$H$ & $\mbox{if } (x_1,x_2,x_3,x_4) = (1,0,0,0)$
	\end{tabular}
	\right.
}
\]
where $H$ is the hyperplane $x_4 = 0$ minus the origin.
That is, the preimage of $(x_1,x_2,x_3,x_4) \neq (1,0,0,0)$ on \Sn{3}
is a line through the origin, minus the origin.
\end{theorem}
\prf 
We work in projective space, where the map $M$ becomes
\begin{equation}
\label{eq:proj}
	 (x_1,x_2,x_3,x_4,x_5) \rightarrow
	 (x_1^2 + x_2^2 + x_3^2 - x_4^2,\ 
	 2x_1 x_4,\ 2x_2 x_4,\ 2x_3 x_4,\ 
	 x_1^2 + x_2^2 + x_3^2 + x_4^2).
\end{equation}
Let $p = (p_1,p_2,p_3,p_4,1) \in S^3 \subset P^4$.
We want to determine the conditions on $q = (q_1,q_2,q_3,q_4,q_5)$
so that $M(q_1,q_2,q_3,q_4,q_5) = p$.
Suppose that $M(q_1,q_2,q_3,q_4,q_5) = p$.
Using (\ref{eq:projequivalence}), we have
\begin{equation}
\label{eq1}
	q_1^2 + q_2^2 + q_3^2 - q_4^2 = kp_1
\end{equation}
\begin{equation}
\label{eq2}
	2q_1q_4	= kp_2
\end{equation}
\begin{equation}
\label{eq3}
	2q_2q_4	= kp_3
\end{equation}
\begin{equation}
\label{eq4}
	2q_3q_4	= kp_4
\end{equation}
\begin{equation}
\label{eq5}
	q_1^2 + q_2^2 + q_3^2 + q_4^2 = k
\end{equation}
for some $k \neq 0$.
$q_5$ is arbitrary, since it does not appear in these equations.
Subtracting (\ref{eq1}) from (\ref{eq5}), 
we have $2q_4^2 = k(1 - p_1)$
or 
\begin{equation}
\label{eq:q4}
q_4 = \pm \sqrt{\frac{k(1-p_1)}{2}}
\end{equation}
%
{\bf Case 1:\ }
Suppose $p_1 = 1$. Then $p = (1,0,0,0,1)$ since $p \in \Sn{3}$.
$q_4 = 0$ by (\ref{eq:q4})
and $M(q) = (q_1^2 + q_2^2 + q_3^2, 0, 0, 0, q_1^2 + q_2^2 + q_3^2)
       = (1,0,0,0,1)$ for any values of $q_1,q_2,q_3$, not all zero.
That is, $M(q) = p$ if and only if $q \in H$ where $H$ is 
the hyperplane $x_4 = 0$ minus the origin.
Equivalently, $M^{-1}(p) = H$.\\
%
{\bf Case 2:\ }
Suppose $p_1 \neq 1$.
Then $p_1 < 1$ since $p \in \Sn{3}$. 
From (\ref{eq:q4}), $q_4 \neq 0$ and $q_4$ is a real number.
From (\ref{eq2}-\ref{eq4}), $q_i = \frac{kp_{i+1}}{2q_4}$ for $i=1,2,3$:
\[
	q = (\frac{kp_2}{2q_4}, \frac{kp_3}{2q_4}, \frac{kp_4}{2q_4}, q_4, q_5)
\]
Using (\ref{eq:projequivalence}), 
\[
	q = \frac{2q_4}{k} q
	  = (p_2, p_3, p_4, \frac{2q_4^2}{k},\frac{2q_4q_5}{k})
\]
and then (\ref{eq:q4}),
\[
	q = (p_2,p_3,p_4,1-p_1, \pm \sqrt{\frac{2(1-p_1)}{k}} q_5)
\]
Since $q_5$ and $k$ are arbitrary,
\[
	q = (p_2,p_3,p_4,1-p_1, k') \hspace{1in} k' \in \Re
\]
We have shown that $M(q) = p$ only if 
$q = (p_2,p_3,p_4,1 - p_1,k')$.

On the other hand, if $q = (p_2,p_3,p_4,1 - p_1,k')$, $k' \in \Re$
and $p_1 \neq 1$,
then 
\[
\scriptsize{M(q) = (p_2^2 + p_3^2 + p_4^2 - (1 - p_1)^2,\ 
	2p_2 (1 - p_1),\ 2p_3 (1 - p_1),\ 2p_4 (1 - p_1),\ 
	p_2^2 + p_3^2 + p_4^2 + (1 - p_1)^2).}
\]
Using $p_1^2 + p_2^2 + p_3^2 + p_4^2 = 1$ ($p \in \Sn{3}$),
\[
M(q) = (2p_1(1 - p_1),\ 2p_2 (1 - p_1),\ 2p_3 (1 - p_1),\ 2p_4 (1 - p_1),\ 
2(1 - p_1))
\]
or $M(q) = (p_1,p_2,p_3,p_4,1)$ using (\ref{eq:projequivalence}).
Thus, $M(q) = p$ if and only if 
$q = (p_2,p_3,p_4,1 - p_1,k')$, $k' \in \Re$.
Translating back from projective space,
$M(q) = p$ if and only if 
$q = t(p_2,p_3,p_4, 1-p_1)$, $t \neq 0 \in \Re$.
Equivalently, $M^{-1}(p) = t(p_2,p_3,p_4, 1-p_1)$, $t \neq 0 \in \Re$.
It is understandable that the preimage does not contain the origin, 
since $M$ is undefined there.
\QED
% $q \in M^{-1}(p)$ if and only if $M(q) = p$.

\begin{defn2}
\label{defn:pole}
The special point $(1,0,0,0)$ is called the {\bf pole} of the map $M^{-1}$.
$(x_2,x_3,x_4,1-x_1)$ is called the {\bf defining point} of the preimage 
$M^{-1}(x_1,x_2,x_3,x_4)$.
\end{defn2}

\noindent Notice the beautiful simplicity of the defining point.

\ifFull
\begin{rmk}
Although the proof technique of Theorem~\ref{thm:inverse}
works for any individual map,
we cannot apply it to the abstract general map (\ref{eq:re4s3})
to reveal the general inverse map.
The above proof requires working in projective space, to take 
advantage of the extra equation it offers;
this would require getting inside the arbitrary polynomials
$a_i$ of (\ref{eq:re4s3}) to translate them into projective space,
an unfeasible task
since these polynomials are of unknown and varying degree.
\end{rmk}
\fi

% Example of another map: using $a_1=x_1 - 1$, $a_2 = x_2^2$, $a_3=0$, 
% and $a_4=x_4$,
% let $M(q_1,q_2,q_3,q_4,q_5) = (q_1^2 - 2q_1 + 1 + q_2^4 - q_4^2,
% 				2q_1q_4 - 2q_4, 2q_2^2 q_4, 0, 
% 				q_1^2 - 2q_1 + 1 + q_2^4 + q_4^2)$.

\section{Designing the curve in Euclidean space}
\label{sec:eucdesign}

Once the quaternions have been mapped into Euclidean space,
the next step is to design the curve in Euclidean space.
This involves interpolating a curve through the 
inverse image lines $M^{-1}(p_i)$.
We will reduce this problem to point interpolation, 
by choosing one point per line
and applying conventional interpolation through a set of points.
Each point should be chosen wisely, taking full advantage of our
flexibility to choose any point on the line.
A more precise statement of step (2b) of our algorithm
for curve design on a surface is now:
%
\begin{description}
\item[(i)] 
	Intelligently choose $k$ points
	$\{q_i\}$ on the $k$ lines $\{M^{-1}(p_i)\}$.
\item[(ii)]
	Design a rational curve $C$ in Euclidean space
	interpolating $\{q_i\}_{i=1,\ldots,k}$, using classical techniques.
\end{description}

To understand our choice of points $\{q_i\}$ on the lines
$\{M^{-1}(p_i)\}$, we must motivate the desirability of designing
a short curve in Euclidean space.
Other factors being equal, we prefer shorter quaternion splines.
We want to minimize any unnecessary spinning of the object:
loop-the-loops or gyration should not be introduced 
unless explicitly designed in the motion.
Since the metric of \Sn{3}\ is the same as the metric of the rotation
matrix group SO(3) (Theorem~\ref{thm:metric}), 
the removal of unnecessary object rotation is equivalent to the
removal of unnecessary length in the quaternion spline.
Thus, we prefer shorter quaternion splines.
Although curve length in Euclidean space is not
equivalent to curve length of the associated quaternion 
spline,\footnote{For example in the extreme case, motion along
	an inverse line $M^{-1}(p_i)$ in $\Re^4$ causes no associated
	motion on \Sn{3}.  However, in general, the arc length
	of the two curves, one in Euclidean space and the other on \Sn{3},
	are strongly related.
	Moreover, the avoidance of motion in Euclidean space always leads 
	to the avoidance of motion on \Sn{3}.}
shorter curves in Euclidean space are associated with shorter
quaternion splines.

By the above analysis, we prefer the curve in Euclidean space to move
efficiently between the lines $\{M^{-1}(p_i)\}$.
A good choice of points $\{q_i\}$ is the set of closest points:

\clearpage
	
\begin{itemize}
\item On the first inverse line, let $q_1$ be the intersection
of the line $M^{-1}(p_1)$ and \Sn{3}.
\item On subsequent lines, let $q_i$ be the closest point to $q_{i-1}$.
\end{itemize}

\add{An undesirable property of this set of closest points is that
it spirals in towards the origin,
continually attracted inwards by the closest criterion.
Although we have not encountered any problems arising from this property
even for datasets with hundreds of quaternions,
some may want an alternative choice for $\{q_i\}$.
An excellent alternative
is to consistently choose $q_i$ to be the point of 
$M^{-1}(p_i)$ on \Sn{3}.	% choose the closest intersection to the
				% previous point on S3
This is a very good approximation to the closest point, 
especially if the data quaternions $\{p_i\}$ are densely sampled.
The appendix develops a formula for the distance of the point on \Sn{3}\ from
the closest point as a function of the angular distance between the associated
quaternions, which formally establishes the quality of this approximation.

The on-\Sn{3} choice of $\{q_i\}$ has the advantage of generating curves
in Euclidean space that are guaranteed to avoid the origin, 
which is important since
the map $M$ back to Riemannian space is undefined at the origin (see~(\ref{eqM})).
Clearly the points $\{q_i\}$ avoid the origin.
Since all points $\{q_i\}$ lie on \Sn{3}\ and consecutive
$p_i$, and thus consecutive $q_i$, do not lie on opposite sides of \Sn{3}
(see Section~\ref{sec:quaternion}),
the interpolating curve of the $\{q_i\}$ also avoids the origin.
}

\section{Mapping back to \Sn{3}}
% \subsection{Mapping the curve back to the surface}
\label{sec:curveimage}

After the curve is designed in $\Re^4$, it must be mapped back to a curve
on \Sn{3}\ using $M$.
The following theorem shows how this mapping is done, segment by segment,
for the important case of a cubic Bezier curve in $\Re^4$.
We concentrate on the cubic Bezier curve since classical design of an 
interpolating curve in $\Re^4$ will generate
a cubic polynomial curve \cite{farin97}.	% Chapter 9
\add{Notice that cubic B-spline curves are also handled by this formula, since a cubic
B-spline curve can be easily translated to a cubic Bezier curve.}
The image of other curves under $M$ (Bezier and B-spline curves of higher
degree, Catmull-Rom, Hermite, etc.) can be computed similarly.

\begin{theorem}
\label{sextic}
Let $c(t)$ be a cubic Bezier curve in 4-space with
control points $b_i = (b_{i1},b_{i2},b_{i3},b_{i4})$, $i=0,\ldots,3$.
The image of $c(t)$ under $M$ is a rational sextic Bezier curve with 
weights 
\begin{equation}
\label{eq:weights}
w_k = \sum_{\begin{array}{c} \mbox{\footnotesize{$0 \leq i \leq 3$}} \\ 
			     \mbox{\footnotesize{$0 \leq j \leq 3$}} \\ 
			     \mbox{\footnotesize{$i+j=k$}}
			     \end{array}}
        \frac{\choice{3}{i} \choice{3}{j}}{\choice{6}{k}}
	\ (b_{i1} b_{j1} + b_{i2} b_{j2} + b_{i3} b_{j3} + b_{i4} b_{j4})
\end{equation}
and control points 
\begin{equation}
\label{eq:control-pts}
c_k = \frac{1}{w_k} 
      \sum_{\begin{array}{c} \mbox{\footnotesize{$0 \leq i \leq 3$}} \\ 
			     \mbox{\footnotesize{$0 \leq j \leq 3$}} \\ 
			     \mbox{\footnotesize{$i+j=k$}}
			     \end{array}} 
        \frac{\choice{3}{i} \choice{3}{j}}{\choice{6}{k}}
	\left( \begin{array}{c}
            b_{i1} b_{j1} + b_{i2} b_{j2} + b_{i3} b_{j3} - b_{i4} b_{j4} \\
            2b_{i1} b_{j4} \\
            2b_{i2} b_{j4} \\
            2b_{i3} b_{j4} 
	\end{array} \right)
\end{equation}
for $k = 0, \ldots, 6$.
\end{theorem}
\prf
Let $B_i^n(t)$ be the $i^{th}$ Bernstein polynomial of degree $n$.
Then $c(t) = \sum_{i=0}^3 B_i^3(t) b_{i}$.
This proof is a simple application of the product rule of Bernstein polynomials
\cite{farin97}:
\[
	B_i^m(t) B_j^n(t) = \frac{\choice{m}{i} \choice{n}{j}}{\choice{m+n}{i+j}}
		B_{i+j}^{m+n}(t).
\]
We will again work in projective space, where we recall that
the map $M$ becomes
\[
	 (x_1,x_2,x_3,x_4,x_5) \rightarrow
	 (x_1^2 + x_2^2 + x_3^2 - x_4^2,\ 
	 2x_1 x_4,\ 2x_2 x_4,\ 2x_3 x_4,\ 
	 x_1^2 + x_2^2 + x_3^2 + x_4^2).
\]
Let $M(c(t)) = (m_1(t),m_2(t),m_3(t),m_4(t),m_5(t))$.
$m_5(t)$ can be simplified using the product rule 
of Bernstein polynomials, as follows:
\[ m_5(t) =  [\sum_{i=0}^3 B_i^3(t) b_{i1}]^2 + 
	\ldots + [\sum_{i=0}^3 B_i^3(t) b_{i4}]^2
     =   \sum_{i=0}^3 \sum_{j=0}^3 
	\frac{\choice{3}{i} \choice{3}{j}}{\choice{6}{i+j}}
       B^6_{i+j}(t) (b_{i1} b_{j1} + \ldots + b_{i4} b_{j4})
\]
Letting $k=i+j$, 
\[ m_5(t) = \sum_{k=0}^6 B_k^6(t) 
	\sum_{\begin{array}{c}  \mbox{\footnotesize{$0 \leq i \leq 3$}} \\ 
			     \mbox{\footnotesize{$0 \leq j \leq 3$}} \\ 
			     \mbox{\footnotesize{$i+j=k$}}
			     \end{array}} 
	\frac{\scriptchoice{3}{i} \scriptchoice{3}{j}}{\scriptchoice{6}{k}}
	(b_{i1} b_{j1} + \ldots + b_{i4} b_{j4}) \]
Computing the other coordinates analogously yields
\[ M(c(t)) = 
   \sum_{k=0}^6 B_k^6(t)
	\sum \frac{\choice{3}{i} \choice{3}{j}}{\choice{6}{k}}
	\left( \begin{array}{c}
            b_{i1} b_{j1} + b_{i2} b_{j2} + b_{i3} b_{j3} - b_{i4} b_{j4} \\
            2b_{i1} b_{j4} \\
            2b_{i2} b_{j4} \\
            2b_{i3} b_{j4} \\
            b_{i1} b_{j1} + b_{i2} b_{j2} + b_{i3} b_{j3} + b_{i4} b_{j4}
	\end{array} \right) \]
% \[         = \sum_{k=0}^6 B_k^6(t) 
%	\sum_{\begin{array}{c} 0 \leq i \leq 3 \\ 
%			     0 \leq j \leq 3 \\ 
%			     i+j=k
%			     \end{array}} 
%       \frac{\choice{3}{i} * \choice{3}{j}}{\choice{6}{k}}
%	\left( \begin{array}{c}
%           b_{i1} b_{j1} + b_{i2} b_{j2} + b_{i3} b_{j3} + b_{i4} b_{j4} \\
%            w_k (\frac{b_{i1} b_{j1} + b_{i2} b_{j2} + b_{i3} b_{j3} - b_{i4} b_{j4}}{w_k}) \\
%            w_k (\frac{2b_{i1} b_{j4}}{w_k}) \\
%            w_k (\frac{2b_{i2} b_{j4}}{w_k}) \\
%            w_k (\frac{2b_{i3} b_{j4}}{w_k})
%	\end{array} \right) \]
%
which is a sextic rational Bezier curve with weights (\ref{eq:weights}) and 
control points (\ref{eq:control-pts}).
\QED

Notice how the map $M$ reveals itself in the formula for the control points
and weights, its numerator in the control points and its denominator
in the weights.
The remaining components of the formulae (\ref{eq:weights}-\ref{eq:control-pts})
are a reflection of the product rule for Bernstein polynomials.

\section{Avoiding the pole}
\label{sec:avoid}

The design of our curve is essentially complete.
However, we must refine one of the steps in the interests of stability.
The mapping of the data into Euclidean space is unstable
if any of the data is close to the pole $(1,0,0,0)$.
In this section, we shall show how the quaternions $\{p_i\}_{i=1,\ldots,k}$
can be moved away from the pole for robust curve design.

\subsection{Ill-conditioning near the pole}
\label{sec:ill}

$M^{-1}$ is not well-behaved near 
its pole $(1,0,0,0)$.
Recall that the image of a point $p \neq (1,0,0,0)$ 
under $M^{-1}$ is a line through the
origin and the defining point of $M^{-1}(p)$ (Definition~\ref{defn:pole}).
As a point $p$ on \Sn{3}\ approaches the pole $(1,0,0,0)$,
the defining point of $M^{-1}(p)$ approaches the origin:
\[ 
\mbox{lim}_{(x_1,x_2,x_3,x_4) \rightarrow (1,0,0,0)} (x_2,x_3,x_4,1-x_1)
= (0,0,0,0) 
\]
Consequently, the defining point becomes an ill-conditioned specification
of $M^{-1}(p)$:
small motions of $p$ can cause large motions of $M^{-1}(p)$.
This is analogous to the solution of a linear
system when the condition number of the matrix becomes large,
and the linear system becomes very sensitive to perturbation of the matrix.

\begin{example}
If $p$ on \Sn{3}\ is at distance $d$ from the pole,
the defining point of $M^{-1}(p)$ is also at distance $d$ from the
origin.
Then a motion of $d+\epsilon$ of $p$ (corresponding to a motion
of the defining point directly towards the origin) can yield
a 90-degree change of orientation of the line $M^{-1}(p)$.
\end{example}

In other words, near the pole on \Sn{3}, there is little correspondence 
between position of $p$ and position of $M^{-1}(p)$.
This leads to quaternion splines that jump wildly about the pole 
or create cusps (Figure~\ref{fig:wild}).
We conclude that we want to move data points away from the pole.
The following definition gives an empirical notion of how far the points
must be moved away.

\begin{figure}
\vspace{3in}
\special{psfile=/usr/people/jj/modelTR/3-spline/img/unstable.ps hoffset=80}
\caption{Unstable behaviour near the pole (cusp is directly at pole) and its correction}
\label{fig:wild}
% s3spline -m 90 -p < data3-cusp &
% s3spline -m 90    < data3-cusp &
% tops unstable.rgb -m 8.5 2 > unstable.ps
\end{figure}

\begin{defn2}
\label{defn:far}
A point $p \in \Sn{3}$ is {\bf too close to the pole}
if the angle formed by the vectors $p$ and $(1,0,0,0)$ is smaller
than 30 degrees. % pi/6 radians
Experimental evidence indicates that points closer to the pole 
can lead to undesirable behaviour in the curve.
% 30 = PI/6; PI/7 is definitely too small (lowerbound), as witnessed by data5-1
% using TOOCLOSEDIST = PI/7;
% we have not found an example that is not treated well by PI/6.
\end{defn2}

% \begin{example}
% data5-1 using $\pi/7$ vs. $\pi/6$.
% \end{example}

\Comment{
Another way of seeing this ill-conditioning is by observing the behaviour
of $M$ near the origin.
$M$ is undefined at the origin, due to the division by zero.
However, $M$ is also badly behaved {\bf near} the origin.
For example, consider approaching near the origin along the line $(0,0,t,.01)$.
In moving the short distance from $(0,0,.01,.01)$ to $(0,0,0,.01)$, 
the image under $M$ moves all the way from $(0,0,0,1)$ to $(-1,0,0,0)$.
That is, a move of distance $.01$ in $\Re^4$ maps to a move of angular distance
$\frac{\pi}{2}$ on \Sn{3}.
The portion of a curve in $\Re^4$ that is near the origin
will be stretched as it is mapped under $M$.
This is not desirable, as shown in Figure~\ref{fig:tooCloseToPole}.

This is not quite as strong or direct an argument for ill-conditioning as the previous
argument, since there is nothing forcing the user to use preimage points 
near the origin.
However, notice that if you do not stay close to the origin, 
then the large changes in the line with small changes on \Sn{3} mean that 
small distances on \Sn{3} will be filled by long curves (and thus not optimal
geodesics) since the preimage curve will be necessarily long.
}

\subsection{Moving data away from the pole}

We shall use a single rotation about the origin 
to move the data points away from the pole $(1,0,0,0)$.
Once the quaternion spline has been robustly designed, 
it will be rotated back by the same amount, 
yielding the desired curve.\footnote{Rotation of a rational 
	Bezier curve can be achieved simply by
	rotation of its control polygon.}
	% another advantage of rational curves.
Let us call the locus of points on \Sn{3}\ such that
no data point is within 30 degrees the {\bf empty region} of \Sn{3}.
Our problem reduces to finding a point $p$ in an empty region of \Sn{3}:
% on \Sn{3}\ such that no data point is within 30 degrees of $p$:
then a rotation of $p$ to the pole will move all data points a safe
distance from the pole.
We first show that there is an empty region to find,
and then how to find a point in an empty region.

\subsection{Existence of an empty region}

First, is there an empty region?
It is theoretically possible that the data points are so densely packed
on the sphere that there is no empty region.
However, the requisite number of data points is huge.
	% large in theory, and
The 'surface area' of \Sn{3} is $2\pi^2$ \cite{kendall61}
while a region 30 degrees wide about a point has area $\pi/6$.
	% A point covers a region 30 degrees wide about it, which has
	% a surface area of $\pi/6$ on the sphere \Sn{3}\ of area $2\pi^2$
	% \cite{kendall61}.
Thus, theoretically only about $12\pi \approx 38$ points are needed to
cover the sphere, leaving no empty region.\footnote{Actually, quite
	a few more points are needed, since the discs about the 38 points
	do not abut perfectly.}
However, this requires a gap of at least 60 degrees between 
consecutive quaternions, a very large change of orientation.
The typical gap between consecutive quaternions for reasonable
motion control is about 20 degrees.
	% independent of the method used to construct the quaternion spline.
Using a gap of 20 degrees, 434,783 points\footnote{The surface area 
	of \Sn{2}\ is $4\pi$, while the surface area
	of a region on \Sn{2}\ 10 degrees wide centered about a point is 
	$\int_{0}^{2\pi} \int_0^{\frac{\pi}{18}}  \sin \phi \ d\phi \ d\theta
	= .0000289$ \cite{lang79}.}
	% Lang, Calculus of several variables, p. 228
	% p. 208 for parameterization, p. 224-5 for ||dx/dphi x dx/dtheta||
	% integral of sin is -cos
are needed to cover \Sn{2}, and even more to cover \Sn{3}.
	% In the examples of Section~\ref{sec:results}, 
	% the distance between quaternions is about --- \ref{}.
Moreover, in order to leave no empty region, the quaternions
must be spread across the entire sphere.
It is far more common for the data to be restricted to a small region
of the sphere.
Thus, there will be an empty region on \Sn{3} in all but the most
pathological cases (over 500,000 quaternions uniformly scattered across
\Sn{3}).

% We need a data point covering every empty region of the sphere.
% Consider the number of points needed to cover just the 2-sphere \Sn{2}.
% A point on \Sn{2}\ covers a region 30 degrees wide about it,
% which has the following surface area:
% \[
%  \int_{0}^{2\pi} \int_0^{\frac{\pi}{6}}  \sin \phi \ d\phi \ d\theta
%	= .841787
% \]
% using 10 degrees or \pi/18, area is only .0000289
% which leads to 431,073 points.
% Therefore, for data sets of size 15,
% it is possible to find a point $p$ with no data points within 30
% degrees of $p$.

\subsection{Finding an empty region}

There are many ways to find a point in an empty region of \Sn{3}.
We shall offer three methods, a very simple randomized method, 
a heuristic method, and an optimal method.
Since there are so many empty regions on \Sn{3},
a random choice of point on \Sn{3}\ is effective 
(where we continue to choose a point on \Sn{3}\ 
at random until a point in an empty region is found).
	% That is, we could randomly choose a point on \Sn{3}, test if it is
	% sufficiently far away from all data points, and if not then randomly choose
	% another point, and so on, until a valid point in an empty region is found.
	% The density of empty regions ensures the efficiency of this method.
		% The results in Section~\ref{sec:results} support the effectiveness
		% of a random choice of empty region \ref{}.
	% How many attempts before success? on all6: 1,1,1,1,4,56
This is an extremely simple method: the only care that must
be taken is with repeatability and coordinate-frame invariance.
If given the same data twice, we would like the same 
quaternion spline to be constructed, 
so we would like to choose the same point in an empty region
to rotate to the pole.
Fortunately, we can take advantage of the predictability 
of pseudo-random number generators.
If we use a pseudo-random number generator with the same seed,
such as the C/C++ 'rand' function, exactly the same sequence of `random'
points will be generated (see Remark~\ref{rmk:random}) 
each time it is called,
leading to the same quaternion spline if the method is repeated.
%
\begin{rmk}
\label{rmk:random}
To find a random point on \Sn{3}, we find four random numbers 
in [-1,1] defining a 4-vector, and then normalize this vector.
A random number in [-1,1] can be generated in C/C++ using the 'rand' function
as follows:
(float) (rand() \% 32767) / 16383 - 1.0.
\end{rmk}
To make the method coordinate-frame invariant,
the data is first rotated into a canonical frame 
(before the random choice of empty region) as follows:
$p_0$ is rotated to the pole, 
then $p_1$ is rotated to the $\{x_2=0,x_3=0\}$ plane (without moving $p_0$),
and finally $p_2$ is rotated to the $\{x_3=0\}$ hyperplane
(without moving $p_0$ or $p_1$).\footnote{Only
	3 points are necessary to define a unique frame of reference,
	since the added degree of freedom is locked down through
	the points lying on \Sn{3}.}
Of course, this rotation is also reversed at the end.
	% This rotation to a canonical frame for coordinate-frame invariance
	% is also useful in the following heuristic method
	% (or alternatively one has to be careful to rotate eigenvectors
	% in a coordinate-frame-independent way).
	% Rotation to a canonical frame allows use of standard rotation
	% matrix with normal n to best-fitting plane as first row,
	% n x e1 as 2nd row, n x e1 x e2 as 3rd row, and 
	% n x e1 x e2 x e3 as fourth row.
	% Similarly, for the optimal method.
	% However, a random choice is not repeatable, so we do not use it. 

There is also an heuristic approach to finding a point of an
empty region, which is not guaranteed to find a point
but usually works well.
This will find not just any empty point, but usually a very good empty
point.
Rather than directly looking for an empty region, we
try to move the data far from the pole by moving the best-fitting plane
of the data as far as possible from the pole.
Since the best-fitting plane is a good representative of most of the
points, the points tend to move far away from the pole.
However, this method is not guaranteed 
since outliers may still be moved close to the pole.
It is efficient, since the best-fitting plane can be computed
quickly from the covariance matrix of the data \cite{ballard82}.
This method is equivalent to choosing the normal of the best-fitting plane
as the empty point.

Finally, one could find the optimal empty point:
the point in the middle of the largest 
empty region which will move the data 
as far away as possible from the pole.
This is a generalization of the classical largest empty circle
problem for points in a plane \cite{shamos85}, 
and a similar technique can be used.
Let $f(p)$ be the distance of $p \in \Sn{3}$ from the nearest quaternion
(measuring distance on the surface).
The maximum of $f(p)$ 
is attained at some vertex of the Voronoi diagram of the quaternions on
\Sn{3}.
This is the desired optimal empty point.
The Voronoi diagram must be built on a surface, which 
is itself an interesting problem.
Luckily, only the Voronoi vertices need to be computed,
not the entire Voronoi diagram.
A Voronoi diagram is built out of point bisectors, and
the bisector of two points on \Sn{3}\ is a great circle of \Sn{3}.
A simplistic approach is to compute a superset of the Voronoi vertices
by intersecting all bisectors of two quaternions, and then 
choose the one furthest from all quaternions.
	% O(n choose 2) = O(n^2) circles; 
	% O(n^4) intersections
	% O(n^5) distance calculations, but can abort early on many

Some discussion is necessary about the issue of 
coordinate-frame invariance.
The observant reader may have noticed that we have established
coordinate-frame invariance for our quaternion splines on \Sn{3},
but should actually be more interested in coordinate-frame invariance 
of the objects in 3-space that are undergoing the motion.
	% (The latter is equivalent to coordinate-frame invariance in the rotation
	% group SO(3).)
Fortunately, these invariances are equivalent, due to the following lemma.
%
\begin{lemma}
Rotation of an object in 3-space by a constant amount
is equivalent to rotation of the associated quaternion on \Sn{3}
by a constant amount.
\end{lemma}
\prf
Let the original orientation of the object and the amount of
rotation be represented by the quaternions $q = (q_1,q_2,q_3,q_4)$ 
and $c = (c_1,c_2,c_3,c_4)$, respectively.
Then the orientation of the object after the rotation is $c q$,
so the associated quaternion of the object
has changed from $q$ to $c q$.
Using the formula for quaternion multiplication (\ref{eq:qmult}),
we notice that $c q$ can also be interpreted as the rotation 
of $q$ on \Sn{3}\ by a constant amount (depending on $c$):
\[
c * q = \left( \begin{array}{cccc}
	c_1 & -c_2 & -c_3 & -c_4 \\
	c_2 &  c_1 & -c_4 &  c_3 \\
	c_3 &  c_4 &  c_1 & -c_2 \\
	c_4 & -c_3 &  c_2 &  c_1
	\end{array} \right)
	\left( \begin{array}{c}
	q_1 \\ q_2 \\ q_3 \\ q_4
	\end{array} \right)
\]
This matrix is a rotation matrix in 4-space.
	% unit rows and columns, orthogonal rows, orthogonal columns
	% unit determinant (yes, checked)
\QED

Finally, note that the maps $M$ and $M^{-1}$ are not affine invariant.
	% Obviously not affine invariant, otherwise we would not need
	% to rotate away from pole, since this would have no effect
	% on the quaternion spline shape.
Thus, in order to make our quaternion spline construction 
coordinate-frame invariant, a rotation to a canonical frame 
is necessary as a preprocessing step, regardless of whether we want 
to rotate away from the pole or not.
In other words, our rotation of the data serves a dual purpose:
avoiding the pole and making the quaternion spline construction
coordinate-frame invariant.

\section{Examples}
\label{sec:eg}

To consolidate our understanding of the algorithm before moving on to its
analysis, we will now provide some examples.
Figure~\ref{fig:eg6} shows six rational quaternion splines designed using the
method of this paper (on input of 5, 5, 6, 10, 66, and 100 quaternions,
respectively).
\add{All of these examples have $C^2$ continuity, since we choose to use cubic B-spline
curves for the interpolation in Euclidean space.}
We can visualize quaternion splines, which reside in 4-space,
using the following trick.
Points of $x_3=0$ are mapped by $M^{-1}$ 
to points of $x_2=0$ in Euclidean space,
which are then mapped back by $M$ to points of $x_3=0$ on \Sn{3}\ again.
Therefore, we can visualize quaternion splines in 3 dimensions
using data samples with $x_3=0$.
	% These figures are still slightly misleading, since they are
	% three-dimensional images of a four-dimensional scene.
We only apply this restriction to $x_3=0$ to the examples
that we want to visualize in this paper.
The quaternion splines that we analyze in Section~\ref{sec:results}
reside in 4-space, with fully general quaternion data.
Figure~\ref{fig:invariance} illustrates the coordinate-frame invariance
of the method: rotated data sets generate the same curve, rotated.

\begin{figure}
\vspace{5in}
\special{psfile=/usr/people/jj/modelTR/3-spline/img/all6.ps hoffset=-20}
\caption{Examples}
\label{fig:eg6}
% s3spline -m 90 < data5-1 &
% s3spline -m 120 < data/5nicepts.dat &
% s3spline -m 120 -a < data/input3 &
% s3spline -m 90 < data10-2 &
% s3spline       < data66-20deg &
% s3spline -m 90 < data100-1 &
% tops all6.rgb -m 8.5 4 > all6.ps
\end{figure}

\begin{figure}
\vspace{3in}
\special{psfile=/usr/people/jj/modelTR/3-spline/img/data5-1+rot.ps hoffset=80}
\caption{Coordinate-frame invariance}
\label{fig:invariance}
% s3spline -m 90 < data5-1 &
% s3spline -m 90 < data5-1-rot &
% tops data5-1+rot.rgb -m 8.5 2 > data5-1+rot.ps
\end{figure}

% Examples of our quaternion splines used in orientation control (for next paper)

\section{Analysis}
\label{sec:results}

In this section, we consider the quality of our rational quaternion
spline and compare it to other quaternion splines.
\cite{barr92} proposes that covariant acceleration
be used as a measure of the quality of a quaternion spline:
the smaller the covariant acceleration, the more desirable the curve.
%
\begin{defn2}
The {\bf covariant acceleration} of a curve segment $c(t)$, $t \in I$, is 
$\int_I \| c''(t) \setminus c(t)\| \ dt$, where
$a \setminus b = a - (\frac{a \cdot b}{b \cdot b}) b$.
\end{defn2}
%
Their motivation is as follows.
The acceleration of a curve on a surface can be decomposed into
normal and covariant components: normal acceleration is necessary
as it keeps the curve on the surface, but covariant acceleration
is not.
%
% They built a quaternion spline that minimized covariant acceleration,
% through constrained optimization, as their interpretation 
% of the `optimal' curve.
% Incidentally, they
% used the term `tangential acceleration', but we prefer to use the 
% more classical term, covariant acceleration.
Low covariant acceleration can also be motivated by appealing to the 
design of short quaternion splines, as discussed in 
Section~\ref{sec:eucdesign}.
The shortest curve through a set of points is necessarily a geodesic,
which has zero covariant acceleration \cite{thorpe79}.
	% (The geodesic is not rational, in general,
	% and would not be desirable even if it was,
	% as the rigidity of perfection necessarily ignores all other
	% constraints.)
	
We will use covariant acceleration as a measure of curve quality.
One should note that covariant acceleration is only a guide to quality
and should not be viewed as an absolute measure.
For example, geodesics with their minimal covariant acceleration
are not necessarily desirable as quaternion splines, as
the rigidity of their perfection can force them into strange shapes
not conducive to motion control.
	% just as the rigid perfection of minimal surfaces can generate
	% unusual shapes.
\Comment{
We compute an approximation to covariant acceleration by discretely sampling the curve:
% and computing the covariant acceleration at each of these samples:
\[ \int_I \| c''(t) \setminus c(t)\| \ dt
\approx \sum_{i=0}^N 
	\| c''(t_0 + i \delta) \setminus c(t_0 + i \delta) \|\  \delta.
\]
}
The best judge of the quality of a quaternion spline
is the motion that it generates.
We have found that our quaternion splines generate very natural motions,
and have shown their effectiveness for animation in \cite{jjjw95}.
	% \footnote{This paper uses
	% a slightly different choice of point on $M^{-1}(p_i)$.}
% Their use in motion control will be further explored in \cite{jj00}.

\subsection{Choice of point on $M^{-1}(p_i)$}

We first explore the advantage of the flexibility 
to choose any points on the lines $M^{-1}(p_i)$.
We compare the following strategies for choosing the point on $M^{-1}(p_i)$:
the closest point,
% (the method outlined in Section~\ref{sec:eucdesign} and the preferred method),
the intersection of $M^{-1}(p_i)$ with \Sn{3}, and
the defining point of $M^{-1}(p_i)$.
% which is basically an arbitrary choice of point on the line in Euclidean space.
We find that these choices are progressively worse, as expected.
All of these data sets have 100 points.
	% Recall that (1) was not used since it has trouble with the definition
	% of tangents and large data sets.
	% Therefore, we use small data sets that do not require subdivision
	% for this comparison.

\begin{table}[h]
\begin{tabular}{|c|c|c|c|c|}  	\hline
Data set & \multicolumn{3}{c|}{Covariant acceleration of our curve} & Improvement factor\\ \hline
	 & Closest & On \Sn{3} & Defining point & \\ \hline
1 & 23.9 & 394   & 410   &  17  \\ \hline
2 & 30.0 & 166   & 190	 &  6.3 \\ \hline
3 & 14.0 & 31.5  & 36.3  &  2.6 \\ \hline
% Data set 1: data100-2 (see script)
% Data set 2: test100-55deg
% Data set 3: test100-30deg
% 5000 probes
\end{tabular}
\caption{Curve quality with various choices of point on $M^{-1}(p_i)$}
\label{tab:cov}
\end{table}

\subsection{Trim curves}

We next compare our rational quaternion spline with trim curves.
Recall from Section~\ref{sec:intro} that trim curves also use modeling 
in Euclidean space, in this case parameter space, 
but use the surface parameterization as the map to the surface.
This comparison should capture the advantage of using a well-designed map to the
surface, as well as the advantage of 
	% a map from the domain $\Re^4$ rather than $\Re^3$, with its associated 
flexibility in choosing an inverse point along a line.
% We use the standard rational parameterization of \Sn{3},
% which is fundamentally equivalent to using stereographic projection as the 
% map to and from \Sn{3}.
% \begin{equation}
% \label{Snparam}
%	S(t_1,t_2,t_3) = 
%	\frac{1}{t_1^2 + t_2^2 + t_3^2 + 1} 
%	(2t_1, 2t_2, 2t_3, t_1^2 + t_2^2 + t_3^2 - 1) \ \ \ \ t_i \in (-\infty,\infty)
% \end{equation}
% inverse of this parameterization is stereographic projection
%
% A better comparison of trim curves to Euclidean curves might be to
% use a Bezier representation of the sphere.
%
Typical results from these tests are presented in Table~\ref{tab:cov},
ordered by the amount of improvement.
These data sets, like all of the others we use in this section, 
are generated randomly, with restrictions on distance between consecutive
quaternions.
This shows the consistent improvement when using the new approach,
usually about an order of magnitude.

\begin{table}
\begin{tabular}{|c|c|c|c|}  	\hline
Data set & \multicolumn{2}{c|}{Covariant acceleration} & Improvement factor \\ \hline
	 & Our curve  & Trim curve & \\ \hline
1 & 23.9 & 764 & 32 \\ \hline
2 & 25.0 & 457 & 18 \\ \hline
3 & 41.3 & 352 & 8.5\\ \hline
4 & 15.9 & 90.6 & 5.7 \\ \hline
5 & 14.8 & 42.4 & 2.9 \\ \hline
% Data set 1: data100-2 (see script)
% Data set 2: test100-60deg
% Data set 3: test200-50deg
% Data set 4: test500-20deg
% Data set 5: test300-20deg
% 5000 probes
\end{tabular}
\caption{A comparison with trim curves}
\label{tab:cov2}
\end{table}

\subsection{Park and Ravani's method}

We now compare to a nonrational quaternion spline.
It is meaningless to compare with Barr et. al.'s methods \cite{barr92,rama97}
since they explicitly minimize covariant acceleration.
Therefore, we choose to compare to another good nonrational spline,
Park and Ravani's method \cite{park97}.
The results are in Table~\ref{tab:cov3}.
	% We can only use small inputs since, in our implementation,
	% the intermediate values in the Lie method 
	% % (led by u) 
	% become progressively larger at an explosive rate,
	% overflowing after about 15 points.
	% However, these small inputs are enough to show our rational quaternion spline
	% performs well against the nonrational Lie method.

% Is covariant acceleration computation of Ravani correct? Awfully large.

\begin{table}
\begin{tabular}{|c|c|c|c|}  	\hline
Data set & \multicolumn{2}{c|}{Covariant acceleration} & Improvement factor \\ \hline
	 & Our curve & Lie curve & \\ \hline
1 & 0.364 & 652 & 1791 \\ \hline
2 & 0.643 & 555 & 863 \\ \hline
3 & 13.3  & 645 & 48 \\ \hline
% Data set 1: test5-20deg (see script)
% Data set 2: test10-20deg
% Data set 3: data10-1
\end{tabular}
\caption{A comparison with Park and Ravani}
\label{tab:cov3}
\end{table}

\subsection{Execution times}

As one would expect from its reliance on classical NURBS technology
and rational computations, our method is very efficient.
Table~\ref{tab:time} gives some typical results,
running on a 225MHz SGI Octane workstation.
The first six results are for the examples in Figure~\ref{fig:eg6}.
Times of 0 are possible since the profiler we use measures in intervals
of 30 microseconds.
Clearly, curve design is entirely interactive and responsive.

\begin{table}
\begin{tabular}{|c|c|c|}  	\hline
Data set & \# of data points & Time (seconds) \\ \hline
1 & 5 & 0.00 \\ \hline
2 & 5 & 0.00 \\ \hline
3 & 6 & 0.00 \\ \hline
4 & 10 & 0.00 \\ \hline
5 & 66 & 0.03 \\ \hline
6 & 100 & 0.03 \\ \hline
7 & 200 & 0.12 \\ \hline	% dataset: test200-50deg (see script)
8 & 500 & 0.27 \\ \hline	% dataset: test500-20deg 
\end{tabular}
\caption{Execution times}
\label{tab:time}
\end{table}

\section{Conclusions and future work}
\label{sec:conclusions}

We have developed a rational quaternion spline of high quality.
It is a B-spline of arbitrary continuity, and is easily built
using the classical construction of interpolating B-splines in Euclidean 
space.\footnote{The curve can be any rational curve, not necessarily expressed
	as a B-spline curve, but we assume that the B-spline curve will be the
	most commonly used.}
The dependence of the method on existing B-spline technology leads to
great efficiency in both the implementation and the execution of the
method, which can promote interactive control for animation or robot motion.
	% or spacecraft control
Previous quaternion splines have required new tools (e.g., slerping,
constrained optimization, biarcs) that prevent them from being
incorporated into existing NURBS modelers.

\add{An exploration of the use of this new quaternion spline for motion editing 
and spacetime optimization would be most interesting, 
motivated by its closed form, simple derivative computation, and efficiency.}\

We believe that there is additional promise in the Euclidean space approach
for curve design on surfaces.
In particular, there are two forms of inherent flexibility in the method
that deserve further study:
map flexibility (the choice of map to the surface)
and interpolation flexibility (the choice of point on the inverse image curve
during Euclidean space interpolation).
In the future, we also want to explore a direct interpolation of the set
of inverse image curves by a curve.
	% General problem of designing a curve that interpolates curves, rather than points.
% A full understanding of the relationship between curves in Euclidean
% and Riemannian space is also an interesting challenge.

\section{Acknowledgements}

I appreciate comments of Sebastian Grassia on the paper.

\bibliographystyle{plain}
\begin{thebibliography}{Lozano-Perez 83}

\bibitem[Ballard 82]{ballard82}
Ballard, D. and C. Brown (1982)
Computer Vision.
Prentice-Hall (Englewood Cliffs, NJ).

\bibitem[Barr 92]{barr92}
Barr, A. and B. Currin and S. Gabriel and J. Hughes (1992)
Smooth Interpolation of Orientations with Angular Velocity Constraints
using Quaternions.
SIGGRAPH '92, 313--320.

\bibitem[Dickson 52]{dickson52}
Dickson, L.E. (1952) History of the Theory of Numbers: Volume II,
Diophantine Analysis.  Chelsea (New York).

\bibitem[Dietz 93]{dietz93}
Dietz, R. and J. Hoschek and B. Juttler (1993)
An algebraic approach to curves and surfaces on the sphere and
on other quadrics.
Computer Aided Geometric Design 10, 211-229.

\bibitem[Duff 85]{duff85}
Duff, T. (1985)
Quaternion splines for animating orientation.
1985 Monterey Computer Graphics Workshop, 54--62.

\bibitem[Farin 97]{farin97}
Farin, G. (1997)
Curves and Surfaces for CAGD: A Practical Guide (4th edition).
Academic Press (New York).

\bibitem[Foley 96]{foley96}
Foley, J. and A. van Dam and S. Feiner and J. Hughes (1996)
Computer Graphics: Principles and Practice (2nd edition in C).
Addison-Wesley (Reading, Massachusetts).

\bibitem[Golub 89]{golubvanloan89}
Golub, G. and C. Van Loan (1989)
Matrix Computations.
2nd edition, Johns Hopkins University Press (Baltimore).

\bibitem[Goldstein 50]{goldstein50}
Goldstein, H. (1950)
Classical Mechanics.
Addison-Wesley (Reading, Massachusetts).

\bibitem[Harris 92]{harris92}
Harris, J. (1992)
Algebraic Geometry: A First Course.
Springer-Verlag (New York).

\bibitem[Hart 94]{hart94}
Hart, J. and G. Francis and L. Kauffman (1994)
Visualizing Quaternion Rotation.
ACM Transactions on Graphics 13(3), July, 256--276.

\bibitem[Herstein 75]{herstein75}
Herstein, I. (1975) Topics in Algebra.
2nd edition, John Wiley (New York).

\bibitem[Hoschek 92]{hoschekSeemann92}
Hoschek, J. and G. Seemann (1992)
Spherical splines.
Mathematical Modeling and Numerical Analysis, 26(1), 1--22.

\bibitem[Johnstone and Williams 95]{jjjw95}
Johnstone, J. and J. Williams (1995)
Rational Control of Orientation for Animation.
{\em Graphics Interface '95}, Quebec City, 179--186.

\bibitem[Johnstone 99b]{jj98b}
Johnstone, J.K. (1999)
The Most Natural Map to the $n$-sphere.
Manuscript.
Available at http://www.cis.uab.edu/info/faculty/jj/cos.html.

\bibitem[Kendall 61]{kendall61}
Kendall, M.G. (1961)
A Course in the Geometry of n Dimensions.
Charles Griffin (London).

\bibitem[Kim 95]{kim95}
Kim, M.-J. and M.-S. Kim and S. Shin (1995)
A General Construction Scheme for Unit Quaternion Curves
with Simple Higher Order Derivatives.
SIGGRAPH '95, 369--376.

\bibitem[Kreyszig 63]{kreyszig63}
Kreyszig, E. (1963)
Differential Geometry.
Dover (New York).

\bibitem[Lang 79]{lang79}
Lang, S. (1979)
Calculus of Several Variables, 2nd edition.
Addison-Wesley (Reading, Mass).

\bibitem[Misner 73]{misner73}
Misner, C. and K. Thorne and J. Wheeler (1973)
Gravitation.
W.H. Freeman (San Francisco).

\bibitem[Nam 95]{nam95}
Nam, K.-W. and M.-S. Kim (1995)
Hermite interpolation of solid orientations based on a smooth blending
of two great circular arcs on SO(3).
Proc. of CG International '95.

\bibitem[Nielson 92]{nielson92}
Nielson, G. and R. Heiland (1992)
Animated rotations using quaternions and splines on a 4D sphere.
Programming and Computer Software, 145--154.

\bibitem[Nielson 93]{nielson93}
Nielson, G. (1993)
Smooth interpolation of orientations.
In Models and Techniques in Computer Animation, Springer-Verlag (New York),
75--93.

\bibitem[Park 95]{park95}
Park, F. and B. Ravani (1995)
Bezier Curves on Riemannian Manifolds and Lie Groups with
Kinematics Applications.
Transactions of the ASME 117, 36--54.

\bibitem[Park 97]{park97}
Park, F. and B. Ravani (1997)
Smooth Invariant Interpolation of Rotations.
ACM Transactions on Graphics 16(3), 277--295.

\bibitem[Pletinckx 89]{pletinckx89}
Pletinckx, D. (1989) 
Quaternion calculus as a basic tool in computer graphics.
The Visual Computer 5, 2--13.

\bibitem[Popovic 99]{popovic99}
Popovic, Z. and A. Witkin (1999)
Physically Based Motion Transformation.
SIGGRAPH '99, 11--20.

\bibitem[Ramamoorthi 97]{rama97}
Ramamoorthi, R. and A. Barr (1997)
Fast Construction of Accurate Quaternion Splines.
SIGGRAPH '97, Los Angeles, 287--292.

\bibitem[Schlag 91]{schlag91}
Schlag, J. (1991) Using geometric constructions to interpolate
orientation with quaternions.  In Graphics Gems II, Academic Press (New York),
377--380.

\bibitem[Preparata 85]{shamos85}
Preparata, F. and M. Shamos (1985)
Computational Geometry: An Introduction.
Springer-Verlag (New York).

\bibitem[Shoemake 85]{shoemake85}
Shoemake, K. (1985) Animating rotation with quaternion curves.
SIGGRAPH '85, San Francisco, 19(3), 245--254.

\bibitem[Thorpe 79]{thorpe79}
Thorpe, J. (1979)
Elementary Topics in Differential Geometry.
Springer-Verlag (New York).

\bibitem[Wang 93]{wang93}
Wang, W. and B. Joe (1993)
Orientation Interpolation in Quaternion Space using Spherical Biarcs.
Graphics Interface '93, 24--32.

\bibitem[Wang 94]{wang94}
Wang, W. (1994)
Rational Spherical Curves.
Technical Report, Dept. of Computer Science, University of Hong Kong.

\bibitem[Witkin 88]{witkin88}
Witkin, A. and M. Kass (1988)
Spacetime Constraints.
Computer Graphics 22(4) (Proc. SIGGRAPH '88), 159--168.

% \bibitem[Woo 97]{opengl97}
% Woo, M. and J. Neider and T. Davis (1997)
% OpenGL Programming Guide (2nd edition).
% Addison-Wesley Developers Press (Reading, Mass.),
% pp. 464-8.
\end{thebibliography}

\section{Appendix}

In this appendix, we establish that the choice of $q_i$ as the point on \Sn{3}
(Section~\ref{sec:eucdesign})
is a good approximation to the closest point to $q_{i-1}$.
Consider two unit quaternions $p_1 = (a,b,c,d)$ and $p_2 = (A,B,C,D)$.
Let $\epsilon$ be their difference in $x_1$-coordinate and $\theta$
be the angle between $p_1$ and $p_2$.
Consider the difference between the choice of inverse image as closest point $q_2$
and point $q_2'$ on \Sn{3} (Figure~\ref{fig:QQdist}).
The distance between $q_2$ and $q_2'$ is $1-\cos \alpha$, where
$\alpha$ is the angle between the inverse image lines $M^{-1}(p_1)$ and
$M^{-1}(p_2)$ (see Figure~\ref{fig:QQdist} again).
$\cos \alpha$ can be reexpressed as follows:
\[
	\cos \alpha = \frac{(b,c,d,1-a)}{\sqrt{2-2a}}
	\cdot
	\frac{(B,C,D,1-A)}{\sqrt{2-2A}}
	= \frac{\cos \theta + 1 - a - A}
	       {2\sqrt{1-a-A+aA}}
\]
since $\cos\theta = (a,b,c,d) \cdot (A,B,C,D)$.
Reexpressing $A$ as $a + \epsilon$,
\[
\cos \alpha = \frac{\cos \theta + 1 - 2a - \epsilon}
	       {2\sqrt{1-2a-\epsilon+a^2 + \epsilon a}}
	= \frac{\cos \theta + 1 - 2a - \epsilon}
	       {2\sqrt{(1-a-\frac{\epsilon}{2})^2 - \frac{\epsilon^2}{4}}}
\]
Suppose that $\frac{\epsilon^2}{4}$ is very small and can be ignored,
and that $\theta$ is small so that $\cos \theta$ is close to 1.
This is true if the quaternion data is densely sampled.
Then 
\[
\cos \alpha \approx \frac{2 - 2a - \epsilon}{2(1-a-\frac{\epsilon}{2})} = 1,\ \ 
1-\cos \alpha \approx 0
\]
and the distance between $q_2$ and $q_2'$, $1 - \cos\alpha$, is small.  
We conclude that the choice of $q_i$ as the 'point on \Sn{3}'
is a good approximation to the choice of $q_i$ as the closest point to $q_{i-1}$,
especially when the quaternion data is densely sampled.

\begin{figure}
\vspace{2.5in}
\special{psfile=/usr/people/jj/modelTR/3-spline/img/dist1-cosAlpha.ps
	 hoffset=150}
\caption{The distance between $q_2$ and $q_2'$ is $1-\cos\alpha$}
% file: dist1-cosAlpha.showcase
% tops dist1-cosAlpha.rgb -m 6.5 1.5 > dist1-cosAlpha.ps
\label{fig:QQdist}
\end{figure}

\end{document}

\ifFull
The main battering ram for new breakthrough should be the rationality
of the quaternion spline and its good quality (better than rational trim curve
or nonrational Lie methods or rational biarc).
So present the paper as a new high-quality rational quaternion spline.
We can talk about extensions to other surfaces in the conclusions, but
it is not yet a general method for design of curves on surfaces,
so save that argument for when it blossoms beyond S3.
The argument of mapping to Euclidean space still holds and can motivate
the method (paragraph 3 on).
But start with the rational quaternion spline problem and proceed
to the Euclidean space approach rather than the other way around.
This also deflects criticism as extension of Dietz.
It is not a massive change from the approach of Dietz of curves on
surfaces (although we choose points in Euclidean space, rather than 
curves directly from curves, which changes many things)
and the general approach is not the main point anyway.
\fi

\Comment{
Note: 
Rational maps to a nontrivial surface are bound to be nonlinear
and algebraically stated (this, we might change)
so not coordinate-frame invariant.
(Think of Bezier curve's coordinate-frame invariance, which depends
on its geometric description by de Casteljau or alternatively
its use of linear interpolation in the Bezier basis algebraic version.)
Thus, the trick of this paper of using a rotation of every input
to a canonical position as a preprocessing step is a perhaps universal
technique that we can use to enforce coordinate-frame invariance
when using maps to a surface in Riemannian modeling.
}

% \section{Motion control}
% \label{sec:motion}
% \subsection{Orientation keyframes for animation and robot motion}

% Call the new quaternion splines `Euclidean quaternion splines'.
% Call the general class of new interpolating curves on the surface
% `Euclidean curves on the surface'.

% {\bf Separate paper on applications of S3 splines to orientation control,
% sing examples from graphics and robotics.}


% \section{Curve interpolation}

% \section{Interactive editing of a curve}




\Comment{
\subsection{The best rotation}

What single rotation will move the entire set of data points far from the pole?
It makes sense to find a good representative for all the data
points and rotate it farthest away from the pole.
We choose the best-fitting plane.
The added benefit of this choice is that the best-fitting plane
can be computed efficiently using the covariance matrix.

% see end of document for commentary on best-fitting plane itself

To maximize the distance of the best-fitting plane from the pole,
its normal should be oriented towards the pole (Figure~\ref{fig:avoid}).
We need to decide whether to rotate $+v_{min}$ or $-v_{min}$ to the pole
$(1,0,0,0)$, choosing the rotation that moves the best-fitting plane
farthest away from the pole.
Under either rotation, the best-fitting plane 
will be rotated to a plane $x_1 = c$ for some $c \in \Re$.
Since the sample mean lies on the best-fitting plane,
we can compute its image under the two rotations
(or just the $x_1$-coordinate of its image) 
and choose the rotation that yields the smaller $x_1$-coordinate.

\begin{example}
\label{eg:resolve}
The choice of $\pm v_{min}$ is an important one.
Consider a set of data points on the plane $x_1=.9$.
The best-fitting plane is $x_1=.9$
and the direction of minimum dispersion is $\pm(1,0,0,0)$.
Rotating $(1,0,0,0)$ to the pole does not move the points,
leaving them undesirably close to the pole,
but rotating $(-1,0,0,0)$ to the pole achieves the
goal of flipping the data points to the other side.
\end{example}

\begin{figure}
\vspace{1in}
\caption{Input set (a) before and (b) after rotation}
\label{fig:avoid}
\end{figure}
% figure of an example with best-fitting plane shown

You need a tie-breaking scheme for assigning
the eigenvectors to positive and negative coordinate axes,
so that the method is coordinate-frame invariant.
The present idea is to project the sample points onto
the eigenvector and choose the direction of the extreme
point as the positive axis.
If there is a tie, choose the extreme point with the smaller
index.  (Of course, this makes the method variant under
*reversal* of the data, which is not so nice.)
Other problem with this method: what if sample points are coplanar?
Then some eigenvector is degenerate, and we have problems
making consistent rotation.

The rotation matrix that rotates $v_{min}$
to $(1,0,0,0)$ is computed easily, using the fact that the rows
of a rotation matrix are mapped to the coordinate axes under the
rotation.
Thus, $\frac{v_{min}}{\|v_{min}\|}$ is the first row of the desired rotation matrix,
and the other rows are chosen orthogonal to $v_{min}$ using
cross product.

A corollary of the choice of best-fitting plane rotation
is that the method is bi-invariant (coordinate-frame invariant).
Bi-invariant, since no matter where a point set is rotated,
it will be perturbed to the same optimal (for that point set)
orientation for mapping to Euclidean space.
Claim: no matter how you rotate the input, it is perturbed to the same
place before it is mapped to image space.
The map $M$ is not invariant to rotation, but the overall method is.

Since the best-fitting plane of $\{p_i\}_{i=1,\ldots,k}$ 
minimizes distance from the point set in the least-squares sense,
there may be outliers that are far from the best-fitting plane.
These outliers could be moved close to the pole by the rotation,
even as the best-fitting plane is moved far from it.
A solution to this problem requires subdivision of the problem
into two or more quaternion splines.
After all, a valid rotation can always be found if the data
set is small enough.
For example, a set of 4 quaternions can always be rotated to
the hyperplane $x_1=0$, which is far from the pole.
Subdivision requires an understanding of how to prescribe tangent
information.

} % end of /Comment


\Comment{
Best-fitting plane commentary

\begin{defn2}
The {\bf best-fitting plane} of a set of points $\{r_i\}_{i=1,\ldots,k}$
is the plane that minimizes $\sum_{i=1}^k \mbox{dist}^2 (r_i,\mbox{plane})$.
The dispersion (or {\bf variance}) of a set of scalars
$\{x_i\}_{i=1,\ldots,k}$ is $\sum_{i=1}^k \frac{(x_i -
\bar{x})^2}{k-1}$, where $\bar{x} = \sum_{i=1}^k x_i / k$ is the mean.
This is a measure of the spread of the sample.
The dispersion of a set of points in the direction $v$ is the dispersion
of the orthogonal projection of the points onto $v$.
The {\bf covariance matrix} of a set of points $\{r_i\}_{i=1,\ldots,k}$ 
is $\sum_{i=1}^k \frac{(r_i - \bar{r})(r_i - \bar{r})^T}{k-1}$,
where $\bar{r}$ is the sample mean of the points.
\end{defn2}

The normal of the best-fitting plane of the data points $\{p_i\}_{i=1,\ldots,k}$
is the direction of minimum dispersion of $\{p_i\}$.
This is the direction that the plane can best afford to ignore,
and hence the normal direction.
The covariance matrix is a generalization of dispersion to many dimensions.
Not surprisingly, it can be used to compute properties of dispersion.
In particular, the direction of minimum dispersion of 
the point set $\{p_i\}$ is the minimum 
eigenvector (say $v_{min}$) of the covariance matrix of $\{p_i\}$ \cite{ballard82}.
Hence, the best-fitting plane is the plane through the sample mean
with normal $v_{min}$.
}

% Other optimization measures (strain energy, measures from spacecraft dynamics, etc.).

Improvement over trim curves:
The image-space approach to curve design on surfaces offers
added flexibility in the choice of map and the choice of interpolation
over the classical trim-curve approach to curve design on surfaces.
Using the important case of curves on \Sn{3}, 
we have shown that this added flexibility improves curve design.
By choosing the most natural map $M$ rather than the surface parameterization,
better curves result, so map flexibility is an improvement.
By choosing the closest-point sequence of points on the inverse lines
of the generalization of stereographic projection
rather than the unique surface parameterization inverse,
better curves result, so interpolation flexibility is an improvement.
