\documentclass[12pt]{article} 
\usepackage{times}
\usepackage[pdftex]{graphicx}
\makeatletter
\def\@maketitle{\newpage
 \null
 \vskip 2em                   % Vertical space above title.
 \begin{center}
       {\Large\bf \@title \par}  % Title set in \Large size. 
       \vskip .5em               % Vertical space after title.
       {\lineskip .5em           %  each author set in a tabular environment
        \begin{tabular}[t]{c}\@author 
        \end{tabular}\par}                   
  \end{center}
 \par
 \vskip .5em}                 % Vertical space after author
\makeatother

% default values are 
% \parskip=0pt plus1pt
% \parindent=20pt

\newcommand{\SingleSpace}{\edef\baselinestretch{0.9}\Large\normalsize}
\newcommand{\DoubleSpace}{\edef\baselinestretch{1.4}\Large\normalsize}
\newcommand{\Comment}[1]{\relax}  % makes a "comment" (not expanded)
\newcommand{\Heading}[1]{\par\noindent{\bf#1}\nobreak}
\newcommand{\Tail}[1]{\nobreak\par\noindent{\bf#1}}
\newcommand{\QED}{\vrule height 1.4ex width 1.0ex depth -.1ex\ \vspace{.3in}} % square box
\newcommand{\arc}[1]{\mbox{$\stackrel{\frown}{#1}$}}
\newcommand{\lyne}[1]{\mbox{$\stackrel{\leftrightarrow}{#1}$}}
\newcommand{\ray}[1]{\mbox{$\vec{#1}$}}          
\newcommand{\seg}[1]{\mbox{$\overline{#1}$}}
\newcommand{\tab}{\hspace*{.2in}}
\newcommand{\se}{\mbox{$_{\epsilon}$}}  % subscript epsilon
\newcommand{\ie}{\mbox{i.e.}}
\newcommand{\eg}{\mbox{e.\ g.\ }}
\newcommand{\figg}[3]{\begin{figure}[htbp]\vspace{#3}\caption{#2}\label{#1}\end{figure}}
\newcommand{\be}{\begin{equation}}
\newcommand{\ee}{\end{equation}}
\newcommand{\prf}{\noindent{{\bf Proof}:\ \ \ }}
\newcommand{\choice}[2]{\mbox{\footnotesize{$\left( \begin{array}{c} #1 \\ #2 \end{array} \right)$}}}      
\newcommand{\scriptchoice}[2]{\mbox{\scriptsize{$\left( \begin{array}{c} #1 \\ #2 \end{array} \right)$}}}
\newcommand{\tinychoice}[2]{\mbox{\tiny{$\left( \begin{array}{c} #1 \\ #2 \end{array} \right)$}}}
\newcommand{\ddt}{\frac{\partial}{\partial t}}
\newcommand{\Sn}[1]{\mbox{{\bf S}$^{#1}$}}
\newcommand{\calP}[1]{\mbox{{\bf {\cal P}}$^{#1}$}}

\newtheorem{theorem}{Theorem}	
\newtheorem{rmk}[theorem]{Remark}
\newtheorem{example}[theorem]{Example}
\newtheorem{conjecture}[theorem]{Conjecture}
\newtheorem{claim}[theorem]{Claim}
\newtheorem{notation}[theorem]{Notation}
\newtheorem{lemma}[theorem]{Lemma}
\newtheorem{corollary}[theorem]{Corollary}
\newtheorem{defn2}[theorem]{Definition}
\newtheorem{observation}[theorem]{Observation}
\newtheorem{implementation}[theorem]{Implementation note}

% \font\timesr10
% \newfont{\timesroman}{timesr10}
% \timesroman


\newif\ifTalk
\Talkfalse
\newif\ifJournal
\Journalfalse
\newif\ifFuture		% issues useful in future papers
\Futurefalse

\setlength{\oddsidemargin}{0pt}
\setlength{\topmargin}{0in}
\setlength{\textheight}{8.6in}
\setlength{\textwidth}{6.875in}
\setlength{\columnsep}{5mm}
\markright{Orientation control (\today) \hfill}
\pagestyle{myheadings}

% -----------------------------------------------------------------------------
\title{Controlling orientation during a motion:\\ 
       an approach using rational quaternion splines}
%       a discussion of orientation control with rational quaternion splines}
% Getting oriented
%:\\A rational approach to orientation control for animation 
%       and motion planning}
% Rational orientation planning I:  free motion
% Rational orientation planning II: motion among obstacles
% Rational orientation planning III: 
\author{J.K. Johnstone}
% \thanks{This work was supported by the National Science Foundation under grant CCR-0203586.}\\
% Computer and Information Sciences\\
% University of Alabama at Birmingham\\
% University Station, Birmingham, AL 35294.}

\begin{document}
\maketitle

% -----------------------------------------------------------------------------

\begin{abstract}
Eventual goal: orientation obstacles.
But first we need to clarify quaternion spline theory, which define the
orientation obstacles.
\end{abstract}

\section{Introduction}

This paper studies the design of quaternion splines for controlling the orientation
of a rigid object.
The impact of this subject in computer graphics is felt predominantly 
in computer animation (from character animation \cite{catmull, elaborations on sig78}
to the editing of motion capture \cite{gleicher} to camera control).
Orientation control is also of interest in robot motion (from spacecraft docking 
\cite{junkins?} to motion planning AI for game design \cite{game design book}).

The problem of orientation control addressed in this paper is a classical 
interpolation problem posed in the context of orientations.
%
\begin{defn2}
{\bf Orientation control} is the specification of an object's orientation along
an entire path when only a small finite collection of orientations are given.
% {\bf Orientation planning} is orientation control in the presence of obstacles.
\end{defn2}
%
Since the best representation for the orientation of a rigid object in computer animation
is the unit quaternion,
and a unit quaternion can be identified with a point on $S^3$ (the unit sphere 
in 4-space), orientation control can be identified with the design of a curve on $S^3$
that interpolates a sequence of points on $S^3$ (the unit quaternions representing
the given orientations of the rigid object), which is called a quaternion spline.

A challenging aspect of the problem of quaternion spline design is
the design of a rational quaternion spline.
A rational quaternion spline may have benefits in efficiency
(rational polynomials are the most efficient representation in graphics)
and compatibility (due to their efficiency and elegance, rational polynomials are
the dominant representation for models).
The design of a rational quaternion spline is challenging because 
of the spline's nonlinear constraint to the sphere $S^3$.
We explore the ramifications of a reductionist solution:
reducing the interpolation of quaternions on $S^3$ to the interpolation of
points in $\Re^4$ through the intermediary of a rational map from $\Re^4$ to $S^3$.

The main challenges of this approach are threefold: the development of a rational map from
$\Re^4$ to $S^3$; coping with the inevitable poles of this map;
and derivative control.
New solutions to all of these problems are proposed in this paper:
namely, the use of inverse stereographic projection as a rational map from 
$\Re^4$ to $S^3$; a divide-and-conquer approach to perturbation away from the pole,
using principal component analysis; 
% affine-invariant through the use of an opening normalization so that all 
% rotation-equivalent datasets start from the same position.
and the ability to control arbitrary derivatives of the quaternion spline.
To realize the rotation of data points necessary for pole avoidance,
we also propose some techniques for rotation in higher dimensions.

\section{Algorithm}

Preprocessing: get a good source of orientations.
Even the flythrough isn't quite enough.
A purely theoretical paper is another possibility for now.
Bracketted items are refinement stages.

\begin{itemize}
\item Encode object orientations as quaternions. (Can use GUI and/or translation.)
\item {[Normalize quaternion dataset (nonlinear maps are commonly not affine-invariant).]}
\item {[Decompose into subsets (perhaps on the fly).]}
\item {[Perturb subset away from pole (poles are common to maps involving the sphere).]}
\item Map subset away from sphere.
      For subsets after first subset, map end derivatives too to guarantee continuity.
      For last subset if closed curve is wanted, map end derivatives of first subset too.
\item Interpolate freely.
\item Map curve back to sphere. (As Bezier curve, if you want.)
\item {[Undo perturbation and normalization.]}
\item Decode quaternion spline into smooth changes in object orientation.
\end{itemize}

Publish this software (with C1 continuity at the moment) and this paper.

\section{Related literature}

Previous work on quaternion splines (Shoemake, Barr); rational quaternion splines (JJ);
curve design on $S^2$ (Juttler); double quaternions (Wagner, Pottmann).

\section{Representation of orientation}

The unit quaternion is the preferred representation for orientation in animation.
Other popular representations, namely the rotation matrix, Euler angles and 
spinor matrices, are appropriate in other contexts, such as the graphics pipeline, 
robot arm control, atomic orientation in high-energy physics(?) or relativity theory,
but the quaternion dominates in computer animation.
This section defines all of these representations, with an emphasis on the quaternion,
and analyzes their strengths and weaknesses.

\section{Rational maps to the sphere}

\subsection{The poles of rational spherical maps}

Stereographic projection clearly has a pole at the point of projection, (1,0,0,0).

\section{Perturbation away from the pole}

To apply our method, whose first step is to map the point data away from $S^3$, 
the point data should not lie near the map's pole.
This motivates a perturbation phase before the actual algorithm begins,
which is described in this section.

We observe that perturbation is simple with pointsets that span a small
region of the sphere, since they can be rotated far away from the pole.
Pole avoidance, or perturbation, 
is equivalent to finding an empty region of the sphere (a region containing
no points) since the pole can then be rotated to the middle of this empty region.
This empty region must be large enough that no point lies near its center.
It is certainly more difficult to find an empty region for larger datasets, and
may even be impossible in theory (although in practice, a huge dataset is required
for $S^3$ to have no sufficiently large empty region).
One approach is to compute the Voronoi diagram of the point set and choose
one of the Voronoi vertices, but this requires quite a bit of work
and no algorithm has presently been developed for Voronoi diagrams 
of points on $S^3$.\footnote{An algorithm for points on $S^2$ is available from 
  Renka(?).  Note that $S^3$ is a 3-manifold, complicating matters.}
However, empty regions are easy to find among small point sets.

This suggests a divide-and-conquer strategy in which the dataset is partitioned
into small sets, and a quaternion spline is built for each small set.
% assumes a maximal distance between consecutive samples
The two central issues in a partitioning strategy are
the implementation of partitioning, and the perturbation of each subset
away from the pole.
Two related issues are derivative control, allowing the subcurves to be spliced 
together smoothly (which will be covered in the next section);
and normalization, in which the method is made affine-invariant despite the
non-affine-invariance of each perturbation.
% Since this perturbation is not affine-invariant, we must also normalize
% the entire dataset at the beginning, to achieve affine-invariance for the
% entire process.

INTERESTING PROBLEM: EMPTY REGIONS ON S3 USING VORONOI DIAGRAMS.
EXPLORE 3D VORONOI DIAGRAMS FOR HINTS.

The approach for each subcurve is:
\begin{itemize}
\item Find a pointset that spans no more than x degrees.
\item Compute best-fitting hyperplane of this pointset, and use its normal as the 
      empty point.
\item Rotate pole to the empty point.
\end{itemize}

The quadratic maps off the sphere that we are considering are not invariant
under rotation.
Therefore, the overall approach will not be affine-invariant unless we normalize first.
This normalization is done as follows:

\begin{itemize}
\item START HERE
\end{itemize}

The weaponry of rotation is intriguing, since we are working in a higher dimension.

\section{Derivative control}

To control derivatives of the quaternion spline as well, 
we need to map these derivatives under stereographic projection,
then design a curve freely that interpolates them using classical techniques.

Consider a curve $C(t) = (x_1(t),x_2(t),x_3(t),x_4(t))$ on $S^3$
and its stereographic projection 
\[
SP(C(t)) = \frac{1}{1-x_4(t)} (x_1(t),x_2(t),x_3(t),0)
\]
The first derivative of this curve is, of course,
$C'(t) = (x'_1(t),x'_2(t),x'_3(t),x'_4(t))$,
which is interpreted geometrically as the curve's tangent.
The first derivative of $SP(C(t))$, the stereographic projection of the curve, is 
\begin{eqnarray*}
(SP(C(t))' & = & SP'(C(t)) \cdot C'(t) \\
           & = & (\frac{x'_1(1-x_4) + x_1x'_4}{(1-x_4)^2},\ 
                  \frac{x'_2(1-x_4) + x_2x'_4}{(1-x_4)^2},\  
                  \frac{x'_3(1-x_4) + x_3x'_4}{(1-x_4)^2}, 0) 
		  \cdot (x'_1, x'_2, x'_3, x'_4) \\
           & = & (\frac{x^{'2}_1(1-x_4) + x_1x'_1x'_4}{(1-x_4)^2},\ 
                  \frac{x^{'2}_2(1-x_4) + x_2x'_2x'_4}{(1-x_4)^2},\  
                  \frac{x^{'2}_3(1-x_4) + x_3x'_3x'_4}{(1-x_4)^2}, 0)
\end{eqnarray*}
%
This allows us to control the derivatives of the quaternion spline as well
as the points, as follows.
Suppose that we want to design a quaternion spline $C(t)$ through the points
$\{p_i\}_{i=0}^n$ with opening tangent $t$.\footnote{If as much tangent 
  data is provided as point data, as in Hermite data,
  then this point set is just two points.
  However, there often will be far fewer tangents than points, such as in the typical case
  when only point data is provided as constraints to the quaternion spline,
  but derivative data is introduced indirectly in the implementation of 
  a divide-and-conquer strategy, as discussed below.}
Assume without loss of generality that $C(0) = p_0$.
Under our approach,
designing $C(t)$ with $C'(0) = t$ is equivalent to 
designing $SP(C(t))$ with 
\[
(SP(C(t)))'(0) = (\frac{t^{2}_1(1-p_{04}) + p_{01}t_1t_4}{(1-p_{04})^2},\ 
                  \frac{t^{2}_2(1-p_{04}) + p_{02}t_2t_4}{(1-p_{04})^2},\  
                  \frac{t^{2}_3(1-p_{04}) + p_{03}t_3t_4}{(1-p_{04})^2}, 0)
\]
Interpolating to a closing tangent is done analogously.
Interpolating to other derivatives can be achieved simply by using higher derivatives
of $SP(C(t))$.
For example, 
\[
(SP(C(t)))'' = use maple
\]

\section{Software}

There are four spaces to visualize: object space, object view (move the mesh in a crowded
room, Ocam's razor suggests mesh is moving, not every object in room;
perhaps use teapot; how are positions set?  how are positions and orientations melded?);
object space, camera view (move the camera in a crowded room; same Ocam assumption; easier
to understand quality of orientation change from this viewpoint perhaps); 
quaternion space (S3);
and 4-space (for free image of quaternion spline).
quaternionSpline.cpp should handle each of these.

A parade route scenario.
Could watch same flythrough from two different perspectives:
1) flying along with camera
2) watching camera fly along from a static viewpoint above (like Harry Potter scene where
they step into the past and observe themselves)

Build an environment with the ability to set the camera viewpoint (from which to watch
the teapot flying by on the parade route);
set keyframe positions and orientations; adjust keyframe positions and orientations;
add keyframe positions and orientations; enter the motion path at a certain keyframe
(not always at beginning).

This environment could also provide a good testbed for motion planning, and visibility
analysis.

- {\bf first build a simple quaternion spline testbed like Graphics Interface},
  to perfect the quaternion spline code:
  two windows, one for quaternion space and other for object space;
  build-in the position curve and the object (teapot);
  visualize keyframes, denser frames sampled off quaternion spline, and moving animation.
  This is also a more exportable version of the software (for web release).
START HERE FOR SOFTWARE, BUT GET WORKING PUBLICATION FIRST.

{\bf the following is a good environment to build in 671}

- then build a flythrough testbed for orientation testing, especially incorporation
  into motion planning, orientation planning, and game design context?

- Build a path and a quaternion spline through the scene.  Don't tie it to the camera
  or to an object yet.
- path construction mode:
  To build the two splines: render a scene as context; build a GUI to move through this
  scene naturally (follow Maya lead, as well as PS2 games); 
  allow positions and orientations to be captured at any time and
  inserted as keyframes (by default, the next keyframe is added, but we should be able to
  *cycle through the keyframes* and choose a position on the timeline, even replacing
  a keyframe should be possible; use Maya as inspiration);
- flythrough mode:
  fly along the path with camera or fly along with object watching from some predetermined
  viewpoint (editable by simply a keystroke from present keyframe in path construction 
  mode); allow toggling between these modes; allow backwards and forwards motion.
- scene input: input file should be a data format (ug,off,wrl,obj,...) and a data file;
  program then inputs this file in the appropriate way.

- motion planning context: input far fewer keyframes and build motion using
  a motion planning algorithm rather than interpolation (requires better extraction
  of scene model than triangles);
  orientation planning context: simply build quaternion spline differently

\clearpage

\section{Abstract}
The efficient and effortless design of smooth, natural motions is important
in animation and general motion planning.
We address the design of interpolating rational quaternion splines and 
obstacle-avoiding rational quaternion splines.

A point on Sn encodes a sum of squares; on S3 and in projective space,
a point encodes a square that is the sum of 4 squares.

We may be able to get away with one spline, using the argument that an empty point
must exist for all but the most egregiously large datasets, at least if an elegant
empty point generator is used.  The main challenge then is closed curves, where
the end derivatives must somehow be controlled.  And the new C++ code must be debugged.

% We must find the dependency between the equations defining the present
% tangent restriction.

Building Hermite quaternion splines, or
robust and continuous quaternion splines.

Strategy:
\begin{itemize}
\item Rotate quaternion dataset into normal form (Section~\ref{sec:normalform}).
\item Split points into subsets (for manageable pole avoidance).
\item For each subset
\begin{itemize}
\item Find a rotation away from pole, and rotate.
\item Map points to 4-space.\footnote{Using your favourite map (Euler or stereographic projection are the obvious choices).}
\item Map data derivatives to 4-space.
\item Interpolate points and derivatives.
\item Map curve (defined by control points) back to S3.
\end{itemize}
\end{itemize}
Alternatively:
\begin{itemize}
\item Normalize and split points into subsets.
\item For each subset
\begin{itemize}
\item Find a rotation and rotate.
\item Map (rotated) points to 4-space.
\item Map (rotated) end-derivatives of previous curve.
      (Note that, for the last segment of a closed curve, there are end-derivatives
       to match on the previous and the next curve.)
\item Interpolate points and end-derivatives.
\item Map curve.
\end{itemize}
\end{itemize}

\section{Dictionary}

\section{Fundamentals for orientation obstacles (see ~/walkthrough/programs preface}

\hspace{.2in} What is an orientation?

What is a unit quaternion, and its relation to Euler's rotation?

What is a point on $S^3$, and its relation to 4-squares theorem?

What is $S^3$ and its relation to $S^2$.

What is a bad orientation (for a teapot and a camera)?

How does a locus of orientations map to the quaternion sphere?

What is the locus of orientations rotating around an axis?

Simple example of mapping a constraint to $S^3$: stay within n degrees of vertical.

Simple example of a path on $S^3$ avoiding this obstacle.

General challenge: building obstacle-avoiding splines on a surface.

General challenge: constructing bitangents of curves on $S^3$.

[Can we express a ruled surface by a generatrix for position and a quaternion spline
for orientation?]

\clearpage

\section{Software issues}

Let $f$ be a map from Euclidean space $\Re^n$ to the sphere $S^3$,
and $f^{-1}$ its inverse map from the sphere to free space.
These maps may be used to transform the construction of a quaternion spline on $S^3$
to the construction of a spline in $Re^n$.
$f^{-1}$ tends to have a pole when $f$ is rational.
The quaternions are moved away from this pole, then mapped to free space,
interpolated in free space, and finally mapped back to the sphere and 
returned to their original position.
Derivative information may be mapped to free space too, if desired.
To streamline the motion away from the pole, the quaternions may be split
up into several subsets, each of which is interpolated separately.
To preserve continuity, derivative information is maintained across the boundary 
between two subcurves during this construction.
There are different choices for the spherical map $f$.
For each choice of $f$, an understanding must be developed 
of how curves (Bezier curves) are mapped by $f$.

{\bf Challenges.}
\begin{itemize}
\item  What are the maps to a sphere? from a sphere?
\item  {\bf Quaternion datasets.}  {\bf Models.} {\bf Keyframes.}
\item  {\bf Flythroughs.}
\item  {\bf Combining positions and orientations.}
\item 
How to put a dataset in normal form, for affine invariant application of 
affine {\em variant} maps.
\item
{\bf Why the inverse of rational maps to the sphere must have a pole.}
\item
A review of rational maps to the sphere. 
\item
How to perturb a pointset away from a pole.
\item
How to rotate in higher dimensions.
(Ironically, the definition of rotation of an object in 3-space,
using a quaternion spline, requires an understanding of rotation in higher dimensions,
especially in 4-space.)
\item
{\bf Implement and see if it can be applied to the construction of orientation obstacles.}
\item
How to build bitangents on S3, and then visibility graphs.
\item
How to build bisectors on S3. 
\item
How to to build cell decomposition of free space among obstacles on S3.
\end{itemize}

{\bf Student projects.}
Mapping a Bezier curve back to the sphere using inverse stereographic projection.

{\bf We need to capture motion data, for good quaternion datasets.}
This could come from Maya animations (keyframes in Learning Maya scripts):
can we extract this data?
It could also come from a real motion, using some gadget to record orientation.
We need to think of good free motions to record, like a tea tray or a camera.
We might be able to record camera motions in a movie to extract good motions,
both for data and for emulation.
In short, we need quaternion-recording gadgets in Maya, games, movies and real motion.
What could you attach to a tray to record its orientation?
Probably a robot arm is the closest, since it can spit out its position and orientation
intermittently.
After all, we want a natural motion that doesn't have to stop every second to laboriously
measure its orientation.  
At the least, we want to record the motion and then play it back for analysis.

The derivatives are provided indirectly from the previous curve.
If we map higher derivatives, a higher degree Bezier curve will be necessary:
there are only two degrees of freedom in a cubic Bezier curve.
This in turn will require a result about mapping this higher degree Bezier curve
back to the sphere.
How do Dietz et. al. handle the poles?

What curves should be used to interpolate in free space?
Consider $C^2$ continuity of 2 Bezier curves, as an example.
To match the 1st and 2nd derivatives at the opening of a curve, two dof are required.
An interpolating cubic curve interpolating a set of data points has two extra degrees of freedom.\footnote{One of degree $n$ has $n-1$ extra degrees of freedom (dof).}
These are typically used to set the tangent at both ends, or the curvature at both ends
(natural spline), but they can be used instead to set both derivatives at one end.
So a cubic Bezier spline has enough dof to preserve $C^2$ continuity 
as each new curve is added, piece by piece.
However, there are no extra degrees of freedom to wrap around to form a closed curve,
since then the last piece must match 2 derivatives at both ends,
which requires 4 dof.
However, a cubic Bezier spline would suffice for the construction
of a $C^1$ continuous closed curve.
In short, a quadratic curve is enough for an open $C^1$ continuous spline in free space
built out of several pieces, or a closed $C^1$ continuous spline in free space
built out of one piece, but a cubic curve is required for a closed $C^1$ continuous spline
in free space built out of several pieces.
Similarly, a cubic curve is enough for a an open $C^2$ continuous spline in free space
built out of several pieces, or a closed $C^2$ continuous spline in free space
built out of one piece, but a quintic curve is required for a closed $C^2$ continuous 
spline in free space built out of several pieces.
All of these degrees are doubled when the curve is mapped to the sphere to become
a quaternion spline.
For example, a closed $C^1$ continuous quaternion spline made out of several pieces
is of degree 6, while $C^2$ continuity requires a degree 10 quaternion spline.
This shows how the quaternion spline's degree increases with desired continuity.
There is no limit to the possible continuity, at the expense of degree.
The use of several pieces, with its associated improvement of robustness, comes
at the expense of degree.
For example, the quaternion spline can be of degree 6 (rather than 10) if it
is made out of one piece.

What is the optimal rotation?

\clearpage

{\bf Exposure sheet}

Maps to the sphere.
\[
	M(x_1,x_2,x_3,x_4) =
	\frac{1}{x_1^2 + x_2^2 + x_3^2 + x_4^2}
	(x_1^2 + x_2^2 + x_3^2 - x_4^2, 2x_1x_4, 2x_2x_4, 2x_3x_4)
\]
\[
SP^{-1}(x_1,x_2,x_3) = \frac{1}{x_1^2 + x_2^2 + x_3^2 + 1} 
                       (2x_1, 2x_2, 2x_3, x_1^2+x_2^2+x_3^2 - 1)
\]

Maps away from the sphere.
\[
M^{-1}|_{S3}(x_1,x_2,x_3,x_4) = \frac{1}{\sqrt{2(1-x_1)}}(x_2,x_3,x_4,1-x_1)
\]
\[
SP(x_1,x_2,x_3,x_4) = \frac{1}{1-x_4} (x_1,x_2,x_3)
\]

\clearpage

\tableofcontents

\clearpage

\section{Spherical introduction}

In the classic 'Flatland',
Edwin Abbott Abbott (sic!) related the ominous appearance of The Sphere in the
cosy confines of two-space.
This was Abbott's way to introduce the concept of higher dimensions.
We will discuss the sphere one dimension higher, the 3-sphere in 4-space.
Cartographers have long wrestled with the problem of mapping the spherical earth
to a flat page.
The sphere has long been a Platonic ideal for shape.
Indeed, all of the Platonic solids\footnote{The Platonic solids are the tetrahedron,
  cube, octahedron, dodecahedron and icosahedron.}
aspire to the sphere, with differing levels of success.\footnote{Geodesation can
  help them to achieve this aspiration.}
We are interested in maps to and from the sphere, 
for reasons both mundane (their application to the design of quaternion splines)
and aesthetic.
One of the most burning questions is: why must they always have poles?

A surface's parameterization may be interpreted as a map to the surface,
since it maps some parameter space to the surface.
For example, the parameterization of a 2-manifold S is a map from $\Re^2$ to S.
The inverse of this parameterization yields a map from the surface.

\subsection{Sphere}

In the case of the sphere, the classical parameterization is
\[
   () x () = 
\]
which can be viewed as a spherical product of a circle and half-circle \cite{barr}.
Unfortunately, this map is not rational.
A rational parameterization is 
\[
   () x () = 
\]
Both of these parameterizations have poles, since the circle shrinks to a point
at the endpoint of the half-circle.
This is reflected in the earth's lines of latitude: the 90th line of latitude at the
North (or South) Pole is degenerate.
Other choices of map to the sphere, such as the Euler map or 
stereographic projection, suffer from poles, as one maps away from the sphere.
This suggests that poles may be an inherent property of maps from the sphere,
certainly of rational maps.
But are they?

\section{Introduction}

In animating a game or film, the motion of many objects and cameras must be controlled.
Our goal is to give the animator the capacity to effortlessly design 
a natural motion that respects 
certain constraints inherent to the moving object or camera.
   % must have its own source of control or someone manipulating it
Consider some examples.
In the motion of a teapot full of Earl Grey from kitchen to table,
tea should not be spilled.
In the motion of an airplane,
banking should be smooth and the cockpit should not turn upside down.
In the flight of a virtual camera through an architectural model or the scene of a game,
the camera should avoid excessive, jerky movements and 
should respect the vertigo of the observer, not turning upside down
or rocking too far from horizontal.
% or a SkyCam flying above a Monday Night football game:
When the camera is to be physically held, the motion
should also respect the cameraman's, or camera dolly's, degrees of freedom.
In the animation of a baseball throw by an articulated character in a computer game,
the elbow and shoulder joints should respect anatomical constraints.
  % surveillance robot;
  % not an approximation of the path of a tumbling thrown shoe:
  % thrown object is not appropriate, since it follows physical laws & isn't free-flying.

The above examples present three challenges for motion design: % free-flying motion
a smooth change of position, a smooth change of orientation, 
and the avoidance of certain illegal orientations intrinsic to the moving object.
Note that we are not considering extrinsic obstacles in the design 
of the motion, either positional or orientational.
For example, we are not considering the design of a motion for the teapot that avoids
the table or a bounding dog, or that respects the orientational constraints 
imposed by placing the pot on the middle shelf of a china cabinet.
However, we do want to respect the orientational constraints of the waiter's hand 
carrying the teapot, since it is considered 'part of the teapot'.
In short, we are presently concentrating on the motion of the object through free space,
free of obstacles.
This decision to emphasize motion in free space can be 
rationalized by the presence of the animator in the design loop:
the animator will guarantee that extrinsic obstacles are avoided.
Although a fully automatic solution would be wonderful,
it is understood that the animator will be tuning the motion:
we offer a tool that simplifies this task by eliminating many faulty motions.

The first of the three challenges, smooth control of position, is well understood:
it is well solved by the classical solution of interpolating polynomial B-splines 
in 3-space.
The second challenge, smooth control of orientation, is reasonably well understood 
but has some need for refinement: {\bf a solution that generates rational quaternion splines
with controllable endpoint derivatives (rational Hermite quaternion splines) 
is still open.}
A new solution, building upon our earlier work on rational quaternion splines, 
is proposed in this paper.
The earlier method generated smooth quaternion splines, 
but did not have fully controllable derivatives.
The third challenge, the avoidance of illegal orientations, is not well understood at all.
Fortunately, it can appeal to related and well-studied problems in motion planning 
for inspiration.
The problem reduces to the design of obstacle-avoiding quaternion splines on the
quaternion sphere $S^3$.

Constraints on position have long been studied in the motion planning literature.

Constraints on orientation could be modelled physically (e.g., a CFD simulation to
identify when tea pours out of the pot when it is moved violently), 
but this adds extra expense and even then does not itself design a motion.

An application is the automated construction of animation curves
in modelers like Maya.

The motion design should be efficient and elegant.

\clearpage

\section{Building a better quaternion spline}

Orientation is controlled by quaternion splines.
Rational orientation control demands rational quaternion splines, 
and smooth orientation control demands high derivative continuity of these splines.
We would like to provide the user with the power to explicitly define derivatives
at data orientations, explicitly controlling torque and other spin parameters.
This control over endpoint derivatives of a quaternion spline is also necessary
if the quaternion spline is built up in parts, which is useful for large data sets.
It fits well into the divide-and-conquer design philosophy suggested by 
one of the challenges of rational quaternion spline design: avoiding poles of the
underlying rational map to and from the sphere.
The best way to avoid a pole is to move away from it, and it is simpler to find a way
to move a small set of quaternions far away from the pole.
Therefore, we want to build a rational quaternion spline with full ability to
interpolate not only points but arbitrary derivatives.

The old method: use Euler map, perturb entire set away from pole at once 
                using random or unguaranteed best-fitting plane perturbation? 
		(SEE CODE TO VERIFY THIS).\\
The new method: use stereographic projection, decompose into subsets, perturb each subset
                away from the pole individually using best-fitting plane heuristic 
		(see options in original paper),
		control derivatives

\section{The problem}

\begin{defn2}
A {\bf quaternion spline} is a curve on $S^3$ that interpolates a set of quaternions.
\end{defn2}

\noindent In order of increasing difficulty,
the paper addresses the following variants of the quaternion spline problem.

\begin{itemize}
\item 
Given a set of quaternions $\{P_i\}_{i=0}^n \subset S^3$,
build an open or closed $C^1$-continuous quaternion spline through this set.
(By interpolating all of the points at once, we should get $C^2$ continuity,
that is maximal continuity [or is $C^5$ continuity maximal?].)

\item 
Given a set of quaternions $\{P_i\}_{i=0}^n \subset S^3$
and two end tangents $T_0$ and $T_1$
such that the tangent $T_0$ lies in the tangent plane of $S^3$ at $P_0$
and $T_1$ lies in the tangent plane of $S^3$ at $P_n$,
build an open or closed $C^1$-continuous quaternion spline $C(t)$, $t \in [0,1]$ 
through this set such that $C'(0) = T_0$ and $C'(1) = T_1$.

\item 
Given a set of quaternions $\{P_i\}_{i=0}^n \subset S^3$
and a set of tangents $\{T_i\}_{i=0}^n$
such that the tangent $T_i$ lies in the tangent plane of $S^3$ at $P_i$,
build an open or closed $C^1$-continuous quaternion spline through this set
that honours the set of tangents (i.e., if $C(t_i) = P_i$, then $C'(t_i) = T_i$).
\end{itemize}
%
and the most general statement of the problem, which encompasses the above problems:
%
\begin{itemize}
\item
Given 
\begin{enumerate}
\item a set of quaternions $\{P_i\}_{i=0}^n \subset S^3$, and
\item a set of tangents $\{T_i\}_{i=0}^m$ ($m\leq n$) 
      at some or all of these quaternions, such that
\begin{enumerate}
\item
$T_i$ is a tangent at $P_{f(i)}$, 
where $f: \{1,\ldots,m\} \rightarrow \{1,\ldots,n\}$ is a one-to-one map, and
\item
the tangent $T_i$ at $P_{f(i)}$ lies in the tangent plane of $S^3$ at $P_f(i)$
\end{enumerate}
\end{enumerate}
build an open or closed $C^1$-continuous quaternion spline $C(t)$, $t \in [0,1]$ 
through this set that honours the set of tangents (i.e., the first derivative of $C(t)$
at $P_{f(i)}$ is $T_i$).
\end{itemize}
%
All of these problems can be reduced to the following subproblem.
%
\begin{itemize}
\item Given a set of quaternions $\{P_i\}_{i=0}^n \subset S^3$ 
of bounded size (where the bounds are defined below) 
and two end tangents $T_0$ and $T_1$
such that the tangent $T_0$ lies in the tangent plane of $S^3$ at $P_0$
and $T_1$ lies in the tangent plane of $S^3$ at $P_n$,
build an open $C^1$-continuous quaternion spline $C(t)$, $t \in [0,1]$ 
through this set such that $C'(0) = T_0$ and $C'(1) = T_1$.
\end{itemize}
{\bf Explain how each problem reduces to the subproblem.}

\clearpage

\section{Related literature}

Shoemake (SIGGRAPH, arcball); Barr (SIGGRAPH and followup);
Kim (exponential maps); Ravani (Lie algebra);
Pottmann and Wagner (dual quaternion splines);
Dietz and Hoschek (curves on S2, do they map from S2 to R3 or S2 to R2?);
animation literature;  Maya's animation.

with Xiao: smooth visibility graphs

with someone: Voronoi diagrams on $S^3$

\clearpage

\section{The approach}

\noindent 
The approach taken in this paper is to replace curve design [point interpolation] 
on a surface by curve design [point interpolation] in free (Euclidean) space.
[The paper employs the following algorithmic technique to solve this subproblem,
motivated by the fact that quaternion spline design is difficult because it is
curve design constrained to remain on a surface.]
Curve design in Euclidean space has an excellent classical solution.
The reduction of constrained curve design to free curve design
is realized by a map between $S^3$ and $R^n$.
%
\begin{itemize}
\item Let $f$ be a rational map of $\Re^n$ to $S^3$.
\item Map the quaternions Q (and derivatives T at Q) by $f^{-1}$, 
      to $f^{-1}(Q)$ and $f^{-1}(T)$.
\item Build a cubic interpolating spline $C(t)$ 
      through $f^{-1}(Q)$ and $f^{-1}(T)$.
      This is the traditional interpolation algorithm in Euclidean $n$-space.
      [The classical solution has two extra degrees of freedom, 
      which are used to set the tangents.]
\item Map the spline under $f$, yielding a rational spline $f(C(t))$.
\end{itemize}

Since this approach relies on the simplicity of curve design in Euclidean space,
the quaternions must be mapped to points in some Euclidean space.
However, any Euclidean space will do: 3-space and 4-space are the two natural candidates.
% Consider the rational map of $Re^n$ to $S^3$.
It is possible to find one-to-one maps of a 3-manifold to a 3-manifold.
However, any map of a 4-manifold to a 3-manifold will be many-to-one,
simply because of the change in dimension.
Thus, the preimage of a typical point on the 3-manifold will be a curve,
not a point.

Applying this observation to our problem,
if a map between $S^3$ and 4-space is used, there will be ambiguity
in mapping the quaternions to 4-space, because the mapping is many-to-one.
We will see that this indirectly manifests itself in difficulties with 
derivative interpolation in the Hermite version of the problem.
A map between $S^3$ and 3-space has no such difficulties.

%
Because of inherent(?) poles of rational maps to the sphere, two additional steps
are needed:
\begin{enumerate}
\item Let $f$ be a rational map of $\Re^4$ to $S^3$.
\item Rotate the quaternions Q and end tangents T away from the pole.
\item Map the rotated $\hat{Q}$ and $\hat{T}$ to $f^{-1}(\hat{Q})$ and $f^{-1}(\hat{T})$.
\item Build a cubic interpolating Bezier curve $C(t)$ 
      through $f^{-1}(\hat{Q})$ and $f^{-1}(\hat{T})$.
\item Map $C(t)$ under $f$, yielding a rational Bezier curve $f(C(t))$.
\item Rotate $C(t)$ back, undoing the original rotation of $Q$ and $T$.
\end{enumerate}


Our previous work solved only the open quaternion spline problem,
did not allow a divide-and-conquer approach,
and had unresolved difficulties with poles of the map.
None of the approaches for coping with the pole were satisfactory,
not being implemented (emptiest point), not always working (BFP),
or being unrepeatable (random point).
The implementation of the earlier ideas has also been streamlined
with more efficient formulae.

\clearpage

A desirable property of any interpolating curve is the ability to connect
two segments smoothly.  
Equivalently, we wish to introduce the power to interpolate derivatives at endpoints.
By interpolating tangent directions at one endpoint, open $G^1$-continuous curves
can be built up segment by segment.
By interpolating tangent directions at both endpoints, closed $G^1$-continuous curves
can be built up segment by segment.
By interpolating tangents at one (resp., both) endpoints, 
open (resp., closed) $C^1$-continuous curves can be built.
By interpolating up to $i$th derivatives at one (resp., both) endpoints,
open (resp., closed) $C^i$-continuous curves can be built.
We begin by showing how to build $C^1$-continuous curves.
This is necessary to build a quaternion spline up by parts, which is in turn necessary
to guarantee the success of certain perturbation strategies.

Suppose that we have a desired beginning tangent $T_0$ 
at the beginning endpoint of an $S^3$ curve. 
    % but are willing to relax this constraint to a beginning tangent 
    % along the direction defined by $T_0$.
How is a quaternion spline designed with this opening tangent?
Let the space curve be defined by the control points $\{b_i\}$,
the $S^3$ curve by the control points $\{B_i\}$ and the weights $\{w_i\}$,
and the quaternion spline by the input quaternions $\{P_i\}$.
   % First recall some basic facts about end tangents.
   % The tangent at the beginning endpoint of a Bezier curve (e.g., the space curve) 
   % is defined by the first two control points and the degree.
   % The tangent at the beginning endpoint of a rational Bezier curve (e.g., the $S^3$ curve)
   % is defined by the first two control points and weights, and the degree:
   % the beginning tangent on the $S^3$ curve is $6(w_1/w_0)(B_1 - B_0)$.
In designing an interpolating cubic B-spline curve, 
there are two extra degrees of freedom.
These are often equated with the positions of the second and penultimate control points,
$b_1$ and $b_{n-1}$ (or equivalently, the end tangents).
This power to set $b_1$ will be used to interpolate the tangent direction.

     % Back to our specific problem.
We have developed formulae for translating the control points $\{b_i\}$ of the space
curve to the control points $\{\breve{b}_i\}$ and weights $\{w_i\}$ of the $S^3$ curve.
Notice that $\breve{b}_i$ and $w_i$ are defined in terms of $\{b_j\}_{j=1}^i$.
In particular, $\breve{b}_1$ and $w_1$ are defined in terms of $b_0$ and $b_1$.
Since $\breve{b}_1$ defines the opening tangent on one curve, 
and $b_1$ the opening tangent on the other,
this formula relates the tangents on both curves and allows the constraint on the
opening tangent of the $S^3$ curve to be translated to a constraint on the opening
tangent of the space curve.
This is where we want the constraint,
so that the space curve can be designed with the right tangent.

NOTE: we don't want zero weights, so don't want $b_0$ = origin (so $b_0 \cdot b_0 = 0$)
and don't want $b_0 \cdot b_1 = 0$ (i.e., b0 and b1 collinear).
These are coordinate dependent!!
(By multiplying by them, they may disappear in a legal way.)

Add a section elaborating on Theorem~\ref{thm:imagecurve} below 
(once the details of how we map and perturb are clear):
we shall guarantee that these weights are not zero.
In particular, we shall place the control points $b_i$ of the space curve
near $S^3$ ($b_{3i}$ shall lie on $S^3$) so that $b_i \cdot b_i$ is nonzero 
and close to 1.
Other constraints on the relationship between the angles of $b_i$ and $b_j$ can be 
deduced or forced to preserve the positivity of these weights.

\clearpage

\section{Preamble to 'Rational maps to the sphere' for Orientation control paper}

A key component of the proposed solution is a rational map to the sphere
and its companion map off of the sphere.
The latter map, from $S^3$ to some Euclidean $n$-space, frees the quaternions from
their surface constraint, while the former map reimposes this constraint.
Since the map from the sphere is applied to a finite pointset, while the map
to the sphere is applied to a polynomial curve, rationality is only necessary for the
map to the sphere, to preserve rationality of the curve, yielding a rational
quaternion spline.
If the user is willing to settle for nonrational curves,
a nonrational map could be used,
but rational Bezier curves offer many advantages, including simplicity,
efficiency, and pleasant properties such as the variation-diminishing and convex-hull
properties.

A map between $S^3$ and $\Re^3$ is more appropriate than a map between $S^3$ and $\Re^4$
for the inverse map.
The inverse will be unique, while in the latter case the map from $S^3$ to $\Re^4$ is
inherently one-to-many, since we are moving from a 3-manifold to a 4-manifold.

\section{Rational maps to the sphere}
% $S^3$ maps, to and fro

[Note: The only published paper is GI and Eurographics.
We have moved beyond these papers by at least the 30\% guideline.
This material is valid for a journal publication, but regardless we want a resolution
of this issue in a technical report.
Additional material: stereographic projection, image of rational Bezier,
generalized construction, visualization.]

In this paper, we consider rational maps to the 3-sphere and their inverses.
Since the inverse maps often (always?) have poles, we explore this issue as well.
Derivatives are computed when the maps are applied to a parametric curve on 
the 3-sphere.
[It is hard to motivate the interest in this derivative result, so this should
be delayed to another paper.]
The image of a Bezier curve under these maps, expressed as a rational Bezier curve,
is computed, which allows any Bezier curve to be projected onto the sphere.
(Can we compute the image of a rational Bezier curve too? This would be additional.)
We discuss the generalization to construction of maps to other surfaces.

\clearpage

\subsection{Previous work}

Mathematicians have long been interested in maps from the sphere to Euclidean space,
because of problems in cartography,
and there are many well-known examples \cite{kreyszig59}.
However, since properties such as conformality (preservation of angles between 
intersecting curves) and equiareality (preservation of area) were more important than
rationality (the ability to express the map in terms of quotients of polynomials)
in the design of these cartographic maps, most of these maps
are not rational (e.g., the Mercator projection, Lambert projection,
mapping of Sanson, and mapping of Bonne \cite{kreyszig59}).
A rational map to the sphere is important in the design of curves on the
sphere, since it generates rational curves.

\clearpage

\subsection{Stereographic projection}
\label{sec:stereo}

The most well-known map involving the sphere is stereographic projection.
It has been used at least since Hipparch in 160 B.C. \cite{kreyszig59}. % p. 205
% probably discovered by Hipparch too
Stereographic projection maps from \Sn{n}\ to the hyperplane $x_{n+1}=0$ in $(n+1)$-space.
A point of \Sn{n}\ is perspectively projected from 
the north pole of \Sn{n}\ to $x_{n+1}=0$  % \cite{thorpe79}. % or kreyszig59
(Figure~\ref{fig:stereo}).
% Note that there is a variant of stereographic projection where
% the projection plane $z=0$ is replaced by the tangent plane opposite the pole
% $z=-1$ \cite{kreyszig59}.
The image of the north pole $(0,\ldots,0,1)$ is undefined.\footnote{It is well defined
  in projective space, where the pole maps to the line at infinity of the hyperplane.}
Stereographic projection is a map away from the sphere, 
but its inverse is a rational map to the sphere.

\begin{figure}[ht]
\begin{center}
\includegraphics[scale=.25]{img/fig:stereo.jpg}
\end{center}
\caption{Stereographic projection in 3-space}
\label{fig:stereo}
\end{figure}

\begin{lemma}
Stereographic projection 
$f_{SP}: \Sn{n} - (0,\ldots,0,1) \rightarrow x_{n+1}=0 \subset \Re^{n+1}$ 
from the sphere
and its inverse $f_{SP}^{-1}: x_{n+1}=0 \rightarrow \Sn{n} - (0,\ldots,0,1)$ 
to the sphere
are defined by:
\begin{eqnarray*}
f_{SP}(x_1,\ldots,x_{n+1}) & = & \frac{1}{1-x_{n+1}} (x_1,\ldots,x_n,0) \\
f_{SP}^{-1}(x_1,\ldots,x_n,0) & = &
	\frac{1}{x_1^2 + \cdots + x_n^2 + 1} 
	(2x_1, \ldots, 2x_n, x_1^2 + \cdots + x_n^2 - 1)
\end{eqnarray*}	% see thorpe79, p. 125
\end{lemma}
\vspace{.1in}
\prf
The projector line $(1-t)p + tq$ through $p = (0,\ldots,0,1)$ and 
$q = (x_1,\ldots,x_{n+1})$ intersects $x_{n+1}=0$ 
when $1-t + tx_{n+1} = 0$ or $t = \frac{1}{1-x_{n+1}}$.
This defines the forward map.
Since stereographic projection is one-to-one and onto,
it has an inverse that is also one-to-one.
The projector line $(1-t)p + t(r,0)$ through $p = (0,\ldots,0,1)$ and $(r,0)$
intersects \Sn{n} when $\|(1-t)(0,\ldots,0,1) + t(r,0)\|_2^2 = 1$,
or $(1-t)^2 + t^2\|r\|_2^2 = 1$, or $t^2(\|r\|^2 + 1) = 2t$, or 
$t=0$ and $t=\frac{2}{\|r\|^2 + 1}$.
The second root yields the inverse point.
\QED

Some other properties of the stereographic projection are interesting, although
more relevant to cartography than to our study.
It is a conformal map and, like any conformal map of \Sn{2}\ to the plane,
it maps loxodromes (curves of constant direction on the sphere) to straight lines.
It is the only conformal map from $S^2$ to a plane that preserves
circles (circles are mapped to circles or straight lines).

\clearpage

\begin{lemma}
Let $C(t) = (x_1(t),x_2(t),x_3(t),x_4(t))$ be a parametric curve that lies on $S^3$.
The first derivative of the stereographic projection of the curve $C(t)$ is 
\[
(\frac{x^{'2}_1(1-x_4) + x_1x'_1x'_4}{(1-x_4)^2},\ 
                \frac{x^{'2}_2(1-x_4) + x_2x'_2x'_4}{(1-x_4)^2},\  
                  \frac{x^{'2}_3(1-x_4) + x_3x'_3x'_4}{(1-x_4)^2}, 0)
\]
\end{lemma}
\prf
\begin{eqnarray*}
(f_{SP}(C(t))' & = & f_{SP}'(C(t)) \cdot C'(t) \\
           & = & (\frac{x'_1(1-x_4) + x_1x'_4}{(1-x_4)^2},\ 
                  \frac{x'_2(1-x_4) + x_2x'_4}{(1-x_4)^2},\  
                  \frac{x'_3(1-x_4) + x_3x'_4}{(1-x_4)^2}, 0) 
		  \cdot (x'_1, x'_2, x'_3, x'_4) \\
           & = & (\frac{x^{'2}_1(1-x_4) + x_1x'_1x'_4}{(1-x_4)^2},\ 
                  \frac{x^{'2}_2(1-x_4) + x_2x'_2x'_4}{(1-x_4)^2},\  
                  \frac{x^{'2}_3(1-x_4) + x_3x'_3x'_4}{(1-x_4)^2}, 0)
\end{eqnarray*}
\QED

Higher derivatives are found simply by repeating this process.

Since $x_{n+1}=0$ is an embedding of $n$-space in $\Re^{n+1}$,
stereographic projection and its inverse may be interpreted as maps between 
$\Sn{n}$ and $\Re^n$.

DEFINE BEZIER CURVE, BERNSTEIN POLYNOMIALS, PROJECTIVE SPACE.
REVIEW RATIONAL BEZIER CURVE AND BEZIER CURVE IN PROJECTIVE SPACE.

HOW ABOUT ARBITRARY BEZIER CURVE?

\begin{theorem}
\label{thm:imagecurve}
The image under inverse stereographic projection 
of a cubic Bezier curve with control points $b_i = (b_{i1},\ldots,b_{i4})$
is a sextic rational Bezier curve 
with control points $\breve{b}_i = $ 
and weights $w_i = $.
This maps an arbitrary cubic curve to a curve on \Sn{3}.
\end{theorem}
\prf
Let $c(t) = \sum_{i=0}^3 b_i B_i^3(t)$ be the cubic Bezier curve.
Since the image of $c(t)$ is a rational Bezier curve,
it is easier to work in projective space.
% When $c(t)$ is expressed in projective space, it becomes
% \[
% \sum_{i=0}^3 (b_{i1},b_{i2},b_{i3},b_{i4}, 1) B_i^3(t)
% \]
Inverse perspective projection becomes
\[
f_{SP}(x_1,x_2,x_3,x_4,1) = ---
\]
\QED

% image under stereographic projection
\Comment{
\prf
Let $c(t) = \sum_{i=0}^3 b_i B_i^3(t)$ be the cubic Bezier curve.
Since the image of $c(t)$ is a rational Bezier curve,
it is easier to work in projective space.
% When $c(t)$ is expressed in projective space, it becomes
% \[
% \sum_{i=0}^3 (b_{i1},b_{i2},b_{i3},b_{i4}, 1) B_i^3(t)
% \]
Perspective projection becomes
\[
f_{SP}(x_1,x_2,x_3,x_4,1) = (x_1,x_2,x_3,0,1-x_4)
\]
% Let the image curve in projective space be $f_{SP}(c(t)) = (f_1(t),\ldots,f_5(t))$.
Note that 
\[
1 - \sum_{i=0}^3 b_{i4} B_i^3(t) = \sum_{i=0}^3 B_i^3(t) - \sum_{i=0}^3 b_{i4} B_i^3(t)
= \sum_{i=0}^3 (1-b_{i4})B_i^3(t)
\]
using the partition of unity property of Bernstein polynomials.
Therefore, 
\[
f_{SP}(c(t)) = \sum_{i=0}^3 (b_{i1}, b_{i2}, b_{i3}, 0, 1-b_{i4}) B_i^3(t)
\]
This is a cubic rational Bezier curve with weights $w_i = 1-b_{i4}$
and control points $\frac{1}{w_i} (b_{i1}, b_{i2}, b_{i3}, 0)$.
\QED
}

[Be careful using stereographic projection, since the conclusions of comp.tex are that
stereographic projection leads to an inferior quaternion spline.
Doesn't this imply that we have implemented the map of a Bezier curve under
stereographic projection too?]

CREATE BIB FILE.

\clearpage

\subsection{Development of Euler's map}

Develop from first principles using 4 squares theorem.
See ratqspline.tex and characterization.tex(?).
Show that all maps to sphere are versions of 4 squares (but then what about stereographic
projection?).

\clearpage

\subsection{Euler's map}

\begin{defn2}
The Euler map $M: \Re^4 - \{0\} \rightarrow \Sn{3}$ to the sphere is defined by:
\begin{equation}
\label{eqM}
	M(x_1,x_2,x_3,x_4) =
	\frac{1}{x_1^2 + x_2^2 + x_3^2 + x_4^2}
	(x_1^2 + x_2^2 + x_3^2 - x_4^2, 2x_1x_4, 2x_2x_4, 2x_3x_4)
\end{equation}
%
If we let 
${\cal M}(p,q): = (p_1q_1 + p_2q_2 + p_3q_3 - p_4q_4, 2p_1q_4, 2p_2q_4, 2p_3q_4)$,
the Euler map may be re-expressed as:
\begin{equation}
\label{eq:calM}
M(P) = \frac{{\cal M}(P,P)}{P \cdot P}
\end{equation}
which is valuable in Theorem~\ref{thm:imagecurve} on the image of a Bezier curve
under the Euler map.
\end{defn2}

The Euler map has both bilinear and quadratic characteristics.
In Euclidean space, every coordinate of ${\cal M}$ is a bilinear map,
while in projective space, every coordinate of $M$ is a quadratic form.

Since the Euler map transforms a 4-space into a 3-manifold, losing a dimension,
the inverse of a point is a 1-manifold.

\begin{lemma}
\label{lem:inverse}
The inverse Euler map $M^{-1}: \Sn{3} - (1,0,0,0) \rightarrow \Re^4$ from the sphere
is defined by:
\begin{equation}
M^{-1}(x_1,x_2,x_3,x_4) = (1-t)(0,0,0,0) + t(x_2,x_3,x_4,1-x_1), t \neq 0
\end{equation}
In other words, the image of a point is a line through the origin.
The map has a special behaviour at the pole $(1,0,0,0)$:
\begin{equation}
M^{-1}(1,0,0,0) = \mbox{ the hyperplane $x_4=0$ without the origin}
\end{equation}
\end{lemma}
\prf
\QED

It is often desirable for the inverse of a point to be a unique point.
In this case, it is possible to choose a particular point on the inverse line.
A natural choice is to choose the point on \Sn{3}, since every inverse line must intersect
\Sn{3} and the choice of \Sn{3} adds a certain symmetry.
This leads to the following restricted inverse map.

\begin{lemma}
The restricted inverse Euler map $M^{-1}_{S3}: S^3 - \{(1,0,0,0)\} \rightarrow S^3$,
defined by $M^{-1}_{S3}(P) = M^{-1}(P) \cap S^3$, has the form
\begin{equation}
(x_1,x_2,x_3,x_4) \mapsto \frac{1}{\sqrt{2-2x_1}}(x_2,x_3,x_4,1-x_1)
\end{equation}
\end{lemma}
\prf
An intersection of the inverse line with $S^3$
is the point $\frac{(x_2,x_3,x_4,1-x_1)}{\|x_2,x_3,x_4,1-x_1\|_2}$.
\[
\|(x_2,x_3,x_4,1-x_1)\|_2^2 = x_2^2 + x_3^2 + x_4^2 + (1-x_1)^2
= x_1^2 + x_2^2 + x_3^2 + x_4^2 + 1 - 2x_1
\]
Since $(x_1,x_2,x_3,x_4) \in S^3$,
\[
\|(x_2,x_3,x_4,1-x_1)\|_2^2 = 2 - 2x_1
\]
\QED

% It is a map from $S^3$ to $S^3$.
% This is not a rational map, but it need not be.

\begin{lemma}
Let $C(t) = (x_1(t),x_2(t),x_3(t),x_4(t))$ be a parametric curve that lies on \Sn{3}.
The restricted inverse Euler map of $C(t)$ is another curve on \Sn{3}.
The first derivative of this curve $M^{-1}|_{S3}(C(t))$ is:
\[
(\frac{x'_2}{(2-2x_1)^{1/2}} + \frac{x'_1x_2}{(2-2x_1)^{3/2}}, 
 \frac{x'_3}{(2-2x_1)^{1/2}} + \frac{x'_1x_3}{(2-2x_1)^{3/2}},
 \frac{x'_4}{(2-2x_1)^{1/2}} + \frac{x'_1x_4}{(2-2x_1)^{3/2}},
 \frac{-x'_1}{2\sqrt{2-2x_1}})
\]
\end{lemma}
\prf
\begin{eqnarray*}
(M^{-1}(C(t))' & = & (M^{-1})'(C(t)) \cdot C'(t)
\end{eqnarray*}
Consider the first coordinate of $(M^{-1})'(C(t))$.
The second and third coordinates are computed analogously.
Recall that the restricted inverse of the Euler map is
\[
M^{-1}|_{S3}(C(t)) = \frac{(x_2,x_3,x_4,1-x_1)}{\sqrt{2-2x_1}}
\]
If we let $g(t) = (2-2x_1(t))^{\frac{1}{2}}$, then
\[
g'(t) = \frac{-x'_1}{\sqrt{2-2x_1}}
\]
and it is clear that the first coordinate of $(M^{-1})'(C(t))$ is:
\[
\frac{x'_2 \sqrt{2-2x_1} + \frac{x'_1 x_2}{\sqrt{2-2x_1}}}{2-2x_1}
= \frac{x'_2}{(2-2x_1)^{1/2}} + \frac{x'_1x_2}{(2-2x_1)^{3/2}}
% = \frac{x'_2(2-2x_1) + x'_1x_2}{(2-2x_1)^{3/2}}
% = \frac{x'_1x_2 - 2x_1x'_2 + 2x'_2}{(2-2x_1)^{3/2}}
\]
The fourth coordinate of $M^{-1}(C(t))$ is 
$\frac{1-x_1}{\sqrt{2-2x_1}} = (\frac{1-x_1}{2})^{1/2}$
and its derivative is:
\[
\frac{1}{2} (\frac{1-x_1}{2})^{-1/2}(\frac{-x'_1}{2})
= \frac{-x'_1}{2\sqrt{2-2x_1}}
\]
This yields the stated derivative.
% Collecting, the derivative of the inverse Euler map is:
% \[
% \frac{1}{\sqrt{2-2x_1}} 
% (\frac{x'_1x_2 - 2x_1x'_2 + 2x'_2}{2-2x_1},
%  \frac{x'_1x_3 - 2x_1x'_3 + 2x'_3}{2-2x_1},
%  \frac{x'_1x_4 - 2x_1x'_4 + 2x'_4}{2-2x_1},
% -\frac{x'_1}{2})
% \]
\QED

Higher derivatives are found simply by repeating this process. 
DO SO IN MAPLE.

\clearpage

\begin{theorem}
\label{thm:imagecurve}
The image of the cubic Bezier curve $\sum_{i=0}^3 b_i B_i^3(t)$ under M 
is the sextic rational Bezier curve $\frac{\sum_{i=0}^6 w_i \breve{b}_i B_i^6(t)}
                                          {\sum_{i=0}^6 w_i B_i^6(t)}$ 
with control points:
\begin{eqnarray}
\breve{b}_0 & = & \frac{{\cal M}(b_0,b_0)}{b_0 \cdot b_0} \nonumber \\
\breve{b}_1 & = & \frac{{\cal M}(b_0,b_1) + {\cal M}(b_1,b_0)}{2b_0\cdot b_1} \nonumber \\
\breve{b}_2 & = & \frac{{\cal M}(b_0,b_2) + 3{\cal M}(b_1,b_1) + {\cal M}(b_2,b_0)}{2b_0 \cdot b_2 + 3b_1 \cdot b_1} \nonumber \\
\breve{b}_3 & = & \frac{{\cal M}(b_0,b_3) + 9{\cal M}(b_1,b_2) + 9{\cal M}(b_2,b_1) + {\cal M}(b_3,b_0)}{2b_0 \cdot b_3 + 18 b_1 \cdot b_2} \nonumber \\
\breve{b}_4 & = & \frac{{\cal M}(b_1,b_3) + 3{\cal M}(b_2,b_2) + {\cal M}(b_3,b_1)}{2b_1 \cdot b_3 + 3 b_2 \cdot b_2} \nonumber \\
\breve{b}_5 & = & \frac{{\cal M}(b_2,b_3) + {\cal M}(b_3,b_2)}{2b_2 \cdot b_3} \nonumber \\
\breve{b}_6 & = & \frac{{\cal M}(b_3,b_3)}{b_3 \cdot b_3} \nonumber
\end{eqnarray}
and weights:
\begin{eqnarray}
w_0 & = & b_0 \cdot b_0 \nonumber \\
w_1 & = & b_0 \cdot b_1 \nonumber \\
w_2 & = & \frac{2b_0 \cdot b_2 + 3b_1 \cdot b_1}{5} \nonumber \\
w_3 & = & \frac{b_0 \cdot b_3 + 9b_1 \cdot b_2}{10} \nonumber \\
w_4 & = & \frac{2b_3 \cdot b_1 + 3b_2 \cdot b_2}{5} \nonumber \\
w_5 & = & b_3 \cdot b_2 \nonumber \\
w_6 & = & b_3 \cdot b_3 \nonumber
\end{eqnarray}
Notice that ${\cal M}$ is the bilinear map of (\ref{eq:calM}).
\end{theorem}
\prf See the appendix.
\QED

\clearpage

\subsection{Appendix: proof of Euler map's image of a Bezier curve}
\label{sec:appendix}

\subsubsection{Bezier derivatives}

If $b(t)$ is a rational Bezier curve of degree $n$ 
with control points $\{b_i\}_{i=0}^n$ and 
weights $\{w_i\}_{i=0}^n$ defined over $t \in [0,1]$, then 
\begin{eqnarray}
\label{eqn:firstderiv}
b'(0) & = & n \frac{w_1}{w_0} (b_1 - b_0) \\
\label{eqn:secondderiv}
b''(0) & = & \frac{n-1}{n} \frac{w_0 w_2}{w_1^2} 
             \frac{\triangle b_0 \times \triangle b_1}{\| \triangle b_0\|^3}
\end{eqnarray}
where $\triangle b_i = b_{i+1} - b_i$.
See Farin \cite{farin02} and Hoschek/Lasser \cite{hoschekLasser0?}. 
% p. 244 of HL has conditions for curvature continuity
Notice the relationship of second derivative to curvature $\kappa$:
\[
    \kappa = \|b''\|
\]
if $b$ is arc-length parameterized, and in general,
\[
    \kappa = \frac{\| b' \times b'' \|}{\| b' \|^3}
\]
$C^2$ continuity guarantees curvature continuity.
See Farin \cite{farin02} or a differential geometry text.
% the formula for curvature of a rational Bezier curve is 
% k = \frac{n-1}{n} \frac{w_0 w_2}{w_1^2} 
%     \frac{\mbox{dist}(b_0,\lyne{b_1 - b_0}}{\mbox{dist}(b_0,b_1)^2}

The product rule of Bernstein polynomials \cite{farin02} states:
\[
B_i^m(t) B_j^n(t) = \frac{\choice{m}{i} \choice{n}{j}}{\choice{m+n}{i+j}} B_{i+j}^{m+n}(t)
\]

\clearpage

\subsubsection{Proof of Theorem~\ref{thm:imagecurve}}

\prf
Let $c(t)$ be the cubic Bezier curve.
Since the image of $c(t)$ is a rational Bezier curve, it is easier 
to work in projective space.
In projective space, the map $M$ becomes
\[
	 (x_1,x_2,x_3,x_4,x_5) \rightarrow
	 (x_1^2 + x_2^2 + x_3^2 - x_4^2,\ 
	 2x_1 x_4,\ 2x_2 x_4,\ 2x_3 x_4,\ 
	 x_1^2 + x_2^2 + x_3^2 + x_4^2).
\]
Let the image curve be expressed in projective space: 
$M(c(t)) = (m_1(t),m_2(t),m_3(t),m_4(t),m_5(t))$.
Each coordinate $m_i(t)$ may be simplified using the product rule 
of Bernstein polynomials \cite{farin02};
consider $m_5(t)$ as an example.
\begin{eqnarray}
m_5(t) & = & [\sum_{i=0}^3 B_i^3(t) b_{i1}]^2 + 
	\ldots + [\sum_{i=0}^3 B_i^3(t) b_{i4}]^2 \nonumber \\
     & = &  \sum_{i=0}^3 \sum_{j=0}^3 
	\frac{\choice{3}{i} \choice{3}{j}}{\choice{6}{i+j}}
       B^6_{i+j}(t) (b_{i1} b_{j1} + \ldots + b_{i4} b_{j4}) \nonumber
\end{eqnarray}
Letting $k=i+j$, 
\[ m_5(t) = \sum_{k=0}^6 B_k^6(t) 
	\sum_{\begin{array}{c}  \mbox{\footnotesize{$0 \leq i \leq 3$}} \\ 
			     \mbox{\footnotesize{$0 \leq j \leq 3$}} \\ 
			     \mbox{\footnotesize{$i+j=k$}}
			     \end{array}} 
	\frac{\scriptchoice{3}{i} \scriptchoice{3}{j}}{\scriptchoice{6}{k}}
	(b_{i1} b_{j1} + \ldots + b_{i4} b_{j4}) \]
The other coordinates can be computed analogously to yield:
\[ M(c(t)) = 
   \sum_{k=0}^6 B_k^6(t)
	\sum_{\begin{array}{c}  \mbox{\footnotesize{$0 \leq i \leq 3$}} \\ 
			     \mbox{\footnotesize{$0 \leq j \leq 3$}} \\ 
			     \mbox{\footnotesize{$i+j=k$}}
			     \end{array}} 
	\frac{\choice{3}{i} \choice{3}{j}}{\choice{6}{k}}
	\left( \begin{array}{c}
            b_{i1} b_{j1} + b_{i2} b_{j2} + b_{i3} b_{j3} - b_{i4} b_{j4} \\
            2b_{i1} b_{j4} \\
            2b_{i2} b_{j4} \\
            2b_{i3} b_{j4} \\
            b_{i1} b_{j1} + b_{i2} b_{j2} + b_{i3} b_{j3} + b_{i4} b_{j4}
	\end{array} \right) \]
This is a sextic rational Bezier curve with weights 
\begin{eqnarray}
\label{eq:weights}
w_k & = & \sum_{\begin{array}{c} \mbox{\footnotesize{$0 \leq i \leq 3$}} \\ 
			     \mbox{\footnotesize{$0 \leq j \leq 3$}} \\ 
			     \mbox{\footnotesize{$i+j=k$}}
			     \end{array}}
        \frac{\choice{3}{i} \choice{3}{j}}{\choice{6}{k}}
	\ (b_{i1} b_{j1} + b_{i2} b_{j2} + b_{i3} b_{j3} + b_{i4} b_{j4}) \nonumber \\
& = & \sum_{\begin{array}{c} \mbox{\footnotesize{$0 \leq i \leq 3$}} \\ 
			     \mbox{\footnotesize{$0 \leq j \leq 3$}} \\ 
			     \mbox{\footnotesize{$i+j=k$}}
			     \end{array}}
        \frac{\choice{3}{i} \choice{3}{j}}{\choice{6}{k}}
	\ b_i \cdot b_j
\end{eqnarray}
and control points 
\begin{eqnarray}
\label{eq:control-pts}
\breve{b}_k & = & \frac{1}{w_k} 
      \sum_{\begin{array}{c} \mbox{\footnotesize{$0 \leq i \leq 3$}} \\ 
			     \mbox{\footnotesize{$0 \leq j \leq 3$}} \\ 
			     \mbox{\footnotesize{$i+j=k$}}
			     \end{array}} 
        \frac{\choice{3}{i} \choice{3}{j}}{\choice{6}{k}}
	\left( \begin{array}{c}
            b_{i1} b_{j1} + b_{i2} b_{j2} + b_{i3} b_{j3} - b_{i4} b_{j4} \\
            2b_{i1} b_{j4} \\
            2b_{i2} b_{j4} \\
            2b_{i3} b_{j4} 
	\end{array} \right) \nonumber \\
& = & \frac{1}{w_k} 
      \sum_{\begin{array}{c} \mbox{\footnotesize{$0 \leq i \leq 3$}} \\ 
			     \mbox{\footnotesize{$0 \leq j \leq 3$}} \\ 
			     \mbox{\footnotesize{$i+j=k$}}
			     \end{array}} 
        \frac{\choice{3}{i} \choice{3}{j}}{\choice{6}{k}}
{\cal M}(b_i, b_j)
\end{eqnarray}
for $k = 0, \ldots, 6$.

By evaluating (\ref{eq:weights}), $w_0$ through $w_3$ can be simplified to
\begin{eqnarray}
(w_0,w_1,w_2,w_3) & = & 
% (b_0 \cdot b_0, 
% \frac{3b_0 \cdot b_1 + 3b_1 \cdot b_0}{6}, 
% \frac{3b_0 \cdot b_2 + 9b_1 \cdot b_1 + 3b_2 \cdot b_0}{15}, 
% \frac{b_0 \cdot b_3 + 9b_1 \cdot b_2 + 9b_2 \cdot b_1 + b3 \cdot b_0}{20}) \nonumber \\
% & = &
(b_0 \cdot b_0, b_0 \cdot b_1, \frac{2b_0 \cdot b_2 + 3b_1 \cdot b_1}{5}, 
\frac{b_0 \cdot b_3 + 9b_1 \cdot b_2}{10}) \nonumber 
\end{eqnarray}

Similarly, by evaluating (\ref{eq:control-pts}), $\breve{b}_0$ through $\breve{b}_3$ 
can be simplified to 
\begin{eqnarray}
\breve{b}_0 & = & % \frac{{\cal M}(b_0,b_0)}{w_0} =
                    \frac{{\cal M}(b_0,b_0)}{b_0 \cdot b_0} \nonumber \\
\breve{b}_1 & = & % \frac{{\cal M}(b_0,b_1) + {\cal M}(b_1,b_0)}{2w_1} =
                    \frac{{\cal M}(b_0,b_1) + {\cal M}(b_1,b_0)}{2b_0\cdot b_1} \nonumber \\
\breve{b}_2 & = & % \frac{{\cal M}(b_0,b_2) + 3{\cal M}(b_1,b_1) + {\cal M}(b_2,b_0)}{5w_2} = 
                    \frac{{\cal M}(b_0,b_2) + 3{\cal M}(b_1,b_1) + {\cal M}(b_2,b_0)}
	                 {2b_0 \cdot b_2 + 3b_1 \cdot b_1} \nonumber \\
\breve{b}_3 & = & % \frac{{\cal M}(b_0,b_3) + 9{\cal M}(b_1,b_2) + 9{\cal M}(b_2,b_1) + {\cal M}(b_3,b_0)}{20w_3} =
                    \frac{{\cal M}(b_0,b_3) + 9{\cal M}(b_1,b_2) + 9{\cal M}(b_2,b_1) + {\cal M}(b_3,b_0)}{2b_0 \cdot b_3 + 18b_1 \cdot b_2} \nonumber
\end{eqnarray}

$\breve{b}_4$ through $\breve{b}_6$ and $w_4$ through $w_6$ can then be determined 
through the symmetry of Bezier curves: 
since the geometry of the cubic Bezier curve does not change 
under reversal of the control points, neither does the algebra of its image.
The leverage of symmetry also generates more efficient formulae.\footnote{For example,
  $w_6$ and $\breve{b}_6$ are both expressed originally as sums of 7 terms.
  Using symmetry, they are now expressed by one term.}
\begin{eqnarray}
w_4 & = & \frac{2b_3 \cdot b_1 + 3b_2 \cdot b_2}{5} \nonumber \\
w_5 & = & b_3 \cdot b_2 \nonumber \\
w_6 & = & b_3 \cdot b_3 \nonumber \\
\breve{b}_4 & = & \frac{{\cal M}(b_3,b_1) + 3{\cal M}(b_2,b_2) + {\cal M}(b_1,b_3)}
	                 {2b_3 \cdot b_1 + 3b_2 \cdot b_2} \nonumber \\
\breve{b}_5 & = & \frac{{\cal M}(b_3,b_2) + {\cal M}(b_2,b_3)}{2b_3\cdot b_2} \nonumber \\
\breve{b}_6 & = & \frac{{\cal M}(b_3,b_3)}{b_3 \cdot b_3} \nonumber
\end{eqnarray}
\QED

\clearpage

\section{Poles}

The special behaviour of $M^{-1}$ at the pole $(1,0,0,0)$
is not unusual with maps from the sphere.
Consider the classical problem in cartography: how to portray the earth
on a 2-dimensional page.
A conformal map from $S^3$ to $\Re^2$ is desired.
Also consider the latitude/longitude parameterization of the sphere, with its two poles.
A parameterization is a map between the surface and $\Re^2$.

\clearpage

\subsection{Postamble to rational maps to the sphere}

The inverse image $M^{-1}(P)$ is a line or plane.
By perturbing the data away from the pole (Section~\ref{sec:perturb}), 
we may assume that $P \neq (1,0,0,0)$ and $M^{-1}(P)$ is a line.
Therefore, the interpolation of $\{M^{-1}(P_i)\}$ is the design of a curve through a
collection of lines.
Rather than directly attacking this non-traditional problem (see the future problems in
Section~\ref{sec:future}), we will reduce it to a traditional point
interpolation problem by choosing a point on each line.
Which point should be chosen on each line?
One solution is to always choose $M^{-1}(P)$ on $S^3$.
This solution (seems to?) promote short curves, a good thing, 
and avoids collapse into the origin, a bad thing.
Note that this is a well-defined choice, since $M^{-1}(P)$ is a line through the origin.
Another advantage of this choice is that it yields a simple inverse.

\clearpage

\section{Mapping derivatives (outside Rational maps to the sphere)}

$C^i$ continuity of the quaternion spline demands an understanding of derivatives.
The mapping of derivatives may be analyzed at the map level 
(rather than at the Bezier control polygon level, which is too complex and low-level).
We need to understand how first and ith derivatives of a curve on S3 map under 
the spherical maps (Euler and stereographic projection).
This may appear backward, since we do not yet know the curve on \Sn{3}.
However, we are alright since we do know the data point $C(0)$ at the beginning
of the curve, and its derivatives there.

If we interpret the maps away from the sphere as applied to a parameterized point
$(x_1(t),x_2(t),x_3(t),x_4(t))$, we can evaluate this map's derivative with respect
to the point's parameter.

For purposes of tangent continuity, 
we need to look at the first derivative of both curves.

\subsection{Mapping derivatives to Euclidean space, too}
\label{sec:derivative}

[From 3-spline/paper/old/splinePreChanges.tex]

As hinted in Section~\ref{sec:eucdesign}, 
we would like to design curves that interpolate not only point data,
but derivative data as well.
Beyond the direct application to situations where derivative information
is known (e.g., pick up the ball on the table at this position while
moving tangent to the table), 
derivative interpolation also allows us to design several curves 
independently yet guarantee that they join smoothly.
The key element in derivative interpolation is the mapping of derivatives
to Euclidean space.
The subsequent design of a rational curve in Euclidean space
interpolating the derivatives as well as the points
is well understood: using either Hermite curves or, if the only derivatives
are end tangents, using the same classical cubic B-spline interpolation that
we have used for points.
In this section, we show how tangents on the surface can be mapped
to tangents in Euclidean space.

Let $C$ be the curve in Euclidean space and $M(C)$ the curve on the surface.
We assume, without loss of generality, that the polynomial curve $C$
and the rational curve $M(C)$ are both represented as Bezier curves.
We shall only present the interpolation of end tangents,
since this is the most common case, and intermediate tangents can be
reduced to this case by subdivision.
Suppose that $M(C)$ must be designed to interpolate the points 
$\{p_i\}_{i=1,\ldots,n}$ and the tangent $T$ at $p_1$.
(Interpolation of a tangent at $p_n$ is analogous.)
We begin with a sketch of the method.
Since the beginning tangent of a Bezier curve is defined by its first 
two control points, the constraint on the beginning tangent of $M(C)$ can be
translated into a constraint on the first two control points of $M(C)$.
This constraint can then be translated back into a constraint
on the first two control points of $C$, using Theorem~\ref{sextic}'s
relationship between the control points of $C$ 
and the control points and weights of $M(C)$.
Moreover, the first control point of $C$ is already known:
it is the image of the first data point, $M^{-1}(p_1) \cap \Sn{3}$,
since Bezier curves are endpoint-interpolating.
Therefore, the tangent constraint on \Sn{3}\ can be translated
into a constraint on the second control point in Euclidean space.
We now proceed with the details.
% The following theorem provides the details of how to build the tangent
% in Euclidean space (technically, the second control point) that
% will correspond to the tangent $T$ in Riemannian space.

Let $b_0 = (b_{01}, b_{02}, b_{03}, b_{04})$ and 
$b_1 = (b_{11}, b_{12}, b_{13}, b_{14})$ be 
the first two control points of $C$,
$c_0$ and $c_1$ the first two control points of $M(C)$,
and $w_0$ and $w_1$ the weights of the first two control points of $M(C)$.
The desired tangent is $T$ on the global curve on \Sn{3},
but $\delta_i T$ on the local Bezier curve component of knot length $\delta_i$
of the subcurve on \Sn{3}.
% local tangent = (knot interval \delta) * global tangent (see p. 105, Farin, 3rd)
We are trying to solve for $b_1$ in Euclidean space.
By the theory of rational Bezier curves \cite{farin97},
\[
	\delta_i T = 6 \frac{w_1}{w_0} (c_1 - c_0)
\]
Applying Theorem~\ref{sextic}, 
\[
	\delta_i T = 6 \frac{w_1}{w_0} \frac{1}{w_1} \frac{3}{6}
	\left(
	\begin{array}{c}
	2(b_{01}b_{11} + b_{02}b_{12} + b_{03}b_{13} - b_{04}b_{14})\\
	2(b_{01}b_{14} + b_{11}b_{04})\\
	2(b_{02}b_{14} + b_{12}b_{04})\\
	2(b_{03}b_{14} + b_{13}b_{04})
	\end{array} \right)
	- 6 \frac{w_1}{w_0} \frac{1}{w_0}
	\left(
	\begin{array}{c}
	b_{01}^2 + b_{02}^2 + b_{03}^2 - b_{04}^2\\
	2b_{01}b_{04}\\
	2b_{02}b_{04}\\
	2b_{03}b_{04}
	\end{array} \right).
\]
We claim that $b_0$ and $w_0$ are known.
$b_0 = M^{-1}(p_1) \cap \Sn{3}$, since the first Bezier control point agrees
with the first point of the curve.
Using (\ref{eq:weights}) of Theorem~\ref{sextic},
$w_0 = b_{01}^2 + b_{02}^2 + b_{03}^2 + b_{04}^2$ 
(so $w_0=1$ since $b_0 \in \Sn{3}$)
and
\[
  w_1 = b_{01}b_{11} + b_{02}b_{12} + b_{03}b_{13} + b_{04}b_{14}
\]
This simplifies the relationship between the tangent $T$ and $b_1$:
\[
	\delta_i T = 6 \left( \begin{array}{c}
	(b_{01}b_{11} + b_{02}b_{12} + b_{03}b_{13} - b_{04}b_{14})
	- \hat{b} 
	  (b_{01}b_{11} + b_{02}b_{12} + b_{03}b_{13} + b_{04}b_{14}) \\
	(b_{01}b_{14} + b_{11}b_{04})
	- 2b_{01}b_{04} 
	  (b_{01}b_{11} + b_{02}b_{12} + b_{03}b_{13} + b_{04}b_{14}) \\
	(b_{02}b_{14} + b_{12}b_{04})
	- 2b_{02}b_{04}
	  (b_{01}b_{11} + b_{02}b_{12} + b_{03}b_{13} + b_{04}b_{14}) \\
	(b_{03}b_{14} + b_{13}b_{04})
	- 2b_{03}b_{04}
	  (b_{01}b_{11} + b_{02}b_{12} + b_{03}b_{13} + b_{04}b_{14}) \\
	\end{array} \right)
\]
where $\hat{b} = b_{01}^2 + b_{02}^2 + b_{03}^2 - b_{04}^2$.
Finally, this can be expressed as a linear system:
\[
6 \left( \begin{array}{cccc}
b_{01}(1-\hat{b}) & b_{02}(1-\hat{b}) & b_{03}(1-\hat{b}) & b_{04}(-1-\hat{b}) \\
b_{04}(1-2b_{01}^2) & -2b_{01}b_{02}b_{04} & -2b_{01}b_{03}b_{04} & b_{01}(1-2b_{04}^2) \\
-2b_{01}b_{02}b_{04} & b_{04} (1-2b_{02}^2) & -2b_{02}b_{03}b_{04} & b_{02}(1-2b_{04}^2) \\
-2b_{01}b_{03}b_{04} & -2b_{02}b_{03}b_{04} & b_{04}(1-2b_{03}^2) & b_{03}(1-2b_{04}^2)
\end{array} \right)
\left( \begin{array}{c} b_{11} \\ b_{12} \\ b_{13} \\ b_{14} \end{array} \right)
= \delta_i T.
\]
%
Notice that all components of the matrix are known.
This system can be solved for the second control point $b_1 = (b_{11},b_{12},b_{13},b_{14})$
using traditional methods for the solution of systems of linear
equations.
We use LU-decomposition with partial pivoting \cite{golubvanloan89}.
The following theorem formalizes this result.

\begin{theorem}
Let $C$ be a cubic Bezier curve in Euclidean space 
on the knot interval of length $\delta_i$,
whose first control
point is $b_0 = (b_{01}, b_{02}, b_{03}, b_{04})$.
If the second control point of $C$ is the solution $x$ to the linear system
\[
6 \left( \begin{array}{cccc}
b_{01}(1-\hat{b}) & b_{02}(1-\hat{b}) & b_{03}(1-\hat{b}) & b_{04}(-1-\hat{b}) \\
b_{04}(1-2b_{01}^2) & -2b_{01}b_{02}b_{04} & -2b_{01}b_{03}b_{04} & b_{01}(1-2b_{04}^2) \\
-2b_{01}b_{02}b_{04} & b_{04} (1-2b_{02}^2) & -2b_{02}b_{03}b_{04} & b_{02}(1-2b_{04}^2) \\
-2b_{01}b_{03}b_{04} & -2b_{02}b_{03}b_{04} & b_{04}(1-2b_{03}^2) & b_{03}(1-2b_{04}^2)
\end{array} \right)
x
= \delta_i T
\]
where $\hat{b} = b_{01}^2 + b_{02}^2 + b_{03}^2 - b_{04}^2$,
then the image $M(C)$ on \Sn{3}\ has the beginning tangent $\delta_i T$.
\end{theorem}
% NOTE: incorporate knot lengths into the computation of the
% desired tangent vector $T$ when using in code, since
% code will involve splines rather than individual Bezier curves.

% \begin{rmk}
% Comment on why we cannot use derivative of $M_I$ to map the
% tangent to Euclidean space.
% Basically, $C(t) \neq M_I(M(C(t))$, see comment after commented-out lemma below.
% Give example of difference between tangent computed this way
% and tangent computed our correct way.
% \end{rmk}

[This lemma is necessary, even though we know the tangent $T$ of the curve
in Euclidean space for the previous subcurve, because $T$ is a tangent of
the curve in Euclidean space associated with a {\em rotated} version
of the previous subcurve on \Sn{3}, and we need the tangent associated
with a {\em different} rotated version of the present subcurve.
Unfortunately, $M^{-1}$ is not affinely invariant 
(tangent ($M^{-1}$(rotation of C(t))) is not the rotation of 
tangent($M^{-1}$(C(t))).]

\Comment{
\begin{lemma}
\label{lem:invtang}
Consider the design of a curve on \Sn{3}\ interpolating the data points 
$\{p_i\}_{i=1,\ldots,n}$ using step (2A).
Assume that none of the data points lies near the pole.
If we want the curve on \Sn{3}\ to have the first derivative 
$(T_1,T_2,T_3,T_4)$ at the data point $p_i = (P_1,P_2,P_3,P_4)$, 
the curve in Euclidean space should 
be designed with the tangent 
\[
 \frac{1}{(2-2P_1)^{3/2}} (T_2 (2-2P_1) + P_2 T_1,\ \ 
 	   		 T_3 (2-2P_1) + P_3 T_1,\ \ 
			 T_4 (2-2P_1) + P_4 T_1,\ \ 
			 T_1 (2-2P_1) + (1-P_1) T_1)
\]
at the point ${\cal M}^I(p_i)$.
\end{lemma}
\prf
Let $D(t) = \{(x_1(t), x_2(t), x_3(t), x_4(t)): t \in I\}$ 
be any curve on \Sn{3} with $D(t_0) = p_i$ and
$D'(t_0) = (T_1,T_2,T_3,T_4)$ for some $t_0 \in I$.
Let $E(t) = \{{\cal M}^I(D(t)): t \in I\}$ be the image of $D(t)$
under ${\cal M}^I$.
\[
	E(t) = \frac{(x_2(t),x_3(t),x_4(t),1-x_1(t))}{\sqrt{2-2x_1(t)}}.
\]
If we let $y(t) = \sqrt{2-2x_1(t)}$,
$E(t) = (\frac{x_2}{y}, \frac{x_3}{y}, \frac{x_4}{y}, \frac{1-x_1}{y})$.
(Every variable implicitly depends on the parameter $t$, but we shall now omit this dependence
for clarity.)
By straightforward differentiation,
\[
 E'(t) = (\frac{x'_2 y - x_2 y'}{y^2}, 
	  \frac{x'_3 y - x_3 y'}{y^2}, 
	  \frac{x'_4 y - x_4 y'}{y^2}, 
	  \frac{-x'_1 y - (1-x_1) y'}{y^2}).
\]
To compute $y'$, notice that $y^2 = 2-2x_1$ and, differentiating,
$2yy' = -2x'_1$.
Thus, $y' = \frac{-x'_1}{y}$.
We can now express
\[
 E'(t) = (\frac{x'_2 y^2 + x_2 x'_1}{y^3},
 	  \frac{x'_3 y^2 + x_3 x'_1}{y^3},
 	  \frac{x'_4 y^2 + x_4 x'_1}{y^3},
 	  \frac{x'_1 y^2 + (1-x_1) x'_1}{y^3}).
\]
Since $p_i$ is far from the pole, $y(t) \neq 0$ in a neighbourhood of $t_0$.
Thus,
\[
 E'(t_0)= (\frac{T_2 (2-2P_1) + P_2 T_1}{(2-2P_1)^{3/2}},
 	   \frac{T_3 (2-2P_1) + P_3 T_1}{(2-2P_1)^{3/2}},
 	   \frac{T_4 (2-2P_1) + P_4 T_1}{(2-2P_1)^{3/2}},
	   \frac{T_1 (2-2P_1) + (1-P_1) T_1}{(2-2P_1)^{3/2}}).
\]
A curve designed with tangent $E'(t_0)$ in Euclidean space
will map to a curve with tangent $(T_1,T_2,T_3,T_4)$ on \Sn{3}.
\QED

Notice that the curve $E(t)$ of the proof is not equivalent to 
the curve $C(t)$ that will be designed in Euclidean space.
In particular, $E(t)$ lies on \Sn{3} since ${\cal M}^I$ maps to \Sn{3},
whereas $C(t)$ is not constrained to \Sn{3}, so it will only
lie on \Sn{3}\ at the data points $\{ {\cal M}^I(p_i) \}_{i=1,\ldots,n}$, 
in general.
However, the tangent behaviour of $C(t)$ and $E(t)$ at the inverse data points 
$\{ {\cal M}^I(p_i)\}_{i=1,\ldots,n}$ {\bf is} the same.
APPARENTLY NOT!
}

\subsubsection{}

[From 3-spline/oldpaper/S3paper.tex]

The technique below does not work, since $M^{-1}(M(C(t))) \neq C(t)$, 
since $M^{-1}$ is not 1-1.
We instead need to understand how the tangent line maps under $M^{-1}$,
where $M^{-1}$ is defined over all of 3-space, not just S3.
This will probably generate a ruled surface (each point maps to a line).
It is not clear how to extract a tangent from this ruled surface.

\title{On the relationship between Euler's 4-squares theorem, 
	stereographic projection, and rational curves on $S^3$}

We want the facility to set the end tangents of curves on the 3-sphere,
so that curves can be smoothly spliced together.
This allows a divide-and-conquer approach to curve fitting.

\begin{lemma}
Let $P=C(\hat{t})=(p1,p2,p3,p4)$ be a point of the S3-curve $C(t)$ 
with tangent $C'(\hat{t}) = (t1,t2,t3,t4)$.
Under the map $M^{-1}$, tangents are mapped as:
	$C'(t)  \rightarrow (M^{-1}(C(t)))'$ \\
	$(x_1,x_2,x_3,x_4) + s(x'_1,x'_2,x'_3,x'_4) \mapsto 
	 M^{-1}(x_1,x_2,x_3,x_4) 
      + s(\frac{x'_2 \alpha - x_2 \alpha'}{\alpha^2},
	  \frac{x'_3 \alpha - x_3 \alpha'}{\alpha^2},
	  \frac{x'_4 \alpha - x_4 \alpha'}{\alpha^2},
	  \frac{-x'_1 \alpha - (1-x_1)\alpha'}{\alpha^2})$
where $\alpha = \sqrt{2 - 2x_1}, \alpha' = -\frac{x'_1}{\alpha}$.
\end{lemma}
\prf
Simple differentiation of $M^{-1}(C(t))$.
\QED

Thus, if we want a spherical curve to have a prescribed end tangent,
we can set the desired endtangent of the endpoint in image space
using this lemma and fit the image curve to this tangent.

see page FOO1

If we are setting the tangent on the spherical curve independently, 
notice that the tangent vector $C'(t_0)$ must be orthogonal to the vector $C(t_0)$
since the tangent at any point of $C(t)$ must lie in the sphere's tangent plane
at this point.)


\clearpage

\subsubsection{Mapping the tangent restriction from $S^3$ to the space curve:
         difficulties with the Euler map}

By combining the formula for the second control point $\breve{b}_1$ of the 
quaternion curve from Theorem~\ref{thm:imagecurve} 
with the tangent formula (\ref{eqn:firstderiv}) for a rational
Bezier curve, $\breve{b}_1$ can be eliminated and a linear system set up.
This system relates the known first control points $b_0$ and $\breve{b}_0$
of the space and quaternion curves with the unknown second control point $b_1$,
and is exactly what is necessary to control the tangent in 4-space so that its image
is the desired tangent $T_0$.
Consider the formula:
\begin{equation}
\label{eq:b1}
\breve{b}_1 = \frac{{\cal M}(b_0,b_1) + {\cal M}(b_1,b_0)}{2b_0 \cdot b_1}
\end{equation}
    % We can use this formula to solve for $b_1$, as follows.
    % First observe that 
and the formula for the opening tangent of a rational Bezier curve: 
\[
T_0 = 6 \frac{w_1}{w_0}(\breve{b}_1 - \breve{b}_0)
            = 6\frac{b_0 \cdot b_1}{b_0 \cdot b_0} (\breve{b}_1 - \breve{b}_0)
\]
Rearranging, we have:
\begin{equation}
\label{eq:b1-2}
        \breve{b}_1 = \frac{b_0 \cdot b_0}{6b_0 \cdot b_1} T_0 + \breve{b}_0
\end{equation}
Equating (\ref{eq:b1}) and (\ref{eq:b1-2}),
\[
\frac{{\cal M}(b_0,b_1) + {\cal M}(b_1,b_0)}{2b_0 \cdot b_1} = 
\frac{b_0 \cdot b_0}{6b_0 \cdot b_1} T_0 + \breve{b}_0
\]
Multiplying by $b_0 \cdot b_1$ and setting $\alpha := \frac{b_0 \cdot b_0}{6}$:
\begin{equation}
\frac{{\cal M}(b_0,b_1) + {\cal M}(b_1,b_0)}{2} - (b_0 \cdot b_1) \breve{b}_0 =
\alpha T_0
\end{equation}
This encodes a linear system of equations:
\begin{eqnarray}
b_{01}b_{11} + b_{02}b_{12} + b_{03}b_{13} - b_{04}b_{14} -
(b_{01}b_{11} + b_{02}b_{12} + b_{03}b_{13} + b_{04}b_{14}) \breve{b}_{01}
& = & \alpha T_{01} \nonumber \\
b_{01}b_{14} + b_{11}b_{04} - 
(b_{01}b_{11} + b_{02}b_{12} + b_{03}b_{13} + b_{04}b_{14}) \breve{b}_{02}
& = & \alpha T_{02} \nonumber \\
b_{02}b_{14} + b_{12}b_{04} - 
(b_{01}b_{11} + b_{02}b_{12} + b_{03}b_{13} + b_{04}b_{14}) \breve{b}_{03}
& = & \alpha T_{03} \nonumber \\
b_{03}b_{14} + b_{13}b_{04} - 
(b_{01}b_{11} + b_{02}b_{12} + b_{03}b_{13} + b_{04}b_{14}) \breve{b}_{04}
& = & \alpha T_{04} \nonumber
\end{eqnarray}
or, solving for $b_1$, the second control point of the space curve:
\begin{equation}
\label{eq:linearsystem}
\left(
\begin{array}{cccc}
b_{01} (1-\breve{b}_{01})     & b_{02} (1-\breve{b}_{01}) & b_{03} (1-\breve{b}_{01}) & b_{04} (-1-\breve{b}_{01}) \\
b_{04} - b_{01}\breve{b}_{02} & -b_{02}\breve{b}_{02}     & -b_{03}\breve{b}_{02}     & b_{01} - b_{04}\breve{b}_{02} \\
-b_{01}\breve{b}_{03}         & b_{04} -b_{02}\breve{b}_{03} & -b_{03}\breve{b}_{03}  & b_{02} - b_{04}\breve{b}_{03} \\
-b_{01}\breve{b}_{04}         & -b_{02}\breve{b}_{04}     & b_{04}-b_{03}\breve{b}_{04} & b_{03} - b_{04}\breve{b}_{04}
\end{array}
\right)
\left(
\begin{array}{c}
b_{11} \\
b_{12} \\
b_{13} \\
b_{14}
\end{array}
\right)
= 
\frac{b_0 \cdot b_0}{6}
\left(
\begin{array}{c}
T_{01} \\
T_{02} \\
T_{03} \\
T_{04}
\end{array}
\right)
\end{equation}
Note that $b_0$ and $\breve{b}_0$ are known by the endpoint interpolation property:
$\breve{b}_0 = P_0$ and
\begin{equation}
b_0 = M^{-1}(P_0) = M^{-1}(b_0) 
     = \frac{1}{\sqrt{2-2\breve{b}_{01}}}(\breve{b}_{02},\breve{b}_{03},\breve{b}_{04},1-\breve{b}_{01})
\end{equation}
Substituting for $b_0$ yields:
\[
A = \frac{1}{\alpha_2}
\left(
\begin{array}{rrrr}
\breve{b}_{02} - \breve{b}_{02}\breve{b}_{01} & \breve{b}_{03} -\breve{b}_{03}\breve{b}_{01} & \breve{b}_{04} - \breve{b}_{04}\breve{b}_{01} & \breve{b}_{01}^2 - 1 \\
1 - \breve{b}_{01} - \breve{b}_{02}^2 & -\breve{b}_{03}\breve{b}_{02}     & -\breve{b}_{04}\breve{b}_{02}     & \breve{b}_{01}\breve{b}_{02} \\
-\breve{b}_{02}\breve{b}_{03} & 1-\breve{b}_{01} - \breve{b}_{03}^2 & -\breve{b}_{04}\breve{b}_{03}  & \breve{b}_{01}\breve{b}_{03} \\
-\breve{b}_{02}\breve{b}_{04}         & -\breve{b}_{03}\breve{b}_{04}     & 1-\breve{b}_{01} - \breve{b}_{04}^2 & \breve{b}_{01}\breve{b}_{04}
\end{array}
\right)
\]
where $\alpha_2 := \sqrt{2-2\breve{b}_{01}} \neq 0$,
so that the linear system is:
\[
Ab_1 = \alpha T_0
\]
Unfortunately, this matrix is singular:
det(A) = $\frac{1}{\alpha_2}(\breve{b}_{01} - 1)^3 
(\breve{b}_{01}^2 + \breve{b}_{02}^2 + \breve{b}_{03}^2 + \breve{b}_{04}^2 - 1) = 0$,
since $\breve{b}_0 \in S^3$.
START HERE.
THIS UNSOLVABLE LINEAR SYSTEM IS THE CRUX OF THE PROBLEM.
STEREOGRAPHIC PROJECTION MAY YIELD A SOLVABLE SOLUTION.
This system may be solved for $b_1$ if the matrix A is nonsingular
(and we must also take care that $b_0 \cdot b_1 \neq 0$ due to the above manipulation).
We may assume that $\breve{b}_0 \neq (1,0,0,0)$, because of our later perturbation
of the data points away from the pole.
Consequently, we may also assume that $b_0$ does not lie in the hyperplane $x_4=0$
(Lemma~\ref{lem:inverse}).
START HERE
Consider the determinant of A:
\[
\]
In the design of the space curve in 4-space,
$b_1$ is a free parameter.
By setting it to the solution of (\ref{eq:linearsystem}),
the image curve on $S^3$ will have the desired end tangent $T_0$.

Is this matrix nonsingular?

The same approach may be used to constrain the closing tangent of the $S^3$ curve.
Rather than using the formulae (\ref{eq:control-pts}), 
which defines the penultimate control point $b_{5}$ in terms of 
all of the control points $\{b_i\}$,
we leverage the symmetry of Bezier curves.
If $C(t)$ is a degree $n$ Bezier curve with control points $\{c_i\}$ and 
another degree $n$ Bezier curve $D(t)$ is defined by the control points $d_i = c_{n-i}$,
then $D(t) = C(1-t)$: the two curves have the same shape, only the time parameter 
is reversed.
Using this symmetry, an end tangent constraint $T_1$ on the quaternion spline
can be translated to a constraint on the penultimate control point $b_{2}$ 
of the space curve, using the above approach.

To recap, the following algorithm designs a Hermite quaternion spline:
---

The quaternion interpolation problem is translated into a Hermite quaternion spline
as follows.
We allow the end tangents to be specified.  If they are not given, 
we can fill them in ourselves.
If a closed quaternion spline is desired, only one end tangent is specified, of course.
(Split quaternions into subsets; perturb first subset using BFP approach;
interpolate first subset, using first prescribed tangent as opening tangent;
perturb ith subset using BFP approach;
interpolate ith subset, using endtangent of previous curve as opening tangent;
...
perturb last subset using BFP approach;
interpolate last subset, using endtangent of previous curve as opening tangent
and prescribed end tangent of entire curve as closing tangent.)
---

WHAT ABOUT $C^2$ CONTINUITY BETWEEN SEGMENTS?

Consider the ramp up to $C^2$ continuity.
$C^1$-continuity required the use of formulae for $\breve{b}_1$, $w_1$ and the first derivative 
of a rational Bezier curve.
$C^2$-continuity requires the use of formulae for $\breve{b}_2$, $w_2$ and the second derivative 
of a rational Bezier curve.
WE NEED TO FIND THE FORMULA FOR 2nd DERIVATIVE OF RATIONAL BEZIER CURVE AT ENDPOINT.

%  This can be used to solved for $B_1$, if it were not for $w_1$.
%  We cannot predict the weight $w_1$, since this circularly involves 
%  knowledge of $b_1$ that we are trying to find.
%  Therefore, we cannot use this formula to set the position of $B_1$ to exactly match $T_0$.
% However, any $B_1$ that satisfies $T_0 = k(B_1 - B_0)$ will match the tangent direction.
% If the $S^3$ curve were polynomial, we would have $T_0 = 6(B_1-B_0)$,
% so it is natural to set 
%                           $B_1 = B_0 + \frac{T_0}{6}$:
% this will nicely match $T_0$ in length as well as direction whenever the weights of the
% quaternion spline are relatively equal, which is commonly the case 
% (SHOW EXPERIMENTAL EVIDENCE).

\clearpage

\section{Rotating away from the pole}

\subsection{Perturbing, isn't it?}

Perturbing a pointset away from the pole.
Want to map away from the pole because both spherical maps are undefined at a pole
((1,0,0,0) for inverse Euler map, (0,0,0,1) for stereographic projection)
and sensitive in its environment (say within d degrees).
The best-fitting plane (BFP) approach is to find the best-fitting plane of the quaternion
pointset and rotate the pole to the unit normal of this plane.
This approach is not valid if there are outliers, since they may be mapped directly
to the pole.
However, the BFP approach is robust for local point sets (point sets 
that do not span 90-d degrees).
Therefore, by splitting a pointset into local pointsets, we can apply the BFP approach.
This requires the ability to build a smooth quaternion spline segment by segment, which
requires the ability to interpolate derivative information at boundaries.

Subset must be small enough to guarantee perturbation by BFP is valid;
In particular, a subset must not span more than about 60 degrees, which guarantees
30 degree clearance from the pole.
Notice that every orientation has 2 reps on $S^3$, so a 90 degree shift on $S^3$
is equivalent to a 180 degree change of orientation, and rotations in 3-space are halved
as they map to $S^3$: WE NEED TO DETERMINE THE CONVERSION CONSTANT BETWEEN THE METRICS.

------------

{\bf PAPER: rational sphere parameterizations inherently have poles} (reference?);
this may be connected to the presence of poles in 
both the rational Euler map and SP; also the presence of poles in Euler angles,
as evidenced by translation to rotation matrix, and gimbal lock.

Move this writeup to a new file 'Rational quaternion splines with Hermite control'.
% Hermite interpolation and more elegant affine invariance

\subsection{Divide and conquer: perturbing one subset at a time}

We could gather points while their angular distance from the first point is less
than k (in our case, 60) degrees.
This would make for larger pointsets in most cases.
(It would be nice to find a natural pointset of quaternions, perhaps from keyframes
of a real animation or from motion capture libraries, in order to determine a
natural spread of a quaternion dataset.)
NEED A NONTRIVIAL DATASET.

\subsection{Affine invariance?}
\label{sec:normalform}

We need to normalize at the beginning so that all rotated variants of the same dataset
begin at the same place.
(The creation of quaternion spline is not itself invariant under rotation of the
quaternions.)
This can be accomplished by imposing a normal form on quaternion datasets
before building quaternion splines.
In particular, rotate the first quaternion to point X, the second quaternion to plane Y,
and the third quaternion to hyperplane Z.

{\bf PAPER: normalizing datasets before applying maps that are not affine invariant.}

\clearpage

\subsection{Rotation in higher dimensions}

START HERE.
THIS SHOULD BE A SEPARATE TECHNICAL REPORT TOO (OR ACM SOUTHEAST PAPER).
A REVIEW OF ROTATION: SO(3), SO(n), Givens, Jacobi, cross product, generalized cross product, determinants

\subsubsection{Generalizations of the cross product}

The cross product of two 3-vectors is a 3-vector orthogonal to both.
We want a cross product of n-1 n-vectors that generates an n-vector
orthogonal to all.
This is a necessary tool for the construction of certain rotation matrices.
Although the cross product of 3-vectors A = (a1,a2,a3) and B = (b1,b2,b3) 
may be defined as (a2b3-a3b2, a3b1-a1b3, a1b2-a2b1), it is more fruitful for the 
purposes of generalization to define it in terms of a determinant:
\[
A \times B = 
\left|
\begin{array}{ccc}
e1 & e2 & e3 \\
a1 & a2 & a3 \\
b1 & b2 & b3
\end{array}
\right|
\]
This leads immediately to a generalized cross product of $n-1$ n-vectors
\[
cross (A1,\ldots,A_{n-1}) = 
\left|
\begin{array}{ccc}
e_1 & \ldots & e_n \\
a_{1,1} & \ldots & a_{1,n} \\
\vdots \\
a_{n-1,1} & \ldots & a_{n-1,n} 
\end{array}
\right|
\]
If we build the nxn real matrix A with arbitrary 1st row 
and ith row = $A_{i-1}$ for $i=2,\ldots,n$, then
$cross (A1, \ldots,A_{n-1}) =
(subdet(A,1,1), -subdet(A,1,2), \ldots, (-1)^{1+n} subdet(A,1,n))$,
where subdet(A,i,j) is the minor of A with the ith row and jth column removed.

\subsubsection{Rotating a vector to a coordinate axis}

This generalized cross product may be used to construct rotation matrices
that rotate an arbitrary vector in n-space to a coordinate axis.
Since rotation matrices are orthogonal, they can be built from rows that
are mutually orthogonal.
Under the operation of a rotation matrix, 
the rows of the rotation matrix are mapped to the coordinate axes.\footnote{This
   is a direct product of the Kronecker-delta behaviour of the dot product of the rows,
   reflecting orthonormality.}
Using this fact, a matrix that rotates the arbitrary unit vector $v \in \Re^n$ 
(where $v$ is not any coordinate axis, since we can deal with these n cases easily)
to the coordinate axis $e_i$ is the matrix with ith row $v$ and other rows
$R_1 = cross (v,e_1,\ldots,e_{n-2}), R_2 = cross(v,R_1,e_2,\ldots,e_{n-2}, 
R_3 = cross (v,R_1,R_2, e_3, \ldots, e_{n-2}), \ldots, 
R_{n-1} = cross (v,R_1,\ldots,R_{n-2})$,
where we start with v and coordinate axes, and then replace a coordinate axis
by a row as soon as it is computed.
%          need n-1 other rows (other than v)
%	  each of these rows needs n-1 vectors to build a generalized cross product
%	      - one of these vectors is always v
%	      - the last vector is the cross product of v and the other n-2 vectors
%	      - one of the vectors can always be e_i
%	  row1 = v
%	  row2 = v x e_1 x ... x e_{n-2}
%	  row3 = v x row2 x e_2 x ... x e_{n-2}
%	  ...
%	  rown = v x row2 x ... x row_{n-1}

    In building a rotation matrix that rotates a vector to a coordinate axis,
    coordinate axes are used to build the rows of the matrix, through the
    generalized cross product.  We can use any coordinate axes (actually,
    any vector not equal to v, but the coordinate axes are the only natural choice).  
    Therefore, we choose the coordinate axes to make the algorithm robust.
    The generalized cross product of three vectors
    a,b,c is less robust if these vectors are close to linearly dependent,
    or equivalently if the underlying matrix is close to singular. (ELABORATE ABOVE
    IN CROSS PRODUCT SECTION.)
    Therefore, we want to use the furthest coordinate axes from v.
    This has the added benefit that it guarantees that the chosen coordinate axes
    do not agree with v (since at most one coordinate axis can be equivalent to v,
    and it would certainly not be one of the two closest), which avoids
    a preprocessing step of checking for this degenerate case when the
    cross product would be undefined.
    (See Vec4::rotToCoord.)

START HERE (compare to qspline.c/'PerturbInput')

We can flip things to change the sign of the determinant, if necessary, to guarantee determinant = 1.  (WHY IS SIGN = 1 IMPORTANT FOR SO(3)?)

This leads to a generalization of SO(3), the group of rotation matrices in 3-space.
Notice that Givens (or Jacobi) rotations \cite{golubvanLoan} are also generalizations
of rotation matrices, but are restricted to rotations about the coordinate axes;
we want a generalization of rotations about arbitrary vectors.
The defining 

\clearpage

\section{Future directions}
\label{sec:future}

Open questions.
\begin{enumerate}
\item exact relationship of torque to quaternion spline derivatives
\end{enumerate}

The open problems suggested by this paper include:

\begin{itemize}
\item An algorithm to find the emptiest point, or the largest empty circle, on $S^3$.
      That is, given a point sample on $S^3$, what is the point furthest from this point
      sample?  This is a classical problem in the plane, where optimal(?) 
      solutions have long been known.  It has also been studied on $S^2$ by Renka.
      The $S^3$ problem is particularly interesting, since it combines a search on
      a surface with a search in a 3-dimensional space.
\item Interpolation of a set of lines, rather than points.
      This would be the natural construction of the space curve.
      We presented one solution to this problem through our choice of a point on each line,
      reducing the problem to point interpolation.
      However, the choice of point on each line did not consider the entire line set,
      at most considering one neighbouring line.
      It would be interesting to develop a more global solution.
\end{itemize}

Elaborations of this work include the following (see Introduction).
\begin{enumerate}
\item
Consider extrinsic obstacles in the design 
of the motion, either positional or orientational.
For example, consider the design of a motion for the teapot that avoids
the table or a bounding dog, or that respects the orientational constraints 
imposed by placing the pot on the middle shelf of a china cabinet.
Equivalently, give the animator the capacity to effortlessly design 
a natural motion that respects 
certain constraints inherent {\em in the environment}.
This becomes motion planning with full degrees of freedom, both positional and
orientational.
Even this 'motion planning' problem for position is interesting, 
since we desire smooth paths, not piecewise linear paths.
(This becomes obstacle-avoiding interpolation.  Pottmann proposes a weak solution
in SIGGRAPH04.  A better solution is needed.)
\item
Addition of constraints like limited G's on the airplane, or acceleration bounds
on the teapot.
\item
Add {\em time-varying} constraints created by other obstacles.
\end{enumerate}

\bibliographystyle{plain}
\bibliography{modeling} % merge Euro95/swept-paper (this has more) and GI95/animation-paper
                        % and move all to ~/software (where it should be CVS'ed)
                        % also merge with 2d-filestructure bib files
                        % once all merged, can decompose into quaternion/visibility/... bibfiles
                        % for now, I am adding to the Euro95 edition
                        % add 2-most/comp.tex bibliography too
\end{document}
