\documentstyle[11pt]{article}
\newif\ifFull
\Fullfalse
\input{header}
\SingleSpace

\setlength{\oddsidemargin}{0pt}
\setlength{\topmargin}{-.25in}	% technically should be 0pt for 1in
\setlength{\headsep}{3em}
\setlength{\textheight}{8.75in}
\setlength{\textwidth}{6.5in}
\setlength{\columnsep}{5mm}		% width of gutter between columns

\markright{Maps to the sphere: \today \hfill}
\pagestyle{myheadings}

% -----------------------------------------------------------------------------

\title{A Characterization of Rational Maps of $\Re^n$ to \Sn{n-1}}
%\title{A Characterization of Rational Maps to the Sphere}
\author{J.K. Johnstone}

\begin{document}
\maketitle

\begin{abstract}
We develop a complete characterization of rational
maps of $\Re^n$ to \Sn{n-1}, the unit sphere in $\Re^n$.
These maps are useful in the design of rational curves on the sphere.
The main result of the paper is that
a map is a rational map of $\Re^n$ to \Sn{n-1}, $n \geq 2$, if and only if
it is of the form:
\[
	(x_1,\ldots,x_n) \mapsto 
	\pi (\frac{a_1^2 + \cdots + a_{n-1}^2 - a_n^2}{a_1^2 + \cdots + a_n^2},
	 \frac{2a_1a_n}{a_1^2 + \cdots + a_n^2},
	 \ldots,
	 \frac{2a_{n-1}a_n}{a_1^2 + \cdots + a_n^2})
\]
where $a_1,\ldots,a_n \in \Re[x_1,\ldots,x_n]$
and $\pi: \{1,\ldots,n\} \rightarrow \{1,\ldots,n\}$ is a permutation.
Euler's Four Squares Theorem is central to this result.
\end{abstract}

\vspace{1in}

\noindent Keywords: 
Riemannian modeling, curves on surfaces, rational maps, 
Euler's Four Squares Theorem, Pythagorean tuples.

\clearpage

\section{Introduction}

The design of curves on surfaces is important in several problems,
such as quaternion splines \cite{barr92,jj95},
trim curves \cite{hoschek89}, and
shortest path planning \cite{lozanoperez83,kim88}.
% \footnote{Path planning 
% 	of a multiple-degree-of-freedom object amongst obstacles in 3-space
% 	can be modeled using configuration space (C-space) obstacles.
% 	These C-space obstacles are surfaces in high
% 	dimensions---often 6 dimensions or more.
% 	Shortest paths involve curves on the C-space obstacles.}
Curve design on surfaces involves modeling in Riemannian space,
as opposed to the conventional modeling of curves in Euclidean space.
Surfaces are Riemannian spaces and Riemannian geometry,
the geometry in a Riemannian space, is significantly different from
Euclidean geometry \cite{kreyszig59}.
Consequently, Riemannian modeling poses several interesting challenges.
% modeling on a surface rather than in free space

A common solution to modeling on a surface is to model in the parameter
space of the surface.
For example, this is the typical solution for the design of trim curves.
This reduces a Riemannian problem to an Euclidean problem.
Consider the surface $A \subset \Re^n$ with parameterization
$\rho:\Re^{n-1} \rightarrow A$.
A curve in parameter space is mapped back to a curve on the 
surface $A$ using the surface parameterization $\rho$.
We shall call this method the {\bf parameter-space} approach to Riemannian
modeling.

Another way of reducing curve design on the surface $A$
to a simpler Euclidean problem is to model in the native Euclidean space
of $A$, $\Re^n$, through the design of a map $f: \Re^n \rightarrow A$.
If $C$ is a curve in $\Re^{n}$, then $f(C)$ is a curve on $A$.
$f(C)$ can be made to interpolate a set of points $p_i$ 
(or higher derivatives)
on $A$ by designing $C$ to interpolate the points $f^{-1}(p_i)$.
We shall call this method the {\bf native-space} approach to Riemannian 
modeling.

Rational curves are the preferred curves in current modeling systems,
due to their simplicity and elegance.
Consequently, we often want to design {\bf rational} maps 
$f: \Re^n \rightarrow A$ in the native-space approach to Riemannian modeling,
so that rational curves can be designed on the surface simply
by designing rational curves in the native Euclidean space.
Notice that $f^{-1}$ need {\bf not} be rational.

The native-space approach allows some added freedom in the design of the curve.
In general, the inverse image $f^{-1}(p)$ of a point on $A$ is a curve,
while $\rho^{-1}(p)$ is typically a point.
Consequently, the design of a curve on $A$ interpolating the points $\{p_i\}$
with the native-space approach
only requires the design of a curve interpolating the curves
$\{f^{-1}(p_i)\}$, as opposed to a curve interpolating
the points $\{\rho^{-1}(p_i)$.
The added freedom in the native-space approach can allow the design
of more optimal curves.
This is illustrated in \cite{dietz93}. (Where is the optimality?)
A second advantage of the native-space approach is more freedom
in the choice of map back to the surface.
Not being limited to the surface parameterization,
the native-space approach can use any rational map of $\Re^n$ to the surface.
This freedom can lead to superior curves on the surface, as illustrated
in \cite{jj+jimbo98}.

Thus, the abstract problem of the design of rational maps 
of $\Re^{n}$ to a surface $A$ is of considerable interest for the 
design of rational curves on $A$ and the general problem of
Riemannian modeling.

Curve design on \Sn{n-1}, the unit sphere in $n$-space,
is an elegant instance of the design of curves on a surface.
It offers an excellent example of the problems and possible approaches
to Riemannian modeling on the simplest and purest surface, 
the sphere.\footnote{The plane is simpler than the sphere,
	but design on the plane is still Euclidean modeling, since
	the plane is equivalent to 2-space.}
Curve design on \Sn{n-1}\ also has special interest because
of the important computer animation problem of 
quaternion spline design for orientation control \cite{barr92,jj95,kim95},
which is nothing but the design of curves on \Sn{3}.
Consequently, the design of rational maps of $\Re^n$ to \Sn{n-1}\ 
is of interest so that the native-space approach can be applied to this
orientation control problem.

In this paper, we develop a complete characterization of rational
maps of $\Re^{n}$ to \Sn{n-1}.
The paper is structured as follows.
Basic definitions are introduced in Section~\ref{sec:defn}.
The relationship between rational maps of $\Re^{n}$ to \Sn{n-1}\
and Pythagorean tuples of polynomials is established in Section~\ref{sec:setup}.
Sections~\ref{sec:n2}-\ref{sec:nn} characterize rational maps 
of $\Re^{n}$ to \Sn{n-1}\ for $n=2,3,4$ and general $n$, respectively.
The cases $n=2,3$ are well understood, but the case $n=4$ 
in Section~\ref{sec:n4} requires new observations 
that are generalized to all $n \geq 2$ in Section~\ref{sec:nn}.
Notice that the case $n=4$, rational maps from $\Re^4$ to \Sn{3},
is also the crucial case for quaternion splines.
Section~\ref{sec:commentary} reviews the results
and the paper ends with some conclusions.
The appendix contains a full proof of Kubota's characterization of Pythagorean
triples, an important result for the case $n=2$.

\section{Definitions}
\label{sec:defn}

\begin{defn2}
\Sn{n-1}\ is the unit sphere in $n$-space $x_1^2 + \ldots + x_{n}^2 - 1 = 0$.
The superscript refers to the dimension of the manifold.
\end{defn2}

\noindent Thus, \Sn{1}\ is a circle and \Sn{2}\ is the canonical sphere in 3-space.

\begin{defn2}
A {\bf rational polynomial} is a quotient of polynomials.
A map 
$(x_1,\ldots,x_n) \mapsto (f_1 (x_1,\ldots,x_n),\ldots,f_m (x_1,\ldots,x_n))$
is {\bf rational} if 
its components $f_i$ are all rational polynomials in $x_1,\ldots,x_n$.
\end{defn2}

\begin{defn2}
$\Re[x_1,\ldots,x_n]$ is the ring of polynomials in the $n$ variables
$x_1,\ldots,x_n$ with real coefficients.
\end{defn2}

\begin{defn2}
A {\bf commutative ring} is a ring in which multiplication is commutative.
\end{defn2}
%
Examples of commutative rings are the integers, the rationals, the reals,
and polynomials in $n$ variables over the integers, rationals, or reals.
Commutative rings subsume all of the domains of interest in this paper.

\begin{defn2}
Let $K$ be a commutative ring.
An $(n+1)$-tuple $(a_1,\ldots,a_{n+1}) \in K^{n+1}$
is a {\bf Pythagorean $(n+1)$-tuple over $K$} 
if $a_1^2 + \ldots + a_n^2 = a_{n+1}^2$.
\end{defn2}
%
This terminology is derived from Pythagorus' theorem on the lengths
of the sides of a right triangle, which of course generate
a Pythagorean triple.
We record the following obvious result for future reference.

\begin{lemma}
\label{lem:perm}
If $(a_1,\ldots,a_{n+1})$ is a Pythagorean $(n+1)$-tuple over $K$,
$(a_{\pi(1)},\ldots,a_{\pi(n)},a_{n+1})$ is also a Pythagorean 
$(n+1)$-tuple over $K$ for any permutation 
$\pi : \{1,\ldots,n\} \rightarrow \{1,\ldots,n\}$.
\end{lemma}


\section{Rational maps from $\Re^{n}$ to \Sn{n-1}}
\label{sec:setup}

% Rational maps to the sphere are attractive 
% because of their tractability.
% Maps to \Sn{n-1}\ (the unit sphere in $n$-space)
% whose domain is all of $n$-space are also attractive,
% because any point can be mapped to the sphere
% and the dimension of the space is preserved in the map.
% In this section, we study rational maps from $\Re^{n}$ to \Sn{n-1}.

A rational map from $\Re^n$ to \Sn{n-1}\ can be expressed as:
\[
	(x_1,\ldots,x_n) \mapsto
	(\frac{f_1(x_1,\ldots,x_n)}{f_{n+1}(x_1,\ldots,x_n)}, \ldots,
	 \frac{f_n(x_1,\ldots,x_n)}{f_{n+1}(x_1,\ldots,x_n)})
\]
where $f_1,\ldots,f_{n+1}$ are polynomials.
Since the image lies on \Sn{n-1},
\[
	f_1^2 + \cdots + f_n^2 = f_{n+1}^2
\]
Thus, $(f_1,\ldots,f_{n+1})$ is a Pythagorean $(n+1)$-tuple of polynomials in
$\Re[x_1,\ldots,x_n]$.
This reduces the study of rational maps from $\Re^{n}$ to \Sn{n-1}\ 
to the study of Pythagorean $(n+1)$-tuples of polynomials in 
$\Re[x_1,\ldots,x_n]$.

\begin{lemma}
\label{lem:iffpyth}
A rational map 
$(x_1,\ldots,x_n) \mapsto (\frac{f_1}{f_{n+1}},\ldots,\frac{f_n}{f_{n+1}})$,
where $f_1,\ldots,f_{n+1} \in \Re[x_1,\ldots,x_n]$, is a map from $\Re^n$
to \Sn{n-1}\ if and only if $(f_1,\ldots,f_{n+1})$ is a Pythagorean
$(n+1)$-tuple of polynomials in $\Re[x_1,\ldots,x_n]$.
\end{lemma}

\section{$\Re^{2}$ to \Sn{1}}
\label{sec:n2}

Consider rational maps from $\Re^2$ to \Sn{1},
and thus Pythagorean triples of polynomials in $\Re[x_1,x_2]$.
It is easily seen that 
\[ (a_1^2 - a_2^2,\ 2a_1a_2,\ a_1^2 + a_2^2)
\]
is a Pythagorean triple
of integers for any choice of integers $a_1$ and $a_2$.
This condition is also necessary, up to a constant factor.
% p. 89, Ebbinghaus, Numbers
%
\begin{lemma}[Classical]
\label{lem:classical}
A triple of integers is a Pythagorean triple if and only if
it is of the form $\alpha(a_1^2 - a_2^2, 2a_1a_2, a_1^2 + a_2^2)$ 
for some integers $a_1,a_2,\alpha$.\footnote{Whenever
	we give a normal form for a Pythagorean $(n+1)$-tuple,
	it is understood that the first $n$ elements are free to undergo
	a permutation.}
\end{lemma}
%
\cite{kubota72} generalized this result
to Pythagorean triples of polynomials in $\Re[x_1,x_2]$.
% where $K$ is a field of characteristic $p \neq 2$.
%
\begin{lemma}[Kubota 1972]
\label{lem:kubota}
A triple of polynomials in $\Re[x_1,x_2]$ is a Pythagorean triple 
if and only if it is of the form 
\begin{equation}
\label{eq:triple}
	\alpha(a_1^2 - a_2^2, 2a_1a_2, a_1^2 + a_2^2)
\end{equation}
for some $a_1,a_2,\alpha \in \Re[x_1,x_2]$.
\end{lemma}
\prf
See the appendix.
\QED
\Comment{
\prf
(If):  A triple of the form (\ref{eq:triple}) is clearly a Pythagorean triple:
\[
	(a_1^2 - a_2^2)^2 + (2a_1a_2)^2 = 
	a_1^4 + a_2^4 + 2a_1^2a_2^2 = (a_1^2 + a_2^2)^2
\]

(Only if): We present Kubota's proof for completeness and reference.
His proof holds for a more general class: Pythagorean triples
over $D$, where $D$ is a unique factorization domain 
not of characteristic 2, such that 2 is either prime 
or invertible in $D$.
$\Re[x_1,x_2]$ is such a unique factorization domain.
${\cal Z}$ is also such a domain, so this proof also functions
as a proof of Lemma~\ref{lem:classical}.

Let $(p_1,p_2,p_3)$ be a Pythagorean triple over $D$.
If $p_1=p_3$, let $(a_1,a_2,\alpha) = (1,0,p_1)$.
Thus, we can assume $p_1 \neq p_3$.
Factor $p_3 - p_1$ into a square component and a squarefree component:
$p_3 - p_1 = gh^2$, where $g,h \in D$ and $g$ and $h$ are both
squarefree.

If $2 \ | \ g$, let $(a_1,a_2,\alpha) = (\frac{hp_2}{p_3 - p_1}, h, \frac{g}{2}$).
Then $a_1,a_2,\alpha$ generate the Pythagorean triple $(p_1,p_2,p_3)$
as in (\ref{eq:triple}).
In particular,
\[
\alpha (a_1^2 - a_2^2)
= \frac{g}{2} 
  (\frac{h^2p_2^2 - h^2(p_3 - p_1)^2}{(p_3-p_1)^2})
\]
and applying $p_1^2 + p_2^2 = p_3^2$ and $p_3 - p_1 = gh^2$,
\[
= \frac{1}{2} (\frac{-2p_1^2 + 2p_1p_3}{p_3 - p_1}) = p_1.
\]
And
\[
\alpha (2a_1 a_2) = gh^2 \frac{p_2}{p_3 - p_1} = p_2
\]
Finally, 
\[
\alpha(a_1^2 + a_2^2)
= \frac{g}{2} 
  (\frac{h^2p_2^2 + h^2(p_3 - p_1)^2}{(p_3-p_1)^2})
= \frac{2p_3^2 - 2p_1p_3}{2(p_3-p_1)} = p_3
\]
Thus, $\alpha (a_1^2 - a_2^2, 2a_1a_2, a_1^2 + a_2^2) = (p_1,p_2,p_3)$.
Since $ga_1^2 = 2 \alpha a_1^2 = p_1 + p_3 \in D$, 
$a_1 \in D$ by Lemma~\ref{lem:applem1} of the appendix.
Clearly $a_2,\alpha \in D$.

If $2 \not | \ g$, let $(a_1,a_2,\alpha) = 
((\frac{hp_2}{p_3 - p_1} + h)/2, \ 
 (\frac{hp_2}{p_3 - p_1} - h)/2, \ g)$.
Then 
\[
\alpha (a_1^2 - a_2^2) = 2g\frac{h^2p_2}{p_3-p_1} = p_2
\]
\[
\alpha (2a_1 a_2) = \frac{g}{2} (\frac{h^2p_2^2}{(p_3-p_1)^2} - h^2) = p_1
\]
\[
\alpha(a_1^2 + a_2^2)
= \frac{gh^2}{4} \frac{2p_2^2 + 2(p_3 - p_1)^2}{(p_3-p_1)^2} = p_3
\]
Thus, $\alpha (a_1^2 - a_2^2, 2a_1a_2, a_1^2 + a_2^2) = (p_2,p_1,p_3)$.
Since $2 \alpha a_1^2 = p_2 + p_3 \in D$ and $2 \alpha$ is squarefree
($2 \not | \ g$), $a_1 \in D$ by Lemma~\ref{lem:applem1}.
Similarly, since $2 \alpha a_2^2 = p_3 - p_2 \in D$,
$a_2 \in D$ by Lemma~\ref{lem:applem1}.
Clearly $\alpha \in D$.
\QED
}

\noindent This yields 
a full characterization of rational maps from $\Re^2$ to \Sn{1}.
%
\begin{theorem}
\label{thm:n2}
A map is a rational map of $\Re^2$ to \Sn{1}\ if and only if 
it is of the form:
\begin{equation}
\label{eq:rationaltoS1}
	(x_{\pi(1)},x_{\pi(2)}) \mapsto (\frac{a_1^2 - a_2^2}{a_1^2 + a_2^2}, 
			   \frac{2a_1 a_2}{a_1^2 + a_2^2})
\end{equation}
where $a_1,a_2 \in \Re[x_1,x_2]$ and $\pi : \{1,2\} \rightarrow \{1,2\}$
is a permutation.
\end{theorem}
\prf
By Lemmas~\ref{lem:perm}, \ref{lem:iffpyth} and \ref{lem:kubota}.
\QED
% (Only if): Let $(x_1,x_2) \mapsto (\frac{f_1}{f_3}, \frac{f_2}{f_3})$,
% $f_1,f_2,f_3 \in \Re[x_1,x_2]$, be a rational map of $\Re^2$ to \Sn{1}.
% Then $(f_1,f_2,f_3)$ is a Pythagorean triple (Lemma~\ref{lem:iffpyth}).
% By Lemmas~\ref{lem:kubota} and \ref{lem:perm}, 
% $(f_{\pi(1)}, f_{\pi(2)}, f_3) = \alpha(a_1^2 - a_2^2, 2a_1a_2, a_1^2 + a_2^2)$ 
% for some $a_1,a_2,\alpha \in \Re[x_1,x_2]$ and permutation 
% $\pi : \{1,2\} \rightarrow \{1,2\}$.
% We conclude that the rational map is of the form (\ref{eq:rationaltoS1}).

% (If): Consider a map of the form (\ref{eq:rationaltoS1}), call it $M$.
% Let $f_1 = a_1^2 - a_2^2$, $f_2 = 2a_1 a_2$, $f_3 = a_1^2 + a_2^2$.
% By Lemma~\ref{lem:kubota}, $(f_1,f_2,f_3)$ is a Pythagorean triple
% over $\Re[x_1,x_2]$.
% By Lemma~\ref{lem:perm}, $(f_{\pi(1)}, f_{\pi(2)}, f_3)$ is also a Pythagorean
% triple.
% By Lemma~\ref{lem:iffpyth}, the map 
% $(x_1,x_2) \mapsto (\frac{f_{\pi(1)}}{f_3}, \frac{f_{\pi(2)}}{f_3})$
% is a rational map of $\Re^2$ to \Sn{1}.
% This map is equivalent to 
% $(x_{\pi(1)},x_{\pi(2)}) \mapsto (\frac{f_1}{f_3}, \frac{f_2}{f_3})$,
% which is our given map $M$.
% Thus, $M$ is a rational map of $\Re^2$ to \Sn{1}.
% \QED

\begin{example}
With $(a_1,a_2) = (1,x_2)$ and the identity permutation,
we have the familiar parameterization of the circle
$(x_1,x_2) \mapsto (\frac{1 - x_2^2}{1 + x_2^2}, 
				 \frac{2x_2}{1 + x_2^2})$.
\end{example}

\begin{example}
With $(a_1,a_2) = (x_1,x_2)$ and the identity permutation,
$(x_1,x_2) \mapsto (\frac{x_1^2 - x_2^2}{x_1^2 + x_2^2}, 
				 \frac{2x_1 x_2}{x_1^2 + x_2^2})$
is a rational map $\Re^2 \rightarrow \Sn{1}$.
On \Sn{1}, $x_1^2 + x_2^2 = 1$, so the restriction of this map
to \Sn{1}\ is $(x_1,x_2) \mapsto (x_1^2 - x_2^2, 2x_1x_2)$.
This map from \Sn{1}\ to \Sn{1}\ is
the simplest nontrivial quadratic spherical map \cite{ono94}. % [pp. 169,173]
\end{example}

\begin{example}
\cite{farouki90} uses Kubota's result to define Pythagorean hodograph curves, 
which have attractive arc length and offset properties.
The hodographs of these curves are components of a Pythagorean triple
of polynomials.
% which among other things have polynomial arc length (in the curve parameter).
\end{example}

\vspace{.5in}

\section{$\Re^{3}$ to \Sn{2}}
\label{sec:n3}

Consider rational maps from $\Re^3$ to \Sn{2},
and thus Pythagorean quadruples of polynomials in \linebreak $\Re[x_1,x_2,x_3]$.
Euler knew that 
\[ (a_1^2+a_2^2-a_3^2-a_4^2,\ 2a_1a_3+2a_2a_4,\ 2a_1a_4-2a_2a_3,
    \ a_1^2+a_2^2+a_3^2+a_4^2)
\]
is a Pythagorean quadruple for any choice of integers $a_1,a_2,a_3,a_4$
(see Lemma~\ref{lem:corEuler4square} below).
\linebreak
\cite{catalan85} observed that this is also a necessary condition.
%
\begin{lemma}[Catalan 1885]
\label{lem:catalan}
A quadruple of integers is a Pythagorean quadruple
if and only if it is of the form
$\alpha (a_1^2+a_2^2-a_3^2-a_4^2,\ 2a_1a_3+2a_2a_4,\ 2a_1a_4-2a_2a_3,
    \ a_1^2+a_2^2+a_3^2+a_4^2)$
for some integers $a_1,a_2,a_3,a_4,\alpha$.
\end{lemma}
\cite{dietz93} generalized this result to Pythagorean quadruples of
polynomials in $\Re[x_1]$, $\Re[x_1,x_2]$ or
$\Re[x_1,x_2,x_3]/<x_1+x_2+x_3-1>$.
We actually need a generalization to $\Re[x_1,x_2,x_3]$,
which would yield a full characterization of rational maps from 
$\Re^3$ to \Sn{2} (Theorem~\ref{thm:n3}).
We do not present a proof of this result, since we will give a more
powerful version of the following theorem (see Theorem~\ref{thm:nn} 
and the commentary after it).

\Comment{
\begin{lemma}
\label{lem:dietz}
A quadruple of polynomials in $\Re[x_1,x_2,x_3]$
is a Pythagorean quadruple if and only if it is of the form
$\alpha (a_1^2+a_2^2-a_3^2-a_4^2,\ 2a_1a_3+2a_2a_4,\ 2a_1a_4-2a_2a_3,
    \ a_1^2+a_2^2+a_3^2+a_4^2)$
for some polynomials $a_1,a_2,a_3,a_4,\alpha$ in $\Re[x_1,x_2,x_3]$.
\end{lemma}
}

\begin{theorem}
\label{thm:n3}
A map is a rational map of $\Re^3$ to \Sn{2}\ if and only if
it is of the form:
\[
(x_{\pi(1)},x_{\pi(2)},x_{\pi(3)}) \mapsto (\frac{a_1^2 + a_2^2 - a_3^2 - a_4^2}{a_1^2 + a_2^2 + a_3^2 + a_4^2},
		       \frac{2a_1a_3 + 2a_2a_4}{a_1^2 + a_2^2 + a_3^2 + a_4^2},
		       \frac{2a_1a_4 - 2a_2a_3}{a_1^2 + a_2^2 + a_3^2 + a_4^2})
\]
where $a_1,a_2,a_3,a_4 \in \Re[x_1,x_2,x_3]$
and $\pi : \{1,2,3\} \rightarrow \{1,2,3\}$ is a permutation.
\end{theorem}
%
\begin{example}
With $(a_1,a_2,a_3,a_4) = (x_1,x_2,x_3,1)$
and the permutation $(1,2,3) \mapsto (3,2,1)$,
\[
(x_1,x_2,x_3) \mapsto 
(\frac{x_1^2 + x_2^2 - x_3^2 - 1}{x_1^2 + x_2^2 + x_3^2 + 1},
 \frac{2x_1x_3 + 2x_2}{x_1^2 + x_2^2 + x_3^2 + 1},
 \frac{2x_1 - 2x_2x_3}{x_1^2 + x_2^2 + x_3^2 + 1})
\]
is a rational map $\Re^3 \rightarrow \Sn{2}$.
This map is used in \cite{dietz93} 
for the design of curves on \Sn{2}.
\end{example}
%
\begin{example}
\label{eg:foreshadow}
With $(a_1,a_2,a_3,a_4)=(x_2,x_1,0,x_3)$ and the identity permutation,
\[
(x_1,x_2,x_3) \mapsto 
(\frac{x_1^2 + x_2^2 - x_3^2}{x_1^2 + x_2^2 + x_3^2},
 \frac{2x_1x_3}{x_1^2 + x_2^2 + x_3^2},
 \frac{2x_2x_3}{x_1^2 + x_2^2 + x_3^2})
\]
is a rational map $\Re^3 \rightarrow \Sn{2}$.
\end{example}

\vspace{.5in}

\section{$\Re^{4}$ to \Sn{3}}
\label{sec:n4}

Consider rational maps from $\Re^4$ to \Sn{3}, and thus
Pythagorean quintuples of polynomials in \linebreak $\Re[x_1,x_2,x_3,x_4]$.
$(a_1,a_2,a_3,a_4,a_5)$ is a Pythagorean quintuple if
$a_1^2 + a_2^2 + a_3^2 + a_4^2 = a_5^2$.
Pythagorean quintuples are not as well understood as Pythagorean triples
and quadruples, and our study will lead to some new characterizations
(Theorems~\ref{thm:necessary4} and \ref{thm:map4}).
We are not aware of any specific study of Pythagorean quintuples.
However, in the number theory literature, there is an extensive study of the
sum of four squares, which is related to Pythagorean quintuples.
This study was driven by the search for a proof that
every positive integer is the sum of the squares of four integers.
This famous result was apparently known by Diophantus in the third century,
since he assumes it implicitly in his writings \cite{dickson52}. % p. 275
Fermat (in characteristic fashion!) claimed that he had a proof.\footnote{"I can
	not give the proof here, which depends upon numerous and abstruse
	mysteries of numbers; for I intend to devote an entire book
	to this subject" \cite[p. 6]{dickson52}.}
However, the first published proof was by Lagrange in 1770 \cite{herstein75}, % p. 375
over a century later,
during which time many mathematicians worked on the problem.
One of these was Euler, who established the following important result in 1748
\cite{herstein75}.   % p. 209 of Ebbing, p. 373 of Herstein
	% p. 210, Ebbinghaus for any commutative ring and Gauss' proof
	% another bulky version in \cite[p. 277]{dickson52}.

\begin{lemma}[Euler's Four Squares Theorem]
\[
\begin{array}{ll}
& (a_1^2 + a_2^2 + a_3^2 + a_4^2) 
(\hat{a}^2_1 + \hat{a}^2_2 + \hat{a}^2_3 + \hat{a}^2_4) = \\
& (a_1 \hat{a}_1 - a_2\hat{a}_2 - a_3\hat{a}_3 - a_4\hat{a}_4)^2 +
   (a_1\hat{a}_2 + a_2\hat{a}_1 + a_3\hat{a}_4 - a_4\hat{a}_3)^2 + \\
& (a_1\hat{a}_3 - a_2\hat{a}_4 + a_3\hat{a}_1 + a_4\hat{a}_2)^2 +
   (a_1\hat{a}_4 + a_2\hat{a}_3 - a_3\hat{a}_2 + a_4\hat{a}_1)^2
\end{array}
\]
where $a_1,a_2,a_3,a_4,\hat{a}_1,\hat{a}_2,\hat{a}_3,\hat{a}_4$ are elements of a
commutative ring.
% \footnote{The original statement was for integers, but it easily generalizes.}
% see p. 210 of Ebbinghaus
\end{lemma}

This result shows that the product of a sum of four squares and a sum of four
squares is another sum of four squares,
which reduces the problem of showing that every integer is the sum
of four squares to the simpler problem
of showing that every prime is the sum of four squares,
since every integer can be expressed as the product of primes.
For our purposes, the most interesting aspect of 
the Four Squares Theorem is its specialization to 
two simpler formulae, which are recipes for
generating Pythagorean quadruples and quintuples.

\begin{lemma}[Euler]
\label{lem:corEuler4square}
\ \ \\
\begin{equation}
\label{eq:euler1}
(a_1^2 + a_2^2 + a_3^2 + a_4^2)^2 = 
(a_1^2 + a_2^2 - a_3^2 - a_4^2)^2 + (2a_1a_3+2a_2a_4)^2 + (2a_1a_4-2a_2a_3)^2
\end{equation}
%
\begin{equation}
\label{eq:aida}
(a_1^2 + a_2^2 + a_3^2 + a_4^2)^2 = 
(a_1^2 + a_2^2 + a_3^2 - a_4^2)^2 + (2a_1a_4)^2 + (2a_2a_4)^2 + (2a_3a_4)^2
\end{equation}
where $a_1,a_2,a_3,a_4$ are elements of any commutative ring.
\end{lemma}
\prf
In the Four Squares Theorem, let 
$(a_1,a_2,a_3,a_4) = (\hat{a}_1,-\hat{a}_2,\hat{a}_3,\hat{a}_4)$
for (\ref{eq:euler1}) and \\
$(\hat{a}_1,\hat{a}_2,\hat{a}_3,\hat{a}_4) = (a_1,-a_2,-a_3,a_4)$ 
for (\ref{eq:aida}).
Other special cases can be found by assigning different signs.
\QED

We have already seen the use of the first formula as a necessary and
sufficient condition for Pythagorean quadruples (Lemma~\ref{lem:catalan}).
The second formula provides a sufficient condition for Pythagorean quintuples.

\begin{lemma}
\label{lem:suff4}
Let $D$ be the integers or the polynomials over $\Re[x_1,x_2,x_3,x_4]$.
\begin{equation}
\label{eq:suff4}
	(a_1^2 + a_2^2 + a_3^2 - a_4^2,\ 2a_1a_4,\ 2a_2a_4,\ 2a_3a_4,\ 
	 a_1^2 + a_2^2 + a_3^2 + a_4^2)
\end{equation}
is a Pythagorean quintuple for any $a_1,a_2,a_3,a_4 \in D$.
\end{lemma}

We want to show that the form (\ref{eq:suff4}) is also a necessary condition,
up to a constant factor.

\begin{rmk}
It is tempting to attempt an extension of Kubota's proof technique 
for Pythagorean triples to prove a stronger version of Lemma~\ref{lem:suff4}:
namely, that a quintuple over $D$ is a Pythagorean quintuple over $D$ if and
only if it is of the form  
\begin{equation}
\label{eq:alphasuff4}
	\alpha (a_1^2 + a_2^2 + a_3^2 - a_4^2,\ 2a_1a_4,\ 2a_2a_4,\ 2a_3a_4,\ 
	 a_1^2 + a_2^2 + a_3^2 + a_4^2)
\end{equation}
for some $a_1,a_2,a_3,a_4,\alpha \in D$.
However, this extension breaks down.
Consider the proof of Kubota's result in the appendix.
Given a Pythagorean quintuple $(p_1,p_2,p_3,p_4,p_5)$ over $D$,
let $p_5 - p_1 = gh^2$ where $g$ and $h$ are squarefree and assume
$2 \ | \ g$ as in the first case of the proof.
We can generate the Pythagorean quintuple 
as in (\ref{eq:alphasuff4}) using $(a_1,a_2,a_3,a_4,\alpha) = 
(\frac{hp_2}{p_5 - p_1}, \frac{hp_3}{p_5 - p_1}, \frac{hp_4}{p_5 - p_1}, 
h, \frac{g}{2}$),
a direct generalization of the $a_i$ used by Kubota.
However, we cannot show that $a_1,a_2,a_3 \in D$:
the trick of using Lemma~\ref{lem:applem1} does not work, 
since $p_1 + p_5 = 2\alpha(a_1^2 + a_2^2 + a_3^2)$,
which is not in the form $ab^2$.
Without $a_i \in D$, (\ref{eq:alphasuff4}) is a useless normal form
since it would also generate Pythagorean quintuples not in $D$.
\end{rmk}
\Comment{
KUBOTA'S PROOF TECHNIQUE BREAKS DOWN:
CANNOT SHOW THAT A_I \in D.
\begin{theorem}
Let $D$ be the integers or the polynomials over $\Re[x_1,x_2,x_3,x_4]$.
A quintuple over $D$ is a Pythagorean quintuple over $D$ if and only if
it is of the form:
\begin{equation}
\label{eq:pythorig}
	\alpha (a_1^2 + a_2^2 + a_3^2 - a_4^2,
		\ 2a_1a_4,\ 2a_2a_4,\ 2a_3a_4,
		\ a_1^2 + a_2^2 + a_3^2 + a_4^2)
\end{equation}
for some $a_1,a_2,a_3,a_4,\alpha \in D$.
\end{theorem}
\prf
(If): Lemma~\ref{lem:corEuler4square}.

(Only if): This proof holds for a more general class: Pythagorean quintuples
over $D$, where $D$ is a unique factorization domain 
not of characteristic 2, such that 2 is either prime 
or invertible in $D$.
$\Re[x_1,x_2,x_3,x_4]$ is such a unique factorization domain,
as is ${\cal Z}$.

Let $(p_1,p_2,p_3,p_4,p_5)$ be a Pythagorean quintuple over $D$.
If $p_1=p_5$, let $(a_1,a_2,a_3,a_4,\alpha) = (1,0,0,0,p_1)$.
Thus, we can assume $p_1 \neq p_5$.
Factor $p_5 - p_1$ into a square component and a squarefree component:
$p_5 - p_1 = gh^2$, where $g,h \in D$ and $g$ and $h$ are both
squarefree.

If $2 \ | \ g$, let $(a_1,a_2,a_3,a_4,\alpha) = 
(\frac{hp_2}{p_5 - p_1}, \frac{hp_3}{p_5 - p_1}, \frac{hp_4}{p_5 - p_1}, 
h, \frac{g}{2}$).
Then $a_1,a_2,a_3,a_4,\alpha$ generate the Pythagorean quintuple 
$(p_1,\ldots,p_5)$ as in (\ref{eq:pythorig}).
In particular,
\[
\alpha (a_1^2 + a_2^2 + a_3^2 - a_4^2)
= \frac{g}{2} 
  (\frac{h^2p_2^2 + h^2p_3^2 + h^2p_4^2 - h^2(p_5 - p_1)^2}{(p_5-p_1)^2})
\]
and applying $p_1^2 + p_2^2 + p_3^2 + p_4^2 = p_5^2$ and $p_5 - p_1 = gh^2$,
\[
= \frac{1}{2} (\frac{-2p_1^2 + 2p_1p_5}{p_5 - p_1}) = p_1.
\]
And
\[
\alpha (2a_i a_4) = gh^2 \frac{p_{i+1}}{p_5 - p_1} = p_{i+1}
\]
for $i=1,2,3$.
Finally, 
\[
\alpha(a_1^2 + a_2^2 + a_3^2 + a_4^2)
= \frac{g}{2} 
  (\frac{h^2p_2^2 + h^2p_3^2 + h^2p_4^2 + h^2(p_5 - p_1)^2}{(p_5-p_1)^2})
= \frac{2p_5^2 - 2p_1p_5}{2(p_5-p_1)} = p_5
\]
Thus, $\alpha (a_1^2 + a_2^2 + a_3^2 - a_4^2, 2a_1a_4, 2a_2a_4, 2a_3a_4, 
a_1^2 + a_2^2 + a_3^2 + a_4^2) = (p_1,p_2,p_3,p_4,p_5)$.
Clearly $a_4,\alpha \in D$, but what about $a_1,a_2,a_3$?

NOT EXTENDED BEYOND HERE.
If $2 \not | \ g$, let $(a_1,a_2,\alpha) = 
((\frac{hp_2}{p_3 - p_1} + h)/2, \ 
 (\frac{hp_2}{p_3 - p_1} - h)/2, \ g)$.
Then 
\[
\alpha (a_1^2 - a_2^2) = 2g\frac{h^2p_2}{p_3-p_1} = p_2
\]
\[
\alpha (2a_1 a_2) = \frac{g}{2} (\frac{h^2p_2^2}{(p_3-p_1)^2} - h^2) = p_1
\]
\[
\alpha(a_1^2 + a_2^2)
= \frac{gh^2}{4} \frac{2p_2^2 + 2(p_3 - p_1)^2}{(p_3-p_1)^2} = p_3
\]
Thus, $\alpha (a_1^2 - a_2^2, 2a_1a_2, a_1^2 + a_2^2) = (p_2,p_1,p_3)$.
Since $2 \alpha a_1^2 = p_2 + p_3 \in D$ and $2 \alpha$ is squarefree
($2 \not | \ g$), $a_1 \in D$ by Lemma~\ref{lem:applem1}.
Similarly, since $2 \alpha a_2^2 = p_3 - p_2 \in D$,
$a_2 \in D$ by Lemma~\ref{lem:applem1}.
Clearly $\alpha \in D$.
\QED
}

Rather than building a necessary and sufficient condition together,
we will separate the necessary and sufficient conditions and
establish a slightly weaker necessary condition for Pythagorean quintuples.

\begin{theorem}
\label{thm:necessary4}
Let $D$ be the integers or the polynomials over $\Re[x_1,x_2,x_3,x_4]$.
	% NB: proof not valid for comm. ring, since comm. ring may not have unit elt 1, which is needed immediately
A Pythagorean quintuple over $D$ can be expressed in the form
\begin{equation}
\label{eq:pyth}
	\alpha (a_1^2 + a_2^2 + a_3^2 - a_4^2,
		\ 2a_1a_4,\ 2a_2a_4,\ 2a_3a_4,
		\ a_1^2 + a_2^2 + a_3^2 + a_4^2)
\end{equation}
for some $a_1,a_2,a_3,a_4,\frac{1}{\alpha} \in D$.
\end{theorem}
\prf
Let $(p_1,p_2,p_3,p_4,p_5)$ be a Pythagorean quintuple over $D$.
If $p_1 = p_5$, let $(a_1,a_2,a_3,a_4,\alpha) = (p_1,0,0,0,\frac{1}{p_1})$.
Assume $p_1 \neq p_5$.
Let
\[
(a_1,a_2,a_3,a_4,\alpha) = (p_2,p_3,p_4,p_5-p_1,\frac{1}{2(p_5 - p_1)})
\]
Then $a_1,a_2,a_3,a_4,\alpha$ generate the Pythagorean quintuple
$(p_1,\ldots,p_5)$ as in (\ref{eq:pyth}).
In particular,
\[
\alpha (a_1^2 + a_2^2 + a_3^2 - a_4^2)
= \frac{p_2^2 + p_3^2 + p_4^2 - (p_5 - p_1)^2}{2(p_5-p_1)}
\]
and applying $p_1^2 + p_2^2 + p_3^2 + p_4^2 = p_5^2$,
\[
= \frac{-2p_1^2 + 2p_1p_5}{2(p_5 - p_1)} = p_1.
\]
And
\[
\alpha (2a_i a_4) = \frac{2p_{i+1}(p_5 - p_1)}{2(p_5 - p_1)} = p_{i+1}
\]
for $i=1,2,3$.
Finally, 
\[
\alpha(a_1^2 + a_2^2 + a_3^2 + a_4^2) 
= \frac{p_2^2 + p_3^2 + p_4^2 + (p_5 - p_1)^2}{2(p_5-p_1)} = p_5
\]
Thus, $(p_1,p_2,p_3,p_4,p_5) = \alpha (a_1^2 + a_2^2 + a_3^2 - a_4^2,
		\ 2a_1a_4,\ 2a_2a_4,\ 2a_3a_4,
		\ a_1^2 + a_2^2 + a_3^2 + a_4^2)$.
\QED

Notice that (\ref{eq:pyth}) is no longer a sufficient condition for
Pythagorean quintuples, because in general $\alpha \not\in D$.
In particular, a quintuple in the form (\ref{eq:pyth}) is not necessarily
a quintuple over $D$.
Try $(a_1,a_2,a_3,a_4,\alpha) = (1,1,1,1,\frac{1}{x_1})$.
	
Combining Lemma~\ref{lem:suff4} (for sufficiency) 
and Theorem~\ref{thm:necessary4} (for necessity),
we have a full characterization
of the rational maps of $\Re^4$ to \Sn{3}.
The presence of $\alpha \not\in D$ in (\ref{eq:pyth}) does not 
affect the characterization, since $\alpha$ cancels.

\begin{theorem}
\label{thm:map4}
A map is a rational map of $\Re^4$ to \Sn{3}\ if and only if
it is of the form:
\begin{equation}
\label{eq:re4s3}
\footnotesize{(x_{\pi(1)},x_{\pi(2)},x_{\pi(3)},x_{\pi(4)})} \mapsto 
\footnotesize{(\frac{a_1^2 + a_2^2 + a_3^2 - a_4^2}{a_1^2 + a_2^2 + a_3^2 + a_4^2},
	 \frac{2a_1a_4}{a_1^2 + a_2^2 + a_3^2 + a_4^2},
	 \frac{2a_2a_4}{a_1^2 + a_2^2 + a_3^2 + a_4^2},
	 \frac{2a_3a_4}{a_1^2 + a_2^2 + a_3^2 + a_4^2})}
\end{equation}
where $a_1,a_2,a_3,a_4 \in \Re[x_1,x_2,x_3,x_4]$
and $\pi : \{1,2,3,4\} \rightarrow \{1,2,3,4\}$ is a permutation.
\end{theorem}
\prf
(Only if): Let $(x_1,x_2,x_3,x_4) \mapsto 
(\frac{f_1}{f_5},\frac{f_2}{f_5},\frac{f_3}{f_5},\frac{f_4}{f_5})$,
where $f_1,\ldots,f_5 \in \Re[x_1,x_2,x_3,x_4]$, be a rational map
of $\Re^4$ to \Sn{3}.
Then $(f_1,\ldots,f_5)$ is a Pythagorean quintuple (Lemma~\ref{lem:iffpyth}).
By Theorem~\ref{thm:necessary4} and Lemma~\ref{lem:perm},
$(f_{\pi(1)},\ldots,f_{\pi(4)},f_5) = \alpha (a_1^2 + a_2^2 + a_3^2 - a_4^2,
2a_1a_4,2a_2a_4,2a_3a_4,$ $a_1^2 + a_2^2 + a_3^2 + a_4^2)$
for some $a_1,a_2,a_3,a_4,\frac{1}{\alpha} \in \Re[x_1,x_2,x_3,x_4]$
and permutation $\pi:\{1,2,3,4\} \rightarrow \{1,2,3,4\}$.
Thus, $(x_{\pi(1)},x_{\pi(2)},x_{\pi(3)},x_{\pi(4)}) \mapsto$
\[
	(\frac{\alpha(a_1^2 + a_2^2 + a_3^2 - a_4^2)}{\alpha(a_1^2 + a_2^2 + a_3^2 + a_4^2)},
	 \frac{\alpha(2a_1a_4)}{\alpha(a_1^2 + a_2^2 + a_3^2 + a_4^2)},
	 \frac{\alpha(2a_2a_4)}{\alpha(a_1^2 + a_2^2 + a_3^2 + a_4^2)},
	 \frac{\alpha(2a_3a_4)}{\alpha(a_1^2 + a_2^2 + a_3^2 + a_4^2)})
\]
\[
= 	(\frac{a_1^2 + a_2^2 + a_3^2 - a_4^2}{a_1^2 + a_2^2 + a_3^2 + a_4^2},
	 \frac{2a_1a_4}{a_1^2 + a_2^2 + a_3^2 + a_4^2},
	 \frac{2a_2a_4}{a_1^2 + a_2^2 + a_3^2 + a_4^2},
	 \frac{2a_3a_4}{a_1^2 + a_2^2 + a_3^2 + a_4^2})
\]
(If): Consider a map of the form (\ref{eq:re4s3}).
This is a rational map of $\Re^4$ to \Sn{3}\ by (\ref{eq:aida})
and Lemma~\ref{lem:iffpyth}.
\QED

\begin{example}
\label{eg:mostnatural}
The most natural choice for $a_i$,
$a_i=x_i$, and the identity permutation,
yields the rational map
\[
(x_1,x_2,x_3,x_4) \mapsto 
	(\frac{x_1^2 + x_2^2 + x_3^2 - x_4^2}{x_1^2 + x_2^2 + x_3^2 + x_4^2},
	 \frac{2x_1x_4}{x_1^2 + x_2^2 + x_3^2 + x_4^2}, 
	 \frac{2x_2x_4}{x_1^2 + x_2^2 + x_3^2 + x_4^2}, 
	 \frac{2x_3x_4}{x_1^2 + x_2^2 + x_3^2 + x_4^2})
\]
of $\Re^4$ to \Sn{3}.
We have used this map for the design of rational curves on \Sn{3} \cite{jj95}.
Motivated by Theorem~\ref{thm:map4}, 
we consider this the most natural map from $\Re^4$ to \Sn{3}.
It is discussed further in \cite{jj98}.
\end{example}
\begin{example}
Although all of our examples have used simple polynomials for $a_i$
($a_i=1$ or $a_i=x_i$),
this is not necessary, of course.
For example, using $(a_1,a_2,a_3,a_4) = (0,2x_2,x_1+x_3,x_4-1)$
and the identity permutation,
\footnotesize{
\[
\frac{1}{x_1^2 + 4x_2^2 + x_3^2 + x_4^2 + 2x_1x_3 - 2x_4 + 1}
	(x_1^2 + 4x_2^2 + x_3^2 - x_4^2 + 2x_1x_3 + 2x_4 - 1,
	 0,
	 2x_2x_4-2x_2,
	 2(x_1+x_3)(x_4-1))
\]
}
\normalsize{is a rational map of $\Re^4$ to \Sn{3}.}
%
% With $(a_1,a_2,a_3,a_4) = (x_1,x_2^2,x_1+x_3,x_4-x_1)$,
% \[
%	(x_1^2 + x_2^4 + x_3^2 - x_4^2 + 2x_1x_3 + 2x_1x_4,\ 
%	 2x_1x_4-2x_1^2,\ 2x_2^2x_4-2x_1x_2^2,\ 
%	 2x_1x_4 - 2x_1x_3 - 2x_1^2 + 2x_3x_4)
% \]
% is a rational map of $\Re^4$ to \Sn{3}.
\end{example}

\vspace{.5in}

\section{$\Re^n$ to \Sn{n-1}}
\label{sec:nn}

The results on rational maps from $\Re^4$ to \Sn{3}\ can be generalized
to rational maps from $\Re^n$ to \Sn{n-1}, $n \geq 2$.
The specialization of Euler's Four Squares Theorem (\ref{eq:aida})
was the key component in the previous development,
since it offered a recipe for Pythagorean quintuples.
Fortunately, there is a generalization of this result by
Ammei (also apparently known by Euler) \cite{dickson52}. % p. 318
% (In fact, Euler also knew of this result.) % See p. 318 of Dickson.

\begin{lemma}[Ammei c. 1817]
\label{lem:ammei}
\begin{equation}
\label{eq:ammei}
	(a_1^2 + \cdots + a_n^2)^2 = 
	(a_1^2 + \cdots + a_{n-1}^2 - a_n^2)^2 + (2a_1a_n)^2 + \ldots + 
	(2a_{n-1}a_n)^2
\end{equation}
where $a_1,\ldots,a_n$ are elements of any commutative ring.
\end{lemma}

This is a sufficient condition for Pythagorean $(n+1)$-tuples.
We can again establish that a slightly weaker version is a necessary condition
for Pythagorean $(n+1)$-tuples.

\begin{theorem}
\label{thm:necessaryn}
Let $n \geq 2$ and $D$ be the integers or the polynomials 
over $\Re[x_1,\ldots,x_n]$.
A Pythagorean $(n+1)$-tuple over $D$ can be expressed in the form
\begin{equation}
\label{eq:pyth2}
	\alpha (a_1^2 + \ldots + a_{n-1}^2 - a_n^2,
		\ 2a_1a_n,\ldots,\ 2a_{n-1}a_n,
		\ a_1^2 + \ldots + a_n^2)
\end{equation}
for some $a_1,\ldots,a_n,\frac{1}{\alpha} \in D$.
\end{theorem}
\prf
Let $(p_1,\ldots,p_{n+1})$ be a Pythagorean $(n+1)$-tuple over $D$.
If $p_1 = p_{n+1}$, let 
$(a_1,\ldots,a_n,\alpha) = (p_1,0,\ldots,0,\frac{1}{p_1})$.
Assume $p_1 \neq p_{n+1}$.
Let 
\[
(a_1,\ldots,a_n,\alpha) = (p_2,\ldots,p_n,p_{n+1}-p_1,
	\frac{1}{2(p_{n+1}-p_1)})
\]
Then $a_1,\ldots,a_n,\alpha$ generate the Pythagorean $(n+1)$-tuple
$(p_1,\ldots,p_{n+1})$ as in (\ref{eq:pyth2}).
In particular,
\[
\alpha (a_1^2 + \ldots + a_{n-1}^2 - a_n^2)
= \frac{p_2^2 + \cdots + p_n^2 - (p_{n+1}-p_1)^2}{2(p_{n+1}-p_1)}
\]
and applying $p_1^2 + \ldots + p_n^2 = p_{n+1}^2$,
\[
= \frac{-2p_1^2 + 2p_1p_{n+1}}{2(p_{n+1} - p_1)}
= p_1
\]
And
\[
\alpha (2a_i a_n) 
= \frac{2p_{i+1}(p_{n+1}-p_1)}{2(p_{n+1}-p_1)}
= p_{i+1}
\]
for $i=1,\ldots,n-1$.
Finally, 
\[ 
\alpha (a_1^2 + \ldots + a_n^2)
= \frac{p_2^2 + \cdots + p_n^2 + (p_{n+1}-p_1)^2}{2(p_{n+1}-p_1)}
= p_{n+1}.
\]
\QED

This leads to our most general result: 
a full characterization of the rational maps of $\Re^n$ to \Sn{n-1}.

\begin{theorem}
\label{thm:nn}
A map is a rational map of $\Re^n$ to \Sn{n-1}, $n \geq 2$, if and only if
it is of the form:
\[
	(x_{\pi(1)},\ldots,x_{\pi(n)}) \mapsto 
	(\frac{a_1^2 + \cdots + a_{n-1}^2 - a_n^2}{a_1^2 + \cdots + a_n^2},
	 \frac{2a_1a_n}{a_1^2 + \cdots + a_n^2},
	 \ldots,
	 \frac{2a_{n-1}a_n}{a_1^2 + \cdots + a_n^2})
\]
where $a_1,\ldots,a_n \in \Re[x_1,\ldots,x_n]$
and $\pi: \{1,\ldots,n\} \rightarrow \{1,\ldots,n\}$ is a permutation.
\end{theorem}

\section{Commentary}
\label{sec:commentary}

Theorem~\ref{thm:nn} applies to all $n \geq 2$ and thus represents 
a solution to the cases $n=2,3,4$ already considered.
Since this theorem is a direct generalization of the $n=4$ case,
there is no new solution for $n=4$.
The solution for $n=2$ is also 
equivalent to the previous solution.
However, there is a new characterization for $n=3$:
a map is a rational map from $\Re^3$ to \Sn{2}\ if and only if it is of
the form $(x_1,x_2,x_3) \mapsto 
(\frac{a_1^2 + a_2^2 - a_3^2}{a_1^2 + a_2^2 + a_3^2}, 
 \frac{2a_1a_3}{a_1^2 + a_2^2 + a_3^2}, 
 \frac{2a_2a_3}{a_1^2 + a_2^2 + a_3^2})$
where $a_1,a_2,a_3 \in \Re[x_1,x_2,x_3]$.
This is not a contradiction: any rational map
from $\Re^3$ to \Sn{2}\ can be either expressed in the form of 
Theorem~\ref{thm:n3} or in the form of Theorem~\ref{thm:nn} with $n=3$.

It turns out that if we set 
$(a_1,a_2,a_3,a_4) = (\hat{a}_2, \hat{a}_1, 0, \hat{a}_3)$
in the characterization of Theorem~\ref{thm:n3},
the characterization of Theorem~\ref{thm:nn} for $n=3$ is produced
(see Example~\ref{eg:foreshadow}).
Thus, the characterization of Theorem~\ref{thm:nn} is stronger than
Theorem~\ref{thm:n3}: not only can a rational map from $\Re^3$ to \Sn{2}\ 
be put in the form of Theorem~\ref{thm:n3}, it can be put in the form
of Theorem~\ref{thm:n3} with $a_3=0$.
The polynomial $a_3$ of Theorem~\ref{thm:n3} is redundant.

We now have a characterization of all rational maps from $\Re^n$ to \Sn{n-1}.
The related findings on Pythagorean $(n+1)$-tuples are reviewed 
in Table~\ref{tab:pyth}.
The second column of this table gives the normal form for Pythagorean
$(n+1)$-tuples for various $n$
and the third (resp., fourth and fifth) column indicates when this
normal form was shown to be a sufficient (resp., necessary and necessary) 
condition for an $(n+1)$-tuple to be
a Pythagorean $(n+1)$-tuple over the integers
(resp., over the integers and polynomial rings).
The star $^*$ on the results from this paper refers to the fact that
$\frac{1}{\alpha} \in D$, not $\alpha \in D$ as in the other results.

\begin{table}[h]
\label{tab:pyth}
\begin{tabular}{|c|c|c|c|c|}
\hline
$n$ & normal form for Pythagorean $(n+1)$-tuples & suff {\cal Z} & nec {\cal Z} & 
\footnotesize{nec $\Re[x_1,\ldots,x_n]$} \\
\hline
$2$ & \tiny{$\alpha(a_1^2 - a_2^2, 2a_1a_2, a_1^2 + a_2^2)$} &
\footnotesize{classical} & 
\footnotesize{classical} & 
\footnotesize{Kubota 1972} \\ 
\hline
$3$ & \tiny{$\alpha (a_1^2+a_2^2-a_3^2-a_4^2, 2(a_1a_3+a_2a_4), 
    2(a_1a_4-a_2a_3), a_1^2+a_2^2+a_3^2+a_4^2)$} &
\footnotesize{Euler 1748} & 
\footnotesize{Catalan 1885} & 
\footnotesize{Dietz 1993} \\ 
\hline
$4$ & \tiny{$\alpha (a_1^2 + a_2^2 + a_3^2 - a_4^2,
		2a_1a_4, 2a_2a_4, 2a_3a_4,
		a_1^2 + a_2^2 + a_3^2 + a_4^2)$} &
\footnotesize{Euler 1748} & 
\footnotesize{$^*$Johnstone 1998} & 
\footnotesize{$^*$Johnstone 1998} \\
\hline
$n$ & \tiny{$\alpha (a_1^2 + \ldots + a_{n-1}^2 - a_n^2,
	2a_1a_n, \ldots, 2a_{n-1}a_n,
	a_1^2 + \ldots + a_n^2)$} & 
\footnotesize{Ammei 1817} & 
\footnotesize{$^*$Johnstone 1998} & 
\footnotesize{$^*$Johnstone 1998} \\ \hline
\end{tabular}
\end{table}

Notice that all of the characterizations of 
rational maps from $\Re^n$ to \Sn{n-1}
can be derived from Euler's
Four Squares Theorem, since the original characterizations of $n=3,4$
are derived from special cases of the Four Squares Theorem
and the general characterization for all $n$ is derived from a generalization
of one of these special cases.

% \section{Quadratic spherical maps}
% All of this taken from Ono, Chapter 5.
\Comment{
Quadratic spherical maps are a class of maps from the $m$-sphere to the
$n$-sphere % that have received some attention
(see \cite{ono94,ebbinghaus90}).	% Ono, Chapter 5; Ebbinghaus, p. 236
% especially in the context
% of Hopf maps, which are quadratic spherical maps with some additional properties

\begin{defn2}
A map $f:\Re^m \rightarrow \Re^n$ is {\bf quadratic} if 
\begin{quote}
1) $f(kx) = k^2 f(x)$ whenever $k \in \Re,\ x \in \Re^m$, and \\
2) the map $(x,y) \mapsto \frac{1}{2} [f(x+y) - f(x) - f(y)],\ x,y\in\Re^m$
is bilinear.
	% restricting K to \Re, X to \Re^m, and Y to \Re^n in Ono, p. 165
\end{quote}
\end{defn2}

\begin{defn2}
A quadratic map $f:\Re^m \rightarrow \Re^n$ is {\bf spherical} 
if $\| f(x) \| = \| x \|^2$ for all $x \in \Re^m$.
	% restricting quadratic forms q_x and q_y to squared Euclidean norms
	% x_1^2 + ... + x_m^2 (or x_n^2)
	% which strictly translates into \| f(x) \|^2 = \| x \|^4,
	% which is simplified to above form
\end{defn2}

If $f:\Re^{m} \rightarrow \Re^{n}$ is a quadratic spherical map, 
then $f:\Sn{m-1} \rightarrow \Sn{n-1}$.
Thus, quadratic spherical maps send the unit sphere into the unit sphere,
and hence their name.

Two other classical quadratic spherical maps are 
$f:\Sn{3} \rightarrow \Sn{2}$ defined by:	% \cite[pp. 171,174]{ono94}
\begin{equation}
\label{eqn:classicalHopf}
	(x_1,x_2,x_3,x_4) \mapsto 
	(x_1^2 + x_2^2 - x_3^2 - x_4^2,\ 
	 2(x_1 x_3 + x_2 x_4),\ 
	 2(x_1 x_4 + x_2 x_3))
\end{equation}
and $f:\Sn{n} \rightarrow \Sn{n}$ defined by: % \cite[p. 193]{ono94}
\begin{equation}
\label{eqn:HopfM}
	(w,x_1,\ldots,x_{n}) \mapsto
	(w^2 - x_1^2 - \cdots - x_{n}^2,\ 
	 2wx_1, \ldots,\ 2wx_{n})
\end{equation}
%
% All of these maps are also Hopf maps.
% (\ref{eqn:classicalHopf}) is called the classical Hopf map.
% Notice the similarity between (\ref{eqn:HopfM})
% and the projection version of stereographic injection 
% (\ref{eqn:stereoinjection}).

\begin{example}
The normal form of Theorem~\ref{thm:n3} is somewhat reminiscent of the quadratic
spherical map (\ref{eqn:classicalHopf}).
However, notice the difference in dimension of the domain (3 in Theorem~\ref{thm:n3}
and 4 in (\ref{eqn:classicalHopf}))
and the difference in sign of the last component.
\end{example}
}

\section{Conclusions}

The main result of the paper has been a characterization of
rational maps of $\Re^n$ to \Sn{n-1}, $n \geq 2$,
resulting in a normal form for all of these maps.
Rational maps of $\Re^n$ to \Sn{n-1}\ are inextricably linked
with Pythagorean $(n+1)$-tuples.
However, our result does not yet establish a normal form for all
Pythagorean $(n+1)$-tuples,
since we decoupled the necessary and sufficient conditions for
Pythagorean $(n+1)$-tuples,
proving a slightly weaker necessary condition.
This characterization of Pythagorean $(n+1)$-tuples
is tantalizingly close but still an open problem.
This paper has also revealed the strong link between Euler's
Four Square Theorem and Pythagorean $(n+1)$-tuples,
and hence rational maps of $\Re^n$ to \Sn{n-1}.

The most well-known map involving the sphere is stereographic projection.
In \cite{jj98}, we discuss the relationship of stereographic projection
to the normal form for rational maps of $\Re^n$ to \Sn{n-1},
as well as further exploring the most natural rational map 
of $\Re^n$ to \Sn{n-1}\ (Example~\ref{eg:mostnatural}).

The application of rational maps of $\Re^n$ to \Sn{n-1}\ to the
design of rational quaternion splines is explored in \cite{jj95,jj+jimbo98}.

\section{Appendix}

For completeness and reference,
we present Kubota's proof of his result on Pythagorean triples.
First, a technical lemma is needed.

\begin{lemma}
\label{lem:applem1}
Let $D$ be a unique factorization domain and $K$ its field of quotients.
Let $a \in D$, $b \in K$.
If $a$ is squarefree and $ab^2 \in D$, then $b \in D$.
\end{lemma}
\prf
Let $b = \frac{m}{n}$ where $m,n \in D$.
$a(\frac{m^2}{n^2}) \in D$ implies $n^2 \ | \ am^2$.
Let $n = n_1 \ldots n_k$ be the prime factorization of $n$,
so $n_1^2 \ldots n_k^2 \ | \ am^2$.
Since $n_i^2 \not | \ a$ ($a$ is squarefree),
$n_i \ | \ m^2$ for all $i$.
Since $n_i$ is prime, $n_i \ | \ m$ for all $i$.
We conclude that $n \ | \ m$ and $b \in D$.
\QED

\begin{lemma}[Kubota 1972]
\label{lem:kubota2}
A triple of polynomials in $\Re[x_1,x_2]$ is a Pythagorean triple 
if and only if it is of the form 
\begin{equation}
\label{eq:triple2}
	\alpha(a_1^2 - a_2^2, 2a_1a_2, a_1^2 + a_2^2)
\end{equation}
for some $a_1,a_2,\alpha \in \Re[x_1,x_2]$.
\end{lemma}
\prf
(If):  A triple of the form (\ref{eq:triple2}) is clearly a Pythagorean triple:
\[
	(a_1^2 - a_2^2)^2 + (2a_1a_2)^2 = 
	a_1^4 + a_2^4 + 2a_1^2a_2^2 = (a_1^2 + a_2^2)^2
\]

(Only if): 
Kubota's proof holds for a more general class: Pythagorean triples
over $D$, where $D$ is a unique factorization domain 
not of characteristic 2, such that 2 is either prime 
or invertible in $D$.
$\Re[x_1,x_2]$ is such a unique factorization domain.
${\cal Z}$ is also such a domain, so this proof also functions
as a proof of Lemma~\ref{lem:classical}.

Let $(p_1,p_2,p_3)$ be a Pythagorean triple over $D$.
If $p_1=p_3$, let $(a_1,a_2,\alpha) = (1,0,p_1)$.
Thus, we can assume $p_1 \neq p_3$.
Factor $p_3 - p_1$ into a square component and a squarefree component:
$p_3 - p_1 = gh^2$, where $g,h \in D$ are squarefree.

If $2 \ | \ g$, let $(a_1,a_2,\alpha) = (\frac{hp_2}{p_3 - p_1}, h, \frac{g}{2}$).
Then $a_1,a_2,\alpha$ generate the Pythagorean triple $(p_1,p_2,p_3)$
as in (\ref{eq:triple2}).
In particular,
\[
\alpha (a_1^2 - a_2^2)
= \frac{g}{2} 
  (\frac{h^2p_2^2 - h^2(p_3 - p_1)^2}{(p_3-p_1)^2})
\]
and applying $p_1^2 + p_2^2 = p_3^2$ and $p_3 - p_1 = gh^2$,
\[
= \frac{1}{2} (\frac{-2p_1^2 + 2p_1p_3}{p_3 - p_1}) = p_1.
\]
And
\[
\alpha (2a_1 a_2) = gh^2 \frac{p_2}{p_3 - p_1} = p_2
\]
Finally, 
\[
\alpha(a_1^2 + a_2^2)
= \frac{g}{2} 
  (\frac{h^2p_2^2 + h^2(p_3 - p_1)^2}{(p_3-p_1)^2})
= \frac{2p_3^2 - 2p_1p_3}{2(p_3-p_1)} = p_3
\]
Thus, $\alpha (a_1^2 - a_2^2, 2a_1a_2, a_1^2 + a_2^2) = (p_1,p_2,p_3)$.
Since $ga_1^2 = 2 \alpha a_1^2 = p_1 + p_3 \in D$, 
$a_1 \in D$ by Lemma~\ref{lem:applem1}.
Clearly $a_2,\alpha \in D$.

If $2 \not | \ g$, let $(a_1,a_2,\alpha) = 
((\frac{hp_2}{p_3 - p_1} + h)/2, \ 
 (\frac{hp_2}{p_3 - p_1} - h)/2, \ g)$.
Then 
\[
\alpha (a_1^2 - a_2^2) = 
\frac{gh^2}{4}[\frac{(p_2+p_3-p_1)^2 - (p_2-p_3+p_1)^2}{(p_3-p_1)^2}] = 
\frac{gh^2}{4}[\frac{4p_2(p_3-p_1)}{(p_3-p_1)^2}] = 
p_2
\]
\[
\alpha (2a_1 a_2) = \frac{g}{2} (\frac{h^2p_2^2}{(p_3-p_1)^2} - h^2) = 
\frac{1}{2}(\frac{p_2^2 - p_3^2 - p_1^2 + 2p_3p_1}{p_3-p_1}) = 
p_1
\]
\[
\alpha(a_1^2 + a_2^2)
= \frac{gh^2}{4} [\frac{2p_2^2 + 2(p_3 - p_1)^2}{(p_3-p_1)^2}] = p_3
\]
Thus, $\alpha (a_1^2 - a_2^2, 2a_1a_2, a_1^2 + a_2^2) = (p_2,p_1,p_3)$.
Since $2 \alpha a_1^2 = p_2 + p_3 \in D$ and $2 \alpha$ is squarefree
($2 \not | \ g$), $a_1 \in D$ by Lemma~\ref{lem:applem1}.
Similarly, since $2 \alpha a_2^2 = p_3 - p_2 \in D$,
$a_2 \in D$ by Lemma~\ref{lem:applem1}.
Clearly $\alpha \in D$.
\QED

\bibliographystyle{plain}
\begin{thebibliography}{Lozano-Perez 83}

\bibitem[Barr 92]{barr92}
Barr, A. and B. Currin and S. Gabriel and J. Hughes (1992)
Smooth Interpolation of Orientations with Angular Velocity Constraints
using Quaternions.
SIGGRAPH '92, 313--320.

\bibitem[Catalan 1885]{catalan85}
Catalan, E. (1885) Bull. Acad. Roy. Belgique 3(9), p. 531.
Referenced in Dickson, L.E. (1952) History of the Theory of Numbers: Volume II,
Diophantine Analysis.  Chelsea (New York), p. 269.

\bibitem[Dickson 52]{dickson52}
Dickson, L.E. (1952) History of the Theory of Numbers: Volume II,
Diophantine Analysis.  Chelsea (New York).

\bibitem[Dietz 93]{dietz93}
Dietz, R. and J. Hoschek and B. Juttler (1993)
An algebraic approach to curves and surfaces on the sphere and
on other quadrics.
Computer Aided Geometric Design 10, 211-229.

\bibitem[Euler 1748]{euler48}
Fuss, P., editor (1843) Corresp. Math. et Phys.,
`Correspondance entre Leonhard Euler et C. Goldbach 1729-1763',
St. Petersburg, Vol. 1, p. 452.  
Referenced in Dickson, L.E. (1952) History of the Theory of Numbers: Volume II,
Diophantine Analysis.  Chelsea (New York), p. 277.
% see p. 209, Ebbinghaus, Numbers

\bibitem[Farouki 90]{farouki90}
Farouki, R. and T. Sakkalis (1990)
Pythagorean Hodographs.
IBM J. Res. Develop. 34, 736--752.

\bibitem[Herstein 75]{herstein75}
Herstein, I. (1975) Topics in Algebra.
2nd edition, John Wiley (New York).

\bibitem[Hoschek 89]{hoschek89}
Hoschek, J. and D. Lasser (1989)
Fundamentals of Computer Aided Geometric Design.
Translated by L. Schumaker.  A.K. Peters (Wellesley, MA).
% p. 593-594.

\bibitem[Johnstone 98a]{jj98}
Johnstone, J. (1998)
The Most Natural Map to the Sphere.
Technical Report 98-02, CIS Dept., UAB.

\bibitem[Johnstone 95]{jj95}
Johnstone, J. and J. Williams (1995)
Rational Control of Orientation for Animation.
Graphics Interface '95, 179--186.

\bibitem[Johnstone 98b]{jj+jimbo98}
Johnstone, J. and J. Williams (1995)
Rational Quaternion Splines,
Technical Report 98-03, CIS Dept., UAB.

\bibitem[Kim 95]{kim95}
Kim, M.-J. and M.-S. Kim and S. Shin (1995)
A General Construction Scheme for Unit Quaternion Curves with Simple
Higher Order Derivatives.
SIGGRAPH '95, 369--376.

\bibitem[Kim 88]{kim88}
Kim, M.-S. (1988)
Motion Planning with Geometric Models.
Ph.D. thesis, Purdue University.

\bibitem[Kreyszig 59]{kreyszig59}
Kreyszig, E. (1959) Differential Geometry.
Dover (New York).

\bibitem[Kubota 72]{kubota72}
Kubota, K. (1972) Pythagorean triples in unique factorization domains.
American Mathematical Monthly 79, 503--505.

\bibitem[Lozano-Perez 83]{lozanoperez83}
Lozano-Perez, T. (1983)
Spatial Planning: A Configuration Space Approach.
IEEE Trans. on Computers C-32 (2), February, 108--120.

\bibitem[Ono 94]{ono94}
Ono, T. (1994) Variations on a Theme of Euler.
Plenum Press (New York).

\end{thebibliography}

\end{document}
