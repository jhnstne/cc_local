\documentstyle[11pt]{article}
\newif\ifFull
\Fullfalse
% use Fulltrue for technical report version, with 1) proof of Kubota's
% version of necessary condition, and 2) footnote about path planning and
% configuration space on page 1
\input{header}
\SingleSpace

\setlength{\oddsidemargin}{0pt}
%\setlength{\topmargin}{-.25in}	% technically should be 0pt for 1in
\setlength{\headsep}{3em}
\setlength{\textheight}{8.75in}
\setlength{\textwidth}{6.5in}
\setlength{\columnsep}{5mm}		% width of gutter between columns

\markright{Maps to the sphere: \today \hfill}
\pagestyle{myheadings}

% -----------------------------------------------------------------------------

\title{A Characterization of Rational Maps of $\Re^n$ to \Sn{n-1}\\for Riemannian Modeling}
% : A Tool for Modeling upon a Surface}
%\title{A Characterization of Rational Maps to the Sphere}
\author{J.K. Johnstone}

\begin{document}
\maketitle

\clearpage

\noindent Consider a rational map from $\Re^4$ to \Sn{3}:
\[
	(x_1,\ldots,x_4) \mapsto
	(\frac{f_1(x_1,\ldots,x_4)}{f_{5}(x_1,\ldots,x_4)}, \ldots,
	 \frac{f_4(x_1,\ldots,x_4)}{f_{5}(x_1,\ldots,x_4)})
\]
where $f_1,\ldots,f_{5}$ are polynomials.
Since the image lies on \Sn{3}, $f_1^2 + \cdots + f_4^2 = f_{5}^2$.
This is called a Pythagorean quintuple.\footnote{This term derives
	from the Pythagorean Theorem on right triangles,
	which involves Pythagorean triples.}
\begin{defn2}
$(a_1,\ldots,a_{5}) \in K^{n+1}$
is a {\bf Pythagorean quintuple over $K$} 
if $a_1^2 + \ldots + a_4^2 = a_5^2$.
\end{defn2}
%
Thus, the study of rational maps from $\Re^{4}$ to \Sn{3}\ 
is equivalent to the study of Pythagorean quintuples of polynomials.
Pythagorean quintuples involve the sum of four squares.
In the number theory literature, there is an extensive study of the
sum of four squares (driven by the search for a proof that
every positive integer is the sum of the squares of four integers \cite{dickson52}).
An important result was developed by Euler,\footnote{This result shows that the product of a sum of four squares and a sum of four
	squares is another sum of four squares.
	This reduces the problem of showing that every {\em integer} is the sum
	of four squares to the simpler problem
	of showing that every {\em prime} is the sum of four squares,
	since every integer can be expressed as the product of primes.}
which we can use to build Pythagorean quintuples, 
and then a characterization of maps to \Sn{3}.

\begin{lemma}[Euler's Four Squares Theorem \cite{herstein75}]	
% p. 373 of Herstein
\[
\begin{array}{ll}
& (a_1^2 + a_2^2 + a_3^2 + a_4^2) 
(\hat{a}^2_1 + \hat{a}^2_2 + \hat{a}^2_3 + \hat{a}^2_4) = \\
& (a_1 \hat{a}_1 - a_2\hat{a}_2 - a_3\hat{a}_3 - a_4\hat{a}_4)^2 +
   (a_1\hat{a}_2 + a_2\hat{a}_1 + a_3\hat{a}_4 - a_4\hat{a}_3)^2 + \\
& (a_1\hat{a}_3 - a_2\hat{a}_4 + a_3\hat{a}_1 + a_4\hat{a}_2)^2 +
   (a_1\hat{a}_4 + a_2\hat{a}_3 - a_3\hat{a}_2 + a_4\hat{a}_1)^2
\end{array}
\]
where $a_1,a_2,a_3,a_4,\hat{a}_1,\hat{a}_2,\hat{a}_3,\hat{a}_4$ are elements of a
commutative ring.
% \footnote{The original statement was for integers, but it easily generalizes.}
% see p. 210 of Ebbinghaus
\end{lemma}

\begin{corollary}
\label{lem:suff4}
$(a_1^2 + a_2^2 + a_3^2 - a_4^2,\ 2a_1a_4,\ 2a_2a_4,\ 2a_3a_4,\ 
 a_1^2 + a_2^2 + a_3^2 + a_4^2)$
is a Pythagorean quintuple for any polynomials $a_1,a_2,a_3,a_4$.
\end{corollary}
\prf
Let $(\hat{a}_1,\hat{a}_2,\hat{a}_3,\hat{a}_4) = (a_1,-a_2,-a_3,a_4)$. 
\QED

\noindent The following lemma establishes a weaker version of the necessary condition
associated with Corollary~\ref{lem:suff4}.
This in turn leads to the desired necessary and sufficient condition for rational maps
of $\Re^4$ to \Sn{3}.

\begin{lemma}
\label{thm:necessary4}
A quintuple of polynomials is Pythagorean only if it can be expressed in the form
\begin{equation}
\label{eq:pyth}
	\alpha (a_1^2 + a_2^2 + a_3^2 - a_4^2,
		\ 2a_1a_4,\ 2a_2a_4,\ 2a_3a_4,
		\ a_1^2 + a_2^2 + a_3^2 + a_4^2)
\end{equation}
for some polynomials $a_1,a_2,a_3,a_4,\frac{1}{\alpha}$.
\end{lemma}
\prf
Let $(p_1,p_2,p_3,p_4,p_5)$ be a Pythagorean quintuple of polynomials.
If $p_1 = p_5$, let\\
$(a_1,a_2,a_3,a_4,\alpha) = (p_1,0,0,0,\frac{1}{p_1})$.
Thus, we may assume without loss of generality that $p_1 \neq p_5$.
Let $(a_1,a_2,a_3,a_4,\alpha) = (p_2,p_3,p_4,p_5-p_1,\frac{1}{2(p_5 - p_1)})$.
Then $a_1,a_2,a_3,a_4,\alpha$ generate the Pythagorean quintuple
$(p_1,\ldots,p_5)$ as in (\ref{eq:pyth}).
In particular,
\[
\alpha (a_1^2 + a_2^2 + a_3^2 - a_4^2)
= \frac{p_2^2 + p_3^2 + p_4^2 - (p_5 - p_1)^2}{2(p_5-p_1)}
\]
and applying $p_1^2 + p_2^2 + p_3^2 + p_4^2 = p_5^2$,
\[
\alpha (a_1^2 + a_2^2 + a_3^2 - a_4^2) = \frac{-2p_1^2 + 2p_1p_5}{2(p_5 - p_1)} = p_1.
\]
Also, 
\[
\alpha (2a_i a_4) = \frac{2p_{i+1}(p_5 - p_1)}{2(p_5 - p_1)} = p_{i+1}
\]
for $i=1,2,3$.
Finally, 
\[
\alpha(a_1^2 + a_2^2 + a_3^2 + a_4^2) 
= \frac{p_2^2 + p_3^2 + p_4^2 + (p_5 - p_1)^2}{2(p_5-p_1)} = p_5
\]
Thus, $(p_1,p_2,p_3,p_4,p_5) = \alpha (a_1^2 + a_2^2 + a_3^2 - a_4^2,
		\ 2a_1a_4,\ 2a_2a_4,\ 2a_3a_4,
		\ a_1^2 + a_2^2 + a_3^2 + a_4^2)$.
\QED

\begin{corollary}
\label{thm:map4}
A map is a rational map of $\Re^4$ to \Sn{3}\ if and only if
it is of the form:
\begin{equation}
\label{eq:re4s3}
\small{(x_1,x_2,x_3,x_4)} \mapsto 
\small{(\frac{a_1^2 + a_2^2 + a_3^2 - a_4^2}{a_1^2 + a_2^2 + a_3^2 + a_4^2},
	 \frac{2a_1a_4}{a_1^2 + a_2^2 + a_3^2 + a_4^2},
	 \frac{2a_2a_4}{a_1^2 + a_2^2 + a_3^2 + a_4^2},
	 \frac{2a_3a_4}{a_1^2 + a_2^2 + a_3^2 + a_4^2})}
\end{equation}
where $a_1,a_2,a_3,a_4$ are polynomials over $x_1,x_2,x_3,x_4$
(or some permutation of this form).\footnote{This result generalizes
	to rational maps from $\Re^n$ to \Sn{n-1}, $n \geq 2$,
	using an identical proof.
	That is, a map is a rational map of $\Re^n$ to \Sn{n-1}, $n \geq 2$, 
	if and only if it is of the form:
\[
	(x_1,\ldots,x_n) \mapsto
	(\frac{a_1^2 + \cdots + a_{n-1}^2 - a_n^2}{a_1^2 + \cdots + a_n^2},
	 \frac{2a_1a_n}{a_1^2 + \cdots + a_n^2},
	 \ldots,
	 \frac{2a_{n-1}a_n}{a_1^2 + \cdots + a_n^2})
\]
where $a_1,\ldots,a_n$ are polynomials over $x_1,\ldots,x_n$.}
\end{corollary}
\prf
Consider a rational map
of $\Re^4$ to \Sn{3}, $(x_1,x_2,x_3,x_4) \mapsto 
(\frac{f_1}{f_5},\frac{f_2}{f_5},\frac{f_3}{f_5},\frac{f_4}{f_5})$,
where $f_1,\ldots,f_5$ are polynomials over $x_1,\ldots,x_4$.
Then $(f_1,\ldots,f_5)$ is a Pythagorean quintuple
so it can be expressed in the normal form
$(f_1,\ldots,f_4,f_5) = \alpha (a_1^2 + a_2^2 + a_3^2 - a_4^2,
2a_1a_4,2a_2a_4,2a_3a_4,$ $a_1^2 + a_2^2 + a_3^2 + a_4^2)$
for some polynomials $a_1,a_2,a_3,a_4,\frac{1}{\alpha}$
by Lemma~\ref{thm:necessary4}.
Thus, the map is of the form (\ref{eq:re4s3}), since the leading $\alpha$
cancels when expressed in 
$(\frac{f_1}{f_5},\frac{f_2}{f_5},\frac{f_3}{f_5},\frac{f_4}{f_5})$.
Notice that it is legal to cancel the $\alpha$, since
$\alpha = \frac{1}{\beta}$ for some polynomial $\beta$,
which implies that $\alpha$ is never zero.

Now consider a map of the form (\ref{eq:re4s3}).
This is a rational map of $\Re^4$ to \Sn{3}\ by Corollary~\ref{lem:suff4}.
\QED

The most natural choice for $a_i$ ($a_i=x_i$) and the identity permutation
yields the rational map
\[
(x_1,x_2,x_3,x_4) \mapsto 
	(\frac{x_1^2 + x_2^2 + x_3^2 - x_4^2}{x_1^2 + x_2^2 + x_3^2 + x_4^2},
	 \frac{2x_1x_4}{x_1^2 + x_2^2 + x_3^2 + x_4^2}, 
	 \frac{2x_2x_4}{x_1^2 + x_2^2 + x_3^2 + x_4^2}, 
	 \frac{2x_3x_4}{x_1^2 + x_2^2 + x_3^2 + x_4^2})
\]
of $\Re^4$ to \Sn{3}.
We call this the {\bf most natural} map from $\Re^4$ to \Sn{3}.
It is discussed further in \cite{jj98}.

\bibliographystyle{plain}
\begin{thebibliography}{Johnstone 98}

\bibitem[Dickson 52]{dickson52}
Dickson, L.E. (1952) History of the Theory of Numbers: Volume II,
Diophantine Analysis.  Chelsea (New York).

\bibitem[Herstein 75]{herstein75}
Herstein, I. (1975) Topics in Algebra.
2nd edition, John Wiley (New York).

\bibitem[Johnstone 98]{jj98}
Johnstone, J. (1998)
The Most Natural Map to the Sphere.
Technical Report 98-02, CIS Dept., UAB.

\end{thebibliography}

\end{document}
