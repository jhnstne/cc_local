\documentstyle[11pt]{article}
\newif\ifFull
\Fullfalse
% use Fulltrue for technical report version, with 1) proof of Kubota's
% version of necessary condition, and 2) footnote about path planning and
% configuration space on page 1
\input{header}
\SingleSpace

\setlength{\oddsidemargin}{0pt}
%\setlength{\topmargin}{-.25in}	% technically should be 0pt for 1in
\setlength{\headsep}{3em}
\setlength{\textheight}{8.75in}
\setlength{\textwidth}{6.5in}
\setlength{\columnsep}{5mm}		% width of gutter between columns

\markright{Maps to the sphere: \today \hfill}
\pagestyle{myheadings}

% -----------------------------------------------------------------------------

\title{A Characterization of Rational Maps of $\Re^n$ to \Sn{n-1}\\for Riemannian Modeling}
% : A Tool for Modeling upon a Surface}
%\title{A Characterization of Rational Maps to the Sphere}
\author{J.K. Johnstone}

\begin{document}
\maketitle

\begin{abstract}
We develop a complete characterization of rational
maps of $\Re^n$ to \Sn{n-1}, the unit sphere in $\Re^n$.
These maps are useful in the design of rational curves on the sphere.
The main result of the paper is that
a map is a rational map of $\Re^n$ to \Sn{n-1}, $n \geq 2$, if and only if
it is of the form:
\[
	(x_1,\ldots,x_n) \mapsto 
	\pi (\frac{a_1^2 + \cdots + a_{n-1}^2 - a_n^2}{a_1^2 + \cdots + a_n^2},
	 \frac{2a_1a_n}{a_1^2 + \cdots + a_n^2},
	 \ldots,
	 \frac{2a_{n-1}a_n}{a_1^2 + \cdots + a_n^2})
\]
where $a_1,\ldots,a_n \in \Re[x_1,\ldots,x_n]$
and $\pi: \{1,\ldots,n\} \rightarrow \{1,\ldots,n\}$ is a permutation.
Euler's Four Squares Theorem is central to this result.
\end{abstract}

\vspace{1in}

\noindent Keywords: 
Curves on surfaces, rational maps, 
Euler's Four Squares Theorem, Pythagorean tuples, Riemannian modeling.

\clearpage

\section{Introduction}

The design of curves on surfaces is important in several problems,
such as quaternion splines \cite{barr92,jj95},
trim curves \cite{hoschek89}, and
shortest path planning \cite{lozanoperez83,kim88}.
\ifFull
\footnote{Path planning 
	of a multiple-degree-of-freedom object amongst obstacles in 3-space
 	can be modeled using configuration space (C-space) obstacles.
 	These C-space obstacles are surfaces in high
 	dimensions---often 6 dimensions or more.
 	Shortest paths involve curves on the C-space obstacles.}
\fi
Since surfaces are Riemannian spaces, % \cite{kreyszig59}
curve design on surfaces is significantly different from 
conventional modeling of curves in Euclidean space.
A common solution to modeling on a surface is to model in the parameter
space of the surface.
This reduces a Riemannian problem to a Euclidean problem,
since parameter space is Euclidean.
% Consider the surface $A \subset \Re^n$ with parameterization
% $\rho:\Re^{n-1} \rightarrow A$.
% A curve in parameter space is mapped back to a curve on the 
% surface $A$ using the surface parameterization $\rho$.
We shall call this method the {\bf parameter-space} approach to Riemannian
modeling.
This is the usual solution for the design of trim curves.

Another way of reducing curve design on a surface $A \subset \Re^n$
to a simpler Euclidean problem is to model in the native Euclidean space
of $A$, $\Re^n$, through the design of a map $f: \Re^n \rightarrow A$.
% (Figure~\ref{fig:approaches}).
% Figure of a) map of curve from parameter space and b) general map of curve
% to surface
If $C$ is a curve in $\Re^{n}$, then $f(C)$ is a curve on $A$.
We shall call this method the {\bf native-space} approach to Riemannian 
modeling.
The native-space approach can be viewed as a generalization of the parameter-space
approach, where the surface parameterization $\rho:\Re^{n-1} \rightarrow A$
is replaced by a map $f: \Re^n \rightarrow A$ to the surface.
The added flexibility inherent in this generalization leads to 
some advantages over the parameter-space approach, 
such as improved curve quality.
These advantages are fully explored in \cite{jj+jimbo99}.
Obviously, one of the essential challenges in the implementation 
of this approach is the design of a good map $f$ to the surface $A$.

Since rational curves are the preferred curves in current modeling systems,
due to their simplicity and elegance,
we want to design {\em rational} maps $f: \Re^n \rightarrow A$ 
so that rational curves can be designed on the surface simply
by designing rational curves in the native Euclidean space.
% Notice that $f^{-1}$ need {\bf not} be rational.

\Comment{
\begin{figure}
\vspace{3.5in}
\special{psfile=/usr/people/jj/modelTR/1-char/img/fig1.ps
	 hoffset=100}
\caption{The (a) parameter-space and (b) native-space approaches to Riemannian modeling}
% file: fig1a.showcase and fig1b.showcase
% tops fig1.rgb -m 6.5 2.5 > fig1.ps
\label{fig:approaches}
\end{figure}
}

Curve design on \Sn{n-1}, the unit sphere in $n$-space,
is an elegant instance of the design of curves on a surface,
offering an example of the problems and possible approaches
to Riemannian modeling on the simplest and purest Riemannian surface, 
the sphere.
% \footnote{The plane is simpler than the sphere,
%	but design on the plane is still Euclidean modeling, since
%	the plane is equivalent to 2-space.}
Curve design on \Sn{3}\ also has special interest because
of the important computer animation problem of 
quaternion spline design for orientation control \cite{barr92,jj95,kim95},
which is nothing but the design of curves on \Sn{3}.
Therefore, rational maps to \Sn{n-1}\ and especially \Sn{3}\ are worthy
of study.

In this paper, we develop a complete characterization of rational
maps of $\Re^{n}$ to \Sn{n-1}.
By understanding rational maps to \Sn{n-1}, we can better design rational
curves on \Sn{n-1}\ and, indeed, better understand the geometry of this
most important of surfaces.
The paper is structured as follows.
Basic definitions are introduced in Section~\ref{sec:defn}.
The relationship between rational maps of $\Re^{n}$ to \Sn{n-1}\
and Pythagorean tuples of polynomials is established in Section~\ref{sec:setup}.
Sections~\ref{sec:n2}-\ref{sec:nn} characterize rational maps 
of $\Re^{n}$ to \Sn{n-1}\ for $n=2,3,4$ and general $n$, respectively.
The cases $n=2,3$ are well understood, but the case $n=4$ 
in Section~\ref{sec:n4} requires new observations 
that are generalized to all $n \geq 2$ in Section~\ref{sec:nn}.
Notice that the case $n=4$, rational maps from $\Re^4$ to \Sn{3},
is also the crucial case for quaternion splines.
The paper ends with some conclusions in Section~\ref{sec:conclusions}.
The appendix contains a full proof of Kubota's characterization of Pythagorean
triples, an important result for the case $n=2$.

\section{Definitions}
\label{sec:defn}

\begin{defn2}
\Sn{n-1}\ is the unit sphere in $n$-space $x_1^2 + \ldots + x_{n}^2 - 1 = 0$.
The superscript $n-1$ refers to the dimension of this manifold.
\end{defn2}

\noindent For example, \Sn{1}\ is a circle and \Sn{2}\ is a typical
sphere in 3-space.

\begin{defn2}
A {\bf rational polynomial} is a quotient of polynomials.
A map 
$(x_1,\ldots,x_n) \mapsto (f_1 (x_1,\ldots,x_n),\ldots,f_m (x_1,\ldots,x_n))$
is {\bf rational} if 
its components $f_i$ are all rational polynomials in $x_1,\ldots,x_n$.
\end{defn2}

\begin{defn2}
$\Re[x_1,\ldots,x_n]$ is the ring of polynomials in the $n$ variables
$x_1,\ldots,x_n$ with real coefficients.
\end{defn2}

\begin{defn2}
A {\bf commutative ring} is a ring in which multiplication is commutative.
\end{defn2}
%
Examples of commutative rings are the integers, the rationals, the reals,
and polynomials in $n$ variables over the integers, rationals, or reals.
Commutative rings subsume all of the domains of interest in this paper.

\begin{defn2}
Let $K$ be a commutative ring.
An $(n+1)$-tuple $(a_1,\ldots,a_{n+1}) \in K^{n+1}$
is a {\bf Pythagorean $(n+1)$-tuple over $K$} 
if $a_1^2 + \ldots + a_n^2 = a_{n+1}^2$.
\end{defn2}
%
This terminology is derived from Pythagoras' theorem on the lengths
of the sides of a right triangle, which of course generate
a Pythagorean triple.
We record the following obvious result for future reference.

\begin{lemma}
\label{lem:perm}
If $(a_1,\ldots,a_{n+1})$ is a Pythagorean $(n+1)$-tuple over $K$,
$(a_{\pi(1)},\ldots,a_{\pi(n)},a_{n+1})$ is also a Pythagorean 
$(n+1)$-tuple over $K$ for any permutation 
$\pi : \{1,\ldots,n\} \rightarrow \{1,\ldots,n\}$.
\end{lemma}


\section{Rational maps from $\Re^{n}$ to \Sn{n-1}}
\label{sec:setup}

% Rational maps to the sphere are attractive 
% because of their tractability.
% Maps to \Sn{n-1}\ (the unit sphere in $n$-space)
% whose domain is all of $n$-space are also attractive,
% because any point can be mapped to the sphere
% and the dimension of the space is preserved in the map.
% In this section, we study rational maps from $\Re^{n}$ to \Sn{n-1}.

A rational map from $\Re^n$ to \Sn{n-1}\ can be expressed as:
\[
	(x_1,\ldots,x_n) \mapsto
	(\frac{f_1(x_1,\ldots,x_n)}{f_{n+1}(x_1,\ldots,x_n)}, \ldots,
	 \frac{f_n(x_1,\ldots,x_n)}{f_{n+1}(x_1,\ldots,x_n)})
\]
where $f_1,\ldots,f_{n+1}$ are polynomials.
Since the image lies on \Sn{n-1},
\[
	f_1^2 + \cdots + f_n^2 = f_{n+1}^2
\]
Thus, $(f_1,\ldots,f_{n+1})$ is a Pythagorean $(n+1)$-tuple of polynomials in
$\Re[x_1,\ldots,x_n]$.
This reduces the study of rational maps from $\Re^{n}$ to \Sn{n-1}\ 
to the study of Pythagorean $(n+1)$-tuples of polynomials in 
$\Re[x_1,\ldots,x_n]$.

\begin{lemma}
\label{lem:iffpyth}
A rational map 
$(x_1,\ldots,x_n) \mapsto (\frac{f_1}{f_{n+1}},\ldots,\frac{f_n}{f_{n+1}})$,
where $f_1,\ldots,f_{n+1} \in \Re[x_1,\ldots,x_n]$, is a map from $\Re^n$
to \Sn{n-1}\ if and only if $(f_1,\ldots,f_{n+1})$ is a Pythagorean
$(n+1)$-tuple of polynomials in $\Re[x_1,\ldots,x_n]$.
\end{lemma}

\section{$\Re^{2}$ to \Sn{1}}
\label{sec:n2}

Consider rational maps from $\Re^2$ to \Sn{1},
and thus Pythagorean triples of polynomials in $\Re[x_1,x_2]$.
It is easily seen that 
\[ (a_1^2 - a_2^2,\ 2a_1a_2,\ a_1^2 + a_2^2)
\]
is a Pythagorean triple
of integers for any choice of integers $a_1$ and $a_2$.
This condition is also necessary, up to a constant factor.
% p. 89, Ebbinghaus, Numbers
%
\begin{lemma}[Classical]
\label{lem:classical}
A triple of integers is a Pythagorean triple if and only if
it is of the form $\alpha(a_1^2 - a_2^2, 2a_1a_2, a_1^2 + a_2^2)$ 
for some integers $a_1,a_2,\alpha$.\footnote{Whenever
	we give a normal form for a Pythagorean $(n+1)$-tuple,
	it is understood that the first $n$ elements are free to undergo
	a permutation.}
\end{lemma}
%
\cite{kubota72} generalized this result
to Pythagorean triples of polynomials in $\Re[x_1,x_2]$.
% where $K$ is a field of characteristic $p \neq 2$.
%
\begin{lemma}[Kubota 1972]
\label{lem:kubota}
A triple of polynomials in $\Re[x_1,x_2]$ is a Pythagorean triple 
if and only if it is of the form 
\begin{equation}
\label{eq:triple}
	\alpha(a_1^2 - a_2^2, 2a_1a_2, a_1^2 + a_2^2)
\end{equation}
for some $a_1,a_2,\alpha \in \Re[x_1,x_2]$.
\end{lemma}
\prf
See the appendix.
\QED

\noindent This yields 
a full characterization of rational maps from $\Re^2$ to \Sn{1}.
%
\begin{theorem}
\label{thm:n2}
A map is a rational map of $\Re^2$ to \Sn{1}\ if and only if 
it is of the form:
\begin{equation}
\label{eq:rationaltoS1}
	(x_{\pi(1)},x_{\pi(2)}) \mapsto (\frac{a_1^2 - a_2^2}{a_1^2 + a_2^2}, 
			   \frac{2a_1 a_2}{a_1^2 + a_2^2})
\end{equation}
where $a_1,a_2 \in \Re[x_1,x_2]$ and $\pi : \{1,2\} \rightarrow \{1,2\}$
is a permutation.
\end{theorem}
\prf
By Lemmas~\ref{lem:perm}, \ref{lem:iffpyth} and \ref{lem:kubota}.
We should be careful about points at which $\alpha(x_1,x_2) = 0$,
since the cancellation of $\alpha$ is not valid at these points.
However, we ignore this issue here, since it will be handled elegantly
by a more general version of this theorem.
See the proof of Theorem~\ref{thm:map4}, which is generalized in Theorem~\ref{thm:nn}.
\QED
% (Only if): Let $(x_1,x_2) \mapsto (\frac{f_1}{f_3}, \frac{f_2}{f_3})$,
% $f_1,f_2,f_3 \in \Re[x_1,x_2]$, be a rational map of $\Re^2$ to \Sn{1}.
% Then $(f_1,f_2,f_3)$ is a Pythagorean triple (Lemma~\ref{lem:iffpyth}).
% By Lemmas~\ref{lem:kubota} and \ref{lem:perm}, 
% $(f_{\pi(1)}, f_{\pi(2)}, f_3) = \alpha(a_1^2 - a_2^2, 2a_1a_2, a_1^2 + a_2^2)$ 
% for some $a_1,a_2,\alpha \in \Re[x_1,x_2]$ and permutation 
% $\pi : \{1,2\} \rightarrow \{1,2\}$.
% We conclude that the rational map is of the form (\ref{eq:rationaltoS1}).

% (If): Consider a map of the form (\ref{eq:rationaltoS1}), call it $M$.
% Let $f_1 = a_1^2 - a_2^2$, $f_2 = 2a_1 a_2$, $f_3 = a_1^2 + a_2^2$.
% By Lemma~\ref{lem:kubota}, $(f_1,f_2,f_3)$ is a Pythagorean triple
% over $\Re[x_1,x_2]$.
% By Lemma~\ref{lem:perm}, $(f_{\pi(1)}, f_{\pi(2)}, f_3)$ is also a Pythagorean
% triple.
% By Lemma~\ref{lem:iffpyth}, the map 
% $(x_1,x_2) \mapsto (\frac{f_{\pi(1)}}{f_3}, \frac{f_{\pi(2)}}{f_3})$
% is a rational map of $\Re^2$ to \Sn{1}.
% This map is equivalent to 
% $(x_{\pi(1)},x_{\pi(2)}) \mapsto (\frac{f_1}{f_3}, \frac{f_2}{f_3})$,
% which is our given map $M$.
% Thus, $M$ is a rational map of $\Re^2$ to \Sn{1}.
% \QED

\begin{example}
With $(a_1,a_2) = (1,x_2)$ and the identity permutation,
we have the familiar parameterization of the circle
$(x_1,x_2) \mapsto (\frac{1 - x_2^2}{1 + x_2^2}, 
				 \frac{2x_2}{1 + x_2^2})$.
\end{example}

\begin{example}
With $(a_1,a_2) = (x_1,x_2)$ and the identity permutation,
$(x_1,x_2) \mapsto (\frac{x_1^2 - x_2^2}{x_1^2 + x_2^2}, 
				 \frac{2x_1 x_2}{x_1^2 + x_2^2})$
is a rational map $\Re^2 \rightarrow \Sn{1}$.
On \Sn{1}, $x_1^2 + x_2^2 = 1$, so the restriction of this map
to \Sn{1}\ is $(x_1,x_2) \mapsto (x_1^2 - x_2^2, 2x_1x_2)$.
This map from \Sn{1}\ to \Sn{1}\ is
the simplest nontrivial quadratic spherical map \cite{ono94}. % [pp. 169,173]
\end{example}

\begin{example}
\cite{farouki90} uses Kubota's result to define Pythagorean hodograph curves, 
which have attractive arc length and offset properties.
The hodographs of these curves are components of a Pythagorean triple
of polynomials.
% which among other things have polynomial arc length (in the curve parameter).
\end{example}

\vspace{.5in}

\section{$\Re^{3}$ to \Sn{2}}
\label{sec:n3}

Consider rational maps from $\Re^3$ to \Sn{2},
and thus Pythagorean quadruples of polynomials in \linebreak $\Re[x_1,x_2,x_3]$.
Euler knew that 
\[ (a_1^2+a_2^2-a_3^2-a_4^2,\ 2a_1a_3+2a_2a_4,\ 2a_1a_4-2a_2a_3,
    \ a_1^2+a_2^2+a_3^2+a_4^2)
\]
is a Pythagorean quadruple for any choice of integers $a_1,a_2,a_3,a_4$
(see Lemma~\ref{lem:corEuler4square} below).
\linebreak
\cite{catalan85} observed that this is also a necessary condition.
%
\begin{lemma}[Catalan 1885]
\label{lem:catalan}
A quadruple of integers is a Pythagorean quadruple
if and only if it is of the form
$\alpha (a_1^2+a_2^2-a_3^2-a_4^2,\ 2a_1a_3+2a_2a_4,\ 2a_1a_4-2a_2a_3,
    \ a_1^2+a_2^2+a_3^2+a_4^2)$
for some integers $a_1,a_2,a_3,a_4,\alpha$.
\end{lemma}
\cite{dietz93} generalized this result to Pythagorean quadruples of
polynomials in $\Re[x_1]$, $\Re[x_1,x_2]$ or
$\Re[x_1,x_2,x_3]/<x_1+x_2+x_3-1>$.
We actually need a generalization to $\Re[x_1,x_2,x_3]$,
to yield a full characterization of rational maps from 
$\Re^3$ to \Sn{2}.
We do not present a proof of this result, since we will give a more
powerful version of the following theorem (see Theorem~\ref{thm:nn} 
and the commentary after it).

\Comment{
\begin{lemma}
\label{lem:dietz}
A quadruple of polynomials in $\Re[x_1,x_2,x_3]$
is a Pythagorean quadruple if and only if it is of the form
$\alpha (a_1^2+a_2^2-a_3^2-a_4^2,\ 2a_1a_3+2a_2a_4,\ 2a_1a_4-2a_2a_3,
    \ a_1^2+a_2^2+a_3^2+a_4^2)$
for some polynomials $a_1,a_2,a_3,a_4,\alpha$ in $\Re[x_1,x_2,x_3]$.
\end{lemma}
}

\begin{theorem}
\label{thm:n3}
A map is a rational map of $\Re^3$ to \Sn{2}\ if and only if
it is of the form:
\[
(x_{\pi(1)},x_{\pi(2)},x_{\pi(3)}) \mapsto (\frac{a_1^2 + a_2^2 - a_3^2 - a_4^2}{a_1^2 + a_2^2 + a_3^2 + a_4^2},
		       \frac{2a_1a_3 + 2a_2a_4}{a_1^2 + a_2^2 + a_3^2 + a_4^2},
		       \frac{2a_1a_4 - 2a_2a_3}{a_1^2 + a_2^2 + a_3^2 + a_4^2})
\]
where $a_1,a_2,a_3,a_4 \in \Re[x_1,x_2,x_3]$
and $\pi : \{1,2,3\} \rightarrow \{1,2,3\}$ is a permutation.
\end{theorem}
%
\begin{example}
With $(a_1,a_2,a_3,a_4) = (x_1,x_2,x_3,1)$
and the permutation $(1,2,3) \mapsto (3,2,1)$,
\[
(x_1,x_2,x_3) \mapsto 
(\frac{2x_1 - 2x_2x_3}{x_1^2 + x_2^2 + x_3^2 + 1},
 \frac{2x_1x_3 + 2x_2}{x_1^2 + x_2^2 + x_3^2 + 1},
 \frac{x_1^2 + x_2^2 - x_3^2 - 1}{x_1^2 + x_2^2 + x_3^2 + 1})
\]
is a rational map $\Re^3 \rightarrow \Sn{2}$.
This map is used in \cite{dietz93} 
for the design of curves on \Sn{2}.
\end{example}
%
\begin{example}
\label{eg:foreshadow}
With $(a_1,a_2,a_3,a_4)=(x_2,x_1,0,x_3)$ and the identity permutation,
\[
(x_1,x_2,x_3) \mapsto 
(\frac{x_1^2 + x_2^2 - x_3^2}{x_1^2 + x_2^2 + x_3^2},
 \frac{2x_1x_3}{x_1^2 + x_2^2 + x_3^2},
 \frac{2x_2x_3}{x_1^2 + x_2^2 + x_3^2})
\]
is a rational map $\Re^3 \rightarrow \Sn{2}$.
\end{example}

\vspace{.5in}

\section{$\Re^{4}$ to \Sn{3}}
\label{sec:n4}

Consider rational maps from $\Re^4$ to \Sn{3}, and thus
Pythagorean quintuples of polynomials in \linebreak $\Re[x_1,x_2,x_3,x_4]$.
Pythagorean quintuples are not as well understood as Pythagorean triples
and quadruples, and our study will lead to some new characterizations
(Theorems~\ref{thm:necessary4} and \ref{thm:map4}).
We are not aware of any specific study of Pythagorean quintuples.
However, in the number theory literature, there is an extensive study of the
sum of four squares, which is related to Pythagorean quintuples.
This study was driven by the search for a proof that
every positive integer is the sum of the squares of four integers.
This famous result was apparently known by Diophantus in the third century,
since he assumes it implicitly in his writings \cite{dickson52}. % p. 275
Fermat (in characteristic fashion!) claimed that he had a proof.\footnote{"I can
	not give the proof here, which depends upon numerous and abstruse
	mysteries of numbers; for I intend to devote an entire book
	to this subject" \cite[p. 6]{dickson52}.}
However, the first published proof was by Lagrange in 1770 \cite{herstein75}, % p. 375
over a century later,
during which time many mathematicians worked on the problem.
One of these was Euler, who established the following important result in 1748
\cite{dickson52,herstein75}.   % p. 209 of Ebbing, p. 373 of Herstein
	% p. 210, Ebbinghaus for any commutative ring and Gauss' proof
	% another bulky version in \cite[p. 277]{dickson52}.

\begin{lemma}[Euler's Four Squares Theorem]
\[
\begin{array}{ll}
& (a_1^2 + a_2^2 + a_3^2 + a_4^2) 
(\hat{a}^2_1 + \hat{a}^2_2 + \hat{a}^2_3 + \hat{a}^2_4) = \\
& (a_1 \hat{a}_1 - a_2\hat{a}_2 - a_3\hat{a}_3 - a_4\hat{a}_4)^2 +
   (a_1\hat{a}_2 + a_2\hat{a}_1 + a_3\hat{a}_4 - a_4\hat{a}_3)^2 + \\
& (a_1\hat{a}_3 - a_2\hat{a}_4 + a_3\hat{a}_1 + a_4\hat{a}_2)^2 +
   (a_1\hat{a}_4 + a_2\hat{a}_3 - a_3\hat{a}_2 + a_4\hat{a}_1)^2
\end{array}
\]
where $a_1,a_2,a_3,a_4,\hat{a}_1,\hat{a}_2,\hat{a}_3,\hat{a}_4$ are elements of a
commutative ring.
% \footnote{The original statement was for integers, but it easily generalizes.}
% see p. 210 of Ebbinghaus
\end{lemma}

This result shows that the product of a sum of four squares and a sum of four
squares is another sum of four squares.
This reduces the problem of showing that every integer is the sum
of four squares to the simpler problem
of showing that every prime is the sum of four squares,
since every integer can be expressed as the product of primes.
For our purposes, the most interesting aspect of 
the Four Squares Theorem is its specialization to 
two simpler formulae, which are recipes for
generating Pythagorean quadruples and quintuples.

\begin{lemma}[Euler]
\label{lem:corEuler4square}
\ \ \\
\begin{equation}
\label{eq:euler1}
(a_1^2 + a_2^2 + a_3^2 + a_4^2)^2 = 
(a_1^2 + a_2^2 - a_3^2 - a_4^2)^2 + (2a_1a_3+2a_2a_4)^2 + (2a_1a_4-2a_2a_3)^2
\end{equation}
%
\begin{equation}
\label{eq:aida}
(a_1^2 + a_2^2 + a_3^2 + a_4^2)^2 = 
(a_1^2 + a_2^2 + a_3^2 - a_4^2)^2 + (2a_1a_4)^2 + (2a_2a_4)^2 + (2a_3a_4)^2
\end{equation}
where $a_1,a_2,a_3,a_4$ are elements of any commutative ring.
\end{lemma}
\prf
In the Four Squares Theorem, let 
$(a_1,a_2,a_3,a_4) = (\hat{a}_1,-\hat{a}_2,\hat{a}_3,\hat{a}_4)$
for (\ref{eq:euler1}) and \\
$(\hat{a}_1,\hat{a}_2,\hat{a}_3,\hat{a}_4) = (a_1,-a_2,-a_3,a_4)$ 
for (\ref{eq:aida}).
Other special cases can be found by assigning different signs.
\QED

We have already seen the use of the first formula as a necessary and
sufficient condition for Pythagorean quadruples (Lemma~\ref{lem:catalan}).
The second formula now provides a sufficient condition for Pythagorean quintuples.

\begin{lemma}
\label{lem:suff4}
Let $D$ be the integers or the polynomials over $\Re[x_1,x_2,x_3,x_4]$.
\begin{equation}
\label{eq:suff4}
	(a_1^2 + a_2^2 + a_3^2 - a_4^2,\ 2a_1a_4,\ 2a_2a_4,\ 2a_3a_4,\ 
	 a_1^2 + a_2^2 + a_3^2 + a_4^2)
\end{equation}
is a Pythagorean quintuple for any $a_1,a_2,a_3,a_4 \in D$.
\end{lemma}

Our challenge is to show that the form (\ref{eq:suff4}) is also a necessary condition.
In particular, we would like to show that a Pythagorean quintuple over $D$ 
can always be expressed in the form  
\begin{equation}
\label{eq:alphasuff4}
	\alpha (a_1^2 + a_2^2 + a_3^2 - a_4^2,\ 2a_1a_4,\ 2a_2a_4,\ 2a_3a_4,\ 
	 a_1^2 + a_2^2 + a_3^2 + a_4^2)
\end{equation}
for some $a_1,a_2,a_3,a_4,\alpha \in D$.
As we move from a necessary and sufficient condition for Pythagorean tuples 
to a necessary and sufficient condition for rational maps from $\Re^n$ to 
\Sn{n-1}, we notice that $\alpha$ disappears through cancellation 
(compare Lemma~\ref{lem:kubota} and Theorem~\ref{thm:n2}).
As a result, we can state a weaker version of the necessary condition,
with a relaxed restriction on $\alpha$,
since our goal is not a characterization of Pythagorean tuples {\em per se},
but a characterization of rational maps.

\begin{theorem}
\label{thm:necessary4}
Let $D$ be the integers or the polynomials over $\Re[x_1,x_2,x_3,x_4]$.
	% NB: proof not valid for comm. ring, since comm. ring may not have unit elt 1, which is needed immediately
A Pythagorean quintuple over $D$ can be expressed in the form
\begin{equation}
\label{eq:pyth}
	\alpha (a_1^2 + a_2^2 + a_3^2 - a_4^2,
		\ 2a_1a_4,\ 2a_2a_4,\ 2a_3a_4,
		\ a_1^2 + a_2^2 + a_3^2 + a_4^2)
\end{equation}
for some $a_1,a_2,a_3,a_4,\frac{1}{\alpha} \in D$.
\end{theorem}
\prf
Let $(p_1,p_2,p_3,p_4,p_5)$ be a Pythagorean quintuple over $D$.
If $p_1 = p_5$, let $(a_1,a_2,a_3,a_4,\alpha) = (p_1,0,0,0,\frac{1}{p_1})$.
Thus, we may assume without loss of generality that $p_1 \neq p_5$.
Let
\[
(a_1,a_2,a_3,a_4,\alpha) = (p_2,p_3,p_4,p_5-p_1,\frac{1}{2(p_5 - p_1)})
\]
Then $a_1,a_2,a_3,a_4,\alpha$ generate the Pythagorean quintuple
$(p_1,\ldots,p_5)$ as in (\ref{eq:pyth}).
In particular,
\[
\alpha (a_1^2 + a_2^2 + a_3^2 - a_4^2)
= \frac{p_2^2 + p_3^2 + p_4^2 - (p_5 - p_1)^2}{2(p_5-p_1)}
\]
and applying $p_1^2 + p_2^2 + p_3^2 + p_4^2 = p_5^2$,
\[
= \frac{-2p_1^2 + 2p_1p_5}{2(p_5 - p_1)} = p_1.
\]
And
\[
\alpha (2a_i a_4) = \frac{2p_{i+1}(p_5 - p_1)}{2(p_5 - p_1)} = p_{i+1}
\]
for $i=1,2,3$.
Finally, 
\[
\alpha(a_1^2 + a_2^2 + a_3^2 + a_4^2) 
= \frac{p_2^2 + p_3^2 + p_4^2 + (p_5 - p_1)^2}{2(p_5-p_1)} = p_5
\]
Thus, $(p_1,p_2,p_3,p_4,p_5) = \alpha (a_1^2 + a_2^2 + a_3^2 - a_4^2,
		\ 2a_1a_4,\ 2a_2a_4,\ 2a_3a_4,
		\ a_1^2 + a_2^2 + a_3^2 + a_4^2)$.
\QED

A quintuple in the form (\ref{eq:pyth}) is not necessarily
even a quintuple over $\Re[x_1,x_2,x_3,x_4]$: for example, 
$(a_1,a_2,a_3,a_4,\alpha) = (1,1,1,1,\frac{1}{x_1})$ yields the
quintuple $(\frac{2}{x_1}, \frac{2}{x_1}, \frac{2}{x_1}, \frac{2}{x_1}, 
\frac{4}{x_1})$.
Thus, (\ref{eq:pyth}) is not a sufficient condition for Pythagorean quintuples
over $\Re[x_1,x_2,x_3,x_4]$.
However, combined with the slightly different sufficient condition of
Lemma~\ref{lem:suff4}, it successfully yields a characterization
of the rational maps of $\Re^4$ to \Sn{3}.

\begin{theorem}
\label{thm:map4}
A map is a rational map of $\Re^4$ to \Sn{3}\ if and only if
it is of the form:
\begin{equation}
\label{eq:re4s3}
\footnotesize{(x_{\pi(1)},x_{\pi(2)},x_{\pi(3)},x_{\pi(4)})} \mapsto 
\footnotesize{(\frac{a_1^2 + a_2^2 + a_3^2 - a_4^2}{a_1^2 + a_2^2 + a_3^2 + a_4^2},
	 \frac{2a_1a_4}{a_1^2 + a_2^2 + a_3^2 + a_4^2},
	 \frac{2a_2a_4}{a_1^2 + a_2^2 + a_3^2 + a_4^2},
	 \frac{2a_3a_4}{a_1^2 + a_2^2 + a_3^2 + a_4^2})}
\end{equation}
where $a_1,a_2,a_3,a_4 \in \Re[x_1,x_2,x_3,x_4]$
and $\pi : \{1,2,3,4\} \rightarrow \{1,2,3,4\}$ is a permutation.
\end{theorem}
\prf
(Only if): Consider a rational map
of $\Re^4$ to \Sn{3}, $(x_1,x_2,x_3,x_4) \mapsto 
(\frac{f_1}{f_5},\frac{f_2}{f_5},\frac{f_3}{f_5},\frac{f_4}{f_5})$,
where $f_1,\ldots,f_5 \in \Re[x_1,x_2,x_3,x_4]$.
Then $(f_1,\ldots,f_5)$ is a Pythagorean quintuple (Lemma~\ref{lem:iffpyth}).
Thus, the components of the map can be expressed in the normal form
$(f_{\pi(1)},\ldots,f_{\pi(4)},f_5) = \alpha (a_1^2 + a_2^2 + a_3^2 - a_4^2,
2a_1a_4,2a_2a_4,2a_3a_4,$ $a_1^2 + a_2^2 + a_3^2 + a_4^2)$
for some $a_1,a_2,a_3,a_4,\frac{1}{\alpha} \in \Re[x_1,x_2,x_3,x_4]$
and permutation $\pi:\{1,2,3,4\} \rightarrow \{1,2,3,4\}$,
by Theorem~\ref{thm:necessary4} and Lemma~\ref{lem:perm}.
Thus, $(x_{\pi(1)},x_{\pi(2)},x_{\pi(3)},x_{\pi(4)}) \mapsto$
\[
	(\frac{\alpha(a_1^2 + a_2^2 + a_3^2 - a_4^2)}{\alpha(a_1^2 + a_2^2 + a_3^2 + a_4^2)},
	 \frac{\alpha(2a_1a_4)}{\alpha(a_1^2 + a_2^2 + a_3^2 + a_4^2)},
	 \frac{\alpha(2a_2a_4)}{\alpha(a_1^2 + a_2^2 + a_3^2 + a_4^2)},
	 \frac{\alpha(2a_3a_4)}{\alpha(a_1^2 + a_2^2 + a_3^2 + a_4^2)})
\]
\[
= 	(\frac{a_1^2 + a_2^2 + a_3^2 - a_4^2}{a_1^2 + a_2^2 + a_3^2 + a_4^2},
	 \frac{2a_1a_4}{a_1^2 + a_2^2 + a_3^2 + a_4^2},
	 \frac{2a_2a_4}{a_1^2 + a_2^2 + a_3^2 + a_4^2},
	 \frac{2a_3a_4}{a_1^2 + a_2^2 + a_3^2 + a_4^2})
\]
Notice that it is legal to cancel the $\alpha$, since
$\alpha = \frac{1}{\beta}$ for some $\beta \in \Re[x_1,x_2,x_3,x_4]$
which implies that $\alpha$ is never zero.\\
(If): Consider a map of the form (\ref{eq:re4s3}).
This is a rational map of $\Re^4$ to \Sn{3}\ by (\ref{eq:suff4})
and Lemma~\ref{lem:iffpyth}.
\QED

\begin{rmk}
It is tempting to use a variant of Kubota's proof technique 
for Pythagorean triples to prove the necessary condition, Theorem~\ref{thm:necessary4}.
\ifFull
(See Theorem~\ref{thm:kubotaversion} of the appendix, where we develop this
version of the result.)
\fi
However, there are two reasons not to do this.
First, our proof of Theorem~\ref{thm:necessary4} is considerably simpler.
Second, if Kubota's technique is used, the restriction
on $\alpha$ must be relaxed even further, to $\alpha = \frac{\beta_1}{\beta_2}$
where $\beta_1,\beta_2 \in \Re[x_1,x_2,x_3,x_4]$;
that is, $\alpha$ is an arbitrary element of the quotient field of
$\Re[x_1,x_2,x_3,x_4]$.
With this weaker version, there is no longer any guarantee that $\alpha$ is 
nonzero everywhere.
\end{rmk}


\begin{example}
\label{eg:mostnatural}
The most natural choice for $a_i$,
$a_i=x_i$, and the identity permutation
yields the rational map
\[
(x_1,x_2,x_3,x_4) \mapsto 
	(\frac{x_1^2 + x_2^2 + x_3^2 - x_4^2}{x_1^2 + x_2^2 + x_3^2 + x_4^2},
	 \frac{2x_1x_4}{x_1^2 + x_2^2 + x_3^2 + x_4^2}, 
	 \frac{2x_2x_4}{x_1^2 + x_2^2 + x_3^2 + x_4^2}, 
	 \frac{2x_3x_4}{x_1^2 + x_2^2 + x_3^2 + x_4^2})
\]
of $\Re^4$ to \Sn{3}.
We have used this map for the design of quaternion splines \cite{jj95}.
We consider this the most natural map from $\Re^4$ to \Sn{3}.
It is discussed further in \cite{jj98}.
\end{example}

\begin{example}
% Although all of our examples have used simple polynomials for $a_i$
% ($a_i=1$ or $a_i=x_i$),
% this is not necessary, of course.
Using $(a_1,a_2,a_3,a_4) = (0,2x_2,x_1+x_3,x_4-1)$
and the identity permutation,
\footnotesize{
\[
\frac{1}{x_1^2 + 4x_2^2 + x_3^2 + x_4^2 + 2x_1x_3 - 2x_4 + 1}
	(x_1^2 + 4x_2^2 + x_3^2 - x_4^2 + 2x_1x_3 + 2x_4 - 1,
	 0,
	 2x_2x_4-2x_2,
	 2(x_1+x_3)(x_4-1))
\]
}
is a rational map of $\Re^4$ to \Sn{3}.
%
% With $(a_1,a_2,a_3,a_4) = (x_1,x_2^2,x_1+x_3,x_4-x_1)$,
% \[
%	(x_1^2 + x_2^4 + x_3^2 - x_4^2 + 2x_1x_3 + 2x_1x_4,\ 
%	 2x_1x_4-2x_1^2,\ 2x_2^2x_4-2x_1x_2^2,\ 
%	 2x_1x_4 - 2x_1x_3 - 2x_1^2 + 2x_3x_4)
% \]
% is a rational map of $\Re^4$ to \Sn{3}.
\end{example}

\vspace{.5in}

\section{$\Re^n$ to \Sn{n-1}}
\label{sec:nn}

Our result on rational maps from $\Re^4$ to \Sn{3}\ can be generalized
to rational maps from $\Re^n$ to \Sn{n-1}, $n \geq 2$.
The specialization of Euler's Four Squares Theorem (\ref{eq:aida})
was a key component in the previous development,
since it offered a recipe for Pythagorean quintuples.
Fortunately, there is a generalization of this result by
Ammei (also known to Euler) \cite{dickson52}. % p. 318
% (In fact, Euler also knew of this result.) % See p. 318 of Dickson.

\begin{lemma}[Ammei]	% c. 1817
\label{lem:ammei}
\begin{equation}
\label{eq:ammei}
	(a_1^2 + \cdots + a_n^2)^2 = 
	(a_1^2 + \cdots + a_{n-1}^2 - a_n^2)^2 + (2a_1a_n)^2 + \ldots + 
	(2a_{n-1}a_n)^2
\end{equation}
where $a_1,\ldots,a_n$ are elements of any commutative ring, $n \geq 2$.
\end{lemma}

This provides a sufficient condition for Pythagorean $(n+1)$-tuples.
We can again establish a necessary condition with a weaker restriction on
$\alpha$.

\begin{theorem}
\label{thm:necessaryn}
Let $n \geq 2$ and $D$ be the integers or the polynomials 
over $\Re[x_1,\ldots,x_n]$.
A Pythagorean $(n+1)$-tuple over $D$ can be expressed in the form
\begin{equation}
\label{eq:pyth2}
	\alpha (a_1^2 + \ldots + a_{n-1}^2 - a_n^2,
		\ 2a_1a_n,\ldots,\ 2a_{n-1}a_n,
		\ a_1^2 + \ldots + a_n^2)
\end{equation}
for some $a_1,\ldots,a_n,\frac{1}{\alpha} \in D$.
\end{theorem}
\prf
This proof is analogous to that of Theorem~\ref{thm:necessary4}.
Let $(p_1,\ldots,p_{n+1})$ be a Pythagorean $(n+1)$-tuple over $D$.
If $p_1 = p_{n+1}$, let 
$(a_1,\ldots,a_n,\alpha) = (p_1,0,\ldots,0,\frac{1}{p_1})$.
Assume without loss of generality that $p_1 \neq p_{n+1}$.
Let 
\[
(a_1,\ldots,a_n,\alpha) = (p_2,\ldots,p_n,p_{n+1}-p_1,
	\frac{1}{2(p_{n+1}-p_1)})
\]
Then $a_1,\ldots,a_n,\alpha$ generate the Pythagorean $(n+1)$-tuple
$(p_1,\ldots,p_{n+1})$ as in (\ref{eq:pyth2}).
In particular,
\[
\alpha (a_1^2 + \ldots + a_{n-1}^2 - a_n^2)
= \frac{p_2^2 + \cdots + p_n^2 - (p_{n+1}-p_1)^2}{2(p_{n+1}-p_1)}
\]
and applying $p_1^2 + \ldots + p_n^2 = p_{n+1}^2$,
\[
= \frac{-2p_1^2 + 2p_1p_{n+1}}{2(p_{n+1} - p_1)}
= p_1
\]
And
\[
\alpha (2a_i a_n) 
= \frac{2p_{i+1}(p_{n+1}-p_1)}{2(p_{n+1}-p_1)}
= p_{i+1}
\]
for $i=1,\ldots,n-1$.
Finally, 
\[ 
\alpha (a_1^2 + \ldots + a_n^2)
= \frac{p_2^2 + \cdots + p_n^2 + (p_{n+1}-p_1)^2}{2(p_{n+1}-p_1)}
= p_{n+1}.
\]
\QED

This leads to our most general result: 
a full characterization of the rational maps of $\Re^n$ to \Sn{n-1}.

\begin{theorem}
\label{thm:nn}
A map is a rational map of $\Re^n$ to \Sn{n-1}, $n \geq 2$, if and only if
it is of the form:
\[
	(x_{\pi(1)},\ldots,x_{\pi(n)}) \mapsto 
	(\frac{a_1^2 + \cdots + a_{n-1}^2 - a_n^2}{a_1^2 + \cdots + a_n^2},
	 \frac{2a_1a_n}{a_1^2 + \cdots + a_n^2},
	 \ldots,
	 \frac{2a_{n-1}a_n}{a_1^2 + \cdots + a_n^2})
\]
where $a_1,\ldots,a_n \in \Re[x_1,\ldots,x_n]$
and $\pi: \{1,\ldots,n\} \rightarrow \{1,\ldots,n\}$ is a permutation.
\end{theorem}
\prf
Identical to Theorem~\ref{thm:map4}'s proof.
\QED

Theorem~\ref{thm:nn} applies to all $n \geq 2$ and thus represents 
a solution to the cases $n=2,3,4$ already considered.
It agrees with our previous characterizations for $n=2,4$,
but it yields a different characterization for $n=3$.
This is not a contradiction: any rational map
from $\Re^3$ to \Sn{2}\ can be expressed either in the form of 
Theorem~\ref{thm:n3} or in the form of Theorem~\ref{thm:nn} with $n=3$.

It turns out that if we set 
$(a_1,a_2,a_3,a_4) = (\hat{a}_2, \hat{a}_1, 0, \hat{a}_3)$
in the characterization of Theorem~\ref{thm:n3},
the characterization of Theorem~\ref{thm:nn} for $n=3$ is produced
(see Example~\ref{eg:foreshadow}).
Thus, the characterization of Theorem~\ref{thm:nn} is stronger than
Theorem~\ref{thm:n3}: not only can a rational map from $\Re^3$ to \Sn{2}\ 
be put in the form of Theorem~\ref{thm:n3}, it can be put in the form
of Theorem~\ref{thm:n3} with $a_3=0$.

\Comment{
We now have a characterization of all rational maps from $\Re^n$ to \Sn{n-1}.
The related findings on Pythagorean $(n+1)$-tuples are reviewed 
in Table~\ref{tab:pyth}.
The second column of this table gives the normal form for Pythagorean
$(n+1)$-tuples for various $n$.
The third column indicates when this normal form was shown to be a sufficient 
condition for an $(n+1)$-tuple to be a Pythagorean $(n+1)$-tuple 
over the integers.
The fourth column indicates when this normal form was shown to be a necessary 
condition for an $(n+1)$-tuple to be a Pythagorean $(n+1)$-tuple 
over the integers.
The fifth column indicates when this normal form was shown to be a necessary
condition for an $(n+1)$-tuple to be a Pythagorean $(n+1)$-tuple 
over polynomial rings.
The results from this paper are not full proofs (and are thus marked
by stars), since $\frac{1}{\alpha} \in D$, not $\alpha \in D$.

\begin{table}[h]
\label{tab:pyth}
\begin{tabular}{|c|c|c|c|c|}
\hline
$n$ & normal form for Pythagorean $(n+1)$-tuples & suff {\cal Z} & nec {\cal Z} & 
\footnotesize{nec $\Re[x_1,\ldots,x_n]$} \\
\hline
$2$ & \tiny{$\alpha(a_1^2 - a_2^2, 2a_1a_2, a_1^2 + a_2^2)$} &
\footnotesize{classical} & 
\footnotesize{classical} & 
\footnotesize{Kubota 1972} \\ 
\hline
$3$ & \tiny{$\alpha (a_1^2+a_2^2-a_3^2-a_4^2, 2(a_1a_3+a_2a_4), 
    2(a_1a_4-a_2a_3), a_1^2+a_2^2+a_3^2+a_4^2)$} &
\footnotesize{Euler 1748} & 
\footnotesize{Catalan 1885} & 
\footnotesize{Dietz 1993} \\ 
\hline
$4$ & \tiny{$\alpha (a_1^2 + a_2^2 + a_3^2 - a_4^2,
		2a_1a_4, 2a_2a_4, 2a_3a_4,
		a_1^2 + a_2^2 + a_3^2 + a_4^2)$} &
\footnotesize{Euler 1748} & 
\footnotesize{$^*$Johnstone 1998} & 
\footnotesize{$^*$Johnstone 1998} \\
\hline
$n$ & \tiny{$\alpha (a_1^2 + \ldots + a_{n-1}^2 - a_n^2,
	2a_1a_n, \ldots, 2a_{n-1}a_n,
	a_1^2 + \ldots + a_n^2)$} & 
\footnotesize{Ammei 1817} & 
\footnotesize{$^*$Johnstone 1998} & 
\footnotesize{$^*$Johnstone 1998} \\ \hline
\end{tabular}
\end{table}
}

\section{Conclusions}
\label{sec:conclusions}

We have developed a characterization of
rational maps of $\Re^n$ to \Sn{n-1}, $n \geq 2$,
resulting in a normal form for all of these maps.
Aside from their basic geometric interest,
a practical motivation for the study of these maps, and rational maps to more
general surfaces, is the application of these maps to the design of
rational curves on the given surface.
The application of rational maps of $\Re^n$ to \Sn{n-1}\ to the
design of rational quaternion splines is explored in \cite{jj95,jj+jimbo99}.
The design and characterization of rational maps to other surfaces
is a problem for future study.

% Don't open a can of worms.  Leave discussion of this issue to \cite{jj98}.
%
% The most well-known map involving the sphere is stereographic projection.
% In \cite{jj98}, we discuss the relationship of stereographic projection
% to the normal form for rational maps of $\Re^n$ to \Sn{n-1}.
% We also further explore the most natural rational map 
% of $\Re^n$ to \Sn{n-1}\ (Example~\ref{eg:mostnatural}),
% including the way in which it generalizes stereographic projection.

This paper has revealed the strong link between
Euler's Four Squares Theorem and Pythagorean $(n+1)$-tuples,
and hence rational maps of $\Re^n$ to \Sn{n-1}.
% The original characterizations of $n=3,4$
% are derived from special cases of the Four Squares Theorem
% and the general characterization for all $n$ is derived from a generalization
% of one of these special cases.
Although rational maps of $\Re^n$ to \Sn{n-1}\ are strongly interconnected
with Pythagorean $(n+1)$-tuples,
our result does not yet establish a normal form for all
Pythagorean $(n+1)$-tuples,
since we decoupled the necessary and sufficient conditions for
Pythagorean $(n+1)$-tuples,
proving a slightly weaker necessary condition.
A characterization of Pythagorean $(n+1)$-tuples
is tantalizingly close but still an open problem.

\section{Appendix}

For completeness and reference,
we present Kubota's proof of his result on Pythagorean triples.
First, a technical lemma is needed.

\begin{lemma}
\label{lem:applem1}
Let $D$ be a unique factorization domain and $K$ its field of quotients.
Let $a \in D$, $b \in K$.
If $a$ is squarefree and $ab^2 \in D$, then $b \in D$.
\end{lemma}
\prf
Let $b = \frac{m}{n}$ where $m,n \in D$.
$a(\frac{m^2}{n^2}) \in D$ implies $n^2 \ | \ am^2$.
Let $n = n_1 \ldots n_k$ be the prime factorization of $n$,
so $n_1^2 \ldots n_k^2 \ | \ am^2$.
Since $n_i^2 \not | \ a$ ($a$ is squarefree),
$n_i \ | \ m^2$ for all $i$.
Since $n_i$ is prime, $n_i \ | \ m$ for all $i$.
We conclude that $n \ | \ m$ and $b \in D$.
\QED

\begin{lemma}[Kubota 1972]
\label{lem:kubota2}
A triple of polynomials in $\Re[x_1,x_2]$ is a Pythagorean triple 
if and only if it is of the form 
\begin{equation}
\label{eq:triple2}
	\alpha(a_1^2 - a_2^2, 2a_1a_2, a_1^2 + a_2^2)
\end{equation}
for some $a_1,a_2,\alpha \in \Re[x_1,x_2]$.
\end{lemma}
\prf
(If):  A triple of the form (\ref{eq:triple2}) is clearly a Pythagorean triple:
\[
	(a_1^2 - a_2^2)^2 + (2a_1a_2)^2 = 
	a_1^4 + a_2^4 + 2a_1^2a_2^2 = (a_1^2 + a_2^2)^2
\]

(Only if): 
Kubota's proof holds for a more general class: Pythagorean triples
over $D$, where $D$ is a unique factorization domain 
not of characteristic 2, such that 2 is either prime 
or invertible in $D$.
$\Re[x_1,x_2]$ is such a unique factorization domain.
${\cal Z}$ is also such a domain, so this proof also acts
as a proof of Lemma~\ref{lem:classical}.

Let $(p_1,p_2,p_3)$ be a Pythagorean triple over $D$.
If $p_1=p_3$, let $(a_1,a_2,\alpha) = (1,0,p_1)$.
Thus, we can assume $p_1 \neq p_3$.
Factor $p_3 - p_1$ into a square component and a squarefree component:
$p_3 - p_1 = gh^2$, where $g,h \in D$ are squarefree.

If $2 \ | \ g$, let $(a_1,a_2,\alpha) = (\frac{hp_2}{p_3 - p_1}, h, \frac{g}{2}$).
Then $a_1,a_2,\alpha$ generate the Pythagorean triple $(p_1,p_2,p_3)$
as in (\ref{eq:triple2}).
In particular,
\[
\alpha (a_1^2 - a_2^2)
= \frac{g}{2} 
  (\frac{h^2p_2^2 - h^2(p_3 - p_1)^2}{(p_3-p_1)^2})
\]
and applying $p_1^2 + p_2^2 = p_3^2$ and $p_3 - p_1 = gh^2$,
\[
= \frac{1}{2} (\frac{-2p_1^2 + 2p_1p_3}{p_3 - p_1}) = p_1.
\]
And
\[
\alpha (2a_1 a_2) = gh^2 \frac{p_2}{p_3 - p_1} = p_2
\]
Finally, 
\[
\alpha(a_1^2 + a_2^2)
= \frac{g}{2} 
  (\frac{h^2p_2^2 + h^2(p_3 - p_1)^2}{(p_3-p_1)^2})
= \frac{2p_3^2 - 2p_1p_3}{2(p_3-p_1)} = p_3
\]
Thus, $\alpha (a_1^2 - a_2^2, 2a_1a_2, a_1^2 + a_2^2) = (p_1,p_2,p_3)$.
Since $ga_1^2 = 2 \alpha a_1^2 = p_1 + p_3 \in D$, 
$a_1 \in D$ by Lemma~\ref{lem:applem1}.
Clearly $a_2,\alpha \in D$.

If $2 \not | \ g$, let $(a_1,a_2,\alpha) = 
((\frac{hp_2}{p_3 - p_1} + h)/2, \ 
 (\frac{hp_2}{p_3 - p_1} - h)/2, \ g)$.
Then 
\[
\alpha (a_1^2 - a_2^2) = 
\frac{gh^2}{4}[\frac{(p_2+p_3-p_1)^2 - (p_2-p_3+p_1)^2}{(p_3-p_1)^2}] = 
\frac{gh^2}{4}[\frac{4p_2(p_3-p_1)}{(p_3-p_1)^2}] = 
p_2
\]
\[
\alpha (2a_1 a_2) = \frac{g}{2} (\frac{h^2p_2^2}{(p_3-p_1)^2} - h^2) = 
\frac{1}{2}(\frac{p_2^2 - p_3^2 - p_1^2 + 2p_3p_1}{p_3-p_1}) = 
p_1
\]
\[
\alpha(a_1^2 + a_2^2)
= \frac{gh^2}{4} [\frac{2p_2^2 + 2(p_3 - p_1)^2}{(p_3-p_1)^2}] = p_3
\]
Thus, $\alpha (a_1^2 - a_2^2, 2a_1a_2, a_1^2 + a_2^2) = (p_2,p_1,p_3)$.
Since $2 \alpha a_1^2 = p_2 + p_3 \in D$ and $2 \alpha$ is squarefree
($2 \not | \ g$), $a_1 \in D$ by Lemma~\ref{lem:applem1}.
Similarly, since $2 \alpha a_2^2 = p_3 - p_2 \in D$,
$a_2 \in D$ by Lemma~\ref{lem:applem1}.
Clearly $\alpha \in D$.
\QED

\ifFull
The following theorem is the weaker version of the necessary condition
for Pythagorean quintuples that would be developed using Kubota's proof technique.
We prefer Theorem~\ref{thm:necessary4}.

\begin{theorem}
\label{thm:kubotaversion}
Let $D$ be the integers or the polynomials over $\Re[x_1,x_2,x_3,x_4]$.
A Pythagorean quintuple over $D$ can be expressed in the form
\begin{equation}
\label{eq:pythorig}
	\alpha (a_1^2 + a_2^2 + a_3^2 - a_4^2,
		\ 2a_1a_4,\ 2a_2a_4,\ 2a_3a_4,
		\ a_1^2 + a_2^2 + a_3^2 + a_4^2)
\end{equation}
for some $a_1,a_2,a_3,a_4 \in D$ and $\alpha \in K$, where
$K$ is the quotient field of $D$.
\end{theorem}
\prf
Let $(p_1,p_2,p_3,p_4,p_5)$ be a Pythagorean quintuple over $D$.
If $p_1=p_5$, let $(a_1,a_2,a_3,a_4,\alpha) = (1,0,0,0,p_1)$.
Thus, we can assume $p_1 \neq p_5$.
Factor $p_5 - p_1$ into a square component and a squarefree component:
$p_5 - p_1 = gh^2$, where $g,h \in D$ and $g$ and $h$ are both
squarefree.

A direct generalization of the $a_i$ used by Kubota would be
$(a_1,a_2,a_3,a_4,\alpha) = 
(\frac{hp_2}{p_5 - p_1}, \frac{hp_3}{p_5 - p_1}, \frac{hp_4}{p_5 - p_1}, 
h, \frac{g}{2}$).
However, this yields $a_1,a_2,a_3 \in K$:
the trick of using Lemma~\ref{lem:applem1} does not work, 
since $p_1 + p_5 = 2\alpha(a_1^2 + a_2^2 + a_3^2)$,
which is not in the form $ab^2$.
This is unacceptable.
Therefore, we transfer the denominator of $a_1,a_2,a_3$ to $\alpha$.
This will make $\alpha \in K$.

Thus, we let $(a_1,a_2,a_3,a_4,\alpha) = 
(hp_2, hp_3, hp_4, h(p_5 - p_1), \frac{g}{2(p_5 - p_1)^2}$).
Then $a_1,a_2,a_3,a_4,\alpha$ generate the Pythagorean quintuple 
$(p_1,\ldots,p_5)$ as in (\ref{eq:pythorig}).
In particular,
\[
\alpha (a_1^2 + a_2^2 + a_3^2 - a_4^2)
= \frac{g}{2} 
  (\frac{h^2p_2^2 + h^2p_3^2 + h^2p_4^2 - h^2(p_5 - p_1)^2}{(p_5-p_1)^2})
\]
and applying $p_1^2 + p_2^2 + p_3^2 + p_4^2 = p_5^2$ and $p_5 - p_1 = gh^2$,
\[
= \frac{1}{2} (\frac{-2p_1^2 + 2p_1p_5}{p_5 - p_1}) = p_1.
\]
And
\[
\alpha (2a_i a_4) = gh^2 \frac{p_{i+1}}{p_5 - p_1} = p_{i+1}
\]
for $i=1,2,3$.
Finally, 
\[
\alpha(a_1^2 + a_2^2 + a_3^2 + a_4^2)
= \frac{g}{2} 
  (\frac{h^2p_2^2 + h^2p_3^2 + h^2p_4^2 + h^2(p_5 - p_1)^2}{(p_5-p_1)^2})
= \frac{2p_5^2 - 2p_1p_5}{2(p_5-p_1)} = p_5
\]
Thus, $\alpha (a_1^2 + a_2^2 + a_3^2 - a_4^2, 2a_1a_4, 2a_2a_4, 2a_3a_4, 
a_1^2 + a_2^2 + a_3^2 + a_4^2) = (p_1,p_2,p_3,p_4,p_5)$.
Clearly $a_1,a_2,a_3,a_4 \in D$ and $\alpha \in K$.
\QED
\fi




\bibliographystyle{plain}
\begin{thebibliography}{Johnstone \& Williams 95}

\bibitem[Barr et. al. 92]{barr92}
Barr, A. and B. Currin and S. Gabriel and J. Hughes (1992)
Smooth Interpolation of Orientations with Angular Velocity Constraints
using Quaternions.
SIGGRAPH '92, 313--320.

\bibitem[Catalan 1885]{catalan85}
Catalan, E. (1885) Bull. Acad. Roy. Belgique 3(9), p. 531.
Referenced in Dickson, L.E. (1952) History of the Theory of Numbers: Volume II,
Diophantine Analysis.  Chelsea (New York), p. 269.

\bibitem[Dickson 52]{dickson52}
Dickson, L.E. (1952) History of the Theory of Numbers: Volume II,
Diophantine Analysis.  Chelsea (New York).

\bibitem[Dietz et. al. 93]{dietz93}
Dietz, R. and J. Hoschek and B. Juttler (1993)
An algebraic approach to curves and surfaces on the sphere and
on other quadrics.
Computer Aided Geometric Design 10, 211-229.

\bibitem[Euler 1748]{euler48}
Fuss, P., editor (1843) Corresp. Math. et Phys.,
`Correspondance entre Leonhard Euler et C. Goldbach 1729-1763',
St. Petersburg, Vol. 1, p. 452.  
Referenced in Dickson, L.E. (1952) History of the Theory of Numbers: Volume II,
Diophantine Analysis.  Chelsea (New York), p. 277.
% see p. 209, Ebbinghaus, Numbers

\bibitem[Farouki \& Sakkalis 90]{farouki90}
Farouki, R. and T. Sakkalis (1990)
Pythagorean Hodographs.
IBM J. Res. Develop. 34, 736--752.

\bibitem[Herstein 75]{herstein75}
Herstein, I. (1975) Topics in Algebra.
2nd edition, John Wiley (New York).

\bibitem[Hoschek \& Lasser 89]{hoschek89}
Hoschek, J. and D. Lasser (1989)
Fundamentals of Computer Aided Geometric Design.
Translated by L. Schumaker.  A.K. Peters (Wellesley, MA).
% p. 593-594.

\bibitem[Johnstone \& Williams 95]{jj95}
Johnstone, J. and J. Williams (1995)
Rational Control of Orientation for Animation.
Graphics Interface '95, 179--186.

\bibitem[Johnstone 98]{jj98}
Johnstone, J. (1998)
The Most Natural Map to the Sphere.
Technical Report 98-02, CIS Dept., UAB.

\bibitem[Johnstone \& Williams 99]{jj+jimbo99}
Johnstone, J. and J. Williams (1999)
Rational Quaternion Splines.
Technical Report 99-03, CIS Dept., UAB.

\bibitem[Kim et. al. 95]{kim95}
Kim, M.-J. and M.-S. Kim and S. Shin (1995)
A General Construction Scheme for Unit Quaternion Curves with Simple
Higher Order Derivatives.
SIGGRAPH '95, 369--376.

\bibitem[Kim 88]{kim88}
Kim, M.-S. (1988)
Motion Planning with Geometric Models.
Ph.D. thesis, Purdue University.

\bibitem[Kreyszig 59]{kreyszig59}
Kreyszig, E. (1959) Differential Geometry.
Dover (New York).

\bibitem[Kubota 72]{kubota72}
Kubota, K. (1972) Pythagorean triples in unique factorization domains.
American Mathematical Monthly 79, 503--505.

\bibitem[Lozano-Perez 83]{lozanoperez83}
Lozano-Perez, T. (1983)
Spatial Planning: A Configuration Space Approach.
IEEE Trans. on Computers C-32 (2), February, 108--120.

\bibitem[Ono 94]{ono94}
Ono, T. (1994) Variations on a Theme of Euler.
Plenum Press (New York).

\end{thebibliography}

\end{document}
