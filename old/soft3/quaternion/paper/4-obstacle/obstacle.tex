\documentstyle[11pt]{article}
\newif\ifFull
\Fullfalse
\input{header}
\SingleSpace

\setlength{\oddsidemargin}{0pt}
%\setlength{\topmargin}{-.25in}	% technically should be 0pt for 1in
\setlength{\headsep}{3em}
\setlength{\textheight}{8.75in}
\setlength{\textwidth}{6.5in}
\setlength{\columnsep}{5mm}		% width of gutter between columns

\markright{Orientation obstacles: \today \hfill}
\pagestyle{myheadings}

% -----------------------------------------------------------------------------

\title{Avoiding Obstacles in Orientation Space}
\author{J.K. Johnstone}

\begin{document}
\maketitle

\begin{abstract}
We will study the addition of an obstacle-avoidance constraint 
to the design of curves on a surface.
\end{abstract}

\vspace{1in}

\noindent Keywords: 

\clearpage

\section{Introduction}

In some applications, we need to design 
curves on a surface that avoid certain regions of the surface.
For example, consider the design of quaternion splines on \Sn{3},
which are used to control the orientation of an object.

Some orientations may be illegal: an orientation of a cup
that spills its contents, an orientation of a camera that turns
the world upside-down, or an orientation
that is outside the kinematic constraints of a robot arm.
These forbidden orientations can be modeled as forbidden regions
(or obstacles) on the surface \Sn{3}.
In general, portions of a surface may be undesirable:
for example, visible regions of a car body may be undesirable
for the design of functional objects that should be invisible.

% Note: orientation constraints arising from kinematic and carrying constraints
% (maintaining general free space flavor of motion planning:
% that is, constraints inherent to the arm, not to the environment).

The avoidance of obstacles in the plane has received much attention,
especially for polygonal obstacles (e.g., \cite{lozano79,hersh88,ghosh91}).
Some of these techniques can be lifted to a Riemannian context.
For example, consider the classical visibility graph algorithm for polygons
in 2-space.
A basic idea that can be generalized is that of following tangents
to the obstacles and common tangents between obstacles.
However, the tangent or line-of-sight must now be translated to a Riemannian 
context on a surface.
By adding intermediate points in this way, and using convex-hull
properties of Bezier curves and B-splines, obstacles could be avoided.

Generalization of Euclidean motion planning algorithms to a surface.

\section{An example of an obstacle on \Sn{3}}

\begin{itemize}
\item
Claim: great sphere of \Sn{3}\ represents orientations orthogonal to a given
	orientation.
\item	
Consider obstacles on \Sn{3} bounded by great spheres.
\item
Compute common tangents of great spheres on \Sn{3}.\
\end{itemize}

\section{Example on a real application}

What is a naturally occuring sophisticated orientation obstacle?
Moving mug of beer is too simple: only one hemispherical obstacle.


\bibliographystyle{plain}
\begin{thebibliography}{Johnstone \& Williams 95}

\bibitem[Ghosh 91]{ghosh91}
Ghosh, S. and D. Mount (1991)
An Output-Sensitive Algorithm for Computing Visibility Graphs.
SIAM J. Computing 20(5), 888-910.

\bibitem[Hershberger 88]{hersh88}
Hershberger, J. and L. Guibas (1988)
An $O(n^2)$ Shortest Path Algorithm for a Non-Rotating Convex Body.
Journal of Algorithms 9, 18--46.

\bibitem[Johnstone \& Williams 95]{jj95}
Johnstone, J. and J. Williams (1995)
Rational Control of Orientation for Animation.
Graphics Interface '95, 179--186.

\bibitem[Johnstone \& Williams 99]{jj+jimbo99}
Johnstone, J. and J. Williams (1999)
Rational Quaternion Splines.
Technical Report 99-03, CIS Dept., UAB.

\bibitem[Lozano-Perez 79]{lozano79}
Lozano-Perez, T. and M. Wesley (1979)
An Algorithm for Planning Collision-Free Paths among Polyhedral Obstacles.
CACM 22(10), 560--570.

\bibitem[Lozano-Perez 83]{lozano83}
Lozano-Perez, T. (1983)
Spatial Planning: A Configuration Space Approach.
IEEE Transactions on Computers, C-32 (2), February, 108--120.

\end{thebibliography}

\end{document}
