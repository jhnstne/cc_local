\documentclass[12pt]{article} 
\usepackage{times}
\usepackage[pdftex]{graphicx}
\makeatletter
\def\@maketitle{\newpage
 \null
 \vskip 2em                   % Vertical space above title.
 \begin{center}
       {\Large\bf \@title \par}  % Title set in \Large size. 
       \vskip .5em               % Vertical space after title.
       {\lineskip .5em           %  each author set in a tabular environment
        \begin{tabular}[t]{c}\@author 
        \end{tabular}\par}                   
  \end{center}
 \par
 \vskip .5em}                 % Vertical space after author
\makeatother

% default values are 
% \parskip=0pt plus1pt
% \parindent=20pt

\newcommand{\SingleSpace}{\edef\baselinestretch{0.9}\Large\normalsize}
\newcommand{\DoubleSpace}{\edef\baselinestretch{1.4}\Large\normalsize}
\newcommand{\Comment}[1]{\relax}  % makes a "comment" (not expanded)
\newcommand{\Heading}[1]{\par\noindent{\bf#1}\nobreak}
\newcommand{\Tail}[1]{\nobreak\par\noindent{\bf#1}}
\newcommand{\QED}{\vrule height 1.4ex width 1.0ex depth -.1ex\ \vspace{.3in}} % square box
\newcommand{\arc}[1]{\mbox{$\stackrel{\frown}{#1}$}}
\newcommand{\lyne}[1]{\mbox{$\stackrel{\leftrightarrow}{#1}$}}
\newcommand{\ray}[1]{\mbox{$\vec{#1}$}}          
\newcommand{\seg}[1]{\mbox{$\overline{#1}$}}
\newcommand{\tab}{\hspace*{.2in}}
\newcommand{\se}{\mbox{$_{\epsilon}$}}  % subscript epsilon
\newcommand{\ie}{\mbox{i.e.}}
\newcommand{\eg}{\mbox{e.\ g.\ }}
\newcommand{\figg}[3]{\begin{figure}[htbp]\vspace{#3}\caption{#2}\label{#1}\end{figure}}
\newcommand{\be}{\begin{equation}}
\newcommand{\ee}{\end{equation}}
\newcommand{\prf}{\noindent{{\bf Proof}:\ \ \ }}
\newcommand{\choice}[2]{\mbox{\footnotesize{$\left( \begin{array}{c} #1 \\ #2 \end{array} \right)$}}}      
\newcommand{\scriptchoice}[2]{\mbox{\scriptsize{$\left( \begin{array}{c} #1 \\ #2 \end{array} \right)$}}}
\newcommand{\tinychoice}[2]{\mbox{\tiny{$\left( \begin{array}{c} #1 \\ #2 \end{array} \right)$}}}
\newcommand{\ddt}{\frac{\partial}{\partial t}}
\newcommand{\Sn}[1]{\mbox{{\bf S}$^{#1}$}}
\newcommand{\calP}[1]{\mbox{{\bf {\cal P}}$^{#1}$}}

\newtheorem{theorem}{Theorem}	
\newtheorem{rmk}[theorem]{Remark}
\newtheorem{example}[theorem]{Example}
\newtheorem{conjecture}[theorem]{Conjecture}
\newtheorem{claim}[theorem]{Claim}
\newtheorem{notation}[theorem]{Notation}
\newtheorem{lemma}[theorem]{Lemma}
\newtheorem{corollary}[theorem]{Corollary}
\newtheorem{defn2}[theorem]{Definition}
\newtheorem{observation}[theorem]{Observation}

% \font\timesr10
% \newfont{\timesroman}{timesr10}
% \timesroman


\setlength{\oddsidemargin}{0pt}
\setlength{\topmargin}{0in}
\setlength{\textheight}{8.6in}
\setlength{\textwidth}{6.875in}
\setlength{\columnsep}{5mm}
% \markright{Sweeping frustra (\today) \hfill}
% \pagestyle{myheadings}

% -----------------------------------------------------------------------------
\title{Tech report 2: Sweeping view frustra along rational motions: blind motion planning}
% Johnstone and Williams
\begin{document}
\maketitle

\begin{abstract}
Think camera control in CSI, Las Vegas, Smallville, Final Fantasy.
Think automated motion planning through a virtual scene.
\end{abstract}

Suppose that we have the capacity to build a rational interpolating quaternion spline.
\begin{itemize}
\item Define the Bezier paths of the four corners of the back view frustrum.
      (we know the offset vector from the center, explored in drawCameraWindowGlobal)
      (Eurographics 1995 paper gave us the Bezier representation of the quaternion spline
       as rotation matrix: degree 12 spline)
      (this should yield the answer using P(t) + O(t)*offsetVector)
\item Intersect the Bezier paths with the polyhedral scene.
      (should we intersect with individual triangles? no, too many triangles)
      (should we gather mesh into a Brep? not the highest priority)
      (we need to store the mesh in a spatial decomposition like an octree)
      (how expensive is Bezier curve intersection detection with a triangle? this is
       a generalization of ray casting, perhaps called curve casting)
\end{itemize}

Challenges: 
\begin{itemize}
\item octree spatial decomposition for triangular mesh (may be available already)
\item development of other mesh formats, especially architectural
\item Vienna city scene, New Orleans mesh
\item collection/editing tool for orientation keyframes
\item orientation constraints
\item bitangents on S3 for shortest path around obstacles
\item very interesting problem: smooth interpolation of points while avoiding obstacles
      (examined in SIGGRAPH 2004 paper but punted by using optimization)
\item motion planning through a visibility structure!!
\end{itemize}

Tangential curve challenges:
\begin{itemize}
\item smooth soft shadows
\item release of the software for tangential curves and surfaces
\item revisit this in class
\end{itemize}

\section{Vertex paths}

The path followed by a vertex of the camera (back view frustrum)
under a rational motion may be defined by a rational Bezier curve.
We now develop the details of this rational Bezier representation of the vertex path.
The position curve $P(t)$ defines the path followed by the center of the camera 
(back view frustrum).
This is a polynomial path built through standard interpolation of the keyframe positions.
Let $V$ be the vector from the center to a corner of the back view frustrum.
The rational quaternion spline $Q(t)$ defines the camera's orientation along the path.
It may be translated to a parameterized rotation matrix $O(t)$,
represented in the Bernstein polynomial basis (see Eurographics paper \cite{jjjimbo95b}).
The vertex path may be represented by $P(t) + O(t)V$, which may be expressed
in the Bernstein basis, and thus as a rational Bezier curve.
In particular, suppose that the positions and orientations of the keyframes
are $\{P_i\}_{i=0}^n$ and quaternions $\{Q_i\}_{i=0}^n$.

Extract relevant parts of Euro95: Lemma 6.2 and variant of Theorem 7.1

\section{Vertex path intersection}

Intersection of rational Bezier curve and triangle.

Identification of the relevant triangles using a spatial decomposition.

Consider the intersection of a rational Bezier curve and a triangle.
The polyhedral mesh should be entered into a spatial decomposition (octree or BSP, probably octree)
with leaves containing faces (possibly split into subfaces to accomodate better
storage in octree).
The curve is then subdivided into the octree.
Since we are only interested in the first intersection, not all intersections,
we first consider the closer half of each subdivision, only descending to the further
half when it has been determined that the first half does not intersect.
The curve is subdivided and passed down the octree until it is small enough
or until there are a small number of triangles in the octree cell.
Once we have identified a small number of triangles to intersect with a small curve segment, the curve/triangle intersection is done as follows.

Get an axis-aligned bounding box (ABB) of the triangle.
Get an ABB of the curve segment.
Subdivide until ABB of curve is small enough (and maximum curvature is low: to avoid
missing an intersection from a very small loop).
Replace the curve by a line segment between its two endpoints.
Intersect the line segment with the plane of the triangle.
Then, if an intersection exists, test it against the three edges of the triangle.
Notice that the curve-triangle intersection may be parallelized (although we will
only have a small number of these intersections if the octree decomposition is done 
correctly).

\section{Rational camera control}

Consider the following camera control problem.
Given certain desired viewpoints for a camera, we are to move the camera through the scene
so that it hits these predefined cues.
Suppose that the camera path is designed as if the motion is in free space,
which is a much simpler motion plan.
As the camera moves along the computed path, it will collide with the scene 
(i.e., the view will cut through some of the scene).
We want to detect these camera collisions (and eventually correct them).

% Application: camera control through a scene, given certain desired viewpoints.
%    i.e., you specify certain viewpoints in a scene from which to build a flythrough
%    that hits certain cues.
% Issue: camera is not an object like typical animations;
%        however, orientation is a key input to a camera, and collision is well defined:
% Source of positions: camera viewpoints at hit points.
% Source of orientations: camera viewpoints at hit points.

This camera collision detection problem is formalized as follows.
The position of the camera is the center of the back of the frustrum.
The orientation of the camera is the normal to the frustrum's back plane.
A bad camera configuration is one whose back plane of the viewing frustrum 
intersects the scene (the side planes will almost always do so, with no ill effects).
We only have to keep track of the back plane of the view frustrum: this makes possible
a swept envelope approach, as follows.
Consider the back plane of the view frustrum moving through space with a moving camera.
We are interested in its intersections with the scene, which reflect ugly cutthroughs
of the scene in the rendering of the flythrough.
The four edges of the quadrilateral defining the back plane define four ruled surfaces
as they sweep through space.
These are rational ruled surfaces when the motion is rational.
We could consider the intersection of these four ruled surfaces with the scene,
but we can simplify further.
Suppose that the back plane of the view frustrum starts out in a free position,
not intersecting the scene.
Then one of its line segments can intersect the scene if and only if one of its vertices
(an endpoint of the line segment) intersects the scene.
Moreover, the vertex intersection detects the initial collision.
Therefore, it is sufficient to test the intersection of the four vertex curves
(the four rational curves defined by the motion of the four vertices of the back plane
of the view frustrum) with the scene.
Intersection points reveal collisions where the motion must be altered.
In short, a camera path is safe iff the 4 rational endcurves of its view frustrum
backplane do not intersect the scene (where the camera starts out safe).

% WE NEED RULED SURFACE INTERSECTION?: 
Since a scene object could interfere with a frustrum edge without
interfering with a frustrum vertex, we must compute the intersection of the moving edge
not the moving vertex, and hence the intersection of a ruled surface with the scene
rather than a curve with the scene.
% Therefore, think of ruled surface intersection methods.

Initial challenges:
\begin{itemize}
\item find realistic hit points;
      how close do they have to be to allow dumb interpolation rather than
      smart motion planning?
\item find 4 rational endcurves of view frustrum from position/orientation 
\item verify that motions generated from these keyframes are locally perturbable
      to collision-free motions
\end{itemize}

Another interpretation of this paper: DUMB OR BLIND MOTION PLANNING (this 
strongly motivates rational motion).
Collision detection using the rational swept volumes generated by the rational motions;
this would be a 'blind' replacement of the keen-sighted, or general, motion planner 
proposed below, which incorporates the environment into its planning;
dumb or blind in the sense that a motion is generated (with no knowledge of obstacles)
and tested for collision, then regenerated
based on the found collision and retested, etc.;  it is a valid strategy for local
motions that are not expected to require large adjustments; the global motion, at a
crude level, would be built by a different planner.
% A rational motion is defined for a rigid object by interpolating 
% positions/orientations; these orientations are found how?

\section{Related work}

Drucker et. al. \cite{drucker92} on procedural camera movements.
Symposium on interactive 3d graphics 1992, p. 67-70.
See Shoemake's citings for various papers on camera control.

\section{Followup paper}

How should the motion be altered when a collision is detected?
Once a collision is detected, there are several options to correct the motion.
As a first crack, assume that a small perturbation will correct the motion:
that is, the distance of our present path from a safe path is small.
If a collision is detected between keyframes i and i+1, we could perturb i,
we could perturb i+1, or we could add a new keyframe local to the collision.
We prefer the latter option, but there are various options here.
We will start by using the first collision only, and move perpendicular to the
present curve's tangent at this collision, which seems to correct in the best
direction to avoid a collision (we don't want to move in the present tangent direction).
The direction in this normal plane could be determined using which vertex of the four
back vertices has intersected: we could move within the normal plane 
towards the nonintersecting vertices.  
Alternatively, we could intersect the entire backplane quadrilateral with the scene
and determine a motion in this plane of the quadrilateral (which is also the normal plane)
that is guaranteed to remove collision.
Experiments with these approaches are necessary.

Flowchart: generation of flythrough $\rightarrow$ collision detection $\rightarrow$ 
path correction.
We could generate better paths by forcing paths down center of halls by shrinking
the corridors inward a set amount.  This shrinking could be removed if the path
correction gets too difficult.

6/22/05 thoughts:
At present, we are finding the intersection of ruled surface triangles 
(the ruled surface defined by an edge of the frustrum), of which there are 8 between each pair
of frustrum samples, since there are 4 edges.
By adding four more triangles, two each that define the frustrum at each sample, we can find
intersections with the face of the frustrum as well as the vertices and edges, which detects
the unusual but certainly possible 
collision where the scene enters through the frustrum face without touching the edges, as
in a camera that is taking a closeup of a concave protrusion from the wall and gets too close,
shattering the camera lens.  This solves this nagging problem.

The first approach to collision correction is the introduction of new keyframes in the 
region of a collision.
We will detect the entire interval of a continuous collision, 
and measure the depth of collision along this interval to find the maximum depth.
Depth is measured as the maximum distance of a colliding vertex from a scene face that the
frustrum intersects.
This depth dictates the size of the correction.
The time location of the correction depends on the length of the collision interval:
if the interval is short, a new keyframe is introduced at the midpoint of the interval,
or possibly at the point of deepest collision;
if the interval is long, new keyframes are introduced at regular distances along the 
collision interval.
At the time location of the correction, the direction of correction is chosen as an 
average of the vectors from the colliding vertices (at this time location) to the frustrum center.

The time location, size and direction of the corrections are all a matter of experimentation.
We also need to try techniques where orientation is changed at the introduced keyframes.
At the moment, a keyframe introduced at time $t'$ will be given orientation $Q(t')$,
where $Q(t)$ is the quaternion spline.

\section{Future papers}

\begin{itemize}
\item Orientation planning with orientation constraints on camera.

Equivalently, ACTIVE ORIENTATION OBSTACLES.
Design of quaternion splines that avoid active orientation obstacles
(and the definition of these orientation obstacles using quaternion splines);
an active orientation obstacle is an obstacle based on the movement of the robot,
not on the environment, such as not spilling milk;

See obstacle.tex

\item high level motion algorithm (decomposition of scene into cells)

\item PASSIVE ORIENTATION OBSTACLES AND INTELLIGENT PLANNING.
Generation of collision-free motions, using
passive orientation obstacles defined by the environment; this is harder
and would solve a problem mentioned way back in Lozano-Perez.

\item camera or object paths that are sensitive to the visibility structure of the scene
\item camera paths that remain invisible to a certain light or object
\item camera paths that keep a certain object visible
\end{itemize}

Ross reading:
\begin{itemize}
\item my quaternion spline papers
\item * my Bezier curve software
\item Bezier intersection papers, ruled surface intersection paper
\item rational Bezier
\item * introduction to robot motion (Latombe)
\item give me game design examples, TV examples
\item * build more architectural flythroughs
\item what underlying data structure to store the scene?
\end{itemize}

%%%%%%%%%%%%%%%%%%%%%%%%%%%%%%%%%%%%%%%%%%%%%%%%%%%%%%%%%%%%%%%%%%%%%%%%%%%%%%%%%%%%%

\section{Introduction}

\subsection{Coordination of position and orientation in the motion}

{\bf Elaborate on this simple coordination strategy in next motion paper:}
DON'T USE SAME KNOT SEQUENCE FOR POSITION AND ORIENTATION CURVES.
DEFINE CURVES TOTALLY INDEPENDENTLY AND USE KEYFRAME CORRESPONDENCE TO
COORDINATE MOTION.
FOR EXAMPLE, IF POSITION KEYFRAMES HAVE KNOTS 0,1,2,3,4,5 AND 
ORIENTATION KEYFRAMES HAVE KNOTS 0 .1 .6 10 20 21, THEN
SAMPLE AT 3.5 ON POSITION CURVE CORRESPONDS TO SAMPLE AT 15 ON ORIENTATION CURVE.

\clearpage

% written May 11, 2005
\section{}

The motion is described by a changing position and a changing orientation,
which are represented by a position spline and a quaternion spline respectively.
The position spline is built from position keyframes, and the quaternion spline
from orientation keyframes.
Suppose that at some stage of the motion, the position is P = (p1,p2,p3) and the
orientation is Q = (q1,q2,q3,q4).
This quaternion is translated to a rotation matrix M.
The quaternion or rotation matrix represents the change in orientation undergone by
the object from a canonical orientation.

Suppose that the motion is of a camera.
Then gluLookAt can be used to encode the motion.
gluLookAt depends on an eye position, a direction vector, and an up vector.
The eye position is P.
% what is the interpretation of the orientation? change of orientation from canonical?
Canonically, the camera looks down the negative z-axis.
Under a rotation by M, the third row is transformed into the z-axis (see ROCOCO),
so the direction vector is the negated third row of M.
Canonically, the camera's up vector is the y-axis.
Under a rotation by M, the second row is the preimage of the y-axis (see ROCOCO),
so the up vector is the second row of M.

\end{document}
