\documentclass[12pt]{article} 
\usepackage{times}
\usepackage[pdftex]{graphicx}
\makeatletter
\def\@maketitle{\newpage
 \null
 \vskip 2em                   % Vertical space above title.
 \begin{center}
       {\Large\bf \@title \par}  % Title set in \Large size. 
       \vskip .5em               % Vertical space after title.
       {\lineskip .5em           %  each author set in a tabular environment
        \begin{tabular}[t]{c}\@author 
        \end{tabular}\par}                   
  \end{center}
 \par
 \vskip .5em}                 % Vertical space after author
\makeatother

% default values are 
% \parskip=0pt plus1pt
% \parindent=20pt

\newcommand{\SingleSpace}{\edef\baselinestretch{0.9}\Large\normalsize}
\newcommand{\DoubleSpace}{\edef\baselinestretch{1.4}\Large\normalsize}
\newcommand{\Comment}[1]{\relax}  % makes a "comment" (not expanded)
\newcommand{\Heading}[1]{\par\noindent{\bf#1}\nobreak}
\newcommand{\Tail}[1]{\nobreak\par\noindent{\bf#1}}
\newcommand{\QED}{\vrule height 1.4ex width 1.0ex depth -.1ex\ \vspace{.3in}} % square box
\newcommand{\arc}[1]{\mbox{$\stackrel{\frown}{#1}$}}
\newcommand{\lyne}[1]{\mbox{$\stackrel{\leftrightarrow}{#1}$}}
\newcommand{\ray}[1]{\mbox{$\vec{#1}$}}          
\newcommand{\seg}[1]{\mbox{$\overline{#1}$}}
\newcommand{\tab}{\hspace*{.2in}}
\newcommand{\se}{\mbox{$_{\epsilon}$}}  % subscript epsilon
\newcommand{\ie}{\mbox{i.e.}}
\newcommand{\eg}{\mbox{e.\ g.\ }}
\newcommand{\figg}[3]{\begin{figure}[htbp]\vspace{#3}\caption{#2}\label{#1}\end{figure}}
\newcommand{\be}{\begin{equation}}
\newcommand{\ee}{\end{equation}}
\newcommand{\prf}{\noindent{{\bf Proof}:\ \ \ }}
\newcommand{\choice}[2]{\mbox{\footnotesize{$\left( \begin{array}{c} #1 \\ #2 \end{array} \right)$}}}      
\newcommand{\scriptchoice}[2]{\mbox{\scriptsize{$\left( \begin{array}{c} #1 \\ #2 \end{array} \right)$}}}
\newcommand{\tinychoice}[2]{\mbox{\tiny{$\left( \begin{array}{c} #1 \\ #2 \end{array} \right)$}}}
\newcommand{\ddt}{\frac{\partial}{\partial t}}
\newcommand{\Sn}[1]{\mbox{{\bf S}$^{#1}$}}
\newcommand{\calP}[1]{\mbox{{\bf {\cal P}}$^{#1}$}}

\newtheorem{theorem}{Theorem}	
\newtheorem{rmk}[theorem]{Remark}
\newtheorem{example}[theorem]{Example}
\newtheorem{conjecture}[theorem]{Conjecture}
\newtheorem{claim}[theorem]{Claim}
\newtheorem{notation}[theorem]{Notation}
\newtheorem{lemma}[theorem]{Lemma}
\newtheorem{corollary}[theorem]{Corollary}
\newtheorem{defn2}[theorem]{Definition}
\newtheorem{observation}[theorem]{Observation}
\newtheorem{implementation}[theorem]{Implementation note}

% \font\timesr10
% \newfont{\timesroman}{timesr10}
% \timesroman


\setlength{\oddsidemargin}{0pt}
\setlength{\topmargin}{0in}
\setlength{\textheight}{8.6in}
\setlength{\textwidth}{6.875in}
\setlength{\columnsep}{5mm}

\title{Publication goals}
\begin{document}
\maketitle

% -------------------------------------------------------------------------------------

\section{Overarching, long-term goals}

\begin{itemize}
\item Visibility in a smooth scene.
\item {\bf Motion planning: given a 3D scene filled with objects, to plan a good
           motion through this scene.  Examples of scenes are architectural buildings
           and cityscapes.  Good may mean collision-free or it may mean optimizing
           a visibility criterion.}
\item (Smooth reconstruction [mostly for scene population].)
\item Architectural models.
\item (Game design, as a way of packaging these research topics.)
\end{itemize}

An attraction of motion planning research is that it allows the construction
of smooth paths through a polyhedral scene.
It does not require the development of a smooth scene as testbed for research.

Individual tasks.
Given a Unigrafix model (or WRL or ...), parse and display it (ugviewerGUI.cpp, wrlviewer.cpp).
Given a scene, hand-generate keyframes of a motion through it (keyfly.cpp).
[Given a scene, build a geometric model of it sufficient for motion planning.]
[Given a scene, automatically generate keyframes for a motion through it, guided by 
    certain hitpoints, such as go counterclockwise around the outer hall, then down
    the central hall and fly out the window to the roof.]
Given keyframes for a motion, generate the motion (flythrough.cpp).
Given a motion, perturb it into a collision-free motion.
Given a motion, correct it to avoid illegal orientations.
Given an object that is to be avoided, build a motion through a scene that 
hides from this object.

\section{Handling stasis and linearity in curve design}

We will not assume that consecutive orientation keyframes are distinct,
and we will simply keep the quaternion spline constant over this interval.
In practice, we will remove the duplicate entries during construction 
of the quaternion spline, then use bookkeeping to keep track of the 
trivial segments where the quaternion does not change.
START HERE
An issue arises: if the orientation stays constant for a while, do we want
to build a separate spline for each nonstationary segment, rather than building
a complete spline that, for example, anticipates future motions as it approaches
a static segment.
Should the orientation be moving in this way to anticipate future developments?
I would argue no.
Consider the case of a position motion between three points, the first and third on
the same side of the second point.
A straight line between A and B, then a pause, then a straight line between B and C
would seem OK.
The pause removes the discontinuity of motion normally noticeable in a non-paused motion.

We presently assume that consecutive position keyframes are distinct.
{\bf We should allow consecutive identical position keyframes in the same way.}

The issue of adding linear segments to a curve interpolation
(that is, the data is points and linear segments between some of these points),
which arises in the interpolation of mostly-smooth architectural data,
has a straightforward solution: the linear segment becomes an endtangent constraint
on the set of points before the linear segment, and an endtangent constraint
on the set of points after the linear segment.
The set of points before and after the linear segment are interpolated independently.

\clearpage

\section{Projects}

\begin{enumerate}
\item Visibility.
\begin{itemize}
\item Smooth bitangency: construction of curve bitangents and 
      surface bitangent developables [parallel construction, software release]
\item Visibility analysis of smooth scenes: visual events, cell decomposition, 
      shadow computation, occlusion culling [paper on 2d visibility next, 
      then 3d visibility among spheres]
\item Total characterization of the visibility of every point of a smooth scene.
\end{itemize}
\item Motion planning
\begin{itemize}
\item {\bf Rational quaternion splines [characterization of rational maps, 
      furthest point issue, generalized cross product, software release]}
\item Rational collision detection [camera's frustrum path]
\item Orientation planning [smooth orientation obstacles, bitangents]
\item Motion planning in a smooth scene [scene decomposition]
\end{itemize}
\item Miscellaneous
\begin{itemize}
\item Molecular modeling for Jere Segrest.
\item Kernels/hulls.
\item Silhouettes.
\item Bitangents of subdivision curves.
\item Ruled surface intersection.
\end{itemize}
\end{enumerate}

The goal of our motion research is {\bf rational motion}.
The first issue is the design of a {\bf rational interpolating quaternion spline},
for rational orientation control to complement classical techniques for rational position control.
A rational approach to the design of an interpolating quaternion spline involves the
{\large {\bf construction of a rational map to the 3-sphere }} and {\bf the construction of an empty circle}
to avoid robustness breakdown around a pole.
WHAT IS THE SWEET SPOT OF DATASET SIZE FOR EFFICIENT BRUTE FORCE EMPTY POINT SOLUTION?
An alternative to the construction of an optimal empty circle is a {\bf divide and conquer 
approach} to the quaternion spline design, in which care must be taken with 
{\bf derivative continuity} across the boundary between two quaternion splines.
Computation in the 4 dimensions of the quaternion spline also involves a 
{\large {\bf 222 generalized cross product 222}}.
CAN WE GENERALIZE THE SIN RELATIONSHIP FOR OUR GENERALIZED CROSS PRODUCT?
To test the method, we would like to extract (position and) 
{\bf orientation keyframes from mocap libraries} to test the interpolation of 
realistic motion keyframes.
GOOD PROJECT FOR AMBER AND ROSS: GIVE THEM THE CMU AND UWISC WEBSITES.
We now have the capacity for rational interpolating motion.

The next issue is rational interpolating {\bf orientation-constrained} motion.
Given an object with orientation constraints (e.g., a pitcher of milk),
the problem is to build a motion that respects these constraints.
Our approach is first to {\large {\bf 444 formalize the orientation constraints 444}}, 
then {\bf build orientation obstacles} on the quaternion sphere 
that represent the constraints;
then to {\bf build a quaternion spline on the quaternion sphere that avoids these obstacles},
which is motion planning on a surface.
A simplified related question is the {\bf interpolation of points in the plane that respect/avoid
certain obstacles}.  This had a pseudo-answer in SIGGRAPH 2004(?) using optimization of 
a distance penalty, but we are looking for a non-iterative solution.
A shortest path motion involves the {\bf computation of bitangents on the 3-sphere}
and the construction of a visibility graph there.
A safest path motion involves the {\bf computation of curve bisectors on the 3-sphere}.
We also want to consider non-optimal solutions (neither shortest nor safest).
We now have the capacity for rational interpolating orientation-constrained motion.

The next issue is collision-free motion through a polyhedral scene,
which integrates the free motion with its environment.
This is subtly different from interactive collision detection under mouse control,
where local motions that cause collision are disallowed.
We are considering the global design of a motion that avoids collisions.
The presence of the rational motion makes collision detection simpler.
In our case, we consider architectural models as the polyhedral scene.
(A preprocessing step is the {\bf parsing of polyhedral scenes into triangle data structures}.)
The first case that we consider is {\bf camera motion} (camera control),
where collision is simplified since it must only be detected with the backplane of the 
view frustrum (explain).
Given a rational camera motion through a polyhedral scene,
we first {\large {\bf 000 detect collisions 000}} with the scene (using an octree for additional efficiency).
(Make sure that we are using the full power of the rational representation of the motion
to add efficiency.)
Next, the {\bf motion is corrected} to remove the collisions.
This model assumes a relatively free environment where local changes do not introduce 
new collisions on the other side of the room, and an initial motion that is close to 
collision-free so that only minor corrections are necessary.
For example, the individual keyframes from which the motion is generated should be collision-free.

Rather than building the motion blindly and then testing for collisions,
a more elegant approach would be design a motion that respects the free space of the scene
in the first place.
However, this requires a model of the free space and 6-dimensional path planning through
configuration space.

Rather than keyframing by hand, keyframes could be generated automatically if a global structure
of the scene's free space is understood.
We could {\bf identify rooms} in the scene and generate a graph structure of the scene.
Then a motion could be specified simply by the desired start and end rooms.

The last major issue is {\bf rational visibility-preserving motion} and
rational visibility-guided motion.
The former keeps some object visible throughout the motion.
The latter plans a motion guided by the desire to see part or all of certain objects in the scene
at some point during the motion (or perhaps to guarantee that every corner of a room is seen 
in a surveillance sweep).

The goal of our {\bf visibility research} is the recognition and categorization of visual events
in a smooth scene.
Visual events can occur when bitangency occurs, so we first build {\bf tools for the construction
of bitangent structures}, both in 2-space and 3-space.
This is tangential curves and surfaces.
Next we show (trivially) how these tangential curves and surfaces are used to build bitangent
structures.
Bitangency is a necessary but not sufficient condition for a visual event, so the next stage
is to {\large {\bf 222 filter the bitangents down to visual events 222}}.
We first consider visual events amongst curves in 2-space.
An extreme case of visibility is the kernel, which sees everything: we consider the construction
of {\large {\bf 333 kernels in 2-space 333}}, and reduce it to the convex hull.
We consider the construction of {\bf kernels in 3-space}, which is more difficult to compute.
The capstone of this work is the computation of visual events amongst surfaces in 3-space.

Review of papers:

\begin{itemize}
\item construction of a rational map to the 3-sphere
\item generalized cross product 
\item detect collisions 
\item filter the bitangents down to visual events 
\item kernels in 2-space
\item formalize the orientation constraints 
\end{itemize}

\clearpage

\section{Immediate goals}

\begin{enumerate}
\item Goal: journal paper on spherical map characterization (power of Euler)
\item {\bf paper on furthest point on S3 and Voronoi diagrams on S3}
\item Goal: journal papers on building a clean flythrough: 
\begin{enumerate}
\item rational camera control (getting the vertex paths of the camera)
\item collision detection (getting the intersections with the scene)
\item collision correction (adding keyframes to correct the flythrough 
          so that it is collision-free)
\end{enumerate}
\item Goal: 2d kernel journal paper with Wenping (look over his comments and suggest SIAM talk)
\item Goal: 2d visual events/smooth soft shadows
\end{enumerate}

Immediate action item: 
get the basic quaternion spline writeup and software cleaned up and running, 
yielding a rational rotation matrix function.

Paper: interpolation of point data with some parts of the curve predefined (e.g., want
       straight line segments between some points as in reconstruction of architecture)

{\bf Paper: if we have position/orientation keyframes, then we can interpolate this data
       by a valid position and quaternion spline.
       What about if we have a predefined position curve and we want to assign some
       orientation keyframes to it?  For example, the helical curve with orientations
       known at every 90 degrees.  This is a new problem.
       An existing spline has existing knots: changing the knots will change the curve 
       if we are not careful.
       We want to reparameterize the curve to match the desired orientation spline.
       The alternative is to force the quaternion spline to inherit the position spline's
       knots, which is probably not so good.

       Why can't we disentangle the parameterization of the position and orientation?
       We have two parametric curves and certain parameter values where they must
       meet.  We could adjust the speed along the paths to force them to reach the
       meetpoint at the right time.  A problem with this is that there will be an
       abrupt change of speed at the meetpoint if we aren't careful (we should gradually
       change the speed along the path).
       Travel along the position curve at the desired speed (say constant speed).
       Scale speed along the quaternion spline between each pair of keyframes
       to arrive at keyframe at the correct time.  If you don't like this, then
       inherit the knots from the position spline.  Either way, let the position
       spline dominate, since its curve is predetermined or more important.
       The only time that exact coordination between the position and orientation
       must be achieved is for tight motion planning.  Here we really need to
       plan motion in a combined configuration space that includes both position
       and orientation.  This is certainly not the problem we are solving in our 
       interpolation problem.  We simply want a valid motion that does reasonable
       things with the orientation.
}

(Get Gavini promo information from phikappaphi.org for vita.)

People assigned to projects:
Ross: smooth motion
Doug: meshing (hopefully smooth visibility meshing)
Conan: motion or visibility: whichever I am being most productive with
Aqueasha: software release and molecular modeling
Michael: software release and architectural formats/flythroughs
Lei: motion or visibility (or GPU coadvised with Puri?)
who for GPU? Daniel co-advised with Puri?

Offices:
Conan/Lei: 143
Doug: ME/143
Ross: 100/143
Aqueasha/Michael: 143
at least 4 people in 143

\clearpage

\section{Challenges}

\begin{itemize}
\item rational quaternion spline:
\item canonical rotation (just necessary)
\item *emptiest point (separate TR, incorporate in quaternion spline paper and software)
\item full quaternion spline built in pieces, w. derivative control (software release, TR)
\item 
\item flythrough with collision detection:
\item parsing of architectural mesh formats (software release, TR)
\item Vienna city scene, New Orleans mesh (just fun)
\item collection/editing tool for orientation keyframes (software release)
\item *vertex paths
\item octree spatial decomposition for triangular mesh
\item (*)curve cast at triangle
\item intersection of ruled surface with scene?
\item automatic definition of good keyframes, perhaps based on visibility structure
\item given a flythrough, painting the parts of the scene that were seen by the flythrough
\item 
\item construction of orientation obstacles
\item bitangents on S3 for shortest path around obstacles
\item very interesting problem: smooth interpolation of points while avoiding obstacles
      (examined in SIGGRAPH 2004 paper but punted by using optimization)
\item 
\item smooth soft shadows: full construction of 2D visibility structure; software, journal
\item motion planning through a visibility structure!!
\end{itemize}

\section{Software release}

\begin{itemize}
\item Bezier curves
\item Quaternion splines
\item Tangential curves
\item Tangential surfaces
\end{itemize}

\begin{itemize}
\item {\bf CVS of Bezier/quaternion/tangential/visibility by March 15}
\item quaternion spline release by end of March
\item Bezier curve release by end of April
\item tangential curve and 2D visibility structure release by end of May
\end{itemize}

\clearpage

\section{Journal papers (J) and technical reports (TR)}

\begin{enumerate}
\item {\bf Bitangency}
\item 2D bitangents/tangents from a point/2D visibility graph (J)
\item tangential curves of subdivision curves (J)
\item 3D silhouettes and bitangent developables (J)
\item 2D kernels (J, with Wenping)                                      *
\item 3D kernels (J, with Wenping, Myung-Soo, Gershon)
\item 2D visual events/smooth soft shadows/visibility structure (J)     *
\item 3D visual events (see Xiao)
\item first gather existing papers and software: SMI01, Dag04
\item
\item {\bf Rational motion}
\item rational map to sphere characterization (J)                       *
\item emptiest point (TR or J)
\item full quaternion spline review (TR)
\item camera control collision detection with rational vertex paths (J) *
\item definition of orientation obstacles (J)
\item bitangents on S3 for avoiding orientation obstacles (J)
\item interpolating points while avoiding obstacles
\item
\item {\bf Contour reconstruction}
\item is Jill interested? reconstructing the reconstruction method for general topology (J)
\end{enumerate}

\begin{itemize}
\item visit tangential curves/surfaces in class
\end{itemize}

\end{document}
