\documentclass[12pt]{article} 
\usepackage{times}
\usepackage[pdftex]{graphicx}
\makeatletter
\def\@maketitle{\newpage
 \null
 \vskip 2em                   % Vertical space above title.
 \begin{center}
       {\Large\bf \@title \par}  % Title set in \Large size. 
       \vskip .5em               % Vertical space after title.
       {\lineskip .5em           %  each author set in a tabular environment
        \begin{tabular}[t]{c}\@author 
        \end{tabular}\par}                   
  \end{center}
 \par
 \vskip .5em}                 % Vertical space after author
\makeatother

% default values are 
% \parskip=0pt plus1pt
% \parindent=20pt

\newcommand{\SingleSpace}{\edef\baselinestretch{0.9}\Large\normalsize}
\newcommand{\DoubleSpace}{\edef\baselinestretch{1.4}\Large\normalsize}
\newcommand{\Comment}[1]{\relax}  % makes a "comment" (not expanded)
\newcommand{\Heading}[1]{\par\noindent{\bf#1}\nobreak}
\newcommand{\Tail}[1]{\nobreak\par\noindent{\bf#1}}
\newcommand{\QED}{\vrule height 1.4ex width 1.0ex depth -.1ex\ \vspace{.3in}} % square box
\newcommand{\arc}[1]{\mbox{$\stackrel{\frown}{#1}$}}
\newcommand{\lyne}[1]{\mbox{$\stackrel{\leftrightarrow}{#1}$}}
\newcommand{\ray}[1]{\mbox{$\vec{#1}$}}          
\newcommand{\seg}[1]{\mbox{$\overline{#1}$}}
\newcommand{\tab}{\hspace*{.2in}}
\newcommand{\se}{\mbox{$_{\epsilon}$}}  % subscript epsilon
\newcommand{\ie}{\mbox{i.e.}}
\newcommand{\eg}{\mbox{e.\ g.\ }}
\newcommand{\figg}[3]{\begin{figure}[htbp]\vspace{#3}\caption{#2}\label{#1}\end{figure}}
\newcommand{\be}{\begin{equation}}
\newcommand{\ee}{\end{equation}}
\newcommand{\prf}{\noindent{{\bf Proof}:\ \ \ }}
\newcommand{\choice}[2]{\mbox{\footnotesize{$\left( \begin{array}{c} #1 \\ #2 \end{array} \right)$}}}      
\newcommand{\scriptchoice}[2]{\mbox{\scriptsize{$\left( \begin{array}{c} #1 \\ #2 \end{array} \right)$}}}
\newcommand{\tinychoice}[2]{\mbox{\tiny{$\left( \begin{array}{c} #1 \\ #2 \end{array} \right)$}}}
\newcommand{\ddt}{\frac{\partial}{\partial t}}
\newcommand{\Sn}[1]{\mbox{{\bf S}$^{#1}$}}
\newcommand{\calP}[1]{\mbox{{\bf {\cal P}}$^{#1}$}}

\newtheorem{theorem}{Theorem}	
\newtheorem{rmk}[theorem]{Remark}
\newtheorem{example}[theorem]{Example}
\newtheorem{conjecture}[theorem]{Conjecture}
\newtheorem{claim}[theorem]{Claim}
\newtheorem{notation}[theorem]{Notation}
\newtheorem{lemma}[theorem]{Lemma}
\newtheorem{corollary}[theorem]{Corollary}
\newtheorem{defn2}[theorem]{Definition}
\newtheorem{observation}[theorem]{Observation}
\newtheorem{implementation}[theorem]{Implementation note}

% \font\timesr10
% \newfont{\timesroman}{timesr10}
% \timesroman


\setlength{\oddsidemargin}{0pt}
\setlength{\topmargin}{0in}
\setlength{\textheight}{8.6in}
\setlength{\textwidth}{6.875in}
\setlength{\columnsep}{5mm}
\markright{\hfill \today \hfill}
\pagestyle{myheadings}

% -----------------------------------------------------------------------------

% paper 1: rational maps to the sphere for motion design
% paper:   finding furthest point
% paper 2: orientation control for rational motion design: only as a technical report.
% paper 3: collision detection for rational motions: this is the one to publish

% -----------------------------------------------------------------------------

\title{A Reductive Technique for Rational Motion Design\\[5pt]
               % A Reductive Technique for Rational Orientation Control\\[5pt]
       {\small UAB Technical Report\\2005}}
               % (originally February-March 2005) 
\author{John K. Johnstone\thanks{Supported in part by the National Science Foundation
        under grant CCR0203586.}\ \ \ et. al.}
               % \ and James Williams}
\begin{document}
\maketitle

\begin{abstract}
This paper shows how to build a
rational quaternion spline that interpolates a sequence of quaternions.
This spline may be used to control the orientation of a rigid object during a motion.
A significant aspect of the proposed technique is 
that it reduces the interpolation of orientations
to the interpolation of positions, a much simpler problem.
Another significant aspect of the technique is its rationality.\footnote{{\bf Discuss 
    rationality as driving force and modus operandi behind my research at website.}}
Rational orientation control leads to rational motion design, which is beneficial
in the processing of motions, such as collision detection.

The paper's design of a rational interpolatory quaternion spline 
depends on a rational map to the 3-sphere.
A separate paper studies the design of rational maps to $S^3$, while
this paper develops the inverse maps, the avoidance of poles of these inverse maps,
the mapping of Bezier curves, and the mapping of derivatives.
\end{abstract}

{\bf Keywords}: motion design, orientation control, 
                quaternion spline, map to $S^3$.

		% emphasize the reliance on classical techniques.
		% secondarily, emphasize the higher continuity.

\vspace{1in}

Software: visualize the orientation control by showing an object twisting around a circle.
          The 3-sphere cannot be fruitfully visualized, except abstractly.

\clearpage

\tableofcontents


\clearpage

From rm0.tex:

\section{Preamble to 'Rational maps to the sphere' for Orientation control paper}

{\bf Move to orientation control paper or RQ1(?).}

A key component of the proposed solution is a rational map to the sphere
and its companion map off of the sphere.
The latter map, from $S^3$ to some Euclidean $n$-space, frees the quaternions from
their surface constraint, while the former map reimposes this constraint.
Since the map from the sphere is applied to a finite pointset, while the map
to the sphere is applied to a polynomial curve, rationality is only necessary for the
map to the sphere, to preserve rationality of the curve, yielding a rational
quaternion spline.
If the user is willing to settle for nonrational curves,
a nonrational map could be used,
but rational Bezier curves offer many advantages, including simplicity,
efficiency, and pleasant properties such as the variation-diminishing and convex-hull
properties.

\section{Rational maps to the sphere}
% $S^3$ maps, to and fro

{\bf Move to RQ1 paper.}

[Note: The only published paper is GI and Eurographics.
We have moved beyond these papers by at least the 30\% guideline.
This material is valid for a journal publication, but regardless we want a resolution
of this issue in a technical report.
Additional material: stereographic projection, image of rational Bezier,
generalized construction, visualization.]

In this paper, we consider rational maps to the 3-sphere and their inverses.
Since the inverse maps often (always?) have poles, we explore this issue as well.
Derivatives are computed when the maps are applied to a parametric curve on 
the 3-sphere.
[It is hard to motivate the interest in this derivative result, so this should
be delayed to another paper.]
The image of a Bezier curve under these maps, expressed as a rational Bezier curve,
is computed, which allows any Bezier curve to be projected onto the sphere.
(Can we compute the image of a rational Bezier curve too? This would be additional.)
We discuss the generalization to construction of maps to other surfaces.

\section{Originally in RM0 introduction: move to RQ1}

The study of rational maps to the sphere is motivated by a problem in motion design.
% Consider motion, % of a rigid object, or of a rigid component of a nonrigid object,
% which is change in position and orientation.
A common task in motion design is the interpolation of a collection of quaternions
by a quaternion spline, which then represents the orientation component of a motion.
The unit quaternion is a preferred representation for orientation
in computer graphics, physics \cite{misner}, and robotics.
Since unit quaternions are points on the unit sphere $S^3$ in 4-space, 
the orientation component of a motion may be captured by a curve on $S^3$, called
a quaternion spline.
The design of a quaternion spline is made particularly challenging because 
the curve is constrained to remain on the sphere $S^3$.
This awkward constrained interpolation problem may be reduced to 
a classical free interpolation problem through the use of a map from Euclidean space 
to the sphere, as follows.
Let $f$ be a map from Euclidean space (typically 3-space or 4-space) to $S^3$,
and let $Q$ be a set of quaternions.

\begin{itemize}
% \item Map the quaternions $Q$ under $f^{-1}$.
\item Build a curve C that interpolates the points $f^{-1}(Q)$, a classical
      interpolation problem.
\item Map the curve C under $f$, yielding a quaternion spline $f(C)$ 
      that interpolates $Q$, embedded in $S^3$.
\end{itemize}

In this solution, the two challenges of point interpolation and
constraint to the sphere are separated,
unlike other solutions that handle both challenges together \cite{animationtext}.
The first step concentrates solely on point interpolation,
while the constraint to the sphere is handled by the spherical map.
Another advantage of this solution is its potential for rationality.
The quaternion spline's constraint to $S^3$ has typically led to nonrational solutions.
If $f$ is a rational map to the sphere
and the interpolating curve $C$ is rational (as it will be in the 
classical construction \cite{farin}),
the quaternion spline $f(C)$ will also be rational, with all of the 
attendant advantages of rational curves.
% (based on constrained optimization, trigonometric slerping, or Lie exponential maps.)
Rational quaternion splines are particularly attractive for analysis of the 
resulting motion, such as collision detection.
A third advantage is that the above construction of rational quaternion splines 
is more efficient than some other nonrational solutions \cite{barr92,barrafter92}.

[{\em Figure of a quaternion spline and its associated motion.
Figure of algorithm's solution, step by step (perhaps using stereographic projection).}]

\clearpage

%%%%%%%%%%%%%%%%%%%%%%%%%%%%%%%%%%%%%%%%%%%%%%%%%%%%%%%%%%%%%%%%%%%%%%%%%%%%%%%%%%%%%

\section{Introduction}

\subsection{Motion design}

An application that involves moving objects may benefit from the ability
to design the motion algorithmically.
For example, motion design might be used to animate an object in computer animation,
to control a camera in an architectural or city flythrough,
to move an object in robot motion planning, or to guide an object in game design.
This paper considers the design of a motion for a rigid object.
% or of a rigid component of an articulated object,
We should be clear from the outset that we are not talking here about
physical simulations of motion, using the laws of motion, or motion capture,
but rather the algorithmic definition of motion.

The motion of a rigid object is defined by a change in position 
and a change in orientation.
% The design of a motion depends on control of position and orientation.
The change in position may be represented by the parameterized path 
of a reference vertex on the object,
while the change in orientation may be represented by 
a parameterized local coordinate frame of the moving object.
The local coordinate frame may be represented in various ways, such as 
a quaternion or a rotation matrix.
In other words, a motion is described by a curve in position space and a curve
in orientation space.

Rather than directly designing the entire motion, it is traditional
to design only a certain number of crucial or representative keyframes 
and build the rest of the motion through interpolation of these keyframes.
Keyframe interpolation is a natural strategy in motion planning, 
where a motion may first be expressed by a higher level planner 
as a motion through a discrete set of cells, 
and the local motions between cells are left to a lower level planner.
It is also a natural strategy in computer animation, where
keyframe animation has been popular in animation ever since the
earliest days of Disney animations. %, even before there was computer animation.
Since each keyframe is composed of a position and an orientation of the object,
keyframe interpolation reduces to position interpolation and orientation interpolation.
Position interpolation is well understood,
with many elegant solutions such as the interpolating cubic B-spline \cite{farin02}.
The subject of this paper is 
% a method for defining the orientation component of a motion through 
orientation interpolation.

\subsection{Orientation interpolation}

The orientation interpolation problem is often abstracted as follows.
   % Lie algebra setup of Ravani is a bit different
An orientation, or coordinate frame, of the moving object is 
represented by a unit quaternion.
The unit quaternion is interpreted as a point on the unit 3-sphere \Sn{3} in 4-space.
   % (Section~\ref{sec:quaternionTheory} has a full discussion of the quaternion 
   % and its relationship to orientation.)
The orientation component of a motion is then encoded by a quaternion spline,
or a curve on \Sn{3}.
   % So the entire orientation curve must lie on \Sn{3}.
Orientation interpolation becomes the design of a curve that interpolates
certain quaternions while remaining on \Sn{3}.
   % (since every intermediate point of the curve must also represent a unit quaternion).
In other words, orientation interpolation becomes
position interpolation with an additional constraint to a surface.

% START HERE 

Orientation interpolation may be interpreted as a constrained version of position
interpolation.
In this model, an orientation is a quaternion, a quaternion is a point on the 3-sphere
\Sn{3} in 4-space,
and the interpolation of orientations is the interpolation of points
by a curve constrained to lie on \Sn{3}.
The main challenge of orientation interpolation has been this constraint to a surface.
% It is this constraint to \Sn{3} that defines the main challenge.
It has elicited solutions such as relaxation onto the sphere through optimization
techniques, and curve design where linear interpolation is replaced
by spherical linear interpolation (slerping).
% try to mention approaches used in the rational qspline work too

\Sn{n}\ is the unit sphere in $(n+1)$-space 
$x_1^2 + \ldots + x_{n+1}^2 - 1 = 0$, a manifold of dimension $n$.

In this paper, we propose a different approach: 
a reduction of orientation interpolation to position interpolation,
in which the surface constraint is eliminated through the intermediary of a rational
map to \Sn{3}.
% The problem is reduced to a pure unconstrained point interpolation problem.
% {\bf (The interpolation of points constrained to \Sn{3} is replaced by
% a typical unconstrained interpolation of points.)}
The primary benefits of this approach are its rationality, its high continuity, 
its efficiency, and its reliance on classical position interpolation techniques.

Rational motions lead to efficient design and storage, and
prove useful in collision detection.

\subsection{Orientation interpolation through position interpolation}
\label{sec:oritopos}

The main result of this paper is a method for lifting position 
interpolation to orientation interpolation.
Suppose that we want to interpolate a collection of orientations, 
modeled as quaternions $\{Q_i\} \subset \Sn{3}$.
Let $f: \Re^n \rightarrow \Sn{3}$ be a rational map.
Suppose that we solve the following position interpolation problem:
build a curve $C$ that interpolates $\{f^{-1}(Q_i)\}$.
% (the solution of a pure position interpolation problem)
Then this solution can be used to solve the original orientation interpolation problem:
the curve $f(C)$ will interpolate $\{Q_i\}$ and lie on \Sn{3}.
% (the solution of our orientation interpolation problem).

The crucial element of this solution is the rational map to \Sn{3}.
It effectively eliminates the constraint to the surface \Sn{3} that has 
so challenged other methods.
In particular, it isolates the two challenges
of orientation interpolation, point interpolation and constraint to the 3-sphere.
The first step addresses the interpolation, while the second step addresses the constraint
to the 3-sphere.
Since the interpolation does not need to worry about the surface constraint, it can use
classical interpolation in free Euclidean space.
The map to the 3-sphere handles the surface constraint.

Another major advantage of this reductive technique is that it relies
on, and inherits good properties from, the classical solutions to 
position interpolation.
Let us consider these properties one by one.
\begin{itemize}
\item
The interpolating curve $C$ has solutions with {\em high continuity}.
The common solution of an interpolating cubic B-spline \cite{farin02} 
has $C^2$ continuity,
but higher continuity may be achieved at the expense of higher degree.
If care is taken (Section~\ref{sec:deriv}), the orientation curve $f(C)$ inherits
this high continuity.
\item
% there are other rational methods, but rationality is only one of the advantages
The interpolating curve $C$ is {\em rational}\footnote{A rational polynomial 
  is a quotient of polynomials.
  A rational curve is a curve whose parameterization may be described by 
  rational polynomials.}
(indeed, it is polynomial).
Since the map to the 3-sphere is also rational,
the orientation curve $f(C)$ will be rational.
Rational position and orientation control leads to rational motion,\footnote{A rational 
  motion is a motion that may be described by rational polynomials.}
  % i.e., the parameterization of the position 
  % and orientation can be described by rational polynomials.
a major benefit in subsequent analysis of the motion (e.g., collision detection).
\item 
The construction of the interpolating curve $C$ is {\em efficient}.
For example, the construction of an interpolating cubic B-spline reduces
to the solution of a tridiagonal linear system \cite{farin02}.
Since mapping to and from the sphere is also efficient, 
the overall method is efficient.
The method's interactivity is valuable in the application of motion design
to 3D games and motion planning.
It is even useful in computer animation (where time is less critical) in that 
a motion may be interactively recomputed as keyframes are adjusted,
allowing the quick evaluation of alternative motions.
\item The interpolating curve $C$ has {\em end control}.
For example, the interpolating cubic B-spline has enough degrees of freedom
to define the end tangents to user specification.
The orientation curve $f(C)$ can leverage this ability, producing orientation curves
with user-specified end derivatives (Section~\ref{sec:deriv}).
\item The interpolating curve $C$ has locality and symmetry properties:
moving one data point will only affect a local neighbourhood of the curve,
and reversing the data points yields the same curve shape 
(with reversed parameterization).
With care, the orientation curve $f(C)$ will inherit these properties.
DOES OUR SOLUTION NEED TO BE ADJUSTED TO HAVE SYMMETRY?
\end{itemize}

In summary, the reduction of orientation interpolation to position interpolation,
through the use of a rational map to \Sn{3},
brings benefits of smoothness, rationality, efficiency, end control, and locality.
A more intangible advantage is its reliance on the trusted and familiar tool
of point interpolation.
% on the comfort of an old friend,

\subsection{The main challenges}

The proposed reduction of orientation interpolation to position interpolation
requires the solution of four challenges.
The first challenge is the design of a rational map to the 3-sphere.
This problem is attacked in a companion paper \cite{jjrm05},
where a characterization of all rational maps $f$ from Euclidean $m$-space 
to the $n$-sphere is developed,
and two specific maps are developed further in Section~\ref{sec:map}.
The second challenge is a construction of the inverse map $f^{-1}$,
a characterization of its poles, and an understanding of how derivatives
are mapped under $f^{-1}$ (Section~\ref{sec:inverse}).
The third challenge is the requirement to map the data away from the pole of $f^{-1}$
in the interests of a robust method, which is discussed in Section~\ref{sec:pole}.
% (which leads to the division of the large interpolation problem
% into several sub-interpolation problems),
And finally, the fourth challenge is the construction of the image of a curve under $f$,
particularly for Bezier curves (Section~\ref{sec:bezierimage}).
The issue of cusps is explored in Section~\ref{sec:cusp}.
The complete algorithm, including all refinements, is presented in Section~\ref{sec:alg},
and conclusions are offered in Section~\ref{sec:conclude}.
The rest of this section reviews the quaternion in more detail 
and the previous work on orientation interpolation.

The kernel of the ideas presented in this paper were first presented 
at Graphics Interface 1995 \cite{jjjimbo95}.
The present paper makes several additions and elaborations to this result.
It also embeds the method more squarely within the context of motion design.

\subsection{Quaternions and orientation}
\label{sec:quaternionTheory}

This section reviews some key properties of the quaternion.
% as an orientation representation.
The unit quaternion is a preferred representation for the orientation
of a rigid object in computer animation.
\Comment{
Other popular representations, namely the rotation matrix, Euler angles and 
{\bf spinors}(?) (spinor matrices?), are appropriate in other contexts, 
such as the graphics pipeline, 
robot arm control{\bf (?)}, 
	% perhaps see junkins, optimal spacecraft rotational maneuvers
atomic orientation in high-energy physics {\bf(?)} and relativity theory {\bf(?)},
but the quaternion dominates in computer animation.
% This section defines all of these representations, with an emphasis on the quaternion,
% and analyzes their strengths and weaknesses.
}

% this is verbatim from 1999 technical report:

Although the quaternion was invented by William Hamilton in 1843
	% H. Coxeter, Non-Euclidean Geometry, Univ. of Toronto Press, 1961, p. 122
as a 4-dimensional generalization of complex numbers,
Cayley soon recognized that the quaternion can also be used to represent an orientation
\cite{coxeter61}.
	% Cayley quickly realized the quaternion's use for rotation,
	% establishing a connection to the rotation matrix 
	% (see Coxeter, Non-Euclidean Geometry, p. 122).
% The encoding of an orientation in a quaternion depends on a result of Euler.
In 1752, Euler showed that 
any change of orientation may be realized by a single rotation about a fixed axis
\cite{goldstein50}. % (Figure~\ref{fig:eulerRotation}).	
	% goldstein, p. 118
	% also junkins, optimal spacecraft rotational maneuvers, p. 26
% \begin{theorem}[Euler]
% A rigid body can be moved from an arbitrary initial orientation
% to an arbitrary final orientation by a single rotation of the body
% about a fixed axis.
% \end{theorem}
%
% figure: fig:eulerRotation here
%
By expressing the orientation of a rigid body as 
a change of orientation from a canonical orientation, 
  % (perhaps the orientation of the input object)
Euler's result shows that an orientation may be represented by 
a unit rotation axis $v$ and a rotation angle $\theta$.
This axis/angle pair may then be encoded in a unit quaternion by
$(\cos \frac{\theta}{2}, v \sin \frac{\theta}{2})$.
The axis and angle can be extracted easily from the quaternion $(q_1,q_2,q_3,q_4)$
using $\theta = 2 cos^{-1}(q_1)$ and $v = \mbox{unit}(q_2,q_3,q_4)$.

% The rotation of an object by a quaternion need not be 
% computed using its literal interpretation as rotation about an axis.
% Quaternion algebra offers an easier solution, which we now review.
Quaternion algebra may be used to combine two orientations and to rotate a point
by a quaternion.
The quaternion $(a,b,c,d)$ is shorthand for $a + bi + cj + dk$,
where $i$, $j$ and $k$ satisfy the relationship 
	% generalizations of the imaginary unit $i$
\begin{equation}
\label{eq:ijk}
	i^2 = j^2 = k^2 = ijk = -1.   
\end{equation}
If a quaternion $q = (a,b,c,d)$ is expressed as a scalar component $s=a$
and a vector component $v = (b,c,d)$, $q = (s,v)$,
the following formula for quaternion multiplication is easily developed from (\ref{eq:ijk}):
\begin{equation}
\label{eq:qmult}
	[s_1,v_1] * [s_2,v_2] =
	[s_1*s_2 - v_1 \cdot v_2,\ 
	 s_1 * v_2 + s_2 * v_1 + v_1 \times v_2]
\end{equation}
This allows the combination of two orientations.
It also enables rotation of a point:
the point $p \in \Re^3$ is rotated by the unit quaternion $[s,v]$
to the point 
\[
[s,-v] * [0,p] * [s,v].
\]
This is a quaternion, but it may be interpreted as a point in 3-space 
since its scalar part is necessarily 0.
	% quaternions form a noncommutative division ring (ring whose nonzero elements
	% form a group under multiplication: Hernstein, p. 125
Another way to rotate a point is through the observation that 
the unit quaternion $(q_1,q_2,q_3,q_4)$ is equivalent to the rotation matrix
\cite{shoemake85}
\[
\left( 
\begin{array}{ccc}
1-2(q_2^2+q_3^2)   & 2(q_2q_3 + q_1q_4) & 2(q_2q_4 - q_1q_3) \\
2(q_2q_3 - q_1q_4) & 1-2(q_2^2+q_4^2)   & 2(q_3q_4 + q_1q_2) \\
2(q_2q_4 + q_1q_3) & 2(q_3q_4 - q_1q_2) & 1-2(q_2^2 + q_3^2)
\end{array}
\right)
\]

The amount and speed of a body's rotation through 3-space may be directly measured,
and directly controlled, through the length of the quaternion curve on \Sn{3}.
In particular, the metric of \Sn{3}\ is the same as the
metric of the rotation group SO(3) \cite{misner73}.
	% That is, arc length on \Sn{3}\ is proportional (equivalent?) to
	% the (Frobenius norm?) of the associated rotation matrix.
     % \label{thm:metric}
For example, constant speed motion on the quaternion curve yields
constant speed of rotation of the rigid body;
and the length of a quaternion curve measures the amount of spinning of the rigid body.
This is an important tool for motion design in animation or robotics.
The following result captures the essence of the equal metric argument.

\begin{lemma}
\label{lem:constant}
If a rigid object in 3-space is rotated by a fixed amount,
then the representation of this object's orientation as a quaternion is also 
rotated by a fixed amount.
\end{lemma}
\prf
Let the rotation of the object in 3-space be encoded 
by the quaternion $c = (c_1,c_2,c_3,c_4)$
and let the original orientation of the object 
be encoded by the quaternion $q = (q_1,q_2,q_3,q_4)$.
  % If the original orientation of the object and the amount of
  % rotation are represented by the quaternions $q = (q_1,q_2,q_3,q_4)$ 
  % and $c = (c_1,c_2,c_3,c_4)$, respectively,
The orientation of the object after the rotation is $c*q$.
% and the quaternion associated with the orientation of the object 
% has changed from $q$ to $c q$.
But, using the formula for quaternion multiplication (\ref{eq:qmult}),
% notice that $c q$ can be interpreted as the rotation 
% of $q$ on \Sn{3}\ by a fixed amount (depending on $c$):
\[
c * q = \left( \begin{array}{cccc}
	c_1 & -c_2 & -c_3 & -c_4 \\
	c_2 &  c_1 & -c_4 &  c_3 \\
	c_3 &  c_4 &  c_1 & -c_2 \\
	c_4 & -c_3 &  c_2 &  c_1
	\end{array} \right)
	\left( \begin{array}{c}
	q_1 \\ q_2 \\ q_3 \\ q_4
	\end{array} \right)
\]
This matrix is a rotation matrix in 4-space (unit and orthogonal rows, unit and orthogonal
columns, unit determinant).
Therefore, just as the actual object in 3-space has rotated by a fixed amount,
the quaternion in 4-space representing the object's orientation
has also rotated by a fixed amount.
	% unit rows and columns, orthogonal rows, orthogonal columns
	% unit determinant (yes, checked)
\QED

Care must be taken in encoding an orientation by a quaternion,
since the quaternion $(\cos \frac{\theta}{2}, v \sin \frac{\theta}{2})$
and its antipode $-(\cos \frac{\theta}{2}, v \sin \frac{\theta}{2}) 
= (\cos \frac{2\pi - \theta}{2}, -v \sin \frac{2\pi - \theta}{2})$
represent the same orientation
(the rotation axis has been flipped and the angle reversed).
That is, \Sn{3}\ is a double covering of orientation space.
(A natural explanation for this double covering is provided by the belt or plate trick
\cite{hart94}, where a rotation of 720 degrees is required to return to an
original configuration.)
  % Hart, p. 263-264
This ambiguity in representation is not a problem in motion design, however,
% However, when using quaternions for motion design, 
since only one of the two quaternion representations will be appropriate 
in any given context:
the quaternion with the smaller angular gap on \Sn{3} to the previous quaternion.
This choice is based on the understanding that 
a motion should not introduce spinning unless it is explicitly designed in,
and the fact that a smaller angular gap between consecutive quaternions
creates less spinning than a larger angular gap (see above).
Consider the difference between a small rotation and the complementary
rotation that represents an almost complete pirouette.
If the designer's intent is to include the pirouette, then this should be
made explicit in the motion control by adding intermediate quaternions
that imply this pirouette.

\Comment{
The quaternion has certain advantages over other representations
for orientation, such as the rotation matrix and Euler angles.
Unlike Euler angles, quaternions do not experience gimbal lock,
can be combined easily, and have an effectively unique representation
for each orientation (see discussion above).
Unlike rotation matrices, quaternions have a concise representation,
4 scalars rather than 9.\footnote{A quaternion is still slightly 
	larger than it needs to be since the rotation axis $v$, being a unit vector, 
	is fully determined by only 2 of its elements.
	That is, $v$ and $\theta$ could be encoded using only 3 scalars.
	However, the redundancy of 4 scalars is necessary for
	the quaternion's geometric interpretation on a unit sphere 
	and the metric equivalence of \Sn{3} and SO(3).}
and, unlike both Euler angles and rotation matrices, 
quaternions have a natural geometric interpretation through identification
with \Sn{3}, which is a crucial element in algorithmic development.
}

Once the power of the quaternion as an orientation representation
were recognized by the graphics community, it quickly became used 
for orientation control.
The previous work in this area is reviewed in the next section.

% We end this section with definitions relevant to this paper.

% DEFINE BEZIER CURVE, BERNSTEIN POLYNOMIALS.
% REVIEW RATIONAL BEZIER CURVE AND BEZIER CURVE IN PROJECTIVE SPACE,

% \begin{defn2}
% A map $(x_1,\ldots,x_n) \mapsto (f_1 (x_1,\ldots,x_n),\ldots,f_m (x_1,\ldots,x_n))$
% is {\bf rational} if 
% the $f_i$ are all rational polynomials in $x_1,\ldots,x_n$, $i=1,\ldots,m$.
% \end{defn2}

% ------------------------------------------------------------------------------------

\subsection{Related work}
\label{sec:prevwork}

START HERE

Be impartial and objective: state the result and sometimes put it in context.

The construction of quaternion splines has a relatively long history,
starting with the slerping work of Shoemake.

% THIS IS VERBATIM FROM SPLINE.TEX, THE 1999 TECHNICAL REPORT

There is a rich literature on quaternion splines.
\cite{shoemake85} introduced them as a solution 
for keyframe animation in 1985.
He used spherical linear interpolation (slerping) between points on \Sn{3}.
Many others have also used slerping 
\cite{duff85,pletinckx89,schlag91,nielson92,nielson93,kim95,nam95}.
The spherical linear interpolation between points $P_0$ and $P_1$ on $S^3$ is:
\[ P(t) = \frac{\sin((1-t)\theta) P_0 + \sin(t \theta) P_1}{\sin \theta}
\]
where $\cos \theta = P_0 \cdot P_1$.
This represents an arc of the great circle between $P_0$ and $P_1$.
The de Casteljau algorithm can be used to build traditional Bezier curves.
A spherical analog to the de Casteljau algorithm based upon
spherical linear interpolation is used to build quaternion splines,
mimicking Bezier \cite{shoemake85,kim95}, B-spline 
\cite{duff85,nielson92,nielson93,kim95}, Hermite \cite{kim95,nam95},
cardinal spline \cite{pletinckx89}, and Catmull-Rom \cite{schlag91} curves.
ADD GABRIEL REFERENCE.
Since slerping is a nonrational operation,
all of the methods based on slerping generate nonrational curves.
Most of these curves are defined only by a geometric construction, and have
no closed form algebraic definition.
Computation of derivatives of these curves is complicated,
as is the imposition of $C^2$ continuity.
Kim et. al. \cite{kim95} provide solutions to the latter two problems,
using Lie algebra and its exponential map.
Our use of conventional Bezier or B-spline curves completely removes 
these problems with continuity or derivative calculation.
\cite{park97} again uses Lie algebra to design the quaternion spline,
working on the SO(3) manifold rather than the \Sn{3}\ manifold.

\cite{barr92} uses constrained optimization to develop optimal
quaternion splines, optimizing the constraint that the curve lie on \Sn{3}.
They also introduce low covariant acceleration
as a desirable property of a quaternion spline,
and incorporate it into their constraints.
This ability to incorporate extra constraints into the optimization
is a nice feature of their algorithm.
Their quaternion spline is nonrational with no closed-form expression,
and their numerical optimization can be expensive.
Their approach is refined for added efficiency in \cite{rama97}.

\cite{wang93, wang94} and \cite{nielson93} are close in spirit 
to this paper, since they design rational quaternion splines.
However, these curves are limited in scope.
\cite{wang93,wang94} and \cite{nielson93} develop quadratic curves
with $G^1$ continuity, and \cite{wang94} develops sextic curves
with $C^1$ continuity.
At least $C^2$ continuity is desirable, especially for animation.
	% In \cite{wang93,wang94}, the input points must be augmented with 
	% tangents, since the problem is posed as a Hermite interpolation problem.
The quaternion splines of \cite{wang93} are built from biarcs 
(great circles of the sphere), and those of \cite{nielson93} from circular arcs.
Both Wang and Nielson's methods also involve heuristic, data-dependent choices
that can be difficult to make,
such as the choice of a spherical biarc from
a one-parameter family of valid spherical biarcs, or a center of projection.
Our method generalizes the work of Wang and Nielson, 
by creating rational curves of arbitrary even degree 
(all rational curves on $S^3$ have even degree \cite{wang94}) 
and arbitrary continuity,
based on traditional NURBS and without any data-dependent choices.

The Euclidean-space approach has been used before
for the design of curves on surfaces \cite{dietz93,wang94}.
	% \cite{dietz93} and \cite{wang94} design rational curves on quadrics
	% in 3-space.
	% They both use a variant of the Euclidean-space approach.
The classical solution to the design of trim curves on a surface 
is another example, where the trim curve is designed in the parameter space 
of the surface and then mapped back to the surface using the
parameterization. % \cite{foley96}
Trim curves are not a good solution to quaternion splines, however.
It turns out that a parameterization of \Sn{3}\ is not a good map
to the surface for quaternion spline design (see Section~\ref{sec:results}).
Its domain is also wrong ($\Re^3$ rather than $\Re^4$) as discussed
in Sections~\ref{sec:intro} and~\ref{sec:eucdesign}.

from hermiteQ.tex

Quaternion splines:
\begin{itemize}
\item Shoemake \cite{shoemake85} (SIGGRAPH, arcball)
\item Barr \cite{barr92}, Ramamoorthi and Barr \cite{rama97}: optimization
\item Kim (exponential maps) \cite{kim95}: exponential maps.
\item Kim and Nam, CAD 27, 1995: Interpolating solid orientations with circular
  blending quaternion curves.
\item Ravani (Lie algebra);
\item Grassia \cite{grassia98}.
\item Pottmann and Wagner (dual quaternion splines);
\item Roschel, CAD 30, 169--178, Rational motion design: a survey.
\item animation literature;  
\item Maya's animation.
\item see other citings in Shoemake 1985 SIGGRAPH website.
\item Juttler and Wagner in 2002 Handbook of CAGD on kinematics and animation.
\item INCLUDE REFERENCES IN REVIEW OF TOG PAPER.
\end{itemize}

Combining orientation with position (and perhaps velocity)
\begin{itemize}
\item Brotman \cite{brotman88},
\item dual quaternion work of Pottmann, Wagner.
\end{itemize}

Address the relative independence of position and orientation (except for knot
sequences) in our approach, in contrast to others such as Brotman.
We aren't building motions with complete physical realism (or is this not 
a concession we have to make here because it is not an issue?).

Spherical curves:
\begin{itemize}
\item Wang and Joe \cite{wang93}: spherical biarcs.
\item Dietz and Hoschek \cite{dietz93} (curves on S2, do they map from S2 to R3 or S2 to R2?);
\item Juttler and Wang \cite{jw03}.
\item See Gfrerrer in CAGD99 16(1), p. 21-37: Rational interpolation on a hypersphere.
\item See references in Juttler and Wang 2003 CAGD article: 621-636, such as 
  Choi on spin representation.
\end{itemize}

\clearpage

\section{Rational maps to the 3-sphere}
\label{sec:map}

The first challenge in reducing orientation interpolation to position interpolation
%  the first step in the design of a rational quaternion spline
is the design of a good rational map from Euclidean space to the 3-sphere \Sn{3}.
In a companion paper \cite{jjrm05}, we completely characterize rational maps from
any Euclidean space to any sphere.
This section explores two of these maps to \Sn{3}.

\subsection{The Euler map to the sphere}

The complete characterization of rational maps from 
Euclidean space to \Sn{n-1} is as follows.
$S^{n-1}$ is the unit sphere centered at the origin in $n$-space,
$x_1^2 + \ldots + x_n^2 = 1$.

\begin{theorem}
\label{thm:ratmap}
The map $f: \Re^m \rightarrow \Sn{n-1}$ is rational if and only if
it is some coordinate permutation of:
\begin{equation}
\label{eq:re4s3}
% \footnotesize{(x_{\pi(1)},x_{\pi(2)},x_{\pi(3)},x_{\pi(4)})} \mapsto 
f(x_1,\ldots,x_m) = 
% \footnotesize{
(\frac{a_1^2 + \cdots + a_{n-1}^2 - a_n^2}{a_1^2 + \cdots + a_n^2},
	 \frac{2a_1a_n}{a_1^2 + \cdots + a_n^2},
	 \ldots, 
	 \frac{2a_{n-1}a_n}{a_1^2 + \cdots + a_n^2})
% }
\end{equation}
where $a_1,\ldots,a_n \in \Re[x_1,\ldots,x_m]$.
% and $\pi : \{1,2,3,4\} \rightarrow \{1,2,3,4\}$ is a permutation.
\end{theorem}
% note: some of the ai may be constant (see below in stereographic injection case)
% so the image of the origin may not always be undefined.

A natural choice in Theorem~\ref{thm:ratmap} is $a_i = x_i$ and the identity permutation.
This yields the following map to the 3-sphere.\footnote{We name this map after Euler,
  because a number theoretical result of Euler is the key element in the
  characterization of rational maps to the sphere.}
%
WHY IS THE EULER MAP ABBREVIATED M? WOULDN'T E BE MORE NATURAL?
\begin{defn2}
The {\bf Euler map to the sphere} $M: \Re^4 - \{0\} \rightarrow \Sn{3}$ is defined by:
\begin{equation}
\label{eqM}
	M(x_1,x_2,x_3,x_4) =
	\frac{1}{x_1^2 + x_2^2 + x_3^2 + x_4^2}\ 
	(x_1^2 + x_2^2 + x_3^2 - x_4^2,\ 2x_1x_4,\ 2x_2x_4,\ 2x_3x_4)
\end{equation}
%
If we let 
${\cal M}(p,q): = (p_1q_1 + p_2q_2 + p_3q_3 - p_4q_4,\ 2p_1q_4,\ 2p_2q_4,\ 2p_3q_4)$,
the Euler map may be re-expressed as:
\begin{equation}
\label{eq:calM}
M(P) = \frac{{\cal M}(P,P)}{P \cdot P}
\end{equation}
\end{defn2}

The Euler map has both bilinear and quadratic characteristics:
every coordinate of ${\cal M}$ is a bilinear map,
but in projective space, every coordinate of $M$ is a quadratic form.
% The form (\ref{eq:calM}) is valuable in Theorem~\ref{thm:imagecurve} below.

\subsection{Stereographic injection}
\label{sec:sp}

Another simple rational map to the sphere may be built from stereographic projection,
the most well-known map {\em from} the sphere,
used at least since Hipparch in 160 B.C. \cite{kreyszig59}. % p. 205
  % probably discovered by Hipparch too
Stereographic projection involves perspective projection from a point on the sphere 
to a hyperplane through the origin (opposite the point).
Since it is one-to-one and onto, stereographic projection has a well-defined inverse,
which we call {\bf stereographic injection}.
Stereographic injection is a rational map from the hyperplane % $x_{n+1}=0$
(interpreted as an Euclidean $n$-space)
% (an embedding of n-space in $\Re^{n+1}$) 
to \Sn{n}.
%
% a map from \Sn{n}\ to a hyperplane through the origin in Euclidean $n+1$-space,
% using perspective projection onto the hyperplane from a point on the sphere
% (furthest from the hyperplane).
%
For compatibility with the Euler map, 
we use a variant of stereographic projection with center of projection $(1,0,\ldots,0)$
and projection plane $x_1=0$.\footnote{For example, they now have the same pole $(1,0,\ldots,0)$.}

% figure fig:stereo

\begin{lemma}
\label{defn:si}
Stereographic projection 
$f_{SP}: \Sn{n} - (1,0,\ldots,0) \rightarrow x_1=0 \subset \Re^{n+1}$ 
is defined by\\
$f_{SP}(x_1,\ldots,x_{n+1}) = \frac{1}{1-x_1} (0,x_2,\ldots,x_{n+1})$.
% see thorpe79, p. 125
Stereographic injection $f_{SP}^{-1}: x_1=0 \rightarrow \Sn{n} - (1,0,\ldots,0)$ 
is defined by
$f_{SP}^{-1}(0,x_1,\ldots,x_{n}) = \frac{1}{x_1^2 + \cdots + x_n^2 + 1} 
	(x_1^2 + \cdots + x_n^2 - 1, 2x_1, \ldots, 2x_n)$.
% see thorpe79, p. 125
\end{lemma}
\vspace{.1in}
\prf
The projector line $(1-t)p + tq$ through $p = (1,0,\ldots,0)$ and 
$q = (x_1,\ldots,x_{n+1}) \in \Sn{n}$ intersects $x_1=0$ 
when $1-t + tx_1 = 0$ or $t = \frac{1}{1-x_1}$.
This defines stereographic projection.
For the inverse map, the projector line $(1-t)p + t(0,r)$ through $p = (1,0,\ldots,0)$ 
and $(0,r)$ intersects \Sn{n} when $\|(1-t)(1,0,\ldots,0) + t(0,r)\|^2 = 1$,
% or $(1-t)^2 + t^2\|r\|_2^2 = 1$,
or $t^2(\|r\|^2 + 1) = 2t$, or 
$t=0,\frac{2}{\|r\|^2 + 1}$.
The second root yields the inverse point.
\QED

\noindent The specializations of these maps to \Sn{3} are a map from the sphere:
\begin{equation}
\label{eq:sp}
f_{SP}(x_1,x_2,x_3,x_4) = \frac{1}{1-x_1}(0,x_2,x_3,x_4)
\end{equation}
and a map to the sphere:
\begin{equation}
\label{eq:si}
f^{-1}_{SP}(0,x_1,x_2,x_3) = 
 \frac{1}{x_1^2 + x_2^2 + x_3^2 - 1}(x_1^2+x_2^2+x_3^2-1,\ 2x_1,\ 2x_2,\ 2x_3).
\end{equation}

% Notice that the image of the pole $(1,0,\ldots,0)$ under stereographic projection
% is well defined in projective space, where it is the line at infinity of the hyperplane.

Stereographic injection defines a parameterization of the sphere,
since it maps $n$-space (which may be interpreted as a parameter space) to an 
$n$-manifold.
Thus, this rational map to the sphere may also be interpreted
as the map derived from a rational parameterization of the sphere.
Finally, it may be interpreted as the rational map to \Sn{3} that uses
$(a_1,a_2,a_3,a_4) = (x_1,x_2,x_3,1)$ and $m=3$ in Theorem~\ref{thm:ratmap}.

As we develop the reduction of orientation interpolation to position interpolation,
we shall illustrate the mechanism using these two rational maps to the sphere,
the Euler map and stereographic injection.
Other rational maps from Theorem~\ref{thm:ratmap} may be used if desired.

\clearpage

\section{Inverse maps away from the 3-sphere}
\label{sec:inverse}

The second challenge is a construction of the inverse map,
which maps the quaternions off the sphere in preparation for position interpolation.
We consider the inverse of the Euler map, the inverse of stereographic injection,
and then discuss the construction of inverse maps in general.
In later sections, we consider how derivatives are mapped by these inverses.

Because of our work interpreting the rational map to the sphere (\ref{eq:si}) 
as stereographic injection,
its inverse map is already known: it is stereographic projection (\ref{eq:sp}).
% The application of stereographic projection to the quaternions maps them off the sphere.
% A curve is then interpolated through the inverse images, unconstrained to the sphere.
% This is a simple result because the inverse is known and the inverse image of a point
% is a point.
The inverse of the Euler map is more interesting.

\subsection{Inverse of the Euler map}
\label{sec:inveuler}

Since the Euler map transforms 4-space into a 3-manifold,
a counting argument would indicate that the inverse of a typical point 
should be a 1-manifold.
The next lemma shows that it is a line.

\begin{lemma}
\label{lem:inverse}
The inverse of the Euler map, $M^{-1}: \Sn{3} \rightarrow \Re^4$, is defined by:
\begin{equation}
\label{eq:eulerinv}
M^{-1}(x_1,x_2,x_3,x_4) = (1-t)(0,0,0,0) + t(x_2,x_3,x_4,1-x_1),\ t \in \Re,\ t \neq 0
\end{equation}
That is, the image of a point is a line through the origin
and $(x_2,x_3,x_4,1-x_1)$.
The map has a special behaviour at the pole $(1,0,0,0)$:
\begin{equation}
\label{eq:eulerpole}
M^{-1}(1,0,0,0) = \mbox{ the hyperplane $x_4=0$ without the origin}
\end{equation}
\end{lemma}
\prf 
See the appendix.
\QED

% Example of another map: using $a_1=x_1 - 1$, $a_2 = x_2^2$, $a_3=0$, 
% and $a_4=x_4$,
% let $M(q_1,q_2,q_3,q_4,q_5) = (q_1^2 - 2q_1 + 1 + q_2^4 - q_4^2,
% 				2q_1q_4 - 2q_4, 2q_2^2 q_4, 0, 
% 				q_1^2 - 2q_1 + 1 + q_2^4 + q_4^2)$.

The position interpolation problem is to build a curve $C$ that
interpolates $\{M^{-1}(Q_i)\}$ (Section~\ref{sec:oritopos}).
Since interpolation of lines is largely unexplored,
it would be more natural to reduce the line interpolation of $\{M^{-1}(Q_i)\}$ 
to point interpolation.
{\bf Explore interpolation of lines, which also arises in tangential curves.}
This can be done by choosing a point on each line.
A natural solution is to choose the closest point on each line to the previous point.
% if $p_i$ is the point chosen on $M^{-1}(q_i)$, $p_i$ is the closest point to $p_{i-1}$.
The first point can be chosen arbitrarily. %, perhaps on \Sn{3}.
This point selection promotes shorter curves in Euclidean space, 
which promotes shorter curves on the 3-sphere,\footnote{We realize that curve length 
        in Euclidean space is not equivalent to curve length on \Sn{3}.
        %  of the associated quaternion spline.
        For example, in the extreme case, motion along
	an inverse line $M^{-1}(Q_i)$ in $\Re^4$ causes no associated
	motion on \Sn{3}.  However, in general, the arc length
	of the two curves, one in Euclidean space and the other on \Sn{3},
	is strongly related.
	Moreover, the avoidance of motion in Euclidean space always leads 
	to the avoidance of motion on \Sn{3}.}
which discourages unnecessary spinning
of the moving object (Section~\ref{sec:quaternionTheory}). 
However, % there is one problem with this choice: 
the points
will spiral in towards the origin, because all the inverse lines pass through the origin
and grow closer together there.
This could be a problem since the Euler map is undefined at the origin;
moreover, the points get very close together.

An excellent alternative is simply to choose the point of $M^{-1}(Q_i)$ on \Sn{3},
rather than the closest point.
This is a good approximation to the closest point, especially when the quaternions
are densely sampled,
and does not suffer from implosion into the origin.
By choosing the point of \Sn{3} on the inverse line, the following inverse map results.
%
\begin{defn2}
The {\bf restricted inverse Euler map} $M^{-1}_{S3}: S^3 - \{(1,0,0,0)\} \rightarrow S^3$,
defined by $M^{-1}_{S3}(P) = M^{-1}(P) \nolinebreak \cap \nolinebreak S^3$, has the form
\begin{equation}
\label{eq:riem}
(x_1,x_2,x_3,x_4) \mapsto \frac{1}{\sqrt{2-2x_1}}(x_2,x_3,x_4,1-x_1)
\end{equation}
\end{defn2}
%
To see this, note that $\frac{(x_2,x_3,x_4,1-x_1)}{\|x_2,x_3,x_4,1-x_1\|}$ is
an intersection of the inverse line with $S^3$, and
\[
\|(x_2,x_3,x_4,1-x_1)\|^2 = x_2^2 + x_3^2 + x_4^2 + (1-x_1)^2
= x_1^2 + x_2^2 + x_3^2 + x_4^2 + 1 - 2x_1 = 2 - 2x_1.
\]
% Since $(x_1,x_2,x_3,x_4) \in S^3$,
% \[
% \|(x_2,x_3,x_4,1-x_1)\|_2^2 = 2 - 2x_1
% \]

The restricted inverse Euler map is simple, avoids the problematic origin, 
and promotes short quaternion splines on \Sn{3}.
The appendix  develops a formula for the distance of the point on \Sn{3}\ from
the closest point as a function of the angular distance between the associated
quaternions, which formally establishes the quality of this approximation to the closest
point (Section~\ref{sec:closest}).

The restricted inverse Euler map has the amusing characteristic that it maps the
quaternions back to \Sn{3}.
However, progress has been made:
the interpolation may now proceed without the constraint to the 3-sphere.
% Note that this is a well-defined choice, since $M^{-1}(P)$ is a line through the origin.

% This is not a rational map, but it need not be.

{\bf Image of quaternions on sphere, their inverse images as lines, their inverse images
as points.}

\Comment{
\begin{defn2}
\label{defn:pole}
The special point $(1,0,0,0)$ is called the {\bf pole} of the map $M^{-1}$.
$(x_2,x_3,x_4,1-x_1)$ is called the {\bf defining point} of the preimage 
$M^{-1}(x_1,x_2,x_3,x_4)$.
\end{defn2}

\noindent Notice the beautiful simplicity of the defining point.
}

\subsection{General inverse map}

Now consider the inverse of the general rational map (\ref{eq:re4s3}) to \Sn{3}.
%  One might hope for a general inverse map, the inverse of the general 
%  rational map to the sphere (\ref{eq:re4s3}).
The proof technique of Lemma~\ref{lem:inverse}
requires working in projective space, 
% works for any individual map, but it
so it cannot be applied to the abstract general map (\ref{eq:re4s3})
to reveal the general inverse map.
To work in projective space
% to take advantage of the extra equation it offers;
would require getting inside the arbitrary polynomials
$a_i$ of (\ref{eq:re4s3}) to translate them into projective space through homogenization,
an infeasible task since these polynomials are of unknown and varying degree.
However, the proof technique does work for any {\em particular} choice of rational
map to the sphere, so there is full flexibility in applying this technique
to any rational map to the sphere to find its inverse, following the template
of Section~\ref{sec:inveuler}.

\subsection{Inverse images of derivatives}
\label{sec:deriv}

We now show how to map a set of derivatives under the inverse map.
The mapped derivatives are used as further input to the position interpolation problem.
This allows added control of the quaternion curve, and it also allows 
two separately built quaternion curves to be joined smoothly.
The latter capacity will become useful in divide-and-conquer techniques proposed
in Section~\ref{sec:pole}.

\begin{lemma}
\label{lem:firstderiv}
Suppose that we are using the Euler map to the sphere.
To design a quaternion curve
$C(t)$, $t \in [t_1,t_2]$, with opening point $C(t_1) = (c_1,c_2,c_3,c_4)$ 
and opening first derivative $C'(t_1) = (c'_1,c'_2,c'_3,c'_4)$,
we should design a curve $M^{-1}|_{S3}(C(t))$ in free space with opening first derivative
% The restricted inverse Euler map of $C(t)$ is another curve on \Sn{3}.
% The first derivative of this curve $M^{-1}|_{S3}(C(t))$ is:
\[
(\frac{c'_2}{\sqrt{2-2c_1}} + \frac{c'_1c_2}{(2-2c_1)^{3/2}},\ 
 \frac{c'_3}{\sqrt{2-2c_1}} + \frac{c'_1c_3}{(2-2c_1)^{3/2}},\ 
 \frac{c'_4}{\sqrt{2-2c_1}} + \frac{c'_1c_4}{(2-2c_1)^{3/2}},\ 
 \frac{-c'_1}{2\sqrt{2-2c_1}}).
\]
\end{lemma}
\prf
See Section~\ref{sec:firstderiv} of the appendix.
\QED

The transfer of higher derivatives from the quaternion curve to the 
image curve in Euclidean space is another simple derivative computation.
For example, the second derivative of the curve in free space is defined by the
following lemma.
% are found simply by repeating this process. 

\begin{lemma}
\label{lem:secderiv}
To design a quaternion curve $C(t)$ with opening second derivative 
$C''(t_1) = (c''_1,c''_2,c''_3,c''_4)$, we should design a curve 
$M^{-1}|_{S3}(C(t))$ in free space with opening second derivative
\[
\left(
\begin{array}{c}
\frac{c''_2}{\sqrt{2-2c_1}} + \frac{2c'_1 c'_2 + c''_1 c_2}{(2-2c_1)^{3/2}}
+ \frac{3(c'_1)^2 c_2}{(2-2c_1)^{5/2}}\\

\frac{c''_3}{\sqrt{2-2c_1}} + \frac{2c'_1 c'_3 + c''_1 c_3}{(2-2c_1)^{3/2}}
+ \frac{3(c'_1)^2 c_3}{(2-2c_1)^{5/2}}\\

\frac{c''_4}{\sqrt{2-2c_1}} + \frac{2c'_1 c'_4 + c''_1 c_4}{(2-2c_1)^{3/2}}
+ \frac{3(c'_1)^2 c_4}{(2-2c_1)^{5/2}}\\

\frac{-c''_1}{2\sqrt{2-2c_1}} - \frac{(c'_1)^2}{4(2-2c_1)^{3/2}}
\end{array}
\right)
\]
\end{lemma}
\prf
See Section~\ref{sec:firstderiv} of the appendix.
\QED

The only limit to the transfer of derivatives from quaternion spline to free curve is
the degree of the quaternion spline and free curve.
Typically, the free curve is designed as a cubic curve,
and it only makes sense to transfer first and second derivatives.
However, if the control of higher derivatives is desired, 
a free curve of higher degree may be used.
For example, the design of a quintic interpolating curve allows the end 
second derivatives to be set.
Derivatives at the other end of the curve are controlled in exactly the same way.

Of course, this process may be applied to any other rational map to the sphere and its
inverse.
For illustrative purposes, the next lemma considers stereographic projection/injection. 

\begin{lemma}
Suppose that we are using stereographic injection to the sphere.
To design a quaternion curve $C(t)$, $t \in [t_1,t_2]$
with opening point $C(t_1) = (c_1,c_2,c_3,c_4)$, opening first derivative 
$C'(t_1) = (c'_1,c'_2,c'_3,c'_4)$, and opening second derivative 
$C''(t_1) = (c''_1,c''_2,c''_3,c''_4)$,
we should design a curve $f_{SP}^{-1}(C(t))$ in free space with opening first derivative
\[
\frac{1}{(1-c_1)^2}
\left(
\begin{array}{c}
0\\
c'_2(1-c_1) + c_2c'_1\\
c'_3(1-c_1) + c_3c'_1\\
c'_4(1-c_1) + c_4c'_1
\end{array}
\right)
\]
and opening second derivative 
\[
\frac{1}{(1-c_1)^3}
\left(
\begin{array}{c}
0\\
(c''_2 - c_1 c''_2 + c''_1 c_2)(1-c_1) + 2c'_1(c'_2 - c_1 c'_2 + c'_1 c_2)\\
(c''_3 - c_1 c''_3 + c''_1 c_3)(1-c_1) + 2c'_1(c'_3 - c_1 c'_3 + c'_1 c_3)\\
(c''_4 - c_1 c''_4 + c''_1 c_4)(1-c_1) + 2c'_1(c'_4 - c_1 c'_4 + c'_1 c_4)
\end{array}
\right)
\]
\end{lemma}
\prf
Simply take the derivative and second derivative of (\ref{eq:sp}) and evaluate at $t=t_1$.
\QED

\clearpage

\section{Rotating away from the pole} % avoiding the pole
\label{sec:pole}

The inverse of a rational map to the sphere inherently has a single pole \cite{jjrm05},
in the neighbourhood of which data is mapped poorly.
This is not a weakness of the method: it is an inherent by-product of using
rational maps to the sphere, which is at the heart of the method's strength.
For example, the three inverse maps that we have considered, 
the inverse Euler map (\ref{eq:eulerpole}),
the restricted inverse Euler map (\ref{eq:riem}), 
and stereographic projection (\ref{eq:sp}), 
have poles at (1,0,0,0).
They are either undefined or their behaviour is unusual at the pole.

To avoid degeneracies associated with the pole, all quaternions 
will be rotated away from the pole before design of the quaternion spline.
This section considers this issue.

As an example of ill behaviour in the neighbourhood of a pole,
consider the inverse Euler map (\ref{eq:eulerinv}).
% Lemma~\ref{lem:inverse} established that the 
% image of a point $p = (x_1,x_2,x_3,x_4) \neq (1,0,0,0)$ is a line through the
% origin and the defining point $(x_2,x_3,x_4,1-x_1)$.
As $p = (x_1,x_2,x_3,x_4) \in \Sn{3}$ approaches the pole $(1,0,0,0)$,
the defining point $(x_2,x_3,x_4,1-x_1)$ of the inverse line $M^{-1}(p)$ 
approaches the origin
% \[ 
% \mbox{lim}_{(x_1,x_2,x_3,x_4) \rightarrow (1,0,0,0)} (x_2,x_3,x_4,1-x_1)
% = (0,0,0,0) 
% \]
and the line becomes ill-conditioned:
% the defining point becomes an ill-conditioned specification of $M^{-1}(p)$:
small motions of $p$ can cause large motions of $M^{-1}(p)$.
% \footnote{This 
%  is analogous to the solution of a linear
%  system when the condition number of the matrix becomes large,
%  and the linear system becomes very sensitive to perturbation of the matrix.}
%
\Comment{
The following example illustrates.

\begin{example}
If $p$ on \Sn{3}\ is at distance $d$ from the pole,
the defining point of $M^{-1}(p)$ is also at distance $d$ from the
origin.
Then a motion of $d+\epsilon$ of $p$ (corresponding to a motion
of the defining point directly towards the origin) can yield
an arbitrary change of orientation of the line $M^{-1}(p)$.
\end{example}
}
%
Since there is little correspondence, near the pole,
between the position of a quaternion and its inverse,
orientation curves built out of quaternions near the pole will 
jump wildly about the pole (Figure~\ref{fig:wild}).  % or create cusps 
% Therefore, the quaternions must be moved away from the pole.
Experimental evidence indicates that a quaternion $p \in \Sn{3}$ is far enough from 
the pole $P$ if the angle formed by the vectors $p$ and $P$ is larger than 20 degrees
({\bf or is it 30 degrees?}).
% used to be 30 degrees, but calculations below about existence of an empty region
% use 20 degrees so we must have determined that this is far enough 
% pi/6 radians
% Experimental evidence indicates that points closer to the pole 
% can lead to undesirable behaviour in the curve.
% 30 = PI/6; PI/7 is definitely too small (lowerbound), as witnessed by data5-1
% using TOOCLOSEDIST = PI/7;
% we have not found an example that is not treated well by PI/6.

% \begin{example}
% data5-1 using $\pi/7$ vs. $\pi/6$.
% \end{example}

% figure fig:wild

% An inherent trait of the inverse map of a rational map to the sphere is
% the presence of a pole.
%  We can also interpret the behaviour of stereographic projection at this pole
%  to be defined but unusual: in projective space, the pole maps to the line
%  at infinity rather than a point.
% A pole is an inevitable artifact of the inverse of a rational map to the sphere, 
% as discussed in \cite{jjrm05}.

The solution to the inverse map's instability around the pole
is simple: the quaternions $\{Q_i\}$ are mapped far away from the pole
before they are mapped to $\{f^{-1}(Q_i)\}$.
%
% Since the inverse map is not well behaved in a neighbourhood of about 20 degrees
% of its pole,
% the quaternions $\{Q_i\}$ need to be rotated away from the pole
% before they are mapped to $\{f^{-1}(Q_i)\}$.
%
A rotation of the quaternions away from the pole 
is equivalent to the discovery of an empty region on the sphere 
where we can move the pole.
In particular, we wish to compute an empty point on the sphere \Sn{3}, defined as follows.
%
\begin{defn2}
A point $p \in \Sn{3}$ is {\bf empty} with respect to a set of points $Q \subset \Sn{3}$
if its angular distance from each point $Q_i \in Q$ is at least 20 degrees.
\end{defn2}
%
By rotating the pole to the empty point
(actually, rotating the quaternions in the opposite direction),
the quaternions are moved sufficiently far away from the pole.

There are many ways to find an empty point on \Sn{3},
and this is the topic of a companion paper \cite{jjempty05}.
  % which shows how to find the optimal empty point, a heuristic empty point,
  % and a random empty point.
The most elegant solution is an optimal empty point, which basically involves
a computation of the vertices of the Voronoi diagram of the points on \Sn{3}.
This solution is ({\bf seems until we test?}) most appropriate for data sets 
of medium size since very small data sets do not require its power and
very large data sets tax its higher complexity.
The most appropriate solution for very large data sets is ({\bf may be?})
a divide and conquer technique.
This is possible because we know how to stitch subcurves together with high
continuity using the end-derivative control results of Section~\ref{sec:deriv}.
The simple idea behind the divide and conquer technique is to grow a dataset until
it deviates by too much from the initial quaternion, and then choose the antipode
of the initial quaternion as the empty point.
The remaining part of the dataset is solved recursively.
Details are in \cite{jjempty05}.
This paper also explores the existence of empty points,
and shows that an empty point will exist whenever the data set is smaller
than ---.
A dataset larger than this limit would implicitly 
demand the divide and conquer technique, not just because it is too large to
compute an empty point efficiently, but because an empty point simply does not exist
because of the density of the dataset.

Note that the rotation required by this pole avoidance is nontrivial,
since rotation in 4-space is different than rotation in 3-space: 
a companion paper \cite{jjrotation05} addresses this topic.

% One comes to the conclusion that either a sophisticated algorithm (based on the 
% construction of an optimal empty point) or a very simple algorithm (supported by divide
% and conquer) will be most appropriate to find the empty point, 
% depending on the size of the dataset.
% It is not advisable to use only the trivial divide and conquer method,
% since divide and conquer is overkill in most situations, 
% causing trouble and forcing derivative interpolation
% (and moreover the companion paper would then have no purpose :)).

\Comment{
This section is structured as follows.
We first discuss the problems associated with quaternions near the pole 
(Section~\ref{sec:behaviour}), then the issue of coordinate frame invariance
of our quaternion spline under rotation (Section~\ref{sec:invariance}).
Section~\ref{sec:polerotate} discusses the actual rotation away from the pole.
Finally, a divide and conquer strategy (Section~\ref{sec:divide}) is suggested by one 
of the methods for rotating the quaternions away from the pole.
% This motivates the treatment of derivatives in Section~\ref{sec:deriv}.
}

\Comment{
Another way of seeing this ill-conditioning is by observing the behaviour
of $M$ near the origin.
$M$ is undefined at the origin, due to the division by zero.
However, $M$ is also badly behaved {\bf near} the origin.
For example, consider approaching near the origin along the line $(0,0,t,.01)$.
In moving the short distance from $(0,0,.01,.01)$ to $(0,0,0,.01)$, 
the image under $M$ moves all the way from $(0,0,0,1)$ to $(-1,0,0,0)$.
That is, a move of distance $.01$ in $\Re^4$ maps to a move of angular distance
$\frac{\pi}{2}$ on \Sn{3}.
The portion of a curve in $\Re^4$ that is near the origin
will be stretched as it is mapped under $M$.
This is not desirable, as shown in Figure~\ref{fig:tooCloseToPole}.

This is not quite as strong or direct an argument for ill-conditioning as the previous
argument, since there is nothing forcing the user to use preimage points 
near the origin.
However, notice that if you do not stay close to the origin, 
then the large changes in the line with small changes on \Sn{3} mean that 
small distances on \Sn{3} will be filled by long curves (and thus not optimal
geodesics) since the preimage curve will be necessarily long.
}

% ----------------------------------------------------------------------------------

\clearpage

\subsection{Preserving invariance under rotation}
\label{sec:invariance}

Care must be taken when rotating the quaternions away from the pole,
since the inverse maps off the sphere are not invariant under rotation
(otherwise we would not need to rotate away from the pole!).
Yet the design of the orientation curve should be invariant to rotation
(or equivalently, invariant to the coordinate frame):
if given two quaternion datasets that differ only by a rotation,
Q and rot(Q), two quaternion splines $S$ and $S'$ should be generated
that differ only by this rotation: $S' = \mbox{rot}(S)$.
	% Obviously not affine invariant, otherwise we would not need
	% to rotate away from pole, since this would have no effect
	% on the quaternion spline shape.
% That is, before we consider the issue of rotation, we need to consider the issue of
% invariance under rotation.

The solution is to impose a canonical reference frame: 
the quaternions are rotated into the reference frame before 
the orientation curve is designed.
Of course, this rotation is reversed once the orientation curve has been designed.

The rotation into a reference frame is implemented as follows.
The first three quaternions are rotated into a canonical position.
Although a coordinate frame in 4-space is typically defined by the position of 
four points, only three points are necessary to define a unique frame of reference
for the quaternions,
since the added degree of freedom is locked down through the points lying 
on a sphere.\footnote{Think of the one degree of freedom in moving points on \Sn{1}
  although the points live in 2-space.}
To rotate a quaternion dataset $Q = \{q_i\}_{i=0}^n$ into the canonical frame,
\begin{itemize}
\item $q_0$ is rotated to (-1,0,0,0),  % may as well start away from the pole
\item then $q_1$ is rotated to the $\{x_2=0,x_3=0\}$ plane (without moving $q_0$),
\item then $q_2$ is rotated to the $\{x_2=0\}$ hyperplane (without moving $q_0$ or $q_1$).
\end{itemize}
% Of course, this rotation is also reversed at the end.
	% This rotation to a canonical frame for coordinate-frame invariance
	% is also useful in the following heuristic method
	% (or alternatively one has to be careful to rotate eigenvectors
	% in a coordinate-frame-independent way).
	% Rotation to a canonical frame allows use of standard rotation
	% matrix with normal n to best-fitting plane as first row,
	% n x e1 as 2nd row, n x e1 x e2 as 3rd row, and 
	% n x e1 x e2 x e3 as fourth row.
	% Similarly, for the optimal method.
	% However, a random choice is not repeatable, so we do not use it. 
The rotation of a point to another point, and the rotation of a point to a plane
while preserving the position of other points, is treated in \cite{jjrotation05}.

The observant reader may have noticed that we have established
coordinate-frame invariance for our quaternion splines on \Sn{3},
but should actually be more interested in coordinate-frame invariance 
of the objects in 3-space that are undergoing the motion.
	% (The latter is equivalent to coordinate-frame invariance in the rotation
	% group SO(3).)
Fortunately, these invariances are equivalent, as can be seen 
from the proof of Lemma~\ref{lem:constant}.
% if the object orientations are rotated by a fixed amount (from orientation q_i to cq_i),
% then the quaternion is also rotated by a fixed amount

\clearpage

\section{Bezier images}
\label{sec:bezierimage}

The present state of the design of the orientation curve is as follows.
The quaternions are first rotated into a reference frame, then away from the pole.
The quaternions are then mapped using the inverse map 
(perhaps accompanied with the mapping of some derivatives)
and freely interpolated.
The resulting curve must be mapped back to the sphere.
This is a straightforward mapping in theory: the subtlety arises from
preserving the control structure of the curve.
This section considers the mapping of a Bezier curve as an illustration
of how the control structure can be preserved under the mapping.

The mapping of a Bezier curve under a rational map to the sphere
yields a rational Bezier curve.
We illustrate the technique using the image of a cubic Bezier curve 
under the Euler map : other rational maps to the sphere, and other Bezier curves,
are dealt with analogously.
Under the Euler map, the degree of the curve doubles, since the map is quadratic.

\begin{theorem}
\label{thm:imagecurve}
The image of the cubic Bezier curve $\sum_{i=0}^3 b_i B_i^3(t)$ under the Euler map $M$ 
is the sextic rational Bezier curve $\frac{\sum_{i=0}^6 w_i \breve{b}_i B_i^6(t)}
                                          {\sum_{i=0}^6 w_i B_i^6(t)}$ 
with the following control points (where ${\cal M}$ is the bilinear map 
of (\ref{eq:calM})):
\begin{eqnarray}
\breve{b}_0 & = & \frac{{\cal M}(b_0,b_0)}{b_0 \cdot b_0} \nonumber \\
\breve{b}_1 & = & \frac{{\cal M}(b_0,b_1) + {\cal M}(b_1,b_0)}{2b_0\cdot b_1} \nonumber \\
\breve{b}_2 & = & \frac{{\cal M}(b_0,b_2) + 3{\cal M}(b_1,b_1) + {\cal M}(b_2,b_0)}{2b_0 \cdot b_2 + 3b_1 \cdot b_1} \nonumber \\
\breve{b}_3 & = & \frac{{\cal M}(b_0,b_3) + 9{\cal M}(b_1,b_2) + 9{\cal M}(b_2,b_1) + {\cal M}(b_3,b_0)}{2b_0 \cdot b_3 + 18 b_1 \cdot b_2} \nonumber \\
\breve{b}_4 & = & \frac{{\cal M}(b_1,b_3) + 3{\cal M}(b_2,b_2) + {\cal M}(b_3,b_1)}{2b_1 \cdot b_3 + 3 b_2 \cdot b_2} \nonumber \\
\breve{b}_5 & = & \frac{{\cal M}(b_2,b_3) + {\cal M}(b_3,b_2)}{2b_2 \cdot b_3} \nonumber \\
\breve{b}_6 & = & \frac{{\cal M}(b_3,b_3)}{b_3 \cdot b_3} \nonumber
\end{eqnarray}
and weights:
\begin{eqnarray}
w_0 & = & b_0 \cdot b_0 \nonumber \\
w_1 & = & b_0 \cdot b_1 \nonumber \\
w_2 & = & \frac{2b_0 \cdot b_2 + 3b_1 \cdot b_1}{5} \nonumber \\
w_3 & = & \frac{b_0 \cdot b_3 + 9b_1 \cdot b_2}{10} \nonumber \\
w_4 & = & \frac{2b_3 \cdot b_1 + 3b_2 \cdot b_2}{5} \nonumber \\
w_5 & = & b_3 \cdot b_2 \nonumber \\
w_6 & = & b_3 \cdot b_3 \nonumber
\end{eqnarray}
\end{theorem}
\prf See the appendix.
\QED

HOW ABOUT ARBITRARY BEZIER CURVE?

\begin{theorem}
\label{thm:imagesp}
The image under inverse stereographic projection 
of a cubic Bezier curve with control points $b_i = (b_{i1},\ldots,b_{i4})$
is a sextic rational Bezier curve 
with control points $\breve{b}_i = $ 
and weights $w_i = $.
This maps an arbitrary cubic curve to a curve on \Sn{3}.
\end{theorem}
\prf 
See the appendix.  (NOT COMPLETED.)
\QED

\clearpage

\section{The algorithm}
\label{sec:alg}

We can now present the complete algorithm for design of the orientation
component of a motion.
The problem's input is a set of quaternions $\{Q_i\}_{i=0}^n$ 
representing desired orientations along the motion, 
a set of opening derivatives $\{D_{0j}\}_{j=0}^p$ at $Q_0$,
and a set of ending derivatives $\{D_{nj}\}_{j=0}^q$ at $Q_n$.\footnote{These derivatives 
   must be valid derivatives for a curve
   on \Sn{3}.  For example, a first derivative must 
   lie in \Sn{3}'s tangent plane at $Q_0$.}
A rational Bezier curve with $C^k$ continuity is designed that 
interpolates the quaternion dataset and observes the end derivative data, as follows.
Suppose that the rational map $f$ to \Sn{3} is being used.

\begin{enumerate}
\item Rotate the quaternions and derivatives into a reference frame 
      (Section~\ref{sec:invariance}).
\item Rotate the quaternions and derivatives away from the pole of $f^{-1}$
      (Section~\ref{sec:pole}).
\item Map the rotated quaternions and derivatives under the inverse map $f^{-1}$, 
      away from the sphere (Section~\ref{sec:inverse}).
\item Build a $C^k$-continuous Bezier curve interpolating the points
      $f^{-1}(rot(\{Q_i\})$, the opening derivatives $f^{-1}(rot(\{D_{0j}\}))$,
      and the ending derivatives $f^{-1}(rot(\{D_{1j}\}))$ \cite{farin02}.
      This curve will need to be of degree {\bf $---$}.
\item Map the Bezier curve of step (5) onto the sphere using the rational map $f$ 
      (Section~\ref{sec:bezierimage}).
\item Rotate the rational Bezier curve of step (6), reversing the rotations 
      of steps (1) and (2).
\end{enumerate}

The quaternion dataset must be small enough to have an empty point.
If not, the dataset must be decomposed into subsets that do have an empty point,
and the subcurves spliced together with $C^k$ continuity using end derivative control.

A word about the definition of knot sequences for the orientation curve is required.
When interpolating $\{f^{-1}(Q_i)\}$,
the knot sequence for this curve should not be determined from these points,
but instead from the points $\{Q_i\}$, so that the final curve will be well parameterized.
Moreover, when calculating the knot sequence, distance on the surface \Sn{3}
should be used rather than Euclidean distance (e.g., for chord length or centripetal
parameterization).
The distance metric on the surface is simple:
\[
\mbox{dist}(A,B) = \theta r = \theta = cos^{-1}(A \cdot B)
\]
where $\theta$ is the angle subtended by the points A and B on \Sn{3}
and $r$ is the radius of the sphere.

% this is elaborated in next motion paper:
% DON'T USE SAME KNOT SEQUENCE FOR POSITION AND ORIENTATION CURVES.
% DEFINE CURVES TOTALLY INDEPENDENTLY AND USE KEYFRAME CORRESPONDENCE TO
% COORDINATE MOTION.
% FOR EXAMPLE, IF POSITION KEYFRAMES HAVE KNOTS 0,1,2,3,4,5 AND 
% ORIENTATION KEYFRAMES HAVE KNOTS 0 .1 .6 10 20 21, THEN
% SAMPLE AT 3.5 ON POSITION CURVE CORRESPONDS TO SAMPLE AT 15 ON ORIENTATION CURVE.

% To control the speed of rotation of the tumbling body,
% we would like to control speed along the orientation curve.
% We cannot generate perfectly regular changes of orientation,
% since no rational curve can be parameterized by a rational function
% of its arc length (Farouki and Sakkalis \cite{farouki91}).  NOT RELEVANT
% However, we do have a great deal of control over speed, through the curve's knot 
% sequence, as discussed below.
% Moreover, this control is simple and intuitive.

% NO: don't emphasize the power to control speed, it is pretty much of a wash once
% the knot sequence is coordinated with the position curve
\Comment{
\subsection{Choice of knot sequence}

Since the orientation curve is a Bezier spline, we can use its knot sequence to control
the speed along the curve.
For example, we might like to approximate an arc-length parameterization
(since it is impossible to attain exactly)
for a uniform change of orientation.
Chord-length parameterizations assign knot intervals proportional
to the Euclidean distance between data points.
Since our points and curve lie on a sphere, it is more appropriate
to use a non-Euclidean metric measuring distance on the sphere:
\[
\mbox{dist}(A,B) = \theta r = \theta = cos^{-1}(A \cdot B)
\]
(where $\theta$ is the angle subtended by the points A and B on the unit sphere).
Non-Euclidean variants of other parameterizations, such as centripetal,
could also be used.

In many cases, perfectly regular tumbling will not be our goal,
and manipulation of the knot sequence can just as easily achieve other effects.
The above non-Euclidean metric will still be useful in these contexts.

This control of the speed along the curve via the knot sequence 
is another benefit of using a Bezier representation of the quaternion spline.
}

\Comment{
\subsection{Choice of frames based on arc length}

THIS IS NOT VALID: THE ORIENTATION CURVE MUST BE COORDINATED WITH THE POSITION CURVE,
SO WE DO NOT HAVE THE FREEDOM TO CHOOSE THE FRAMES TO GUARANTEE SMOOTHLY 
VARYING ORIENTATION, SINCE THAT WOULD ALSO DEFINE POSITIONS, AND IN A NONDESIRABLE WAY.
Another simple mechanism for effective speed control in the animation
is to choose intermediate frames based on arc-length.
Suppose that, for example, we wish to simulate constant speed along
the orientation curve ({\em i.e.}, arc length parameterization).
Based on computation of arc length, the orientation of intermediate frames
can be chosen at approximately equal spacing along the orientation curve.
The length of a segment should be computed by subdivision
(until an acceptable degree of accuracy is achieved)
rather than by exact computation via the integral 
$\int_a^b \|c'(t)\| dt$ (for the segment $c(t)$).
This method is possible since we have a closed form Bezier representation
of the orientation curve. 
}

% -----------------------------------------------------------------------

\subsection{Examples}

TEST HUGE RANDOM DATASETS AND SEE HOW MANY SUBSETS ARE GENERATED.

\clearpage

\subsection{Cusps}
\label{sec:cusp}

% Verbatim from GI95.

% DOES C^2 CONTINUOUS IMPLY NO CUSPS?
% NO, BECAUSE YOU COULD HAVE COINCIDENT CONTROL POINTS
% IN GENERAL, THERE IS NO CONTRADICTION BETWEEN CUSPS
% AND C^I CONTINUITY SINCE CUSPS SIMPLY MEAN THAT THE
% DERIVATIVE BECOMES ZERO, NOT DISCONTINUOUS.
% THUS, WE CANNOT USE C^I CONTINUITY TO RULE OUT CUSPS.

A cusp in an orientation curve is associated with an abrupt change
of orientation, which is undesirable in a motion.
All of the orientation curves created by other methods can contain cusps
(e.g., Shoemake calls them kinks \cite{shoemake85}), and so can ours.
However, the next lemma shows that our orientation curves rarely have cusps.
It also gives testable conditions for cusps, which can be used to avoid them altogether,
as discussed below.

Consider the design of orientation curves using the Euler map back to the sphere.
The position interpolation (of the quaternion inverse images) will not generate cusps:
only the mapping back to the sphere might introduce them.
% Farin 3rd edition p. 63
	% proof: the control polygon of the hodograph of a nonplanar cubic Bezier
	% will consist of 3 points, not all coplanar with the origin.
	% So the plane of the conic hodograph does not contain the origin,
	% implying that it does not intersect the origin (since a conic is 
	% always a plane curve).  Thus, the cubic Bezier does not have a cusp,
	% since the condition for a cusp is that the hodograph passes through
	% the origin.
The following lemma shows that $M$ only introduces cusps in unusual situations.

\begin{lemma}
\label{lem:cusp}
If $M$ introduces a cusp in the curve $C(t)$ at $t=t_0$,
then one of the following conditions must hold:
\begin{itemize}
\item
	$C(t_0) = (0,0,0,0)$
\item
	$C'(t_0) = (0,0,0,0)$
\item
	$C(t_0) \cdot C'(t_0) = 0$ and $C_4(t_0) = C'_4(t_0) = 0$
\item
	$\ddt(M_1(C(t_0))) = kM_1(C(t_0))$, where $M_1$ is defined below
	and $k = \frac{\ddt(M_1(C)) \cdot M_1(C)}{M_1(C) \cdot M_1(C)}$.
\end{itemize}
\end{lemma}
\prf
See the appendix.
\QED

% These situations should be easy to avoid during design
% (especially since design is interactive with our method).

Thus, $M$ can introduce a cusp in $C(t)$ only when the curve $C(t)$ passes through
the origin, its hodograph $C'(t)$ passes through the origin,
the vectors to the curve $C(t_0)$ and hodograph $C'(t_0)$ 
{\em at the same parameter value $t_0$}, are orthogonal and lie in the hyperplane $x_4=0$,
or the vectors to the
curve $M_1(C(t))$ and its tangent, at the same parameter value $t_0$, 
are multiples in the special ratio $k$ of the theorem.
These are all pathological conditions. %  which most curves will not contain.

We can test for these conditions and, if they are detected,
perturb the end derivatives to generate a slightly different orientation curve
(that still interpolates the original quaternions) that avoids these conditions.
This will be successful since the conditions are fragile.

\clearpage

\section{Conclusions}
\label{sec:conclude}

The proposed algorithm for orientation interpolation is based on a reduction 
to the well understood problem of position interpolation,
yielding a simple, efficient, smooth and rational solution with
end derivative control.
The fundamental tool in this reduction is a rational map to the 3-sphere 
(Section~\ref{sec:map}).

The method is actually a family of methods, based on different choices for
the rational map to the sphere and different choices of the interpolating curve.
The choice of interpolating curve dictates the continuity of the solution.
For example, the classical cubic interpolating B-spline yields $C^2$ continuity
and two extra degrees of freedom for end derivative control,
while a quintic interpolating curve offers the potential of $C^4$ continuity
and ---- extra degrees of freedom for end derivative control.

Once a map (Section~\ref{sec:map}) and its inverse (Section~\ref{sec:inverse}) 
have been defined, there are two remaining issues: 
how to avoid the pole of the inverse map (Section~\ref{sec:pole}),
and how to map a Bezier curve (Section~\ref{sec:bezierimage}).

\vspace{1in}

IMPLEMENT QUINTIC INTERPOLATING CURVE WITH QUARTIC CONTINUITY.

This reduction brings several benefits, all inherited from position interpolation:
simplicity, rationality, continuity, and efficiency.

Line interpolation is an interesting future research topic 
(see Section~\ref{sec:inveuler}).

Derivative control is useful even if the divide-and-conquer strategy is not used,
as it allows more control of the quaternion spline, such as in traditional cubic spline
interpolation.

See 'The Approach' section in hermiteQ.tex (p. 20-21 in January 2005 version)

----

Move up:
This paper extends the conference paper \cite{jjwilliams95} by developing
general rational maps to the sphere, examining in detail the issue of pole avoidance
(inherent to any rational method) and providing four solutions, 
providing a divide-and-conquer technique that simplifies pole avoidance and
still preserves continuity, cleaning up the theory of the image of a Bezier curve,
and in general streamlining the method.
Full software incorporating these improvements is now available at our website
{\bf www.cis.uab.edu/jj}.
(Advertise computer camp, graduate program director, Bezier code.)

---

Add above:
Note that there is typically only a need for the design of open orientation curves,
since a motion is usually not cyclical.
This simplifies the design of orientation curves through divide and conquer,
since as we design the next subcurve, it only has to splice smoothly with its predecessor.
However, the use of divide-and-conquer technique to design a {\em closed} orientation
curve through subcurves requires some subtlety,
because the splicing of the final segment into the curve requires continuity
with both the preceding curve (as always) {\em and} the following curve.
This requires extra degrees of freedom for end derivative control.
Consider $C^2$ continuity as an example.
An open orientation curve may be designed using cubic interpolating curves,
since their two remaining degrees of freedom may be used to specify two end derivatives
at the beginning of a subcurve.
However, with a closed curve, 
the final subcurve will need four degrees of freedom, and so cubic interpolating
curves cannot be used for the design of this final subcurve.
Either the last subcurve, or for uniformity every subcurve, must use a quintic 
interpolating curve for its four degrees of freedom(??).

---

other journal paper: smooth visibility graphs

other journal paper: Voronoi diagrams on $S^3$

\clearpage

AND HERE

\bibliographystyle{plain}
\begin{thebibliography}{99}

\bibitem[Ballard 82]{ballard82}
Ballard, D. and C. Brown (1982)
Computer Vision.
Prentice-Hall (Englewood Cliffs, NJ).

\bibitem[Barr 92]{barr92}
Barr, A. and B. Currin and S. Gabriel and J. Hughes (1992)
Smooth Interpolation of Orientations with Angular Velocity Constraints
using Quaternions.
SIGGRAPH '92, 313--320.

\bibitem{brotman88}
Brotman, L and A. Netravali (1988)
Motion interpolation by optimal control.
SIGGRAPH '88, 309--315.

\bibitem{coxeter61}
Coxeter, H. (1961)
Non-Euclidean Geometry.

\bibitem[Dickson 52]{dickson52}
Dickson, L.E. (1952) History of the Theory of Numbers: Volume II,
Diophantine Analysis.  Chelsea (New York).

\bibitem[Dietz 93]{dietz93}
Dietz, R. and J. Hoschek and B. Juttler (1993)
An algebraic approach to curves and surfaces on the sphere and
on other quadrics.
Computer Aided Geometric Design 10, 211-229.

\bibitem[Duff 85]{duff85}
Duff, T. (1985)
Quaternion splines for animating orientation.
1985 Monterey Computer Graphics Workshop, 54--62.

\bibitem[Farin 02]{farin02}
Farin, G. (2002)
Curves and Surfaces for CAGD: A Practical Guide (5th edition).
Morgan Kaufmann (San Francisco).

\bibitem{farouki91}
Farouki, R.T. and T. Sakkalis (1991) Real rational curves are not
"unit speed".  Computer Aided Geometric Design 8, 151--157.

\bibitem[Foley 96]{foley96}
Foley, J. and A. van Dam and S. Feiner and J. Hughes (1996)
Computer Graphics: Principles and Practice (2nd edition in C).
Addison-Wesley (Reading, Massachusetts).

\bibitem{gabriel85}
Gabriel, S. and J. Kajiya (1985) Spline interpolation in curved space.
In SIGGRAPH '85 course notes `State of the art image synthesis'.

\bibitem[Golub 89]{golubvanloan89}
Golub, G. and C. Van Loan (1989)
Matrix Computations.
2nd edition, Johns Hopkins University Press (Baltimore).

\bibitem{goldstein50}
Goldstein, H. (1950)
Classical Mechanics.
Addison-Wesley (Reading, Massachusetts).
REPLACE BY 3rd EDITION.

\bibitem[Grassia 98]{grassia98}
Grassia, F. S. (1998)
Practical parameterization of rotations using the exponential map.
Journal of Graphics Tools 3(3), March, 29--48.

\bibitem{hanson}
Hanson, A. SIGGRAPH Course Notes on Quaternions.

\bibitem[Hart 94]{hart94}
Hart, J. and G. Francis and L. Kauffman (1994)
Visualizing Quaternion Rotation.
ACM Transactions on Graphics 13(3), July, 256--276.

\bibitem[Herstein 75]{herstein75}
Herstein, I. (1975) Topics in Algebra.
2nd edition, John Wiley (New York).

\bibitem[Hoschek 92]{hoschekSeemann92}
Hoschek, J. and G. Seemann (1992)
Spherical splines.
Mathematical Modeling and Numerical Analysis, 26(1), 1--22.

\bibitem{hoschekLasser0?}
Hoschek.

\bibitem[Johnstone and Williams 95]{jjjimbo95}
Johnstone, J. and J. Williams (1995)
Rational Control of Orientation for Animation.
{\em Graphics Interface '95}, Quebec City, 179--186.

\bibitem[Johnstone 99b]{jj98b}
Johnstone, J.K. (1999)
The Most Natural Map to the $n$-sphere.
Manuscript.
Available at http://www.cis.uab.edu/info/faculty/jj/cos.html.

\bibitem{jjrm05}
Johnstone, J. (2005) Rational maps to the sphere for motion design.
Technical report, Dept. of Computer and Information Sciences, UAB, February 2005.

\bibitem{jjempty05}
Johnstone, J. (2005) Finding an empty point on \Sn{3}.
Technical report, Dept. of Computer and Information Sciences, UAB, 2005.

\bibitem{jjrotation05}
Johnstone, J. (2005) Rotation in 4-space.
Technical report, Dept. of Computer and Information Sciences, UAB, 2005.

\bibitem{jw03}
J\"uttler, B. and W. Wang (2003)
The shape of spherical quartics.
CAGD 20 (8-9), December, 621--636.

\bibitem[Kendall 61]{kendall61}
Kendall, M.G. (1961)
A Course in the Geometry of n Dimensions.
Charles Griffin (London).

\bibitem[Kim 95]{kim95}
Kim, M.-J. and M.-S. Kim and S. Shin (1995)
A General Construction Scheme for Unit Quaternion Curves
with Simple Higher Order Derivatives.
SIGGRAPH '95, 369--376.

\bibitem[Kreyszig 63]{kreyszig63}
Kreyszig, E. (1963)
Differential Geometry.
Dover (New York).

\bibitem{kreyszig59}
Kreyszig, E. (1959) Differential Geometry.
Dover (New York).

\bibitem[Lang 79]{lang79}
Lang, S. (1979)
Calculus of Several Variables, 2nd edition.
Addison-Wesley (Reading, Mass).

\bibitem[Misner 73]{misner73}
Misner, C. and K. Thorne and J. Wheeler (1973)
Gravitation.
W.H. Freeman (San Francisco).

\bibitem[Nam 95]{nam95}
Nam, K.-W. and M.-S. Kim (1995)
Hermite interpolation of solid orientations based on a smooth blending
of two great circular arcs on SO(3).
Proc. of CG International '95.

\bibitem[Nielson 92]{nielson92}
Nielson, G. and R. Heiland (1992)
Animated rotations using quaternions and splines on a 4D sphere.
Programming and Computer Software, 145--154.

\bibitem[Nielson 93]{nielson93}
Nielson, G. (1993)
Smooth interpolation of orientations.
In Models and Techniques in Computer Animation, Springer-Verlag (New York),
75--93.

\bibitem[Park 95]{park95}
Park, F. and B. Ravani (1995)
Bezier Curves on Riemannian Manifolds and Lie Groups with
Kinematics Applications.
Transactions of the ASME 117, 36--54.

\bibitem[Park 97]{park97}
Park, F. and B. Ravani (1997)
Smooth Invariant Interpolation of Rotations.
ACM Transactions on Graphics 16(3), 277--295.

\bibitem[Pletinckx 89]{pletinckx89}
Pletinckx, D. (1989) 
Quaternion calculus as a basic tool in computer graphics.
The Visual Computer 5, 2--13.

\bibitem[Popovic 99]{popovic99}
Popovic, Z. and A. Witkin (1999)
Physically Based Motion Transformation.
SIGGRAPH '99, 11--20.

\bibitem[Ramamoorthi 97]{rama97}
Ramamoorthi, R. and A. Barr (1997)
Fast Construction of Accurate Quaternion Splines.
SIGGRAPH '97, Los Angeles, 287--292.

\bibitem[Schlag 91]{schlag91}
Schlag, J. (1991) Using geometric constructions to interpolate
orientation with quaternions.  In Graphics Gems II, Academic Press (New York),
377--380.

\bibitem[Preparata 85]{shamos85}
Preparata, F. and M. Shamos (1985)
Computational Geometry: An Introduction.
Springer-Verlag (New York).

\bibitem[Shoemake 85]{shoemake85}
Shoemake, K. (1985) Animating rotation with quaternion curves.
SIGGRAPH '85, San Francisco, 19(3), 245--254.

\bibitem[Wang 93]{wang93}
Wang, W. and B. Joe (1993)
Orientation Interpolation in Quaternion Space using Spherical Biarcs.
Graphics Interface '93, 24--32.

\bibitem[Wang 94]{wang94}
Wang, W. (1994)
Rational Spherical Curves.
Technical Report, Dept. of Computer Science, University of Hong Kong.

\bibitem[Witkin 88]{witkin88}
Witkin, A. and M. Kass (1988)
Spacetime Constraints.
Computer Graphics 22(4) (Proc. SIGGRAPH '88), 159--168.

% \bibitem[Woo 97]{opengl97}
% Woo, M. and J. Neider and T. Davis (1997)
% OpenGL Programming Guide (2nd edition).
% Addison-Wesley Developers Press (Reading, Mass.),
% pp. 464-8.

\end{thebibliography}

CREATE BIB FILE.

\clearpage

\section{Appendix}
\label{sec:appendix}

\subsection{Proof of Lemma~\ref{lem:inverse}}

This proof works in projective space.
% We shall make use of projective space in some of the proofs in this paper.
Real {\bf projective $n$-space} $P^n$ is the space 
$\{ (x_1,x_2,\ldots,x_{n+1}) : x_i \in \Re, \mbox{not all zero} \}$
under the equivalence relation 
\begin{equation}
\label{eq:projequivalence}
(x_1,\ldots,x_{n+1}) = k(x_1,\ldots,x_{n+1}),\ \ k \neq 0 \in \Re.
\end{equation}
The point $(x_1,\ldots,x_{n+1})$ in projective $n$-space, $x_{n+1} \neq 0$,
is equivalent to the point $(\frac{x_1}{x_{n+1}},\ldots,\frac{x_n}{x_{n+1}})$
in $n$-space.
The point $(x_1,\ldots,x_n,0)$ in projective $n$-space represents the point
at infinity in the direction $(x_1,\ldots,x_n)$.
All equations must be homogenized (by appropriate multiplication 
with the extra homogeneous coordinate) as they are translated to projective space.
% To translate from $n$-space to projective $n$-space, the point 
% $(x_1,\ldots,x_n)$ is typically transformed into the point $(x_1,\ldots,x_n,1)$.
% See \cite{harris92} for more details on projective space.
% \bibitem[Harris 92]{harris92}
% Harris, J. (1992)
% Algebraic Geometry: A First Course.
% Springer-Verlag (New York).

\vspace{.1in}

\prf
In projective space $P^4$, the map $M$ becomes
\begin{equation}
\label{eq:proj}
	 M(x_1,x_2,x_3,x_4,x_5) =
	 (x_1^2 + x_2^2 + x_3^2 - x_4^2,\ 
	 2x_1 x_4,\ 2x_2 x_4,\ 2x_3 x_4,\ 
	 x_1^2 + x_2^2 + x_3^2 + x_4^2).
\end{equation}
Let $p = (p_1,p_2,p_3,p_4,1) \in S^3 \subset P^4$.
We want to determine the conditions on $q = (q_1,q_2,q_3,q_4,q_5) \in P^4$
so that $M(q_1,q_2,q_3,q_4,q_5) = p$.
Suppose that $M(q_1,q_2,q_3,q_4,q_5) = p$.
Using (\ref{eq:proj}) and (\ref{eq:projequivalence}), we have
\begin{equation}
\label{eq1}
	q_1^2 + q_2^2 + q_3^2 - q_4^2 = kp_1
\end{equation}
\begin{equation}
\label{eq2}
	2q_1q_4	= kp_2
\end{equation}
\begin{equation}
\label{eq3}
	2q_2q_4	= kp_3
\end{equation}
\begin{equation}
\label{eq4}
	2q_3q_4	= kp_4
\end{equation}
\begin{equation}
\label{eq5}
	q_1^2 + q_2^2 + q_3^2 + q_4^2 = k
\end{equation}
for some $k \neq 0$.
$q_5$ is arbitrary, since it does not appear in these equations.
Subtracting (\ref{eq1}) from (\ref{eq5}), 
we have $2q_4^2 = k(1 - p_1)$
or 
\begin{equation}
\label{eq:q4}
q_4 = \pm \sqrt{\frac{k(1-p_1)}{2}}
\end{equation}
%
{\bf Case 1:\ }
Suppose $p_1 = 1$. Then $p = (1,0,0,0,1)$ since $p \in \Sn{3}$.
$q_4 = 0$ by (\ref{eq:q4})
and $M(q) = (q_1^2 + q_2^2 + q_3^2, 0, 0, 0, q_1^2 + q_2^2 + q_3^2)
       = (1,0,0,0,1)$ for any values of $q_1,q_2,q_3$, not all zero.
That is, $M(q) = p$ if and only if $q \in H$ where $H$ is 
the hyperplane $x_4 = 0$ minus the origin.
Equivalently, $M^{-1}(p) = H$.
This is the special behaviour at the pole.\\
%
{\bf Case 2:\ }
Suppose $p_1 \neq 1$.
Then $p_1 < 1$ since $p \in \Sn{3}$. 
From (\ref{eq:q4}), $q_4 \neq 0$ and $q_4$ is a real number.
From (\ref{eq2}-\ref{eq4}), $q_i = \frac{kp_{i+1}}{2q_4}$ for $i=1,2,3$.
% \[
%	q = (\frac{kp_2}{2q_4}, \frac{kp_3}{2q_4}, \frac{kp_4}{2q_4}, q_4, q_5)
% \]
By the properties of projective space, 
\[
	q = \frac{2q_4}{k} q
	  = (p_2, p_3, p_4, \frac{2q_4^2}{k},\frac{2q_4q_5}{k})
\]
then using (\ref{eq:q4}),
\[
	q = (p_2,p_3,p_4,1-p_1, \pm \sqrt{\frac{2(1-p_1)}{k}} q_5)
\]
Since $q_5$ and $k \neq 0$ are arbitrary,
\[
	q = (p_2,p_3,p_4,1-p_1, k') \hspace{1in} k' \in \Re
\]
We have shown that $M(q) = p$ only if 
$q = (p_2,p_3,p_4,1 - p_1,k')$.

On the other hand, if $q = (p_2,p_3,p_4,1 - p_1,k')$, $k' \in \Re$
and $p_1 \neq 1$,
then 
\[
\scriptsize{M(q) = (p_2^2 + p_3^2 + p_4^2 - (1 - p_1)^2,\ 
	2p_2 (1 - p_1),\ 2p_3 (1 - p_1),\ 2p_4 (1 - p_1),\ 
	p_2^2 + p_3^2 + p_4^2 + (1 - p_1)^2).}
\]
Using $p_1^2 + p_2^2 + p_3^2 + p_4^2 = 1$,
\[
M(q) = (2p_1(1 - p_1),\ 2p_2 (1 - p_1),\ 2p_3 (1 - p_1),\ 2p_4 (1 - p_1),\ 
2(1 - p_1))
\]
or $M(q) = (p_1,p_2,p_3,p_4,1)$ using (\ref{eq:projequivalence}).
Thus, $M(q) = p$ if and only if 
$q = (p_2,p_3,p_4,1 - p_1,k')$, $k' \in \Re$.
Translating back from projective space,
$M(q) = p$ if and only if 
$q = t(p_2,p_3,p_4, 1-p_1)$, $t \neq 0 \in \Re$.
% Equivalently, $M^{-1}(p) = t(p_2,p_3,p_4, 1-p_1)$, $t \neq 0 \in \Re$.
It is understandable that the preimage does not contain the origin, 
since $M$ is undefined there.
\QED
% $q \in M^{-1}(p)$ if and only if $M(q) = p$.

\subsection{Closest point vs. point on \Sn{3}}
\label{sec:closest}

In this section, we establish that the choice of $p_i$ as the point on \Sn{3}
(Section~\ref{sec:inveuler})
is a good approximation to the closest point to $p_{i-1}$.
Consider two unit quaternions $q_1 = (a,b,c,d)$ and $q_2 = (A,B,C,D)$.
Let $\epsilon$ be their difference in $x_1$-coordinate and $\theta$
be the angle between $q_1$ and $q_2$.
Consider the difference between the choice of inverse image as closest point $p_2$
and point $p_2'$ on \Sn{3} (Figure~\ref{fig:QQdist}).
The distance between $p_2$ and $p_2'$ is $1-\cos \alpha$, where
$\alpha$ is the angle between the inverse image lines $M^{-1}(q_1)$ and
$M^{-1}(q_2)$ (see Figure~\ref{fig:QQdist} again).
$\cos \alpha$ can be reexpressed as follows:
\[
	\cos \alpha = \frac{(b,c,d,1-a)}{\sqrt{2-2a}}
	\cdot
	\frac{(B,C,D,1-A)}{\sqrt{2-2A}}
	= \frac{\cos \theta + 1 - a - A}
	       {2\sqrt{1-a-A+aA}}
\]
since $\cos\theta = (a,b,c,d) \cdot (A,B,C,D)$.
Reexpressing $A$ as $a + \epsilon$,
\[
\cos \alpha = \frac{\cos \theta + 1 - 2a - \epsilon}
	       {2\sqrt{1-2a-\epsilon+a^2 + \epsilon a}}
	= \frac{\cos \theta + 1 - 2a - \epsilon}
	       {2\sqrt{(1-a-\frac{\epsilon}{2})^2 - \frac{\epsilon^2}{4}}}
\]
Suppose that $\frac{\epsilon^2}{4}$ is very small and can be ignored,
and that $\theta$ is small so that $\cos \theta$ is close to 1.
This is true if the quaternion data is densely sampled.
Then 
\[
\cos \alpha \approx \frac{2 - 2a - \epsilon}{2(1-a-\frac{\epsilon}{2})} = 1,\ \ 
1-\cos \alpha \approx 0
\]
and the distance between $p_2$ and $p_2'$, $1 - \cos\alpha$, is small.  
We conclude that the choice of $p_i$ as the 'point on \Sn{3}'
is a good approximation to the choice of $p_i$ as the closest point to $p_{i-1}$,
especially when the quaternion data is densely sampled.

\begin{figure}
\vspace{2.5in}
\special{psfile=/usr/people/jj/modelTR/3-spline/img/EulerRotThm.ps
	 hoffset=150}
\caption{Any change of orientation can be expressed as a rotation about a fixed axis}
% file: EulerRotThm.showcase
% tops EulerRotThm.rgb -m 6.5 1.5 > EulerRotThm.ps
\label{fig:eulerRotation}
\end{figure}

\begin{figure}
\vspace{2.5in}
\special{psfile=/usr/people/jj/modelTR/3-spline/img/dist1-cosAlpha.ps
	 hoffset=150}
\caption{The distance between $p_2$ and $p_2'$ is $1-\cos\alpha$}
% file: dist1-cosAlpha.showcase
% tops dist1-cosAlpha.rgb -m 6.5 1.5 > dist1-cosAlpha.ps
\label{fig:QQdist}
\end{figure}

\subsection{Proof of Lemma~\ref{lem:firstderiv}}
\label{sec:firstderiv}

\prf
Let $C(t) = (x_1(t),x_2(t),x_3(t),x_4(t))$ be the quaternion curve.\footnote{We 
  hypothesize that the quaternion spline is already known,
  which is of course not true at this stage.
  However, it yields the desired constraint on the image curve when we restrict
  attention to behaviour at the opening of the quaternion spline, which we do know.}
Its image is 
\[
M^{-1}|_{S3}(C(t)) = \frac{1}{\sqrt{2-2x_1(t)}} (x_2(t),x_3(t),x_4(t),1-x_1(t))
\]
Consider the first coordinate of the derivative of the image curve:
% This proof is simply a matter of taking the derivative of this curve.
% If we let $g(t) = (2-2x_1(t))^{\frac{1}{2}}$, then
% \[
% g'(t) = \frac{-x'_1}{\sqrt{2-2x_1}}
% \]
\[
\frac{x'_2 \sqrt{2-2x_1} + \frac{x'_1 x_2}{\sqrt{2-2x_1}}}{2-2x_1}
= \frac{x'_2}{\sqrt{2-2x_1}} + \frac{x'_1x_2}{(2-2x_1)^{3/2}}
% = \frac{x'_2(2-2x_1) + x'_1x_2}{(2-2x_1)^{3/2}}
% = \frac{x'_1x_2 - 2x_1x'_2 + 2x'_2}{(2-2x_1)^{3/2}}
\]
The second and third coordinates are computed analogously.
Since the fourth coordinate of the image curve is
$\frac{1-x_1}{\sqrt{2-2x_1}} = (\frac{1-x_1}{2})^{1/2}$,
the fourth coordinate of the derivative is:
\[
\frac{1}{2} (\frac{1-x_1}{2})^{-1/2}(\frac{-x'_1}{2})
= \frac{-x'_1}{2\sqrt{2-2x_1}}
\]
Therefore, the derivative of the image curve is:
\[
(\frac{x'_2}{\sqrt{2-2x_1}} + \frac{x'_1x_2}{(2-2x_1)^{3/2}},\ 
 \frac{x'_3}{\sqrt{2-2x_1}} + \frac{x'_1x_3}{(2-2x_1)^{3/2}},\ 
 \frac{x'_4}{\sqrt{2-2x_1}} + \frac{x'_1x_4}{(2-2x_1)^{3/2}},\ 
 \frac{-x'_1}{2\sqrt{2-2x_1}}) 
\]
By evaluating this derivative at $t=0$, we have the result.
% Collecting, the derivative of the inverse Euler map is:
% \[
% \frac{1}{\sqrt{2-2x_1}} 
% (\frac{x'_1x_2 - 2x_1x'_2 + 2x'_2}{2-2x_1},
%  \frac{x'_1x_3 - 2x_1x'_3 + 2x'_3}{2-2x_1},
%  \frac{x'_1x_4 - 2x_1x'_4 + 2x'_4}{2-2x_1},
% -\frac{x'_1}{2})
% \]
\QED

Proof of Lemma~\ref{lem:secderiv} on second derivative:

\prf
The first derivative of the image curve is 
\[
(\frac{x'_2}{\sqrt{2-2x_1}} + \frac{x'_1x_2}{(2-2x_1)^{3/2}},\ 
 \frac{x'_3}{\sqrt{2-2x_1}} + \frac{x'_1x_3}{(2-2x_1)^{3/2}},\ 
 \frac{x'_4}{\sqrt{2-2x_1}} + \frac{x'_1x_4}{(2-2x_1)^{3/2}},\ 
 \frac{-x'_1}{2\sqrt{2-2x_1}}) 
\]
Consider the first coordinate of the second derivative of the image curve:
\[
\frac{x''_2 \sqrt{2-2x_1} + x'_1x'_2 (2-2x_1)^{-1/2}}{2-2x_1} +
\frac{(x''_1 x_2 + x'_1 x'_2)(2-2x_1)^{3/2} + 3x'_1 x_2 \sqrt{2-2x_1}(x'_1)}{(2-2x_1)^3}
\]
\[
= \frac{x''_2}{\sqrt{2-2x_1}} + \frac{2x'_1 x'_2 + x''_1 x_2}{(2-2x_1)^{3/2}}
+ \frac{3(x'_1)^2 x_2}{(2-2x_1)^{5/2}}
\]
The second and third coordinates are analogous.
The fourth coordinate of the second derivative of the image curve is:
\[
\frac{-x''_1(2)\sqrt{2-2x_1} + x'_1(2-2x_1)^{-1/2} (-x'_1)}{4(2-2x_1)}
\]
\[
= \frac{-x''_1}{2\sqrt{2-2x_1}} - \frac{(x'_1)^2}{4(2-2x_1)^{3/2}}
\]
This yields the result.
\QED

\clearpage

\subsection{Proof of Theorem~\ref{thm:imagecurve}}

\prf
Let $c(t)$ be the cubic Bezier curve.
Since the image of $c(t)$ is a rational Bezier curve, it is easier 
to work in projective space.
In projective space, the map $M$ becomes
\[
	 (x_1,x_2,x_3,x_4,x_5) \rightarrow
	 (x_1^2 + x_2^2 + x_3^2 - x_4^2,\ 
	 2x_1 x_4,\ 2x_2 x_4,\ 2x_3 x_4,\ 
	 x_1^2 + x_2^2 + x_3^2 + x_4^2).
\]
Let the image curve be expressed in projective space: 
$M(c(t)) = (m_1(t),m_2(t),m_3(t),m_4(t),m_5(t))$.
Each coordinate $m_i(t)$ may be simplified using the product rule 
of Bernstein polynomials \cite{farin02}.
The product rule of Bernstein polynomials \cite{farin02} states:
\[
B_i^m(t) B_j^n(t) = \frac{\choice{m}{i} \choice{n}{j}}{\choice{m+n}{i+j}} B_{i+j}^{m+n}(t)
\]
Consider $m_5(t)$ as an example.
\begin{eqnarray}
m_5(t) & = & [\sum_{i=0}^3 B_i^3(t) b_{i1}]^2 + 
	\ldots + [\sum_{i=0}^3 B_i^3(t) b_{i4}]^2 \nonumber \\
     & = &  \sum_{i=0}^3 \sum_{j=0}^3 
	\frac{\choice{3}{i} \choice{3}{j}}{\choice{6}{i+j}}
       B^6_{i+j}(t) (b_{i1} b_{j1} + \ldots + b_{i4} b_{j4}) \nonumber
\end{eqnarray}
Letting $k=i+j$, 
\[ m_5(t) = \sum_{k=0}^6 B_k^6(t) 
	\sum_{\begin{array}{c}  \mbox{\footnotesize{$0 \leq i \leq 3$}} \\ 
			     \mbox{\footnotesize{$0 \leq j \leq 3$}} \\ 
			     \mbox{\footnotesize{$i+j=k$}}
			     \end{array}} 
	\frac{\scriptchoice{3}{i} \scriptchoice{3}{j}}{\scriptchoice{6}{k}}
	(b_{i1} b_{j1} + \ldots + b_{i4} b_{j4}) \]
The other coordinates can be computed analogously to yield:
\[ M(c(t)) = 
   \sum_{k=0}^6 B_k^6(t)
	\sum_{\begin{array}{c}  \mbox{\footnotesize{$0 \leq i \leq 3$}} \\ 
			     \mbox{\footnotesize{$0 \leq j \leq 3$}} \\ 
			     \mbox{\footnotesize{$i+j=k$}}
			     \end{array}} 
	\frac{\choice{3}{i} \choice{3}{j}}{\choice{6}{k}}
	\left( \begin{array}{c}
            b_{i1} b_{j1} + b_{i2} b_{j2} + b_{i3} b_{j3} - b_{i4} b_{j4} \\
            2b_{i1} b_{j4} \\
            2b_{i2} b_{j4} \\
            2b_{i3} b_{j4} \\
            b_{i1} b_{j1} + b_{i2} b_{j2} + b_{i3} b_{j3} + b_{i4} b_{j4}
	\end{array} \right) \]
This is a sextic rational Bezier curve with weights 
\begin{eqnarray}
\label{eq:weights}
w_k & = & \sum_{\begin{array}{c} \mbox{\footnotesize{$0 \leq i \leq 3$}} \\ 
			     \mbox{\footnotesize{$0 \leq j \leq 3$}} \\ 
			     \mbox{\footnotesize{$i+j=k$}}
			     \end{array}}
        \frac{\choice{3}{i} \choice{3}{j}}{\choice{6}{k}}
	\ (b_{i1} b_{j1} + b_{i2} b_{j2} + b_{i3} b_{j3} + b_{i4} b_{j4}) \nonumber \\
& = & \sum_{\begin{array}{c} \mbox{\footnotesize{$0 \leq i \leq 3$}} \\ 
			     \mbox{\footnotesize{$0 \leq j \leq 3$}} \\ 
			     \mbox{\footnotesize{$i+j=k$}}
			     \end{array}}
        \frac{\choice{3}{i} \choice{3}{j}}{\choice{6}{k}}
	\ b_i \cdot b_j
\end{eqnarray}
and control points 
\begin{eqnarray}
\label{eq:control-pts}
\breve{b}_k & = & \frac{1}{w_k} 
      \sum_{\begin{array}{c} \mbox{\footnotesize{$0 \leq i \leq 3$}} \\ 
			     \mbox{\footnotesize{$0 \leq j \leq 3$}} \\ 
			     \mbox{\footnotesize{$i+j=k$}}
			     \end{array}} 
        \frac{\choice{3}{i} \choice{3}{j}}{\choice{6}{k}}
	\left( \begin{array}{c}
            b_{i1} b_{j1} + b_{i2} b_{j2} + b_{i3} b_{j3} - b_{i4} b_{j4} \\
            2b_{i1} b_{j4} \\
            2b_{i2} b_{j4} \\
            2b_{i3} b_{j4} 
	\end{array} \right) \nonumber \\
& = & \frac{1}{w_k} 
      \sum_{\begin{array}{c} \mbox{\footnotesize{$0 \leq i \leq 3$}} \\ 
			     \mbox{\footnotesize{$0 \leq j \leq 3$}} \\ 
			     \mbox{\footnotesize{$i+j=k$}}
			     \end{array}} 
        \frac{\choice{3}{i} \choice{3}{j}}{\choice{6}{k}}
{\cal M}(b_i, b_j)
\end{eqnarray}
for $k = 0, \ldots, 6$.

By evaluating (\ref{eq:weights}), $w_0$ through $w_3$ can be simplified to
\begin{eqnarray}
(w_0,w_1,w_2,w_3) & = & 
% (b_0 \cdot b_0, 
% \frac{3b_0 \cdot b_1 + 3b_1 \cdot b_0}{6}, 
% \frac{3b_0 \cdot b_2 + 9b_1 \cdot b_1 + 3b_2 \cdot b_0}{15}, 
% \frac{b_0 \cdot b_3 + 9b_1 \cdot b_2 + 9b_2 \cdot b_1 + b3 \cdot b_0}{20}) \nonumber \\
% & = &
(b_0 \cdot b_0, b_0 \cdot b_1, \frac{2b_0 \cdot b_2 + 3b_1 \cdot b_1}{5}, 
\frac{b_0 \cdot b_3 + 9b_1 \cdot b_2}{10}) \nonumber 
\end{eqnarray}

Similarly, by evaluating (\ref{eq:control-pts}), $\breve{b}_0$ through $\breve{b}_3$ 
can be simplified to 
\begin{eqnarray}
\breve{b}_0 & = & % \frac{{\cal M}(b_0,b_0)}{w_0} =
                    \frac{{\cal M}(b_0,b_0)}{b_0 \cdot b_0} \nonumber \\
\breve{b}_1 & = & % \frac{{\cal M}(b_0,b_1) + {\cal M}(b_1,b_0)}{2w_1} =
                    \frac{{\cal M}(b_0,b_1) + {\cal M}(b_1,b_0)}{2b_0\cdot b_1} \nonumber \\
\breve{b}_2 & = & % \frac{{\cal M}(b_0,b_2) + 3{\cal M}(b_1,b_1) + {\cal M}(b_2,b_0)}{5w_2} = 
                    \frac{{\cal M}(b_0,b_2) + 3{\cal M}(b_1,b_1) + {\cal M}(b_2,b_0)}
	                 {2b_0 \cdot b_2 + 3b_1 \cdot b_1} \nonumber \\
\breve{b}_3 & = & % \frac{{\cal M}(b_0,b_3) + 9{\cal M}(b_1,b_2) + 9{\cal M}(b_2,b_1) + {\cal M}(b_3,b_0)}{20w_3} =
                    \frac{{\cal M}(b_0,b_3) + 9{\cal M}(b_1,b_2) + 9{\cal M}(b_2,b_1) + {\cal M}(b_3,b_0)}{2b_0 \cdot b_3 + 18b_1 \cdot b_2} \nonumber
\end{eqnarray}

$\breve{b}_4$ through $\breve{b}_6$ and $w_4$ through $w_6$ can then be determined 
through the symmetry of Bezier curves: 
since the geometry of the cubic Bezier curve does not change 
under reversal of the control points, neither does the algebra of its image.
The leverage of symmetry also generates more efficient formulae.\footnote{For example,
  $w_6$ and $\breve{b}_6$ are both expressed originally as sums of 7 terms.
  Using symmetry, they are now expressed by one term.}
\begin{eqnarray}
w_4 & = & \frac{2b_3 \cdot b_1 + 3b_2 \cdot b_2}{5} \nonumber \\
w_5 & = & b_3 \cdot b_2 \nonumber \\
w_6 & = & b_3 \cdot b_3 \nonumber \\
\breve{b}_4 & = & \frac{{\cal M}(b_3,b_1) + 3{\cal M}(b_2,b_2) + {\cal M}(b_1,b_3)}
	                 {2b_3 \cdot b_1 + 3b_2 \cdot b_2} \nonumber \\
\breve{b}_5 & = & \frac{{\cal M}(b_3,b_2) + {\cal M}(b_2,b_3)}{2b_3\cdot b_2} \nonumber \\
\breve{b}_6 & = & \frac{{\cal M}(b_3,b_3)}{b_3 \cdot b_3} \nonumber
\end{eqnarray}
\QED

\subsection{Proof of Theorem~\ref{thm:imagesp}}

\prf
Let $c(t) = \sum_{i=0}^3 b_i B_i^3(t)$ be the cubic Bezier curve.
Since the image of $c(t)$ is a rational Bezier curve,
it is easier to work in projective space.
% When $c(t)$ is expressed in projective space, it becomes
% \[
% \sum_{i=0}^3 (b_{i1},b_{i2},b_{i3},b_{i4}, 1) B_i^3(t)
% \]
In projective space, inverse perspective projection becomes
\[
f^{-1}_{SP}(x_1,x_2,x_3,0,1) = (2x_1, 2x_2, 2x_3,\ x_1^2 + x_2^2 + x_3^2 - 1, 
                                \ x_1^2 + x_2^2 + x_3^2 + 1)
\]
The fourth coordinate of the image curve may be expanded using the product rule
for Bernstein polynomials:
\begin{eqnarray*}
& [\sum_{i=0}^3 B_i^3(t) b_{i1}]^2 + 
[\sum_{i=0}^3 B_i^3(t) b_{i2}]^2 + 
[\sum_{i=0}^3 B_i^3(t) b_{i3}]^2 - 1 \\
= & \sum_{i=0}^3 \sum_{j=0}^3 
	\frac{\choice{3}{i} \choice{3}{j}}{\choice{6}{i+j}}
       B^6_{i+j}(t) (b_{i1} b_{j1} + b_{i2} b_{j2} + b_{i3} b_{j3} - 1) 
\end{eqnarray*}
Letting $k=i+j$, this becomes
\[ \sum_{k=0}^6 B_k^6(t) 
	\sum_{\begin{array}{c}  \mbox{\footnotesize{$0 \leq i \leq 3$}} \\ 
	                        \mbox{\footnotesize{$0 \leq j \leq 3$}} \\ 
			        \mbox{\footnotesize{$i+j=k$}}
			     \end{array}} 
	\frac{\scriptchoice{3}{i} \scriptchoice{3}{j}}{\scriptchoice{6}{k}}
	(b_{i1} b_{j1} + b_{i2} b_{j2} + b_{i3} b_{j3} - 1) 
\]
The fifth coordinate may be expanded analogously, yielding the weights
\[
	\sum_{\begin{array}{c}  \mbox{\footnotesize{$0 \leq i \leq 3$}} \\ 
	                        \mbox{\footnotesize{$0 \leq j \leq 3$}} \\ 
			        \mbox{\footnotesize{$i+j=k$}}
			     \end{array}} 
	\frac{\scriptchoice{3}{i} \scriptchoice{3}{j}}{\scriptchoice{6}{k}}
	(b_{i1} b_{j1} + b_{i2} b_{j2} + b_{i3} b_{j3} + 1) 
\]
The first, second, and third coordinates must be degree elevated,
from degree 3 to 6.
START HERE BY ADDRESSING DEGREE ELEVATION.
\QED

% image under stereographic projection
\Comment{
\prf
Let $c(t) = \sum_{i=0}^3 b_i B_i^3(t)$ be the cubic Bezier curve.
Since the image of $c(t)$ is a rational Bezier curve,
it is easier to work in projective space.
% When $c(t)$ is expressed in projective space, it becomes
% \[
% \sum_{i=0}^3 (b_{i1},b_{i2},b_{i3},b_{i4}, 1) B_i^3(t)
% \]
Perspective projection becomes
\[
f_{SP}(x_1,x_2,x_3,x_4,1) = (x_1,x_2,x_3,0,1-x_4)
\]
% Let the image curve in projective space be $f_{SP}(c(t)) = (f_1(t),\ldots,f_5(t))$.
Note that 
\[
1 - \sum_{i=0}^3 b_{i4} B_i^3(t) = \sum_{i=0}^3 B_i^3(t) - \sum_{i=0}^3 b_{i4} B_i^3(t)
= \sum_{i=0}^3 (1-b_{i4})B_i^3(t)
\]
using the partition of unity property of Bernstein polynomials.
Therefore, 
\[
f_{SP}(c(t)) = \sum_{i=0}^3 (b_{i1}, b_{i2}, b_{i3}, 0, 1-b_{i4}) B_i^3(t)
\]
This is a cubic rational Bezier curve with weights $w_i = 1-b_{i4}$
and control points $\frac{1}{w_i} (b_{i1}, b_{i2}, b_{i3}, 0)$.
\QED
}

\begin{figure}[ht]
\begin{center}
% \includegraphics[scale=.25]{img/fig:stereo.jpg}
\end{center}
\caption{Stereographic projection in 3-space}
\label{fig:stereo}
\end{figure}

\begin{figure}[h]
\vspace{3in}
% \special{psfile=/usr/people/jj/modelTR/3-spline/img/unstable.ps hoffset=80}
\caption{Unstable behaviour near the pole (cusp is directly at pole) and its correction}
\label{fig:wild}
% s3spline -m 90 -p < data3-cusp &
% s3spline -m 90    < data3-cusp &
% tops unstable.rgb -m 8.5 2 > unstable.ps
\end{figure}

\subsection{Proof of Lemma~\ref{lem:cusp}}

\prf
$M$ may be expressed as the composition of two maps (in affine space):
\[
M_1: (p,q,r,s) \rightarrow (p^2 + q^2 + r^2 -s^2 , 2ps,2qs,2rs)
\]
and
\[
M_2: V \rightarrow V/\|V\|
\]
The problem now reduces to determining when these two maps introduce cusps.
We will show that if $M_1(C(t_0))$ is a cusp, then
$C(t_0)$ or $C'(t_0)$ is the origin or $C(t_0) \cdot C'(t_0) = 0$;
while $M_2(C(t_0))$ is a cusp when $C'(t_0) = kC(t_0),\ k \in \Re$.

Consider the map $M_1$.
A curve $C(t)$ has a cusp at $t=t_0$ if $\ddt(C(t_0)) = (0,0,0,0)$.
% \cite{farin93}.
Let $C(t) = (p(t),q(t),r(t),s(t))$.
Suppose that $M_1(C(t_0))$ is a cusp.
\begin{eqnarray}
\label{eq:d1}
\ddt(M_1(p,q,r,s)) & = & 
\mbox{\footnotesize{$\left( \begin{array}{c}
	2pp' + 2qq' + 2rr'  - 2ss' \\
	2p's + 2ps' \\
	2q's + 2qs' \\
        2r's + 2rs' 
\end{array} \right)$}} \nonumber \\
& = & (0,0,0,0)
\end{eqnarray}
If $s \neq 0$, this yields
$       p' = -p(s'/s)$,
$       q' = -q(s'/s)$,
$       r' = -r(s'/s)$,
and substituting into $2pp' + 2qq' + 2rr'  - 2ss' = 0$ yields
$        \frac{-2s'}{s}(p^2+q^2+r^2+s^2) = 0 $.
If $s'\neq 0$, this reduces to
$ p^2 + q^2 + r^2 + s^2 = 0$
or $(p,q,r,s)$ is the origin.

If $s = 0$ and $s' \neq 0$,
(\ref{eq:d1}) again reduces to $(p,q,r,s) = (0,0,0,0)$.
If $s = s' = 0$,
(\ref{eq:d1}) reduces to $pp' + qq' + rr' = 0$ 
(or $(p,q,r,s) \cdot (p',q',r',s') = 0$).
If $s \neq 0$ and $s' = 0$,
(\ref{eq:d1}) reduces to $(p',q',r',s') = (0,0,0,0)$.

Next consider $M_2$.
Suppose that $M_2(C(t_0))$ is a cusp.
We assume that $C(t_0) \neq (0,0,0,0)$.
\[ 
\ddt(M_2(C(t_0))) 
= \ddt(\frac{C(t_0)}{\|C(t_0)\|})
= \ddt(\frac{C(t_0)}{\sqrt{C(t_0) \cdot C(t_0)}})
\]
\[
= \frac{\|C\| C' - (\frac{C \cdot C'}{\sqrt{C \cdot C}}) C}{C \cdot C}
= (0,0,0,0)
\]
Multiplying by $\|C(t_0)\|^3$,
\[
\|C\|^2 C' - (C \cdot C') C = (0,0,0,0)
\]
\[
C' = kC
\]
where $k = \frac{C \cdot C'}{C \cdot C}$.
In other words,
the map $M_2$ only introduces cusps into the curve $C(t)$ when
$C'(t_0) = kC(t_0)$.

Note that $M_2$ preserves cusps: that is, if $C(t_0)$ is a cusp,
then $M_2(C(t_0))$ is also a cusp.
% (if $C'(t)=0$ then $C'(t) = kC(t)$).
\QED

\clearpage

\subsection{Bezier derivatives}

NONE OF THIS IS APPROPRIATE FOR THE PAPER. 
HOWEVER, THIS IS GOOD REFERENCE MATERIAL FOR THE IMPLEMENTATION AND PERHAPS 671.

If $b(t)$ is a rational Bezier curve of degree $n$ 
with control points $\{b_i\}_{i=0}^n$ and 
weights $\{w_i\}_{i=0}^n$ defined over $t \in [0,1]$, then 
\begin{eqnarray}
\label{eqn:firstderiv}
b'(0) & = & n \frac{w_1}{w_0} (b_1 - b_0) \\
\label{eqn:secondderiv}
b''(0) & = & \frac{n-1}{n} \frac{w_0 w_2}{w_1^2} 
             \frac{\triangle b_0 \times \triangle b_1}{\| \triangle b_0\|^3}
\end{eqnarray}
where $\triangle b_i = b_{i+1} - b_i$.
See Farin \cite{farin02} and Hoschek/Lasser \cite{hoschekLasser0?}. 
% p. 244 of HL has conditions for curvature continuity
Notice the relationship of second derivative to curvature $\kappa$:
\[
    \kappa = \|b''\|
\]
if $b$ is arc-length parameterized, and in general,
\[
    \kappa = \frac{\| b' \times b'' \|}{\| b' \|^3}
\]
$C^2$ continuity guarantees curvature continuity.
See Farin \cite{farin02} or a differential geometry text.
% the formula for curvature of a rational Bezier curve is 
% k = \frac{n-1}{n} \frac{w_0 w_2}{w_1^2} 
%     \frac{\mbox{dist}(b_0,\lyne{b_1 - b_0}}{\mbox{dist}(b_0,b_1)^2}

\clearpage

{\bf Build a video of the rational quaternion spline technique.}

{\bf Work the software back up to snuff.}

Challenges.
\begin{itemize}
\item emptiest point (using bisectors intersection)
\item true computation of VD on S3
\item how easy is it to find an empty point?
\item line interpolation
\item brief biographical sketch for Ross
\end{itemize}

% from hermiteQ.tex
{\bf Exposure sheet}

Maps to the sphere.
\[
	M(x_1,x_2,x_3,x_4) =
	\frac{1}{x_1^2 + x_2^2 + x_3^2 + x_4^2}
	(x_1^2 + x_2^2 + x_3^2 - x_4^2, 2x_1x_4, 2x_2x_4, 2x_3x_4)
\]
\[
SP^{-1}(x_1,x_2,x_3) = \frac{1}{x_1^2 + x_2^2 + x_3^2 + 1} 
                       (2x_1, 2x_2, 2x_3, x_1^2+x_2^2+x_3^2 - 1)
\]

Maps away from the sphere.
\[
M^{-1}|_{S3}(x_1,x_2,x_3,x_4) = \frac{1}{\sqrt{2(1-x_1)}}(x_2,x_3,x_4,1-x_1)
\]
\[
SP(x_1,x_2,x_3,x_4) = \frac{1}{1-x_4} (x_1,x_2,x_3)
\]

% OLD STEREOGRAPHIC PROJECTION (WHEN WE FIRST INTRODUCED PROJECTION FROM NORTH POLE:
\Comment{

% Traditionally, a point is projected from the north pole $(0,\ldots,0,1)$ to $x_{n+1}=0$.
% We rotate the point of projection to $(1,0,\ldots,0)$ and the projection plane
% to $x_1=0$, for compatibility with the Euler map.
% (They then have the same pole $(1,0,\ldots,0)$.)
\Comment{
   It is a map from \Sn{n}\ to the hyperplane $x_{n+1}=0$,
   in which a point of \Sn{n}\ is perspectively projected from 
   the north pole of \Sn{n}\ to $x_{n+1}=0$  % \cite{thorpe79}. % or kreyszig59
   (Figure~\ref{fig:stereo}).
% Note that there is a variant of stereographic projection where
% the projection plane $z=0$ is replaced by the tangent plane opposite the pole
% $z=-1$ \cite{kreyszig59}.
}

OLD VERSION OF SP, PROJECTING FROM (0,\ldots,0,1):
\begin{lemma}
Stereographic projection 
$f_{SP}: \Sn{n} - (0,\ldots,0,1) \rightarrow x_{n+1}=0 \subset \Re^{n+1}$ 
is defined by:
\begin{eqnarray*}
f_{SP}(x_1,\ldots,x_{n+1}) & = & \frac{1}{1-x_{n+1}} (x_1,\ldots,x_n,0) \\
\end{eqnarray*}	% see thorpe79, p. 125
Stereographic injection $f_{SP}^{-1}: x_{n+1}=0 \rightarrow \Sn{n} - (0,\ldots,0,1)$ 
is defined by:
\begin{eqnarray*}
f_{SP}^{-1}(x_1,\ldots,x_n,0) & = &
	\frac{1}{x_1^2 + \cdots + x_n^2 + 1} 
	(2x_1, \ldots, 2x_n, x_1^2 + \cdots + x_n^2 - 1)
\end{eqnarray*}	% see thorpe79, p. 125
\end{lemma}
\vspace{.1in}
\prf
The projector line $(1-t)p + tq$ through $p = (0,\ldots,0,1)$ and 
$q = (x_1,\ldots,x_{n+1}) \in \Sn{n}$ intersects $x_{n+1}=0$ 
when $1-t + tx_{n+1} = 0$ or $t = \frac{1}{1-x_{n+1}}$.
This defines stereographic projection.
For the inverse map, the projector line $(1-t)p + t(r,0)$ through $p = (0,\ldots,0,1)$ 
and $(r,0)$ intersects \Sn{n} when $\|(1-t)(0,\ldots,0,1) + t(r,0)\|^2 = 1$,
% or $(1-t)^2 + t^2\|r\|_2^2 = 1$,
or $t^2(\|r\|^2 + 1) = 2t$, or 
$t=0,\frac{2}{\|r\|^2 + 1}$.
The second root yields the inverse point.
\QED

\begin{defn2}
\label{defn:si}
{\rm 
{\bf Stereographic injection} is the map 
% $\sigma^{-1}_{(1,0,\ldots,0)}: x_{1}=0 \rightarrow \Sn{n} - (1,0,\ldots,0)$:
% \begin{equation}
% \label{eq:injection}
% (0,x_1,\ldots,x_n) \mapsto
% 	\frac{1}{x_1^2 + \cdots + x_n^2 + 1} 
%	(x_1^2 + \cdots + x_n^2 - 1, 2x_1, \ldots, 2x_n)
% \end{equation}
$\sigma^{-1}: \Re^n \rightarrow \Sn{n} - (1,0,\ldots,0)$:
\begin{equation}
\label{eq:si}
(x_1,\ldots,x_n) \mapsto
	\frac{1}{x_1^2 + \cdots + x_n^2 + 1} 
	(x_1^2 + \cdots + x_n^2 - 1, 2x_1, \ldots, 2x_n)
\end{equation}
}
\end{defn2}

% By rotation, we can easily generalize to a rational map from $\Re^n$
% to $\Sn{n} - \{q\}$, where $q$ is an arbitrary point of \Sn{n}
	% by choosing a different center of
	% projection on \Sn{n}, say $q$, and projecting onto the associated hyperplane:
	% the hyperplane through the origin and parallel to $q$'s tangent plane.
	% This is simply a rotation of the conventional stereographic projection map.
% whose choice dictates the hyperplane embedding of $\Re^n$ in $\Re^{n+1}$.

\bibitem[Thorpe 79]{thorpe79}
Thorpe, J. (1979)
Elementary Topics in Differential Geometry.
Springer-Verlag (New York).
}

% --------------------------------------------------------------------------------

{\bf Unnecessary additional information about stereographic projection.}

Some other properties of the stereographic projection are interesting, although
more relevant to cartography than to our study.
It is a conformal map and, like any conformal map of \Sn{2}\ to the plane,
it maps loxodromes (curves of constant direction on the sphere) to straight lines.
It is the only conformal map from $S^2$ to a plane that preserves
circles (circles are mapped to circles or straight lines).

The standard rational parameterization of \Sn{n}\ 
can be developed from stereographic injection.
Since stereographic injection is a one-to-one, 
(almost) onto map from an $n$-dimensional
hyperplane to the $n$-dimensional sphere, 
it induces a parameterization of the sphere.
For example, in the case of \Sn{3}, this parameterization is:
\begin{equation}
	S(t_1,t_2,t_3) = 
	\frac{1}{t_1^2 + t_2^2 + t_3^2 + 1} 
	(2t_1, 2t_2, 2t_3, t_1^2 + t_2^2 + t_3^2 - 1) \ \ \ \ t_i \in (-\infty,\infty)
\end{equation}
% For the circle \Sn{1}, this parameterization is a variant of the
% classical $(\frac{1-t^2}{1+t^2}, \frac{2t}{1+t^2})$.

% --------------------------------------------------------------------------------

\Comment{
{\bf The discussion of embedded stereographic injection is a moot point now,
since we have shown that stereographic injection is exactly a case of 
Theorem~\ref{thm:ratmap}.}

Stereographic injection is very similar to the rational map of $\Re^{n+1}$ to
\Sn{n}\ in Theorem~\ref{thm:ratmap} generated by the choice 
$(a_1,\ldots,a_{n+1}) = (x_1,\ldots,x_{n},1)$ and the identity permutation:
%
\begin{equation}
\label{eq:rat1}
(x_1,\ldots,x_{n+1}) \mapsto 
	\frac{1}{x_1^2 + \cdots + x_{n}^2 + 1}
	(x_1^2 + \cdots + x_{n}^2 - 1,
	 2x_1, \ldots, 2x_{n})
\end{equation}
%
Since every hyperplane $x_{n+1}=k$ of (\ref{eq:rat1}) gets mapped like 
(\ref{eq:si}), this is a many-to-one map with many embedded copies of
stereographic injection.
TECH REPORT ONLY: (\ref{eq:rat1}) will be called {\bf embedded stereographic injection}.

\subsection{A comparison}

We now compare the Euler map $M$ with stereographic injection.
$M$ works over a larger domain than stereographic injection ($\Re^{n+1}$
rather than $\Re^n$).
Therefore, the most we can expect is that $M$ contains some embedding of
stereographic injection.
It does: the restriction of $M$ to the hyperplane $x_{n+1}=1$
is equivalent to stereographic injection.
However, $x_{n+1}=1$ is the only hyperplane restriction that is equivalent to 
stereographic injection.
For example, the restriction of $M$ to the hyperplane $x_{n+1}=5$ is
\[
	(x_{1},\ldots,x_n) \rightarrow
	\frac{1}{x_1^2 + \cdots + x_n^2 + 25}
	(x_1^2 + \cdots + x_{n}^2 - 25, 10x_1, \ldots, 10x_n).
\]
Thus, $M$ can be characterized as an extension of stereographic injection:
its restriction to the hyperplane $x_{n+1}=1$ is equivalent
	% the hyperplane restriction $M_{x_{n+1}=1}$ is equivalent to $\sigma^{-1}$
while its behaviour on other hyperplanes of $\Re^{n+1}$ is different
from $\sigma^{-1}$.

TECH REPORT ONLY: 
A comparison of $M$ and stereographic injection in projective space initially 
seems to reveal a deeper relationship between the two maps,
but this is misleading.
Translating $M$ to projective space (with projective coordinate $x_{n+2}$), 
we have
% where little changes since the terms are already homogeneous of degree 2:
\begin{equation}
\label{eq:projM}
(x_1,\ldots,x_{n+1},x_{n+2}) \mapsto 
	(x_1^2 + \cdots + x_{n}^2 - x_{n+1}^2,
	 2x_1x_{n+1}, \ldots, 2x_nx_{n+1}, x_1^2 + \cdots + x_{n+1}^2)
\end{equation}
Similarly, translating stereographic injection to projective space 
(with projective coordinate $x_{n+1}$), we have
\begin{equation}
\label{eq:projsi}
	(x_1,\ldots,x_{n+1}) \mapsto
	(x_1^2 + \cdots + x_n^2 - x_{n+1}^2,\ 2x_{1}x_{n+1}, \ldots,\ 2x_{n}x_{n+1},\ 
	x_1^2 + \cdots + x_{n+1}^2)
\end{equation}
	% These look virtually identical!
SKIP DISCUSSION OF EMBEDDED SI:
These seem to have the same relationship as embedded stereographic
injection and stereographic injection,
as every hyperplane $x_{n+2}=k$ of (\ref{eq:projM}) gets mapped 
like stereographic injection (\ref{eq:projsi}).
However, we are now dealing with the very different case of 
hyperplanes in the projective coordinate $x_{n+2}$.
Even more importantly, $x_{n+1}$ is a typical % affine?
coordinate in (\ref{eq:projM}) but the added projective coordinate in (\ref{eq:projsi}).
The interchange of projective and other coordinates is a significant
change to a map. %, unlike the minor effect of interchanging two affine coordinates.
For example, the two points $(0,0,1)$ and $(0,1,0)$ in projective 2-space
are identical except for an interchange of coordinates,
yet $(0,0,1)$ is the origin while $(0,1,0)$ is a point at infinity
along the $y$-axis.
Thus, $M$ is indeed significantly different from stereographic injection.

We conclude that the Euler map to the sphere is a powerful extension of
the inverse map of stereographic projection.

[Be careful using stereographic projection, since the conclusions of comp.tex are that
stereographic projection leads to an inferior quaternion spline.
Doesn't this imply that we have implemented the map of a Bezier curve under
stereographic projection too?]

Can also check in hermiteQ.pdf.

LATER:
There are arguments for both maps, the Euler map and the inverse of stereographic
projection, but the acid test is their use to build a quaternion spline.
We will consider both options in this paper.
Other maps can be used as well, using a development modeled after the construction
for the two maps.

\begin{itemize}
\item
We prefer a map with domain $\Re^4$, so that the data points will be
mapped to one-dimensional curves in Euclidean space and the 
curve design in Euclidean space will enjoy more flexibility.
\item
A map between $S^3$ and $\Re^3$ is more appropriate than a map between $S^3$ and $\Re^4$
for the inverse map.
The inverse will be unique, while in the latter case the map from $S^3$ to $\Re^4$ is
inherently one-to-many, since we are moving from a 3-manifold to a 4-manifold.
\end{itemize}

}

\end{document}
