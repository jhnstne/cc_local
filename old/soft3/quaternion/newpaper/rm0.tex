\documentclass[12pt]{article}
\usepackage{times}
\usepackage[pdftex]{graphicx}
\usepackage{amssymb}
\makeatletter
\def\@maketitle{\newpage
 \null
 \vskip 2em                   % Vertical space above title.
 \begin{center}
       {\Large\bf \@title \par}  % Title set in \Large size. 
       \vskip .5em               % Vertical space after title.
       {\lineskip .5em           %  each author set in a tabular environment
        \begin{tabular}[t]{c}\@author 
        \end{tabular}\par}                   
  \end{center}
 \par
 \vskip .5em}                 % Vertical space after author
\makeatother

% default values are 
% \parskip=0pt plus1pt
% \parindent=20pt

\newcommand{\SingleSpace}{\edef\baselinestretch{0.9}\Large\normalsize}
\newcommand{\DoubleSpace}{\edef\baselinestretch{1.4}\Large\normalsize}
\newcommand{\Comment}[1]{\relax}  % makes a "comment" (not expanded)
\newcommand{\Heading}[1]{\par\noindent{\bf#1}\nobreak}
\newcommand{\Tail}[1]{\nobreak\par\noindent{\bf#1}}
\newcommand{\QED}{\vrule height 1.4ex width 1.0ex depth -.1ex\ \vspace{.3in}} % square box
\newcommand{\arc}[1]{\mbox{$\stackrel{\frown}{#1}$}}
\newcommand{\lyne}[1]{\mbox{$\stackrel{\leftrightarrow}{#1}$}}
\newcommand{\ray}[1]{\mbox{$\vec{#1}$}}          
\newcommand{\seg}[1]{\mbox{$\overline{#1}$}}
\newcommand{\tab}{\hspace*{.2in}}
\newcommand{\se}{\mbox{$_{\epsilon}$}}  % subscript epsilon
\newcommand{\ie}{\mbox{i.e.}}
\newcommand{\eg}{\mbox{e.\ g.\ }}
\newcommand{\figg}[3]{\begin{figure}[htbp]\vspace{#3}\caption{#2}\label{#1}\end{figure}}
\newcommand{\be}{\begin{equation}}
\newcommand{\ee}{\end{equation}}
\newcommand{\prf}{\noindent{{\bf Proof}:\ \ \ }}
\newcommand{\choice}[2]{\mbox{\footnotesize{$\left( \begin{array}{c} #1 \\ #2 \end{array} \right)$}}}      
\newcommand{\scriptchoice}[2]{\mbox{\scriptsize{$\left( \begin{array}{c} #1 \\ #2 \end{array} \right)$}}}
\newcommand{\tinychoice}[2]{\mbox{\tiny{$\left( \begin{array}{c} #1 \\ #2 \end{array} \right)$}}}
\newcommand{\ddt}{\frac{\partial}{\partial t}}
\newcommand{\Sn}[1]{\mbox{{\bf S}$^{#1}$}}
\newcommand{\calP}[1]{\mbox{{\bf {\cal P}}$^{#1}$}}

\newtheorem{theorem}{Theorem}	
\newtheorem{rmk}[theorem]{Remark}
\newtheorem{example}[theorem]{Example}
\newtheorem{conjecture}[theorem]{Conjecture}
\newtheorem{claim}[theorem]{Claim}
\newtheorem{notation}[theorem]{Notation}
\newtheorem{lemma}[theorem]{Lemma}
\newtheorem{corollary}[theorem]{Corollary}
\newtheorem{defn2}[theorem]{Definition}
\newtheorem{observation}[theorem]{Observation}
\newtheorem{implementation}[theorem]{Implementation note}

% \font\timesr10
% \newfont{\timesroman}{timesr10}
% \timesroman


\setlength{\oddsidemargin}{0pt}
\setlength{\topmargin}{0in}
\setlength{\textheight}{8.6in}
\setlength{\textwidth}{6.875in}
\setlength{\columnsep}{5mm}
\markright{\hfill \today \hfill}
\pagestyle{myheadings}

% -----------------------------------------------------------------------------
\title{A characterization of rational maps to the sphere\\[5pt]
       {\small UAB Technical Report}} % \\January-February 2005}}
\author{John K. Johnstone\thanks{Supported in part by the National 
        Science Foundation under grant CCR0203586.}}
        % Johnstone and Williams
\begin{document}
\maketitle

% ask Ian Knowles for a suggestion of a mathematical journal

\begin{abstract}
A characterization of rational maps from Euclidean space 
to the hypersphere \Sn{n-1} in $n$-space
is developed, including a formula that generates all of these maps.
This normal form explains the presence of point singularities in these maps.
A characterization of arbitrary Pythagorean tuples is also developed,
generalizing earlier characterizations of Pythagorean triples and quadruples.
% The definition of a rational quaternion spline reduces to the discovery
% of a rational map to the 3-sphere \Sn{3} \cite{jjjimbo95a}.
\end{abstract}

\noindent
Keywords: sphere, rational map, pole, Pythagorean tuple, 
          Euler's Four Squares Theorem.

\section{Introduction}

A study of maps to the sphere, the subject of this paper,
allows computation in Euclidean space to be translated to the sphere.
This is useful since computation on the sphere is common.
% Much computation is carried out on the sphere,
% throughout mathematics and computer science 
For example, in our research, 
maps to the sphere arise in the design of quaternion splines,
which are modeled on the 3-sphere in 4-space.
(Quaternion splines are used for orientation control in motion design.)
A map to the 3-sphere reduces the problem of curve design on the 3-sphere
to the simpler problem of curve design in Euclidean space.
Basically, the curve is designed in Euclidean space 
and then mapped to the sphere \cite{rq1}.
%
In general, a better understanding of the sphere, including maps to the sphere, 
is beneficial because of the sphere's centrality to mathematics.

In this paper, we concentrate on rational maps to the sphere.
We develop a characterization of rational maps from Euclidean space
to the hypersphere in arbitrary dimensions, $f : \mathbb{R}^m \rightarrow \Sn{n}$.
Rational maps to the hypersphere are well understood in 2-space and 3-space,
but they are not well understood in 4-space or in higher dimensions.
The restriction to rationality defines the main challenge.
It is motivated by the many advantages of rational over nonrational constructs,
such as efficiency, conciseness, representational ability, and 
compatibility with rational numerical techniques.
For example, in our work, a rational map allows the design of rational Bezier
quaternion splines.

Our tool for building a normal form for rational maps to the sphere
is the Pythagorean tuple, and along the way we develop a complete characterization
of the Pythagorean tuple as well, extending previous characterizations of
Pythagorean triples and quadruples.
%
The development of a normal form for rational maps to the sphere also
explains the necessary presence of point singularities (or poles) in these maps,
which is a fundamental issue in working with them.
% One of the most fundamental [burning] questions is: why do rational maps to the
% sphere always have poles?
 
The rest of the paper unfolds as follows.
% The rest of the paper is structured as follows.
Section~\ref{sec:prev} reviews previous work on maps to the sphere.
Section~\ref{sec:ratpy} establishes the equivalence of rational maps 
to the sphere and Pythagorean tuples.
Section~\ref{sec:prevpyth} reviews previous work on Pythagorean triples 
and quadruples, and introduces Euler's Four Squares Theorem, a tool
for the construction of more general tuples.
Section~\ref{sec:normpyth} establishes a normal form for general Pythagorean 
tuples, including a better characterization of Pythagorean quadruples.
Section~\ref{sec:char} provides the advertised normal form for rational maps 
to the sphere.
It also applies the normal form to the construction of
important rational maps, some old and some new.  
% It also shows how to generate known Pythagorean tuples from the normal form.
Section~\ref{sec:pole} discusses the inevitability of poles
and Section~\ref{sec:conclude} ends the paper with conclusions.
We end this introduction with two formal definitions,
of the sphere and the rational map.

% in other papers:
% an analysis of the inverse maps % from the sphere to Euclidean space 
% (which are typically not rational), 
% [We could include the image of a rational Bezier curve under these maps, 
% if more is needed.  But I think that this is more natural in the paper on quaternion 
% splines, which also addresses avoidance of the poles, divide and conquer, and hence
% derivatives. Delay discussion of stereographic projection to this second paper too,
% since only its inverse is a map to the sphere and we heavily use s.p. in building'
% rational quaternion splines]
%

\begin{defn2}
\Sn{n}\ is the unit sphere in $(n+1)$-space $x_1^2 + \ldots + x_{n+1}^2 - 1 = 0$,
a manifold of dimension $n$.
\end{defn2}

\begin{defn2}
A map $(x_1,\ldots,x_n) \mapsto (f_1 (x_1,\ldots,x_n),\ldots,f_m (x_1,\ldots,x_n))$
is {\bf rational} if 
the $f_i$ are all rational polynomials in $x_1,\ldots,x_n$, $i=1,\ldots,m$.
A {\bf rational polynomial} is a quotient of polynomials.
\end{defn2}

% -----------------------------------------------------------------------------

\section{Previous work on maps to the sphere}
\label{sec:prev}

Cartographers have long wrestled with the problem of mapping the spherical earth
to a flat page.
As a result, there are many maps from the 2-sphere to the plane in the literature,
whose inverses are maps from a Euclidean space to the 2-sphere.
The many famous examples, such as the Mercator projection, the mapping of Sanson,
the mapping of Bonne, and the Lambert projection \cite{kreyszig59},
illustrate the challenge of developing rational maps.
None of these classical maps are rational since, in the design 
of cartographic maps, 
properties such as conformality (preservation of angles between 
intersecting curves) and equiareality (preservation of area) are more important 
than rationality.
% (the ability to express the map in terms of quotients of polynomials).
For example, the Mercator projection maps the point
$(\cos u \cos v,\ \cos u \sin v,\ \sin u)$ on the 2-sphere 
to the point $(v,\ \log \tan (\frac{u}{2} + \frac{\pi}{4}))$ on the plane.

A rare rational map that proves the exception to these famous nonrational maps
is the even more classical map of stereographic projection
(used at least since Hipparch in 160 B.C.). %  \cite{kreyszig59}
This is a rational map from the sphere to the plane
whose inverse is also rational. 
Stereographic projection is the map from 
$\Sn{n} - (0,\ldots,0,1)$ to $x_{n+1}=0$ defined by:
\[ (x_1,\ldots,x_{n+1}) \mapsto \frac{1}{1-x_{n+1}} (x_1,\ldots,x_n,0) \]
% The projector line $tp + (1-t)q$ through $p = (x_1,\ldots,x_{n+1})$ 
% and $q = (0,\ldots,0,1)$ intersects $x_{n+1}=0$ 
% when $tx_{n+1} + (1-t) = 0$ or $t = \frac{1}{1-x_{n+1}}$.
in which a point of \Sn{n}\ is perspectively projected from 
the north pole of \Sn{n}\ to $x_{n+1}=0$.  % \cite{thorpe79}. % or kreyszig59
Its inverse from
$x_{n+1}=0$ to $\Sn{n} - (0,\ldots,0,1)$ is defined by:
\begin{equation}
\label{eq:spinverse}
(x_1,\ldots,x_n,0) \mapsto
	\frac{1}{x_1^2 + \cdots + x_n^2 + 1} 
	(2x_1, \ldots, 2x_n, x_1^2 + \cdots + x_n^2 - 1).
\end{equation}
% see thorpe79, p. 125
(To establish this result, note that the projector line through $(r,0)$ 
and $q = (0,\ldots,0,1)$, $t(r,0) + (1-t)q$
intersects \Sn{n} at $t=0$ and $t=\frac{2}{\|r\|^2 + 1}$.)

Another source of maps to the sphere is parameterizations of the sphere.
After all, a parameterization of an $n$-manifold is a map of $n$-space
to the manifold.
The classical rational parameterization of \Sn{2}, constructed using the
spherical product of a semi-circle and a circle \cite{barr81},
% $(a(u),b(u)) \times (c_1(v_1,...,v_{n-1}),\ldots,c_n(v_1,\ldots,v_{n-1}) 
% = (a c_1, \ldots, a c_{n-1}, b c_n)$.
% The spherical product is a general scheme for sweeping one curve along another,
% while modulating its size.
is the following rational map of 2-space to the sphere:
\begin{equation}
\label{eq:s2param}
S(u,v) = (\frac{1-u^2}{1+u^2}(\frac{1-v^2}{1+v^2}),\  
          \frac{1-u^2}{1+u^2}(\frac{2v}{1+v^2}),\  \frac{2u}{1+u^2}),\ 
u \in [-1,1],\ v \in (-\infty,+\infty)
\end{equation}
The restriction of $u$ to $[-1,1]$
can be removed without problem, thereby making this a true map of all of Euclidean
space to the sphere: the map then double covers the sphere.
This paramerization of the sphere can be generalized to arbitrary dimensions, using
repeated spherical products.
% Rational parameterizations are of interest for the development of rational Bezier or
% NURBS representations of the sphere.

% -----------------------------------------------------------------------------

\section{A Reduction to Pythagorean Tuples}
\label{sec:ratpy}

A key observation in the characterization of rational maps to the sphere is their
equivalence to Pythagorean tuples.
A Pythagorean tuple satisfies a sum of squares relationship,
generalizing the sum of squares relationship between the sides of a right triangle
in the Pythagorean Theorem.

\begin{defn2}
A {\bf Pythagorean $n+1$-tuple over $K$} % n+1-tuple is clear from context
is a tuple $(f_1,\ldots,f_{n+1}) \in K^{n+1}$
where $f_1^2 + \ldots + f_n^2 = f_{n+1}^2$.
\end{defn2}

% n --> m
Consider a rational map to the sphere $f: \mathbb{R}^m \rightarrow \Sn{n-1}$:
\[
	f(x_1,\ldots,x_m) =
	(\frac{f_1(x_1,\ldots,x_m)}{f_{n+1}(x_1,\ldots,x_m)}, \ldots,
	 \frac{f_n(x_1,\ldots,x_m)}{f_{n+1}(x_1,\ldots,x_m)})
\]
where $f_1,\ldots,f_{n+1}$ are polynomials.
Since $f$ maps to the unit sphere at the origin:
\begin{eqnarray}
       (\frac{f_1}{f_{n+1}})^2 + \cdots + (\frac{f_n}{f_{n+1}})^2 & = & 1 \\
	f_1^2 + \cdots + f_n^2 & = & f_{n+1}^2.
\end{eqnarray}
That is, the components of the rational map define a Pythagorean $(n+1)$-tuple
over the ring of polynomials $\mathbb{R}[x_1,\ldots,x_m]$.
Therefore, rational maps to \Sn{n-1} can be identified with Pythagorean $(n+1)$-tuples.

\begin{lemma}
\label{lem:ratpy}
The rational map $f: \mathbb{R}^m \rightarrow \mathbb{R}^n$ defined by 
$f = (\frac{f_1}{f_{n+1}},\ldots,\frac{f_n}{f_{n+1}})$ (where $f_1,\ldots,f_{n+1}$
are polynomials) is a rational map 
to the sphere \Sn{n-1} if and only if
the $n+1$-tuple $(f_1,\ldots,f_{n+1})$ of polynomials is Pythagorean.
\end{lemma}

\Comment{
\begin{defn2}
$\mathbb{R}[x_1,\ldots,x_n]$ is the ring of polynomials in the $n$ variables
$x_1,\ldots,x_n$ with real coefficients.
\end{defn2}
}

% This reduces the study of rational maps from $\mathbb{R}^{n}$ to \Sn{m-1}\ 
% to the study of Pythagorean $(m+1)$-tuples over polynomials $\mathbb{R}[x_1,\ldots,x_n]$.

% \begin{lemma}
% \label{lem:iffpyth}
% A rational map $f: \mathbb{R}^n \rightarrow \mathbb{R}^m$ defined by 
% $f(x_1,\ldots,x_n) = (\frac{f_1(x_1,\ldots,x_n)}{f_{m+1}(x_1,\ldots,x_n)}, \ldots,
%	 \frac{f_m(x_1,\ldots,x_n)}{f_{m+1}(x_1,\ldots,x_n)})$,
% where $f_1,\ldots,f_{m+1} \in \mathbb{R}[x_1,\ldots,x_n]$, is a rational map from $\mathbb{R}^n$
% to the sphere \Sn{m-1}\ if and only if $(f_1,\ldots,f_{m+1})$ is a Pythagorean
% $(m+1)$-tuple over polynomials $\mathbb{R}[x_1,\ldots,x_n]$.
% \end{lemma}

% notice that m+1-tuples map to m-1-sphere

Pythagorean tuples and sums of squares have been heavily studied in number theory.
An examination of this literature will lead us to a characterization of rational maps
to the sphere.
We begin with a review of Pythagorean triples and Pythagorean quadruples,
which are well known.
We then consider Euler's fundamental work on the sum of four squares,
which can be related to Pythagorean quintuples.
This leads to a characterization of Pythagorean quintuples, then general 
Pythagorean tuples, and finally to rational maps to \Sn{3}
% Although this work does not yield a full characterization of Pythagorean quintuples,
% it is sufficient to characterize rational maps to \Sn{3}.
and rational maps to \Sn{n}.
% by generalizing the technique for Pythagorean quintuples, 
% using Ammei's generalization of Euler's result.
Special attention is given to Pythagorean quintuples and the associated
maps to \Sn{3}, since they define the first step of generalization past
the known territory of Pythagorean quadruples.
(They also have particular import for our research with quaternion splines
on the 3-sphere.)

% Since there is flexibility in which Euclidean space is involved, we consider a general
% Euclidean space.

% -----------------------------------------------------------------------------

\section{Previous Work on Pythagorean Tuples}
\label{sec:prevpyth}

\subsection{Pythagorean triples}

It is a classical result for $K = \mathbb{Z}$ (the field of integers),
generalized to $K = \mathbb{P}$ (the ring of polynomials) 
by Kubota \cite{kubota72},
that a triple over K is Pythagorean if and only if
it is of the form 
\[
\alpha(a_1^2 - a_2^2, 2a_1a_2, a_1^2 + a_2^2)
\]
for some $a_1,a_2,\alpha \in K$.
  % p. 89, Ebbinghaus, Numbers
(Whenever we give a normal form for a Pythagorean $(n+1)$-tuple,
it is understood that the first $n$ elements are free to undergo
a permutation.)
Farouki and Sakkalis \cite{farouki90} use
Kubota's result to define Pythagorean hodograph curves
with attractive arc length and offset properties,
whose hodographs are components of a Pythagorean triple
of polynomials.
  % which among other things have polynomial arc length (in the curve parameter).
To illustrate the association between Pythagorean triples and rational maps to \Sn{1},
note that the standard rational parameterization for \Sn{1} 
may be developed from the normal form for Pythagorean triples 
using $(a_1,a_2,\alpha) = (1,t,1)$:
$(\frac{1-t^2}{1+t^2}, \frac{2t}{1+t^2})$.

\subsection{Pythagorean quadruples}

A quadruple over $K$ is Pythagorean if and only if it is of the form
\begin{equation}
\label{eq:quad}
\alpha (a_1^2+a_2^2-a_3^2-a_4^2,\ 2a_1a_3+2a_2a_4,\ 2a_1a_4-2a_2a_3,
\ a_1^2+a_2^2+a_3^2+a_4^2)
\end{equation}
for some $a_1,a_2,a_3,a_4,\alpha \in K$.
This normal form for Pythagorean quadruples was established 
for $K = \mathbb{Z}$ by Catalan \cite{catalan85}.
It was generalized to $K = \mathbb{R}[x_1]$, 
$\mathbb{R}[x_1,x_2]$ and 
$\mathbb{R}[x_1,x_2,x_3]/\hspace{-.05in}<\hspace{-.05in}
x_1+x_2+x_3-1
\hspace{-.05in}>$ 
by Dietz et. al. \cite{dietz93}, who used the rational map to \Sn{2}\ 
resulting from the choice $(a_1,a_2,a_3,a_4) = (x_1,x_2,x_3,1)$.
% (with a permutation (1,2,3) to (3,2,1)) 
The standard rational parameterization (\ref{eq:s2param}) of \Sn{2} mentioned
in Section~\ref{sec:prev}
may be developed from the normal form (\ref{eq:quad})
using $(a_1,a_2,a_3,a_4,\alpha) = (1,uv,u,v,1)$.
% and a permutation.
% $\frac{1}{1+u^2+v^2+u^2v^2}(1-u^2-v^2+u^2v^2, 2v - 2u^2v,2u+2uv^2)$.

\subsection{The sum of four squares}

There is an extensive study of the sum of four squares in the 
number theory literature, which can be related to Pythagorean quintuples
since these are of the form $f_1^2 + f_2^2 + f_3^2 + f_4^2 = f_5^2$.
Moreover, this development can be generalized to arbitrary Pythagorean tuples.

The study of four squares in the eighteenth century
was driven by the elusive search for a proof that
every positive integer is the sum of the squares of four integers.
This famous result was apparently known to Diophantus in the third century,
since he assumes it implicitly in his writings \cite{dickson52}, % p. 275
and Fermat % (in characteristic fashion!) 
claimed that he had a proof.
% \footnote{"I can
% 	not give the proof here, which depends upon numerous and abstruse
%	mysteries of numbers; for I intend to devote an entire book
%	to this subject" \cite[p. 6]{dickson52}.}
However, the first published proof was by Lagrange in 1770 \cite{herstein75}, % p. 375
over a century later,
during which time many mathematicians worked on the problem.
One of the mathematicians who was intensely interested in this problem was Euler, 
and he established the following important result in 1748 \cite{weil84}.   
        % p. 226 of Weil
        % p. 209 of Ebbing, p. 373 of Herstein herstein75
	% p. 210, Ebbinghaus for any commutative ring and Gauss' proof
	% another bulky version in \cite[p. 277]{dickson52}.

\begin{theorem}[Euler's Four Squares Theorem]
\label{thm:euler}
\[
\begin{array}{ll}
& (a_1^2 + a_2^2 + a_3^2 + a_4^2) 
(\hat{a}^2_1 + \hat{a}^2_2 + \hat{a}^2_3 + \hat{a}^2_4) = \\
& (a_1 \hat{a}_1 + a_2\hat{a}_2 + a_3\hat{a}_3 + a_4\hat{a}_4)^2 +
  (a_1 \hat{a}_2 - a_2\hat{a}_1 - a_3\hat{a}_4 + a_4\hat{a}_3)^2 + \\
& (a_1\hat{a}_3 + a_2\hat{a}_4 - a_3\hat{a}_1 - a_4\hat{a}_2)^2 +
  (a_1\hat{a}_4 - a_2\hat{a}_3 + a_3\hat{a}_2 - a_4\hat{a}_1)^2
\end{array}
\]
for any integers $a_1,a_2,a_3,a_4,\hat{a}_1,\hat{a}_2,\hat{a}_3,\hat{a}_4$.
%(but it can easily be generalized to polynomials, or any elements of a commutative ring).
% \footnote{The original statement was for integers, but it easily generalizes.}
% see p. 210 of Ebbinghaus
\end{theorem}
%
\Comment{
% version in Herstein (called Lagrange's identity) and Ebbinghaus
% the version directly attributed to Euler in Dickson and Weil is cleaner, so we use it
USING THE DICKSON FORM:
% another version of Euler's 4-squares theorem
\begin{lemma}[Euler's Four Squares Theorem \cite{herstein75}]
\label{lem:euler}
% p. 373 of Herstein
\[
\begin{array}{ll}
& (a_1^2 + a_2^2 + a_3^2 + a_4^2) 
(\hat{a}^2_1 + \hat{a}^2_2 + \hat{a}^2_3 + \hat{a}^2_4) = \\
& (a_1 \hat{a}_1 - a_2\hat{a}_2 - a_3\hat{a}_3 - a_4\hat{a}_4)^2 +
   (a_1\hat{a}_2 + a_2\hat{a}_1 + a_3\hat{a}_4 - a_4\hat{a}_3)^2 + \\
& (a_1\hat{a}_3 - a_2\hat{a}_4 + a_3\hat{a}_1 + a_4\hat{a}_2)^2 +
   (a_1\hat{a}_4 + a_2\hat{a}_3 - a_3\hat{a}_2 + a_4\hat{a}_1)^2
\end{array}
\]
where $a_1,a_2,a_3,a_4,\hat{a}_1,\hat{a}_2,\hat{a}_3,\hat{a}_4$ are elements of a
commutative ring.
% \footnote{The original statement was for integers, but it easily generalizes.}
% see p. 210 of Ebbinghaus
\end{lemma}

\begin{corollary}
\label{lem:suff4}
$(a_1^2 + a_2^2 + a_3^2 - a_4^2,\ 2a_1a_4,\ 2a_2a_4,\ 2a_3a_4,\ 
 a_1^2 + a_2^2 + a_3^2 + a_4^2)$
is a Pythagorean quintuple for any polynomials $a_1,a_2,a_3,a_4$.
\end{corollary}
\prf
Let $(\hat{a}_1,\hat{a}_2,\hat{a}_3,\hat{a}_4) = (a_1,-a_2,-a_3,a_4)$. 
\QED
}
%
This result, which is also called Lagrange's identity \cite{herstein75}, % p. 373
establishes that the product of two sums of four squares is another sum of four squares,
reducing the problem of showing that every integer is the sum of four squares to 
the simpler problem of showing that every prime is the sum of four squares.
  % since every integer may be expressed as the product of primes.

However, for our purposes, 
Euler's Four Squares Theorem is a Pythagorean quintuple generator.

\begin{corollary}
\label{cor:euler}
\begin{equation}
\label{eq:aida}
(a_1^2 + a_2^2 + a_3^2 - a_4^2)^2 + (2a_1a_4)^2 + (2a_2a_4)^2 + (2a_3a_4)^2
= (a_1^2 + a_2^2 + a_3^2 + a_4^2)^2
\end{equation}
for any integers $a_1,a_2,a_3,a_4$.
  % where $a_1,a_2,a_3,a_4$ are elements of any commutative ring.
\end{corollary}
\prf
(\ref{eq:aida}) is generated by substituting 
$(\hat{a}_1,\hat{a}_2,\hat{a}_3,\hat{a}_4) = (a_1,a_2,a_3,-a_4)$.
This is the next obvious choice after 
$(\hat{a}_1,\hat{a}_2,\hat{a}_3,\hat{a}_4) = (a_1,a_2,a_3,a_4)$,
which yields a tautology.
% This proves the result for integers.
% The Four Squares Theorem may easily be generalized to polynomials, 
% or any elements of a commutative ring (see Definition~\ref{defn:commring} below), 
% such as the rationals and rational polynomials.
% It can be further generalized to rational numbers or rational polynomials
% by placing these numbers or polynomials over a least common multiple,
% which then factors out.
\QED

Ammei \cite{dickson52} % p. 318 (circa 1817)
generalized Corollary~\ref{cor:euler} as follows,
which we can use as a general Pythagorean tuple generator.

\begin{lemma}[Ammei]	% c. 1817
\label{lem:ammei}
\begin{equation}
\label{eq:ammei}
(a_1^2 + \cdots + a_{n-1}^2 - a_n^2)^2 + (2a_1a_n)^2 + \ldots + (2a_{n-1}a_n)^2 
= (a_1^2 + \cdots + a_n^2)^2
\end{equation}
where $n \geq 2$, $a_1,\ldots,a_n \in K$ and $K$ is any commutative ring, 
\end{lemma}

% A {\bf commutative ring} is a ring in which multiplication is commutative. % obvious
Examples of commutative rings are the integers $\mathbb{Z}$, 
rational numbers $\mathbb{Q}$, real numbers $\mathbb{R}$, polynomials $\mathbb{P}$
(over the integers, rationals, or reals),
and rational polynomials $\mathbb{QP}$ (over the integers, rationals, or reals).

\Comment{
    \section{A Characterization of Rational Maps of $\mathbb{R}^4$ to \Sn{3}}

    Euler's Four Squares Theorem may be used to develop
    a weak\footnote{It is weak in the sense of its relaxed restriction on $\alpha$,
    as discussed below.} 
    normal form for Pythagorean quintuples.

    % VERSION FOR DICKSON FORM OF 4-SQUARE THEOREM  
    We get the following formula by letting 
    $(\hat{a}_1,\hat{a}_2,\hat{a}_3,\hat{a}_4) = (a_1,-a_2,-a_3,a_4)$
    in the Four Squares Theorem:
    % \label{lem:corEuler4square}
    \begin{equation}
    \label{eq:aida}
    (a_1^2 + a_2^2 + a_3^2 - a_4^2)^2 + (2a_1a_4)^2 + (2a_2a_4)^2 + (2a_3a_4)^2
    = (a_1^2 + a_2^2 + a_3^2 + a_4^2)^2
    \end{equation}
    % where $a_1,a_2,a_3,a_4$ are elements of any commutative ring.
}

% This exhausts the related previous work on Pythagorean tuples.

% -------------------------------------------------------------------------------

\section{A normal form for Pythagorean tuples}
\label{sec:normpyth}

We are not aware of any specific study of Pythagorean quintuples.
However, Corollary~\ref{cor:euler} leads to a normal form for Pythagorean 
quintuples.

\begin{lemma}
\label{lem:quint}
A quintuple over K is Pythagorean if and only it is of the form
\begin{equation}
\label{eq:pythquint}
	\alpha (a_1^2 + a_2^2 + a_3^2 - a_4^2,
		\ 2a_1a_4,\ 2a_2a_4,\ 2a_3a_4,
		\ a_1^2 + a_2^2 + a_3^2 + a_4^2)
\end{equation}
for some $a_1,a_2,a_3,a_4 \in K$, $\alpha\neq 0 \in K$, 
where $K = \mathbb{Q}, \mathbb{R}$ or $\mathbb{QP}$.\footnote{We need a
  commutative ring closed under quotient (so $\mathbb{Z}$ and $\mathbb{P}$ 
  do not work) because we introduce quotients in the proof.}
\end{lemma}
\prf
(If:)
Suppose that $Q = \alpha (a_1^2 + a_2^2 + a_3^2 - a_4^2,
		\ 2a_1a_4,\ 2a_2a_4,\ 2a_3a_4,
		\ a_1^2 + a_2^2 + a_3^2 + a_4^2)$ is a quintuple over $K$.
By Corollary~\ref{cor:euler}, $\frac{1}{\alpha} Q$ is a Pythagorean quintuple.
But any nonzero multiple of a Pythagorean tuple is also a Pythagorean tuple,
so $Q$ is a Pythagorean quintuple.

(Only if:) 
Let $Q = (p_1,p_2,p_3,p_4,p_5)$ be a Pythagorean quintuple over $K$.
% and let $f(a_1,a_2,a_3,a_4,\alpha)$ be a shorthand for the normal form
% $\frac{1}{\alpha} (a_1^2 + a_2^2 + a_3^2 - a_4^2,
%		 \ 2a_1a_4,\ 2a_2a_4,\ 2a_3a_4,
%		 \ a_1^2 + a_2^2 + a_3^2 + a_4^2)$.

If $p_1 = p_5$, then $p_1^2 + \ldots p_4^2 = p_5^2$ 
implies $p_2^2 + p_3^2 + p_4^2 = 0$ or $p_2=p_3=p_4=0$
and $Q = (p_1,0,0,0,p_1)$.\footnote{Care must be taken 
  to choose $K$ so that $p_2^2 + p_3^2 + p_4^2 = 0$ has no nontrivial solutions.
  For example, if $K = \mathbb{C}$, $(p_2,p_3,p_4)=(i,1,0)$ 
  would be a nontrivial solution.}
The substitution $(a_1,a_2,a_3,a_4,\alpha) = (p_1,0,0,0,\frac{1}{p_1})$ into the
normal form yields $Q$.

If $p_1 \neq p_5$, then the substitution 
$(a_1,a_2,a_3,a_4,\alpha) = 
 (\frac{p_2}{2(p_5-p_1)},
  \frac{p_3}{2(p_5-p_1)},
  \frac{p_4}{2(p_5-p_1)},
  \frac{1}{2}, 2(p_5-p_1))$ into the normal form yields $Q$.
To see this, note that
\begin{eqnarray*}
  \alpha(a_1^2 + a_2^2 + a_3^2 - a_4^2)
& = & 2(p_5-p_1) \frac{p_2^2 + p_3^2 + p_4^2 - (p_5 - p_1)^2}{(2(p_5-p_1))^2} \\
& = & \frac{(p_5^2 - p_1^2) - (p_5 - p_1)^2}{2(p_5 - p_1)} \\
& = & \frac{-2p_1^2 + 2p_1p_5}{2(p_5 - p_1)} = p_1
\end{eqnarray*}
where we use the Pythagorean property $p_1^2 + p_2^2 + p_3^2 + p_4^2 = p_5^2$
in the second equation.
Secondly, note that
\[
\alpha (2a_i a_4) = 2(p_5-p_1)\frac{p_{i+1}}{p_5-p_1}\frac{1}{2}
= p_{i+1}
\]
for $i=1,2,3$.
Finally, note that 
\begin{eqnarray*}
\alpha(a_1^2 + a_2^2 + a_3^2 + a_4^2)
& = & \frac{p_2^2 + p_3^2 + p_4^2 + (p_5 - p_1)^2}{2(p_5-p_1)} 
= \frac{(p_1^2 + p_2^2 + p_3^2 + p_4^2) + p_5^2 -2p_1p_5}{2(p_5-p_1)} \\
& = & \frac{2p_5^2 - 2p_1p_5}{2(p_5-p_1)} 
= p_5
\end{eqnarray*}
\QED

\Comment{
\noindent The following lemma establishes a weaker version of the necessary condition
associated with Corollary~\ref{lem:suff4}.
This in turn leads to the desired necessary and sufficient condition for rational maps
of $\mathbb{R}^4$ to \Sn{3}.

\begin{lemma}
\label{thm:necessary4}
A quintuple of polynomials is Pythagorean only if it can be expressed in the form
\begin{equation}
\label{eq:pyth}
	\alpha (a_1^2 + a_2^2 + a_3^2 - a_4^2,
		\ 2a_1a_4,\ 2a_2a_4,\ 2a_3a_4,
		\ a_1^2 + a_2^2 + a_3^2 + a_4^2)
\end{equation}
for some polynomials $a_1,a_2,a_3,a_4,\frac{1}{\alpha}$.
\end{lemma}
\prf
Let $(p_1,p_2,p_3,p_4,p_5)$ be a Pythagorean quintuple of polynomials.
If $p_1 = p_5$, let\\
$(a_1,a_2,a_3,a_4,\alpha) = (p_1,0,0,0,\frac{1}{p_1})$.
Thus, we may assume without loss of generality that $p_1 \neq p_5$.
Let $(a_1,a_2,a_3,a_4,\alpha) = (p_2,p_3,p_4,p_5-p_1,\frac{1}{2(p_5 - p_1)})$.
Then $a_1,a_2,a_3,a_4,\alpha$ generate the Pythagorean quintuple
$(p_1,\ldots,p_5)$ as in (\ref{eq:pyth}).
In particular,
\[
\alpha (a_1^2 + a_2^2 + a_3^2 - a_4^2)
= \frac{p_2^2 + p_3^2 + p_4^2 - (p_5 - p_1)^2}{2(p_5-p_1)}
\]
and applying $p_1^2 + p_2^2 + p_3^2 + p_4^2 = p_5^2$,
\[
\alpha (a_1^2 + a_2^2 + a_3^2 - a_4^2) = \frac{-2p_1^2 + 2p_1p_5}{2(p_5 - p_1)} = p_1.
\]
Also, 
\[
\alpha (2a_i a_4) = \frac{2p_{i+1}(p_5 - p_1)}{2(p_5 - p_1)} = p_{i+1}
\]
for $i=1,2,3$.
Finally, 
\[
\alpha(a_1^2 + a_2^2 + a_3^2 + a_4^2) 
= \frac{p_2^2 + p_3^2 + p_4^2 + (p_5 - p_1)^2}{2(p_5-p_1)} = p_5
\]
Thus, $(p_1,p_2,p_3,p_4,p_5) = \alpha (a_1^2 + a_2^2 + a_3^2 - a_4^2,
		\ 2a_1a_4,\ 2a_2a_4,\ 2a_3a_4,
		\ a_1^2 + a_2^2 + a_3^2 + a_4^2)$.
\QED
} % of \Comment

\Comment{
  OTHER VERSION:
  This formula is very close to a normal form for Pythagorean quintuples.

  \begin{theorem}
    \label{thm:necessary4}
    A quintuple over K is a Pythagorean quintuple if and only it is of the form
    \begin{equation}
      \label{eq:pythquint}
      \frac{1}{\alpha} (a_1^2 + a_2^2 + a_3^2 - a_4^2,
      \ 2a_1a_4,\ 2a_2a_4,\ 2a_3a_4,
      \ a_1^2 + a_2^2 + a_3^2 + a_4^2)
    \end{equation}
    for some $a_1,a_2,a_3,a_4,\alpha\neq 0 \in K$, 
    where $K$ is the ring of integers
    or polynomials over $\mathbb{R}[x_1,x_2,\ldots,x_n]$.
    % NB: proof not valid for comm. ring, since comm. ring may not have unit elt 1, 
    % which is needed immediately
  \end{theorem}
  \prf
  (If:)
   Suppose that $Q = \frac{1}{\alpha} (a_1^2 + a_2^2 + a_3^2 - a_4^2,
		\ 2a_1a_4,\ 2a_2a_4,\ 2a_3a_4,
		\ a_1^2 + a_2^2 + a_3^2 + a_4^2)$ is a quintuple over $K$.
   By Corollary~\ref{cor:euler}, $\alpha Q$ is a Pythagorean quintuple.
   But any nonzero multiple of a Pythagorean tuple is also a Pythagorean tuple,
   so $Q$ is also a Pythagorean quintuple.

   (Only if:) 
   Let $Q = (p_1,p_2,p_3,p_4,p_5)$ be a Pythagorean quintuple over $K$.
   % and let $f(a_1,a_2,a_3,a_4,\alpha)$ be a shorthand for the normal form
   % $\frac{1}{\alpha} (a_1^2 + a_2^2 + a_3^2 - a_4^2,  
   %		 \ 2a_1a_4,\ 2a_2a_4,\ 2a_3a_4,
   %		 \ a_1^2 + a_2^2 + a_3^2 + a_4^2)$.
   If $p_1 = p_5$, then $Q = (p_1,0,0,0,p_1)$
   and the substitution $(a_1,a_2,a_3,a_4,\alpha) = (p_1,0,0,0,p_1)$ into the
   normal form yields $Q$.
   If $p_1 \neq p_5$, then the substitution 
   $(a_1,a_2,a_3,a_4,\alpha) = (p_2,p_3,p_4,p_5 - p_1, 2(p_5 - p_1))$ into the normal
   form yields $Q$.
   To see this, note that
   \[
   \frac{a_1^2 + a_2^2 + a_3^2 - a_4^2}{\alpha}
   = \frac{p_2^2 + p_3^2 + p_4^2 - (p_5 - p_1)^2}{2(p_5-p_1)}
   = \frac{(p_5^2 - p_1^2) - (p_5 - p_1)^2}{2(p_5 - p_1)}
   = \frac{-2p_1^2 + 2p_1p_5}{2(p_5 - p_1)} 
   = p_1
   \]
   where we use the Pythagorean property in the second equation.
   Secondly, note that
   \[
   \frac{2a_i a_4}{\alpha} = \frac{2p_{i+1}(p_5 - p_1)}{2(p_5 - p_1)} = p_{i+1}
   \]
   for $i=1,2,3$.
   Finally, note that 
   \[
   \frac{a_1^2 + a_2^2 + a_3^2 + a_4^2}{\alpha}
   = \frac{p_2^2 + p_3^2 + p_4^2 + (p_5 - p_1)^2}{2(p_5-p_1)} 
   = \frac{(p_1^2 + p_2^2 + p_3^2 + p_4^2) + p_5^2 -2p_1p_5}{2(p_5-p_1)} 
   = \frac{2p_5^2 - 2p_1p_5}{2(p_5-p_1)} 
   = p_5
   \]
   \QED

   Unfortunately, 
   (\ref{eq:pythquint}) is not a sufficient condition for Pythagorean quintuples
   over $\mathbb{R}[x_1,x_2,\ldots,x_n]$, since
   a quintuple in the form (\ref{eq:pythquint}) is not necessarily
   even a quintuple over $\mathbb{R}[x_1,x_2,x_3,x_4]$.
   For example, $(a_1,a_2,a_3,a_4,\alpha) = (1,1,1,1,x_1)$ yields the
   quintuple $(\frac{2}{x_1}, \frac{2}{x_1}, \frac{2}{x_1}, \frac{2}{x_1}, \frac{4}{x_1})$,
   which is not ---.
   However, it successfully yields a characterization
   of the rational maps of $\mathbb{R}^4$ to \Sn{3}.\footnote{The weaker restriction on $\alpha$
     suffices because of the cancellation of $\alpha$ as we translate from Pythagorean
     tuples to rational maps, as illustrated in the proof.}

   It suffices to establish a weaker version,
   with a relaxed restriction on $\alpha$,
   since our goal is not a characterization of Pythagorean tuples {\em per se},
   but a characterization of rational maps to the sphere.
} % \Comment

\noindent This result generalizes, using Ammei's Lemma~\ref{lem:ammei}.

\begin{theorem}
\label{thm:general}
An $n+1$-tuple over K, $n \geq 2$, is Pythagorean if and only if 
it is of the form 
\begin{equation}
\label{eq:pyth2}
	\alpha (a_1^2 + \ldots + a_{n-1}^2 - a_n^2,
		\ 2a_1a_n,\ldots,\ 2a_{n-1}a_n,
		\ a_1^2 + \ldots + a_n^2)
\end{equation}
for some $a_1,\ldots,a_n \in K$, $\alpha \neq 0 \in K$,
where $K = \mathbb{Q}, \mathbb{R}$ or $\mathbb{QP}$.
\end{theorem}
\prf
The proof is completely analogous to the proof of Lemma~\ref{lem:quint},
replacing quintuple by $n+1$-tuple and 5 by $n+1$.
\QED

Theorem~\ref{thm:general} provides a stronger characterization of
Pythagorean quadruples.
Since Theorem~\ref{thm:general} applies for all $n \geq 2$,
it characterizes Pythagorean triples, quadruples, and quintuples
that have already been considered.
It agrees with our previous characterizations for $n=2,4$ (Pythagorean triples
and quintuples),
but it yields another characterization for $n=3$, as can be seen 
by comparing with (\ref{eq:quad}).
This is not a contradiction: any Pythagorean quadruple
may be expressed either in the form of (\ref{eq:quad})
or in the form of Theorem~\ref{thm:general} with $n=3$:
\[
\alpha (a_1^2 + a_{2}^2 - a_3^2, \ 2a_1a_3,\ 2a_2a_3, \ a_1^2 + a_2^2 + a_3^2)
\]
%
It turns out that if we set 
$(a_1,a_2,a_3,a_4) = (\hat{a}_2, \hat{a}_1, 0, \hat{a}_3)$
in the characterization (\ref{eq:quad}),
the characterization of Theorem~\ref{thm:general} for $n=3$ is produced.
Thus, the characterization of Theorem~\ref{thm:general} is actually stronger than
(\ref{eq:quad}): not only can a Pythagorean quadruple
be put in the form of (\ref{eq:quad}), it can be put in the form
of (\ref{eq:quad}) with $a_3=0$.

\section{A Characterization of Rational Maps to the Sphere}
\label{sec:char}

Through the identification of Pythagorean $n+1$-tuples and rational maps 
to the sphere \Sn{n-1} (Lemma~\ref{lem:ratpy}),
Theorem~\ref{thm:general} leads to the sought-after 
characterization of rational maps to \Sn{n-1}.
This is the main result of this paper.
 
\Comment{
\prf
(Only if): Consider a rational map
of $\mathbb{R}^m$ to \Sn{3}, $f(x_1,\ldots,x_m) \mapsto 
(\frac{f_1}{f_5},\frac{f_2}{f_5},\frac{f_3}{f_5},\frac{f_4}{f_5})$,
where $f_1,\ldots,f_5 \in \mathbb{R}[x_1,\ldots,x_m]$.
Then $(f_1,\ldots,f_5)$ is a Pythagorean quintuple by Lemma~\ref{lem:ratpy} and 
can be expressed as some permutation of
$\alpha (a_1^2 + a_2^2 + a_3^2 - a_4^2,
2a_1a_4,2a_2a_4,2a_3a_4,$ $a_1^2 + a_2^2 + a_3^2 + a_4^2)$
for some rational polynomials $a_1,a_2,a_3,a_4,\alpha \neq 0$.
% Thus, $(x_{\pi(1)},x_{\pi(2)},x_{\pi(3)},x_{\pi(4)}) \mapsto$
% \[
%	(\frac{\alpha(a_1^2 + a_2^2 + a_3^2 - a_4^2)}{\alpha(a_1^2 + a_2^2 + a_3^2 + a_4^2)},
%	 \frac{\alpha(2a_1a_4)}{\alpha(a_1^2 + a_2^2 + a_3^2 + a_4^2)},
%	 \frac{\alpha(2a_2a_4)}{\alpha(a_1^2 + a_2^2 + a_3^2 + a_4^2)},
%	 \frac{\alpha(2a_3a_4)}{\alpha(a_1^2 + a_2^2 + a_3^2 + a_4^2)})
% \]
% \[
% = 	(\frac{a_1^2 + a_2^2 + a_3^2 - a_4^2}{a_1^2 + a_2^2 + a_3^2 + a_4^2},
%	 \frac{2a_1a_4}{a_1^2 + a_2^2 + a_3^2 + a_4^2},
%	 \frac{2a_2a_4}{a_1^2 + a_2^2 + a_3^2 + a_4^2},
%	 \frac{2a_3a_4}{a_1^2 + a_2^2 + a_3^2 + a_4^2})
% \]
% Notice that it is legal to cancel the $\alpha$, since
% $\alpha = \frac{1}{\beta}$ for some $\beta \in \mathbb{R}[x_1,x_2,x_3,x_4]$
% which implies that $\alpha$ is never zero.\\
% (If): Consider a map of the form (\ref{eq:re4s3}).
% This is a rational map of $\mathbb{R}^4$ to \Sn{3}\ by (\ref{eq:suff4})
% and Lemma~\ref{lem:iffpyth}.
\QED
}

\begin{theorem}
\label{thm:ratmap}
The map $f: \mathbb{R}^m \rightarrow \Sn{n-1}$ is rational if and only if
it is of the following form (up to coordinate permutation):
\begin{equation}
\label{eq:remsn-1}
f(x_1,\ldots,x_m) = 
(\frac{a_1^2 + \cdots + a_{n-1}^2 - a_n^2}{a_1^2 + \cdots + a_n^2},
	 \frac{2a_1a_n}{a_1^2 + \cdots + a_n^2},
	 \ldots, 
	 \frac{2a_{n-1}a_n}{a_1^2 + \cdots + a_n^2})
\end{equation}
where $a_1,\ldots,a_n \in \mathbb{QP}$ (rational polynomials over the
integers, rationals or reals).
\end{theorem}
\prf
Consider a rational map
of $\mathbb{R}^m$ to \Sn{n-1}, $(x_1,\ldots,x_m) \mapsto 
(\frac{f_1}{f_{n+1}},\ldots,\frac{f_n}{f_{n+1}})$,
where $f_1,\ldots,f_{n+1}$ are polynomials over $x_1,\ldots,x_m$.
Then $(f_1,\ldots,f_{n+1})$ is a Pythagorean $n+1$-tuple 
over $\mathbb{QP}$ (Lemma~\ref{lem:ratpy})
so it can be expressed in the normal form
$(f_1,\ldots,f_{n+1}) = \alpha (a_1^2 + \ldots + a_{n-1}^2 - a_n^2,
2a_1a_n,\ldots,2a_{n-1}a_n,a_1^2 + \ldots + a_n^2)$
for some $a_1,\ldots,a_n \in \mathbb{QP}$, $\alpha \neq 0 \in \mathbb{QP}$
by Theorem~\ref{thm:general}.
Thus, the map is of the form (\ref{eq:remsn-1}), since the leading $\alpha$
cancels when expressed in 
$(\frac{f_1}{f_{n+1}},\ldots,\frac{f_n}{f_{n+1}})$.
Notice that it is legal to cancel the $\alpha$, since $\alpha \neq 0$.

Now consider a map of the form (\ref{eq:remsn-1}).
This is a rational map of $\mathbb{R}^m$ to \Sn{n-1}\ by Lemma~\ref{lem:ratpy} 
and Theorem~\ref{thm:general}.
\QED

\Comment{
This provides a sufficient condition for Pythagorean $(n+1)$-tuples.
We can again establish a necessary condition with a weaker restriction on
$\alpha$.

\begin{theorem}
\label{thm:necessaryn}
Let $n \geq 2$ and $D$ be the integers or the polynomials 
over $\mathbb{R}[x_1,\ldots,x_n]$.
A Pythagorean $(n+1)$-tuple over $D$ can be expressed in the form
\begin{equation}
\label{eq:pyth2}
	\alpha (a_1^2 + \ldots + a_{n-1}^2 - a_n^2,
		\ 2a_1a_n,\ldots,\ 2a_{n-1}a_n,
		\ a_1^2 + \ldots + a_n^2)
\end{equation}
for some $a_1,\ldots,a_n,\frac{1}{\alpha} \in D$.
\end{theorem}
\prf
This proof is analogous to that of Theorem~\ref{thm:necessary4}.
Let $(p_1,\ldots,p_{n+1})$ be a Pythagorean $(n+1)$-tuple over $D$.
If $p_1 = p_{n+1}$, let 
$(a_1,\ldots,a_n,\alpha) = (p_1,0,\ldots,0,\frac{1}{p_1})$.
Assume without loss of generality that $p_1 \neq p_{n+1}$.
Let 
\[
(a_1,\ldots,a_n,\alpha) = (p_2,\ldots,p_n,p_{n+1}-p_1,
	\frac{1}{2(p_{n+1}-p_1)})
\]
Then $a_1,\ldots,a_n,\alpha$ generate the Pythagorean $(n+1)$-tuple
$(p_1,\ldots,p_{n+1})$ as in (\ref{eq:pyth2}).
In particular,
\[
\alpha (a_1^2 + \ldots + a_{n-1}^2 - a_n^2)
= \frac{p_2^2 + \cdots + p_n^2 - (p_{n+1}-p_1)^2}{2(p_{n+1}-p_1)}
\]
and applying $p_1^2 + \ldots + p_n^2 = p_{n+1}^2$,
\[
= \frac{-2p_1^2 + 2p_1p_{n+1}}{2(p_{n+1} - p_1)}
= p_1
\]
And
\[
\alpha (2a_i a_n) 
= \frac{2p_{i+1}(p_{n+1}-p_1)}{2(p_{n+1}-p_1)}
= p_{i+1}
\]
for $i=1,\ldots,n-1$.
Finally, 
\[ 
\alpha (a_1^2 + \ldots + a_n^2)
= \frac{p_2^2 + \cdots + p_n^2 + (p_{n+1}-p_1)^2}{2(p_{n+1}-p_1)}
= p_{n+1}.
\]
\QED

This leads to our most general result: 
a full characterization of the rational maps of $\mathbb{R}^n$ to \Sn{n-1}.

\begin{theorem}
\label{thm:nn}
A map is a rational map of $\mathbb{R}^n$ to \Sn{n-1}, $n \geq 2$, if and only if
it is of the form:
\[
	(x_{\pi(1)},\ldots,x_{\pi(n)}) \mapsto 
	(\frac{a_1^2 + \cdots + a_{n-1}^2 - a_n^2}{a_1^2 + \cdots + a_n^2},
	 \frac{2a_1a_n}{a_1^2 + \cdots + a_n^2},
	 \ldots,
	 \frac{2a_{n-1}a_n}{a_1^2 + \cdots + a_n^2})
\]
where $a_1,\ldots,a_n \in \mathbb{R}[x_1,\ldots,x_n]$
and $\pi: \{1,\ldots,n\} \rightarrow \{1,\ldots,n\}$ is a permutation.
\end{theorem}
\prf
Identical to Theorem~\ref{thm:map4}'s proof.
\QED
}

\Comment{
We now have a characterization of all rational maps from $\mathbb{R}^n$ to \Sn{n-1}.
The related findings on Pythagorean $(n+1)$-tuples are reviewed 
in Table~\ref{tab:pyth}.
The second column of this table gives the normal form for Pythagorean
$(n+1)$-tuples for various $n$.
The third column indicates when this normal form was shown to be a sufficient 
condition for an $(n+1)$-tuple to be a Pythagorean $(n+1)$-tuple 
over the integers.
The fourth column indicates when this normal form was shown to be a necessary 
condition for an $(n+1)$-tuple to be a Pythagorean $(n+1)$-tuple 
over the integers.
The fifth column indicates when this normal form was shown to be a necessary
condition for an $(n+1)$-tuple to be a Pythagorean $(n+1)$-tuple 
over polynomial rings.
The results from this paper are not full proofs (and are thus marked
by stars), since $\frac{1}{\alpha} \in D$, not $\alpha \in D$.

\begin{table}[h]
\label{tab:pyth}
\begin{tabular}{|c|c|c|c|c|}
\hline
$n$ & normal form for Pythagorean $(n+1)$-tuples & suff {\cal Z} & nec {\cal Z} & 
\footnotesize{nec $\mathbb{R}[x_1,\ldots,x_n]$} \\
\hline
$2$ & \tiny{$\alpha(a_1^2 - a_2^2, 2a_1a_2, a_1^2 + a_2^2)$} &
\footnotesize{classical} & 
\footnotesize{classical} & 
\footnotesize{Kubota 1972} \\ 
\hline
$3$ & \tiny{$\alpha (a_1^2+a_2^2-a_3^2-a_4^2, 2(a_1a_3+a_2a_4), 
    2(a_1a_4-a_2a_3), a_1^2+a_2^2+a_3^2+a_4^2)$} &
\footnotesize{Euler 1748} & 
\footnotesize{Catalan 1885} & 
\footnotesize{Dietz 1993} \\ 
\hline
$4$ & \tiny{$\alpha (a_1^2 + a_2^2 + a_3^2 - a_4^2,
		2a_1a_4, 2a_2a_4, 2a_3a_4,
		a_1^2 + a_2^2 + a_3^2 + a_4^2)$} &
\footnotesize{Euler 1748} & 
\footnotesize{$^*$Johnstone 1998} & 
\footnotesize{$^*$Johnstone 1998} \\
\hline
$n$ & \tiny{$\alpha (a_1^2 + \ldots + a_{n-1}^2 - a_n^2,
	2a_1a_n, \ldots, 2a_{n-1}a_n,
	a_1^2 + \ldots + a_n^2)$} & 
\footnotesize{Ammei 1817} & 
\footnotesize{$^*$Johnstone 1998} & 
\footnotesize{$^*$Johnstone 1998} \\ \hline
\end{tabular}
\end{table}
}

% -----------------------------------------------------------------------------

% \section{Generating rational maps to the sphere}
% \label{sec:generation}

Consider applying this normal form to generate rational maps to the sphere.
The natural choice of polynomials $(a_1,\ldots,a_n) = (x_1,\ldots,x_n)$ 
in this normal form yields a map from
$n$-space to \Sn{n-1}, which we call {\em Euler's map to the sphere}:
\begin{equation}
\label{eq:euler}
  M(x_1,\ldots,x_n) = \frac{1}{x_1^2 + \cdots + x_n^2} 
  (x_1^2 + \cdots + x_{n-1}^2 - x_n^2, 2x_1x_n, \ldots, 2x_{n-1}x_n)
\end{equation}
%
%
The inverse of the stereographic projection map (see (\ref{eq:spinverse})),
expressed as a map from $n$-space to \Sn{n},
is achieved by the assignment 
$(a_1,\ldots,a_n) = (x_1,\ldots,x_{n-1},1)$ and a permutation.
%
% \[
% f(x_1,\ldots,x_n) = \frac{1}{x_1^2 + \cdots + x_{n-1}^2 + 1}
%    (2x_1, \ldots, 2x_{n-1}, x_1^2 + \cdots + x_{n-1}^2 - 1)
% \]
%
The standard rational parameterization (\ref{eq:s2param}) of the 2-sphere \Sn{2}
is achieved by the assignment 
\begin{equation}
\label{eq:ratsphereass}
(a_1,a_2,a_3) = (\frac{2v(1-u^2)}{4u^2+4v^2},
                  \frac{2u(1+v^2)}{4u^2+4v^2},
                  \frac{1}{2}).
\end{equation}
% \alpha = 4u^2+4v^2
This illustrates that some maps require sophisticated choices for the 
polynomials in the normal form.
These difficult choices % for substitution into the normal form 
may be developed using the proof of Lemma~\ref{lem:quint} and its generalization,
as follows.

\begin{lemma}
Given a Pythagorean $(n+1)$-tuple $(p_1,\ldots,p_{n+1})$,
the assignment to the normal form of Theorem~\ref{thm:general} that realizes 
this tuple is $(a_1,\ldots,a_n,\alpha) = (p_1,0,\ldots,0,\frac{1}{p_1})$ 
if $p_1 = p_{n+1}$ and
$(a_1,\ldots,a_n,\alpha) = 
    (\frac{p_2}{2(p_{n+1}-p_1)},\ldots,
     \frac{p_n}{2(p_{n+1}-p_1)},
     \frac{1}{2},
                2(p_{n+1}-p_1) )$
if $p_1 \neq p_{n+1}$.
\end{lemma}
\prf
See the proof of Lemma~\ref{lem:quint}.
\QED

{\bf The fact that $a_n \not\equiv 0$ in this lemma makes it particularly
important to resolve the pole issue for $a_n \not\equiv 0$.}

The assignment (\ref{eq:ratsphereass}) is easily found from this lemma,
after observing that the Pythagorean tuple associated with
the rational parameterization of the sphere is
$(p_1,p_2,p_3,p_4) = (1-u^2-v^2+u^2v^2,\ 2v(1-u^2),\ 2u(1+v^2),\ 1+u^2+v^2+u^2v^2)$.
Notice that the parameter range of the sphere parameterization is not indicated
by the normal form, so the map will be many-to-one.
Examination of the assignment of parameter intervals to the normal form is a topic
for future study.

% \section{Inverse maps}
% Development of general inverse will be very difficult.
% See comment in section on development of inverse of M.
% Practically, we should do it for specific maps only.
% It suffices to establish the poles, which can be done without the inverse maps.

% -----------------------------------------------------------------------------

\section{The Inevitability of Poles}
\label{sec:pole}

The normal form for rational maps to the sphere clarifies why the inverse
of a rational map to the sphere always has a pole.
Consider a rational map from $\mathbb{R}^m$ to \Sn{n}.
We expect a manifold of dimension $m-n$ to map to each point of the sphere,
simply by a counting argument.
For example, with a map from $\mathbb{R}^n$ to \Sn{n-1} like 
the Euler map (\ref{eq:euler}),
the preimage of a 
typical point of the sphere is a line.
However, for some anomalous points of the sphere (poles), 
the dimension of the preimages
is larger than expected, resulting in a poorly defined inverse at this point.
For example, with the Euler map,
the entire hyperplane $x_n=0$ maps to $(1,0,\ldots,0)$:
the preimage of $(1,0,\ldots,0)$ is a hyperplane of dimension $n-1$ 
rather than a line of dimension one.

This is not a vagary of the Euler map: it is a general property of
rational maps to the sphere.
Considering the normal form (\ref{eq:remsn-1}) for rational maps to the sphere,
the entire hypersurface $a_n(x_1,\ldots,x_m)=0$ will map to the same point,
$(1,0,\ldots,0)$.
This is a manifold of dimension $m-1$ rather than the expected manifold 
of dimension $m-n+1$.
The points satisfying $a_n = 0$ lead to poles in the inverse map 
associated with a rational map to the sphere.

{\bf start here}

What if $a_n \not\equiv 0$?
Poles %, or points of the sphere that have more preimages than they should,
are present even in maps built from the normal form 
with constant polynomial $a_n \equiv k \neq 0$.
For example, the normal form for the rational parameterization (\ref{eq:s2param})
of \Sn{2} has $a_n = \frac{1}{2}$ (see Section~\ref{sec:char}).
Yet this map has a pole at $(0,0,1)$.
(This is the part of the parameterization associated with the circle of radius 0 
in the sweeping of a circle using the spherical product;
or, in common parlance, the degenerate line of latitude.
There is another pole at the other end of the sphere.)
$(0,0,1)$ is a pole because the entire line $u=1$ maps to $(0,0,1)$,
a manifold of dimension one rather than the expected manifold of dimension zero 
for this map of $\mathbb{R}^2$ to \Sn{2}.
{\bf But this is not a proof.  How could we predict a pole there?}
...
{\bf What if we work in projective space and homogenize $a_n$ (to the other $a_i$)
  and then set it to 0?}

Another example is the inverse of
stereographic projection, which has $a_n \equiv 1$ in the normal form.
If we include points at infinity, this map has a pole at $(0,0,1)$:
the entire line $w=0$ in projective space maps to $(0,0,1)$.
{\bf How could we predict a pole there?}

The accumulation of preimages at certain points is not surprising.
% It is an artifact of wrapping a Euclidean space about the sphere.
It is well known that mappings of the sphere to Euclidean space have these poles,
which have long been the bane of cartographers as they map the spherical earth to
the flat page.
But now we have an explanation in terms of the normal form (\ref{eq:remsn-1}).

\Comment{
\section{Visualization}
\label{sec:viz}

The rational maps to the sphere generated by our characterization may be compared
using a visualization technique that measures how they spread Euclidean space
across the sphere.
This is most easily done for \Sn{2}, so we will concentrate the visualization there.
However, the visualization of higher dimensions \cite{brown mathematician}
could be used to consider other rational maps.

There are two techniques.
In the first, we sample some region of Euclidean m-space uniformly 
and visualize the number of samples that are mapped to each area of \Sn{n}
(the more samples, the lighter).
Recall that typically many points of Euclidean space will map to a given point of 
the sphere (Section~\ref{sec:pole}).
This technique is particularly useful if it is known how an application restricts 
the domain,
which may be used to choose the region of Euclidean space that is considered.
We do this for 3-space and 2-space (the hyperplane $z=0$ in 3-space).
Some applications are only interested in mapping the n-sphere back to itself: 
the domain is restricted to \Sn{n}.
(For example, the application of these maps to quaternion splines is sometimes
restricted to maps of \Sn{3} to \Sn{3}.)
In the second visualization, we sample \Sn{n} and visualize the distance that 
each sample moves.
This is a local measure of the map, but still evocative of distinctions between
the maps.

RESURRECT MAP TEXTURES AND IMPORT PROSE, IMAGES AND SOFTWARE.

The maps that we consider are the following classical products of the characterization:
1) Euler spherical map
2) inverse of stereographic projection
3) standard rational parameterization of sphere
4) fancier one
5) Dietz map?
}

% -----------------------------------------------------------------------------

\section{Conclusions}
\label{sec:conclude}

We have developed a complete characterization of rational maps to the sphere,
a clarification of the poles of their inverse maps,
and a full characterization of Pythagorean tuples, 
extending past the known Pythagorean
triple and quadruple characterizations, as well as improving the Pythagorean
quadruple characterization.
The main results are Theorem~\ref{thm:general} and Theorem~\ref{thm:ratmap}.

An essential observation in the characterization of rational maps to the sphere 
is the equivalence between rational maps to the sphere and Pythagorean tuples.

Pythagorean quintuples are not as well understood as Pythagorean triples
and quadruples, and our study leads to some new characterizations.
% (Theorems~\ref{thm:necessary4} and \ref{thm:map4}).

\begin{rmk}
Euler's Four Squares Theorem also anticipates the normal form
for Pythagorean quadruples (Section~\ref{sec:prevwork}):
\begin{equation}
\label{eq:euler1}
(a_1^2 + a_2^2 + a_3^2 + a_4^2)^2 = 
(a_1^2 + a_2^2 - a_3^2 - a_4^2)^2 + (2a_1a_3+2a_2a_4)^2 + (2a_1a_4-2a_2a_3)^2
\end{equation}
using $(a_1,a_2,a_3,a_4) = (\hat{a}_1,-\hat{a}_2,\hat{a}_3,\hat{a}_4)$ in 
Theorem~\ref{thm:euler}.
\end{rmk}

Euler's Four Squares Theorem anticipates the quaternion ({\bf elaborate})...,
which is the subject of the natural follow-on to this paper, 
to the application of rational maps in the design of quaternion splines
for motion control.

\begin{rmk}
Euler's Four Squares Theorem anticipates
the quaternion (a century in advance of its invention by Hamilton),
in the sense that it encodes the product formula for quaternions,
perhaps the most defining characteristic of the quaternion:
% (compare Lemma~\ref{lem:euler}):
\[
\begin{array}{ll}
& (a_1 + a_2 i + a_3 j + a_4 k) 
(\hat{a}_1 + \hat{a}_2 i + \hat{a}_3 j + \hat{a}_4 k) = \\
& (a_1 \hat{a}_1 - a_2\hat{a}_2 - a_3\hat{a}_3 - a_4\hat{a}_4) +
   (a_1\hat{a}_2 + a_2\hat{a}_1 + a_3\hat{a}_4 - a_4\hat{a}_3) i + \\
&  (a_1\hat{a}_3 - a_2\hat{a}_4 + a_3\hat{a}_1 + a_4\hat{a}_2) j +
   (a_1\hat{a}_4 + a_2\hat{a}_3 - a_3\hat{a}_2 + a_4\hat{a}_1) k
\end{array}
\]
Thus, it is quite natural to use Euler's Four Squares Theorem
to characterize Pythagorean tuples (through generalization) and rational maps, 
which are then used to build quaternion splines.
\end{rmk}

\bibliographystyle{plain}
\begin{thebibliography}{99}

\bibitem{barr81}
Barr, A. (1981) Superquadrics and Angle-Preserving Transformations.
IEEE Computer Graphics and Applications.

\bibitem{barr92}
Barr, A. and B. Currin and S. Gabriel and J. Hughes (1992)
Smooth Interpolation of Orientations with Angular Velocity Constraints
using Quaternions.
SIGGRAPH '92, 313--320.

\bibitem{catalan85}
Catalan, E. (1885) Bull. Acad. Roy. Belgique 3(9), p. 531.
Referenced in Dickson, L.E. (1952) History of the Theory of Numbers: Volume II,
Diophantine Analysis.  Chelsea (New York), p. 269.

\bibitem{dickson52}
Dickson, L.E. (1952) History of the Theory of Numbers: Volume II,
Diophantine Analysis.  Chelsea (New York).

\bibitem{dietz93}
Dietz, R. and J. Hoschek and B. Juttler (1993)
An algebraic approach to curves and surfaces on the sphere and
on other quadrics.
Computer Aided Geometric Design 10, 211-229.

% Ebbinghaus, H.-D. et. al., Numbers, Springer-Verlag (New York), 1990.

\bibitem{euler48}
Fuss, P., editor (1843) Corresp. Math. et Phys.,
`Correspondance entre Leonhard Euler et C. Goldbach 1729-1763',
St. Petersburg, Vol. 1, p. 452.  
Referenced in Dickson, L.E. (1952) History of the Theory of Numbers: Volume II,
Diophantine Analysis.  Chelsea (New York), p. 277.
% see p. 209, Ebbinghaus, Numbers

\bibitem{farouki90}
Farouki, R. and T. Sakkalis (1990)
Pythagorean Hodographs.
IBM J. Res. Develop. 34, 736--752.

\bibitem{herstein75}
Herstein, I. (1975) Topics in Algebra.
2nd edition, John Wiley (New York).

\bibitem{kreyszig59}
Kreyszig, E. (1959) Differential Geometry.
Dover (New York).

\bibitem{kubota72}
Kubota, K. (1972) Pythagorean triples in unique factorization domains.
American Mathematical Monthly 79, 503--505.

\bibitem{weil84}
Weil, A. (1984) Number Theory.
Birkh\"auser (Boston).

\end{thebibliography}

In technical report version, we could add my proof of Kubota's result, 
as in Appendix of characterization.ps.

\end{document}

% See 'A Characterization of Rational Maps of Rn to $S^{n-1}$ for Riemannian Modeling'
% and its file characterization.tex in 1-char.  (print it out, and the short version)

% ----------------------

% The sphere has long been a Platonic ideal for shape.
% Indeed, all of the Platonic solids
% aspire to the sphere, with differing levels of success.
%
% In our research, the issue that drew us to the sphere was quaternion splines
% (a tool for orientation control in animation and motion design),
% which are modeled on the 3-sphere in 4-space.

% In the classic 'Flatland',
% Edwin Abbott Abbott (sic!) related the ominous appearance of The Sphere in the
% cosy confines of two-space.
% This was Abbott's way of introducing the concept of higher dimensions.

% Elie Cartan: The Theory of Spinors

% The first component $a_1^2 + a_2^2 + a_3^2 - a_4^2$ of the Pythagorean quintuple 
% is a non-positive definite fundamental form (p. 3) and the last component is
% the fundamental form of 4-space (p. 3).
% One of the components of the Pythagorean quadruple generator (Remark 13) is also 
% a fundamental form for a pseudo-Euclidean space.
