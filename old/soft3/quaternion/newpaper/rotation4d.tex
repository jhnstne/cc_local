\documentclass[10pt]{article} 
\usepackage{times}
\usepackage[pdftex]{graphicx}
\makeatletter
\def\@maketitle{\newpage
 \null
 \vskip 2em                   % Vertical space above title.
 \begin{center}
       {\Large\bf \@title \par}  % Title set in \Large size. 
       \vskip .5em               % Vertical space after title.
       {\lineskip .5em           %  each author set in a tabular environment
        \begin{tabular}[t]{c}\@author 
        \end{tabular}\par}                   
  \end{center}
 \par
 \vskip .5em}                 % Vertical space after author
\makeatother

% default values are 
% \parskip=0pt plus1pt
% \parindent=20pt

\newcommand{\SingleSpace}{\edef\baselinestretch{0.9}\Large\normalsize}
\newcommand{\DoubleSpace}{\edef\baselinestretch{1.4}\Large\normalsize}
\newcommand{\Comment}[1]{\relax}  % makes a "comment" (not expanded)
\newcommand{\Heading}[1]{\par\noindent{\bf#1}\nobreak}
\newcommand{\Tail}[1]{\nobreak\par\noindent{\bf#1}}
\newcommand{\QED}{\vrule height 1.4ex width 1.0ex depth -.1ex\ \vspace{.3in}} % square box
\newcommand{\arc}[1]{\mbox{$\stackrel{\frown}{#1}$}}
\newcommand{\lyne}[1]{\mbox{$\stackrel{\leftrightarrow}{#1}$}}
\newcommand{\ray}[1]{\mbox{$\vec{#1}$}}          
\newcommand{\seg}[1]{\mbox{$\overline{#1}$}}
\newcommand{\tab}{\hspace*{.2in}}
\newcommand{\se}{\mbox{$_{\epsilon}$}}  % subscript epsilon
\newcommand{\ie}{\mbox{i.e.}}
\newcommand{\eg}{\mbox{e.\ g.\ }}
\newcommand{\figg}[3]{\begin{figure}[htbp]\vspace{#3}\caption{#2}\label{#1}\end{figure}}
\newcommand{\be}{\begin{equation}}
\newcommand{\ee}{\end{equation}}
\newcommand{\prf}{\noindent{{\bf Proof}:\ \ \ }}
\newcommand{\choice}[2]{\mbox{\footnotesize{$\left( \begin{array}{c} #1 \\ #2 \end{array} \right)$}}}      
\newcommand{\scriptchoice}[2]{\mbox{\scriptsize{$\left( \begin{array}{c} #1 \\ #2 \end{array} \right)$}}}
\newcommand{\tinychoice}[2]{\mbox{\tiny{$\left( \begin{array}{c} #1 \\ #2 \end{array} \right)$}}}
\newcommand{\ddt}{\frac{\partial}{\partial t}}
\newcommand{\Sn}[1]{\mbox{{\bf S}$^{#1}$}}
\newcommand{\calP}[1]{\mbox{{\bf {\cal P}}$^{#1}$}}

\newtheorem{theorem}{Theorem}	
\newtheorem{rmk}[theorem]{Remark}
\newtheorem{example}[theorem]{Example}
\newtheorem{conjecture}[theorem]{Conjecture}
\newtheorem{claim}[theorem]{Claim}
\newtheorem{notation}[theorem]{Notation}
\newtheorem{lemma}[theorem]{Lemma}
\newtheorem{corollary}[theorem]{Corollary}
\newtheorem{defn2}[theorem]{Definition}
\newtheorem{observation}[theorem]{Observation}
\newtheorem{implementation}[theorem]{Implementation note}

% \font\timesr10
% \newfont{\timesroman}{timesr10}
% \timesroman


\newif\iftalk
\talkfalse
\newif\ifjournal
\journalfalse

\markright{\hfill \today \hfill}
\pagestyle{myheadings}

\setlength{\oddsidemargin}{0pt}
\setlength{\topmargin}{0in}
\setlength{\textheight}{8.6in}
\setlength{\textwidth}{6.875in}
\setlength{\columnsep}{5mm}

% -----------------------------------------------------------------------------

\title{A generalization of the cross product}
% for higher dimensions (this is implied by generalization)
% A generalized cross product for rotation and intersection in higher dimensions
% \\[5pt]{\small UAB Technical Report\\ March 2005}}
\author{John K. Johnstone\thanks{Supported in part by the National Science Foundation
        under grant CCR0203586.}}
\begin{document}
\maketitle

% Candidate for Eurographics (perhaps mini-session).
% To be submitted to Journal of Graphics Tools (jgt).

BE MORE PUNCHY

\iftalk
What do you get when you cross a mountain climber and a mosquito?
Nothing, you can't cross a scaler and a vector.
\fi

\begin{abstract}
We develop a generalization of the cross product that works effectively in arbitrary dimensions.
We establish its connection to volume, which leads to a generalization of the Pythagorean Theorem.
We apply the generalized cross product to the construction of normals and rotation matrices, and
the intersection of hyperplanes, all in arbitrary dimensions.
% We also consider a further generalization of the new cross product.

The n-volume defined by n vectors in n-space can be computed from the determinant of these vectors.
The n-volume defined by n+1 points in n-space can also be computed from the determinant,
by padding the matrix with 1's.
But how do you compute the (n-1)-volume defined by n vectors in n-space, such as the area
of the parallelogram spanned by two vectors in 3-space?  
This straddles dimensions.
The solution is to express the volume in terms of the volumes of its projections onto the
coordinate hyperplanes (which are fully dimensioned).
It turns out that these volumes enjoy a Pythagorean relationship.
It also turns out that this (n-1)-volume is encoded by the length of the generalized cross
product vector.

The generalized cross product can be further generalized(!) to work with $n-k$ vectors in n-space
for $k>1$.

Why do the projected volumes and volume form a Pythagorean tuple?
The Euclidean proof of the Pythagorean Theorem shows why for the degenerate case of length,
but the general case is much more difficult.

The Pythagorean relationship between volumes is a positive 
generalization of the Pythagorean Theorem,
in contrast to the negative generalization encoded by Fermat's Last Theorem
(there are no solutions to $x^n + y^n = z^n$ for $n>2$).
\end{abstract}

% Although rotation in 3-space is well understood in computer graphics, rotation
% in higher dimensions is less so.
% Higher-dimensional rotation is necessary, for example,
% for the treatment of quaternion splines, which reside in 4-space.

% -----------------------------------------------------------------------------

\section{Introduction}

% The cross product of two vectors in 3-space defines everything about the
% parallelogram spanned by the two vectors.
% It defines the normal of the parallelogram ...
% It defines the area of the parallelogram ...

The cross product of two vectors is a useful tool in 3-space, especially for computer graphics.
It is used to build surface normals for lighting, to build rotation matrices, to intersect planes, 
and 
% \cite{normal computation in jgt}
many other geometric tasks.
Unfortunately, unlike the inner (dot) product and outer product,
the cross product is only defined in 3-space.
As Serge Lang cautions at the head of the section on the cross product:
``This section applies only in 3-space!'' \cite{lang71}. % p. 32
    % However, all of these tasks are needed in higher dimensions as well: is there
    % a generalization of the cross product that will have the same utility in higher dimensions?
    % And how exactly will it be used?
This paper develops a generalization of the cross product to arbitrary dimensions.

We establish some of the key properties of this new cross product,
and show how it can be applied to build normals and rotations and intersect planes
in higher dimensions.
{\bf We also consider an application to quaternion splines, an example of a higher dimensional
problem that benefits from this new tool. (really?)}

\iftalk
3 vector products: inner product, outer product, cross product
\fi

The cross product of two vectors in 3-space is a vector in 3-space, defined as follows:
\[
  A \times B = (a_1,a_2,a_3) \times (b_1,b_2,b_3) 
             = (a_2b_3 - a_3b_2, a_3b_1 - a_1b_3, a_1b_2 - a_2b_1).
\]
It has two major properties , one concerning orthogonality and the other area:
\begin{equation}
A \times B \mbox{ is orthogonal to } A \mbox{ and } B
\end{equation}
unless $A$ and $B$ are linearly dependent, in which case it is the zero vector; and 
\begin{equation}
\label{eq:area}
\|A \times B\| = \mbox{area of parallelogram spanned by } A \mbox{ and } B
\end{equation}
These two properties have many important applications, which we explore below.

\begin{itemize}
\item definition of gcp (this is the useful result)
\item proof that it is orthogonal
\item proof that its length is volume (and thus a generalization of Pythagorean Theorem);
      this is what gives the result its meat.
\item application to rotation matrix (EASY, but shows how it immediately applies)
\item application to quaternion splines (I think this is to do with the rotation of 
      input quaternions away from the pole)
\end{itemize}

\clearpage

\subsection{The cross product in 3-space}

% The most notable feature of the cross product is that 
Another way of posing the orthogonality property is as follows:
\begin{itemize}
\item The cross product of two vectors is the normal of the plane spanned by these two vectors.
\end{itemize}
% The second major property of the cross product is that its length 
% is the area of the parallelogram spanned by the two vectors:
The area property is a direct result of the relationship of the cross product with
$\sin\theta$, $\|A \times B\| = \|A\| \|B\| |\sin\theta|$, 
where $\theta$ is the angle between $A$ and $B$ \cite{lang71}. % p. 34
The area property in 3-space can be translated into a useful area property
in 2-space.
Consider a triangle $\bigtriangleup P_1P_2P_3$ in 2-space, 
and two of its legs $P_2 - P_1$ and $P_3 - P_2$.
\begin{itemize}
\item The z-component of $(P_2 - P_1, 0) \times (P_3 - P_2, 0)$ is twice the signed area
of the triangle $\bigtriangleup P_1P_2P_3$.
\end{itemize}
This is a simple consequence of the parallelogram area property;
its main advantage is that it doesn't require a norm calculation (and therefore no square root,
only addition and multiplication).
%
The signage of the triangle area allows it to be used in the computation of polygonal area:
the area of a polygon in 2-space is the sum of the areas of all the triangles formed by 
connecting the origin and a (consistently oriented) leg of the polygon \cite{orourke94}.
The sign of the triangle area is also used to distinguish a left turn from a right turn:
the turn at $P_2$ in $P_1P_2P_3$ is left if and only if the signed area of the triangle is positive
\cite{orourke94}.% p. 31
This result is used to maintain a consistent orientation for the triangles in a triangular mesh,
as well as detecting concavities in a polygon (as a right turn).

A fourth, less central, property of the cross product progresses from area to volume through 
a combination with the inner product:
\begin{itemize}
\item $A \cdot B \times C =$ volume of parallelepiped spanned by $A$, $B$ and $C$.
\end{itemize}
This is called the triple scalar product \cite{wrede72}.
% triple scalar product, p. 104, Wrede
% triple scalar product, p. 50, Goodbody
% triple product: Orourke, p. 46

These properties establish the importance of the cross product in the geometry of 3-space
and its direct relation to normals, areas (both in 2-space and 3-space),
triangle orientations, and volumes.
%

\subsection{The need for a generalization}

% mention later after introduction of determinant: 
% volume spanned by $d-1$ vectors in $d$-space, rather than the 
% determinant's volume spanned by $d$ vectors
We would like to define a generalized cross product with the following properties:
\begin{itemize}
\item it works with vectors in $d$-space,
\item it creates a vector from $d-1$ input vectors,
\item the created vector is the normal of the hyperplane spanned by the input vectors
      (if they are linearly independent),
\item the length of the created vector is the volume of the parallelepiped spanned by the
      input vectors.
\end{itemize}
These should yield the following secondary properties:
\begin{itemize}
\item the signed volume and the orientation of a simplex in $d-1$-space may be defined 
  using the generalized cross product,
\item the dot product with another vector yields the volume spanned by these $d$ vectors,
      generalizing the triple scalar product, and
\item further generalizations of the cross product work with fewer vectors and likewise
      define vectors whose length is the volumes of the spanned parallelepipeds.
\end{itemize}

The generalized cross product should have applications to the construction of rotation
matrices in higher dimensions, and the intersection of planes.

\clearpage

% -------------------------------------------------------------------------------------------

\section{Determinants and volume}

The classical tool to compute volumes is the determinant,
so we should study it since volume computation is one of our goals.
The determinant of $d$ vectors in $d$-space defines the volume 
of the parallelepiped spanned by these vectors \cite{strang88}. % p. 234
\Comment{
Let $A_1,\ldots,A_d \in \Re^d$.
\[
\left|
\begin{array}{c}
A_1 \\
\ldots\\
A_d
\end{array}
\right|
 = \mbox{volume (parallelepiped spanned by $A_1,\ldots,A_d$)}
\]
}
The equivalence of determinant and volume uses two observations:
if the box is orthogonal, the equivalence is easily established, and 
if the box is not orthogonal, it can be transformed to an orthogonal box by a
Gram-Schmidt orthogonalization process that preserves determinant
(the typical operation being the subtraction of a multiple of one vector/row from
another vector/row to realize a projection onto a subspace) \cite{strang88}.

Incidentally, the triple scalar product $A \cdot B \times C$ 
can be shown equivalent to the determinant $|A B C|$ \cite{wrede72}, % p. 104
which explains its connection to volume.

In anticipation of simplicial volumes, we mention here a subtly different formula
for the volume of a simplex in $d$-space \cite[p. 27]{orourke94}:
\[
\left|
\begin{array}{cc}
A_1 & 1 \\
\ldots \\
A_{d+1} & 1 
\end{array}
\right|
    = d! \mbox{ * volume of simplex with vertices $A_1,\ldots,A_{d+1}$} 
\]

For example, the area of the triangle with vertices $(x_1,y_1)$, $(x_2,y_2)$, and
$(x_3,y_3)$ is 
\[
\frac{1}{2}
\left|
\begin{array}{ccc}
x_1 & y_1 & 1 \\
x_2 & y_2 & 1 \\
x_3 & y_3 & 1 
\end{array}
\right|
\]

\section{Related literature}

other attempts to combine more than two vectors: p. 51 of Goodbody, Cartesian Tensors
% A.M. Goodbody, Cartesian Tensors, Ellis Horwood (Chichester), 1982.

\clearpage

\section{A generalization of the cross product}
\label{sec:gcp}

Our preferred definition of the cross product of two 3-vectors 
$A = (a_1,a_2,a_3)$ and $B = (b_1,b_2,b_3)$
is in terms of a determinant:
\[
A \times B = 
\left|
\begin{array}{ccc}
e_1 & e_2 & e_3 \\
& A \\
& B
\end{array}
\right|
=
\left|
\begin{array}{ccc}
e_1 & e_2 & e_3 \\
a_1 & a_2 & a_3 \\
b_1 & b_2 & b_3
\end{array}
\right|
= (a_2b_3 - a_3b_2, a_3b_1 - a_1b_3, a_1b_2 - a_2b_1)
\]
where $e_1 = (1,0,0)$, $e_2 = (0,1,0)$ and $e_3 = (0,0,1)$ are the 
unit coordinate vectors.
This suggests the following generalization.

\begin{defn2}
The {\bf generalized cross product} of $d-1$ vectors in $d$-space
$\{A_i = (a_{i,1},\ldots,a_{i,d})\}_{i=1}^{d-1}$ is:
\begin{equation}
\label{eqn:gcp}
cross (A_1,\ldots,A_{d-1}) = 
\left|
\begin{array}{ccc}
e_1 & \ldots & e_d \\
    & A_1 \\
    & \vdots \\
    & A_{d-1}
\end{array}
\right|
=
\left|
\begin{array}{ccc}
e_1 & \ldots & e_d \\
a_{1,1} & \ldots & a_{1,d} \\
\vdots \\
a_{d-1,1} & \ldots & a_{d-1,d} 
\end{array}
\right|
\end{equation}
We shall use the shorthand {\bf gcp} for the generalized cross product.
\end{defn2}

\clearpage

\subsection{Orthogonality}

We first want to show that, like the cross product in 3-space, 
the gcp is orthogonal to all of its input vectors:
\[
\mbox{cross}(A_1,\ldots,A_{d-1}) \cdot A_i = 0 \hspace{.2in} \mbox{ for } i=1,\ldots,d-1.
\]
Letting 
\[
A = \left(
\begin{array}{c}
x \\ A_1 \\ \ldots \\ A_{d-1}
\end{array}
\right)
\in \Re^{dxd}
\]
be any matrix whose $i^{th}$ row is $A_{i-1}$ for $i=2,\ldots,d$
(the first row is arbitrary),
the generalized cross product may be reinterpreted as 
\[ 
(subdet(A,1,1), -subdet(A,1,2), \ldots, (-1)^{1+d} subdet(A,1,d))
\]
where $subdet(A,i,j)$ is the minor of $A$ with the $i^{th}$ row and $j^{th}$
column removed.
Thus, the dot product of the vector $A_i$ ($1 \leq i \leq d-1$) 
with the generalized cross product is:
\[
a_{i,1} subdet(A,1,1) - a_{i,2} subdet(A,1,2) + \ldots + (-1)^{1+d} a_{i,d} subdet(A,1,d))
\]
If we now choose the matrix $A$ to have $A_i$ as its first row,
this is an expansion of the determinant of $A$:
\[
\left|
\begin{array}{c}
A
\end{array}
\right|
=
\left|
\begin{array}{c}
A_i \\ A_1 \\ \ldots \\ A_{d-1}
\end{array}
\right|
\]
This determinant vanishes, since the first row is identical to the $(i+1)$st row.
That is, 
\[
A_i \cdot \mbox{cross}(A_1,\ldots,A_{d-1})
=
\left|
\begin{array}{c}
A_i \\ A_1 \\ \ldots \\ A_{d-1}
\end{array}
\right| 
= 0
\]
which proves our result:
the generalized cross product $\mbox{cross}(A_1,\ldots,A_{d-1})$ 
is orthogonal to all of its defining vectors $A_i$.

% No it doesn't: distance formula for cross product involves sin of angle.
% The generalized cross product preserves unit vectors, like the cross product.
% That is, if $A_1,\ldots,A_{d-1}$ are unit vectors, 

The generalized cross product of $A_1,\ldots,A_{d-1}$ becomes less robust
as the matrix (\ref{eqn:gcp}) becomes more singular, or equivalently
as the defining vectors $A_i$ become closer to linearly dependent.
When the choice of $A_i$ is flexible, this robustness property should
be observed in choosing the $A_i$.

\clearpage

\subsection{The connection to volume}
% Length is volume

We want to show that the length of the generalized cross product has meaning.
In particular, we want to show that the length of the gcp is
the volume of the parallelepiped spanned by its input vectors:
\begin{equation}
\label{eq:volume}
\| gcp(A_1,\ldots,A_{d-1}) \| = \mbox{ volume of parallelepiped spanned by } A_1, \ldots, A_{d-1}
\end{equation}
This is known for the cross product in 3-space, as we saw in (\ref{eq:area}).

Consider the known 3-dimensional case. [COULD DO IN GENERAL RIGHT OFF THE BAT].
Expanding the determinant expression for the cross product, we get:
\begin{eqnarray}
\label{eq:gcpdet}
A \times B & = & 
\left|
\begin{array}{ccc}
e_1 & e_2 & e_3 \\
    & A \\
    & B
\end{array}
\right|\\
& = &
e_1 
\left| 
\begin{array}{cc} 
a_2 & a_3 \\
b_2 & b_3
\end{array}
\right|
+ e_2 
\left| 
\begin{array}{cc} 
a_1 & a_3 \\
b_1 & b_3
\end{array}
\right|
+
e_3 
\left| 
\begin{array}{cc} 
a_1 & a_2 \\
b_1 & b_2
\end{array}
\right| \\
& = & ( 
\left| 
\begin{array}{cc} 
a_2 & a_3 \\
b_2 & b_3
\end{array}
\right|
, 
\left| 
\begin{array}{cc} 
a_1 & a_3 \\
b_1 & b_3
\end{array}
\right|
, 
\left| 
\begin{array}{cc} 
a_1 & a_2 \\
b_1 & b_2
\end{array}
\right|
)\\
& = & (
\label{eq:gcpproj}
\left|
\begin{array}{c}
\mbox{proj}_{x_1=0}(A)\\
\mbox{proj}_{x_1=0}(B)
\end{array}
\right|
,
\left|
\begin{array}{c}
\mbox{proj}_{x_2=0}(A)\\
\mbox{proj}_{x_2=0}(B)
\end{array}
\right|
,
\left|
\begin{array}{c}
\mbox{proj}_{x_3=0}(A)\\
\mbox{proj}_{x_3=0}(B)
\end{array}
\right|
)\\
& = &
(\mbox{vol} (\mbox{par} (\mbox{proj}_x (A_1,A_2))),\ \ 
\mbox{vol} (\mbox{par} (\mbox{proj}_y (A_1,A_2))),\ \ 
\mbox{vol} (\mbox{par} (\mbox{proj}_z (A_1,A_2))))
\end{eqnarray}
where $\mbox{par}(A_1,\ldots,A_n)$ 
is shorthand for the parallelepiped spanned by $A_1,\ldots,A_n$.

This gives a very interesting interpretation to the coordinates of the cross product:
% Since determinants encode volume,
they are the volumes of the parallelepipeds spanned
by the input vectors projected onto the coordinate planes $x_i=0$.
% each of the components of (\ref{eq:gcpproj}) may be interpreted as the volume
% of a parallelepiped spanned by its rows.
% Combining these two facts, the components of the cross product vector are the
% parallelepiped 2-volumes (areas) spanned by the three projection of the input vectors
% onto the coordinate hyperplanes $x_i=0$.
From (\ref{eq:area}), we know that 
\[
   \| A \times B \|^2 = \mbox{vol}^2 (\mbox{par} (A_1,A_2))
\]
But we have just shown that
\[
   \| A \times B \|^2 = 
   \mbox{vol}^2 (\mbox{par} (\mbox{proj}_x (A_1,A_2))) + 
   \mbox{vol}^2 (\mbox{par} (\mbox{proj}_y (A_1,A_2))) + 
   \mbox{vol}^2 (\mbox{par} (\mbox{proj}_z (A_1,A_2))) 
\]
Therefore,
% This suggests a reinterpretation of (\ref{eq:area}) and (\ref{eq:volume})
% as a sum of squares relationship:
\[
   \mbox{vol}^2 (\mbox{par} (A_1,A_2)) = 
   \mbox{vol}^2 (\mbox{par} (\mbox{proj}_x (A_1,A_2))) + 
   \mbox{vol}^2 (\mbox{par} (\mbox{proj}_y (A_1,A_2))) + 
   \mbox{vol}^2 (\mbox{par} (\mbox{proj}_z (A_1,A_2))) 
\]
% the square of the area of the parallelogram spanned by two vectors (in 3-space)
% is the sum of the squares of the areas of its projections (onto coordinate hyperplanes).
This is a generalization of the Pythagorean theorem: there is a sum of squares relationship
now between volumes rather than lengths.
Notice that the Pythagorean theorem can be interpreted as a result on the sum of squares
relationship between a line and its projections.
The generalization of this theorem suggested by the above analysis is from a line to 
a parallelopiped and from length to volume, yielding a sum of squares relationship between
the volume of a parallelopiped and the volumes of its projections onto the coordinate planes.
We would like to state this result in its highest generality.
Therefore, we would like to prove the equivalence between the length of the generalized cross
product and the volume of the parallelopiped spanned by its constituent vectors.
{\bf We need to directly prove the sum of squares relationship between a volume
and its projections, which will indirectly yield the equivalence of cross product length and
volume.}

% Proving that the length of the gcp is the volume of the parallelepiped spanned
% by its input vectors is equivalent to proving that the volume of a parallelepiped and
% the volumes of its projections enjoy a sum of squares relationship.

\subsubsection{A generalization of the Pythagorean theorem}

Notice that the Pythagorean theorem is a special case of this result, in 2-space.
The vector $A_1$ in 2-space defines a right triangle with hypotenuse $A_1$,
and the Pythagorean theorem can be restated as follows:
\[
   \mbox{vol}^2 (\mbox{par} (A_1)) = 
   \mbox{vol}^2 (\mbox{par} (\mbox{proj}_x (A_1))) + 
   \mbox{vol}^2 (\mbox{par} (\mbox{proj}_y (A_1)))
\]
The 1-volume of the parallelepiped spanned by a single vector is its length.
The Pythagorean theorem states that the square of the 1-volume of a vector 
is equal to the sum of squares of the 1-volumes of its projections.
{\em Picture of a 2-vector and its projections.}
Therefore, the proposed length property of the generalized cross product is 
a generalization of the Pythagorean theorem to higher dimensions.
Equivalently, the volumes of a parallelepiped and its projections define a
Pythagorean tuple \cite{}.

% The Pythagorean theorem relates geometry to number theory.

\subsubsection{Proving it (START HERE)}

{\em The drafting literature probably says something about the relationship of the
area of an original shape to the area of its projections onto coordinate planes
(side,front,above views). Following this proof, we may be able to generalize
to volumes in higher dimensions.
This is the remaining fly in the ointment.  Discuss with other mathematicians and colleagues.  Then this paper is done.}

What happens to parallelepiped volume under projection?

NOT CRUCIAL TO PROVE IN THE SENSE THAT WE WANT TO EMPHASIZE THE APPLICATIONS.
A PROOF WOULD BE THE ICING (AND UNDOUBTEDLY A SEPARATE PAPER).

\clearpage

\subsection{Signed simplicial volume}

The original cross product was used to compute simplicial 
area (hypervolume) in a lower dimension, using zero settings for the final dimension and
then applying the length of the cross product.
This is valid since the simplex in d-space is defined by d vectors and the cross product
only deals with $d-1$ vectors, forcing a drop in dimension to handle the simplex.
The generalization of this idea is to compute the volume of a simplex in $d-1$-space
by computing the length of the generalized cross product of the $d-1$ defining vectors
of the simplex, all lifted into $d$-space by setting the $d$th component to 0.
{\bf OPEN PROBLEM}.
That's it, except for the fact that we are computing parallelepiped volumes rather than
simplicial volumes, so the simplicial volume is some constant factor of the result,
with the constant depending on dimension (undoubtedly $1/d!$).

How does this compare to adding a row of 1's instead of a row of coordinate vectors:
get the same result about volume, but the former yields a scalar and the latter a vector,
so the latter must be used to define hyperplane normals.

\subsection{Orientation}

Orientation = Left-handed or right-handed coordinate system (see Strang p. 234).

\subsection{A generalization of the triple scalar product}

The absorption of a dot product into the generalized cross product
to yield a higher dimensional volume (a determinant) should be straightforward.
Follow the example of the scalar triple product proof of p. 104 of Wrede.
{\bf STILL OPEN PROBLEM}.

\subsection{Further generalizations of the cross product}

This is where we continue the addition of rows of canonical vectors $e_i$
and see what happens to the determinant.
{\bf This could be challenging, but is one of the more interesting developments.}
{\bf KEY OPEN PROBLEM}.

\clearpage

\section{Application 1: Higher-dimensional rotation of a vector to a coordinate axis}

% Rotation is a different beast in 4-space
% The weaponry of rotation is intriguing, since we are working in a higher dimension.

The generalized cross product may be used to construct a rotation matrix
that sends an arbitrary vector to an arbitrary coordinate axis in n-space.
The ability to find a vector orthogonal to a collection of other vectors is crucial
in the construction of rotation matrices.
The construction proceeds as follows.
A rotation matrix is built from rows that are mutually orthogonal,
since a rotation matrix is orthogonal.
Under the rotation, 
the rows of the rotation matrix are mapped to the coordinate axes:
\[
\left(
\begin{array}{c}
R_1^t \\ R_2^t \\ \ldots \\ R_n^t
\end{array}
\right)
R_i
= 
\left(
\begin{array}{c}
R_1^t R_i \\ \ldots \\ R_n^t R_i
\end{array}
\right)
=
\left(
\begin{array}{c}
0 \\ \cdots \\ 0 \\ 1 \\ 0 \\ \cdots \\ 0
\end{array}
\right)
=
e_i
\]
because of the orthogonality of the rows.
% This is a direct product of the Kronecker-delta behaviour of the dot product 
% of the rows, reflecting orthonormality.
Consider a matrix that rotates the arbitrary\footnote{Assume without loss of 
      generality that $v$ is not a coordinate axis.
      We can deal with the $d$ coordinate axes easily.} 
unit vector $v \in \Re^n$ to the coordinate axis $e_i$.
A candidate\footnote{There are an infinite number
  of rotations that achieve the desired result, 
  since there are remaining degrees of freedom in rotating
  the frame about the coordinate axis.} 
is the matrix whose $i^{th}$ row is $v$ and whose other rows are
\begin{eqnarray}
\label{eqn:R1}
R_1 = cross (v,e_1,         \ldots,e_{d-2}) \\
R_2 = cross (v,R_1,e_2,     \ldots,e_{d-2}) \\
R_3 = cross (v,R_1,R_2,e_3, \ldots,e_{d-2}) \\
\ldots \hfill \\
\label{eqn:Rlast}
R_{d-1} = cross (v,R_1,\ldots,R_{d-2})
\end{eqnarray}
all normalized to unit vectors.
These rows are mutually orthogonal, since $v$ is included in every cross product
and a row is introduced into the cross product as soon as it is generated.
This matrix rotates $v$ to $e_i$ (i.e., $Mv = e_i$), 
since $v$ is orthogonal to all rows except the $i^{th}$.
% Note that we start with $v$ and the coordinate axes, then replace each 
% coordinate axis by a row as soon as the row is computed.
%          need n-1 other rows (other than v)
%	  each of these rows needs n-1 vectors to build a generalized cross product
%	      - one of these vectors is always v
%	      - the last vector is the cross product of v and the other d-2 vectors
%	      - one of the vectors can always be e_i
%	  row1 = v
%	  row2 = v x e_1 x ... x e_{d-2}
%	  row3 = v x row2 x e_2 x ... x e_{d-2}
%	  ...
%	  rown = v x row2 x ... x row_{d-1}

The choice of coordinate axes in the construction (\ref{eqn:R1})-(\ref{eqn:Rlast})
of the rotation to a coordinate axis is flexible\footnote{Actually, we 
  can use any vectors not equal to $v$, although coordinate axes are the 
  natural choice.}
and we should take care to make the most robust choice (see the end of 
Section~\ref{sec:gcp}).
Therefore, in (\ref{eqn:R1}), 
the $d-2$ furthest coordinate axes from $v$ should be used.
  % \footnote{Whether or not $e_i$ is included in the furthest coordinate axes 
  % is irrelevant. We are simply looking for orthogonal vectors to $v$.}
This has the added benefit of guaranteeing that the chosen coordinate axes
do not agree with $v$, which avoids a preprocessing step.\footnote{{\bf See
% recall that we are working in d-space, with d coordinate axes.
    Vec4::rotToCoord where we check for this degenerate case 
    when the cross product would be undefined.}}
For added robustness, the newly introduced row should replace the remaining 
coordinate axis closest to it, to avoid dependence between the $R_j$ and $e_j$.
Under this improved scheme, a matrix that rotates $v$ to $e_i$ is the matrix
whose $i^{th}$ row is $v$ and whose other rows are (in any order)
\begin{eqnarray}
R_1 = unit(cross (v,e_{\pi(1)},         \ldots,e_{\pi(d-2)})) \\
R_2 = unit(cross (v,R_1,e_{\pi(2)},     \ldots,e_{\pi(d-2)})) \\
R_3 = unit(cross (v,R_1,R_2,e_{\pi(3)}, \ldots,e_{\pi(d-2)})) \\
\ldots \hfill \\
R_{d-1} = unit(cross (v,R_1,\ldots,R_{d-2}))
\end{eqnarray}
where $unit(v)$ is the unit vector associated with the vector $v$,
$e_{\pi(1)},\ldots,e_{\pi(d-2)})$ are the $d-2$ coordinate axes furthest from $v$,
and $e_{\pi(i)}$ is the remaining coordinate axis closest to $R_i$.

Since rotations should have determinant +1, % e.g., Cartan, Theory of Spinors, p.9
the rows $R_i$ should be permuted to make this so.

\subsection{More general rotations}

Once it is known how to rotate a vector to a coordinate axis, more general
rotations quickly follow.
A rotation of the coordinate axis $e_i$ into an arbitrary vector $v$
is simply the transpose of a rotation of $v$ into $e_i$.
A rotation of the arbitrary vector $v$ to the arbitrary vector $w$ 
(or equivalently a rotation of the point $v$ to the point $w$)
is the product $AB$ of a rotation $B$ of $v$ to a coordinate axis $e_i$ 
and a rotation $A$ of $e_i$ to $w$.

\subsection{A review of rotation}

THIS IS AN OPTIONAL SECTION.
Why do we need a review of rotation?  To put the higher-dimensional rotation section in context, I guess.

A REVIEW OF ROTATION: SO(3), SO(n), Givens, Jacobi, cross product, determinants
A REVIEW OF ROCOCO (and the construction of a rotation matrix using it 
and cross product).

This leads to a generalization of SO(3), the group of rotation matrices in 3-space.
Notice that Givens (or Jacobi) rotations \cite{golubvanLoan} are also 
generalizations
of rotation matrices, but are restricted to rotations about the coordinate axes;
we want a generalization of rotations about arbitrary vectors.

\clearpage

\section{Application 2: Hyperplane intersection}

The intersection of $d-1$ hyperplanes defined by $d-1$ normals is parallel 
to the generalized cross product of these normals.
(This is used in the construction of bisectors in d-space.)

For example, the intersection of two planes in 3-space with normals N1 and N2 
is the line in direction $N1 \times N2$.


\section{Hyperplane definition: normals}

A hyperplane is defined by a point and a normal.
In general, $d-1$ vectors in $d$-space span a hyperplane.
This hyperplane can be defined using the generalized cross product,
since the gcp of the spanning vectors yields the hyperplane normal.
The spanning vectors might come from partial derivatives of a surface
or from polyhedral edges.

\section{Conclusions and future work}

Compare to Clifford algebra, which rotates about (n-2)-dimensional 'axis' in n-space
and expresses rotation in terms of reflections (see Penrose).

Compare to tensors.

\bibliographystyle{plain}
\begin{thebibliography}{99}

\bibitem{lang71}
Lang, S. (1971) Linear Algebra.
2nd edition, Addison-Wesley (Reading, MA).

\bibitem{orourke94}
O'Rourke, J. (1994)
Computational Geometry in C.
Cambridge University Press (New York).

\bibitem{strang88}
Strang, G. (1988) Linear Algebra and its Applications.
3rd edition, Harcourt Brace Jovanovich (San Diego).

\bibitem{wrede72}
Wrede, R. (1972) Introduction to Vector and Tensor Analysis.
Dover (New York).

\end{thebibliography}

{\bf Compare to qspline.c/'PerturbInput'.}

\end{document}
