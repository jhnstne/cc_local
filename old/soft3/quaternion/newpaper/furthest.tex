\documentclass[12pt]{article} 
\usepackage{times}
\usepackage[pdftex]{graphicx}
\input{header}

\setlength{\oddsidemargin}{0pt}
\setlength{\topmargin}{0in}
\setlength{\textheight}{8.6in}
\setlength{\textwidth}{6.875in}
\setlength{\columnsep}{5mm}
\markright{\hfill \today \hfill}
\pagestyle{myheadings}

% -----------------------------------------------------------------------------

\title{Finding an empty point on \Sn{3}\\[5pt]{\small UAB Technical Report\\
       February-March 2005}}
\author{John K. Johnstone\thanks{Supported in part by the National Science Foundation
        under grant CCR0203586.}\ and James Williams}
\begin{document}
\maketitle

\begin{abstract}
An algorithm to find the emptiest point, or the largest empty circle, on $S^3$.
That is, given a point sample on $S^3$, what is the point furthest from this point
sample?  This is a classical problem in the plane, where optimal(?) 
solutions have long been known.  It has also been studied on $S^2$ by Renka.
The $S^3$ problem is particularly interesting, since it combines a search on
a surface with a search in a 3-dimensional space.
\end{abstract}

\section{Related work}

Review the literature on empty points in 2-space, 3-space, on a surface.
Review the literature on Voronoi diagrams.
Note that $S^3$ is a 3-manifold, complicating matters.
An algorithm for points on $S^2$ is available from 
Renka(?).  
Explore 3D Voronoi diagrams for hints.

\section{Bisectors}

Discuss bisectors in Euclidean space and on the sphere. (See assignment I gave
to Ross.)

\section{Solutions}

The optimal solution is to find
the point in the middle of the largest empty region.
{\bf Janardan claims to have a solution to this in his LM (layered manufacturing) work.}
This is a generalization of the classical largest empty circle
problem for points in a plane \cite{shamos85}, 
and a similar technique using the Voronoi diagram may be used.\footnote{Renka \cite{renka}
  has developed an algorithm for the Voronoi diagram of points on the lower-dimensional
  $S^2$.}
Let $f(p)$ be the distance of $p \in \Sn{3}$ from the nearest quaternion.
(Note that distance must be measured on the surface.)
The maximum of $f(p)$ 
is attained at some vertex of the Voronoi diagram of the quaternions on \Sn{3},
which is the desired optimal empty point.
The Voronoi diagram must be built on the surface \Sn{3}, which 
is itself an interesting problem.
Luckily, only the Voronoi vertices need to be computed,
not the entire Voronoi diagram.
A Voronoi diagram is built out of point bisectors, and
the bisector of two points on \Sn{3}\ is a great circle of \Sn{3}.
A simplistic approach is to compute a superset of the Voronoi vertices
by intersecting all bisectors of two quaternions, and then 
choose the one furthest from all quaternions.
	% O(n choose 2) = O(n^2) circles; 
	% O(n^4) intersections
	% O(n^5) distance calculations, but can abort early on many
{\bf Implement this (good student project).}

One approach is to compute the Voronoi diagram of the point set and choose
one of the Voronoi vertices, but this requires quite a bit of work
and no algorithm has presently been developed for Voronoi diagrams 
of points on $S^3$.

An heuristic approach can be used to find a point of an
empty region, which is not guaranteed to find an empty point, 
but usually finds a very good empty point.
The idea is to move the data far from the pole by moving the best-fitting plane
of the data as far as possible from the pole.
In particular, the unit normal of the best-fitting hyperplane 
is chosen as the empty point (FIGURE OF DATASET, ITS BESTFITTING PLANE, AND ITS NORMAL AS POINT ON SPHERE).
Since the best-fitting plane is a good representative of most of the
points, the points tend to move far away from the pole.
Of course, outliers may be moved close to the pole.
This method is efficient, since the best-fitting plane may be computed
easily using principal component analysis:
the normal of the best-fitting plane is the eigenvector associated with the
smallest eigenvalue of the covariance matrix of the data \cite{ballard82}.
% The normal of the best-fitting plane is equivalent to the vector 
% such that the projection of the data points onto this vector has minimal variance,
% i.e., the vector a that minimizes a^t \Sigma a.

A simple randomized method is also worthy of mention.
Since there are so many empty regions on \Sn{3} (see below),
even a random choice of point on \Sn{3}\ is often effective.
A point on \Sn{3}\ is chosen at random until a point in an empty region 
is found.\footnote{To find a random point on \Sn{3}, find four random numbers 
  in [-1,1] defining a 4-vector, and then normalize this vector.}
   % A random number in [-1,1] can be generated in C/C++ using the 'rand' function
   % as follows:
   % (float) (rand() \% 32767) / 16383 - 1.0.
	% That is, we could randomly choose a point on \Sn{3}, test if it is
	% sufficiently far away from all data points, and if not then randomly choose
	% another point, and so on, until a valid point in an empty region is found.
	% The density of empty regions ensures the efficiency of this method.
		% The results in Section~\ref{sec:results} support the effectiveness
		% of a random choice of empty region \ref{}.
	% How many attempts before success? on all6: 1,1,1,1,4,56
This is a very simple method.
Care must be taken with repeatability.
If given the same data twice, we would like to choose the same point in an empty region
to rotate to the pole.
Fortunately, we can take advantage of the predictability 
of pseudo-random number generators.
If a pseudo-random number generator is used with the same seed
(such as the C 'rand' function), exactly the same sequence of `random' points will be 
generated each time it is called,
leading to the same quaternion spline if the method is repeated.

EVALUATE THESE THREE METHODS EXPERIMENTALLY.

\subsection{Existence of an empty region}
\label{sec:empty}

The very existence of an empty region is a question worthy of examination.
Is there an empty region whose center is at least $20^{\circ}$ from the
nearest quaternion?
It is theoretically possible that the data points are so densely packed
on the sphere that there is no empty region.
However, the requisite number of data points is huge.
	% large in theory, and
\Comment{
The 'surface area' of \Sn{3} is $2\pi^2$ \cite{kendall61}
while a region 30 degrees wide about a point has area $\pi/6$.
	% A point covers a region 30 degrees wide about it, which has
	% a surface area of $\pi/6$ on the sphere \Sn{3}\ of area $2\pi^2$
	% \cite{kendall61}.
Thus, theoretically only about $12\pi \approx 38$ points are needed to
cover the sphere, leaving no empty region.\footnote{Actually, quite
	a few more points are needed, since the discs about the 38 points
	do not abut perfectly.}
However, this requires a gap of at least 60 degrees between 
consecutive quaternions, a very large change of orientation.
The typical gap between consecutive quaternions for reasonable
motion control is about 20 degrees.
	% independent of the method used to construct the quaternion spline.
}
If there is a maximum gap of $20^{\circ}$ between a quaternion and its
closest neighbour, 434,783 points\footnote{The surface area 
	of \Sn{2}\ is $4\pi$, while the surface area
	of a region on \Sn{2}\ 10 degrees wide centered about a point is 
	$\int_{0}^{2\pi} \int_0^{\frac{\pi}{18}}  \sin \phi \ d\phi \ d\theta
	= .0000289$.} % \cite{lang79}
	% Lang, Calculus of several variables, p. 228
	% p. 208 for parameterization, p. 224-5 for ||dx/dphi x dx/dtheta||
	% integral of sin is -cos
are needed to cover \Sn{2}, and even more to cover \Sn{3}.
	% In the examples of Section~\ref{sec:results}, 
	% the distance between quaternions is about --- \ref{}.
Moreover, in order to leave no empty region, the quaternions
must be spread across the entire sphere.
It is far more common for the data to be restricted to a small region
of the sphere.
Thus, there will be a valid empty region on \Sn{3} in all but the most
pathological cases (over 400,000 quaternions uniformly scattered across \Sn{3}).

% We need a data point covering every empty region of the sphere.
% Consider the number of points needed to cover just the 2-sphere \Sn{2}.
% A point on \Sn{2}\ covers a region 30 degrees wide about it,
% which has the following surface area:
% \[
%  \int_{0}^{2\pi} \int_0^{\frac{\pi}{6}}  \sin \phi \ d\phi \ d\theta
%	= .841787
% \]
% using 10 degrees or \pi/18, area is only .0000289
% which leads to 431,073 points.
% Therefore, for data sets of size 15,
% it is possible to find a point $p$ with no data points within 30
% degrees of $p$.

\Comment{
% It is certainly more difficult to find an empty region for larger datasets, and
% may even be impossible in theory (although in practice, a huge dataset is required
% for $S^3$ to have no sufficiently large empty region).
One approach is to compute the Voronoi diagram of the point set and choose
one of the Voronoi vertices, but this requires quite a bit of work
and no algorithm has presently been developed for Voronoi diagrams 
of points on $S^3$.
However, empty regions are easy to find among small point sets.
}

\subsection{A divide and conquer technique for empty point (Building the quaternion spline up from smaller pieces)}
\label{sec:divide}

A simpler technique for finding an empty point is available if the dataset is small,
or if we can divide the dataset up into subsets and find an empty point for each subset
rather than an empty point for the entire set.
This approach is particularly important for very large datasets,
since empty points may not even exist for huge datasets.
(That is, the emptiest point is still too close to the data for our purposes.)

{\bf Adjust the following writeup for the purity of finding empty points, not embedded within
a quaternion spline paper.}
A robust and simple method may be designed if we are willing to build many
quaternion splines independently, each using only a subset of the quaternions.
Notice that, if the dataset is partitioned into subsets $\{q_i\}_{i=0}^n$ such that
the angle between $q_0$ and every quaternion $q_i$, $i \geq 1$ is less than $\theta$
degrees (i.e., every quaternion lies in a disk of radius $\theta$ about $q_0$),
then every quaternion $q_i$ of this dataset lies at a distance of at least 180 - $\theta$
degrees from $q_0$'s antipodal point.
This means that it is simple to find an empty point:
$q_0$'s antipodal point is an empty point of a subset $\{q_i\}_{i=0}^n$ that varies
by at most 160 (150?) degrees from $q_0$.\footnote{We measure distance
  from $q_0$ so that the subset may be designed trivially.  Larger subsets
  might be built if an arbitrary quaternion was used as the pivot.}
This suggests a divide-and-conquer strategy in which the dataset is partitioned
into small sets, and a quaternion spline is built for each small set.
% assumes a maximal distance between consecutive samples
The advantage of this divide and conquer approach is that it is guaranteed yet simple.
The second and third approaches above are simple but not guaranteed.
The first approach above is guaranteed but not simple.

Here is the algorithm in a C-like syntax.
This algorithm is called recursively, but the opening call is qSpline (n, q, NIL, NIL).

IN LIGHT OF THE ALGORITHM IN THE NEXT SECTION (when in the quaternion spline paper), 
THIS ALGORITHM CAN BE SHORTENED TO DEAL WITH SUBSET DEFINITION.

\begin{verbatim}
void qSpline (int n, Quaternion *q, float openingDeriv[NDERIV][3], qSpline &QS)
{
  i=1;
  subset = {q_0};
  nSubset = 1;
  while (angle(q[0],q[i]) < 160)
   {
     subset = subset $\cup$ \{q_i\};
     i++; nSubset++;
   }
  rotate subset and derivatives so that -q[0] is moved to the pole (1,0,0,0) (or whatever the pole of the inverse map is)
  if (openingDeriv == NIL)
       design a quaternion spline S through this rotated subset
  else design a quaternion spline S through this rotated subset with these rotated opening derivatives
  
  rotate the quaternion spline back
  measure the end derivatives of S
  if (nSubset < n)
   {
    qSpline (n-nSubset, q + nSubset, end derivatives of S, QSnext)
    QS = S spliced with QSnext
   }
  if (n - nSubset == 1)
    split the last two subsets into equal sizes
}
\end{verbatim}

We assume that object orientations (quaternions) are sampled frequently enough
so that consecutive quaternions span less than 160 degrees
(so subsets are of nontrivial size).
Note that quaternions at 180 degrees represent the same orientation (Section~\ref{sec:quaternionTheory}),
so no pair of consecutive quaternions need to be more than 90 degrees apart:
if they are, the second quaternion may be replaced by its antipode.

This method does require some additional theoretical development,
since the quaternion splines for each subset will need to be spliced 
together smoothly.
In particular, we must consider the control of derivatives at the endpoints of a
quaternion spline.

\clearpage

\section{2nd document}

The emptiest point on $S^3$ is valuable in the design of a quaternion
spline for motion design.
The emptiest point on $S^3$ may be computed from Voronoi diagrams,
and Voronoi diagrams are built from intersection of bisectors.
Therefore, we want to understand bisectors and their intersections.
We actually don't need to build the full Voronoi diagram, 
so computing intersections of bisectors will be enough.
This is a paper and pencil exercise.  Once you understand it on paper,
we can implement (last question only).

\begin{itemize}
\item Compute the bisector of two points in 2-space.
\item Compute the bisector of two points in 3-space.
\item Compute the bisector of two points in 4-space.
\item Compute the bisector of two points on $S^1$.
\item Compute the bisector of two points on $S^2$.
\item Compute the bisector of two points on $S^3$.
\item Compute the intersection of bisectors in 2-space.
\item Compute the intersection of bisectors in 3-space.
\item Compute the intersection of bisectors in 4-space.
\item Compute the intersection of bisectors on $S^1$.
\item Compute the intersection of bisectors on $S^2$.
\item Compute the intersection of bisectors on $S^3$.
\end{itemize}

\clearpage

\section{Furthest points on the sphere}

April 19, 2005.

Given S, a finite set of points on a sphere, we are interested in computing
the furthest point P from S.
It is known that the furthest point is attained at a vertex of the 
Voronoi diagram of the points \cite[p. 251]{preparatashamos85},
since the distance function from the nearest point is downward-convex within 
a Voronoi cell.\footnote{We don't need to add intersections of the convex hull 
  with the Voronoi diagram to the search set,
  since we do not need to restrict the furthest point to the convex hull of the points,
  as the sphere is bounded, unlike Euclidean space.}
That is, if the furthest point lay strictly inside a Voronoi cell or on a Voronoi edge, 
its distance to the nearest point could be increased by moving towards the cell boundary 
or towards a Voronoi vertex, which is a contradiction.
This reduces the furthest point problem on the sphere to computing the Voronoi vertices.

The Voronoi diagram is defined by point bisectors.
The bisector of two points in 2-space is a line,
the bisector of two points in 3-space is a plane and, in general,
the bisector of two points in n-space is a hyperplane.
A hyperplane may be defined by a normal vector and a point on the hyperplane.
Of course, the bisector of $A$ and $B$ in n-space is defined by the normal vector $B-A$
and the point $\frac{A+B}{2}$.
[We shall identify the bisector of two points with the normal vector of the hyperplane.]

The bisector of two points on the sphere in 3-space is a great circle and, in general, 
the bisector of two points on the hypersphere $S^{n-1}$ in n-space is 
the intersection of the bisector of these two points in Euclidean space with $S^{n-1}$.
FIGURE OF TWO POINTS AND GREAT CIRCLE ON S2.
This is the intersection of a hyperplane through the origin
  % since it may also be interpreted as the
  % bisector of the angle subtended by the two points on the sphere.}
with the sphere, or a great hypercircle.\footnote{The
  hyperplane must pass through the origin, since the origin is clearly equidistant
  from the two points.}
Note that this means that the bisector of two points on the sphere is 
calculated by first finding the bisector in Euclidean space 
and then intersecting with the sphere.

A vertex of the Voronoi diagram is defined by the intersection of enough
bisectors to generate a point.
How many is this?
We have seen that a bisector may be identified with a hyperplane, 
whether in Euclidean space or on the sphere.
The intersection of $n$ hyperplanes in $n$-space yields a 0-dimensional 
point.\footnote{After all, each hyperplane (represented by a single linear equation) 
  removes one degree of freedom.}
The intersection of $n-1$ hyperplanes in $n$-space through the origin
yields a 1-dimensional line through the origin
which, when intersected with the hypersphere $S^{n-1}$, also yields a 0-dimensional point.
Therefore, a Voronoi vertex of the Voronoi diagram of a set of points in Euclidean space 
is built from the intersection of $n$ bisectors,
while a Voronoi vertex of the Voronoi diagram of a set of points on the sphere $S^{n-1}$
is built from the intersection of $n-1$ bisectors.

Rather than computing the intersection of great circles as
\[
(H_1 \cap S^{n-1}) \cap (H_2 \cap S^{n-1}) \ldots (H_{n-1} \cap S^{n-1})
\]
we compute the intersection of hyperplanes through
\[
(H_1 \cap H_2 \cap \ldots \cap H_{n-1}) \cap S^{n-1}.
\]

This suggests the following algorithm for computing the furthest point
of the pointset $S \subset S^{n-1}$ in $n$-space.
Rather than intersecting $n-1$ great circles, we intersect $n-1$ hyperplanes
and then intersect the resulting line with the sphere.

\begin{itemize}
\item For every pair of points from S, define the bisector 
      (with respect to Euclidean space, i.e., a hyperplane):
\[
B = \{\mbox{bisector}(P,Q) : P,Q \in S\}.
\]
\item For every subset of $n-1$ bisectors,
      find the intersection line of these $n-1$ bisectors:
\[
L = \{b_{i,1} \cap \cdots \cap b_{i,n-1}: b_{i,j} 
                     \mbox{ are distinct elements from } B\}.
\]
\item Intersect each of these lines with the sphere $S^{n-1}$, 
      yielding a superset of the Voronoi vertices:
\[
V = \{ {\cal L} \cap S^{n-1}: {\cal L} \in L\}.
\]
\item Measure the distance of every point of $V$ from the point set,
      and choose the point of $V$ that maximizes this distance.
\end{itemize}

Some observations will streamline this algorithm.
Notice that the intersection of a line through the origin with $S^{n-1}$ in Step 3
is simply a normalization: $tV \rightarrow \frac{V}{\|V\|}$.

We need an elegant way to compute the intersection of $n-1$ hyperplanes in
$n$-space, which is a line.\footnote{In one sense, this is simply a linear system.  
  But it is an underdetermined system,
  since there are $n-1$ linear equations in $n$ variables, 
  so typical methods for the solution of linear systems, such as Gaussian elimination,
  are not useful.}
The bisectors may be identified with the normal vectors of the associated hyperplanes,
since all of these hyperplanes pass through the origin.
A hyperplane is the locus orthogonal to a given vector,
so the intersection of $n-1$ hyperplanes is a line orthogonal to all $n-1$ normals.
The cross product offers a mechanism in 3-space for finding the vector orthogonal
to two ($n-1$) other vectors.
We need to generalize this result to arbitrary dimensions.

INSERT GENERAL CROSS PRODUCT DISCUSSION HERE (see rotation4d.tex's section 1.1.1).
(IT IS NECESSARY FOR GENERAL ROTATION AS WELL.)

What is the complexity of this approach?
Notice that at the moment there are $\|B\| = \choice{p}{2} = O(p^2)$ bisectors
(where $\|S\| = p$) and 
$\choice{\|B\|}{n-1}$ lines defined by the subsets.
In 4-space, this is $\choice{\|B\|}{3} = \choice{O(p^2)}{3} = O(p^6)$ lines,
which is far too many.
We need to reduce this amount.
IS THIS A FEASIBLE COMPLEXITY WHEN THE FURTHEST POINT IS HARD TO FIND?
[Suppose that we have a relatively small number, say 200, of quaternions: can we find 
a valid empty point robustly without the exact furthest point?
Consider the case where the object spins around an axis, so that the quaternions
march across the sphere and quickly violate the less-than-160-degree-span criterion,
although the sphere is very sparsely covered.
This is evidence that the naive divide and conquer algorithm will split up into subsets
long before it needs to.]

The distance of a point of $V$ from the point set may be maintained during the algorithm,
rather than computed from first principles at the end (which could take $O(n)$ time
per point in a naive implementation, and so $O(n^2)$ total time just 
for distance computation).
In particular, a record should be maintained of the generator points 
for each bisector in $B$, 
then the generator points for each line in $L$ (as the union of the generator points
of its defining bisectors), and finally the generator points for every point in $V$
(equivalent to the generator points of the defining line from $L$).
Then the point of $V$ is equidistant from all of its generators.
This shows that only one point of S needs to be carried along with each structure,
since the final point of $V$ will be equidistant from all of its generators.
Let $P$ be called a {\em representative} of bisector(P,Q).
{\em But some other point of S may be closer to $v \in V$ than its generators.}

START HERE
In Theorem 6.15 of Preparata-Shamos, how is the correct Voronoi vertex chosen quickly?
Easy: each Voronoi cell knows which point it is associated with: the distance only
needs to be calculated between the Voronoi vertex and this point.
The difference is that the work has been accomplished to determine that the vertex is
truly a Voronoi vertex and not an impostor, which we are not presently doing
with our blind intersection of every tuple of bisectors (a triple of bisectors
in the case of $S^3$).
{\em Therefore, we must determine how to eliminate impostors: read about the construction
of Voronoi diagrams. ROSS, DO THIS READING.}

Many of the points found in the above algorithm won't be Voronoi vertices.
Think of Euclidean 2-space version.
Can we avoid computing these vertices?
For example, do we only need to use subsets of bisectors with overlapping pairs:
for example, in 2-space, bisector(A,B) and bisector(A,C) may interact,
but perhaps bisector(A,B) and bisector(C,D) do not, in general.

% talk: only want to stop when 0 dimensional

\section{Furthest points on the 3-sphere}

We are after the vertices of the Voronoi diagram of a set of points.

The Voronoi diagram of a set of points is built from point bisectors, 
and Voronoi vertices are built from intersection of these point bisectors.
The number of point bisectors that must be intersected to generate a point (e.g.,
a Voronoi vertex) 
depends on the dimension of the space or manifold that we are working in.
The bisector of two points is a hyperplane (a linear structure of dimension n-1 
in n-space).
The codimension of a manifold of dimension m in n-space is n-m.
The intersection of a linear structure of codimension $m$ and a linear structure
of codimension $n$ is a linear structure of codimension $m+n$.
For example, the intersection of a line (codimension 2) and plane (codimension 1)
in 3-space is a point (codimension 3).
In 2-space or on a 2-manifold, the intersection of two point bisectors (hyperplanes)
yields a point.
In 3-space or on a 3-manifold, the intersection of three point bisectors (hyperplanes)
yields a point.

We are after the intersection of enough bisectors to generate a point.

We are after the intersection of enough hyperplanes to generate a line,
and the elegant construction of this line.
Then the introduction of the spherical restriction, and so the intersection
of this line with a sphere, will generate points.

\clearpage

\subsection{The bisector of two points in 2-space}

The bisector of two points $A$, $B$ in 2-space is a line, the perpendicular bisector.
It is the line through the midpoint $(A+B)/2$ with normal $B-A$.
This yields the implicit equation, since the line $ax+by+c=0$ has normal $(a,b)$, 
and any point on the line may be substituted into the equation to solve for $c$.
The parametric equation of the bisector is also wanted, since intersection is
most easily done with one implicit and one parametric representation.
The vector orthogonal to the vector $(x,y)$ is $(-y,x)$, so the parametric equation
of the bisector is $((a0+b0)/2, (a1+b1)/2) + t(a1-b1,b0-a0)$.

\subsection{The bisector of two points in 3-space} 

The bisector of two points $A$ and $B$ in 3-space is a plane.
It is the plane through the midpoint $(A+B)/2$ with normal $B-A$.
This representation is enough to compute the intersection of two of these bisectors.
The implicit equation is $(B-A) \cdot (x,y,z) + d = 0$ where $d$ is solved for by plugging
in the midpoint.
The parametric equation of a plane is not as natural or straightforward.

\subsection{The bisector of two points in 4-space}

This is a hyperplane or 3-plane in 4-space.
The point on the hyperplane is $(A+B)/2$ and the normal is $B-A$.

{\bf In general, the bisector of two points in n-space is a hyperplane.}

\subsection{The bisector of two points on $S^1$}

This is a point equidistant from the two points.
One can bisect the angle subtended by the points to find it.
So if the two points are $(\cos \theta_1, \sin \theta_1)$ and 
$(\cos \theta_2, \sin \theta_2)$, the bisector is
$(\cos ((\theta_1 + \theta_2)/2), \sin ((\theta_1 + \theta_2)/2)$.

\subsection{The bisector of two points on $S^2$}

This is a great circle of the 2-sphere.
A great circle is defined by a plane through the origin, 
or equivalently by the normal of this plane.
The great circle is the circle in this plane with the sphere's center and unit radius.
Let the two points be A and B.
The normal of the great circle's plane is B-A.
Equivalently, this is the intersection of the bisector of A and B in 3-space
with the 2-sphere.

Generalization: intersection of bisector in n-space with n-1-sphere.
(This bisector inherently goes through the origin.)

% Consider the great circle C defined by A and B, and the bisector P of A and B
% with respect to the 1-sphere C.
% P may also be found by as the unit vector in the direction (A+B)/2,
% or equivalently the intersection of the bisector line with the 2-sphere,
% where the bisector line is found by computing the bisector of A and B confined
% to the plane of A, B, and the 2-sphere center.
% Then the bisector with respect to the 2-sphere is the great circle through P
% orthogonal to B-A.
% This great circle is defined by P, and ---

\subsection{The bisector of two points on $S^3$}

This is the intersection of the 3-plane bisector of two points in 4-space with
the 3-sphere.

Notion of codimension.

{\bf In general, the bisector of two points on $S^n$ in $n+1$-space 
is a 'great sphere of dimension n-1', 
the intersection of a hyperplane with $S^n$.}

\subsection{Intersection of two bisectors in 2-space}

Line intersection.

\subsection{Intersection of two bisectors in 3-space}

Plane intersection.
This is best approached using the point/normal representation of the planes.
The intersection of (P1,N1) and (P2,N2) is (P,N1xN2), where P is found as follows.
Choose an arbitrary line in plane 1, such as $P1 + t(N1 \times A)$ where $A$ is
an arbitrary vector not equal to N1.
Then intersect this parametric line with implicit plane 2.

\subsection{Intersection of three bisectors in 4-space}

We need the intersection of 3 bisectors to get down to a line (see table).

We need a higher-dimensional cross product to use an analogy of the 3-space solution.

(move this discussion of hyperplanes up) 
Note that a hyperplane is defined by the family of vectors normal to a given vector.
That is, all of the lines through a point defined by all vectors normal to a given vector
$\cup \{P + tV_i: V_i \in V \}$ where $V = \{V_i: V_i \mbox{ is normal to } W\}$.
Since it is everything normal to a vector.
This is how it removes one dimension.

Using this idea, a hyperplane is defined by a vector and the intersection
of three hyperplanes in 4d, which is a line, can be defined by a (higher-dimensional)
cross product of these 3 hyperplanes' vectors.

\subsection{Intersection of n point bisectors in n+1-space}

START HERE: THESE ARE THE TWO KEY POINTS.
In general, the intersection of n+1 point bisectors in n+1-space will generate a point
and the intersection of n point bisectors in n+1-space will generate a line 
({\bf proof through codimension?})
Use the cross product of the n hyperplane's vectors to define the vector of the
line of intersection.

\subsection{Intersection of two bisectors on 1-sphere}

Undefined.

\subsection{Intersection of two bisectors on 2-sphere}

This is great circle intersection, which is equivalent to 
plane-plane intersection (a line) intersected with the 2-sphere.

\subsection{Intersection of three bisectors on 3-sphere}

Intersect the sphere with the 
line generated from the intersection of 3 point bisectors in 4-space.

\subsection{The dimension of hyperplane intersections}

Table~\ref{tab:dim} notes the dimension of various hyperplane intersections.
The second column is the dimension of the space.
The third column reflects the dimension of the intersection of two hyperplanes,
the fourth the dimension of the intersection of two hyperplanes and $S^{n-1}$,
the fifth the dimension of the intersection of three hyperplanes, 
and the last column the dimension of the intersection of three hyperplanes and $S^{n-1}$.

\begin{table}
\label{tab:dim}
\begin{tabular}{c|c|c|c|c|c}
hyperplane $H$ in $n$-space & $n$ & $H_1 \cap H_2$ & $H_1 \cap H_2 \cap S^{n-1}$ & $H_1 \cap H_2 \cap H_3$ & $H_1 \cap H_2 \cap H_3 \cap S^{n-1}$ \\ \hline
line in 2-space       & 1 & 0 & - & - & - \\
plane in 3-space      & 2 & 1 & 0 & 0 & - \\
hyperplane in 4-space & 3 & 2 & 1 & 1 & 0
\end{tabular}
\end{table}

What is the complexity of finding the furthest point in this brute force fashion? 

\bibliographystyle{plain}
\begin{thebibliography}{99}

\bibitem{renka}
Renka.

\end{thebibliography}

\end{document}
