\documentclass[12pt]{article} 
\usepackage{times}
\usepackage[pdftex]{graphicx}
\makeatletter
\def\@maketitle{\newpage
 \null
 \vskip 2em                   % Vertical space above title.
 \begin{center}
       {\Large\bf \@title \par}  % Title set in \Large size. 
       \vskip .5em               % Vertical space after title.
       {\lineskip .5em           %  each author set in a tabular environment
        \begin{tabular}[t]{c}\@author 
        \end{tabular}\par}                   
  \end{center}
 \par
 \vskip .5em}                 % Vertical space after author
\makeatother

% default values are 
% \parskip=0pt plus1pt
% \parindent=20pt

\newcommand{\SingleSpace}{\edef\baselinestretch{0.9}\Large\normalsize}
\newcommand{\DoubleSpace}{\edef\baselinestretch{1.4}\Large\normalsize}
\newcommand{\Comment}[1]{\relax}  % makes a "comment" (not expanded)
\newcommand{\Heading}[1]{\par\noindent{\bf#1}\nobreak}
\newcommand{\Tail}[1]{\nobreak\par\noindent{\bf#1}}
\newcommand{\QED}{\vrule height 1.4ex width 1.0ex depth -.1ex\ \vspace{.3in}} % square box
\newcommand{\arc}[1]{\mbox{$\stackrel{\frown}{#1}$}}
\newcommand{\lyne}[1]{\mbox{$\stackrel{\leftrightarrow}{#1}$}}
\newcommand{\ray}[1]{\mbox{$\vec{#1}$}}          
\newcommand{\seg}[1]{\mbox{$\overline{#1}$}}
\newcommand{\tab}{\hspace*{.2in}}
\newcommand{\se}{\mbox{$_{\epsilon}$}}  % subscript epsilon
\newcommand{\ie}{\mbox{i.e.}}
\newcommand{\eg}{\mbox{e.\ g.\ }}
\newcommand{\figg}[3]{\begin{figure}[htbp]\vspace{#3}\caption{#2}\label{#1}\end{figure}}
\newcommand{\be}{\begin{equation}}
\newcommand{\ee}{\end{equation}}
\newcommand{\prf}{\noindent{{\bf Proof}:\ \ \ }}
\newcommand{\choice}[2]{\mbox{\footnotesize{$\left( \begin{array}{c} #1 \\ #2 \end{array} \right)$}}}      
\newcommand{\scriptchoice}[2]{\mbox{\scriptsize{$\left( \begin{array}{c} #1 \\ #2 \end{array} \right)$}}}
\newcommand{\tinychoice}[2]{\mbox{\tiny{$\left( \begin{array}{c} #1 \\ #2 \end{array} \right)$}}}
\newcommand{\ddt}{\frac{\partial}{\partial t}}
\newcommand{\Sn}[1]{\mbox{{\bf S}$^{#1}$}}
\newcommand{\calP}[1]{\mbox{{\bf {\cal P}}$^{#1}$}}

\newtheorem{theorem}{Theorem}	
\newtheorem{rmk}[theorem]{Remark}
\newtheorem{example}[theorem]{Example}
\newtheorem{conjecture}[theorem]{Conjecture}
\newtheorem{claim}[theorem]{Claim}
\newtheorem{notation}[theorem]{Notation}
\newtheorem{lemma}[theorem]{Lemma}
\newtheorem{corollary}[theorem]{Corollary}
\newtheorem{defn2}[theorem]{Definition}
\newtheorem{observation}[theorem]{Observation}

% \font\timesr10
% \newfont{\timesroman}{timesr10}
% \timesroman


\setlength{\oddsidemargin}{0pt}
\setlength{\topmargin}{0in}
\setlength{\textheight}{8.6in}
\setlength{\textwidth}{6.875in}
\setlength{\columnsep}{5mm}
\markright{\hfill \today \hfill}
\pagestyle{myheadings}

% -----------------------------------------------------------------------------

\title{Finding an empty point on \Sn{3}\\[5pt]{\small UAB Technical Report\\
       February-March 2005}}
\author{John K. Johnstone\thanks{Supported in part by the National Science Foundation
        under grant CCR0203586.}\ and James Williams}
\begin{document}
\maketitle

\begin{abstract}
An algorithm to find the emptiest point, or the largest empty circle, on $S^3$.
That is, given a point sample on $S^3$, what is the point furthest from this point
sample?  This is a classical problem in the plane, where optimal(?) 
solutions have long been known.  It has also been studied on $S^2$ by Renka.
The $S^3$ problem is particularly interesting, since it combines a search on
a surface with a search in a 3-dimensional space.
\end{abstract}

\section{Related work}

Review the literature on empty points in 2-space, 3-space, on a surface.
Review the literature on Voronoi diagrams.
Note that $S^3$ is a 3-manifold, complicating matters.
An algorithm for points on $S^2$ is available from 
Renka(?).  
Explore 3D Voronoi diagrams for hints.

\section{Bisectors}

Discuss bisectors in Euclidean space and on the sphere. (See assignment I gave
to Ross.)

\section{Solutions}

The optimal solution is to find
the point in the middle of the largest empty region.
{\bf Janardan claims to have a solution to this in his LM (layered manufacturing) work.}
This is a generalization of the classical largest empty circle
problem for points in a plane \cite{shamos85}, 
and a similar technique using the Voronoi diagram may be used.\footnote{Renka \cite{renka}
  has developed an algorithm for the Voronoi diagram of points on the lower-dimensional
  $S^2$.}
Let $f(p)$ be the distance of $p \in \Sn{3}$ from the nearest quaternion.
(Note that distance must be measured on the surface.)
The maximum of $f(p)$ 
is attained at some vertex of the Voronoi diagram of the quaternions on \Sn{3},
which is the desired optimal empty point.
The Voronoi diagram must be built on the surface \Sn{3}, which 
is itself an interesting problem.
Luckily, only the Voronoi vertices need to be computed,
not the entire Voronoi diagram.
A Voronoi diagram is built out of point bisectors, and
the bisector of two points on \Sn{3}\ is a great circle of \Sn{3}.
A simplistic approach is to compute a superset of the Voronoi vertices
by intersecting all bisectors of two quaternions, and then 
choose the one furthest from all quaternions.
	% O(n choose 2) = O(n^2) circles; 
	% O(n^4) intersections
	% O(n^5) distance calculations, but can abort early on many
{\bf Implement this (good student project).}

One approach is to compute the Voronoi diagram of the point set and choose
one of the Voronoi vertices, but this requires quite a bit of work
and no algorithm has presently been developed for Voronoi diagrams 
of points on $S^3$.

An heuristic approach can be used to find a point of an
empty region, which is not guaranteed to find an empty point, 
but usually finds a very good empty point.
The idea is to move the data far from the pole by moving the best-fitting plane
of the data as far as possible from the pole.
In particular, the unit normal of the best-fitting hyperplane 
is chosen as the empty point (FIGURE OF DATASET, ITS BESTFITTING PLANE, AND ITS NORMAL AS POINT ON SPHERE).
Since the best-fitting plane is a good representative of most of the
points, the points tend to move far away from the pole.
Of course, outliers may be moved close to the pole.
This method is efficient, since the best-fitting plane may be computed
easily using principal component analysis:
the normal of the best-fitting plane is the eigenvector associated with the
smallest eigenvalue of the covariance matrix of the data \cite{ballard82}.
% The normal of the best-fitting plane is equivalent to the vector 
% such that the projection of the data points onto this vector has minimal variance,
% i.e., the vector a that minimizes a^t \Sigma a.

A simple randomized method is also worthy of mention.
Since there are so many empty regions on \Sn{3} (see below),
even a random choice of point on \Sn{3}\ is often effective.
A point on \Sn{3}\ is chosen at random until a point in an empty region 
is found.\footnote{To find a random point on \Sn{3}, find four random numbers 
  in [-1,1] defining a 4-vector, and then normalize this vector.}
   % A random number in [-1,1] can be generated in C/C++ using the 'rand' function
   % as follows:
   % (float) (rand() \% 32767) / 16383 - 1.0.
	% That is, we could randomly choose a point on \Sn{3}, test if it is
	% sufficiently far away from all data points, and if not then randomly choose
	% another point, and so on, until a valid point in an empty region is found.
	% The density of empty regions ensures the efficiency of this method.
		% The results in Section~\ref{sec:results} support the effectiveness
		% of a random choice of empty region \ref{}.
	% How many attempts before success? on all6: 1,1,1,1,4,56
This is a very simple method.
Care must be taken with repeatability.
If given the same data twice, we would like to choose the same point in an empty region
to rotate to the pole.
Fortunately, we can take advantage of the predictability 
of pseudo-random number generators.
If a pseudo-random number generator is used with the same seed
(such as the C 'rand' function), exactly the same sequence of `random' points will be 
generated each time it is called,
leading to the same quaternion spline if the method is repeated.

EVALUATE THESE THREE METHODS EXPERIMENTALLY.

\subsection{Existence of an empty region}
\label{sec:empty}

The very existence of an empty region is a question worthy of examination.
Is there an empty region whose center is at least $20^{\circ}$ from the
nearest quaternion?
It is theoretically possible that the data points are so densely packed
on the sphere that there is no empty region.
However, the requisite number of data points is huge.
	% large in theory, and
\Comment{
The 'surface area' of \Sn{3} is $2\pi^2$ \cite{kendall61}
while a region 30 degrees wide about a point has area $\pi/6$.
	% A point covers a region 30 degrees wide about it, which has
	% a surface area of $\pi/6$ on the sphere \Sn{3}\ of area $2\pi^2$
	% \cite{kendall61}.
Thus, theoretically only about $12\pi \approx 38$ points are needed to
cover the sphere, leaving no empty region.\footnote{Actually, quite
	a few more points are needed, since the discs about the 38 points
	do not abut perfectly.}
However, this requires a gap of at least 60 degrees between 
consecutive quaternions, a very large change of orientation.
The typical gap between consecutive quaternions for reasonable
motion control is about 20 degrees.
	% independent of the method used to construct the quaternion spline.
}
If there is a maximum gap of $20^{\circ}$ between a quaternion and its
closest neighbour, 434,783 points\footnote{The surface area 
	of \Sn{2}\ is $4\pi$, while the surface area
	of a region on \Sn{2}\ 10 degrees wide centered about a point is 
	$\int_{0}^{2\pi} \int_0^{\frac{\pi}{18}}  \sin \phi \ d\phi \ d\theta
	= .0000289$.} % \cite{lang79}
	% Lang, Calculus of several variables, p. 228
	% p. 208 for parameterization, p. 224-5 for ||dx/dphi x dx/dtheta||
	% integral of sin is -cos
are needed to cover \Sn{2}, and even more to cover \Sn{3}.
	% In the examples of Section~\ref{sec:results}, 
	% the distance between quaternions is about --- \ref{}.
Moreover, in order to leave no empty region, the quaternions
must be spread across the entire sphere.
It is far more common for the data to be restricted to a small region
of the sphere.
Thus, there will be a valid empty region on \Sn{3} in all but the most
pathological cases (over 400,000 quaternions uniformly scattered across \Sn{3}).

% We need a data point covering every empty region of the sphere.
% Consider the number of points needed to cover just the 2-sphere \Sn{2}.
% A point on \Sn{2}\ covers a region 30 degrees wide about it,
% which has the following surface area:
% \[
%  \int_{0}^{2\pi} \int_0^{\frac{\pi}{6}}  \sin \phi \ d\phi \ d\theta
%	= .841787
% \]
% using 10 degrees or \pi/18, area is only .0000289
% which leads to 431,073 points.
% Therefore, for data sets of size 15,
% it is possible to find a point $p$ with no data points within 30
% degrees of $p$.

\Comment{
% It is certainly more difficult to find an empty region for larger datasets, and
% may even be impossible in theory (although in practice, a huge dataset is required
% for $S^3$ to have no sufficiently large empty region).
One approach is to compute the Voronoi diagram of the point set and choose
one of the Voronoi vertices, but this requires quite a bit of work
and no algorithm has presently been developed for Voronoi diagrams 
of points on $S^3$.
However, empty regions are easy to find among small point sets.
}

\subsection{A divide and conquer technique for empty point (Building the quaternion spline up from smaller pieces)}
\label{sec:divide}

A simpler technique for finding an empty point is available if the dataset is small,
or if we can divide the dataset up into subsets and find an empty point for each subset
rather than an empty point for the entire set.
This approach is particularly important for very large datasets,
since empty points may not even exist for huge datasets.
(That is, the emptiest point is still too close to the data for our purposes.)

{\bf Adjust the following writeup for the purity of finding empty points, not embedded within
a quaternion spline paper.}
A robust and simple method may be designed if we are willing to build many
quaternion splines independently, each using only a subset of the quaternions.
Notice that, if the dataset is partitioned into subsets $\{q_i\}_{i=0}^n$ such that
the angle between $q_0$ and every quaternion $q_i$, $i \geq 1$ is less than $\theta$
degrees (i.e., every quaternion lies in a disk of radius $\theta$ about $q_0$),
then every quaternion $q_i$ of this dataset lies at a distance of at least 180 - $\theta$
degrees from $q_0$'s antipodal point.
This means that it is simple to find an empty point:
$q_0$'s antipodal point is an empty point of a subset $\{q_i\}_{i=0}^n$ that varies
by at most 160 (150?) degrees from $q_0$.\footnote{We measure distance
  from $q_0$ so that the subset may be designed trivially.  Larger subsets
  might be built if an arbitrary quaternion was used as the pivot.}
This suggests a divide-and-conquer strategy in which the dataset is partitioned
into small sets, and a quaternion spline is built for each small set.
% assumes a maximal distance between consecutive samples
The advantage of this divide and conquer approach is that it is guaranteed yet simple.
The second and third approaches above are simple but not guaranteed.
The first approach above is guaranteed but not simple.

Here is the algorithm in a C-like syntax.
This algorithm is called recursively, but the opening call is qSpline (n, q, NIL, NIL).

IN LIGHT OF THE ALGORITHM IN THE NEXT SECTION (when in the quaternion spline paper), 
THIS ALGORITHM CAN BE SHORTENED TO DEAL WITH SUBSET DEFINITION.

\begin{verbatim}
void qSpline (int n, Quaternion *q, float openingDeriv[NDERIV][3], qSpline &QS)
{
  i=1;
  subset = {q_0};
  nSubset = 1;
  while (angle(q[0],q[i]) < 160)
   {
     subset = subset $\cup$ \{q_i\};
     i++; nSubset++;
   }
  rotate subset and derivatives so that -q[0] is moved to the pole (1,0,0,0) (or whatever the pole of the inverse map is)
  if (openingDeriv == NIL)
       design a quaternion spline S through this rotated subset
  else design a quaternion spline S through this rotated subset with these rotated opening derivatives
  
  rotate the quaternion spline back
  measure the end derivatives of S
  if (nSubset < n)
   {
    qSpline (n-nSubset, q + nSubset, end derivatives of S, QSnext)
    QS = S spliced with QSnext
   }
  if (n - nSubset == 1)
    split the last two subsets into equal sizes
}
\end{verbatim}

We assume that object orientations (quaternions) are sampled frequently enough
so that consecutive quaternions span less than 160 degrees
(so subsets are of nontrivial size).
Note that quaternions at 180 degrees represent the same orientation (Section~\ref{sec:quaternionTheory}),
so no pair of consecutive quaternions need to be more than 90 degrees apart:
if they are, the second quaternion may be replaced by its antipode.

This method does require some additional theoretical development,
since the quaternion splines for each subset will need to be spliced 
together smoothly.
In particular, we must consider the control of derivatives at the endpoints of a
quaternion spline.

\bibliographystyle{plain}
\begin{thebibliography}{99}

\bibitem{renka}
Renka.

\end{thebibliography}

\end{document}
