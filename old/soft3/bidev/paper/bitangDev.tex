\documentclass[12pt]{article}
\usepackage{latex8}
\usepackage{times}
\usepackage{epsfig}
\input{header}

\newif\ifTalk
\Talktrue
\newif\ifJournal
\Journaltrue
\newif\ifFuture		% issues that are useful for future papers
\Futurefalse

\newcommand{\plucker}{Pl\"{u}cker\ }
\newcommand{\tang}{tangential surface\ }
\newcommand{\tangs}{tangential surfaces\ }
\newcommand{\Tang}{Tangential surface\ }
\newcommand{\atang}{tangential $a$-surface\ }
\newcommand{\btang}{tangential $b$-surface\ }
\newcommand{\ctang}{tangential $c$-surface\ }
\newcommand{\atangs}{tangential $a$-surfaces\ }
\newcommand{\btangs}{tangential $b$-surfaces\ }
\newcommand{\ctangs}{tangential $c$-surfaces\ }
\newcommand{\bisilhouette}{bisilhouette\ }
\newcommand{\Bisilhouette}{bisilhouette}
\newcommand{\bisilhouettes}{bisilhouettes\ }

\SingleSpace
\setlength{\headsep}{.5in}
	% \setlength{\oddsidemargin}{0pt}
	% \setlength{\topmargin}{-.2in}	% should be 0pt for 1in
\setlength{\textheight}{8in}
	% \setlength{\textwidth}{6.5in}
	% \setlength{\columnsep}{5mm}	% width of gutter between columns
\markright{The bitangent developable: \today \hfill}
\pagestyle{myheadings}

% -----------------------------------------------------------------------------

\title{The bitangent developable}
\author{John K. Johnstone\\
	Geometric Modeling Lab\\
	Computer and Information Sciences\\
	The University of Alabama at Birmingham\\
	University Station, Birmingham, AL 35294}

\begin{document}
\maketitle

% -----------------------------------------------------------------------------

\begin{abstract}
This paper explores the application of the tangential surface
to the construction of the bitangent developable,
a fundamental surface in the analysis of visibility, lighting and motion.
% The bitangent developable of the surfaces $G$ and $H$ is represented
% as a lofting between its curves of tangency with $G$ and $H$.
We also discuss applications of the bitangent developable.
\end{abstract}

\ifJournal
\noindent {\bf Keywords}: tangential surface,
	  bitangent developable, visibility, lighting, motion planning.
\fi

% \tableofcontents
% \clearpage
% \listoffigures
% \clearpage

% -----------------------------------------------------------------------------

\section{Bitangent developables}
\label{sec:bitangdev}

A natural application of the tangential surface system
is the computation of bitangent developables.
Bitangent developables of surfaces
are the generalization of bitangent lines of curves.
They are important in lighting and visibility analysis
since they bound the umbra and penumbra and, in general, mark a change
of visibility in a scene.
% Bitangent developables are necessary for visibility analysis
% (as they define visual events), lighting from a surface light source,
% shortest path motion, and other problems in geometric modeling,
% graphics, robotics, and computational geometry that involve the
% interaction between surfaces.

\begin{defn2}
A {\bf bitangent plane} of two surfaces is a plane that is tangent to
both surfaces.
The bitangent planes of two surfaces are organized into one-parameter families.
A {\bf bitangent developable} is the envelope of one of these
families of bitangent planes.
\end{defn2}

Each intersection curve of the tangential surface systems of $S_1$ and $S_2$
corresponds to a bitangent developable of $S_1$ and $S_2$.
Bitangency in primal space naturally becomes intersection in dual space,
since a bitangent plane is a tangent plane that is common to two surfaces
and an intersection point is a point that is common to two surfaces.

Theory of directrix curves and primal/dual relationships.

{\bf The main challenge is splicing the intersection curve together across
dual spaces.
To answer this challenge, 
we need to better understand the transition between dual spaces,
that is, their shared boundaries.}

% It is also safer to avoid clipping the box exactly,
% since with numerical precision this will often cut away part of the box too.

\clearpage

\section{Introduction}

The bitangent developable, a structure that interrelates two surfaces,
acts as a crucial analytical tool for the study of collections
of surfaces in applications such as visibility analysis, lighting, and motion.
The bitangent developable can be calculated directly 
from the intersection curves of tangential surface systems in dual space,
either from their tangent spaces (Theorem~\ref{thm:bidev})
or, more simply, using the parametric equivalence between points in primal
and dual space (Theorem~\ref{thm:bidev2}).


The computation of the bitangent developables of a pair of surfaces
can be reduced to a surface intersection problem in dual space.
This paper develops a practical implementation of this idea.

Section~\ref{sec:uses} motivates the importance of bitangency, in general,
and the bitangent developable, in particular.
A solution for the bitangent developable in primal space is presented
in Section~\ref{sec:comparison}, exhibiting its inferiority to a solution in dual
space.
The heart of the paper is Sections~\ref{sec:bitangdev}-\ref{sec:bitangcompute}.
Section~\ref{sec:bitangdev} begins to show how the bitangent developable
is computed in dual space using the tangent spaces of the intersection curves
of tangential surfaces.
After suggesting a representation for the bitangent developable in
Section~\ref{sec:rep},
the fundamental elements of this representation (the curves of tangency
that act as directrix curves) are computed in two ways in 
Section~\ref{sec:curveoftang}, 
leading to a more efficient computation of the bitangent developable,
and a final algorithm in Section~\ref{sec:bitangcompute}.

Bitangents between surfaces are necessary for visibility analysis
(as they define visual events), lighting from a surface light source,
shortest path motion, and other problems in geometric modeling,
graphics, robotics, and computational geometry that involve the
interaction between surfaces.

\clearpage

% -----------------------------------------------------------------------------

\section{Definitions}

\begin{defn2}
A {\bf bitangent plane} of two surfaces is a plane that is tangent to
both surfaces.
\end{defn2}

Each bitangent plane is part of a one-parameter family of bitangent planes.

\begin{defn2}
The bitangent planes of two surfaces $G$ and $H$ 
are organized into one-parameter families.
A {\bf bitangent developable} of $G$ and $H$ is 
the envelope of one of these families of bitangent planes \cite{koenderink}.
\end{defn2}

\ifJournal
\begin{defn2}
A bitangent plane of $G$ and $H$ is {\bf inner} if 
$G$ and $H$ lie on opposite sides of the bitangent plane
(in the local neighbourhood of the points of tangency).
It is {\bf outer} if 
$G$ and $H$ (locally) lie on the same side of the bitangent plane.

A bitangent developable of $G$ and $H$ is {\bf inner} (resp., outer) if 
it is the envelope of a family of inner (resp., outer) bitangent planes
of $G$ and $H$.
\end{defn2}
\fi

\begin{defn2}
A {\bf bitangent line} of two surfaces is a line that is tangent to both
surfaces.
\end{defn2}

\begin{defn2}
\label{defn:lofting}
A ruled surface R can be defined by {\bf lofting} between two curves
$C_1$ and $C_2$:
\[
	R(s,t) = (1-s) C_1(t) + sC_2(t)
\]
for $s \in \Re,\ t \in I \subset \Re$. % (Figure~\ref{fig:loft}).
$C_1$ and $C_2$ are called {\bf directrix curves}.
\end{defn2}

	% hand-drawn symbolic figure
	% OR loft.rgb (from loft.c++ program)
\begin{figure}[b]
\caption{Lofting}
\label{fig:loft}
\end{figure}
	
Notice that the directrix curves lie on the ruled surface,
and that the lines 
\begin{equation}
\label{eq:generator}
	\{ L_{\alpha}(s) = (1-s) C_1(\alpha) + s C_2(\alpha) \}_{\alpha \in I}
\end{equation}
define the generators of the ruled surface.

\ifJournal
\begin{example}
$(1-s)(\cos t, \sin t, 0) + s (\cos t, \sin t, 1)$ is a lofting between
two circles, defining a cylinder.
A typical generator is $(1-s)(1,0,0) + s(1,0,1)$.
\end{example}
\fi

\clearpage

% -----------------------------------------------------------------------------

\section{Representing a bitangent developable}
\label{sec:rep}

The bitangent developable is a ruled surface,
and its representation will reflect this fact.
We shall use one of the classical representations for a ruled surface.

We will represent the bitangent developable by lofting.
Which directrix curves should we use?
Since the defining characteristic of the bitangent developable is 
its tangency with two surfaces, 
the most natural choice for its two directrix curves is the two
curves of tangency.
%
\begin{itemize}
\item A bitangent developable will be represented by lofting between
      the curves of tangency with its defining surfaces.
\end{itemize}
%
There are many advantages to this representation, aside from its naturalness.
% This representation of the bitangent developable 
It provides immediate access to the curves of tangency and,
if properly parameterized, 
the bitangent lines generating the bitangent developable
(through the generators (\ref{eq:generator})).
The curves of tangency are useful to many
applications that use the bitangent developable.
For example, in lighting, the curves of tangency define the boundaries
of the umbra and penumbra.
Individual bitangent lines are also useful,
as we have observed in Section~\ref{sec:uses},
such as in the definition of shortest paths.

The final advantage of this representation is that it is
easily computed in dual space, which we now explore.
In particular, the curves of tangency of a bitangent developable
have a simple characterization in dual space.

\clearpage

% -----------------------------------------------------------------------------

\section{Dual theory I [The bitangent developable in dual space]}
\label{sec:bitangdev}

The computation of bitangency is ideally suited to dual space,
since bitangency in primal space becomes intersection in dual space,
a well studied operation in geometric modeling.
In this section, we explore the relationship between bitangency 
and intersection, in preparation for the computation of the bitangent
developable in dual space in Section~\ref{sec:bitangcompute}.

For simplicity of presentation, we shall temporarily ignore issues at infinity
and consider the image of a tangent space to be a single tangential surface.
We will expand the theory to the multiple surface representation of 
(\ref{eq:tangspace}) at the end of Section~\ref{sec:bitangcompute}.

Since the bitangent developable is the envelope of a family of
bitangent planes, we begin by considering the primal-dual relationship
for a bitangent plane (Lemma~\ref{lem:bitangplane}), 
then for a family of bitangent planes (Corollary~\ref{cor:family}).
Since the bitangent developable is composed of bitangent lines,
we continue with the primal-dual relationship for a bitangent line 
(Lemma~\ref{lem:biline}).
Finally, we address the primal-dual relationship for the bitangent developable
(Theorem~\ref{thm:bidev}).

\begin{lemma}
\label{lem:bitangplane}
A bitangent plane of $G$ and $H$ dualizes to an intersection point of
the tangential surfaces $G^*$ and $H^*$.
\ifJournal (Figure~\ref{fig:bitangplane}). \fi
\end{lemma}
\prf
A bitangent plane is simultaneously a tangent plane of $G$ and
a tangent plane of $H$.
Hence, it dualizes to a point of $G^*$ and $H^*$ (Definition~\ref{defn:tangsurf}).
\QED

	% hand-drawn symbolic figure
\ifJournal
\begin{figure}[b]
\vspace{1in}
\caption{A bitangent plane and the associated intersection point in dual space}
\label{fig:bitangplane}
\end{figure}
\fi

\begin{table*}[b]
\centering
\begin{tabular}{|l|l|}
\hline
primal space & dual space \\
\hline \hline
bitangent plane of $G$ and $H$ & intersection point of $G^*$ and $H^*$ \\ \hline
1-parameter family of bitangent planes of $G$ and $H$ &
	intersection curve of $G^*$ and $H^*$ \\ \hline
bitangent line of $G$ and $H$ & tangent of intersection curve \\ \hline
bitangent developable of $G$ and $H$ & tangent space of intersection curve \\ \hline
\end{tabular}
\caption{Bitangency in primal space is intersection in dual space}
\label{tbl:duality}
\end{table*}

\begin{corollary}
\label{cor:family}
A one-parameter family of bitangent planes of $G$ and $H$ dualizes
to an intersection curve of $G^*$ and $H^*$.
\end{corollary}
\prf
As the bitangent plane sweeps across the surfaces,
it traces out a curve in dual space.
\QED

\begin{lemma}
\label{lem:biline}
A bitangent line of $G$ and $H$ dualizes to the tangent of an intersection
curve of $G^*$ and $H^*$.
\end{lemma}
\prf
A bitangent line is the intersection of two consecutive bitangent planes.
Moving the bitangent plane infinitesimally in primal space is equivalent to
moving infinitesimally along the associated intersection curve in dual space.
By the principle of duality \cite{pedoe70},	% Pedoe, p. 244
an intersection in primal space becomes a join in dual space.
Hence, the intersection of two infinitesimally close bitangent planes
dualizes to the join of two infinitesimally close points of the intersection 
curve, a tangent.
\QED

\ifJournal
\begin{rmk}
We recall that lines dualize to lines \cite{pedoe70}.\footnote{The
	% Pedoe, p. 277; Hilbert/Cohn-Vossen, p. 121
	intersection of 2 planes $P$ and $Q$  
	in primal space (a line) dualizes to the join of the dual points $P^*$ 
	and $Q*$ in dual space (also a line).}
We know then that a bitangent of $A$ and $B$ dualizes to a line.
\end{rmk}
\fi

% We are finally ready to consider the dual image of a bitangent developable.

\begin{theorem}
\label{thm:bidev}
A bitangent developable of $G$ and $H$ dualizes to the tangent space of
an intersection curve of $G^*$ and $H^*$.
\end{theorem}
\prf
Suppose the bitangent developable is the envelope of the 
family of bitangent planes $\alpha(t)$,
which dualizes to an intersection curve $C$ (Corollary~\ref{cor:family}).
The envelope of $\alpha(t)$ is generated
by the intersections of consecutive surfaces 
$\mbox{lim}_{\epsilon \rightarrow 0} (\alpha(t) \cap \alpha(t + \epsilon))$.
Since the intersection of consecutive bitangent planes is a bitangent line,
the bitangent developable is generated by a sweeping bitangent line,
each of which dualizes to a tangent of the intersection curve $C$
(Lemma~\ref{lem:biline}).
Thus, the entire bitangent developable dualizes to the tangent space of $C$.
\QED

These primal-dual relationships are reviewed in Table~\ref{tbl:duality}.
Before continuing our discussion of the computation of the bitangent developable
in dual space, we divert to a discussion of the representation
for a bitangent developable in the next section.
This representation will suggest a
better method for computing the bitangent developable.

\clearpage

% -----------------------------------------------------------------------------

\section{Dual theory II [The curves of tangency in dual space]}
\label{sec:curveoftang}

A bitangent developable of $G$ and $H$ is associated with
an intersection curve of $G^*$ and $H^*$.
We have seen that the tangent lines along the intersection curve define the bitangent
developable (Theorem~\ref{thm:bidev}).
Each point of this intersection curve not only has an associated tangent line,
but also an associated tangent plane on $G^*$
and an associated tangent plane on $H^*$.
It will be the tangent {\em planes} along the intersection curve
that define the curves of tangency.

\begin{defn2}
Consider an intersection curve $C$ of the surfaces $G^*$ and $H^*$.
The {\bf $G^*$-tangent space} of $C$ is the set of tangent planes of $G^*$ along 
$C$ (i.e., the tangent space of $C$ with respect to the surface $G^*$).
\end{defn2}

\begin{lemma}
\label{lem:curveoftang}
% Consider a bitangent developable of $G$ and $H$.
% Let $C$ be the associated intersection curve of $G^*$ and $H^*$ in dual space.
The curve of tangency on $G$ of a bitangent developable of $G$ and $H$
dualizes to
the $G^*$-tangent space of an intersection curve of $G^*$ and $H^*$
(the same intersection curve as in Theorem~\ref{thm:bidev}).
\end{lemma}
\prf
Proof omitted for lack of space.
\QED
\ifJournal	% result is quickly superseded by Corollary~\ref{cor:curveoftang2}
\prf
READ OVER THIS PROOF.
Consider an intersection curve of $G^*$ and $H^*$ and the associated
bitangent developable $D$ in primal space.
Let $P$ be a point of the intersection curve and $T$ the tangent plane of $G^*$
at $P$.
We must show that $T^*$ is a point of the bitangent directrix curve on $G$,
i.e., a point on $G$ and $D$.
Any tangent plane of $G^*$ dualizes to a point on $G$, by straightforward 
duality, so $T^* \in G$.
Since $T$ contains the tangent line of the intersection curve at $P$
(which is the intersection of the tangent planes of $G^*$ and $H^*$ at $P$),
$T^*$ lies on a bitangent of $G$ and $H$ (Lemma~\ref{lem:biline})
and on $D$ (Theorem~\ref{thm:bidev}).
(We are applying here another aspect of the principle of duality:
the 'contains' relationship in dual space becomes
the 'lies-on' relationship in primal space \cite{pedoe70}.) % p. 244
\QED
\Comment{
	Consider a line generator of the developable.
	We want to isolate its two points of tangency with the primal surfaces.
	These two points dualize to two planes.
	The line generator is the dual of a tangent line of the intersection curve.
	Since both primal points lie on the line generator,
	both desired dual planes intersect the tangent line at a point of the
	intersection curve.
	{\bf Which planes through the line in dual space are associated with the
	two special points of the line in primal space that lie on the primal surfaces?
	Naturally, the two planes that are tangent to the dual surfaces,
	since points dualize to tangent planes.}
}
\fi

\ifJournal
This leads to a subtle reinterpretation of a solution:
rather than mapping 
the curve tangent spaces of the intersection curves back to bitangent
developables, we can map the surface tangent spaces of the intersection curves
back to curves of tangency, which act as directrix curves that define
the bitangent developable.
\fi

Lemma~\ref{lem:curveoftang} is an appealing generalization of 
Theorem~\ref{thm:bidev},
replacing a curve tangent space by a surface tangent space.
However, there turns out to be a more efficient characterization of the curves
of tangency.
We begin by reexpressing the relationship between the bitangent
line and the intersection curve in dual space.
In Lemma~\ref{lem:biline}, we related the bitangent line to the
tangent of an intersection point.
We now leverage the equivalence between the parameter values of a 
tangent plane and its dual point to relate
the bitangent line directly to the intersection point.

\begin{lemma}
\label{lem:bitang2}
The bitangent line associated with the 
intersection point $P = G^*(s_1,t_1) = H^*(s_2,t_2)$
is $\seg{G(s_1,t_1)H(s_2,t_2)}$.
\end{lemma}
\prf
The tangent of the intersection curve at $P$ dualizes to 
the bitangent line associated with $P$ (Lemma~\ref{lem:biline}).
We will show that this tangent also dualizes to $\seg{G(s_1,t_1)H(s_2,t_2)}$.
Using the principle of duality, 
the tangent planes of $G^*(s_1,t_1)$ and $H^*(s_2,t_2)$ dualize to the points
$G(s_1,t_1)$ and $H(s_2,t_2)$
\ifJournal
THIS STATEMENT NEEDS TO BE PROVEN IN THE JOURNAL VERSION, SINCE IT IS THE
DUAL OF THE ORIGINAL DUALITY.
\fi
and the intersection of these tangent planes
dualizes to the join of $G(s_1,t_1)$ and $H(s_2,t_2)$, $\seg{G(s_1,t_1)H(s_2,t_2)}$.
But this intersection of tangent planes is nothing but the tangent of the intersection
curve at $P$.
\QED

\ifJournal
\prf
The tangent planes at $A^*(s_1,t_1)$ and $B^*(s_2,t_2)$ dualize to the points
$A(s_1,t_1)$ and $B(s_2,t_2)$.
Thus, using the principle of duality, the intersection of these tangent planes
dualizes to the join of $A(s_1,t_1)$ and $B(s_2,t_2)$.
But the intersection of the tangent planes is the tangent of the intersection
curve at $A^*(s_1,t_1)$, which dualizes to a bitangent by Lemma~\ref{lem-line}.
\QED
\prf
Let $P$ be an intersection point $A^*(s_1,t_1) = B^*(s_2,t_2)$ in dual space.
We know that the tangent $T$ of the intersection curve at $P$ dualizes to a
bitangent (Table~\ref{tbl:duality2}).
This tangent T of $A^* \cap B^*$ at $P$ is the intersection of the tangent plane 
$T_A$ of $A^*$ and the tangent plane $T_B$ of $B^*$ at $P$:
$T = T_A \cap T_B$.
By the principle of duality (intersection becomes join), 
$T^* = \mbox{join} (T_A^*, T_B^*)$.
% the dual of the tangent is the join of these points.
That is, the associated bitangent is the join of 
$T_A^*$ and $T_B^*$.
But the tangent planes $T_A$ and $T_B$ dualize to a point of $A$ 
and a point of $B$, respectively, with the same parameter values as $P$,
yielding the desired result.
% But the parameter values of $T_A^*$ and $T_B^*$ are the same as $P$,
% yielding the desired result.
\QED
\fi

\ifJournal
This is a surprising result since a point normally dualizes to a plane 
(by definition!).
However, a point has an associated tangent line and tangent plane,
both in primal space and dual space.
Moreover, a point on $G$ can be associated with a point
on $G^*$ through identical parameter values.
It is these associations that allow us to probe beneath the fundamental
duality between points and planes to relate point with lines (Lemma~\ref{lem:bitang2})
and points with points (Corollary~\ref{cor:curveoftang2} below).
\fi

\begin{corollary}
\label{cor:curveoftang2}
The curves of tangency of the bitangent developable of $G$ and $H$
associated with the intersection curve $G^*(\alpha(t)) = H^*(\beta(t))$
are $G(\alpha(t))$ and $H(\beta(t))$.
\end{corollary}
% any explanation (proof) only confuses the issue

This leads directly to a different computation of the bitangent developable
from Theorem~\ref{thm:bidev}, requiring no calculation
of curve tangent spaces.

\begin{theorem}
\label{thm:bidev2}
The bitangent developable associated with the intersection curve
$G^*(\alpha(t)) = H^*(\beta(t))$ is 
\[
	(1-s) G(\alpha(t)) + s H(\beta(t))
\]
\end{theorem}
\prf
A bitangent developable is the lofting of its curves of tangency
with $G$ and $H$.
\QED

\clearpage

% -----------------------------------------------------------------------------

\section{Computing a bitangent developable}
\label{sec:bitangcompute}

Theorem~\ref{thm:bidev2} shows that the calculation of bitangent 
developables is equivalent to 
the calculation of intersection curves of tangential surfaces.
There is not even any dualization involved in the translation of 
the intersection curves in dual space back to bitangent developables in
primal space.
\ifJournal
This is a benefit of the equivalence between parameter values of a tangent plane
in dual space and a point in primal space.
\fi

Theorem~\ref{thm:bidev2} suggests the following simple algorithm
for calculating the bitangent developables of $G$ and $H$:
\begin{enumerate}
\item Compute the tangential surfaces $G^*$ and $H^*$.
\item Intersect $G^*$ and $H^*$.
\item Each intersection curve $G^*(\alpha(t)) = H^*(\beta(t))$ 
      defines a bitangent developable $(1-s) G(\alpha(t)) + s H(\beta(t))$.
\end{enumerate}

Incorporating the triumvirate representation of tangent space (\ref{eq:tangspace})
using 3 boxed tangential surfaces, this algorithm becomes:

\begin{enumerate}
\item Compute the dual representation (\ref{eq:tangspace}) of the tangent
	spaces of $G$ and $H$:
\[
	(G^*_{\fbox{a}}, G^*_{\fbox{b}}, G^*_{\fbox{c}}),\ 
	(H^*_{\fbox{a}}, H^*_{\fbox{b}}, H^*_{\fbox{c}})
\]
\item Intersect the tangential surfaces in each of the 3 dual spaces, 
	passively clipping to each box (Section~\ref{sec:tangsurf}):
\[
\begin{tabular}{c}
	$G^*_{\fbox{a}} \cap H^*_{\fbox{a}}$ \\
	$G^*_{\fbox{b}} \cap H^*_{\fbox{b}}$ \\
	$G^*_{\fbox{c}} \cap H^*_{\fbox{c}}$
\end{tabular}
\]
\item Merge the intersection curves from the 3 dual spaces.
\item Each intersection curve $G^*(\alpha(t)) = H^*(\beta(t))$ 
      defines a bitangent developable $(1-s) G(\alpha(t)) + s H(\beta(t))$.
\end{enumerate}

Figure~\ref{fig:intcurve} shows the component of the bitangent developable 
arising from intersection in one of the dual spaces (c-space).
The intersection curves are marked in the right figure.
Figure~\ref{fig:bitangdev} shows the pair of bitangent developables (one cone-like, the
other cylinder-like) arising from
all intersection curves in the three dual spaces.

	% 2 tops (shaded); a-envelope, b-envelope, and c-envelope of 2Top.pts3 together, original position
\begin{figure}[b]
\caption{Two surfaces and their bitangent developables}
\label{fig:bitangdev}
\end{figure}

Since the intersection of step (2) is confined to a small box, it is more 
efficient.
We use the classical subdivision algorithm of Lane and Riesenfeld \cite{lane80}
for surface intersection, 
with some elements of \cite{nat90}.
In our implementation, the surfaces in primal space
are Bezier surfaces and the tangential surfaces in dual space
are rational Bezier surfaces \cite{jj01b}.
The merging of step (3) is similar to the merging required in a subdivision
algorithm for surface intersection.

\clearpage

% -----------------------------------------------------------------------------

\section{Applications [On the many uses of bitangency]}
\label{sec:uses}

We would now like to explore problems that can be better solved using the
dual representation (\ref{eq:tangspace}) of the tangent space of a surface
that we have just introduced.
One of these problems is bitangency, an important but largely neglected 
problem related to the tangent space.
Before we discuss its solution, we explore some of the applications
of the bitangent to better understand exactly what we want to compute.

% -----------------------------------------------------------------------------

\subsection{Qualitative visibility}

	% Figure 2 of bisilhouette paper
	% 2sphereAB.rgb, from 2sphere.c++ program
	% (a) 2sphereA.rgb: general setup
	% (b) 2sphereB.rgb: looking from a point on the bitangent plane
\begin{figure}[b]
\caption{Crossing a bitangent plane may not change visibility: a bitangent plane of 2 spheres (left) and a crossing (right)}
\label{fig:vis1}
\end{figure}

	% Figure 3 of bisilhouette paper
	% 2sphereEnv.rgb, from 2sphere.c++ program
	% (a) 2sphereEnvANew.rgb: one of the envelopes of 2 spheres
	% (b) 2sphereEnvB.rgb: looking from a point just left of the envelope
	% (c) 2sphereEnvC.rgb: looking from a point just right of the envelope
\begin{figure}[b]
\caption{But crossing a bitangent developable will change visibility: a bitangent developable (top) and a crossing (bottom)}
\label{fig:vis2}
\end{figure}

The bitangent planes of two surfaces define the viewpoints where the line
of sight grazes both surfaces.
At these locations, the qualitative visibility (i.e., the objects
that are visible from the eye) may potentially change.
Yet most crossings of a bitangent plane do not change the visibility
(Figure~\ref{fig:vis1}).
What additional constraint causes such a change?
For the qualitative visibility to change, the eye must cross the
envelope of this family (Figure~\ref{fig:vis2}).
% We conclude that computation of the envelopes of bitangent planes 
% is important for visibility analysis.
This relationship of bitangency to visibility analysis is formalized
in the following lemma.

\begin{defn2}
The {\bf qualitative visibility} at a point of a scene is 
the set of objects that are visible when the eye is at that point.
\end{defn2}

\begin{lemma}
\label{lem:cross}
The qualitative visibility of a scene will change only if the 
the eye crosses a bitangent developable of two surfaces.
\end{lemma}

This is not a sufficient condition, since
the visibility will still not change if another object blocks the line
of sight between the two surfaces or the eye lies between the two surfaces.
However, Lemma~\ref{lem:cross} is a necessary condition,
and computation of the bitangent developable is the most difficult
computation in analyzing qualitative visibility.
Notice that qualitative visibility is important to efficient rendering, 
since it defines the objects that we can avoid rendering.

Visibility analysis of smooth surfaces has received far less attention
than visibility analysis of polyhedral scenes \cite{durand00}.
The development of the underlying theory has been begun, with work by Petitjean on 
aspect graphs of algebraic surfaces 
\cite{petitjean96} and Durand and his colleagues on
visibility complexes of smooth convex objects \cite{durand97},
which develop the global structure of the smooth visibility problem.
However, to make visibility analysis of smooth environments practical, 
efficient construction of 
fundamental elements such as the bitangent developable of smooth surfaces
will be necessary.

% -----------------------------------------------------------------------------

\subsection{Lighting}

Bitangency also impacts lighting.
% Consider a surface light source, an interesting
% generalization of the point light source.
In a scene lit by a surface light source, the umbra and penumbra
is bounded by bitangent developables of the light and the scene's objects.
Thus, the accurate simulation of lighting from a surface light source
also requires the computation of bitangent developables.

\ifJournal
The umbra is bounded by outer bitangent developables,
and the penumbra by inner bitangent developables.
Give pair of spheres example.
\fi

\ifJournal
\begin{figure}[b]
\vspace{1in}
\caption{Bitangency and lighting}
\label{fig:lighting}
\end{figure}
\fi

% -----------------------------------------------------------------------------

\subsection{Shortest path motion}

The final application of bitangency that we explore
% although by no means the only remaining application of bitangency, 
is motion planning.
Consider a point\footnote{A
	surface robot can be reduced to a point robot, 
	by growing the obstacles through Minkowski sum.}
robot moving among smooth obstacles.
Shortest paths are composed of straight lines through free space and
curves on the surface of obstacles.
One category of straight line in a shortest path
is a bitangent line between two obstacles.

A bitangent line will skirt past a pair of obstacles in the most efficient
manner, in a 3D analog of the bitangent lines in the 2D shortest path of
Figure~\ref{fig:motion}.
Thus, a knowledge of the bitangent lines of a scene is necessary
for shortest path motion planning.
Bitangent lines will be directly available from our representation
of the bitangent developable (Section~\ref{sec:rep}).

	% Figure 1 of 'A parametric solution to common tangents'
	% svgraph vg4b.rawctr, path only; jjdush.gif
\begin{figure}[b]
\caption{Bitangency and shortest paths in 2D}
\label{fig:motion}
\end{figure}

Bitangency arises in a second way in motion planning.
A common strategy in motion planning is to simplify the problem by replacing
robots or obstacles by their convex hulls.
The convex hull of a smooth surface $G$ is composed from parts of
the original surface and envelopes
of tangent planes that are tangent at two points of the surface---
i.e., bitangent developables of $G$ and $G$ (sic).
Thus, the computation of bitangent developables is important
to the construction of convex hulls, and hence to motion planning.

These applications indicate the importance of the bitangent developable
(the envelope of bitangent planes) and the bitangent line.
We will discover in Section~\ref{sec:rep} that 
the bitangent lines are directly available from a good representation 
of the bitangent developable.
Thus, the rest of the paper will concentrate on the efficient construction
of the bitangent developable.

\clearpage

% -----------------------------------------------------------------------------

\section{Related work [A solution in primal space]}
\label{sec:comparison}

% Let us compare our computation of the bitangent developable in dual space
% using tangential surfaces with a computation in primal space.

The bitangent developable could be computed in primal space,
without recourse to tangential surfaces, as follows.
However, this primal solution is much more expensive.

Suppose that $G$ and $H$ are polynomial parametric surfaces of degree $(m,n)$.
% In a generalization of the standard solution for 
% the bitangents of two plane curves,
A bitangent plane of $G$ and $H$ can be associated with a point pair
$(p,q) = (G(s,t),H(u,v))$, where
$p$ is one of the points of tangency of the bitangent plane with $G$
and $q$ is one of the points of tangency with $H$.
\ifJournal
\footnote{The normal of this bitangent plane is the normal of $G$ at $p$.}
\fi
These point pairs $(p,q)$ are identified by the fact that the line $\lyne{pq}$
is orthogonal to both the normal at $p$ and the normal at $q$,
and thus are the solutions of the following system of equations:
\begin{equation}
\label{eq:otherbi}
\begin{tabular}{c}
$(G(s,t) - H(u,v)) \cdot (\frac{\partial G}{\partial s} \times 
			 \frac{\partial G}{\partial t}) (s,t) = 0$ \\
$(G(s,t) - H(u,v)) \cdot (\frac{\partial H}{\partial u} \times 
			 \frac{\partial H}{\partial v}) (u,v) = 0$ \\
\end{tabular}			 
\end{equation}
or
\[
\begin{tabular}{c}
$f(s,t,u,v) = 0$\\
$g(s,t,u,v) = 0$
\end{tabular}
\]
This reduces the calculation of bitangent planes to the 
intersection of implicit surfaces of degree $(3m-1,3n-1)$ in 4-space.
In contrast, the solution in dual space will involve the 
simpler intersection of parametric surfaces of degree
$(3m-1,3n-1)$ in 3-space.

However, the biggest difference lies in the computation of the envelope
of these bitangent planes.
The decomposition of the solution set (\ref{eq:otherbi}) into
1-parameter families and the computation of the envelopes of these families
is much more difficult than our trivial computation of the envelopes
directly from the intersection curves in dual space
(Theorem~\ref{thm:bidev2}).
% THESE COULD BE THE BITANGENTS DIRECTLY, BUT STILL HARD TO EXTRACT THE BISILHOUETTE
% Equally importantly, there is no simple way to get the \Bisilhouette,
% or its directrix curves, from the representation (\ref{eq:otherbi})
% of the bitangent planes, unlike our algorithm.
% The explicit computation of the envelope of these two surfaces (\ref{eq:otherbi}) 
% in 4-space (e.g., \cite{boltyanskii64,martin90}) is much more difficult than

We now begin to consider the solution in dual space in detail.

\section{Silhouette}

\ifJournal
A closely related construct to the \bisilhouette is the silhouette,
which is a special case of the \bisilhouette
where one of the surfaces becomes a point.
There has been much work on the silhouette of smooth surfaces.
Schweitzer and Cobb \cite{schweitzer82} use the intersection of a plane with
Catmull's normal surface to compute the silhouette.
Sederberg and Zundel \cite{sederberg89}
compute the silhouette of an algebraic surface using polar surfaces.
Elber and Cohen \cite{elber90} compute the silhouette of a parametric
surface using subdivision.
Krishnan and Manocha \cite{krishnan94} express the silhouette as a determinant.
There are also many algorithms for silhouettes of polyhedral meshes.
Of these, \cite{zorin00} is notable since the silhouette is computed 
in dual space, by intersecting a plane with a dual mesh.
Points at infinity are dealt with by identifying projective 3-space with
the boundary of a hypercube in 4-space.
\fi

% -----------------------------------------------------------------------------

\bibliographystyle{plain}
\begin{thebibliography}{99}

\bibitem{catmull74}
Catmull, E. (1974)
A Subdivision Algorithm for Computer Display of Curved Surfaces.
Ph.D. thesis, University of Utah.

\bibitem{durand97}
Durand, F., G. Drettakis and C. Puech (1997)
The 3D Visibility Complex: a unified data structure for global
visibility of scenes of polygons and smooth objects.
9th Canadian Conference on Computational Geometry.

\bibitem{durand00}
Durand, F. (2000)
A Multidisciplinary Survey of Visibility.
SIGGRAPH 2000 Course Notes on Visibility: Problems, Techniques
and Applications.
Also available from http://graphics.lcs.mit.edu/~fredo/.

\bibitem{farin97}
Farin, G. (1997)
Curves and Surfaces for CAGD: A Practical Guide (4th edition).
Academic Press (New York).

\bibitem{harris92}
Harris, J. (1992)
Algebraic Geometry: A First Course.
Springer-Verlag (New York).

\bibitem{hartshorne}
Hartshorne, R. (1977)
Algebraic Geometry.
Springer-Verlag (New York).

\bibitem{zorin00}
Hertzmann, A. and D. Zorin (2000)
Illustrating Smooth Surfaces.
SIGGRAPH 2000, 517--526.

\bibitem{hoschek83}
Hoschek, J. (1983)
Dual Bezier curves and surfaces.
In {\em Surfaces in Computer Aided Geometric Design},
R. Barnhill and W. Boehm, eds.,
North Holland (Amsterdam), 147--156.

\bibitem{jj01a}
Johnstone, J. (2001)
A Parametric Solution to Common Tangents.
International Conference on Shape Modelling and Applications (SMI2001),
240--249.

\ifJournal
\bibitem{jj01c}
Johnstone, J. (2001)
Smooth Visibility from a Point.
39th Annual ACM Southeast Conference, 296--302.
\fi

\bibitem{jj01b}
Johnstone, J. (2001)
Bezier Tangential Surfaces.
Technical Report.

\bibitem{koenderink}
Koenderink, J. (1990)
Solid Shape.
MIT Press (Cambridge, MA).

\bibitem{lane80}
Lane, J. and R. Riesenfeld (1980)
A Theoretical Development for the Computer Generation and Display
of Piecewise Polynomial Surfaces.
IEEE Transactions on Pattern Analysis and Machine Intelligence 2(1),
35--46.

\bibitem{nat90}
Natarajan, B. (1990)
On Computing the Intersection of B-Splines.
Proc. of 6th Annual Symposium on Computational Geometry,
157--167.

\bibitem{pedoe70}
Pedoe, D. (1970)
Geometry: A Comprehensive Course.
Dover (New York).

\bibitem{petitjean96}
Petitjean, S. (1996)
The Enumerative Geometry of Projective Algebraic Surfaces and
the Complexity of Aspect Graphs.
International Journal of Computer Vision 19(3), 1--28.

\bibitem{thorpe79}
Thorpe, J. (1979)
Elementary Topics in Differential Geometry.
Springer (New York).

\end{thebibliography}

\end{document}
 
