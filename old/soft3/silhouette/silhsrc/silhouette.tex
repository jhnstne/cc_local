\documentclass[12pt]{article}
\usepackage{times}
\usepackage[pdftex]{graphicx}

\newcommand{\Comment}[1]{\relax}  % makes a "comment" (not expanded)
\newcommand{\prf}{\noindent{{\bf Proof}:\ \ \ }}
\newcommand{\QED}{\vrule height 1.4ex width 1.0ex depth -.1ex\ \vspace{.1in}} %square box
\newcommand{\choice}[2]{\mbox{\footnotesize{$\left( \begin{array}{c} #1 \\ #2 \nd{array} \right)$}}}
\newcommand{\tinychoice}[2]{\mbox{\tiny{$\left( \begin{array}{c} #1 \\ #2 \end{rray} \right)$}}}
\newtheorem{theorem}{Theorem}
\newtheorem{lemma}[theorem]{Lemma}
\newtheorem{defn2}[theorem]{Definition}

\title{Smooth silhouettes}
\author{John K. Johnstone\thanks{Computer and Information Sciences,
University of Alabama at Birmingham,
University Station, Birmingham, AL 35294, jj@cis.uab.edu.
This work was supported in part by the National Science Foundation
under grant CCR-0203586.}}

% \date{}

\begin{document}
\maketitle

Examine the silhouette bibliography at Calgary.

Silhouettes of smooth surfaces have received far less attention than silhouettes of polyhedral
meshes in the recent NPR literature.
Nevertheless, there are some advantages to a smooth silhouette algorithm:
\begin{itemize}
\item If the input surface is a smooth surface, there is no need to mesh it to find its silhouette.
\item The accuracy of the silhouette is controlled by the silhouette algorithm, and is 
      independent of the underlying surface/mesh,
      rather than requiring a remeshing (which will be impossible if a finer remeshing is required,
      unless the mesh is a subdivision surface that can be subdivided again)
\item The smooth surface is a perfect representation of the shape, and the silhouette curve can
      be computed in an adaptive fashion, using less points in areas of lower curvature.
      Unless the mesh sampling is adaptive to curvature (which most meshes are not),
      the silhouette will not be adaptive either.
\end{itemize}
The best example of a smooth silhouette algorithm is Elber and Cohen's.
Just as there are many polyhedral silhouette algorithms, it is desirable to have more than one
smooth silhouette algorithm to choose from.
In this paper, we propose another smooth silhouette algorithm based on dual space.
It can be viewed as a smooth analogue to the silhouette algorithm of Hertzmann and Zorin.

The challenges that must be overcome by this algorithm are:
\begin{itemize}
\item efficiency (dealt with by precomputing tangential surface systems and possibly by octrees)
\item 
\end{itemize}

We recently introduced a robust dual representation for the tangent space of a rational surface,
in which the tangent space is encoded by a collection of surfaces.
The dual relationship between the tangent space of a surface and a surface is classical,
but a practical and robust treatment of this relationship has been lacking.
A practical implementation of this dual relationship has been achieved for polyhedral meshes
by Hertzmann and Zorin.
Lifting this result to smooth surfaces presents some new challenges.

Working with smooth surfaces, 
This surface representation of the tangent space 
sharing the tangent space across three clipped Bezier surfaces.
This representation allows a tangent space to be treated as a surface
The strength of this construct is its application to problems in visibility analysis, 
most of which involve relationships between the tangent spaces of the objects in the scene.
A classical example of this type of visibility problem is the silhouette.
This construct was designed for the computation of visibility 

\end{document}
