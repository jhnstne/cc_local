\documentclass[11pt]{article}
\newif\ifVideo
\Videofalse
\newif\ifTalk
\Talkfalse
\input{header}
\newcommand{\plucker}{Pl\"{u}cker\ }
\DoubleSpace

\setlength{\oddsidemargin}{0pt}
\setlength{\topmargin}{-.2in}	% should be 0pt for 1in
% \setlength{\headsep}{.5in}
\setlength{\textheight}{8.5in}
\setlength{\textwidth}{6.5in}
\setlength{\columnsep}{5mm}	% width of gutter between columns
\markright{The dual curve: \today \hfill}
\pagestyle{myheadings}
% -----------------------------------------------------------------------------

\title{Silhouettes}
\author{J.K. Johnstone\thanks{Geometric Modeling Lab, 125 Campbell, 
	Computer and Information Sciences, UAB, Birmingham, AL 35294.}}

\begin{document}
\maketitle

\begin{abstract}
The multiple tangents of a curve are associated with singularities of the dual curve.
Let us consider a curve C and its dual curve, associated with its tangent
developable.
In general, objects such as silhouettes and common tangents of curves or surfaces
can be viewed as special subsets of lines from a ruled surface or
special ruled surfaces from a multiply infinite collection of lines.
It is these special kissing lines that we are intent upon in shortest path
motion.
\end{abstract}

\end{document}

