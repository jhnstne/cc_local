\documentstyle[11pt,letterhead]{bletter}
\signature{John K. Johnstone\\jj@cs.jhu.edu}
\begin{document}
\begin{letter}
{Dr. Jaroslaw R. Rossignac\\
IBM Research, J2-C03\\
P.O. Box 704\\
Yorktown Heights, NY  10598
}

\noindent \opening{Dear Dr. Rossignac:}

I am enclosing 5 copies of a revised version of the paper
`On the Lower Degree Intersections of Two Natural Quadrics'
by C.-K. Shene and myself (TOG 92-1),
along with 5 copies of this letter explaining how we have
handled the referees' requests.
I apologize for the tardiness of this reply.
The addition of material from Part II took extra time,
and communication between the two authors was complicated
when the first author moved to a different university
without access to electronic mail for a long time.

%*****************************************

First to your comments.
We have followed your request to condense the paper.
The section `Tangency and Disjointness', 
the first lemma of Section 4.2, %(not crucial to development)
and the case of parallel diagonals in Section 4.3 (with steps 6 and 10
of the ensuing Algorithm) 
have all been removed, the latter because of a new observation.
We have generally cleaned up the paper, such as making explicit 
Definitions and annotating algorithms and figures, mostly in Section 4.
The rest of the paper was carefully perused for possible condensation,
but none was possible.
Although the topic is similar to the conference paper
`On the Planar Intersection of Natural Quadrics', 
the presentation of this journal paper is very different,
using clearer terminology, more rigorous proofs, and 
substantially different methods of developing results
that are both more rigorous and more clear.
Moreover, all proofs are either crucial to
understanding or necessary for future proofs.

% Theorem 7.1 (Common Inscribed Sphere section)
% although there exist other proofs, this is the only published proof
% moreover, it is shorter than the others and very different from any others.

%the proof of Lemma 4.1 is necessary for the development
%	and is a much better proof than the one in the ACM paper.
%	The proof of Lemma 4.2 is necessary for intuition and understanding,
%	although it exists in the ACM paper.
%	The proof of Lemma 4.4 is not even in the ACM paper.
	
% 	Condensations: Proof of Thm 5.1 has been sketched and a reference
%	to the technical report substituted for most of the proof.

% *******************************************************

We have also followed your request to extract
results from our proposed `Part II' of this paper, and put them into
this paper.
We have added a section, Section~8 on `Enumerating all Conic Intersection
Types' from this material.
We will not be publishing `Part II' of this paper.

%*****************************************

On to the referees' comments.
To address the sole concern of Referee 1,
we have added a footnote to the proof of Lemma 4.1 that
explains why the Dandelin sphere is cut symmetrically.

One of the two concerns of Referee 3 was that the algorithm detected 
degenerate intersection, but did not compute it.
In order to clarify that the algorithm does indeed compute the intersection,
we have added the explicit characteristics of the computed conics
into step 6 of the Algorithm at the end of Section 4.
The other concern of Referee 3 was that the paper does not consistently
use geometric arguments.
The two theorems with algebraic proofs are Theorems~5.1 and 7.1.
An algebraic proof was used in Theorem~7.1 because
the intent of the proof was to show how the axial plane
could shorten the proof, and a geometric proof would be much longer.
Similarly, a geometric proof of Theorem~5.1 would be much longer.
Also, the initial parts of this proof are geometric
and the remaining algebraic part of the proof is needed for the proof 
of Theorem~7.1.
I hope this explains our occasional use of algebraic arguments.

The concerns of Referee 2 require a longer reply.
First, we have included a remark in the paper (Remark~2.1) addressing
this referee's suggestions.
I shall repeat these arguments and elaborate on them here.

Referee 2 suggests the following algorithm to replace ours:
\begin{enumerate}
\item
	\ [Reduce to a simple case] Map the cones to 
	cylinders.\footnote{By a projective transformation 
	that maps a plane through both cone vertices to the plane at
	infinity, and thus both vertices to infinity.}
	Then map the cylinders (simultaneously) to circular 
	cylinders.\footnote{One way to map a single cylinder 
	to a circular cylinder is to use
	the Gram-Schmidt diagonalization technique on the matrix
	representing the cylinder.}
\item
	If the cylinders' axes intersect at a finite point and the 
	cylinders' radii are equal, the quadric intersection consists
	of two conics.
	If the cylinders' axes intersect at infinity, 
	the quadric intersection consists
	of four lines.
	In all other cases, the intersection is not degenerate.
%
%	The quadric intersection of two cylinders can only be
%	degenerate if the cylinders' axes intersect: if the axes
%	intersect at infinity, the quadric intersection consists
%	of four lines; if the axes intersect at a finite point,
%	the quadric intersection is irreducible when the cylinders'
%	radii are not the same, but it is two conics when the radii 
%	are the same.
\item
	Transform the lines or conics
	computed in the last step (if they exist)
	back to the original coordinate system.
\end{enumerate}

\noindent The main problem with this method lies in the use of transformed 
quadrics, rather than working directly with the original quadrics.
In step (2), the axes of the transformed cylinders have to be extracted 
and tested for intersection, and the radii must be extracted and tested 
for equality.  
However, the transformations of step (1) 
distort information such as axes and radii,
as discussed in Goldman and Wilson (references 8 and 44 of the bibliography).
% For example, `[if] we initially select two cylinders of equal radii
% [represented by matrices], \ldots
% the more manipulations we perform on the cylinders, the more dissimilar
% the radii will become because of accumulated floating point inaccuracies'
% \cite[p. 22]{goldman:1983b}.
Due to the effect of transformations, we are not sure whether a pair
of non-intersecting axes are indeed intersecting in the original
coordinate system, or whether two radii are truly equal.
This is a particularly important problem in the context of the detection
of degenerate intersection, because degenerate intersection is inherently
a fragile condition: a small perturbation of the quadrics will change a
degenerate intersection into an irreducible intersection.
We wish to work with the purest data, which is the original data.

Other problems exist as well.

\begin{itemize}
\item
	Step 1 is expensive.  
	A Gram-Schmidt diagonalization method can be used to transform
	one cylinder to a circular cylinder, but we need to simultaneously
	transform {\em two} cylinders to circular cylinders, using the 
	same transform.  We are not clear how this would be done.
	However, assuming that it can be done (as it probably can),
	the cost will necessarily be considerable.
\item
	The inverse transformation of step (3) 
	may distort the type of the intersection
	conic, because of numerical inaccuracy (see Wilson).
\item
The transformations require complex arithmetic.
(For example, if one cone's vertex is contained in the other cone's
 	interior, any plane through both vertices intersects the cone
 	in two intersecting lines, and if this plane is mapped to the plane
 	at infinity, then the cone becomes a {\em hyperbolic} cylinder.
 	Complex arithmetic is required to transform this hyperbolic cylinder
 	to a circular one.)
% i.e., to map $x^2+y^2$ to $x^2-y^2$
 Complex arithmetic is not only more expensive than real arithmetic,
 but one must check if the lines and conics detected in step (2) 
 are still real after the inverse transformation of step (3).
\item
 The above method forces the use of an algebraic
 representation of the quadrics rather than a geometric representation.
\end{itemize}

In conclusion, referee 2's suggestion is mathematically a good
procedure, but it has problems with robustness and performance.
In general, a mathematically correct procedure is not a good 
computational algorithm.
Other algorithms for general intersection exist in the literature
that transform any pair of
quadric surfaces to some normal forms, compute the intersection and
then transform the curve back (Ocken, Schwartz and Sharir, or Bromwich), 
but we do not use these procedures
for degenerate intersection or natural quadrics because their computation
cost is higher and their robustness is worse than algorithms
such as the one presented in our paper.
% It is well-known that we can transform any pair of
% quadric surfaces to some normal forms, compute the intersection and
% then transform the curve back (see T. J. I'A Bromwich,
% {\em Quadric Forms and Their Classification by Means of Invariant
% Factors}, Cambridge University Press,
% or S. Ocken, J. T. Schwartz and M. Sharir,
% {\em Precise Implementation of CAD Primitives Using Rational
% Parameterizations of Standard Surfaces}, in {\em Planning,
% Geometry, and Complexity of Robot Motion}, edited by Jacob T.
% Schwartz, Micha Sharir and John Hopcroft, Ablex Publishing Co.,
% 1987, pp. 245--266 for a modern formulation.)
% However, we do not use these ``simple'' procedures for natural quadrics
% and degenerate intersection
% because their computation cost is higher and robustness is worse
% than specially designed algorithms such as ours.

%*****************************************

Please do not hesitate to call me at (410) 516-5560 if you have any questions.

\closing{Sincerely,}
\end{letter}
\end{document}

