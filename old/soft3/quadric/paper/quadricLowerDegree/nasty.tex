\documentstyle [12pt]{article}

\newtheorem{example}{Example}[section]
\newtheorem{property}{Property}[section]
\newtheorem{definition}{Definition}[section]
\newtheorem{theorem}{Theorem}[section]
\newtheorem{lemma}{Lemma}[section]
\newtheorem{corollary}{Corollary}[section]
\newtheorem{remark}{Remark}[section]

\newcommand{\DoubleSpace}{\edef\baselinestretch{1.4}\Large\normalsize}
\newcommand{\QED}{\ \ \ \rule{2mm}{3mm}\\}
\newcommand{\arrow}[1]{\vec{\bf #1}}

%\DoubleSpace
\setlength{\oddsidemargin}{0pt}
\setlength{\evensidemargin}{0pt}
\setlength{\headsep}{0pt}
\setlength{\topmargin}{0pt}
\setlength{\textheight}{8.75in}
\setlength{\textwidth}{6.5in}


\begin{document}
\begin{center}\Large\bf
     Response to Reviewer 92-1-2
\end{center}
\vspace{1cm}
     Reviewer 92-1-2 suggested the following algorithm as a replacement of
ours:
\begin{enumerate}
     \item\label{step-1}
          Use a linear transformation transforming both right cones to
          circular cylinders.
          This can be done with the Gram-Schmidt diagonalization technique.
     \item\label{step-2}
          If the cylinder axes do not intersect, intersection is irreducible;
          if the axes intersect at infinity, the intersection consists of four
          lines.
     \item\label{step-3}
          If the cylinder axes intersect at a finite point and the cylinders
          have equal radii, the cylinders intersect in two conics.
     \item\label{step-4}
          Transform the conics computed in the last step back to the original
          coordinate system.
\end{enumerate}

     This algorithm is short and mathematically correct, however, not very
encouraging because the reviewer missed some important issues required to design
a robust algorithm.  The following is our response:

\begin{enumerate}
     \item In step (\ref{step-1}) above, since Gram-Schmidt process is used, we
          assume that the reviewer uses an algebraic model ({\em i.e.},
          using an equation to represent  a quadric).
          In his paper ``Two Approaches to a Computer Model for Quadric
          Surfaces'', {\em IEEE Computer Graphics}, Vol. 3 (1983), No. 6,
          pp. 21--24, Goldman pointed out that because of floating point
          inaccuracy,
          \begin{quote}
               ... from the algebraic model it is not possible to ascertain
               with any certainty the type of surface being modeled; yet this
               decision critically affects our interpretation of all the other
               invariants of the surface. (page 21)
          \end{quote}
          After some discussions, Goldman concluded with
          \begin{quote}
               ... in terms of speed, size, and accuracy, the geometric model
               is far superior to the algebraic model. (page 24)
          \end{quote}
          This is one of the major reasons, among others, that we choose
          a geometric representation as our starting point.  However, a common
          wisdom is that geometric analysis usually requires a case-by-case
          study.

     \item In step (\ref{step-3}), the reviewer's algorithm extracts the radii
          of both transformed cylinders and tests for equality.  This is a
          dangerous step as Goldman has discussed:
          \begin{quote}
               Suppose, for example, that we initially select two cylinders
               of equal radii. If after performing several rigid motions to
               position one of the two cylinders we then extract the two radii
               from the invariants of the representative matrices, these radii
               will no longer be identical.  In fact, the more manipulations we
               perform on the cylinders, the more dissimilar the radii will
               become because of accumulated floating point inaccuracies.
               (page 22)
          \end{quote}

          Goldman continued with
          \begin{quote}
               For example, if the axes intersect and the radii are identical,
               the intersection curve is actually two intersecting ellipses;
               however, if the axes intersect and the radii are not
               identical, the intersection curve is a two branch, nonplanar
               curve.  While these curves may be very close geometrically,
               they are very different topologically.
          \end{quote}

          Thus, applying transformations to the original surfaces and then
          transforming the intersection curve back could have unexpected
          result.  Furthermore, in the new coordinate system, the reviewer's
          algorithm has to extract the axes of the cylinders.  Due to the
          effect of transformations, we are not sure whether a pair of
          non-intersecting axes are indeed intersecting in the original
          coordinate system.
          In our algorithm, since line intersection computations are used,
          which can be implemented accurately and have a meaningful geometric
          tolerance, we avoid using any transformation.

     \item In step (\ref{step-4}), the reviewer's algorithm transforms the two
          ellipses back to the origin coordinate system.
          There are two drawbacks: (1) we don't know the types of the
          intersection conics in the original coordinate system since
          projective transformations change the affine type of a conic
          ({\em i.e.}, transforms an ellipse to any other type).
          Therefore, we have to extract the type with an additional
          step. (2) Transformation may distort the types of the intersection
          conics because of numerical inaccuracy.  In Wilson's study,
          ``Conic Representations for Shape Description'',
          {\em IEEE Computer Graphics and Applications}, Vol. 7 (1987), No. 4,
          pp. 23-30, he discovered that after some transformations,
          \begin{quote}
               [for the implicit form], a quarter of the circles were distorted,
               or classified, into ellipses.  This has important practice
               consequences when one considers manufacturing a design -- a
               circular hole could be drilled or bored, but an elliptical
               hole would have to milled, which is a much more expensive
               operation (page 28).
          \end{quote}
          In our algorithm, the types of the intersection conics are
          determined by raw input without any transformation and hence
          are more accurate than the reviewer's.

     \item The reviewer's algorithm implicitly uses complex arithmetic and hence
          performance is worse than ours.  For example, if any cone's vertex
          is contained in the other cone's interior, any plane through the line
          joining the vertices intersects a cone in two intersecting lines.
          If this plane is chosen to be the plane at infinity, then the cone
          becomes a {\em hyperbolic} cylinder.  To transform this hyperbolic
          cylinder to a circular one so that its radius can be extracted,
          complex arithmetic is unavoidable.  Worse, if in the new coordinate
          system the intersection of cylinders consists of two ellipses,
          extra work must be performed to determine if in the original
          coordinate system the intersection conics are
          real, pure imaginary or consisting of one or two real points.

     \item In step (\ref{step-1}) of the reviewer's algorithm, he/she claims
          \begin{quote}
               This computation involves nothing more than using the
               Gram-Schmidt process to diagonalize a real symmetric matrix,
               and it reduces the problem of intersecting the two ``axial
               natural quadrics'' to the simpler problem of intersecting
               two cylinders.
          \end{quote}
          After taking a plane containing both vertices as the plane at
          infinity, the cones becomes elliptic cylinders.  We are not very
          clear, with a Gram-Schmidt diagonalization process, how to transform
          {\em both} elliptic cylinders to circular cylinders
          at the same time.  The reviewer should be more specific
          at this point.

     \item Note that to push both vertices to infinity, a center computation
          is necessary and this computation makes performance worse.

     \item In general, this is a good mathematical procedure.  Unfortunately,
          it treats robust and performance with simplicity.  A mathematical
          correct procedure in general is not a good computational algorithm.
          For example, it is well-known that we can transform any pair of
          quadric surfaces to some normal forms, compute the intersection and
          then transform the curve back. (See T. J. I'A Bromwich,
          {\em Quadric Forms and Their Classification by Means of Invariant
          Factors}, Cambridge University Press,
          or S. Ocken, J. T. Schwartz and M. Sharir,
          {\em Precise Implementation of CAD Primitives Using Rational
          Parameterizations of Standard Surfaces}, in {\em Planning,
          Geometry, and Complexity of Robot Motion}, edited by Jacob T.
          Schwartz, Micha Sharir and John Hopcroft, Ablex Publishing Co.,
          1987, pp. 245--266 for a modern formulation.)
          We don't use these ``simple'' procedures for natural quadrics
          because their computation cost is higher and robustness is worse
          than specially designed algorithms.  If the reviewer's
          ``computational point of view'' is identical to the perception of
          geometric computing community, he should be able to recognize the
          importance of robustness and performance, instead of only simplicity.

     \item We strongly disagree with the contents in the last paragraph of this
          reviewer's report.
          It is true that users are ``interested in having the computer paint
          pictures for us, and it does not need, nor even understand, geometric
          intuition, ...''.  But, there must be implementers spending
          their time in coding programs that paint the screen.
          Therefore, the easier the algorithm to be understood the lesser bugs
          the final programs my have.  If our algorithm is ``part of an
          advanced high school text book on coordinate geometry'', then
          perhaps high school students can implement a conic intersection
          detection and computation algorithm without much trouble.
          This is a benefit, not a drawback.
          After all, {\em TOG} focuses on solving computer graphics problems,
          not a mathematical journal in which mathematicians search for
          abstraction and generalization.

\end{enumerate}

     Based on the above reasons, it is not worth to add or even mention the
reviewer's algorithm in our paper.  However, we strongly suggest that the
reviewer submits his algorithm to a journal and listens to some other experts'
comments.  Hopefully, responses would be essentially what we have mentioned
above.

\end{document}
