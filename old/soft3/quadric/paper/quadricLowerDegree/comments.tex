CK, here are most of my comments.
	I will discuss them with you over the phone,
	but you might look them over beforehand if you can.
	I am also sending the present form of the paper (again, most
	but not all comments are incorporated) in the next mail,
	so that you have something to refer to during our discussion.
	It would help greatly if you could make a copy of the new paper
	before we talk.
	
	Talk to you soon.

John

**********************

Page numberings are from new version that you faxed to me.

p. 2, near top, `Using a simple geometric argument':
	What is the argument that (space cubic, line) never arises?
	Could put as a footnote here.

p. 2, `For example': I changed `conic intersection'
	to `planar intersection'.

p. 3, last paragraph: I rewrote this discussion.
	The discussion of robustness of geometric representation is only 
	a minor point, and should be made subsidiary to later points in
	the paragraph.
%	(Note: robustness is not the sole motivation of geometric
%	approach, simplicity of the method and thus speed is another reason.)

environment of p. 4: We do not state *anywhere* that we will restrict to
	axial natural quadrics (rather than all natural quadrics), 
	and why this is OK (i.e., why spheres are simple to work with).
	This must be added and would come naturally at the end of Section 3.

p. 4: no return after Notation.

p. 4, Figure 1: annotate with P.

p. 4, added material: I think it is redundant and tedious.
	Remove it.  (Is it ever important that direction vectors are unit
	length, though?)

p. 4, section 4, line 1: changed.

p. 5, paragraph on Dandelin spheres: this should probably be significantly
	shortened, since it is not the place of this paper to review
	all references on Dandelin spheres.
	The first sentence and its footnote could be reduced to:
	`These spheres are called focal spheres or Dandelin spheres,
	after G. Dandelin [6] (see, for example, --- or --- for more on
	the Dandelin sphere).

[to Rossignac]: the proof of Lemma 4.1 is necessary for the development
	and is a much better proof than the one in the ACM paper.
	The proof of Lemma 4.2 is necessary for intuition and understanding,
	although it exists in the ACM paper.
	The proof of Lemma 4.4 is not even in the ACM paper.
	
	It turns out to be impossible to use the ACM paper,
	because of the different terminology in the papers
	(which is less rigorous in the ACM paper)
	and the subtle difference in the results.
	The journal form has been improved and clarified.
	Also the comments made above about importance of keeping 
	the journal proofs.
	Therefore, our condensation has been achieved through other
	means.

	Condensations: Proof of Thm 5.1 has been sketched and a reference
	to the technical report substituted for most of the proof.

	Lateness: first author moved to a different university,
	which made communication difficult initially.
	Moreover, the addition of material from Part II took extra time.
	
p. 6, proof of Lemma 4.1: it is not clear to me why the plane E must cut
	the Dandelin sphere D into two symmetric parts.
	Is an elaboration necessary?

p. 6: I added a note after proof of Lemma 4.1, and after proof of Lemma 4.2.
	Moved Defn 4.1 immediately after Lemma 4.1, which motivates it.
	
p. 7: I made the definition of diagonals into an explicit Definition.
	I made an explicit Example of the following material.
	I believe that this helps the reader to follow the
	development, by highlighting the major concept of the section
	(the diagonal).

p. 7: The paragraph that defines diagonals and 
	the following example paragraph have been changed somewhat.
	
p. 7: end of example: is a `diagonal' that is the line at infinity
	considered a diagonal?

p. 7-8: why is Lemma 4.3 necessary: remove it?

p. 7, Figure 3: annotate the points of intersection as A,B,C,D, 
	in a consistent way so that AC and BD are always the diagonals,
	and if any intersections are at infinity they are B and D.
	Also *remove* the annotation of V1 and V2.

p. 8: I believe that the flow is helped by moving the proof of Lemma 4.4
	into an explicit proof after the lemma, that the reader can skip
	if he wants.
	This allows the reader more freedom and avoids the problem
	of the reader getting annoyed by details if he wants to just
	get the big picture.

p. 8: Lemma 4.4: removed restriction to coplanar axes, which is redundant
	since it is implied by Lemma 4.2.
	This is also done in other places where it is redundant.

p. 8, proof of Lemma 4.4: I don't remember why P \cap H must be the MAJOR
	axis (and not say the minor axis).
	It seems that symmetry only implies that P \cap H is an axis 
	of the conic.  Is an expanded argument needed?

p. 8, last sentence of Section 4.2: I removed it, since without
	the discussion of the next section, this comment suggests
	that you mean the naive algorithm of Goldman and Miller 
	(compute plane sections and see if they coincide).
	I have added a sentence to the next section that replaces
	this sentence.

p. 9, Figure 5: you have the annotation `d = P \cap H';
	I think that the `P \cap H' part does not add anything
	and just confuses the picture.
	Similarly in the first paragraph of Section 4.3.
	I would remove all of them, leaving just `d'.
	
p.8, line 4 of Section 4.3: again, why not minor axis rather than major?

p. 8-9: I have changed the first two paragraphs of Section 4.3.
	
p. 9: made formal definition of height.

p. 10: small changes.

p. 10, `Once the first intersection conic
	is determined, any point from the other diagonal can be used
	to compute the other intersection conic.':
	why is this sentence (new to this version) necessary?
	Why not just cover it in the algorithm?

p. 10, computation of focal length: what is u?  Above it is the distance
	of U from the midpoint of RS, but now there is no midpoint of RS.

p. 11, algorithm: make it clearer that the conic is *computed*, 
	not just detected.
	Thus, specify what the axes, axis lengths, and conic types are.

p. 11, step 10: *need to elaborate*

[for Rossignac]  we have made it clearer that the algorithm indeed computes
	the degenerate intersection if it is detected.
	In the algorithm at the end of Section 4, the intersection conics
	are given.

p. 11, algorithm: I have tried to highlight the main steps of the algorithm,
	by including explanatory headings to steps,
	and by combining steps that logically belong together.

p. 12, Section 5: it seems more logical to make the linear intersections
	section the last subsection of Section 5, since it is one
	of the degenerate cases.  I understand your wish to make linear
	intersections an important topic, but I believe that it still is
	as its own subsection.
	This affects the first paragraph of Section 5, as well as Section 1.

p.12, I removed the first paragraph of subsection 5.1, since it simply repeats
	what was said in the introduction of Section 5.

p. 12, Lemma 5.1: again for flow, I made explicit proof.
	rewording of statement, rewording of proof.

p. 13, Section 5.2: you mention the cylinder/cone and cyl/cyl cases here.
	However, they are generally ignored throughout the paper.
	Would it be better to state that the paper only discusses cones?

p. 13, second paragraph: shortened somewhat.
	Also, there is no formal proof in Theorem 8.2 so I removed that
	statement.  (A proof is not needed.)

p. 14, proof of Thm 5.1: To make the paper shorter (for Rossignac),
	this proof could be sketched and a reference made to the technical
	report.  The reason I believe this is alright is that
	(1) the proof is not intuitive, (2) is only trying to prove that
	the conic passes through Q, and (3) the result is analogous to 
	Lemma 4.4.

p. 15, algorithm: I replaced the wording of the technical report in step 2:
	I believe it is ambiguous otherwise (e.g., length).


Section 7 (Common inscribed sphere): 
	Theorem 7.1: Do you think that the proof could be replaced by
	the ACM proof and thus removed via a reference?
	The proof techniques are very different.

Section 8 (Alternate form of algorithm): 
	I am sure that Rossignac and TOG will
	consider this added four pages to be overkill, especially
	when Rossignac has demanded that we be more concise.
	Therefore, I think this section should be removed.
	Also, it is apparently covered by your ASME paper?

	For my own understanding: 
	Why do you want to compute X as the intersection of AB and CD,
	where A,B,C,D must be computed separately,
	rather than as the intersection of diagonals,
	which are easily defined in terms of skeletal lines,
	which are easily found as a pair of lines through the vertex?

	About the argument that the alternate algorithm moves the
	detection step forward:
	I am sure that referees like the second one will jump all over this,
	and argue that we are splitting hairs here.
	That is, the loss of time in doing detection/computation at the
	same time as in the first algorithm is trivial compared to
	detecting a nondegenerate intersection before you do any height
	computations.

Section 9 (Enumeration): it is enough to add this section:
	This will satisfy the request to add material from Part II,
	and yet also satisfy the request to make the paper more concise
	because we have shortened original material
	and removed one section (tangency) to counterbalance this
	added section.

p. 24: remove `as the second application' if Section 8 is indeed removed.
	
p. 25 (Section 9): you start out talking about axial natural quadrics,
	but only discuss cones in the rest of the section.
	What about cylinders, or should the entire section only mention	
	cones?

small but important changes throughout Section 9 
(e.g., made critical points into a defn)

I have not yet incorporated my comments on Linear Intersections
or the Common Inscribed Sphere sections.
I will have these shortly 
