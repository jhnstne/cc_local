CK, here are my comments.
I have decided that we should wait until tomorrow before we 
consider the paper finished, so that I can take a final look at the entire
paper and address the small remaining questions below.
We've waited long enough that one day won't make any difference.


p. 5, footnote: I have slightly moved the position of the footnote.
	
p. 5, proof: I have removed the bars on top of the axis l in the last
	sentences of the proof.  The rest of the proof does not use
	bars so this is for consistency.

p. 11, algorithm, step 4: I changed the wording to `(but $X$ not on
	any skeletal line)'.
	I changed the ordering of the cases: intersecting diagonals
	is the primary case so it should go first.
	I changed the wording of footnote b.

p. 11, I made the Goldman reference explicitly refer to Goldman.

p. 17, line 6 of proof of Thm 6.1: Magically it appeared again 
	although I have not touched any neighbourhood of it in the file!!
	There must be something funny going on.

I think we should comment somewhere that material on Tangency and 
Disjointedness is available in ACM paper.  I'm not sure where.

end of Section 2, from 'Robustness also motivates our use
	of geometric representations' to end of paragraph:
	Should we remove this statement, since
	there are no obvious transformations in our method,
	(especially in view of our comments that will now immediately
	follow this paragraph, in which we say that the method
	of Referee 2 is not appropriate, because it involves
	transformations: that is, we do not need to guard against
	transformations since our method will not have any).

Section 2 addition: This new remark would be put at the end of Section 2.

\begin{remark}
To better appreciate our approach, it is illuminating to investigate
the shortcomings of a conceptually simple algorithm that appears to
solve the problem:

\begin{enumerate}
\item
	\ [Reduce to a simple case] Map the cones to cylinders,
	(by a projective transformation 
	that maps a plane through both cone vertices to the plane at
	infinity, and thus both vertices to infinity).
	Then map the cylinders (simultaneously) to circular cylinders.
	(One way to map a cylinder to a circular cylinder is to use
	the Gram-Schmidt diagonalization technique on the matrix
	representing the cylinder.)
\item
	The quadric intersection of two cylinders can only be
	degenerate if the cylinders' axes intersect: if the axes
	intersect at infinity, the quadric intersection consists
	of four lines; if the axes intersect at a finite point,
	the quadric intersection is irreducible when the cylinders'
	radii are not the same, but it is two conics when the radii 
	are the same.
\item
	Transform the lines or conics
	computed in the last step (if they exist)
	back to the original coordinate system.
\end{enumerate}

\noindent The problem with this method lies in the use of transformed quadrics,
rather than working directly with the original quadrics.
In step (2), the axes of the transformed cylinders have to be extracted 
and tested for intersection, and the radii must be extracted and tested 
for equality.  
However, the transformations of step (1) 
distort information such as axes and radii,
as discussed in Goldman \cite{goldman:1983b} and Wilson \cite{wilson:1987}.
For example, `[if] we initially select two cylinders of equal radii
[represented by matrices], \ldots
the more manipulations we perform on the cylinders, the more dissimilar
the radii will become because of accumulated floating point inaccuracies'
\cite[p. 22]{goldman:1983b}.
Due to the effect of transformations, we are not sure whether a pair
of non-intersecting axes are indeed intersecting in the original
coordinate system, or whether two radii are truly equal.
This is a particularly important problem in the context of the detection
of degenerate intersection, because degenerate intersection is inherently
a fragile condition: a small perturbation of the quadrics will change a
degenerate intersection into an irreducible intersection.
We wish to work with the purest data, which is the original data.

Another problem is that the inverse transformation of step (3) 
may distort the type of the intersection
conic, because of numerical inaccuracy (Wilson \cite{wilson:1987}).
There is also a problem in the transformations themselves,
which require complex arithmetic.\footnote{For example, 
	if one cone's vertex is contained in the other cone's
	interior, any plane through both vertices intersects the cone
	in two intersecting lines, and if this plane is mapped to the plane
	at infinity, then the cone becomes a {\em hyperbolic} cylinder.
	Complex arithmetic is required to transform this hyperbolic cylinder
	to a circular one.}
	% i.e., to map $x^2+y^2$ to $x^2-y^2$
Complex arithmetic is not only more expensive than real arithmetic,
but one must check if the lines and conics detected in step (2) 
are still real after the inverse transformation of step (3).
IS THIS SO HARD?
A final drawback of the method is that it forces the use of an algebraic
representation rather than a geometric representation.
ELABORATE ON THIS IF WE REMOVE DISCUSSION OF GEOMETRIC REPRESENTATION ABOVE.

Our method works directly with the original quadrics without using
any transformations, it uses only real arithmetic, and it uses geometric
representations of the quadrics.
The predominant calculation is line intersection, which can be implemented
robustly.
\end{remark}


