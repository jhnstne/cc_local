% *********************************************************************
% *  Revision History :                                               *
% *           Mar/01/91 - Many changes, including the following       *
% *                       (1) Lemma 2.1 for perpendicular stuff is    *
% *                           added.                                  *
% *                       (2) Section for Linear Intersection has been*
% *                           rewritten.                              *
% *                       (3) Dandelin sphere, inscribed circle and   *
% *                           so on are added to the characterization  *
% *                           section.  The Goldman-Miller criteria   *
% *                           has been removed to save space.         *
% *                       (4) Robustness added in Conclusion.         *
% *                                                                   *
% *           Mar/02/91 - (1) Author field is changed since many ACM  *
% *                           has the new style and, by the way, I am *
% *                           not supported by NSF grant.             *
% *                       (2) Citation of the Figure 10 has been      *
% *                           replaced with a direct citation to      *
% *                           Hilbert's book.  Hence, if any item is  *
% *                           changed, this direct citation has to be *
% *                           changed accordingly.                    *
% *           Mar/04/91 - (1) Change to 10pt.                         *
% *                       (2) Second revision.                        *
% *********************************************************************

\documentstyle [twocolumn]{article}

\input{table}                 % include TaBlE macros

\newtheorem{example}{Example}[section]
\newtheorem{property}{Property}[section]
\newtheorem{definition}{Definition}[section]
\newtheorem{theorem}{Theorem}[section]
\newtheorem{lemma}{Lemma}[section]
\newtheorem{corollary}{Corollary}[section]

\newcommand{\DoubleSpace}{\edef\baselinestretch{1.4}\Large\normalsize}
\newcommand{\QED}{\rule{2mm}{3mm}}

%\DoubleSpace
\setlength{\oddsidemargin}{0pt}
\setlength{\evensidemargin}{0pt}
\setlength{\headsep}{0pt}
\setlength{\topmargin}{0pt}
\setlength{\textheight}{9.35in}
\setlength{\textwidth}{7in}
\setlength{\columnsep}{0.33in}

\addtolength{\oddsidemargin}{-6mm}
\addtolength{\topmargin}{-1cm}

\def\thefootnote{\fnsymbol{footnote}}        %%%%%% CHANGE TO footnote SYMBOL 

\title{\bf On the Planar Intersection of Natural Quadrics\footnotemark[1]}

\author{Ching-Kuang Shene\ \ \ \ \ \ \ John K. Johnstone\\ \\
        Department of Computer Science, \\
        The Johns Hopkins University,   \\
        Baltimore, Maryland 21218.}

\date{\ }

\pagestyle{empty}

\begin{document}

\maketitle


\footnotetext[1]{This work was supported by National Science Foundation grant 
                 IRI--8910366.\vspace{0.9in}}

\def\thefootnote{\arabic{footnote}} %%%%%%%%%% CHANGE to NUMBER
\setcounter{footnote}{0}            %%%%%%%%%% Reset the footnote counter

% #####################################################################
%                        ABSTRACT
% #####################################################################

\begin{abstract}
     In general, two quadric surfaces intersect in a space quartic curve.
However, the intersection curve frequently degenerates to  a collection of 
plane curves.  These degenerate cases
introduce problems into the general intersection algorithms and are difficult 
to implement in a reliable way.  In this paper, we investigate this problem 
for natural quadrics.  Three algorithms are presented: to detect and compute
1) conic intersections,
2) linear intersections and 3) tangency and disjointedness.
The techniques we present are considerably
different from traditional methods in that only elementary geometric arguments
are used.  Therefore these algorithms are easily implementable and can be 
understood easily without any algebraic geometry background.
\end{abstract}

% ######################################################################
%                       INTRODUCTION
% ######################################################################

\section{Introduction}
\label{section:introduction}
     In general, two quadric surfaces intersect in a space quartic curve.
However, the intersection curve frequently degenerates to a collection of
plane curves.  These degenerate cases
introduce problems into the general intersection algorithms and are difficult 
to implement in a reliable way.  In this paper, we will investigate this 
problem for natural quadrics, using fully geometric methods. 

     Natural quadrics consist of the spheres, the circular cylinders and 
the right cones.  Ever since Hakala's 1980 paper~\cite{hakala:1980}, many 
papers have been dedicated to this restricted class because, in most CAD 
systems, they are still the most useful and natural objects to model 
mechanical parts.  During the past decade, starting with 
Levin~\cite{levin:1976,levin:1979}, algebraic methods based on the pencil of 
two quadrics have been used successfully to attack the intersection problem.  
Other examples are Farouki--Neff--O'Connor~\cite{farouki:1989},
Miller~\cite{miller:1987}, 
Ocken--Schwartz--Sharir~\cite{ocken:1987} and Sarraga~\cite{sarraga:1983}.  
Only recently people have started to look at the conditions that cause 
degenerate intersection curves.  Farouki--Neff--O'Connor~\cite{farouki:1989}
is the first complete study using the pencil technique.  Their paper deals 
with general quadric surfaces.  To the best of 
our knowledge, Goldman--Miller~\cite{goldman:1990} is the only paper 
concerning the conditions that cause the intersection of two natural quadrics 
to degenerate to a conic.  Knowing if two natural quadrics have conic
intersection is useful since, in this case, as M. A. Sabin has pointed out,
the two surfaces can be blended using a family of cyclides 
(Pratt~\cite{pratt:1990}).  Moreover, if two quadrics intersect in conics, 
their blending surface can be a quadric or cubic surface, rather than a
quartic (Warren~\cite{warren:1987}).


     In the Goldman--Miller paper~\cite{goldman:1990}, the pencil method is 
used to establish some simple conditions for conic intersections.  
However, methods based on pencil theory are usually involved which lack 
geometric intuition.  This may be necessary for the general case,
but simpler methods are possible with natural quadrics.  Because of this and 
our belief that geometric methods usually uncover geometric insight, we will 
use elementary geometry to handle the intersection problem.

    Goldman and Miller focus on the detection of planar 
intersections,\footnote{Computation will be discussed in an upcoming paper 
of Goldman and Miller.} whereas detection and computation are done at
the same time with our method.   We also consider the detection and computation
of one point tangency and disjoint quadrics.  Since our approach is completely
geometric (only geometric arguments are used), 
it does not depend on the representation 
(e.g., parametric or implicit).  This provides an advantage over the algebraic
pencil method since it can be adapted to any representation as long as that 
representation is capable of calculating the intersection of lines and planes,
determining whether a point lies inside a surface, and computing the distance 
from a point to a surface along a specific direction.

     Using the pencil technique, Goldman--Warren~\cite{goldman-warren:1990}
generalizes Goldman--Miller's idea to revolutionary quadrics.  They found that
if the axes of two quadrics of revolution intersect at a point equidistant
from the two surfaces, then the intersection between the two quadric surfaces
is degenerate.  Note that this is only a sufficient condition.  The height
test to be discussed in Section~\ref{section:algorithm} can be modified
easily to handle this type of quadric surface and thus provides a necessary
and sufficient condition for conic intersection of two revolutionary quadrics.

     Finally, Piegl~\cite{piegl:1989} uses
a geometric method to distinguish eight cases for cylinder--cone intersection,
in which two of them give conic intersection.  However, from Piegl's
enumeration, it is not  clear how to devise an extension to handle
the cone--cone case, which is more general and more subtle than the 
cylinder--cone case, since conic intersection can be obtained even if the 
vertex of the cone lies inside the other surface.

     This paper is divided into seven sections.  
In Section~\ref{section:algorithm}, our algorithm for detecting and 
calculating conic intersections is presented in its most general form without 
concern for degenerate cases.  Section~\ref{section:special-degenerate} 
discusses how to handle the degenerate cases so that the presentation of our 
algorithm is complete.  Sections~\ref{section:linear} 
and~\ref{section:tangency-disjoint} present simple algorithms for detecting
and computing linear intersections, tangency and disjointedness.  
In Section~\ref{section:characterization}, some characterization 
theorems are presented, which give more geometric insights into the 
intersection problem.  Section~\ref{section:conclusion} gives our conclusion 
and an overview of the method.

     For simplicity, we concentrate on the intersection of two cones.  The 
intersection of a sphere with another surface is very simple; and results for 
the cylinder can be obtained easily from the cone algorithms, since a cylinder
is a cone with vertex at infinity.  
\begin{table}
\caption{Possible Intersection Types of Two Natural Quadrics}
\label{tbl:intersection-type}
$$
\BeginTable
     \BeginFormat
     |5 c              | c                    | c             |5
     \EndFormat
     \_5
     | \it Number of | \it Number of | \it Number of | \\
     | \it Conics    | \it Lines     | \it Points    | \\ \_4
     |      2        |       0       |       0       | \\ \_
     |      1        |       1       |       0       | \\ \_
     |      0        | 1 (tangent line), 2 | 0       | \\ \_
     |      0        |       0       |      1,2      | \\
     \_5
\EndTable
$$
\end{table}

     Since we are concerned with planar intersections, the case of a line and 
a space cubic, as well as all quartic space curves, will not be considered.  
With a simple geometric argument, it can be shown that the planar intersections
in Table~\ref{tbl:intersection-type} are the only possibilities.

     In the rest of this paper, we will use the following notations.
\begin{itemize}
     \item A {\em natural} quadric is a sphere, a circular cylinder, or a 
          right cone.
     \item An {\em axial} quadric is a natural quadric with an axis, namely 
          a cylinder or a cone.
     \item ${\cal C}(V,\ell,\alpha)$ is the cone with vertex $V$, axis $\ell$
          and half angle $\alpha$.
     \item ${\cal Z}(\ell,r)$ is the cylinder with axis $\ell$ and radius $r$.
     \item $\stackrel{\longleftrightarrow}{AB},
          \stackrel{\longrightarrow}{AB},\overline{AB}$ and
          $|\overline{AB}|$ are the line, the half ray, the segment, and the 
          length of the segment determined by two points $A$ and $B$.
     \item A {\em conic} intersection is a composite intersection curve such
          that the highest degree among all component curves is two (Cases 1 
          and 2 in Table~\ref{tbl:intersection-type}).
     \item A {\em linear} intersection is a composite intersection curve such
          that the highest degree among all component curves is one (Case 3 
          of Table~\ref{tbl:intersection-type}).
     \item A {\em point} intersection is a composite intersection such that the
          components are all isolated points (Case 4 of 
          Table~\ref{tbl:intersection-type}).
     \item {\em Tangency} means a point intersection.  Hence a common tangent
          line is a linear intersection.
     \item Two surfaces are {\em congruent} if and only if there exists a 
          rigid motion transforming one into the other.
\end{itemize}

% #########################################################################
%                          MAIN ALGORITHM
% #########################################################################


\section{The Main Algorithm}
\label{section:algorithm}
     In this section, we present our conic intersection algorithm in a
generic form, ignoring degenerate cases.  The idea is to  reduce the surface 
intersection problem to a simple planar line intersection problem, determine 
whether the intersection curve is a conic via some simple inside-outside and 
on-surface tests and, if so, compute the equation of the intersection conic.
The following technical lemma is needed in later work.

\begin{lemma}
\label{lemma:perpendicular-stuff}
     Let $P$ be a plane cutting a revolutionary quadric in a conic $C$.
Then the plane $E$ determined by the major axis of $C$ and the axis of 
revolution of the quadric is perpendicular to $P$.
\end{lemma}
{\bf Proof:} We will establish this lemma for a 
cone ${\cal C}(V,\ell,\alpha)$.
The same proof holds for other revolutionary quadrics, by replacing $V$
by the center of the quadric and $\ell$ by the axis of revolution.
Let $D$ be the plane
perpendicular to $\ell$ through $V$ (Figure~\ref{fig:perpendicular-stuff}).
In general, $P$ intersects $D$ in a line $d$.  Drop a perpendicular $e$ from
$V$ to $d$.  The plane determined by $e$ and $\ell$ is clearly perpendicular
to $P$, and contains the major axis of the conic.  In fact, it is exactly the
plane $E$. \ \ \ \ \QED
\begin{figure}
\vspace{5.5cm}
\caption{Relation of Plane $E$ and Plane $P$}
\label{fig:perpendicular-stuff}
\end{figure}

We can immediately use this lemma to develop a necessary condition for
conic intersection.

\begin{lemma}
\label{lemma:non-skew}
     If two axial natural quadrics have a conic intersection component, their 
axes are coplanar.  
\end{lemma}
{\bf Proof:}  Let $Q_1$ and $Q_2$ be two cones
with a conic intersection.  
(The proof for cylinders is analogous.)
Let $P$ be the plane containing one of the 
intersection conics.
By Lemma~\ref{lemma:perpendicular-stuff},  the plane determined by the major 
axis of the intersection  conic and the axis of $Q_1$ is perpendicular to $P$.
Hence, the axis of $Q_1$ lies on the plane through the axis of the 
intersection conic and perpendicular to $P$.  By the same argument, the axis
of $Q_2$ lies in this plane.\ \ \ \QED

     If the axes of two axial quadrics are distinct and coplanar,
they determine a plane called the {\em axial plane}.  
This plane cuts each surface in a pair of straight lines, called the 
{\em skeletal pair}.  In general, these two 
pairs of lines intersect each other in four points, and this is the 
non-degenerate case that will be discussed in this section.  The lines joining
these four points that do not coincide to the original lines are called the 
{\em diagonals} (Figure~\ref{fig:potential-seg}).  We restrict to diagonals
that do not contain the vertex of either cone.
The collection of diagonals and skeletal pairs are called the
{\em skeletal lines} on the axial plane.\footnote{More precisely, the four
lines of the two skeletal pairs form a {\em complete quadrilateral} with 
three diagonal lines.  However, the diagonal containing the
two vertices of the two given cones will not be used in this paper.}  
Each diagonal is cut by the
two intersection points into three parts, one {\em finite} and two 
{\em infinite segments}.  If either finite or infinite segment lies entirely 
in the interior of both surfaces, it is called a {\em potential segment}.  
Note that, since the end points of a potential segment lie on the intersection 
of the two skeletal pairs and thus on the intersection of the two surfaces, 
and the cylinder and half cone are convex, if any point of the segment lies 
in the interior of both surfaces, so do all interior points of the segment.  
Hence, in order to test whether any one of the segments lies in the interior 
of the intersection, simply test an arbitrary interior point of the segment.

% \vspace{5mm}
\begin{figure}
\vspace{7cm}
\caption{Several Examples of Potential Segments}
\label{fig:potential-seg}
\end{figure}

     In general, we have one or two potential segments; however, a potential
segment may not exist at all.  Furthermore, the number of potential segments,
in some degenerate cases, could be infinite.   
In Figure~\ref{fig:potential-seg} we show some skeletal lines of two cones,
with potential segments indicated by thick lines.

     Suppose two axial quadrics have a conic  intersection.  
For convenience, let it be an ellipse $E$ (see 
Figure~\ref{fig:cone-cone-example}).  
By Lemma~\ref{lemma:perpendicular-stuff}, the plane containing
$E$, $E_P$, is perpendicular to the axial plane ${\cal H}$.
Note that the two intersecting quadrics,
and thus the intersection curve, are symmetric about the axial plane 
${\cal H}$.  Therefore $E$ is cut by ${\cal H}$ into two congruent halves.
Let us consider the vertical projection of $E$ onto ${\cal H}$.  Obviously, 
this vertical projection lies in the line $E_P\cap{\cal H}$.  Furthermore,
the two points $E\cap{\cal H}$ lie on both surfaces and thus are the 
intersection points of the skeletal lines.  $E_P\cap{\cal H}$ must be a 
diagonal, since it is a skeletal line and it does not lie on either surface.
However, since the vertical projection of $E$ 
onto ${\cal H}$ lies in the interior of both surfaces,
it is a potential segment.  If the 
intersection curve $E$ is a hyperbola or a parabola, its projection onto 
${\cal H}$ would be an infinite interval with either one or two components.
These observations allow us to develop an efficient conic 
intersection algorithm.

     Let us find a diagonal and the potential segment on it.  If no potential 
segment can be found, the above observation tells us that there cannot be any 
conic intersection.  If we do find a potential segment, pick any point on it 
and let it be $X$.  The {\em height} from $X$ to either surface is the 
distance from $X$ to that surface, measured along the line perpendicular to 
${\cal H}$ and through $X$ (see Figure~\ref{fig:cone-cone-example}).  This is 
well-defined, because both surfaces are symmetric about ${\cal H}$ and the 
components of the interior of the surface containing $X$ are both convex.  
Hence the perpendicular to ${\cal H}$ through $X$ hits either surface in two 
points and no matter which point is chosen, the height is the same.  
\begin{figure}
\vspace{2.5cm}
\caption{The Conic Intersection of Two Cones}
\label{fig:cone-cone-example}
\end{figure}

\begin{definition}
     $X$ passes the {\em height test} if and only if the heights to both 
surfaces from $X$ are equal.
\end{definition}

     We claim that if the chosen point $X$ passes the height test, then the two
surfaces have a conic intersection.   On the plane through the potential 
segment and perpendicular to ${\cal H}$, there are two conics, one from each 
axial surface.  Let these conics be called $E_1$ and $E_2$.  Note that these 
two conics project onto the same potential segment (see 
Figure~\ref{fig:height-test}).  Let the heights from $X$ to the surfaces be 
$h_1$ and $h_2$.  Recall from elementary geometry that if we know the type of a
central conic, the two end points of its major axis and one more point on the 
conic, then the conic is uniquely determined.\footnote{This can be easily seen
using some algebra.  The distance between the two end points is $2a$, where $a$
is the length of the major axis.  Then the central conic has equation 
$x^2/a^2\pm y^2/b^2=1$, where $b$ is the unknown length of the minor axis.  
Plugging the third point into this equation and solving for $b$, we have the 
complete equation.}  Thus, if $X$ passes the height test, the two ellipses 
$E_1$ and $E_2$ must coincide.  Note that the major axis of this ellipse is
the potential segment, and the minor axis is the line perpendicular to the 
axial plane at the midpoint of the potential segment.  
In summary, we have a simple method of detecting conic intersection.
\begin{figure}[h]
\vspace{5.5cm}
\caption{A Height Test}
\label{fig:height-test}
\end{figure}

\begin{theorem}
\label{theorem:conic-conditions-using-height-test}
     Two axial natural quadrics have a conic 
intersection component if and only if there exists a potential segment such 
that any point of it passes the height test.
\end{theorem}

     We also have a simple method of computing the conic.  It is an ellipse
if the potential segment is finite, a parabola if the potential segment is
one infinite segment, and a hyperbola if it is two infinite segments.
Knowing its type, the conic can be determined by 2 or 3 points: the end points
of its potential segment and the point of the height test.

     If two quadric surfaces intersect in a conic, then the residue curve (the
other component of the intersection) must be a conic also, perhaps a 
degenerate one (in which case see Section~\ref{section:one-coincide-line}).  
If we have exactly two potential segments and both of them 
pass the height test, it is easily seen that the intersection curve consists 
of two conic components.  Because there are at most two diagonals,
with at most two interior tests,
we can find the two needed potential segments.  Then with at most
two height tests, the two conics can be determined, if the two cones
have conic intersection.

% ######################################################################
%                       DEGENERATE CASES
% ######################################################################

\section{Degenerate Conic Intersections}
\label{section:special-degenerate}
     In the last section, our algorithm was presented based on the assumption 
that the two skeletal pairs have exactly four intersection points.  However, 
the number of intersection points could be three, two, or even infinity.  In 
the next three sections, we consider these cases in turn.

% -------------------------------------------------------------------------
%                       Three Point Intersection
% -------------------------------------------------------------------------

\subsection{Three Point Intersection}
\label{section:three-point}
     There are two ways to get three points of intersection.  In the simplest 
case, one of the points is a vertex of a cone.  In this case, there are no
diagonals, and therefore no potential segment and no conic intersection.  In 
the other case, one line of a skeletal pair is parallel to a line from the 
other pair (see Figure~\ref{fig:three-point}).  We will call these parallel 
lines a {\em parallel pair}.  In effect, we still have four intersection 
points, with one point at infinity on the parallel pair.  One of the diagonal 
lines is parallel to the parallel pair, and the potential segment, if it 
exists, on this diagonal consists of only one infinite interval.  If a 
parallel pair is detected, one inside--outside test can determine which 
infinite interval is the desired potential segment.  
If the two surfaces have conic intersection, one of them must be a parabola
since it lies on a plane through the potential segment parallel to one of the
skeletal lines.  Since the major axis is known and the end point of the 
potential segment is the vertex of the parabola, one more point is enough to 
determine the parabola.  Pick any point on the potential segment for the 
height test.  The computation is the same as the four point case.
\begin{figure}
\vspace{7cm}
\caption{Several Examples of Three Point Intersection}
\label{fig:three-point}
\end{figure}

% -----------------------------------------------------------------------
%                         Two Point Intersection
% -----------------------------------------------------------------------

\subsection{Two Point Intersection}
\label{section:two-point}
     If the two skeletal pairs intersect in exactly two points, it is not
difficult to see that we must have two parallel pairs. (See two figures in
Figure~\ref{fig:potential-seg}.)  Since the axis is the
angle bisector of the cone angle, parallel skeletal lines imply that the axes
are also parallel and the half angles of the cones are equal, 
so the two cones are congruent.
Two cases can be distinguished: one of the vertices lies 
in the interior of the other cone (case 1), 
and no vertex lies in the interior of the other cone (case 2).  Note that 
since we have only two intersection points, no vertex can lie on the surface of
the other cone.  Using Figure~\ref{fig:potential-seg}, 
we see that there is exactly one potential segment.
In fact, the other two intersection points of the two parallel pairs are at 
infinity and thus the corresponding diagonal is a line at infinity.  We can 
prove, using the characterization theorem in 
Section~\ref{section:characterization}, that the conic intersection is an 
ellipse in case 1 above, and a hyperbola in case 2.  Based on
this fact, the  intersection conic curves can be computed using the two 
intersection points and the height of some point of the potential segment.  

% -----------------------------------------------------------------------
%                           One Coincide Line
% -----------------------------------------------------------------------

\subsection{One Common Tangent and an Isolated Point}
\label{section:one-coincide-line}
     If the intersection of skeletal lines has exactly one isolated point, we
must have a line in common and this is the common tangent of the two cones
(see Figure~\ref{fig:infinite}).  Note that this common tangent is counted 
twice and thus, using B\'{e}zout's
theorem, the residue curve must be a conic.  Therefore, if two lines, one from
each skeletal pair, coincide and the other two lines intersect at an isolated 
point, the intersection curves consist of the common tangent and a conic.

     In Figure~\ref{fig:infinite}, $P$ is the isolated point and the thick line
is the common tangent.  Some segments with $P$ at one end and a point on the 
common tangent at the other end are shown in the figure.  These segments lie in
the interior of both cones and hence are potential segments.  Therefore, the 
height test cannot be applied since there are infinite number of such segments.
In what follows we will illustrate a method to isolate the appropriate 
potential segment and compute the intersection conic upon it.
\begin{figure}
\vspace{4.5cm}
\caption{An Infinite Number of Potential Segments}
\label{fig:infinite}
\end{figure}

     The idea is simple.  Let ${\cal K}$ be a plane perpendicular to the axial
plane ${\cal H}$, parallel to the common tangent $C$, and intersecting both 
cones (see Figure~\ref{fig:proof-idea}).  Since ${\cal K}$ is parallel to a 
common tangent, it cuts the cones in two intersecting parabolas.  Since both 
parabolas are symmetric about ${\cal H}$, the projection of these intersection
points onto ${\cal H}$ gives a unique point $X$ with common height to both 
cones.  Since ${\cal K}\cap{\cal H}$ is parallel to the
common tangent $C$ and $X$ lies on ${\cal K}\cap{\cal H}$, the line 
$\stackrel{\longleftrightarrow}{PX}$ (recall that $P$ is the isolated point)
cannot be parallel to $C$ and must intersect $C$ at some point $P^\prime$.
Since $P$ and $P^\prime\in C$ lie on both surfaces, 
and $X$ has common height, if we know the type 
of the intersection conic, this conic is determined uniquely.  Therefore the 
plane through $P$ and $X$, and perpendicular to ${\cal H}$ cuts both cones in 
a common conic.  If the plane ${\cal K}$ is chosen carefully, the computation 
is straightforward.
\begin{figure}
\vspace{5cm}
\caption{Two Cones with a Common Tangent and an Isolated Point}
\label{fig:proof-idea}
\end{figure}

     The position of ${\cal K}$ and the type of the intersection conic are
determined by the relative positions of the cones. Suppose two 
non-congruent cones, ${\cal C}_1(V_1,\ell_1,\alpha_1)$ and
${\cal C}_2(V_2,\ell_2,\alpha_2)$, have a common tangent $C$, and
the two skeletal pairs intersect at an isolated point $P$.
Based on the location of the triangle $\bigtriangleup V_1V_2P$, we can
distinguish three cases: $\bigtriangleup V_1V_2P$ lies in the interiors of both
cones, $\bigtriangleup V_1V_2P$ lies in the exteriors of both cones, and
only one cone contains $\bigtriangleup V_1V_2P$ in its interior
(see Figure~\ref{fig:three-case}).  For the first and the third cases, 
${\cal K}$ can be any plane (satisfying the conditions mentioned at the 
beginning of the last paragraph) that has an intersection point in the segment
$\overline{PV_1}$.  The intersection conic is an ellipse.  For the second 
case, ${\cal K}$ must cut the ray $\stackrel{\longrightarrow}{PV_1}$ at a point
not in the segment $\overline{PV_1}$, and the intersection conic is a 
hyperbola.
\begin{figure}
\vspace{4cm}
\caption{The Positions of the Triangle $\bigtriangleup V_1V_2P$}
\label{fig:three-case}
\end{figure}

% #######################################################################
%                         LINEAR INTERSECTION
% #######################################################################

\section{Linear Intersections}
\label{section:linear}

     In this section, we show how to detect and compute linear intersections.
(Recall that under our definition, a linear intersection is an intersection 
whose components are all linear.)  Two cones can have at most two intersecting
lines or a common tangent. (It is interesting to note that a cylinder and a
cone can never have a linear intersection.)

\begin{lemma}
\label{lemma:cone-cone-linear}
     Two distinct cones intersect in two lines if and only if they have a
common vertex and one of the following propositions holds:
\begin{enumerate}
     \item On the axial plane, the two skeletal pairs coincide.
     \item On the axial plane, the two cone angles bounded by skeletal pairs
          overlap.
\end{enumerate}
\end{lemma}
{\bf Proof:} Suppose two cones have two lines in common.
The intersection point of these lines must be the common vertex, and hence
the axial plane is well-defined.  The axial plane intersects the cones in four
lines with a common point.  If the cones have no common interior point, we have
the first proposition; otherwise, the second proposition holds.

     For the converse, the first proposition is easy.  Consider the 
second proposition.  Assume the two cone angles overlap.  
Then the two cones must intersect in a curve.
Let $X$ be a point of this intersection, other
than the common vertex.  Let the cones be
${\cal C}_1(V,\ell_1,\alpha_1)$ and ${\cal C}_2(V,\ell_2,\alpha_2)$.
From $X$ drop 
perpendiculars to $\ell_1$ and $\ell_2$ meeting them at $F_1$ and $F_2$
respectively.  Let $d_1=|\overline{VF_1}|$ and $d_2=|\overline{VF_2}|$.
We have $|\overline{XV}|=d_1/\cos\alpha_1$ from cone ${\cal C}_1$,
and $|\overline{XV}|=d_2/\cos\alpha_2$ from cone ${\cal C}_2$.  Therefore
$d_2=\frac{\cos\alpha_2}{\cos\alpha_1}d_1$ holds.  Now it is not difficult
to see that the projection of $X$ lies on the intersection of the line
perpendicular to $\ell_1$ at $F_1$, and the line perpendicular to $\ell_2$ at
$F_2$.  This implies that the projection is determined by a linear relation of 
$d_1$ and hence the curve of intersection must be a line.  
Finally, the plane though this line and 
perpendicular to the axial plane cuts both cones in two common lines.  Note 
that this proof also provides a computation of the intersection 
lines.\ \ \ \QED

\begin{lemma}
\label{lemma:cone-cone-linear1}
     Two distinct cones intersect in one line if and only if one of the 
following holds.
\begin{enumerate}
     \item They have a common vertex and they are tangent along a common
          tangent line.
     \item They do not have a common vertex but one vertex lies on the other
          cone, they are congruent, and they
          have parallel axes.
\end{enumerate}
\end{lemma}
{\bf Proof:}  If two cones have a single line in common, they must be tangent 
to each other and therefore their axes are coplanar.  Let ${\cal H}$ be the 
axial plane which cuts the two cones in two pairs of lines.  Since the two 
cones are tangent, they share a skeletal line $T$, leaving three skeletal 
lines $T,L_1,L_2$.  If $L_1$ and $L_2$ intersect at the common vertex, we 
have one line intersection.  If they intersect, but not at the vertex, we have
the conic--line intersection that we have studied in
Section~\ref{section:one-coincide-line}.  If $L_1$ and $L_2$
are parallel, then the two cones are congruent.  The converse is not difficult
to see.\ \ \ \QED

     Using these two lemmas, we have a simple algorithm to test whether two 
cones have linear intersections.
\begin{enumerate}
     \item If the cones have a common vertex, the two lines of intersection
          lie in a plane perpendicular to the axial plane, and this plane
          can be found as has been discussed in 
          Lemma~\ref{lemma:cone-cone-linear}.  
          Figure~\ref{fig:two-lines-and-one-line} displays all possibilities.
     \item If the two cones are congruent with parallel axes, then
     \begin{itemize}
          \item If the vertex of ${\cal C}_1$ does not lie on ${\cal C}_2$, and
               vice versa, then ${\cal C}_1$ intersects ${\cal C}_2$ in a conic
               and see Section~\ref{section:two-point}.
          \item If the vertex of one cone lies on the other cone, then they
               are tangent along the line joining the two vertices.
     \end{itemize}
\end{enumerate}
\begin{figure}[h]
\vspace{10cm}
\caption{All Possibilities of Cone Intersections with A Common Vertex}
\label{fig:two-lines-and-one-line}
\end{figure}

% #########################################################################
%                    TANGENCY and DISJOINTEDNESS
% #########################################################################

\section{Tangency and Disjointedness}
\label{section:tangency-disjoint}
     In this section, we will study when two given cones are tangent to each
other in a finite number of points.  Let ${\cal C}_1(V_1,\ell_1,\alpha_1)$ and
${\cal C}_2(V_2,\ell_2,\alpha_2)$ be the two given cones.  The following 
establishes the necessary condition.

\begin{lemma}
\label{lemma:point-tangent-has-skew-axes}
     If two cones are tangent at an isolated point, their axes are skew.
\end{lemma}
{\bf Proof:}  If two cones, with vertices $V_1$ and $V_2$, are tangent at an
isolated point $P$, then the common tangent plane intersects the first and the
second cone at $\stackrel{\longleftrightarrow}{V_1P}$ and
$\stackrel{\longleftrightarrow}{V_2P}$, respectively.  Since $P$ is an isolated
point of tangency, the two half cones lie on different sides of the common
tangent plane.  Therefore their axes must be skew.\ \ \ \QED

     Without loss of generality, we will assume that $\ell_1$ and $\ell_2$ are 
skew and none of the two 
vertices lies in the other cone.  We shall find two planes that bound one cone
tightly and use the other cone to test the tangency against the two bounding 
planes.  From $V_2$, construct the two tangent planes to the cone 
${\cal C}_1$, $T_1$ and $T_2$ (see Figure~\ref{fig:cone-tangent}).  Note that 
since $T_i\ (i=1,2)$ is tangent to the cone ${\cal C}_1$ in a line, $T_i$ 
passes through $V_1$.  Thus $\stackrel{\longleftrightarrow}{V_1V_2}
=T_1\cap T_2$.  The two tangent planes divide the space into four quadrants.
Let the two quadrants containing ${\cal C}_1$ be ${\cal R}$.
\begin{figure}
\vspace{6cm}
\caption{The Construction for Testing Tangency}
\label{fig:cone-tangent}
\end{figure}

\begin{lemma}
\label{lemma:cone-in-region}
     If the axis of the cone ${\cal C}_2$ lies in region ${\cal R}$, the cones
intersect transversally.
\end{lemma}
{\bf Proof:}  Consider the plane containing $V_1$ and the axis of ${\cal C}_2$.
This plane lies in region ${\cal R}$, since the axis of ${\cal C}_2$ does.
Since this plane passes through both vertices, it cuts the two cones
in two pairs of intersecting lines.  If none of the four lines are parallel,
the two cones will intersect transversally.  If one line from each pair are 
parallel, the two remaining lines intersect both parallel lines.  Thus the 
two cones again intersect transversally.\ \ \ \QED

     Using the above lemma, we can assume that the axis $\ell_2$ does not lie 
in the region ${\cal R}$.  Pick any point $S$ from $\ell_2$, 
different from the vertex $V_2$.  From $S$ drop perpendiculars to planes $T_1$
and $T_2$ yielding $S_1\in T_1$ and $S_2\in T_2$.  Let $d_1=|\overline{SS_1}|,
d_2=|\overline{SS_2}|$ and $d=|\overline{SV_2}|$.  The following theorem 
characterizes the disjointedness and tangency of two cones using $d_1,d_2,d$ 
and the half angle $\alpha_2$.

\begin{theorem}[Tangency and Disjointedness]
\label{theorem:tangent-disjoint}
     Suppose ${\cal C}_1$ is tangent to $T_1$ and $T_2$ in two lines $t_1$ and
$t_2$ respectively.  We have the following characterizations for 
disjointedness and tangency for the cone--cone case.
\begin{enumerate}
     \item\label{case:disjoint} $d_1, d_2>d\sin\alpha_2$ : Two cones are 
          disjoint.
     \item\label{case:tangent-1} $d_1>d_2=d\sin\alpha_2$  : If 
          $t_2\cap\!\stackrel{\longleftrightarrow}{V_2S_2}\neq\emptyset$, the 
          two cones are tangent at exactly one point, 
          $t_2\cap\!\stackrel{\longleftrightarrow}{V_2S_2}$.  
          Otherwise, the cones are disjoint.
     \item\label{case:tangent-2} $d_2>d_1=d\sin\alpha_2$  : If 
          $t_1\cap\!\stackrel{\longleftrightarrow}{V_2S_1}\neq\emptyset$, the 
          two cones are tangent at exactly one point,
          $t_1\cap\!\stackrel{\longleftrightarrow}{V_2S_1}$.  
          Otherwise, the cones are disjoint.
     \item\label{case:two-point-tangent} $d_1=d_2=d\sin\alpha_2$ :  The two 
         cones are tangent at $t_1\cap\!\stackrel{\longleftrightarrow}{V_2S_1}$
         and $t_2\cap\!\stackrel{\longleftrightarrow}{V_2S_2}$.  One or both of
          these may not exist.  If both do not exist, the cones are disjoint.
     \item\label{case:intersect} $d_1,d_2< d\sin\alpha_2$ :
          Two cones intersect transversally.
\end{enumerate}
\end{theorem}
{\bf Proof:} The idea behind the proof is a sphere centered at $S$ with radius
$d\sin\alpha_2$.  This sphere is tangent to the cone ${\cal C}_2$.  If the
distances from $S$ to the two planes $T_1$ and $T_2$ are both greater than the
radius of the sphere, the cone lies in the interior of the regions that do not
contain the cone ${\cal C}_1$. Therefore proposition~\ref{case:disjoint} holds.

     Note that the cone ${\cal C}_2$ is tangent to $T_2$ if and only if 
$d\sin\alpha_2=d_2$.  $d_1>d_2=d\sin\alpha_2$ implies that 
${\cal C}_2$ does not intersect $T_1$, except at $V_2$.  Therefore, 
${\cal C}_2$ intersects $T_2$ in one line.  If this line, 
$\stackrel{\longleftrightarrow}{V_2S_2}$, intersects $t_2$, then ${\cal C}_1$ 
and ${\cal C}_2$ is tangent there since this is the only point at which they
intersect.  If these two lines do not intersect at all, the two cones do not 
intersect since they lie in different regions in space.  Therefore 
proposition~\ref{case:tangent-1} holds.  Similar arguments can prove 
proposition~\ref{case:tangent-2} and proposition~\ref{case:two-point-tangent}.

     For proposition~\ref{case:intersect}, the plane $T_2$ intersects the cone
${\cal C}_2$ in two lines.  They cannot both be parallel to $t_2$, which is
the common tangent line of ${\cal C}_1$ and $T_2$.  That is, one of these lines
must intersect $t_2$ and therefore the two cones cannot be tangent to each
other.\ \ \ \QED

% #########################################################################
%                     CHARACTERIZATION THEOREMS
% #########################################################################

\section{Characterization Theorems}
\label{section:characterization}
     In this section, we will present some characterization results for conic
intersection.  Characterization theorems provide necessary and sufficient
conditions for conic intersection.  
Recall that a conic is the intersection of a cone with a plane.
If a sphere, called the Dandelin sphere or the focal sphere, is inscribed in 
the cone tangent to the plane, the tangent point of the inscribed sphere and 
the plane is a focus of the conic (see 
Figure~\ref{fig:dandelin-sphere}).\footnote{It
was Germinal P. Dandelin~\cite{dandelin:1822}, a Belgian mathematician, who
discovered the relationship of the inscribed sphere and the focus of the
conic section.  Dandelin's beautiful idea is covered in 
Hilbert and Cohn--Vossen~\cite{hilbert:1983}.  However, the complete 
foci-directrix relation was found by Pierce Morton~\cite{morton:1830} a few 
years later.  Refer to Drew~\cite{drew:1875} and Macaulay~\cite{macaulay:1895}
for geometric development, and Taylor~\cite{taylor:1881} for historical
material along this line.}  Two spheres can be found, yielding the two foci of
the conic. (In the case of the parabola, only one sphere can be found.)   
If two cones have a conic intersection, this conic lies on a plane, which is 
the plane defining the conic for both cones.  Therefore, in general, we can 
find two pairs of Dandelin spheres tangent to the plane at the same point,
one pair for each focus.  If one of the two cones is replaced by a cylinder, 
the same conclusion is true, but the intersection conic can only be an 
ellipse (Figure~\ref{fig:dandelin-sphere}).  
In summary, we have the following theorem.
% ############################################################################
% #   The following figure using a direct citation, Hilbert and Cohn-Vossen   #
% #   [8].  This '[8]' should be replaced with some other number if bib.     #
% #   items are changed.                          Mar/02/1991  C.-K. Shene   #
% ############################################################################

\begin{figure}
\vspace{6cm}
\caption{The Dandelin Spheres (Taken from Hilbert and Cohn--Vossen [8])}
\label{fig:dandelin-sphere}
\end{figure}

\begin{theorem}[General Characterization]
\label{theorem:general-characterization}
     If two axial natural quadrics have distinct and coplanar axes, the
following propositions are equivalent:
\begin{itemize}
     \item These two surfaces have conic intersection.
     \item There exists a potential segment passing the height test.
     \item There exists a plane and a sphere inscribed in each 
          surface such that both spheres and the plane are tangent 
          at the same point.
\end{itemize}
\end{theorem}

     If the Dandelin spheres and the plane are projected to the axial plane, 
they become circles and a line and the above theorem can be restated as 
follows.

\begin{corollary}
\label{corollary:planar-dandelin-sphere}
     If two axial natural quadrics have distinct and coplanar axes,  the
following propositions are equivalent:
\begin{itemize}
     \item These two surfaces have conic intersection.
     \item There exists a line and a circle inscribed in each skeletal pair
          such that both circles and the line are tangent at the same point.
\end{itemize}
\end{corollary}

     If the two axes intersect, Theorem~\ref{theorem:general-characterization}
and Corollary~\ref{corollary:planar-dandelin-sphere} can be simplified further.
We can always find a sphere inscribed in {\em both} surfaces, or a circle 
inscribed in {\em both} skeletal pairs.  

\begin{theorem}[Intersecting Axes]
\label{theorem:intersecting-axes-characterization}
     Suppose the two axes intersect.  The following propositions are
equivalent:
\begin{itemize}
     \item The two surfaces have conic intersection.
     \item There exists a sphere inscribed in both surfaces.
     \item There exists a circle inscribed in both pairs of the skeletal lines.
\end{itemize}
\end{theorem}
{\bf Proof:} We give a proof for the simplest case.
Proofs for the other cases are similar.
Let ${\cal C}_1(V_1,\ell_1,\alpha_1)$ and ${\cal C}_2(V_2,\ell_2,\alpha_2)$
be two cones, with intersecting axes, that have conic intersection.
Corollary~\ref{corollary:planar-dandelin-sphere} guarantees that, on the axial
plane, there exists a line and a circle inscribed in each skeletal pair such
that both circles and the line are tangent at the same 
point (Figure~\ref{fig:inscribed-circle}).  For simplicity,
we will assume 1) the intersection curve upon the potential segment
$\overline{RS}$ is an ellipse, 2) $V_1$ and $V_2$ lie on different sides of 
$\stackrel{\longleftrightarrow}{RS}$, 3) the two circles are tangent at some 
point of $\overline{RS}$ and hence they are the inscribed circles of 
$\bigtriangleup V_1RS$ and $\bigtriangleup V_2RS$, and 4) $V_1RV_2S$ is a
convex quadrangle.  Now we have
$|\overline{V_1R}|+|\overline{V_2S}|=|\overline{V_1S}|+|\overline{V_2R}|$ and
this relation implies, in turn, that there exists a circle inscribed in the
quadrangle $V_1RV_2S$.
(See, for example, Henrici--Treutlein~\cite{henrici:1897},
p. 73.)  

     Conversely, suppose convex quadrangle $V_1RV_2S$ has an inscribed
circle.  It is easily seen that $|\overline{V_1R}|+|\overline{V_2S}|=
|\overline{V_1S}|+|\overline{V_2R}|$ holds.  Let the inscribed circle of
$\bigtriangleup V_1RS$ (resp., $\bigtriangleup V_2RS$) be tangent to
$\overline{RS}$ at $T_1$ (resp., $T_2$).  Then for $\bigtriangleup V_1RS$ 
(resp.,
$\bigtriangleup V_2RS$) we have $|\overline{RT_1}|=
\frac{1}{2}(|\overline{V_1R}|+|\overline{V_1S}|+|\overline{RS}|)-
|\overline{V_1S}|$ (resp., $|\overline{RT_2}|=
\frac{1}{2}(|\overline{V_2R}|+|\overline{V_2S}|+|\overline{RS}|)-
|\overline{V_2S}|$).  Hence $|\overline{RT_1}|=|\overline{RT_2}|$ holds and 
the two inscribed circles are tangent at the same point $T_1=T_2$.  However,
the inscribed circle of the quadrangle $V_1RV_2S$ is no more than the great
circle of the common inscribed sphere of the two given cones cut by the 
axial plane.  Therefore, having a pair of
tangent Dandelin spheres is equivalent to having a common inscribed sphere for
both cones.\ \ \ \QED
\begin{figure}[h]
\vspace{4cm}
\caption{The Inscribed Circles of $\bigtriangleup V_1RS$ and $\bigtriangleup V_2RS$,
and the Common Inscribed Circle of the Quadrangle $V_1RV_2S$}
\label{fig:inscribed-circle}
\end{figure}
     
     The inscribed sphere gives the Goldman--Miller 
criteria~\cite{goldman:1990} immediately.

% ====================================================================
% Comment out to save space (Mar/01/1991)
% ====================================================================
%
%\begin{corollary}\ \                            % ***** ADD SPACE
%\begin{enumerate}
%     \item Given two cones ${\cal C}_1(V_1,\ell_1,\alpha_1)$ and 
%          ${\cal C}_2(V_2,\ell_2,\alpha_2)$ with 
%          $\ell_1\cap\ell_2\neq \emptyset$,
%          they have conic intersection if and only if
%          $d_1\sin\alpha_1=d_2\sin\alpha_2$ holds, where $d_i$ is the distance
%          from $\ell_1\cap\ell_2$ to $V_i$.
%     \item Given a cone ${\cal C}(V,\ell_1,\alpha)$ and a cylinder 
%          ${\cal Z}(\ell_2,r)$ with $\ell_1\cap\ell_2\neq \emptyset$,
%          they have conic intersection if and only if $d\sin\alpha=r$ 
%          holds, where $d$ is the distance from $\ell_1\cap\ell_2$ to $V$.
%     \item Given two cylinders ${\cal Z}_1(\ell_1,r_1)$ and
%          ${\cal Z}_2(\ell_2,r_2)$ with $\ell_1\cap\ell_2\neq \emptyset$,
%          they have conic intersection if and only if $r_1=r_2$ holds.
%\end{enumerate}
%\end{corollary}

     The common inscribed sphere criteria are very useful.  Let us examine a
limiting case of these 
criteria. If two cones ${\cal C}_1(V_1,\ell_1,\alpha_1)$ 
and ${\cal C}_2(V_2,\ell_2,\alpha_2)$ have conic intersection, they have a 
common inscribed sphere.  Let us fix ${\cal C}_1$ and the vertex $V_2$ of
${\cal C}_2$, and move the inscribed sphere along $\ell_1$.  Then the center
goes to infinity, the length of the radius approaches infinity, 
and ${\cal C}_2$ rotates about $V_2$.  On the axial plane, eventually, the two
skeletal pairs become parallel to each other as a limiting case.  Hence, the
two cones become congruent to each other.  This intuition shows that, for two
cones with parallel axes, they have conic intersection if and only if they
have an inscribed sphere with center at infinity and infinite length radius,
if and only if they are congruent.  We have mentioned this fact in
Section~\ref{section:two-point}.

% #######################################################################
%                          CONCLUSIONS
% #######################################################################

\section{Conclusions}
\label{section:conclusion}
     In order to establish the structure of the method and the interaction
of the sections, we give an outline of the algorithm in Figure~\ref{fig:algo}.
\begin{figure}
{\small
\begin{tabbing}
     xxxxx\=xxxxx\=xxxxx\=xxxxx\=xxxxx\=xxxxx\=\kill
     {\bf PROCEDURE} PlanarIntersection; \\
     \\
     {\bf BEGIN} \\
     \> {\bf IF} axes are coplanar {\bf THEN} \\
     \>\> {\bf CASE} skeletal lines intersection type {\bf OF} \\
     \>\>\>  4 points : see Section 2; \\
     \>\>\>  3 points : see Section 3.1; \\
     \>\>\>  2 points : see Section 3.2; \\
     \>\>\>  1 point and a line : see Section 3.3; \\
     \>\>\>  2 lines : see Section 4; \\
     \>\>\>  1 line  : see Section 4; \\
     \>\>\>  1 point : see Section 4; \\
     \>\> {\bf END} \\
     \> {\bf ELSE} \\
     \>\>  see Section 5 \\
     {\bf END} \\
\end{tabbing}
}
\caption{The Structure of the Method and the Interaction of the Sections}
\label{fig:algo}
\end{figure}

     Using only elementary geometry, this paper has successfully developed a 
fully geometric technique for detecting and computing the planar intersections
of two natural quadric surfaces.  The fundamental method involves a height
test from a segment in the axial plane.

     We include a word about the robustness of the methods in this paper.
The basic method of Section~\ref{section:algorithm} is very robust: the more 
accurately the potential segment and height are computed, the better the 
accuracy of the conic intersection.
The computation degrades gracefully as these are computed less accurately.
The degenerate cases and linear intersections of 
Sections~\ref{section:special-degenerate} and~\ref{section:linear} involve
some fragile conditions: parallel skeletal lines and common vertices.
However, these degrade gracefully into other cases (of Section 2): for example,
if the lines are not parallel, then the conic is a very large ellipse
rather than a parabola, where the parabola is the limit case of the ellipse.
By definition, the tangency conditions of 
Section~\ref{section:tangency-disjoint} are limit conditions
between intersections and disjointedness, and our methods reflect this.

     The techniques of this paper can be generalized to cover more general 
quadric surfaces, and we are working on this problem.  The height test theorem 
(Theorem~\ref{theorem:conic-conditions-using-height-test}) holds for 
revolutionary quadrics.  However, some lemmas presented here will not hold for
more general surfaces; for example, tangency need not imply skew axes, and 
planar intersection need not imply coplanar axes.
Another interesting question is whether one can compute the higher degree
intersection curves of natural quadrics using the techniques presented in this
paper.

% ######################################################################
%                             REFERENCES
% ######################################################################


\begin{thebibliography}{999}

\bibitem{dandelin:1822}
     G. P. Dandelin,
     M\'{e}moire sur quelques propri\'{e}t\'{e}s remarquables de la Focal
     Parabolique,
     {\em Nouveaux M\'{e}moires de l'Acad\'{e}mie Royale des Sciences et
     Belles--lettres de Bruxelles},
     Vol. 2 (1822), pp. 171--202.

\bibitem{drew:1875}
     William H. Drew,
     {\em A Geometrical Treatise on Conic Sections},
     fifth edition,
     Macmillan and Co., 1875.

\bibitem{farouki:1989}
     R. T. Farouki, C. A. Neff and M. A. O'Connor,
     Automatic Parsing of Degenerate Quadric-Surface Intersection,
     {\em ACM Transaction on Graphics},
     Vol. 8 (1989), No. 3, pp. 174--203.

\bibitem{goldman:1990}
     Ronald N. Goldman and James R. Miller,
     Detecting and Calculating Conic Sections in the Intersection of Two
     Natural Quadric Surfaces, Part I: Detection,
     Draft, October 1990.

\bibitem{goldman-warren:1990}
     Ronald N. Goldman and Joe D. Warren,
     A Sufficient Condition for the Intersection of Two Quadrics of 
     Revolution to Degenerate into a Pair of Conic Sections,
     Technical Report, Rice University, December 1990.

\bibitem{hakala:1980}
     D. G. Hakala, R. C. Hillyard, B. E. Nourse and P. J. Malraison,
     Natural Quadrics in Mechanical Design,
     {\em Proceedings of Autofact West 1}, 
     Anaheim, CA., Nov. 1980, pp. 363--378.

\bibitem{henrici:1897}
     J. Henrici and P. Treutlein,
     {\em Lehrbuch der Elementar--Geometrie}, Vol. I,
     Teubner, Leipzig, 1897.

\bibitem{hilbert:1983}
     D. Hilbert and S. Cohn--Vossen,
     {\em Geometry and the Imagination},
     translated by P. Nememyi,
     Chelsea, New York, 1983.

\bibitem{levin:1976}
     Joshua Zev Levin,
     A Parametric Algorithm for Drawing Pictures of Solid Objects Composed
          of Quadric Surfaces,
     {\em Communications of ACM},
     Vol. 19(1976), No. 10, pp. 555--563.

\bibitem{levin:1979}
     Joshua Zev Levin,
     Mathematical Models for Determining the Intersections of Quadric Surfaces,
     {\em Computer Graphics and Image Processing},
     Vol. 11(1979), pp. 73--87.

\bibitem{macaulay:1895}
     Francis S. Macaulay,
     {\em Geometrical Conics},
     Cambridge University Press, London, 1895.

\bibitem{miller:1987}
     James R. Miller,
     Geometric Approaches to Nonplanar Quadric Surface Intersection Curves,
     {\em ACM Transactions on Graphics},
     Vol. 6(1987), No. 4, pp. 274--307.

\bibitem{morton:1830}
     Pierce Morton,
     On the Focus of a Conic Section,
     {\em Transactions of the Cambridge Philosophical Society},
     Vol. 3 (1827--1830), pp. 185--191.

\bibitem{ocken:1987}
     S. Ocken, Jacob T. Schwartz and Micha Sharir,
     Precise Implementation of CAD Primitives Using Rational Parameterizations
     of Standard Surfaces,
     in {\em Planning, Geometry, and Complexity of Robot Motion}, edited by
     Jacob T. Schwartz, Micha Sharir and John Hopcroft,
     Ablex Publishing Co., Norwood, New Jersey, 1987, pp. 245--266.

\bibitem{piegl:1989}
     Les Piegl,
     Geometric Method of Intersecting Natural Quadrics Represented in
     Trimmed Surface Form,
     {\em Computer Aided Design}, Vol. 21 (1989), No. 4, pp. 201--212.

\bibitem{pratt:1990}
     M. J. Pratt,
     Cyclides in Computer Aided Geometric Design,
     {\em Computer Aided Geometric Design},
     Vol. 7 (1990), pp. 221-242.

\bibitem{sarraga:1983}
     Ramon F. Sarraga,
     Algebraic Methods for Intersections of Quadric Surfaces in GMSOLID,
     {\em Computer Vision, Graphics, and Image Processing},
     Vol. 22(1983), pp. 222--238.

\bibitem{taylor:1881}
     Charles Taylor,
     {\em An Introduction to Ancient and Modern Geometry of Conics},
     Cambridge University Press, London, 1881.

\bibitem{warren:1987}
     Joe Warren,
     Blending Quadric Surfaces with Quadric and Cubic Surfaces,
     {\em Proceedings of the Third Annual Symposium on Computational Geometry},
     Waterloo, Ontario, Canada 1987, pp. 341--347.

\end{thebibliography}

% ##############################################################
%                       THE END
% ##############################################################

\end{document}
