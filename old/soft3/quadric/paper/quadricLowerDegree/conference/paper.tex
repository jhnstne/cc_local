\documentstyle [12pt]{article}
\title{
     On the Planar Intersection of Natural Quadrics}

\author{Ching-Kuang Shene\thanks
          {Department of Computer Science, The Johns Hopkins University,
          Baltimore, Maryland 21218.  This work was supported by
          National Science Foundation grant IRI--8910366.}\ \ and
        John K. Johnstone${^*}$
       }

\date{\ }

\newtheorem{example}{Example}[section]
\newtheorem{property}{Property}[section]
\newtheorem{definition}{Definition}[section]
\newtheorem{theorem}{Theorem}[section]
\newtheorem{lemma}{Lemma}[section]
\newtheorem{corollary}{Corollary}[section]

\newcommand{\DoubleSpace}{\edef\baselinestretch{1.4}\Large\normalsize}
\newcommand{\QED}{\rule{2mm}{3mm}}

%\DoubleSpace
\setlength{\oddsidemargin}{0pt}
\setlength{\evensidemargin}{0pt}
\setlength{\headsep}{0pt}
\setlength{\topmargin}{0pt}
\setlength{\textheight}{8.75in}
\setlength{\textwidth}{6.5in}


\begin{document}
\maketitle

% #####################################################################
%                        ABSTRACT
% #####################################################################

\begin{abstract}
     In general, two quadric surfaces intersect in a space quartic curve.
Under certain circumstances, the intersection curve degenerates to  some
special planar forms, which occur frequently.  These degenerate cases usually 
introduce problems into the general intersection algorithms and are difficult 
to implement in a reliable way.  In this paper, we investigate this problem 
for natural quadrics.  Three algorithms are presented: 1) to detect and 
compute tangency and disjointedness, 2) to compute linear intersections and 
3) to compute conic intersections.  Techniques presented here are considerably
different from traditional methods in that only elementary geometric arguments
are used.  Therefore these algorithms are easily implementable and can be 
understood easily without any algebraic geometry background.  Geometric 
arguments can also help us to uncover more properties of the intersection of 
two natural quadrics.
\end{abstract}

% ######################################################################
%                       INTRODUCTION
% ######################################################################


\section{Introduction}
\label{section:introduction}
     In general, two quadric surfaces intersect in a space quartic curve.
Under certain circumstances, the intersection curve degenerates to  some
special planar forms, which occur frequently.  These degenerate cases usually 
introduce problems into the general intersection algorithms and are difficult 
to implement in a reliable way.  In this paper, we will investigate this 
problem for natural quadrics, using fully geometric methods. 

     Natural quadrics consist of the spheres, the circular cylinders and 
the right cones.  Ever since Hakala's 1980 paper~\cite{hakala:1980}, many 
papers have been dedicated to this very restricted class because, in most CAD 
systems, they are still the most useful and natural objects to model 
mechanical parts.  During the past decade, starting with 
Levin~\cite{levin:1976,levin:1979}, algebraic methods based on the pencil of 
two quadrics have been used successfully to attack the intersection problem.  
Other examples are Miller~\cite{miller:1987}, 
Ocken--Schwartz--Sharir~\cite{ocken:1987} and Sarraga~\cite{sarraga:1983}.  
Only recently people have started to look at the conditions that cause 
degenerate intersection curves.  Farouki--Neff--O'Connor~\cite{farouki:1989} 
is the first complete study using the pencil technique.  Their paper deals 
with the general quadric surfaces, not the natural quadrics.  To the best of 
our knowledge, Goldman--Miller~\cite{goldman:1990} is the only paper 
concerning the conditions that cause the intersection of two natural quadrics 
to degenerate to a conic.  

     In the Goldman--Miller paper, Levin's method is used to establish some 
simple conditions for conic intersections.  Although Levin's method is elegant,
it is not without problems.  Since it is basically a local method, results 
obtained from it may not be the most appropriate ones.  Furthermore, complex 
arithmetic must also be carefully avoided in order not to increase the 
computational complexity.  Therefore, methods based on Levin are usually 
involved and lack geometric intuition.  This may be good for the general case,
but simpler methods are possible with natural quadrics.  Because of this and 
our belief that geometric methods usually uncover geometric insight, we will 
use elementary geometry to handle the intersection problem.

    Goldman and Miller focus on the detection of planar intersections and leave
the calculation to a second part, whereas detection and computation are done at
the same time with our method.   We also consider the detection and computation
of one point tangency and disjoint quadrics.  Our approach is a complete 
geometric analysis in that only geometric arguments are used.  Therefore, the 
result is in a higher level form and does not depend on the representation 
(e.g., parametric or implicit).  This provides an advantage over the algebraic
method in that it can be adopted to any representation as long as that 
representation is capable of calculating the intersection of lines and planes,
determining whether a point lies inside a surface, and computing the distance 
from a point to a surface along a specific direction.

     The paper is divided into eight sections.  
In Section~\ref{section:algorithm}, our algorithm for detecting and 
calculating conic intersections is presented in its most general form without 
concern for degenerate cases.  Section~\ref{section:special-degenerate} 
discusses how to handle the degenerate cases so that the presentation of our 
algorithm is complete.  Section~\ref{section:linear} presents a simple 
algorithm  for linear intersections.  Section~\ref{section:tangency-disjoint} 
handles another important problem, the test for tangency or disjointedness.  
In Section~\ref{section:characterization}, some characterization 
theorems are presented.  These theorems are generalizations of those obtained 
in Goldman--Miller's paper, and give geometric insights into the intersection 
problem.  Section~\ref{section:conclusion} gives our conclusion and an overview
of the method.

     For simplicity, we concentrate on the intersection of two cones.  The 
intersection of a sphere with another surface is very simple.  Results for the
cylinder can be obtained easily from the cone algorithms, since a cylinder is 
a cone with vertex at infinity.  
\begin{table}
\begin{center}
\caption{Possible Intersection Types of Two Natural Quadrics}
\label{tbl:intersection-type}
\vspace{3mm}
\begin{tabular}{||c|c|c||} \hline
Number of Conics & Number of Lines & Number of Points \\ \hline
  2 & 0 & 0 \\ \hline
  1 & 1 & 0 \\ \hline
  0 & 1(tangent line),2 & 0 \\ \hline
  0 & 0 & 1,2 \\ \hline
\end{tabular}
\end{center}
\end{table}

     Since we are concerned with planar intersections, the case of a line and 
a space cubic, as well as all quartic space curves, will not be considered.  
With a simple geometric argument, it can be shown that the planar intersections
in Table~\ref{tbl:intersection-type} are the only possibilities.

     In the rest of this paper, we will use the following notations.
\begin{itemize}
     \item A {\em natural} quadric is a sphere, a circular cylinder, or a 
          right cone.
     \item An {\em axial} quadric is a natural quadric with an axis, namely 
          a cylinder or a cone.
     \item ${\cal C}(V,\ell,\alpha)$ is the cone with vertex $V$, axis $\ell$
          and half angle $\alpha$.
     \item $\stackrel{\longleftrightarrow}{AB},
          \stackrel{\longrightarrow}{AB},\overline{AB}$ and
          $|\overline{AB}|$ are the line, the half ray, the segment, and the 
          length of the segment determined by two points $A$ and $B$.
     \item A {\em conic} intersection is a composite intersection curve such
          that the highest degree among all component curves is two (Case 1 
          and 2 in Table~\ref{tbl:intersection-type}).
     \item A {\em linear} intersection is a composite intersection curve such
          that the highest degree among all component curves is one (Case 3 
          of Table~\ref{tbl:intersection-type}).
     \item A {\em point} intersection is a composite intersection such that the
          components are all isolated points (Case 4 of 
          Table~\ref{tbl:intersection-type}).
     \item {\em Tangency} means a point intersection.  Hence a common tangent
          line is a linear intersection.
     \item Two surfaces are {\em congruent} if and only if there exists a 
          rigid motion transforming one into the other.
\end{itemize}

% #########################################################################
%                          MAIN ALGORITHM
% #########################################################################


\section{The Main Algorithm}
\label{section:algorithm}
     In this section, we will present our conic intersection algorithm in its 
most general form, ignoring degenerate cases.  For simplicity, we motivate the
algorithm using only the cone--cone case, although it works for all axial 
quadrics.  The idea is to  reduce the surface intersection problem to a simple 
planar line intersection problem; second, determine whether the intersection 
curve is a conic via some simple inside-outside and on-surface tests and, if 
so, compute the equation of the intersection conic.

     The following lemma shows that if the intersection of two axial natural 
quadric surfaces contains a conic, their axes must be coplanar.

\begin{lemma}
\label{lemma:non-skew}
     If two axial natural quadrics have a conic intersection component, their 
axes are coplanar.  
\end{lemma}
{\bf Proof:}  Let $Q_1$ and $Q_2$ be two axial natural quadrics.  Since $Q_1$ 
is revolutionary, the plane determined by the major axis of the intersection 
conic and the axis of $Q_1$ is perpendicular to the plane containing the 
intersection conic.  Let this plane be $P_1$ and let $P_2$ be the analogous 
plane for $Q_2$.  Because both quadrics contain the intersection conic, $P_1$ 
and $P_2$ must coincide.  Hence, the axes are either parallel, coincide, or 
intersecting.  In order words, they are not skew.\ \ \ \QED

     Let $Q_1$ and $Q_2$ be two axial quadrics with distinct coplanar axes.
Since the axes are coplanar, they determine a plane, called the {\em axial 
plane}.  This plane cuts each surface in a pair of straight lines, called the 
{\em skeletal pair}.  In general, these two 
pairs of lines intersect each other in four points, and this is the 
non-degenerate case that will be discussed in this section.  The lines joining
these four points that do not coincide to the original lines are called the 
{\em diagonals} (Figure~1).
The collection of diagonals and skeletal pairs are called the
{\em skeletal lines} on the axial plane.  Each diagonal is cut by the two 
intersection points into three parts, one {\em finite segment} and two 
{\em infinite segments}.  If either finite or infinite segment lies entirely 
in the interior of both surfaces, it is called a {\em potential segment}.  
Note that, since the end points of a potential segment lie on the intersection 
of the two skeletal pairs and thus on the intersection of the two surfaces, 
and the cylinder and the half cone is convex, if any point of the segment lies 
in the interior of both surfaces, so do all interior points of the segment.  
Hence, in order to test whether any one of the segments lies in the interior 
of the intersection, simply test an arbitrary interior point of the segment.

     In general, we have one or two potential segments; however, a potential
segment may not exist at all.  Furthermore, the number of potential segments,
in some degenerate cases, could be infinite.   
In Figure~\ref{fig:potential-seg} we show some skeletal lines of two cones,
with potential segments indicated by thick lines.

     Suppose the two axial quadrics, $Q_1$ and $Q_2$, have a conic 
intersection.  For convenience, let it be an ellipse $E$.  The plane containing
$E$, $E_P$, is perpendicular to the axial plane ${\cal H}$ (see 
Figure~\ref{fig:cone-cone-example}).  Note that the two intersecting surfaces,
and thus the intersection curve, are symmetric about the axial plane 
${\cal H}$.  Therefore $E$ is cut by ${\cal H}$ into two congruent halves.
Let us consider the vertical projection of $E$ onto ${\cal H}$.  Obviously, 
this vertical projection lies in the line $E_P\cap{\cal H}$.  Furthermore,
the two points $E\cap{\cal H}$ lie on both surfaces and thus are the 
intersection points of the skeletal lines.  $E_P\cap{\cal H}$ must be a 
diagonal, since it is a skeletal line and it does not lie on the intersection 
of the surfaces.  However, since the ellipse $E$ belongs to both surfaces and 
its vertical projection onto ${\cal H}$ lies in the interior of both surfaces,
the projection of $E$ onto ${\cal H}$ is a potential segment.  If the 
intersection curve $E$ is a hyperbola or a parabola, the projection onto 
${\cal H}$ would be an infinite interval with two components or only one 
component.  These observations allow us to develop an efficient conic 
intersection algorithm.

     Let us find a diagonal and the potential segment on it.  If no potential 
segment can be found, the above observation tells us that there cannot be any 
conic intersection.  If we do find a potential segment, pick any point on it 
and let it be $X$.  The {\em height} from $X$ to either surface is the 
distance from $X$ to that surface, measured along the line perpendicular to 
${\cal H}$ and through $X$ (see Figure~\ref{fig:cone-cone-example}).  This is 
well-defined, because both surfaces are symmetric about ${\cal H}$ and the 
components of the interior of the surface containing $X$ are both convex.  
Hence the perpendicular to ${\cal H}$ through $X$ hits either surface in two 
points and no matter which point is chosen, the height is the same.  

\begin{definition}
     $X$ passes the {\em height test} if and only if the heights to both 
surfaces from $X$ are equal.
\end{definition}

     We claim that if the chosen point $X$ passes the height test, then the two
surfaces have a conic intersection.   On the plane through the potential 
segment and perpendicular to ${\cal H}$, there are two conics, one from each 
axial surface.  Let these conics be called $E_1$ and $E_2$.  Note that these 
two conics project onto the same potential segment (see 
Figure~\ref{fig:height-test}).  Let the heights from $X$ to the surfaces be 
$h_1$ and $h_2$.  Recall from elementary geometry that if we know the type of a
central conic, the two end points of its major axis and one more point on the 
conic, then the conic is uniquely determined.\footnote{This can be easily seen
using some algebra.  The distance between the two end points is $2a$, where $a$
is the length of the major axis.  Then the central conic has equation 
$x^2/a^2\pm y^2/b^2=1$, where $b$ is the unknown length of the minor axis.  
Plugging the third point into this equation and solving for $b$, we have the 
complete equation.}  Thus, if $X$ passes the height test, the two ellipses 
$E_1$ and $E_2$ must coincide.  In summary, we have a stronger form of
Lemma~\ref{lemma:non-skew} for the coplanar axes case.

\begin{theorem}
\label{theorem:conic-conditions-using-height-test}
     Two axial natural quadrics with distinct and coplanar axes have a conic 
intersection component if and only if there exists a potential segment such 
that any point of it passes the height test.
\end{theorem}

     If two quadric surfaces intersect in a conic, then the residue curve (the
other component of the intersection) must be a conic also, perhaps a 
degenerate one.  If we have exactly two potential segments and both of them 
pass the height test, it is easily seen that the intersection curve consists 
of two conic components.  
Theorem~\ref{theorem:conic-conditions-using-height-test} leads to an algorithm
for detecting and calculating conic intersection in the non-degenerate case, 
which is given in Table~\ref{table1}.

\begin{table}
\label{table1}
\begin{minipage}{5in}
{\small
\begin{itemize}
     \item {\bf Input:} Two intersecting axial natural quadrics with 
          distinct and coplanar axes.
     \item {\bf Output:} If the intersection contains a conic component, this
          algorithm finds the intersection conic; Otherwise, it will report
          that there is no such intersection.
     \item {\bf Algorithm:}
     \begin{tabbing}
          xxxxx\=xxxxx\=xxxxx\=xxxxx\=xxxxx\=\kill
          \> {\bf BEGIN} \\
          \>\>  Let ${\cal H}$ be the axial plane; \\
          \>\>  Let $\ell_{11}$ and $\ell_{12}$ be the intersection lines
                    of ${\cal H}$ and the first surface; \\
          \>\>  Let $\ell_{21}$ and $\ell_{22}$ be the intersection lines
                    of ${\cal H}$ and the second surface; \\
          \>\>  Let $P_{11,21}=\ell_{11}\cap \ell_{21}, 
                     P_{12,21}=\ell_{12}\cap \ell_{21}$; \\
          \>\>  Let $P_{11,22}=\ell_{11}\cap \ell_{22}, 
                     P_{12,22}=\ell_{12}\cap \ell_{22}$; \\
          \>\>  IF $\overline{P_{11,21}P_{12,22}}$ is a potential segment 
                    {\bf THEN} \\
          \>\>\>    {\bf BEGIN} \\
          \>\>\>\>      Do a height test with the midpoint $M_1$ of \\
          \>\>\>\>\>         the segment $\overline{P_{11,21}P_{12,22}}$; \\
          \>\>\>\>      If it passes the test, we have an ellipse with \\
          \>\>\>\>\>         major axis length
                             $\frac{1}{2}|\overline{P_{11,21}P_{12,22}}|$ and\\
          \>\>\>\>\>         minor axis as determined in the text; \\
          \>\>\>\>      Otherwise, there is no conic intersection. \\
          \>\>\>    {\bf END} \\
          \>\>  {\bf ELSE IF} the infinite part is a potential segment {\bf THEN}\\
          \>\>\>    {\bf BEGIN} \\
          \>\>\>\>      Do a height test with any point $M_2$ of the infinite
                        segment; \\
          \>\>\>\>      If it passes the test, we have an hyperbola with \\
          \>\>\>\>\>         major axis length
                             $\frac{1}{2}|\overline{P_{11,21}P_{12,22}}|$ and\\
          \>\>\>\>\>         minor axis as determined in the text; \\
          \>\>\>\>      Otherwise, there is no conic intersection. \\
          \>\>\>    {\bf END} \\
          \>\> Do the same thing using the other diagonal
               $\stackrel{\longleftrightarrow}{P_{11,22}P_{12,21}}$; \\
          \> {\bf END}.
     \end{tabbing}
\end{itemize}
}
\end{minipage}
\vspace{3mm}
\caption{General Algorithm for Conic Intersection}
\end{table}

% ######################################################################
%                       DEGENERATE CASES
% ######################################################################

\section{Degenerate Conic Intersections}
\label{section:special-degenerate}
     In the last section, our algorithm was presented based on the assumption 
that the two skeletal pairs have exactly four intersection points.  However, 
the number of intersection points could be three, two, or even infinity.  In 
the next three sections, we consider these cases in turn.

% -------------------------------------------------------------------------
%                       Three Point Intersection
% -------------------------------------------------------------------------

\subsection{Three Point Intersection}
\label{section:three-point}
     There are two ways to get three points of intersection.  In the simplest 
case, one of the points is a vertex of a cone.  In this case, there are no
diagonals, and therefore no potential segment and no conic intersection.  In 
the other case, one line of a skeletal pair is parallel to a line from the 
other pair.  We will call these parallel lines a {\em parallel pair}.  In 
effect, we still have four intersection points, with one point at infinity on 
the parallel pair.  One of the diagonal lines is parallel to the parallel pair,
and the potential segment, if it exists, on this diagonal consists of only one
infinite interval (see Figure~\ref{fig:three-point}).  If the two surfaces have
a conic intersection, this conic on the plane through the potential segment 
must be a parabola since the plane is parallel to one of the skeletal lines.  
Since the major axis is known and the end point of the potential segment is the
vertex of the parabola, one more point would be enough.  Therefore, if a 
parallel pair is detected, one inside--outside test can determine which 
infinite interval is the desired potential segment.  Pick any point on this 
potential segment for the height test.  The computation is the same as the 
four point case.

% -----------------------------------------------------------------------
%                         Two Point Intersection
% -----------------------------------------------------------------------

\subsection{Two Point Intersection}
\label{section:two-point}
     If the two skeletal pairs intersect in exactly two points, it is not
difficult to see that we must have two parallel pairs.  Since the axis is the
angle bisector of the cone angle, parallel skeletal lines imply that the axes
are also parallel and the half angles of the cones are equal, 
so the two cones are congruent.
Two cases can be distinguished: one of the vertices lies 
in the interior of the 
other cone, and no vertex lies in the interior of the other cone.  Note that 
since we have only two intersection points, no vertex can lie on the surface of
the other cone (see the last two figures in Figure~\ref{fig:potential-seg}).
Using these figures, we see that there is exactly one potential segment.
In fact, the other two intersection points of the two parallel pairs are at 
infinity and thus the corresponding diagonal is a line at infinity.  We can 
prove, using the characterization theorem in 
Section~\ref{section:characterization}, that the conic intersection is an 
ellipse in the first case above, and a hyperbola in the second case.  Based on
this fact, the  intersection conic curves can be computed using the two points
on their axes and the height of some point of the potential segment.  

% -----------------------------------------------------------------------
%                           One Coincide Line
% -----------------------------------------------------------------------

\subsection{One Common Tangent and an Isolated Point}
\label{section:one-coincide-line}
     If the intersection of skeletal lines has exactly one isolated point, we
must have a line in common and this is the common tangent of the two cones
(see Figure~\ref{fig:infinite}).\footnote{Suppose there is no line in common. 
Then these four skeletal lines have at least four intersection points.}  Note
that this common tangent is counted twice and, thus using B\'{e}zout's
theorem, the residue curve must be a conic.  Therefore, if two lines, one from
each skeletal pair, coincide and the other two lines intersect at an isolated 
point, the intersection curves consist of the common tangent and a conic to be 
determined later.

     In Figure~\ref{fig:infinite}, $P$ is the isolated point and the thick line
is the common tangent.  Some segments with $P$ at one end and a point on the 
common tangent at the other end are shown in the figure.  These segments lie in
the interior of both cones and hence are potential segments.  Therefore, the 
height test cannot be applied since there are infinite number of such segments.
In what follows we will illustrate a method to isolate the appropriate 
potential segment and compute the intersection conic upon it.

     The idea is simple.  Let ${\cal K}$ be a plane perpendicular to the axial
plane ${\cal H}$, parallel to the common tangent $C$, and intersecting both 
cones.  Since ${\cal K}$ is parallel to a common tangent, it cuts
the cones in two intersecting parabolas.  Since both parabolas are symmetric 
about ${\cal H}$, the projection of these intersection points onto ${\cal H}$ 
gives a unique point $X$ with common height to both cones (see 
Figure~\ref{fig:proof-idea}).  Since ${\cal K}\cap{\cal H}$ is parallel to the
common tangent $C$ and $X$ lies on ${\cal K}\cap{\cal H}$, the line 
$\stackrel{\longleftrightarrow}{PX}$ joining $P$, the isolated point, and $X$,
cannot be parallel to $C$ and must intersect $C$ at some point $P^\prime$.
Since $P$ is a fixed point on both surfaces, $P^\prime$ lies in the 
intersection of the two cones, and $X$ has common height, if we know the type 
of the intersection conic, this conic is determined uniquely.  Therefore the 
plane through $P$ and $X$, and perpendicular to ${\cal H}$ cuts both cones in 
a common conic.  If the plane ${\cal K}$ is chosen carefully, the computation 
is straightforward.

     The position of ${\cal K}$ and the type of the intersection conic are
determined by the relative positions of the cones. Suppose two 
non-congruent cones, ${\cal C}_1(V_1,\ell_1,\alpha_1)$ and
${\cal C}_2(V_2,\ell_2,\alpha_2)$, have a common tangent $C$, and on the axial 
plane ${\cal H}$, the two skeletal pairs intersect at an isolated point $P$.
Based on the location of the triangle $\bigtriangleup V_1V_2P$, we can
distinguish three cases: $\bigtriangleup V_1V_2P$ lies in the interiors of both
cones, $\bigtriangleup V_1V_2P$ lies in the exteriors of both cones, and
only one cone contains $\bigtriangleup V_1V_2P$ in its interior
(see Figure~\ref{fig:three-case}).  For the first and the third cases, 
${\cal K}$ can be any plane satisfying the conditions mentioned at the 
beginning of the last paragraph that has an intersection point in the segment
$\overline{PV_1}$.  The intersection conic is an ellipse.  For the second 
case, ${\cal H}$ must cut the ray $\stackrel{\longrightarrow}{PV_1}$ at a point
not in the segment $\overline{PV_1}$, and the intersection conic is a 
hyperbola.

% #######################################################################
%                         LINEAR INTERSECTION
% #######################################################################

\section{Linear Intersections}
\label{section:linear}
     In this section, we will study the linear intersections. (Recall that
under our definition, a linear intersection is an intersection whose components
are all linear.) For the cone--cone case, we either have two intersecting
lines or a common tangent, while for the cylinder--cone case, it is easy to
see that no linear intersection is possible.

\begin{lemma}
\label{lemma:cone-cone-linear}
     Two distinct cones intersect in two lines if and only if they have
common vertex, a common interior point, and each cone has an interior point
that is exterior to the other surface.  Two distinct cones intersect in one
line if and only if one of the following holds.
\begin{enumerate}
     \item They have a common vertex and they are tangent along a common
          tangent line.
     \item They do not have a common vertex but one vertex lies on the other
          cone, they are congruent, and they
          have parallel axes.
\end{enumerate}
\end{lemma}
{\bf Proof:} Suppose two cones have two lines in common.  The intersection 
point of these lines must be the vertex of both cones.  The plane
determined by the two common lines contains the desired interior points.
Conversely, any two cones having a common vertex are either intersecting at
one point (the common vertex), tangent
in one or two lines, or intersecting in two lines.  The interior point 
guarantees that the cones intersect, while the point interior to one surface
but exterior to the other surface makes sure that the two surfaces do not 
include or exclude each other.  Hence they intersect in two lines.

     If two cones have a single line in common, they must be tangent to each 
other and therefore their axes are coplanar.  Let ${\cal H}$ be the axial
plane which cuts the two cones in two pairs of lines.  Since the two cones are
tangent, they share a skeletal line $T$, leaving three skeletal lines 
$T,L_1,L_2$.  If $L_1$ and $L_2$ intersect at the common vertex, we have one 
line intersection.  If they intersect, but not at the vertex, we have the 
conic--line intersection that we have studied in the 
Section~\ref{section:one-coincide-line}.  If $L_1$ and $L_2$
are parallel, then the two cones are congruent.  The converse is not difficult
to see.\ \ \ \QED

     Using these two lemmas, we have a simple algorithm to test whether two 
cones have linear intersections.
\begin{enumerate}
     \item If the cones have a common vertex, cut both cones with the axial
          plane yielding  two skeletal pairs. The two lines of intersection
          lie in a plane perpendicular to the axial plane, and this plane
          is easily found. 
          Figure~\ref{fig:two-lines-and-one-line} displays all possibilities.
     \item If the two cones are congruent with parallel axes, then
     \begin{itemize}
          \item If the vertex of ${\cal C}_1$ does not lie on ${\cal C}_2$, and
               vice versa, then ${\cal C}_1$ intersects ${\cal C}_2$ in a conic
               and see Section~\ref{section:two-point}.
          \item If the vertex of one cone lies on the other cone, then they
               are tangent along the line joining the two vertices.
     \end{itemize}
\end{enumerate}

% #########################################################################
%                    TANGENCY and DISJOINTEDNESS
% #########################################################################

\section{Tangency and Disjointedness}
\label{section:tangency-disjoint}
     In this section, we will study when two given cones are tangent to each
other in a finite number of points.  Let ${\cal C}_1(V_1,\ell_1,\alpha_1)$ and
${\cal C}_2(V_2,\ell_2,\alpha_2)$ be the two given cones.  The following 
establishes the necessary condition.

\begin{lemma}
\label{lemma:point-tangent-has-skew-axes}
     If two cones are tangent at an isolated point, their axes are skew.
\end{lemma}
{\bf Proof:}  If two cones, with vertices $V_1$ and $V_2$, are tangent at an
isolated point $P$, then the common tangent plane intersects the first and the
second cone at $\stackrel{\longleftrightarrow}{V_1P}$ and
$\stackrel{\longleftrightarrow}{V_2P}$, respectively.  Since $P$ is an isolated
point of tangency, the two half cones lie on different sides of the common
tangent plane.  Therefore their axes must be skew.\ \ \ \QED

     We will assume that $\ell_1$ and $\ell_2$ are skew and none of the two 
vertices lies in the other cone.  We shall find two planes that bound one cone
tightly and use the other cone to test the tangency against the two bounding 
planes.  From $V_2$ construct the two tangent planes to the cone 
${\cal C}_1$, $T_1$ and $T_2$ (see Figure~\ref{fig:cone-tangent}).  Note that 
since $T_i\ (i=1,2)$ is tangent to the cone ${\cal C}_1$ in a line, $T_i$ 
passes through $V_1$.  Thus $\stackrel{\longleftrightarrow}{V_1V_2}
=T_1\cap T_2$.  The two tangent planes divide the space into four quadrants.
Let the two quadrants containing ${\cal C}_1$ be ${\cal R}$.

\begin{lemma}
\label{lemma:cone-in-region}
     If the axis of the cone ${\cal C}_2$ lies in region ${\cal R}$, the cones
intersect transversally.
\end{lemma}
{\bf Proof:}  Consider the plane containing $V_1$ and the axis of ${\cal C}_2$.
This plane lies in region ${\cal R}$, since the axis of ${\cal C}_2$ does.
Since this plane passes through both vertices, it cuts the two cones
in two pairs of intersecting lines.  If none of the four lines are parallel,
the two cones will intersect transversally.  If one line from each pair are 
parallel, the two remaining lines intersect both parallel lines.  Thus the 
two cones again intersect transversally.\ \ \ \QED

     Using the above lemma, we can assume that the axis $\ell_2$ does not lie 
in the region ${\cal R}$.  Pick any point $S$ from the axis of ${\cal C}_2$, 
different from the vertex $V_2$.  From $S$ drop perpendiculars to planes $T_1$
and $T_2$ yielding $S_1\in T_1$ and $S_2\in T_2$.  Let $d_1=|\overline{SS_1}|,
d_2=|\overline{SS_2}|$ and $d=|\overline{SV_2}|$.  The following theorem 
characterizes the disjointedness and tangency of two cones using $d_1,d_2,d$ 
and the half angle $\alpha_2$.

\begin{theorem}[Tangency and Disjointedness]
\label{theorem:tangent-disjoint}
     Suppose ${\cal C}_1$ is tangent to $T_1$ and $T_2$ in two lines $t_1$ and
$t_2$ respectively.  We have the following characterizations for 
disjointedness and tangency for the cone--cone case.
\begin{enumerate}
     \item\label{case:disjoint} $d_1, d_2>d\sin\alpha_2$ : Two cones are 
          disjoint.
     \item\label{case:tangent-1} $d_1>d_2=d\sin\alpha_2$  : If $t_2\cap
          \stackrel{\longleftrightarrow}{V_2S_2}\neq\emptyset$, the two 
          cones are tangent at exactly one point, 
          $t_2\cap\stackrel{\longleftrightarrow}{V_2S_2}$.  
          Otherwise, the cones are disjoint.
     \item\label{case:tangent-2} $d_2>d_1=d\sin\alpha_2$  : If $t_1\cap
          \stackrel{\longleftrightarrow}{V_2S_1}\neq\emptyset$, the two 
          cones are tangent at exactly one point,
          $t_1\cap\stackrel{\longleftrightarrow}{V_2S_1}$.  
          Otherwise, the cones are disjoint.
     \item\label{case:two-point-tangent} $d_1=d_2=d\sin\alpha_2$ :  The two 
          cones are tangent at $t_1\cap\stackrel{\longleftrightarrow}{V_2S_1}$
          and $t_2\cap\stackrel{\longleftrightarrow}{V_2S_2}$.  One or both of
          these may not exist.  If both do not exist, the cones are disjoint.
     \item\label{case:intersect} $d_1,d_2< d\sin\alpha_2$ :
          Two cones intersect transversally.
\end{enumerate}
\end{theorem}
{\bf Proof:} The idea behind the proof is a sphere centered at $S$ with radius
$d\sin\alpha_2$.  This sphere is tangent to the cone ${\cal C}_2$.  If the
distances from $S$ to the two planes $T_1$ and $T_2$ are both greater than the
radius of the sphere, the cone lies in the interior of the regions that do not
contain the cone ${\cal C}_1$. Therefore proposition~\ref{case:disjoint} holds.

     Note that the cone ${\cal C}_2$ is tangent to $T_2$ if and only if 
$d\sin\alpha_2=d_2$.  Since $d_1>d_2=d\sin\alpha_2$, this implies that 
${\cal C}_2$ does not intersect $T_1$, except at $V_2$.  Therefore, 
${\cal C}_2$ intersects $T_2$ in one line.  If this line, 
$\stackrel{\longleftrightarrow}{V_2S_2}$, intersects $t_2$, then ${\cal C}_1$ 
and ${\cal C}_2$ is tangent there since this is the only point at which they
intersect.  If these two lines do not intersect at all, the two cones do not 
intersect since they lie in different regions in space.  Therefore 
proposition~\ref{case:tangent-1} holds.  Similar arguments can prove 
proposition~\ref{case:tangent-2} and proposition~\ref{case:two-point-tangent}.

     For proposition~\ref{case:intersect}, the plane $T_2$ intersects the cone
${\cal C}_2$ in two lines.  They cannot both be parallel to $t_2$, which is
the common tangent line of ${\cal C}_1$ and $T_2$.  That is, one of these lines
must intersect $t_2$ and therefore the two cones cannot be tangent to each
other.\ \ \ \QED

% #########################################################################
%                     CHARACTERIZATION THEOREMS
% #########################################################################

\section{Characterization Theorems}
\label{section:characterization}
     In this section, we will present some characterization results for conic
intersection.  Characterization theorems simply tell when two quadrics have a
conic intersection, but they do not offer a method for computing the result.
Essentially, they are yes/no algorithms.
Theorem~\ref{theorem:conic-conditions-using-height-test}, using the height
test, is a kind of characterization theorem, but in this case the conic
intersection can also be immediately computed.

     Recall that a conic is the intersection of a cone with a plane.  If a 
sphere is inscribed in the cone tangent to the plane, then the tangent point of
the inscribed sphere and the plane is a focus of the conic (see page 7--9 of 
Hilbert and Cohn--Vossen~\cite{hilbert:1983}).  Two spheres can be found, one 
above and the other beneath the plane, yielding the two foci of the conic. (In
the case of the parabola, only one sphere can be found.)   If two cones have a 
conic intersection, this conic lies on a plane, which is the plane defining 
the conic for both cones.  Therefore, in general, we can find two pairs of 
spheres tangent to the plane at the same point.  If one of the two cones is 
replaced by a cylinder, the same conclusion is true, but the intersection 
conic can only be an ellipse.  In summary, we have the following theorem.

\begin{theorem}[General Characterization for Conic Intersection]
     If two axial natural quadrics have distinct and coplanar axes, the
following propositions are equivalent:
\begin{itemize}
     \item These two surfaces have conic intersection.
     \item There exists a potential segment passing the height test.
     \item There exists a plane and one sphere inscribed in each surface such 
          that both spheres and the plane are tangent at the same point.
\end{itemize}
\end{theorem}

     If the spheres and the plane are projected to the axial plane, they become
circles and a line and the above theorem can be restated as follows.

\begin{corollary}
     If two axial natural quadrics have distinct and coplanar axes,  the
following propositions are equivalent:
\begin{itemize}
     \item These two surfaces have conic intersection.
     \item There exists a line and a circle inscribed in each skeletal pair
          such that both circles and the line are tangent at the same point.
\end{itemize}
\end{corollary}

     If the two axes intersect, the two inscribed spheres/circles can be 
simplified further. In fact, if two axes intersect, we can always find a 
sphere inscribed in {\em both} surfaces, or a circle inscribed in {\em both} 
skeletal pairs.  Let the distances from the vertices to the intersection of 
the two axes be $d_1$ and $d_2$, and let the half angles of the corresponding 
cones be $\alpha_1$ and $\alpha_2$.  Using the inscribed circle, the 
Goldman--Miller criteria~\cite{goldman:1990} can be obtained easily.  
This is the following theorem.

\begin{theorem}[Characterization If Axes Intersect]
\label{theorem:intersecting-axes-characterization}
     Suppose the two axes intersect.  The following propositions are
equivalent:
\begin{itemize}
     \item The two surfaces have conic intersection.
     \item There exists a sphere inscribed in both surfaces.
     \item There exists a circle inscribed in both pairs of the skeletal lines.
     \item $d_1\sin\alpha_1=d_2\sin\alpha_2$ for the cone--cone case;
           $d\sin\alpha=r$ for the cylinder--cone case, where $d$ is the 
           distance between the vertex of the cone and the intersection point
           of the two axes, and $r$ is the radius of the cylinder
           (Goldman--Miller~\cite{goldman:1990}).
\end{itemize}
\end{theorem}

% #######################################################################
%                          CONCLUSIONS
% #######################################################################

\section{Conclusions}
\label{section:conclusion}
     In order to establish the structure of the method and the interaction
of the sections, we give an outline of the algorithm below.

\vspace{1cm}
\begin{minipage}{5in}
{\small
\begin{tabbing}
     xxxxx\=xxxxx\=xxxxx\=xxxxx\=xxxxx\=xxxxx\=\kill
     \> {\bf PROCEDURE} PlanarIntersection; \\
     \> \\
     \> {\bf BEGIN} \\
     \>\> {\bf IF} axes are coplanar {\bf THEN} \\
     \>\>\> {\bf CASE} type of skeletal lines intersection {\bf OF} \\
     \>\>\>\>  4 points : goto Section 2; \\
     \>\>\>\>  3 points : goto Section 3.1; \\
     \>\>\>\>  2 points : goto Section 3.2; \\
     \>\>\>\>  1 point and a line : goto Section 3.3; \\
     \>\>\>\>  2 lines : goto Section 4; \\
     \>\>\>\>  1 line  : goto Section 4; \\
     \>\>\>\>  1 point : goto Section 4; \\
     \>\>\> {\bf END} \\
     \>\> {\bf ELSE} \\
     \>\>\>  goto Section 5 \\
     \> {\bf END} \\
\end{tabbing}
}
\end{minipage}

     Using only elementary geometry, this paper has successfully developed a 
fully geometric technique for detecting and computing the planar intersections
of two natural quadric surfaces.  Results presented here include three 
separate algorithms for detecting and computing tangency and disjointedness, 
linear intersections, and conic intersections.  

     The techniques of this paper can be generalized to cover more general 
quadric surfaces, and we are working on this problem.  The height test theorem 
(Theorem~\ref{theorem:conic-conditions-using-height-test}) holds for 
revolutionary quadrics.  However, some lemmas presented here will not hold for
more general surfaces; for example, tangency need not imply skew axes, and 
planar intersection need not imply coplanar axes.

     Another interesting question is whether one can compute the higher degree
intersection curves of natural quadrics using the techniques presented in this
paper.

% ######################################################################
%                             REFERENCES
% ######################################################################

\newpage
\begin{thebibliography}{999}

\bibitem{farouki:1989}
     R. T. Farouki, C. A. Neff and M. A. O'Connor,
     Automatic Parsing of Degenerate Quadric-Surface Intersection,
     {\em ACM Transaction on Graphics},
     Vol. 8 (1989), No. 3, pp. 174--203.

\bibitem{goldman:1990}
     Ronald N. Goldman and James R. Miller,
     Detecting and Calculating Conic Sections in the Intersection of Two
     Natural Quadric Surfaces, Part I: Detection,
     Draft, October 1990.

\bibitem{hakala:1980}
     D. G. Hakala, R. C. Hillyard, B. E. Nourse and P. J. Malraison,
     Natural Quadrics in Mechanical Design,
     {\em Proceedings of Autofact West 1}, 
     Anaheim, CA., Nov. 1980, pp. 363--378.

\bibitem{hilbert:1983}
     D. Hilbert and S. Cohn--Vossen,
     {\em Geometry and the Imagination},
     translated by P. Nememyi,
     Chelsea, New York, 1983.

\bibitem{levin:1976}
     Joshua Zev Levin,
     A Parametric Algorithm for Drawing Pictures of Solid Objects Composed
          of Quadric Surfaces,
     {\em Communications of ACM},
     Vol. 19(1976), No. 10, pp. 555--563.

\bibitem{levin:1979}
     Joshua Zev Levin,
     Mathematical Models for Determining the Intersections of Quadric Surfaces,
     {\em Computer Graphics and Image Processing},
     Vol. 11(1979), pp. 73--87.

\bibitem{miller:1987}
     James R. Miller,
     Geometric Approaches to Nonplanar Quadric Surface Intersection Curves,
     {\em ACM Transactions on Graphics},
     Vol. 6(1987), No. 4, pp. 274--307.

\bibitem{ocken:1987}
     S. Ocken, Jacob T. Schwartz and Micha Sharir,
     Precise Implementation of CAD Primitives Using Rational Parametrizations
     of Standard Surfaces,
     in {\em Planning, Geometry, and Complexity of Robot Motion}, edited by
     Jacob T. Schwartz, Micha Sharir and John Hopcroft,
     Ablex Publishing Co., Norwood, New Jersey, 1987, pp. 245--266.

\bibitem{sarraga:1983}
     Ramon F. Sarraga,
     Algebraic Methods for Intersections of Quadric Surfaces in GMSOLID,
     {\em Computer Vision, Graphics, and Image Processing},
     Vol. 22(1983), pp. 222--238.

\end{thebibliography}

% #########################################################################
%                               FIGURES
% #########################################################################

\newpage

% -------------------------------------------------------------
%                    The Main Algorithm
% -------------------------------------------------------------

\begin{figure}
\vspace{3cm}
\caption{Several Examples of Potential Segments}
\label{fig:potential-seg}
\end{figure}

\begin{figure}
\vspace{3cm}
\caption{The Conic Intersection of Two Cones}
\label{fig:cone-cone-example}
\end{figure}

\begin{figure}
\vspace{6cm}
\caption{A Height Test}
\label{fig:height-test}
\end{figure}

% --------------------------------------------------------------
%                  Degenerate Intersections
% --------------------------------------------------------------

% -------------------------------------------------------------------------
%                       Three Point Intersection
% -------------------------------------------------------------------------

\begin{figure}
\vspace{4cm}
\caption{Several Examples of Three Point Intersection}
\label{fig:three-point}
\end{figure}

% -----------------------------------------------------------------------
%                         Two Point Intersection
% -----------------------------------------------------------------------

% -----------------------------------------------------------------------
%                           One Common Line
% -----------------------------------------------------------------------

\begin{figure}
\vspace{4cm}
\caption{An Infinite Number of Potential Segments}
\label{fig:infinite}
\end{figure}

\begin{figure}
\vspace{6.5cm}
\caption{Two Cones with a Common Tangent and an Isolated Point}
\label{fig:proof-idea}
\end{figure}

\begin{figure}
\vspace{4cm}
\caption{The Positions of the Triangle $\bigtriangleup V_1V_2P$}
\label{fig:three-case}
\end{figure}

% --------------------------------------------------------------
%                    Linear Intersection
% --------------------------------------------------------------

\begin{figure}
\vspace{9.5cm}
\caption{All Possibilities of Cone Intersections with A Common Vertex}
\label{fig:two-lines-and-one-line}
\end{figure}

% --------------------------------------------------------------
%                          Tangency
% --------------------------------------------------------------

\begin{figure}
\vspace{7cm}
\caption{The Construction for Testing Tangency}
\label{fig:cone-tangent}
\end{figure}

% ##############################################################
%                       THE END
% ##############################################################

\end{document}

