% --------------------------------------------------------------------
% REVISION HISTORY:
%    1)   Sept/04/91     -- written
%    2)   May/18/93      -- revised
%    3)   June/04/93     -- 2nd revision
% --------------------------------------------------------------------

\documentstyle [12pt]{article}

\input{table}

\def\thefootnote{\fnsymbol{footnote}}

\title{
     On the Lower Degree Intersections of \\
     Two Natural Quadrics\footnotemark[1]\ \footnotemark[2]}

\author{Ching-Kuang Shene\footnotemark[3]\ \footnotemark[4]\ \ \ \
          John K. Johnstone\footnotemark[3]}

\date{\ }

\newtheorem{example}{Example}[section]
\newtheorem{property}{Property}[section]
\newtheorem{definition}{Definition}[section]
\newtheorem{theorem}{Theorem}[section]
\newtheorem{lemma}{Lemma}[section]
\newtheorem{corollary}{Corollary}[section]
\newtheorem{remark}{Remark}[section]

\newcommand{\DoubleSpace}{\edef\baselinestretch{1.4}\Large\normalsize}
\newcommand{\QED}{\ \ \ \rule{2mm}{3mm}\\}
\newcommand{\arrow}[1]{\vec{\bf #1}}

%\DoubleSpace
\setlength{\oddsidemargin}{0pt}
\setlength{\evensidemargin}{0pt}
\setlength{\headsep}{0pt}
\setlength{\topmargin}{0pt}
\setlength{\textheight}{8.75in}
\setlength{\textwidth}{6.5in}


\begin{document}
\maketitle
\footnotetext[1]{This work was supported by
          National Science Foundation grant IRI--8910366.}
\footnotetext[2]{Preliminary results from this work were presented at the
          ACM Symposium on Solid Modeling Foundations and CAD/CAM Applications,
          June 1991 and at 1993 ASME Design Automation Conference,
          September 1993.}
\footnotetext[3]{Authors' address: Department of Computer Science,
          Johns Hopkins University, Baltimore, Maryland 21218--2686, USA.}
\footnotetext[4]{Current address: Department of Mathematics and Computer
          Science, Northern Michigan University,
          Marquette, Michigan 49855--5340, USA}

\def\thefootnote{\arabic{footnote}}
\setcounter{footnote}{0}


% --------------------------------------------------------------------
%                               ABSTRACT
% --------------------------------------------------------------------

\begin{center}
\begin{minipage}{14cm}
{\small
\_$\!$\hrulefill
\noindent

     In general, two quadric surfaces intersect in a space quartic curve.
However, the intersection  frequently degenerates to  a collection of
plane curves.  In this paper, we investigate this problem
for natural quadrics.  Algorithms are presented to detect and compute
conic intersections and linear intersections.
These methods reveal the relationship between the planes of the degenerate
intersections and the quadrics.  Using the theory developed in the paper,
we present a new and simplified proof of a necessary and sufficient
condition for conic intersection.
Finally, we present a simple method for
determining the types of conic in a degenerate intersection
without actually computing the intersection,
and an enumeration of all possible conic types.
Since only elementary geometric routines such as line intersection
are used, all of the above algorithms are intuitive and easily implementable.
\\
\par
Categories and Subject Descriptors: I.3.5 [{\bf Computer Graphics}]:
Computational Geometry and Object Modeling---{\em curve, surface, and object
representations, geometric algorithms}; J.6. [{\bf Computer Applications}]:
Computer-Aided Engineering---{\em computer-aided design}
\\
\par
General Terms: Algorithms, Design
\\
\par
Additional Key Words and Phrases: Natural quadrics, surface intersection,
planar intersection

\_$\!$\hrulefill

}
\end{minipage}
\end{center}


% ----------------------------------------------------------------------
%                       INTRODUCTION
%
% REVISION HISTORY:
%    1)   Sept/04/91     Written
%    2)   Apr/04/93      Revised
% ----------------------------------------------------------------------

\section{Introduction}
\label{section:introduction}

     Natural quadrics consist of the spheres, the circular cylinders and
the right cones.  Many papers have been dedicated to this class because, in
most geometric/solid modeling systems, they are still the most useful and
natural objects to model mechanical parts.  In this paper, we will investigate
the degenerate intersection of natural quadrics.
In general, two quadric surfaces intersect in a space quartic curve.
However, the intersection may degenerate to a collection of plane curves.
Using a simple geometric argument, it can be shown that there are
four types of degenerate intersection for natural quadrics, as outlined
in Table~\ref{tbl:intersection-type}.
These degenerate cases are very frequent and important in geometric/solid
modeling,
because degeneracies are often required by design.
We shall develop a method of detecting and computing degenerate intersections
of natural quadrics.
This detection is important, because
degenerate intersections can be computed more easily, and also allow
simpler treatment of important problems.
For example,
if two natural quadrics have planar intersection, the two surfaces can be
blended using two distinct families of cyclides (Boehm~\cite{boehm:1990},
Pratt~\cite{pratt:1990}, Sabin~\cite{sabin},
Johnstone and Shene~\cite{johnstone-shene:1992} and
Shene~\cite{shene:1992,shene:1993a});
or if two quadrics have planar
intersection, they can be blended using a quadric or cubic surface, rather
than a quartic (Warren~\cite{warren:1987,warren:1989}).

\begin{table}
\caption{Possible Intersection Types of Two Natural Quadrics}
\label{tbl:intersection-type}
$$
\BeginTable
     \BeginFormat
     |5 c              | c                    | c             |5
     \EndFormat
     \_5
     | \it Number of | \it Number of | \it Number of | \\
     | \it Conics    | \it Lines     | \it Points    | \\ \_4
     |      2        |       0       |       0       | \\ \_
     |      1        |       1       |       0       | \\ \_
     |      0        | 1, 2, 3, 4    |       0       | \\ \_
     |      0        |       0       |      1,2      | \\
     \_5
\EndTable
$$
\end{table}

     This paper is divided into ten  sections.
Section~\ref{section:previous-work} reviews related work.
Section~\ref{section:notation} introduces the notation used in the paper.
In Section~\ref{section:conic}, our algorithm for detecting and
calculating conic intersections is presented in its most general form without
concern for degenerate cases.
By observing that natural quadrics with planar intersection must have coplanar
axes, the computation of the intersection is reduced to this plane of the axes.
Then planar intersections are determined by testing the height from a
point on this plane to both surfaces.
These computations only involve line intersections
and simple arithmetic operations.
Section~\ref{section:degen}
discusses how to handle the degenerate cases so that the presentation of our
algorithm is complete.
Section~\ref{section:linear} presents a simple algorithm for detecting
and computing linear intersections.
In Section~\ref{section:simplification}, we present a necessary and sufficient
condition for conic intersection, the common inscribed sphere criterion.
An interesting application of this criterion is discussed in
Section~\ref{section:enumeration}, where we discuss a method to determine
the types of the intersection conics without actually computing them.
Finally, Section~\ref{section:conclusion} gives our conclusion.


% ----------------------------------------------------------------------
%                       PREVIOUS WORK
%
% REVISION HISTORY:
%    1)   Sept/04/91     Written
%    2)   May/14/93      Revised
% ----------------------------------------------------------------------

\section{Previous Work}
\label{section:previous-work}

     The intersection of two surfaces is a fundamental and challenging problem
in geometric/solid modeling.  Many powerful algorithms have been developed for
computing the intersection curve of two algebraic surfaces: for example,
Abhyankar and Bajaj~\cite{abhyankar-bajaj:1989},
Garrity and Warren~\cite{garrity-warren:1989},
Hoffmann~\cite{hoffmann:1989},
Hohmeyer~\cite{hohmeyer:1991} and
Manocha and Canny~\cite{manocha-canny:1991}.
For two quadric surfaces,
the computation is simpler and has theoretically been solved in the last
century (see, for example, Bromwich~\cite{bromwich:1971}).
In the context of geometric/solid modeling, Sabin~\cite{sabin:1976}
is one of the earliest rigorous solutions of this problem.
During the past decade, starting with Levin~\cite{levin:1976,levin:1979},
algebraic methods based on the pencil of two quadrics have been
successfully used to attack quadric intersection.  Other examples are
Farouki, Neff and O'Connor~\cite{farouki:1989}, Miller~\cite{miller:1987},
Ocken, Schwartz and Sharir~\cite{ocken:1987},
O'Connor~\cite{oconnor:1989},
% it does not distinguish the degenerate cases and thus the two
% intersection conics may be parameterized as a space general quartic.
and Sarraga~\cite{sarraga:1983}.

Recent attention has been drawn to the conditions that cause
degenerate intersection curves,
since a general algorithm may be insufficient for degenerate cases.
Farouki, Neff and O'Connor~\cite{farouki:1989}
is a complete study for general quadric surfaces,
using the pencil technique.

Samuel, Requicha and Elkind~\cite{samuel:1976} and
Hakala, Hillyard, Nourse and Malraison~\cite{hakala:1980} are two of the first
studies of natural quadrics by the geometric/solid modeling community.
In his papers~\cite{piegl:1989,piegl:1992}, Piegl distinguishes eight
cases for cylinder--cone intersection, in which two of them give conic
intersection.  However, from this enumeration, it is not  clear how to devise
an extension to the cone--cone case, which is more general and more
subtle than the cylinder--cone case.
% since conic intersection can be obtained
% even if the vertex of the cone lies inside the other surface.
Miller~\cite{miller:1987} also provides an incomplete enumeration.

Concurrently with the development of our work on degenerate intersection
of natural quadrics \cite{shene-johnstone:1991a},
Goldman and Miller~\cite{g-m:1990,g-m:1991a,m-g:1991,m-g:1991a}
developed conditions for degenerate intersection of two natural quadrics.
They use the pencil method, in which the singular
members of the one-parameter family ${\cal Q}_1+\lambda{\cal Q}_2=0$ are
computed, where ${\cal Q}_1$ and ${\cal Q}_2$ are two natural quadrics.
If ${\cal Q}_1\cap{\cal Q}_2$ has planar intersection,
we can find some $\bar{\lambda}$ such that
${\cal Q}_1+\bar{\lambda}{\cal Q}_2$ is of rank two or less and thus can be
factored into the product of two planes.
If the two quadrics have planar
intersections, then the two planar
members in the pencil are computed
(Miller and Goldman~\cite{m-g:1991}).
Finally, these planar members are intersected with the natural quadric to
compute the intersection curve
(Miller and Goldman~\cite{m-g:1991a} and
Johnstone and Shene~\cite{johnstone-shene:1991}).

In our method, we depart from the pencil method and use geometric rather
than algebraic arguments.
These help to reveal the structure of the problem, simplify the development,
and allow detection and computation of the degenerate intersections to
be done at the same time.
One of the most interesting aspects of our method is
that the topology of the degenerate intersection curve
(Table~\ref{tbl:intersection-type}) is classified prior to any calculations
that are used to compute the geometry.  Thus, our method is more robust than
previous methods.
Robustness also motivates our use
of geometric representations for the quadrics and intersection conics
({\em e.g.}, the representation of a conic by its major and minor axis
and their lengths, or a cone by its axis, vertex, and cone angle) in the
rest of this paper:
Goldman \cite{goldman:1983b} and Wilson \cite{wilson:1987} address this
issue well, showing how geometric representations, if they are available
(as they are for quadrics), are more robust than
algebraic (implicit or parametric) representations (also see the following
remark).
% (For example, after a series of rigid motions,
% the radius of a cylinder as calculated indirectly through its implicit
% equation may have accumulated considerable error.)

%We also consider the question of disjointedness: that is, whether
%the surfaces intersect at all.

% Since
% our approach is geometric (only geometric arguments are used), it does not
% depend on the representation ({\em e.g.}, parametric or implicit).
% This provides
% an advantage over the algebraic pencil method since it can be adapted to any
% representation as long as that representation is capable of calculating the
% intersection of lines and planes.
% Also, the simplicity of our method makes
% it generalizable, unlike other methods that are already complicated and
% therefore have to push further.

\begin{remark} \rm
To better appreciate our approach, it is illuminating to investigate
the shortcomings of a conceptually simple algorithm that appears to
solve the problem:

\begin{enumerate}
\item
     \ [Reduce to a simple case] Map the cones to cylinders.
     Then map the cylinders (simultaneously) to circular cylinders.
\item
     If the cylinders' axes intersect at a finite point and the
     cylinders' radii are equal, the intersection of the cylinders consists
     of two conics.
     If the cylinders' axes intersect at infinity,
     the intersection consists of four lines.
     In all other cases, the intersection is not degenerate.
\item
     Transform the lines or conics
     computed in the last step (if they exist)
     back to the original coordinate system.
\end{enumerate}

\noindent The main problem with this method lies in the use of transformed
quadrics, rather than working directly with the original quadrics.
In step (2), the axes of the transformed cylinders have to be extracted
and tested for intersection, and the radii must be extracted and tested
for equality.
However, the transformations of step (1)
distort information such as axes and radii,
as discussed in Goldman \cite{goldman:1983b} and Wilson \cite{wilson:1987}.
Due to the effect of transformations, we are not sure whether a pair
of non-intersecting axes are indeed intersecting in the original
coordinate system, or whether two radii are truly equal.
This is a particularly important problem in the context of the detection
of degenerate intersection, because degenerate intersection is inherently
a fragile condition: a small perturbation of the quadrics will change a
degenerate intersection into an irreducible intersection.
We wish to work with the purest data, which is the original data.

Other problems exist as well, such as the complexity of step (1),
the need for complex arithmetic in steps (1) and (3),\footnote{Complex
     arithmetic is needed to transform a hyperbolic cylinder to a
     circular cylinder.}
and the need to use an algebraic representation
rather than a geometric representation.

Our method works directly with the original quadrics without using
any transformations, it uses only real arithmetic, and it uses geometric
representations of the quadrics.
The predominant calculation is line intersection, which can be implemented
robustly.
$\Box$
\end{remark}


% ----------------------------------------------------------------------
%                           NOTATION
% ----------------------------------------------------------------------

\section{Notation}
\label{section:notation}

     We need to develop some terminology and notation.

\begin{definition}
     A {\bf natural} quadric is a sphere, a circular cylinder, or a right cone.
An {\bf axial} natural quadric is a
natural quadric with an axis, namely a cylinder or a cone.
\end{definition}

\begin{definition}
     A {\bf conic} intersection is a composite intersection curve such that
the highest degree among all component curves is two (Cases 1 and 2 in
Table~\ref{tbl:intersection-type}).  A {\bf linear} intersection is a
composite intersection curve such that the highest degree among all component
curves is one (Case 3 of Table~\ref{tbl:intersection-type}).  A {\bf point}
intersection is a composite intersection such that the components are all
isolated points (Case 4 of Table~\ref{tbl:intersection-type}).  {\bf Tangency}
means a point intersection.  Hence a common tangent line is a linear
intersection.
\end{definition}

\begin{remark} \rm
Detection and computation of one-point and two-point tangency, as well
as the detection of disjointness of two natural quadrics, are discussed
in Shene and Johnstone~\cite{shene-johnstone:1991a,shene-johnstone:1991c}.
$\Box$
\end{remark}

\begin{definition}
     Two surfaces are {\bf congruent} if and only if there exists a rigid
motion transforming one into the other.
\end{definition}

\noindent{\bf Notation:} ${\cal C}(V,\ell,\alpha)$
is the cone with vertex $V$, axis $\ell$ and half angle $\alpha$.
${\cal Z}(\ell,r)$ is the cylinder with axis $\ell$ and radius $r$.
${\cal S}(O,r)$ is the sphere with center $O$ and radius $r$.
$\stackrel{\longleftrightarrow}{AB},
          \stackrel{\longrightarrow}{AB},\overline{AB}$ and
          $|\overline{AB}|$ are the line, the ray, the segment, and the
          length of the segment determined by two points $A$ and $B$.

\vspace{1em}

In the rest of this paper, we will only consider cylinders and cones,
the axial natural quadrics.
Detection and computation of planar intersection with a sphere is very simple,
since all planar curves on a sphere are circles.
If two spheres intersect, their intersection is always a circle,
whose computation is trivial.
For the intersection of a sphere and an axial natural quadric,
since the only circles on a cylinder or cone lie in the planes
perpendicular to the axis,
a necessary condition for planar intersection
is that the center of the sphere
lies on the axis of the axial natural quadric,
which again makes the detection and computation simple.

% --------------------------------------------------------------------
%                           THE MAIN IDEA
% --------------------------------------------------------------------

\section{Conic Intersections}
\label{section:conic}

     In this section, we present our algorithm for detecting and computing
conic intersection in a
generic form, ignoring degenerate cases.  The idea is to  reduce the surface
intersection problem to a simple planar line intersection problem.
Section~\ref{section:fundamental} presents some basic results which play a
fundamental role in our study.  Section~\ref{section:axial-plane} introduces
the concept of the axial plane and proves our first characterization result
for conic intersection.  Section~\ref{section:height} discusses the concept
of height at some point on the axial plane and its computation.


% ********************************************************************
%                       FUNDAMENTAL RESULTS
% ********************************************************************

\subsection{Some Fundamental Results}
\label{section:fundamental}

     In this section, we establish that a necessary condition for conic
intersection is coplanar axes.  The following classical result is useful.

\begin{theorem}[Dandelin Sphere]
\label{thm:dandelin-sphere}
     Let the plane $P$ intersect the axial natural quadric ${\cal Q}$ in a
conic.  There are one or two spheres inscribed in ${\cal Q}$ and tangent to
$P$.  The tangent points of these spheres on $P$ are the foci of the
intersection conic (Figure~\ref{fig:dandelin-sphere-and-foci}).
\end{theorem}

\begin{figure}
\vspace{8.5cm}
\caption{The Dandelin Spheres}
\label{fig:dandelin-sphere-and-foci}
\end{figure}

     These spheres are called {\em focal spheres} or {\em Dandelin spheres},
after G. Dandelin~\cite{dandelin:1822} (see, for example, Hilbert and
Cohn-Vossen~\cite{hilbert:1952} for more on the Dandelin sphere).
The Dandelin sphere has a direct application
in the computation of the intersection conic of a plane and an axial natural
quadric (Johnstone and Shene~\cite{johnstone-shene:1991} and
Miller and Goldman~\cite{m-g:1991a}).

     Using the Dandelin sphere, we can prove the following important
technical lemma.

\begin{lemma}
\label{lemma:perpendicular-stuff}
     Let $P$ be a plane cutting an axial natural quadric in a conic $C$.
Then the plane $E$ determined by the major axis of $C$ and the axis of
the axial natural quadric is perpendicular to $P$.
\end{lemma}
{\bf Proof:} Note that $P$ cannot contain the axis $\ell$ of the axial
natural quadric, since otherwise the intersection is two lines, not a
conic. If $P$ is perpendicular to $\ell$, the intersection is a circle,
and the lemma is obviously true.   If $P$ is not perpendicular to $\ell$,
consider the Dandelin sphere $D$ yielding the focus $F$
(Figure~\ref{fig:perpendicular}).  The plane $E$ containing $\ell$ and
perpendicular to $P$ cuts the surface, the plane $P$ and the Dandelin
sphere\footnote{Since it is an inscribed sphere,
     the center of a Dandelin sphere lies on the axis of the quadric,
     and thus on the plane $E$.  Therefore $E$ cuts the sphere into two
     symmetric parts.}
$D$ into two symmetric parts.
Therefore $\bar{\ell}=E\cap P$ must be an axis
of the intersection conic.  Note that the tangent point $F$ lies on
$\bar{\ell}=E\cap P$ because of symmetry.
Hence $\ell$ must be the major axis.  \QED

\begin{figure}
\vspace{7.5cm}
\caption{Relation of Plane $P$ and Plane $E$}
\label{fig:perpendicular}
\end{figure}

Notice that this lemma implies that the axis of the quadric
intersects the major axis of the plane section $C$.
We can immediately use this lemma to develop a necessary condition for
conic intersection.

\begin{lemma}
\label{lemma:non-skew}
     If two axial natural quadrics have a conic intersection, their
axes are coplanar.
\end{lemma}
{\bf Proof:}  Let ${\cal Q}_1$ and ${\cal Q}_2$ be two axial natural quadrics.
If ${\cal Q}_1\cap{\cal Q}_2$ contains a circle, then the axes must coincide.
Otherwise, let $P$ be the plane containing one of the intersection conics.
By Lemma~\ref{lemma:perpendicular-stuff}, the plane ${\cal H}$ determined by
the major axis of the intersection conic and the axis of ${\cal Q}_1$ is
perpendicular to $P$.  The same is true for ${\cal Q}_2$.  Hence both axes of
${\cal Q}_1$ and ${\cal Q}_2$ lie in plane ${\cal H}$.  \QED

In the sequel, we shall deal exclusively with quadrics with coplanar axes.

\begin{definition}
     If the axes of two axial natural quadrics are distinct and coplanar,
they determine a plane called the {\bf axial plane}.\footnote{If the axes are
identical, then the two quadrics are either identical, disjoint or have an
intersection of two circles that is easy to compute.}  This plane cuts each
surface in a pair of straight lines, called the  {\bf skeletal pair}.
The lines in a skeletal pair are called {\bf skeletal lines}
(Figure~\ref{fig:diagonals}).
\end{definition}




% ********************************************************************
%                           AXIAL PLANE
% ********************************************************************

\subsection{The Axial Plane}
\label{section:axial-plane}

     In this section, we will discuss the axial plane and identify the planes
that must contain the conics of a conic intersection.

\begin{definition}
     Consider the non-degenerate case in which neither vertex of the axial
natural quadrics lies on a skeletal line.\footnote{The degenerate case will
     be considered in Section~\ref{section:degen}.}
Then the four skeletal lines
form a  complete quadrilateral with six intersection points
(e.g., $A$, $B$, $C$, $D$, and the two cone vertices in
Figure~\ref{fig:diagonals}(a)).
(Some of these intersections may be at infinity if some of the skeletal
lines are parallel.  We assume that we are working in the Euclidean plane
with a line at infinity
so that any two lines, even if parallel, intersect in a point.)
These six points generate two lines that are not skeletal lines
and that do not contain either vertex,\footnote{Since the axial plane
     cuts a cylinder in two parallel lines
     and parallel lines meet at the point of infinity along the lines'
     direction,
     a cylinder has its vertex at the point of infinity along the
     direction of any line on its surface.}
which we call {\bf diagonals}.
\end{definition}

\begin{example}
Figure~\ref{fig:diagonals} shows some examples of
skeletal pairs and their diagonals $\stackrel{\longleftrightarrow}{AC}$
and $\stackrel{\longleftrightarrow}{BD}$.
Parts of the diagonals that lie in the
interior of both surfaces are indicated by thick line segments.
In Figures~\ref{fig:diagonals}(c) and \ref{fig:diagonals}(d),
the lines in one skeletal pair are
parallel to their counterparts from the other pair.  Hence two intersection
points are at infinity and the diagonal $\stackrel{\longleftrightarrow}{BD}$
containing them is the line at
infinity.  In Figure~\ref{fig:diagonals}(e), notice that one diagonal
$\stackrel{\longleftrightarrow}{BD}$ includes
an intersection point $D$ at infinity.
\end{example}

\begin{figure}
\vspace{9cm}
\caption{Some Examples of Diagonals}
\label{fig:diagonals}
\end{figure}

We can now characterize the position of the intersection conics precisely.

\begin{lemma}
\label{lemma:coincide-conic}
     If two axial natural quadrics with distinct axes have conic
intersection, then the conics lie in the planes through a diagonal and
perpendicular to the axial plane.
\end{lemma}
{\bf Proof:}
Suppose two axial natural quadrics with distinct axes have a conic
intersection.  Let one of the intersection conics be $C$.  Since the axial
plane ${\cal H}$ cuts both surfaces into two symmetric parts, $C$ is
symmetric about ${\cal H}$.  In particular, if $P$ is the plane containing $C$,
$P\cap{\cal H}$ is the major axis of $C$ (Figure~\ref{fig:major}).
By Lemma~\ref{lemma:perpendicular-stuff}, $P$ is
perpendicular to the axial plane ${\cal H}$.  Since $C$ belongs to both
surfaces, the points of $C\cap {\cal H}$ are intersection points of the two
skeletal pairs, and $P\cap{\cal H}$ must be a diagonal. \QED

\begin{figure}
\vspace{4.5cm}
\caption{The Axial Plane and the Intersection Conic}
\label{fig:major}
\end{figure}



% ********************************************************************
%                             HEIGHT
% ********************************************************************

\subsection{The Height and Its Computation}
\label{section:height}

     Lemma~\ref{lemma:coincide-conic} tells us that the planes through a
diagonal and perpendicular to the axial plane are important.
In particular, if $P$ is such a plane,
through the diagonal $d$, and it intersects
the two given surfaces in $C_1$ and $C_2$ (Figure~\ref{fig:heights}),
then the two quadrics have conic (degenerate) intersection if and only if
$C_1 = C_2$.
We want to develop a simple, indirect method to test if $C_1=C_2$.

$C_1$ and $C_2$ must have
the same major axis, $d$, by symmetry with the axial plane.
Let $R$ and $S$ be the intersections of the diagonal $d$ and
any skeletal pair, and suppose that both $R$ and $S$ are finite
({\em i.e.}, $C_1$ and $C_2$ are central conics).
The case of infinite $R$ or $S$ will be addressed at the end of this section.
$C_1$ and $C_2$ have major axis $\stackrel{\longleftrightarrow}{RS}$,
and the direction of
their minor axes are the same: through the midpoint of $\overline{RS}$ and
perpendicular to ${\cal H}$.  Thus the question reduces
to testing if the length of the minor axes of $C_1$ and $C_2$ are the
same.  We shall use a height computation to check this.

\begin{definition}
\label{defn:height}
Pick any point $U$ on the diagonal $d$,
construct a line  through $U$ and perpendicular to ${\cal H}$, and compute
the distance from $U$ to the quadric surface along this line.
This distance is called the {\bf height} at $U$.
Because the surface is symmetric about the
axial plane ${\cal H}$, the heights from $U$ to either side of the surface
are equal.  Hence the concept of height is well-defined.
\end{definition}

Since there are
two surfaces, we have two heights at $U$, $h_1$ and $h_2$.
$C_1=C_2$ if and only if $h_1=h_2$.  To see this more formally,
let $a=\frac{1}{2}|\overline{RS}|$ be the
semi-major axis length.  If the midpoint of $\overline{RS}$ and
$\stackrel{\longleftrightarrow}{RS}$
are taken to be
the coordinate origin and $x$-axis,
$C_i$'s equation is $\frac{1}{a^2}{x^2}+B_iy^2=1$,
where $B_i$ is unknown.  Then, by plugging $x=u$, the $x$-coordinate of $U$,
and $y=h_i$ into the equation, we have $B_i=\frac{a^2-u^2}{a^2h_i^2}$.
Hence, if $h_1=h_2$, $B_1=B_2$ and thus $C_1=C_2$.

\begin{remark}\rm
\label{rmk:minor}
Notice that $\sqrt{1/|B_i|} = \frac{|ah_i|}{\sqrt{|a^2 - u^2|}}$
is the semi-minor axis length.
If $B_i>0$, the conic is an ellipse, otherwise it is a hyperbola.
$\Box$
\end{remark}

\begin{figure}
\vspace{4cm}
\caption{Equal Height Implies Equal Conics}
\label{fig:heights}
\end{figure}

     The height computation is particularly simple for natural quadrics.  From
$U$, consider the plane
perpendicular to the surface axis $\ell$ meeting it at $K$.  This plane
intersects the surface in a circle.
Let $d_1=\mbox{dist}(U,\ell)=|\overline{UK}|$,
and let $r$ be the radius of the circle.
If the surface is a cone with vertex $V$,
then $r = d_2\tan\alpha$, where $\alpha$ is the cone angle and
$d_2=|\overline{VK}|$ (Figure~\ref{fig:height-comp}(a));
if the surface is a cylinder, $r$ is the radius of the cylinder.
Then the squared height at $U$ is $r^2-d_1^2$
(Figure~\ref{fig:height-comp}(b-c)).
Note that if $U$ lies outside of the
circle, the squared height will be negative.  This is equivalent to saying that
we have an imaginary $y$-coordinate at $U$.


\begin{figure}
\vspace{5cm}
\caption{Height Computation}
\label{fig:height-comp}
\end{figure}

We have shown how to detect and calculate the first intersection conic,
using two height computations.
It would appear that another height computation is needed to calculate the
second intersection conic, using the other diagonal.
However, by calculating the height from the intersection point $X$ of the
two diagonals (Figure~\ref{fig:point-and-line}),
we deal with both conics at the same time and
it is possible to use only two squared height computations in total.

     The above discussion assumes that $R$ and $S$ are finite.  If one of $R$
and $S$, say $S$, is at infinity, the intersection conic must be a parabola
and the computation is very similar, still involving two  height computations.
$R$ is the vertex of the parabola, while the line $d_\perp$ through $R$ and
perpendicular to ${\cal H}$ gives the direction of the directrix.
If $d,d_\perp$ and $R$ are chosen to be the $x$-axis, the $y$-axis and the
origin, the parabola has equation $y^2=4fx$, where $f$ is the unknown focal
length.  Using a height computation at any point $U$ on the diagonal $d$,
we have $f=h_u^2/4u$, where $u$ is the $x$-coordinate of $U$
and $h_u^2$ is the squared height at $U$.
By using the height at $X$, the intersection of the diagonals, only
two height computations are needed to find both intersection
conics.

     We are now ready to present the algorithm for detection and computation
(if applicable) of conic intersection for
${\cal Q}_1\cap{\cal Q}_2$, where ${\cal Q}_i$ is an axial natural quadric with
vertex $V_i$ and axis $\ell_i$.

\begin{center}
\begin{minipage}{5in}
\vspace{8mm}
\begin{center}
     {\bf Algorithm: } Detection and Computation of Conic-Intersection
               (Non-degenerate Case)
\end{center}
{\small
\begin{enumerate}
     \item Test for coplanarity of the axes.
          If not coplanar, then ${\cal Q}_1$ and ${\cal Q}_2$
          do not have conic intersection.
     \item \ [Special case]
          If $\ell_1=\ell_2$ and $V_1\neq V_2$, the intersection consists
          of two circles;  if $\ell_1=\ell_2$ and $V_1=V_2$, the intersection
          is the common vertex or the entire surface.
          Thus, assume $\ell_1 \neq \ell_2$ in the sequel.
     \item \ [Test for degeneracy]
          Let $\ell_{11}$ and $\ell_{12}$ be ${\cal Q}_1$'s skeletal lines
          (easily computed as the lines in the axial plane through $V_1$
          at an angle $\pm \alpha_1$ from the axis $\ell_1$),
          and $\ell_{21}$, $\ell_{22}$ be ${\cal Q}_2$'s skeletal lines.
          If $V_1\in\ell_{21}\cup\ell_{22}$ or $V_2\in\ell_{11}\cup\ell_{12}$,
          we have a degenerate case (see Section~\ref{section:degen}).
     \item \ [Find the intersections of the skeletal lines, which define
          the diagonals]
          Let $P_1=\ell_{11}\cap\ell_{21}, P_2=\ell_{11}\cap\ell_{22},
          P_3=\ell_{22}\cap\ell_{12}$, $P_4=\ell_{21}\cap\ell_{12}$
          (Figure~\ref{fig:point-and-line}).\footnote{Line intersection
          is a simple problem.
          See Goldman~\cite{goldman:1990} for a good treatment.}
          If the diagonals intersect,
       let $X=\stackrel{\longleftrightarrow}{P_1P_3}\cap
          \stackrel{\longleftrightarrow}{P_2P_4}$, the intersection of the
          diagonals.
          If one of the diagonals, $\stackrel{\longleftrightarrow}{P_1P_3}$
          or $\stackrel{\longleftrightarrow}{P_2P_4}$, is the line at
          infinity, then let $X$ be any point on the finite diagonal
       (but $X$ not on any skeletal line).
          If the diagonals are parallel, then ${\cal Q}_1$ and ${\cal Q}_2$
          do not have conic intersection.\footnote{We can show that if
          the quadrics have conic intersection,
          the diagonals will not be parallel.
       In particular, if the axes intersect,
          the diagonals are not parallel by Shene \cite[Lemma~4.5]{shene:1992}.
          If the axes are parallel but distinct,
          the cone angles are equal
          (Theorem~\ref{theorem:parallel-axes} below) which
          implies one of the diagonals is the line at infinity.
          Finally, step 2 deals with identical axes.}
     \item \ [Compute height]
          Compute the squared heights $h_{1}^2$ and $h_{2}^2$ to both
          surfaces at $X$ (using the formula after
          Definition~\ref{defn:height}, $r^2 - d_1^2$).
     \item If $h_{1}^2=h_{2}^2$, we have conic intersection.
          The conics lie in the planes through the diagonals and perpendicular
          to the axial plane:
\begin{enumerate}
     \item If $P_1$ and $P_3$ are both finite, the intersection on
          diagonal $\stackrel{\longleftrightarrow}{P_1P_3}$ is a central conic
       with major axis $\stackrel{\longleftrightarrow}{P_1P_3}$,
       center $\frac{1}{2}(P_1 + P_3)$, and the length of its semi-major
       axis is $\frac{1}{2}|\overline{P_1 P_3}|$.
       The minor axis is the line through the center and perpendicular to
          the axial plane, with length and type
       as computed in Remark~\ref{rmk:minor}.

     \item If $P_3$ is at infinity, the intersection on  diagonal
          $\stackrel{\longleftrightarrow}{P_1P_3}$ is a parabola
          with vertex $P_1$, directrix direction perpendicular to the axial
          plane, and focal length as computed above ($f=h_u^2/4u$).
\end{enumerate}
          The intersection conic on $\stackrel{\longleftrightarrow}{P_2P_4}$
          can be computed similarly.
\end{enumerate}
}
\end{minipage}
\end{center}

\begin{figure}
\vspace{5.5cm}
\caption{Points and Lines Used in Determining the Intersection}
\label{fig:point-and-line}
\end{figure}


% --------------------------------------------------------------------
%                         DEGENERATE CASES
% --------------------------------------------------------------------

\section{Degenerate Cases}
\label{section:degen}

     In the previous section, we ignored the degenerate case when the vertex
of one surface lies on a skeletal line.
Let $S_i$ be the skeletal pair of the quadric ${\cal Q}_i$.
There are four cases:
(1) $V_1\in S_2$ and $V_2\not\in S_1$, (2) $V_1\not\in S_2$
and $V_2\in S_1$, (3) $V_1\in S_2$ and $V_2\in S_1$, and (4) $V_1=V_2$.
We will show that
the first two cases cannot have degenerate intersection, while the third
always delivers a conic intersection and the fourth always delivers
linear intersection.
Our treatment of the fourth case is postponed to
Section~\ref{section:linear}.


% ********************************************************************
%                 EXACTLY ONE VERTEX ON SURFACE
% ********************************************************************

\subsection{Exactly One Vertex Lies on the Other Surface}
\label{section:exactly-one}

\begin{lemma}
\label{lemma:one-vtx-on-surface}
     Let ${\cal Q}_1$ and ${\cal Q}_2$ be two axial natural quadrics with
coplanar  and distinct axes.  If exactly one vertex lies on the other surface,
then ${\cal Q}_1$ and ${\cal Q}_2$ do not have conic or linear intersection.
\end{lemma}
{\bf Proof:}
Suppose $V_1$ lies on the skeletal line $\ell_{21}$ from $S_2$
(Figure~\ref{fig:vtx-on-pair}).
The lines of $S_1$ intersect the other skeletal line $\ell_{22}$ of $S_2$
in two points, say $R$ and $S$.
Suppose that ${\cal Q}_1$ and ${\cal Q}_2$ have conic intersection.
The plane $P$ containing one of the intersection conics must have two points
of intersection with skeletal lines, and the candidate points are
$V_1$, $V_2$, $R$, and $S$ since they must lie on both skeletal pairs
(being on the intersection conic).
$P$ cannot contain $V_1$ or $V_2$, otherwise $P \cap {\cal Q}_1$
or $P \cap {\cal Q}_2$ is linear (rather than conic).
The only candidate points left are $R$ and $S$, but if $P$ contains both
$R$ and $S$ then it will also contain $V_2$, a contradiction.
Hence ${\cal Q}_1\cap {\cal Q}_2$ is not a conic intersection.

\begin{figure}
\vspace{3.5cm}
\caption{$V_1\in S_2$ and $V_2\not\in S_1$}
\label{fig:vtx-on-pair}
\end{figure}

${\cal Q}_1\cap{\cal Q}_2$ is not a linear intersection, either.
If it contains a line, this line must contain both $V_1$ and $V_2$ and thus
the intersection line is $\ell_{21}=\stackrel{\longleftrightarrow}{V_1V_2}$.
Then the axial plane cuts ${\cal Q}_1$ in three lines, two in $S_1$ and
$\ell_{21}=\stackrel{\longleftrightarrow}{V_1V_2}$, which is impossible.
\QED

     In fact, the intersection curve is a space quartic with one isolated
point.  However we will not elaborate this issue here.


% ********************************************************************
%                          COMMON LINE
% ********************************************************************

\subsection{A Double Line and Conic}
\label{section:common-line}

     If $V_1\in S_2, V_2\in S_1$ and $V_1\neq V_2$, we have
a common line $\stackrel{\longleftrightarrow}{V_1V_2}$ on the axial plane
(Figure~\ref{fig:v1v2}).
Since the plane containing this line and perpendicular to the axial plane is
tangent to both surfaces, this  is a double line.  By Bezout's theorem, the
residue curve of the complete quadric intersection is a conic.
In this section we
will find the residue conic.  We will concentrate on the cone--cone case.
The same argument works for the cylinder--cone case.  Since the
intersection of two cylinders is strictly linear, it will not be discussed.
\begin{figure}
\vspace{6.5cm}
\caption{The Common Line $\stackrel{\longleftrightarrow}{V_1V_2}$ on the Axial P
lane}
\label{fig:v1v2}
\end{figure}

If $\ell_1$ is parallel to $\ell_2$,
where $\ell_1$ and $\ell_2$ are the axes of the cones,
the two given cones have equal cone angles.
Their intersection consists of the double line
$\stackrel{\longleftrightarrow}{V_1V_2}$ and a circle at infinity on which
any point is the intersection of two corresponding parallel rulings, one from
each cone.

     In general, $\ell_1$ is not parallel to $\ell_2$.  The common
inscribed sphere is crucial to the computation of the conic intersection
in this case.

\begin{lemma}
\label{lemma:common-sphere}
     Let $O=\ell_1\cap\ell_2$ and $d$ be the distance from $O$ to
$\stackrel{\longleftrightarrow}{V_1V_2}$.  The sphere ${\cal S}$ with center
$O$ and radius $d$ is a common inscribed sphere of both cones
(Figure~\ref{fig:inscribed-sphere}).
\end{lemma}
{\bf Proof:}  Simple. \QED

     Note that $d=d_1\sin\alpha_1=d_2\sin\alpha_2$, where $d_i$ is the
distance from $O$ to $V_i$.
The following result is analogous to Lemma~\ref{lemma:coincide-conic}.

\begin{figure}
\vspace{4.5cm}
\caption{The Inscribed Sphere with center $O$ and radius $d$}
\label{fig:inscribed-sphere}
\end{figure}

\begin{theorem}
\label{theorem:double-line}
     Suppose two cones ${\cal Q}_1(V_1,\ell_1,\alpha_1)$ and
${\cal Q}_2(V_2,\ell_2,\alpha_2)$, $\ell_1$ not parallel to $\ell_2$,
have a common line $\stackrel{\longleftrightarrow}{V_1V_2}$ on the axial plane
${\cal H}$.  Let $P$ be the intersection point of the other two skeletal lines
and $Q$ the tangent point on $\stackrel{\longleftrightarrow}{V_1V_2}$ of the
common inscribed sphere.  Then the intersection conic of
${\cal Q}_1\cap{\cal Q}_2$ lies in the plane through
$\stackrel{\longleftrightarrow}{PQ}$ and perpendicular to the axial plane
${\cal H}$.
\end{theorem}
{\bf Proof:}  We know that the conic must lie in a plane
perpendicular to ${\cal H}$ (Lemma~\ref{lemma:perpendicular-stuff})
and must pass through $P$.  Thus we must only show that the plane
containing the conic passes through $Q$.

     Suppose the axial plane ${\cal H}$ is chosen to be the $xy$-plane with
$O=\ell_1\cap\ell_2$ the origin.  Let ${\cal S}(O,r)$ be a sphere.
Any cone with axis on the $xy$-plane and inscribed sphere ${\cal S}$ can
be obtained by rotating a  cone with equation
$-c^2(x-u)^2+y^2+z^2=0$, where $(u,0,0), u>0$, is the cone vertex and
$\alpha$ the cone angle, $c=\tan\alpha$.

     Let ${\cal S}(O,r)$ be the common inscribed sphere of both cones
(Lemma~\ref{lemma:common-sphere}).  By rotation we can assume without loss of
generality that these two cones have a common skeletal line $y+r=0$.
It suffices to show that the planar members of the pencil
formed by these two transformed cones both pass through
$Q=(0,-r,0)$, the tangent point of the common line and the common inscribed
sphere.

     If the cone $-c^2(x-u)^2+y^2+z^2=0$ has an inscribed sphere
$x^2+y^2+z^2=r^2$, then $c^2u^2=r^2(1+c^2)$.  To see this, note that the cone
angle $\alpha$ must be an acute angle.  Let one of the skeletal lines be
tangent to the inscribed sphere at $T$ (Figure~\ref{fig:relation-of-cru}).
We have $c=\tan\alpha=\frac{|\overline{OT}|}{|\overline{VT}|}$.  Since
$r=|\overline{OT}|$ and $|\overline{TV}|=\sqrt{u^2-r^2}$, we have
$c^2=\frac{r^2}{u^2-r^2}$ and hence $c^2u^2=r^2(1+c^2)$.
Note that this relation is an invariant under rotation transformation, since
$r$ and $c$ are constants, and $u$ is the distance from the cone's vertex to
$O$, which does not change under rotation.
\begin{figure}
\vspace{4.5cm}
\caption{The Relation of $c, r$ and $u$}
\label{fig:relation-of-cru}
\end{figure}

     After rotation, the first cone is a  cone rotated by $\alpha$, or
\begin{equation}
\label{eqn:cone-1}
     (1-c^2)y^2+2cxy+2rcx-2c^2ry-r^2(1+c^2)+z^2=0,
\end{equation}
while the second cone, which is obtained by rotating the  cone
$-d^2(x-v)^2+y^2+z^2=0$, is
\begin{equation}
\label{eqn:cone-2}
     (1-d^2)y^2+2dxy+2rdx-2d^2ry-r^2(1+d^2)+z^2=0,
\end{equation}
where $d=\tan\beta$ or $d=\tan(\pi+\beta)$.
Subtracting (\ref{eqn:cone-2}) from (\ref{eqn:cone-1}) and applying
$c^2u^2=r^2(1+c^2)$ and $d^2v^2=r^2(1+d^2)$, we have

\[ y^2-\frac{2}{c+d}xy-\frac{2r}{c+d}x+2ry+r^2=0. \]
This equation can be factored into two linear factors:

\[ \left(y+r\right)\left(y-\frac{2}{c+d}x+r\right)=0. \]
$y+r=0$ and $y-\frac{2}{c+d}x+r=0$ are clearly the two planar members.
Note that both planes are perpendicular to the $xy$-plane as we expected and
both planes pass through $Q=(0,-r,0)$ as desired. \QED

     Based on this theorem, the following algorithm computes the
intersection conic.

\begin{center}
\begin{minipage}{5in}
\vspace{8mm}
\begin{center}
     {\bf Algorithm:} Common Line and Conic Intersection
\end{center}
{\small
\begin{enumerate}
     \item If $\ell_1//\ell_2$, the intersection consists of only one finite
          component $\stackrel{\longleftrightarrow}{V_1V_2}$.
     \item If $\ell_1$ and $\ell_2$ are not parallel, let $O=\ell_1\cap\ell_2$.
        Let $P$ be the intersection point of the other two skeletal lines.
          From $O$ drop a perpendicular
          to $\stackrel{\longleftrightarrow}{V_1V_2}$ meeting it at $Q$.
       The intersection conic, which lies in the plane through
       $\stackrel{\longleftrightarrow}{PQ}$ perpendicular to the axial
       plane, can be computed as follows:
     \begin{itemize}
          \item $P$ is finite :  Let $X$ be the midpoint of $\overline{PQ}$
               and compute the squared height $h_X^2$ to ${\cal C}_1$
               (or ${\cal C}_2$).  The intersection is a central conic with
               major axis $\stackrel{\longleftrightarrow}{PQ}$, center $X$,
            and the length of its semi-major axis is
               $\frac{1}{2}|\overline{PQ}|$.
               The minor axis is the line through $X$ and perpendicular to
               ${\cal H}$ and the length of the semi-minor axis is
            $\sqrt{|h_X^2|}$.  Note that if
               $h_X^2>0$, the conic is an ellipse, otherwise it is a hyperbola.
          \item $P$ is at infinity : The intersection
            is a parabola with major axis
               $\stackrel{\longleftrightarrow}{PQ}$ and vertex $Q$.  The focal
               length of this parabola can be determined using the technique
               of Section~\ref{section:height}.
     \end{itemize}
\end{enumerate}
}
\end{minipage}
\end{center}

\begin{remark}\rm
     Alternatively, since the plane containing the intersection conic is
perpendicular to the axial plane and passing  through
$\stackrel{\longleftrightarrow}{PQ}$, we can use the algorithms given
in Johnstone and Shene~\cite{johnstone-shene:1991} to compute the intersection
conic by intersecting this plane with one of the two given surfaces.
Another way of computation is directly determining one of the foci on
$\stackrel{\longleftrightarrow}{PQ}$ using a Dandelin sphere
(Shene~\cite{shene:1992}).  $\Box$
\end{remark}



% --------------------------------------------------------------------
%                        LINEAR INTERSECTION
% --------------------------------------------------------------------

\section{Linear Intersections}
\label{section:linear}

This section handles the last degenerate case, $V_1=V_2$.
We will show that $V_1=V_2$ implies linear intersection.
(Recall that under our
definition, a linear intersection is an intersection whose components are all
linear.)  For the cylinder--cylinder case, $V_1=V_2$ means all rulings on both
surfaces are parallel, since the vertex of a cylinder is the point at infinity
along the direction given by the axes.  In this case, the intersection is
either empty, a common tangent, or two parallel lines.  $V_1=V_2$ is obviously
impossible for the cylinder--cone case.  Thus, we need only consider
cone--cone intersection.

\begin{definition}
     Let ${\cal C}_1(V,\ell_1,\alpha_1)$ and ${\cal C}_2(V,\ell_2,\alpha_2)$
be two cones with the common vertex $V$ and distinct axes $\ell_1$ and
$\ell_2$.  Let $P_{11}$ and $P_{12}$ be distinct points on $\ell_1$ such that
$|\overline{P_{11}V}|=|\overline{P_{12}V}|=\cos\alpha_1$, while $P_{21}$ and
$P_{22}$ are distinct points on $\ell_2$ such that
$|\overline{P_{21}V}|=|\overline{P_{22}V}|=\cos\alpha_2$.
The lines on the axial plane through $P_{11}$ and $P_{12}$ and perpendicular
to $\ell_1$, and through $P_{21}$ and $P_{22}$ and perpendicular to $\ell_2$
form a parallelogram $ABCD$ (Figure~\ref{fig:linear}).  Denote the two
diagonals $\stackrel{\longleftrightarrow}{AC}$ and
$\stackrel{\longleftrightarrow}{BD}$ of this parallelogram by $L_1$ and $L_2$
respectively.
The $L_i$ are important and act like diagonals (see
Theorem~\ref{theorem:linear-intersection} below).
\end{definition}

The following lemma characterizes $L_1$ and $L_2$.

\begin{figure}
\vspace{4.5cm}
\caption{Lines $L_1$ and $L_2$}
\label{fig:linear}
\end{figure}

\begin{lemma}
\label{lemma:L1-L2}
     A point $P$, $P\neq V$, on ${\cal H}$ lies on $L_1\cup L_2$ if and only
if the ratio of distances from $V$ to the perpendicular feet of $P$ on
$\ell_1$ and $\ell_2$ is $\frac{\cos\alpha_1}{\cos\alpha_2}$.
\end{lemma}
{\bf Proof:}  See Figure~\ref{fig:linear}. \QED

     $L_1$ and $L_2$ in turn characterize the intersection lines of
${\cal C}_1$ and ${\cal C}_2$.
Notice the similarity with Lemma~\ref{lemma:coincide-conic} again.

\begin{theorem}
\label{theorem:linear-intersection}
     Two cones with identical vertices ${\cal C}_1(V,\ell_1,\alpha_1)$ and
${\cal C}_2(V,\ell_2,\alpha_2)$, $\ell_1\neq\ell_2$, always intersect in
lines.  The lines are the intersection of either cone with the plane through
$L_i,i=1,2$ and perpendicular to the axial plane.
\end{theorem}
{\bf Proof:}  First, we will show that the plane through $L_i$ and
perpendicular to the axial plane ${\cal H}$ contains one pair of intersection
lines of ${\cal C}_1\cap{\cal C}_2$ by computing the squared heights to both
cones at an arbitrary point $T\in L_i, T\neq V$. (This is obviously true for
$T=V$.)  From $T$, drop a perpendicular to $\ell_i$ meeting it at $T_i,i=1,2$
(Figure~\ref{fig:quadrilateral}(a)).  By the above lemma, we can assume
$|\overline{VT_i}|=t\cos\!\alpha_i, i=1,2$, for some $t>0$.
Let $t_i=|\overline{TT_i}|$ and $p=|\overline{VT}|$.  Then the squared height
from $T$ to ${\cal C}_i$ is $h_i^2=t^2\sin^2\!\alpha_i-t_i^2$
(Figure~\ref{fig:quadrilateral}(b)).  From $\bigtriangleup TT_iV$, since
$\angle TT_iV=\frac{\pi}{2}$,
$p^2=t^2\cos^2\!\alpha_i+t_i^2=t^2-t^2\sin^2\!\alpha_i+t_i^2$ and hence
$t^2\sin^2\!\alpha_i-t_i^2=t^2-p^2$.  Therefore, we have $h_1^2=t^2-p^2=h_2^2$.
These equal heights imply that the plane through $L_i$ and perpendicular to
${\cal H}$ cuts the cones in a common intersection curve.  Since this plane
passes though the common vertex $V$, the common intersection curve must be
a pair of lines.  According as $h_1^2=h_2^2$ is positive,
zero, or negative, the intersection lines are distinct real, double real, or
conjugate imaginary.  With the above argument, we have already found four
lines of intersection.  By Bezout's theorem, these are the only intersections
of ${\cal C}_1\cap{\cal C}_2$. \QED

\begin{figure}
\vspace{4cm}
\caption{Quadrangle $TT_1VT_2$ and the Height at $T$}
\label{fig:quadrilateral}
\end{figure}

     The computation procedure is simple and consists of the following steps:
\begin{center}
\begin{minipage}{5in}
\vspace{8mm}
\begin{center}
     {\bf Algorithm:} Linear Intersection ($V_1=V_2$)
\end{center}
{\small
\begin{enumerate}
     \item Determine the axial plane ${\cal H}$.
     \item Determine the parallelogram $ABCD$ (Figure~\ref{fig:linear}).
     \item For points $A$ and $D$, compute the squared heights $h_A^2$ and
          $h_D^2$.  One can use the method in the proof of
          Theorem~\ref{theorem:linear-intersection}.
     \item Based on the sign of $h_A^2$ we have three cases:
     \begin{enumerate}
          \item $h_A^2>0$ : On the line through $A$ and perpendicular to
               the axial plane, choose the two points at distance $h_A$
               from $A$.  Connecting these two points to $V$ gives two
               intersection lines.
          \item $h_A^2=0$ : The intersection line is $L_1$, the line through
               $V$ and $A$.
          \item $h_A^2<0$ : The intersection lines are imaginary.  Thus, for
               our purpose, the intersection consists of only one point,
               the vertex $V$.
     \end{enumerate}
     \item Do the same thing using $D$ and $h_D^2$.
\end{enumerate}
}
\end{minipage}
\end{center}


% -----------------------------------------------------------------
%           SIMPLIFICATION OF GOLDMAN-MILLER THEORY
% -----------------------------------------------------------------

\section{A Necessary and Sufficient Condition for Conic Intersection: The
     Common Inscribed Sphere}
\label{section:simplification}

     It is well known by descriptive geometers that two axial natural
quadrics with intersecting axes have conic intersection if they have a
common inscribed sphere.  For example, Bereis~\cite{bereis:1964} and
Rehbock~\cite{rehbock:1964} use this property implicitly, while
Gordon and Sementsov-Ogievskii~\cite[p. 283]{gordon:1980} state
it explicitly and indicate that it is also true for revolute quadrics.
However, in this literature,
the existence of a common inscribed sphere is only
stated as a sufficient condition for conic
intersection.\footnote{In~\cite{goldman-warren:1990} Goldman and Warren
point out that this condition is provably not necessary for revolute
quadrics.}

     Recently, Shene and Johnstone~\cite{shene-johnstone:1991a} and Goldman and
Miller~\cite{g-m:1990,g-m:1991a} have shown that
the common inscribed sphere is
actually a necessary and sufficient condition.  The method and proof of
Goldman and Miller is an exhaustive algebraic analysis, while the method and
proof of Shene and Johnstone is geometric.
In this section, using Lemma~\ref{lemma:perpendicular-stuff} and
Lemma~\ref{lemma:non-skew} as our starting point, we will provide a new and
very short algebraic proof for the common inscribed sphere
criterion.\footnote{To the best of our knowledge, this is also the only proof
published in a journal.}
The purpose of this proof is to illustrate how the use of the axial plane
can simplify previous proofs.
Another algebraic proof can be found in Goldman and
Warren~\cite{goldman-warren:1990}.

\begin{remark}\rm
     For a complete geometric proof of this criterion, the interested reader
should consult Shene and Johnstone~\cite{shene-johnstone:1991b} and
Shene~\cite{shene:1992}.  Moreover, a geometric proof that establishes the
equivalence among the existence of a planar intersection, the existence of a
common inscribed sphere and the existence of a Dupin blending cyclide is
presented in Shene~\cite{shene:1992,shene:1993a}. $\Box$
\end{remark}

     The following lemma provides all of the necessary theory for our proof.

\begin{lemma}
\label{lemma:factor}
     Two axial natural quadrics with coplanar and distinct axes have conic or
linear intersection if and only if the parallel projection of their
intersection curve to the axial plane factors.
The projection curve is of second degree.
\end{lemma}
{\bf Proof:}  Suppose the axial plane is taken to be the $xy$-plane.
Because of symmetry, there will be no $z$-term in the equations of the given
axial natural quadrics, $Q_1(x,y)+z^2=0$ and $Q_2(x,y)+z^2=0$,
where $Q_1(x,y)$ and $Q_2(x,y)$ are second degree polynomials of $x$ and $y$.
Note that eliminating the $z$ terms from both equations is equivalent to
performing a parallel projection of the intersection curve to the
$xy$-plane.  Thus the projection curve
has equation $Q_1(x,y)-Q_2(x,y)=0$, which is obviously a conic.

     $(\Rightarrow)$ Recall that the intersection conics lie on two
planes both perpendicular to the axial plane.  Thus the projection generates
two straight lines.  Since the projection curve is of second degree and it
has two lines as components, it must factor.

     $(\Leftarrow)$ If the projection curve factors into two lines,
the intersection curve is contained in two planes through the lines and
perpendicular to the axial plane.  Each such plane cuts the surfaces
in a conic or two lines. \QED

     By Lemma~\ref{lemma:non-skew}, we can assume without loss of generality
that the two given axial natural quadrics have coplanar axes.
We have two theorems, based on whether the axes are parallel or intersecting.

\begin{theorem}[Common Inscribed Sphere Criterion]
\label{theorem:intersecting}
     Two axial natural quadrics with intersecting and distinct axes have
conic intersection if and only if they have a common inscribed sphere.
\end{theorem}
{\bf Proof:}
We will prove this theorem using two
cones.  The cases of cone--cylinder and cylinder--cylinder are similar.
We can simplify the calculation by making the axial plane the $xy$-plane.
We can also assume without loss of generality that the intersection point of
the two axes is the coordinate origin.
By an appropriate rotation bringing one cone's axis to the $x$-axis,
the corresponding equation becomes $-c^2(x-u)^2+y^2+z^2=0$, where $(u,0)$ is
the cone's vertex.  The second cone, which is the rotation of a cone
$-d^2(x-v)^2+y^2+z^2=0$ by angle $\theta\neq 0,\pi$, has the equation
\[ -d^2(x\cos\theta-y\sin\theta-v)^2+(x\sin\theta+y\cos\theta)^2+z^2=0. \]
Eliminating the $z^2$ term gives the following projection curve on the
$xy$-plane:
\[ [c^2-d^2+(1+d^2)\sin^2\theta]x^2-(1+d^2)\sin^2\theta y^2
   +2\sin\theta\cos\theta(1+d^2)xy \]
\[ +2(d^2v\cos\theta-c^2u)x
   -2d^2v\sin\theta y+(c^2u^2-d^2v^2)=0.                      \]
The discriminant of this conic (after some reduction) is
\begin{eqnarray*}
\Delta&=&\left| \begin{array}{ccc}
     c^2-d^2+(1+d^2)\sin^2\theta & \sin\theta\cos\theta(1+d^2) &
                     d^2v\cos\theta-c^2u \\
     \sin\theta\cos\theta(1+d^2) & -(1+d^2)\sin^2\theta & -d^2v\sin\theta \\
     d^2v\cos\theta-c^2u & -d^2v\sin\theta & c^2u^2-d^2v^2
                 \end{array} \right| \\
     &=& \sin^2\theta(1+c^2)(1+d^2)\left[
                    \frac{d^2v^2}{1+d^2}-\frac{c^2u^2}{1+c^2}
                       \right].
\end{eqnarray*}
Therefore the projection curve factors if and only if $\Delta=0$,  if and only
if $\frac{d^2v^2}{1+d^2}=\frac{c^2u^2}{1+c^2}$.  However, from the proof of
Theorem~\ref{theorem:double-line}, we have $r_1^2=\frac{c^2u^2}{1+c^2}$,
where $r_1$ is the radius of the inscribed sphere of the first cone with
center the origin.
Similarly, for the second cone, we have
$r_2^2=\frac{d^2v^2}{1+d^2}$,
where $r_2$ is the radius of the inscribed sphere of the second cone
with center the origin.
Now it is obvious that $\Delta=0$ if and only
if the radii of the inscribed spheres are equal.  That is, if and only if
the two given cones have a common inscribed sphere. \QED

    The following corollary, also reported in
Goldman and Miller~\cite{g-m:1990,g-m:1991a}, is a simple
consequence of Theorem~\ref{theorem:intersecting}.

\begin{corollary}
     Consider two axial natural quadrics with $\ell_1$ and $\ell_2$,
$\ell_1\neq\ell_2, O=\ell_1\cap\ell_2$,  as their axes.
\begin{itemize}
     \item Two cones ${\cal C}_1(V_1,\ell_1,\alpha_1)$ and
          ${\cal C}_2(V_2,\ell_2,\alpha_2)$ have a conic intersection
          if and only if $r=d_1\sin\alpha_1=d_2\sin\alpha_2$, where
          $d_i=|\overline{OV_i}|$ and $r$ the radius of the common
          inscribed sphere.
     \item A cone ${\cal C}(V,\ell_1,\alpha)$ and a cylinder
          ${\cal Z}(\ell_2,r)$ have a conic intersection if and only if
          $d_1\sin\alpha_1=r$, where $d_1=|\overline{OV_1}|$
          and $r$ the radius of the common inscribed sphere.
     \item Two cylinders ${\cal Z}_1(\ell_1,r_1)$ and
          ${\cal Z}_2(\ell_2,r_2)$ have conic intersection if and only
          if $r_1=r_2$.
\end{itemize}
\end{corollary}
{\bf Proof:}
The radius of the inscribed sphere of ${\cal C}_i$ with center $O$
is $d_i\sin\alpha_i$.
\QED

\begin{theorem}[Parallel Axes]
\label{theorem:parallel-axes}
     Two axial natural quadrics with parallel and distinct axes have conic
intersection if and only if they have equal cone angles.
\end{theorem}
{\bf Proof:}
Again assume that the axial plane is the $xy$-plane.
The cone with axis
the $x$-axis and vertex the origin has equation $-c^2x^2+y^2+z^2=0$,
while the other cone with axis parallel to the $x$-axis and vertex $(u,v)$
has equation $-d^2(x-u)^2+(y-v)^2+z^2=0$.  Note that since
the axes are distinct,
$v\neq 0$.  Eliminating the $z^2$ term gives the following
projection curve on the $xy$-plane:
\[ (c^2-d^2)x^2+2d^2ux-2vy+(v^2-d^2u^2)=0. \]
This conic's discriminant is the following:
\[ \Delta=\left| \begin{array}{ccc}
                    c^2-d^2 & 0 & d^2u \\
                        0   & 0 & -v   \\
                       d^2u & -v & v^2-d^2u^2
                 \end{array} \right| = -v^2(c^2-d^2). \]
Thus, the projection curve factors if and only if $\Delta=0$, if and only if
$c^2=d^2$. \QED

\begin{remark} \rm
Theorem~\ref{theorem:parallel-axes} also satisfies the
common inscribed sphere criterion, since
the two cones have a common inscribed sphere with center at the point at
infinity along the two parallel axes. $\Box$
\end{remark}


% --------------------------------------------------------------------
%              Enumerating All Conic Intersection Types
%
% REVISION HISTORY:
%    Written   -- Feb/02/93
%    Revised   -- May/09/93
% --------------------------------------------------------------------

\section{Enumerating All Conic Intersection Types}
\label{section:enumeration}

     Although two axial natural quadrics can intersect in two conics, some
combinations such as two parabolas are impossible.  In this
section, as an application of the common inscribed sphere criterion,
we will investigate the possible combinations that can be
obtained from two cones.
We will also give a very simple method for determining the types of conic
in a degenerate intersection (Table~\ref{tbl:transform}).

The technique we will use is to fix one cone and rotate the other about the
center of the common inscribed sphere.  By locating the positions
at which the
intersection type can change, and determining the intersection type at these
points and in the arc between any two adjacent positions, we will obtain a
clear picture of the possible degenerate intersections of two cones.

     Suppose two cones ${\cal C}_1(V_1,\ell_1,\alpha_1)$ and
${\cal C}_2(V_2,\ell_2,\alpha_2)$, $\alpha_1<\alpha_2$, have
conic intersection (and intersecting axes).
Then they have a common
inscribed sphere with center $O=\ell_1\cap\ell_2$.  Let $I$ be the intersection
circle of the common inscribed sphere and the axial plane.
Fix cone ${\cal C}_1$ and rotate ${\cal C}_2$'s vertex
$V_2$ about $O$ in the axial plane.
The locus of $V_2$ is a circle $J$ with
radius $|\overline{OV_2}|$ and center $O$.  Because $\alpha_1<\alpha_2$, $V_1$
lies outside of $J$.

\begin{definition}
Parallel to each skeletal line of ${\cal C}_1$, construct
a tangent line of $I$.  These two tangent lines meet at a point $V_1^\prime$,
which also lies outside of $J$.  Since these two pairs of parallel lines form a
parallelogram and $J$ has larger radius, $J$ cuts the four sides of the
parallelogram in eight points.  We call them the
{\bf critical points}\index{critical point}\index{point,critical},
since the types of the intersection conics change
when $V_2$ crosses these points.
\end{definition}

If $V_2$ is one of the critical points,
two skeletal lines, one from each skeletal pair, are either parallel or
identical.  In the former case, one of the intersection conics is a parabola,
while in the latter case one of the intersection conics degenerates to a double
line.  Between any two consecutive critical points, it is not difficult to see
that the types of the intersection conics do not change.
We have three cases to consider:
(1) $\alpha_1+\alpha_2<\frac{\pi}{2}$, (2) $\alpha_1+\alpha_2=\frac{\pi}{2}$,
and (3) $\alpha_1+\alpha_2>\frac{\pi}{2}$ (Figure~\ref{fig:enum}).
\begin{figure}
\vspace{4.5cm}
\caption{Enumeration of All Types of Intersection Conics}
\label{fig:enum}
\end{figure}

     One should have no difficulty verifying the results documented in
Table~\ref{tbl:case-1}.
In this table, $\cal E, P, H$ and $\cal L$ indicate an ellipse, a parabola, a
hyperbola and a double line respectively.  Note that this table is based on
the assumption that $\alpha_1<\alpha_2$.  For $\alpha_1=\alpha_2$, we have
$V_1^\prime=A=H$ and $V_1=D=E$ and the corresponding parts in the table still
give correct results.
\begin{table*}
\caption{Intersection Type for Two Cones $(\alpha_1<\alpha_2)$}
\label{tbl:case-1}
$$
\BeginTable
     \OpenUp22
     \def\E{$\cal E$}
     \def\P{$\cal P$}
     \def\H{$\cal H$}
     \def\L{$\cal L$}
     \BeginFormat
     |5   c   | c   |   c   | c  |5
     \EndFormat
     \_5
     |\it Range | $\alpha_1+\alpha_2<\frac{\pi}{2}$ |
                  $\alpha_1+\alpha_2=\frac{\pi}{2}$ |
                  $\alpha_1+\alpha_2>\frac{\pi}{2}$ | \\ \_3
     | $(H,A)$             | \use3 \E+\E                       | \\ \_
     | $A$ and $H$         | \use3 \E+\P                       | \\ \_
     | $(A,B)$ and $(G,H)$ | \use3 \E+\H                       | \\ \_
     | $B$ and $G$         | \E+\L |        | \H+\P            | \\
     & \use2 \=                    &        & \=               & \\0
     | $(B,C)$ and $(F,G)$ | \E+\E | \P+\L  | \H+\H            | \\
     & \use2 \=                    &        & \=               & \\0
     | $C$ and $F$         | \E+\P |        | \H+\L            | \\ \_
     | $(C,D)$ and $(E,F)$ | \use3 \E+\H                       | \\ \_
     | $D$ and $E$         | \use3 \E+\L                       | \\ \_
     | $(D,E)$             | \use3 \E+\E                       | \\ \_5
\EndTable
$$
\end{table*}

     We can transform Table~\ref{tbl:case-1} into a format that is even easier
to use.  In particular, we can replace the computation of the critical points
by the computation of $\cos\theta=\arrow{v}_1\cdot\arrow{v}_2$, where
$\arrow{v}_1$ and $\arrow{v}_2$ are the normalized vectors of
$\stackrel{\longrightarrow}{OV}_1$ and $\stackrel{\longrightarrow}{OV}_2$, and
$\theta$ is the angle between $\arrow{v}_1$ and $\arrow{v}_2$.
Suppose circle $J$ intersects one of the skeletal lines of
${\cal C}_1$ at $X_1$ and $Y_1$ with $X_1 \in \overline{V_1 Y_1}$
(Figure~\ref{fig:transform}(a)).
Then, $\angle OX_1Y_1=\angle OY_1X_1=\alpha_2$.  Therefore,
$\angle V_1OX_1=\alpha_2-\alpha_1$ and
$\angle V_1OY_1=\angle V_1OX_1+\angle X_1OY_1=
(\alpha_2-\alpha_1)+(\pi-2\alpha_2)=\pi-(\alpha_1+\alpha_2)$.
Thus, if $V_2$ is identical to
$X_1$ ({\em resp.}, $Y_1$), we have $\theta=\alpha_2-\alpha_1$
({\em resp.}, $\theta=\pi-(\alpha_1+\alpha_2$)).
Note that $X_1$ and $Y_1$ are two of the four points $A,B,C$ and $D$.
Finally, let $X_2$ and $Y_2$ be the symmetric images with respect
to the line through
$O$ and perpendicular to $\stackrel{\longleftrightarrow}{V_1V_1^\prime}$
(Figure~\ref{fig:transform}(b)).
Notice that
$\angle V_1OX_2=\pi-\angle V_1OX_1=\pi-(\alpha_2-\alpha_1)$ and
$\angle V_1OY_2=\pi-\angle V_1OY_1=\alpha_1+\alpha_2$.
We can now identify the critical points with values of angle $\theta$,
splitting the problem into the following three cases:
\begin{figure}
\vspace{4cm}
\caption{Determining the Angles of the Critical Points}
\label{fig:transform}
\end{figure}
\begin{enumerate}
     \item \ [$\alpha_1+\alpha_2<\frac{\pi}{2}$]:
          $X_1$, $Y_1$, $X_2$ and $Y_2$ correspond to $D$, $B$, $A$ and $C$
          respectively (Figure~\ref{fig:enum}(a)), and the
          $\theta$'s corresponding to $V_1$, $D$, $C$, $B$, $A$ and
       $V_1^\prime$ are \\$0 < \alpha_2-\alpha_1 < \alpha_1+\alpha_2
          < \pi-(\alpha_1+\alpha_2) < \pi-(\alpha_2-\alpha_1) < \pi$,
       respectively.

     \item \ [$\alpha_1+\alpha_2=\frac{\pi}{2}$]:
          $X_1$, $Y_1=Y_2$ and $X_2$ correspond to $D$, $B=C$ and $A$
          respectively (Figure~\ref{fig:enum}(b)), and the
          $\theta$'s corresponding to $V_1$, $D$, $B=C$, $A$ and $V_1^\prime$
          are \\$0 < \alpha_2-\alpha_1 < \alpha_1+\alpha_2=\frac{\pi}{2}
          < \pi-(\alpha_2-\alpha_1) < \pi$, respectively.

     \item \ [$\alpha_1+\alpha_2>\frac{\pi}{2}$]:
          $X_1$, $Y_2$, $X_2$ and $Y_2$ correspond to $D$, $B$, $A$ and $C$
          as in the first case (Figure~\ref{fig:enum}(c)), however
          the $\theta$'s corresponding to $V_1$, $D$, $C$, $B$, $A$ and
          $V_1^\prime$ are $0 < \alpha_2-\alpha_1 < \pi-(\alpha_1+\alpha_2)
          < \alpha_1+\alpha_2 < \pi-(\alpha_2-\alpha_1) < \pi$, respectively.
\end{enumerate}

     For the sake of simplicity, let $c_{-}$ and $c_{+}$ be
$\cos(\alpha_2-\alpha_1)$ and $\cos(\alpha_1+\alpha_2)$ respectively.
Using $c_{-}$, $c_{+}$ and $\cos\theta$, Table~\ref{tbl:case-1} is transformed
to the more manageable form of Table~\ref{tbl:transform}
(where ${\cal C}$ indicates a circle).

\begin{table*}
\caption{Intersection Types for Two Cones $(\alpha_1<\alpha_2)$}
\label{tbl:transform}
$$
\BeginTable
     \OpenUp22
     \ninepoint
     \def\E{$\cal E$}
     \def\P{$\cal P$}
     \def\H{$\cal H$}
     \def\L{$\cal L$}
     \def\C{$\cal C$}
     \BeginFormat
     |5 c | c | c | c | c | c | c | c | c | c | c | c |5
     \EndFormat
     " \use{12} \JustCenter (a) $\alpha_1+\alpha_2<\frac{\pi}{2}$ " \\ \_5
     |\it $\cos\theta$ | 1 " $\dots$ " $c_{-}$
          " $\dots$ " $c_{+}$ " $\dots$
          " $-c_{+}$ " $\dots$ " $-c_{-}$
          " $\dots$ " $-1$ | \\ \_3
     | \it Type | \C+\C " \E+\E " \E+\L " \E+\H " \E+\P
          " \E+\E " \E+\L " \E+\H " \E+\P " \E+\E " \C+\C | \\
     \_5
\EndTable
$$

\vspace{0.8cm}
$$
\BeginTable
     \OpenUp22
     \ninepoint
     \def\E{$\cal E$}
     \def\P{$\cal P$}
     \def\H{$\cal H$}
     \def\L{$\cal L$}
     \def\C{$\cal C$}
     \BeginFormat
     |5 c | c | c | c | c | c | c | c | c | c |5
     \EndFormat
     " \use{10} \JustCenter (b) $\alpha_1+\alpha_2=\frac{\pi}{2}$ " \\ \_5
     |\it $\cos\theta$ | 1 " $\dots$ " $c_{-}$
          " $\dots$ " 0 " $\dots$
          " $-c_{-}$ " $\dots$ " $-1$ | \\ \_3
     | \it Type | \C+\C " \E+\E " \E+\L " \E+\H " \P+\L
          " \E+\H " \E+\P " \E+\E " \C+\C | \\
     \_5
\EndTable
$$
\vspace{0.8cm}
$$
\BeginTable
     \OpenUp22
     \ninepoint
     \def\E{$\cal E$}
     \def\P{$\cal P$}
     \def\H{$\cal H$}
     \def\L{$\cal L$}
     \def\C{$\cal C$}
     \BeginFormat
     |5 c | c | c | c | c | c | c | c | c | c | c | c |5
     \EndFormat
     " \use{12} \JustCenter (c) $\alpha_1+\alpha_2>\frac{\pi}{2}$ " \\ \_5
     |\it $\cos\theta$ | 1 " $\dots$ " $c_{-}$
          " $\dots$ " $-c_{+}$ " $\dots$
          " $c_{+}$ " $\dots$ " $-c_{-}$
          " $\dots$ " $-1$ | \\ \_3
     | \it Type | \C+\C " \E+\E " \E+\L " \E+\H " \H+\L
          " \H+\H " \H+\P " \E+\H " \E+\P " \E+\E " \C+\C | \\
     \_5
\EndTable
$$
\end{table*}

In conclusion,
if  the common inscribed sphere criterion indicates that the cones have
conic intersection, then the types of the intersection conics can be
determined immediately as follows.
First, from the cone vertices and the center of the
common inscribed sphere, we can compute the vectors $\arrow{v}_1$ and
$\arrow{v}_2$ and hence
$\cos\theta=\arrow{v}_1\cdot\arrow{v}_2$.  Then, based on
the relation between $\alpha_1+\alpha_2$ and $\frac{\pi}{2}$,
we can choose an appropriate
table in Table~\ref{tbl:transform} and search using $\cos\theta$
to find the types of the intersection conics.



% -----------------------------------------------------------------------
%                          CONCLUSIONS
%
% REVISION HISTORY:
%    1)   Sept/04/93     Written
%    2)   May/14/93      Remove all references to Part II
% -----------------------------------------------------------------------

\section{Conclusions}
\label{section:conclusion}

     This paper has successfully developed a fully geometric technique for
detecting and computing the planar intersections of two natural quadric
surfaces.  The fundamental method involves a comparison of two heights from a
point in the axial plane to the surfaces.  In developing the method, the
structure of planar intersection has been revealed, such as that the axes are
coplanar and the planes containing the intersection curves are perpendicular
to this plane.

     The techniques of this paper can be generalized to cover more general
quadric surfaces, and we are working on this problem.   However, some lemmas
presented here will not hold for more general surfaces; for example, planar
intersection need not imply coplanar axes.  Another interesting question is
whether one can compute the higher degree intersection curves of natural
quadrics using the techniques presented in this paper.


% ---------------------------------------------------------------------
%                       THE BIBLIOGRAPHY
% ---------------------------------------------------------------------

\newpage
\begin{thebibliography}{999}

\bibitem{abhyankar-bajaj:1989}
     Abhyankar, S. S., and Bajaj, C. L.
     Automatic Parameterization of Rational Curves and Surfaces IV:
          Algebraic Space Curves,
     {\em ACM Transactions on Graphics},
     Vol. 8 (1989), No. 4 (October), pp. 325--334.

\bibitem{bereis:1964}
     Bereis, R.
     {\em Darstellende Geometrie I},
     Akademie-Verlag, Berlin, 1964.

\bibitem{boehm:1990}
     Boehm, W.
     On Cyclides in Geometric Modeling,
     {\em Computer Aided Geometric Design},
     Vol. 7 (1990), pp. 243--255.

\bibitem{bromwich:1971}
     Bromwich, T. J. I'A
     {\em Quadratic Forms and Their Classification
          by Means of Invariant-Factors},
     Hafner Publishing Company, New York, 1971.

\bibitem{dandelin:1822}
     Dandelin, G. P.
     M\'{e}moire sur quelques propri\'{e}t\'{e}s remarquables de la Focal
     Parabolique,
     {\em Nouveaux M\'{e}moires de l'Acad\'{e}mie Royale des Sciences et
     Belles-lettres de Bruxelles},
     Vol. 2 (1822), pp. 171--202.

\bibitem{farouki:1989}
     Farouki, R. T., Neff, C. A., and O'Connor, M. A.
     Automatic Parsing of Degenerate Quadric-Surface Intersection,
     {\em ACM Transaction on Graphics},
     Vol. 8 (1989), No. 3, pp. 174--203.

\bibitem{garrity-warren:1989}
     Garrity, T., and Warren, J. D.
     On Computing the Intersection of a Pair of Algebraic Surfaces,
     {\em Computer Aided Geometric Design},
     Vol. 9 (1989), No. 2 (May), pp. 137--153.

%\bibitem{graphics-gems}
%     Glassner, A., editor.
%     {\em Graphics Gems},
%     Academic Press, New York, 1990.

\bibitem{goldman:1983b}
     Goldman, R. N.
     Two Approaches to a Computer Model for Quadric Surfaces,
     {\em IEEE Computer Graphics and Applications},
     Vol. 3 (1983), No. 6, pp. 21--24.

\bibitem{goldman:1990}
     Goldman, R. N.
     Intersection of Two Lines in Three-Space,
     in {\em Graphics Gems}, edited by Andrew S. Glassner,
     Academic Press, Boston, 1990, p. 304.

\bibitem{g-m:1990}
     Goldman, R. N., and Miller, J. R.
     Detecting and Calculating Conic Sections in the Intersection of Two
     Natural Quadric Surfaces, Part I: Detection,
     Draft, October 1990.

\bibitem{g-m:1991a}
     Goldman, R. N., and Miller, J. R.
     Combining Algebraic Rigor with Geometric Robustness for the Detection
          and Calculation of Conic Sections in the Intersection of Two
          Natural Quadric Surfaces,
     in {\em ACM Symposium on Solid Modeling Foundations and
          CAD/CAM Applications},
     June 1991, pp. 221--231.

\bibitem{goldman-warren:1990}
     Goldman, R. N., and Warren, J. D.
     A Sufficient Condition for the Intersection of Two Quadrics of Revolution
     to Degenerate into a Pair of Conic Sections,
     Technical Report, Rice University, December 1990.

\bibitem{gordon:1980}
     Gordon, V. O., and Sementsov-Ogievskii, M. A.
     {\em A Course in Descriptive Geometry},
     translated from Russian by Leonid Levant,
     MIR Publishers, Moscow, 1980.

\bibitem{hakala:1980}
     Hakala, D. G., Hillyard, R. C., Nourse, B. E., and Malraison, P. J.
     Natural Quadrics in Mechanical Design,
     {\em Proceedings of Autofact West 1},
     Anaheim, CA., Nov. 1980, pp. 363--378.

\bibitem{hilbert:1952}
     Hilbert, D. and Cohn-Vossen, S.
     {\em Geometry and the Imagination},
     translated from the German edition
     {\em Anschauliche Geometrie}
     by P. Nemenyi,
     Chelsea, New York, 1952.

\bibitem{hoffmann:1989}
     Hoffmann, C. M.
     {\em Geometric and Solid Modeling: An Introduction},
     Morgan Kaufmann, San Mateo, CA, 1989.

\bibitem{hohmeyer:1991}
     Hohmeyer, M. E.
     A Surface Intersection Algorithm Based on Loop Detection,
     {\em International Journal of Computational Geometry \& Applications},
     Vol. 1 (1991), No. 4 (December), pp. 473--490.

\bibitem{johnstone-shene:1991}
     Johnstone, J. K. and Shene, C.-K.
     Computing the Intersection of a Plane and a Natural Quadric,
     {\em Computers \& Graphics},
     Vol. 16 (1992), No. 2, pp. 179--186.

\bibitem{johnstone-shene:1992}
     Johnstone, J. K. and Shene, C.-K.
     Dupin Cyclides as Blending Surfaces for Cones,
     to appear in {\em Mathematics of Surfaces V},
     Oxford University Press.

\bibitem{levin:1976}
     Levin, J. Z.
     A Parametric Algorithm for Drawing Pictures of Solid Objects Composed
          of Quadric Surfaces,
     {\em Communications of ACM},
     Vol. 19 (1976), No. 10, pp. 555--563.

\bibitem{levin:1979}
     Levin, J. Z.
     Mathematical Models for Determining the Intersections of Quadric Surfaces,
     {\em Computer Graphics and Image Processing},
     Vol. 11 (1979), pp. 73--87.

\bibitem{manocha-canny:1991}
     Manocha, D., and Canny, J.
     A New Approach for Surface Intersection,
     {\em International Journal of Computational Geometry \& Applications},
     Vol. 1 (1991), No. 4, pp. 491--516.

\bibitem{miller:1987}
     Miller, J. R.
     Geometric Approaches to Nonplanar Quadric Surface Intersection Curves,
     {\em ACM Transactions on Graphics},
     Vol. 6 (1987), No. 4, pp. 274--307.

\bibitem{m-g:1991}
     Miller, J. R., and Goldman, R. N.
     Detecting and Calculating Conic Sections in The Intersection of
     Two Natural Quadric Surfaces, Part II: Calculation,
     Draft, February 1991.

\bibitem{m-g:1991a}
     Miller, J. R., and Goldman, R. N.
     Using Tangent Balls to Find Plane Sections of Natural Quadric Surfaces,
     {\em IEEE Computer Graphics and Applications},
     Vol. 12 (1992), No. 2 (March), pp. 68--82.

\bibitem{ocken:1987}
     Ocken, S., Schwartz, J. T., and Sharir, M.
     Precise Implementation of CAD Primitives Using Rational Parameterizations
     of Standard Surfaces,
     in {\em Planning, Geometry, and Complexity of Robot Motion}, edited by
     Jacob T. Schwartz, Micha Sharir and John Hopcroft,
     Ablex Publishing Co., Norwood, New Jersey, 1987, pp. 245--266.

\bibitem{oconnor:1989}
     O'Connor, M. A.
     Natural Quadrics: Projections and Intersections,
     {\em IBM Journal of Research and Development},
     Vol. 33 (1989), No. 4 (July), pp. 417--446.

\bibitem{piegl:1989}
     Piegl, L.
     Geometric Method of Intersecting Natural Quadrics Represented in Trimmed
     Surface Form,
     {\em Computer Aided Design}, Vol. 21 (1989), No. 4, pp. 201--212.

\bibitem{piegl:1992}
     Piegl, L.
     Constructive Geometric Approach to Surface-Surface Intersection,
     in {\em Geometry Processing for Design and Manufacturing},
     edited by Robert E. Barnhill, SIAM, Philadelphia, 1992, pp. 137--159.

\bibitem{pratt:1990}
     Pratt, M. J.
     Cyclides in Computer Aided Geometric Design,
     {\em Computer Aided Geometric Design},
     Vol. 7 (1990), pp. 221--242.

\bibitem{pratt:1992}
     Pratt, M. J.
     The Virtues of Cyclides in CAGD,
     in {\em Mathematical Methods in Computer Aided Geometric Design II},
     edited by Tom Lyche and Larry L. Schumaker,
     Academic Press, Boston, 1992, pp. 457--473.

\bibitem{rehbock:1964}
     Rehbock, F.
     {\em Darstellende Geometrie},
     second edition, Springer-Verlag, New York, 1964.

\bibitem{sabin:1976}
     Sabin, M. A.
     A Method for Displaying the Intersection Curve of Two Quadric Surfaces,
     {\em The Computer Journal},
     Vol. 19 (1976), No. 4, pp. 336--338.

\bibitem{sabin}
     Sabin, M. A.
     as communicated to Pratt in \cite{pratt:1990}.

\bibitem{samuel:1976}
     Samuel, N. M., Requicha, A. A. G., and Elkind, S. A.
     Methodology and Results of an Industrial Part Survey,
     Technical Memorandum No. 21,
     Production Automation Project,
     University of Rochester, Rochester, NY, 1976.

\bibitem{sarraga:1983}
     Sarraga, R. F.
     Algebraic Methods for Intersections of Quadric Surfaces in GMSOLID,
     {\em Computer Vision, Graphics, and Image Processing},
     Vol. 22 (1983), pp. 222--238.

\bibitem{shene:1992}
     Shene, C.-K.
     Planar Intersection and Blending of Natural Quadrics,
     Ph.D. thesis,
     Department of Computer Science,
     The Johns Hopkins University,
     Baltimore, Maryland, August 1992.

\bibitem{shene:1993a}
     Shene, C.-K.
     Planar Intersection, Common Inscribed Sphere and Blending Dupin Cyclides
     (Extended Abstract),
     {\em Second ACM/IEEE Symposium on Solid Modeling and
     Applications}, May 1993.

\bibitem{shene-johnstone:1991a}
     Shene, C.-K., and Johnstone, J. K.
     On the Planar Intersection of Natural Quadrics,
     in {\em ACM Symposium on Solid Modeling Foundations and
     CAD/CAM Applications}, June 1991, pp. 233--242.

\bibitem{shene-johnstone:1991c}
     Shene, C.-K., and Johnstone, J. K.
     On the Lower Degree Intersections of Two Natural Quadrics,
     Technical Report JHU--91/15,
     Department of Computer Science,
     The Johns Hopkins University,
     September 1991.

\bibitem{shene-johnstone:1991b}
     Shene, C.-K., and Johnstone, J. K.
     When Does a Quadrilateral Have an Inscribed Circle?
     Technical Report JHU--91/22,
     Department of Computer Science,
     The Johns Hopkins University,
     December 1991.

\bibitem{warren:1987}
     Warren, J. D.
     Blending Quadric Surfaces with Quadric and Cubic Surfaces,
     {\em Proceedings of the Third Annual Symposium on Computational Geometry},
     Waterloo, Ontario, Canada 1987, pp. 341--347.

\bibitem{warren:1989}
     Warren, J. D.
     Blending Algebraic Surfaces,
     {\em ACM Transactions on Graphics},
     Vol. 8 (1989), No. 4 (October), pp. 263--278.

\bibitem{wilson:1987}
     Wilson, P. R.
     Conic Representations for Shape Description,
     {\em IEEE Computer Graphics and Applications},
     Vol. 7 (1987), No. 4, pp. 23--40.

\end{thebibliography}


% --------------------------- THE END --------------------------------

\end{document}

