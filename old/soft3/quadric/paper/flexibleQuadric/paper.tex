\documentstyle[12pt]{article} 

\newif\ifFull
\Fulltrue

\makeatletter
\def\@maketitle{\newpage
 \null
 %\vskip 2em                   % Vertical space above title.
 \begin{center}
       {\Large\bf \@title \par}  % Title set in \Large size. 
       \vskip .5em               % Vertical space after title.
       {\lineskip .5em           %  each author set in a tabular environment
	\begin{tabular}[t]{c}\@author 
	\end{tabular}\par}                   
  \end{center}
 \par
 \vskip .5em}                 % Vertical space after author
\makeatother

% non-indented paragraphs with xtra space
% set the indentation to 0, and increase the paragraph spacing:
\parskip=8pt plus1pt
\parindent=0pt
% default values are 
% \parskip=0pt plus1pt
% \parindent=20pt
% for plain tex.

\newenvironment{summary}[1]{\if@twocolumn
\section*{#1} \else
\begin{center}
{\bf #1\vspace{-.5em}\vspace{0pt}} 
\end{center}
\quotation
\fi}{\if@twocolumn\else\endquotation\fi}

% \renewenvironment{abstract}{\begin{summary}{Abstract}}{\end{summary}}

\newcommand{\SingleSpace}{\edef\baselinestretch{0.9}\Large\normalsize}
\newcommand{\DoubleSpace}{\edef\baselinestretch{1.4}\Large\normalsize}
\newcommand{\Comment}[1]{\relax}  % makes a "comment" (not expanded)
\newcommand{\Heading}[1]{\par\noindent{\bf#1}\nobreak}
\newcommand{\Tail}[1]{\nobreak\par\noindent{\bf#1}}
\newcommand{\QED}{\vrule height 1.4ex width 1.0ex depth -.1ex\ } % square box
\newcommand{\arc}[1]{\mbox{$\stackrel{\frown}{#1}$}}
\newcommand{\lyne}[1]{\mbox{$\stackrel{\leftrightarrow}{#1}$}}
\newcommand{\ray}[1]{\mbox{$\vec{#1}$}}
\newcommand{\seg}[1]{\mbox{$\overline{#1}$}}
\newcommand{\tab}{\hspace*{.2in}}
\newcommand{\se}{\mbox{$_{\epsilon}$}}  % subscript epsilon
\newcommand{\ie}{\mbox{i.e.}}
\newcommand{\eg}{\mbox{e.\ g.\ }}
\newcommand{\figg}[3]{\begin{figure}[htbp]\vspace{#3}\caption{#2}\label{#1}\end{figure}}
\newcommand{\be}{\begin{equation}}
\newcommand{\ee}{\end{equation}}
\newcommand{\prf}{\noindent{{\bf Proof} :\ }}

\newtheorem{rmk}{Remark}[section]
\newtheorem{example}{Example}[section]
\newtheorem{conjecture}{Conjecture}[section]
\newtheorem{claim}{Claim}[section]
\newtheorem{notation}{Notation}[section]
\newtheorem{lemma}{Lemma}[section]
\newtheorem{theorem}{Theorem}[section]
\newtheorem{corollary}{Corollary}[section]
\newtheorem{defn2}{Definition}

\ifFull
\SingleSpace
\else
\DoubleSpace
\fi

\setlength{\oddsidemargin}{0pt}
\setlength{\evensidemargin}{0pt}
\setlength{\headsep}{0in}
\setlength{\topmargin}{0pt}
\setlength{\textheight}{8.75in}
\setlength{\textwidth}{6.5in}

% \input{clock}
% \setclock

\title{Quadric control nets}
\author{John K. Johnstone}
%\date{Version: \today \clock}

\begin{document}

% HEADER ON EACH PAGE
\markright{\today \hfill}
%	% 	   \fbox{{\bf Johnstone, \today}} \hrulefill}
\setlength{\headsep}{.5in}
\pagestyle{myheadings}

\maketitle

% \title
% `Flexible models through constrained control nets' OR
% `Placing constraints on control nets to achieve tractable surfaces that also
% incorporate flexibility'.

\begin{abstract}
Goal: define the constraints on a control net for it to define a patch of a
	quadric surface (or some other well understood and tractable surface
	like the cyclide or torus or ruled).
	Then the surface will be guaranteed to be locally (i.e., within
	each of its patches) tractable (e.g., for analysis operations such as
	intersection).
	It will also be possible to relax an arbitrary control net onto
	an approximating quadric, or series of quadric patches, by relaxing
	onto a nearby grid that satisfies the constraints for a quadric surface.
	Finally, it will be possible to make flexible surfaces that maintain
	some rigidity (and possibly later some volume preserving properties)
	by moving one or a few of the points of the control net and 
	letting the rest of the control net readjust to maintain the quadric
	constraints.

{\bf Series of papers from this topic}

Part 1: `What is a quadric?  An answer from a CAGD point of view'
	or `A Geometric Characterization of Quadric Control Nets'
	(constraints that are necessary and hopefully sufficient for a quadric)

Part 2: `Relaxing an arbitrary control net onto a quadric'

Part 3: `Flexible surfaces: relaxing a quadric control net to a nearby quadric
	net after movement of a point'

Part 4: `Translation to and from a quadric control net and its implicit/parametric
	equation'
	(given the implicit or parametric equation of a quadric, we would like
	to be able to efficiently determine its control net; and vice versa,
	given a quadric control net, we would like to determine
	its equation for analysis)

Part 5: `Joining constrained patches of the same and different type'
	(how to connect up a quadric patch with another quadric patch, or
	with a constrained patch of another type)

	We would also want the underlying constraints to be flexible enough
	to change over the surface, so that on one patch the surface would relax
	to a quadric while on a neighbouring patch it relaxes to a cyclide.
	This latter property is akin to uniformly shaped beta-splines (same bias
	and tension over entire spline) and continuously shaped beta-splines
	(varying bias and tension over spline), except that the relationships
	would be constant over one patch, then a different constant over the
	next patch, possibly with a `blending' patch in between to smoothly
	join one set of relationships to the next set of relationships.
\end{abstract}

\clearpage

\section{Introduction}

A quadric surface can be represented in many different 
ways. Three popular representations are by an implicit surface, 
by a parametric surface, and by a control net. 
As implicit surfaces, quadrics have a complete algebraic characterization:
they are the surfaces defined by the zero set of a trivariate polynomial 
of degree 2; that is, $\{Z(f) | f = f(x,y,z) = ax^2 + \ldots + iz + j = 0\}$.
As parametric surfaces, they also have a complete algebraic characterization:
they are the surfaces with biquadratic parameterizations with two base points
(assuming a proper parameterization)
\cite{Seder90a,Seder90b,Seder85}.
[There is a difficulty in identifying surfaces by parameterizations, because
of reparameterization.]
Thus, looking at a surface's implicit equation or parametric equation, 
it is (easily) possible to determine whether or not the surface is a quadric.
However, as Bezier control nets, quadrics do not have a geometric 
characterization: that is, `when does a Bezier control net represent a quadric?'
What are the special internal relationships between control points that exist
when a control net represents a quadric?

This would be useful in various contexts.
\begin{itemize}
\item 
One would like to recognize a quadric when represented as a control net,
so that one can take advantage of its quadric-ness.
\item
A typical CAGD operation is to massage a surface
by moving some of its control points.
Suppose that you start with a quadric control net because computations on it
are very efficient (e.g., intersection and other analysis operations are 
much better understood on quadrics than arbitrary NURBS surfaces).
Upon massaging the surface by moving one of its control points,
we would like to preserve its quadric-ness.
This requires an understanding of the 
relationships (constraints) between control points in a quadric,
so that one can determine the necessary movements of other control points in order
to preserve a quadric surface.
\end{itemize}

We might also like to determine the constraints on control nets for
even more special classes, the seven subclasses of quadrics:
cylinder (elliptic, parabolic, and hyperbolic), ellipsoid, cone,
hyperbolic paraboloid, hyperboloid of one sheet, hyperboloid of two sheets,
and elliptic paraboloid.




% ***************************************************


\section{What type of control net?}

What type of control net is 
the most appropriate for the representation of quadrics?
The options are Bezier or NURBS, triangular or tensor product,
and the degree of the patch.
We start with Bezier control nets for simplicity,
and with triangular patches because of their more universal
usefulness.
We need rational patches because most quadrics are not polynomial.
Thus, we are considering rational triangular Bezier control nets
of degree $n$.
Is there a natural choice for the degree $n$?
We could allow variable degree.
Suppose that however, for simplicity, we consider patches of a fixed degree.
We clearly cannot represent quadric patches with arbitrary boundaries
using a control net of fixed degree.\footnote{A
	rational triangular Bezier patch of degree $n$ has degree $n$
	boundary curves.}
Nevertheless, using a control net of fixed degree $d$, we would hope that
we could represent any quadric patch with boundaries of degree $d$.
Unfortunately this is not true for degree 2 control nets:
any quadric {\em surface} can be expressed as a triangular rational quadratic
Bezier surface (also called a Steiner surface) \cite[p. 84]{Seder83},
but not every triangular {\em patch} of a quadric with conic boundaries
can be expressed as a
Steiner patch \cite[p. 347]{FAR93}.\footnote{A Steiner patch is a 
	triangular rational quadratic Bezier patch.
	A Steiner surface is the unbounded surface defined by this 
	parameterization.}
In order to be represented as a Steiner patch, the three boundary curves
of the triangular quadric patch must meet
at a point of the quadric, and the tangents at this common point must 
be coplanar \cite[p. 34]{Seder85}, \cite[p. 7]{Boehm91}.

In order to represent {\em all} triangular quadric patches,
one can use triangular rational {\em quartic} Bezier patches
\cite[p. 347]{FAR93}.
However, in general these will not be proper (or faithful) parameterizations
\cite[p. 34]{Sed87}.

Warren and Lodha's characterization is excellent: they can represent
any quadric patch that is the image of a triangle.
This is a natural restriction on the type of boundaries.

I feel that degenerate Steiner patches are
uninteresting as representation of quadric surface patches, 
because of their unnatural restriction on the boundary curves.
They are the best choice for entire quadric surfaces, however,
because of their low degree.

\section{What type of constraints?}

What type of constraints do we want to find?
They should be low level, telling us directly about the geometry of
control points, otherwise it will be impossible to make
local changes to correct local perturbations.
For example, consider some known constraints on rational quadratic
triangular Bezier patches that are not useful:
it is known that `a necessary condition [for such a patch to parametrize
a quadric] is that the planes containing the 3 boundary curves of
the patch must meet in a point lying on the quadric, [and that]
a sufficient condition is that the boundary curves, when extended,
also contain this point.' \cite{Teller91}
Therefore, a constraint could be to ensure that the extended boundary
curves all contain the point of intersection of the three planes
of the boundary curves.
But how could you maintain this constraint after a perturbation
of the control net?
Moving a control point would change one (or more, if it is a corner
point) of the boundary planes, and thus move the point of intersection
of the three boundary planes: we want to correct the 3 boundary curves
(with a small local movement) to go through this new point.
How?
One can see that the problem with the constraint is that it is too
indirect: it deals with high-level concepts such as boundary curves 
and `containing a point', rather than in the direct geometry of
the control net.

Therefore, messy and inelegant as it may be,
we actually prefer down-and-dirty low-level characterizations
that may not contain any geometric significance or elegance.

Two preliminary low-level approaches:
\begin{itemize}
\item
	{[implicitization]}
	Bezier control points $\rightarrow$
	rational parameterization $\rightarrow$
	implicitized polynomial;
	analyze the conditions that make the higher-level terms
	(cubic, quartic, ...) disappear and how these relate back
	to the control points.
\item
	{[base points]}
	Bezier control points $\rightarrow$
	rational parameterization;
	analyze the conditions that force base points and
	how these relate back to the control points.
\end{itemize}

\section{Applications of the constraints}

Suppose that we know the constraints for a control net to represent a quadric.
We must immediately ask how we will use this information to achieve flexibility.
Upon the motion of a control point, there are an infinite number of motions
of the other control points that will maintain the constraints.
Which motion should be chosen?
How can we avoid an infinite rippling effect, i.e., 
in order to satisfy the original patch we move one of its other control points,
which upsets another quadric patch, which needs to be corrected, ad infinitum?

If we discover the constraints and we discover how to mend a patch
upon disruption of those constraints, then we must ask where this
quadric-preserving collection of patches will be used.
Is our initial motivation of a flexible surface realistic?
For instance, how could it be used in the context of brain-atlas `morphing'?

% ***************************************************

\section{\fbox{Burning question and next step}}

You move one of the six control points of a rational quadratic
triangular Bezier patch of a quadric.
Where are the other control points pulled 
to preserve the quadric patch?

Eventually want a GL package showing the coordinated movement of
the six control points.

Let's show these bastards that they're wrong to write me off.
And let's have some fun doing it. Hockey is now my life, but I need
to do something to keep me entertained during the day.

Let's first consider rational quadric
parameterizations rather than Bezier reps of quadrics:
given a parameterization of a quadric (say $(u,v,uv)$ of a hyperbolic
paraboloid), how can it be perturbed while remaining quadric
(i.e., with same number of base points)?

\section{Previous knowledge on constraints}

There are some examples of how certain curves or surfaces
place structure on a Bezier net.

{\em Pythagorean-triple hodograph cubic curves}:
	their control polygons satisfy the relationship
	$L_2 = \sqrt{L_1L_3}$ and $\theta_1 = \theta_2$ 
	($L$ for length of an edge, $\theta$ for angle between edges).
	(Farouki/Sakkalis 1990).

{\em Cyclides}
	A rational biquadratic Bezier surface represents a generalized
	cyclide (a projective image of a Dupin cyclide) if and only if
	(Degen, Generalized cyclides for use in CAGD):
\begin{itemize}
\item
	the three planes spanned by the control points from a column
	(resp., row) belong to a pencil with the axis a (resp., b),
	and a and b are skew lines; and
\item	
	for each pair of triples from two columns (resp., rows),
	there exists a perspectivity with center on the axis b (resp., a)
	mapping these triples of control points onto each other.
\end{itemize}

	A sufficient condition for a Dupin cyclide net is given in 
	(Zhou and Strasser, A NURBS representation for cyclides, 1991):
	the NURBS control net for an entire cyclide (p. 87)
	and the knot vectors for defining subparts bounded by parametric lines
	are given.

	The control points and weights for a Bezier rational
	biquadratic patch of a cyclide with parametric-line boundaries
	is given in (Pratt, Cyclides in CAGD).

	A basic Bezier patch on a cyclide (from which other patches
	can be determined by de Casteljau) is given in 
	(Boehm, On cyclides in geometric modeling).

{\em Quadrics}

There are several papers on this topic: \cite{Dietz93,Boehm93,Hoschek-Seemann92,Boehm-Hansford92,Hoschek92,Wang92,Teller91,Boehm91,Lodha90,Dahmen89,Cobb88,Farin87,Piegl87,Piegl85,Seder85,Goldman83}.

\begin{enumerate}
\item
	`A necessary condition for a Steiner surface to be a quadric
	is that the planes on which the three boundary curves lie must
	intersect in a point on the surface -- namely, the center of projection.'
	\cite[p. 34]{Seder85}
\item
	(The same result stated another way.)
	`A rational triangular quadratic patch lies on a quadric if
	the three boundaries meet in one point, q, where their three tangents 
	are coplanar, and q corresponds three times to the parameter value
	$\infty$.' \cite[p. 7]{Boehm91}
\item
	Paraboloid patches are simpler to characterize, in terms of relationships
	on their control points \cite[p. 5]{Boehm91}:
	`[For an elliptic paraboloid, hyperbolic paraboloid, parabolic cylinder],
	the projections of the B-points along the axis into the tangent plane
	$\tau$ form a regular triangular net and a, b, c are relative distances 
	from $\tau$ in the direction of the axis....
	[For a parabolic cylinder], the B-net is projected into the figure
	of the de Casteljau algorithm.'
\item
	`In order to represent an arbitrary triangular patch of a quadric,
	rational quartic triangular patches are necessary.' \cite[p. 14]{Boehm91}
	That is, rational quadratic and biquadratic patches are not enough.
\item
	\cite{Lodha90}
	use a tetrahedron and a projective transformation of a functional
	representation to represent quadrics.
	Because the boundary curves of the quadric lie on the tetrahedral faces,
	the apex of the tetrahedron (which they call the focal vertex)
	must be a point of the quadric surface (by Sederberg's observation).
	When they connect up quadric patches, they do so by connecting up
	tetrahedra along mutual faces (which seems to be an artificial constraint)
	and thereby get into the restriction that neighbouring patches must
	share the same focal vertex, and thus only represent star-shaped objects.
\end{enumerate}

\section{Constraints imposed by quadrics}

Suppose that you have a Bezier patch that 
represents a quadric: what must it look like?
What properties make a quadric: 
	(1) any line intersects 
the surface in at most two real points.
Is this a sufficient condition?  
That is, are there nonalgebraic surfaces for which 
any line intersects them at most twice?
	(2) parallel circles (or lines)
	    circles sweeping through space along a line,
	    quadratic radius fn, same orientation.

	    doesn't apply to sphere or hyperbolic paraboloid

Quadrics have two base points: what property
forces two base points?

\section{Base points}

For a conic, there are practically no 
constraints on the control net:
A conic is generated by any Bezier curve with 
3 control points, in arbitrary position and with
arbitrary weights.
The simple structure of the Bezier 
control net of a conic is a result of 
the equivalence of conics and rational
quadratic parameterizations.
However, there is no such equivalence
for quadrics: quadrics form a strict
subset of rational quadratic triangular 
Bezier patches (Steiner patches \cite{Seder85}) [in
barycentric coordinates] and of
rational biquadratic (tensor product)
parameterizations [in Cartesian coordinates].
In particular, Steiner patches are generally
of degree 4 (=$n^2$) and biquadratic 
parameterizations are generally of degree 
8 (=$2mn$) \cite{Seder83}.  Specifically, they are of
degree $n^2 - b$ (resp., $2mn - b$?) where $b$	
is the number of base points (counted with	
appropriate multiplicity) on the entire parametric
surface, each of which reduces the implicit	
degree by one \cite{Warren90a}.\footnote{This result assumes
	that the parameterization is in barycentric
	coordinates: x(s,t,u),y(s,t,u),....
	This must be important, since it is simple to
	create examples where a Cartesian parameterization
	of a quadric does not have any w=0 components and thus
	no base points: the ellipsoid 
	$(a(1-u^2-v^2),2bu,2cv,1+u^2+v^2)$, for example		
	\cite[p. 217]{Gray93}.  Unless parameter values
	can be complex, which is true: a rational surface param should
	be viewed as a mapping $P^2 \rightarrow P^3$ 
	\cite[Section 3.2]{Manocha92}.}
Thus, quadrics are 
Steiner patches with two base points (or
biquadratic patches with six base points?).	

\cite{Seder90a} shows that the two base points of a sphere
with parameterization (x,y,z,w) = 
$(2s,1-s^2-t^2,2t,1+s^2+t^2)$ are (1,i,0) and
(i,1,0).  Note that these base points are both
complex and infinite.

\cite{Seder90a} also observes that although Steiner patches 
with one base point are cubic surfaces, they
cannot express general cubic surfaces, only ruled
ones.  This suggests that care be taken to ensure
that any final method gives {\em complete} coverage
of quadrics.

There are techniques that are sufficient
to enforce base points on a patch.
However, these are not necessary conditions
and therefore do not lead to a characterization
of quadric control nets.
For example, a Steiner patch has six control points.
Using a zero weight at a corner (interior?)
control point induces a base point \cite{Warren92}??
Thus, as a Steiner patch with two base points,
the quadric can be viewed as a Steiner patch
with two zero weights.
The most basic way to enforce base points is through
the method of undetermined coefficients, specifying
that the parametric equations evaluate to 0 at the
base points (and filling out the sufficient number
of linear equations, that is the number of coefficients
in the parameterization, by forcing the surface 
through certain other points).

The area of the Newton polygon of a patch
is related to its implicit degree \cite{Warren92}.
So in changing a quadric patch, we want to
retain the area of its Newton polygon, which
might be related to maintaining the same number
of zero weights.

Cyclides have base points, since they are quartic
although they have a biquadratic parameterization \cite{Zhou91}.
(Thus, cyclides lead to the examination
of surfaces with base points too.)

Given the control points of a patch (e.g.,
Steiner patch), is there any elegant way of 
determining how many base points it has
(i.e., the infinite surface that contains the patch)?
In other words, is there an elegant relationship
between the number of base points and the
weighted control polyhedron of a patch?
For a flexible quadric, this would be the 
invariant to maintain as the control points 
and weights change.

\section{Doodles}

What does the algebraic form of the triangular
Bezier surface mean, since it is in barycentric
coordinates?  For example, what does it mean
to evaluate x(s,t,u) (so that we can check if 
it is 0)?  
Answer: Just plug in some values of s,t,u=1-s-t,
like you plug in some values of s and t for
tensor product Bezier surfaces.

Note: a base point is a parameter value, not 
a point of the surface, of course (this
parameter value is associated with curves on
the surface, not a point).

Do the number of base points of a Bezier surface
increase as it is degree-elevated, or is
the number of base points calculated based
on the minimal degree of the surface?
If the former, then where are base points
introduced during degree elevation?

At a base point, all of x,y,z,w evaluate to 0.
So we are looking for common solutions of 
the equations x(s,t,u)=0 (in barycentric coords,
or x(s,t)=0 in Cartesian), y(s,t,u)=0,
z(s,t,u)=0 and w(s,t,u)=0.  
For a base point, we need w=0, which is the
denominator in the rational Bezier representation:
$\sum w_{ijk} B_{ijk}(u,v,w)$.
Since the weights $w_{ijk}$ are nonnegative, as are
the Bernstein polynomials over the interval of		
interest (NOT TRUE: IT IS INVALID TO RESTRICT TO THE INTERVAL OF 
INTEREST: THE BASE POINTS MAY BE OUTSIDE THE PATCH), 
each term $w_{ijk} B_{ijk}(u,v,w)$ must be zero.
The Bernstein polynomials are only zero on 
the sides of the triangle (u=0, v=0 or w=0).
Thus, base points only occur on the boundaries 
of the patch or when all weights are zero.
Indeed, $B_{ijk}$ with one of i,j,k zero (i.e., $B_{00n}$,
$B_{0n0}$ and $B_{n00}$) are only zero on one of the sides 
of the triangle.
Thus, unless all of the weights are zero,
the only way to simultaneously zero the
terms $w_{00n} B_{00n}$, $w_{0n0} B_{0n0}$, and $w_{n00} B_{n00}$
is for the base point to lie at one of the vertices
of the triangle (thus zeroing two of the Bernsteins)
and the weight associated with the other corner Bernstein
to be zero.
We conclude that {\em in order for a triangular Bezier patch
to have a base point, it must have at least one zero
weight; and if not all weights are zero, it must lie
at a corner of the patch}.
In particular, all Steiner patches (and in general
all triangular Bezier patches of degree 2 or higher)
that represent a quadric must have this form.

\section{Collected facts about base points}

\begin{itemize}
\item
(Existence) A parameterization has base points iff
its auxiliary resultant (resultant of its auxiliary
eqns) is identically zero \cite{Chionh92a,Chionh92c}.

\item
(Identification) It is claimed that Chionh's Ph.D. thesis contains
algorithms to identify base points.
\cite{Manocha92} contains an algorithm using u-resultants 
and factorization for computing base points,
and a faster but probabilistic algorithm that 
uses resultants and gcd.
\end{itemize}

\section{Bibliography for base points}

[1] T. Sederberg, Steiner patches. [Seder85]

[2] T. Sederberg, Ph.D. thesis. [Seder83]

[3] J. Warren, Creating multisided rational
	Bezier surfaces using base points. [Warren92]

[4] J. Warren, A bound on the implicit degree
	of polygonal Bezier surfaces.  [Warren90a]

[5] Zhou and Strasser, A NURBS representation of 
	cyclides.	[Zhou91]

[6] Chionh and Goldman, Using multivariate resultants
	to find the implicit equation of a rational
	surface. [Chionh92a]

[7] Chionh and Goldman, Implicitizing rational
	surfaces with base points by applying
	perturbations and the factors of zero theorem. [Chionh92c]

[8] Chionh, Ph.D. thesis, Base points, resultants,
	and the implicit representation of rational
	surfaces.	

[9] D. Manocha, Ph.D. thesis, Algebraic and numeric
	techniques in modeling and robotics.  [Manocha92]

[10] T. Sederberg, IEEE CG+A, July 1990,
	Techniques for cubic algebraic surfaces, part one. [Seder90a]

\section{Rational parameterizations of quadric surfaces}

We need to consider the parameterizations of quadrics,
both to analyze base points, and to consider translation
to and from parametric form and control net.

\begin{itemize}
\item
Sphere:	$(1-s^2-t^2,2s,2t,1+s^2+t^2)$
	[Manocha Ph.D. thesis, p. 51]
\item
Cylinder:
\item
Hyperbolic paraboloid: $(u,v,uv)$ 	[p. 211+350, Gray, 
			 Modern Diff Geom of Curves and Surfaces]
\item
Cone:
\item
Ellipsoid: $(a(1-u^2-v^2),2bu,2cv,1+u^2+v^2)$  [p. 217, Gray]
\item
Elliptic paraboloid: $(u,v,a(u^2+v^2))$	  [p. 221, Gray]
\item
Hyperboloid of one sheet:
\item
Hyperboloid of two sheets:
\end{itemize}

\bibliographystyle{plain}
\bibliography{/users/cogito/jj/bib/modeling}

\end{document}
