% --------------------------------------------------------------------
% Paper Submitted to COMPUTERS & GRAPHICS
%
% Paper History:
%    1)   Nov/22/91      First draft
%    2)   May/31/92      Second draft
%    3)   June/27/92     Final Version
%    4)   Oct/23/92      Submitted
%    5)   Nov/10/92      Acknowledged
%    6)   May/04/93      Provisionally accepted
%    7)   June/09/93     Final Revision
% --------------------------------------------------------------------

\documentstyle [12pt]{article}

\title{
     Computing the Intersection of a Plane \\
     and a Revolute Quadric}

\author{Ching-Kuang Shene\\
        Department of Mathematics and Computer Science \\
        Northern Michigan University \\
        Marquette, MI 49855--5340, USA\\[0.5cm]
        John K. Johnstone \\
        Department of Computer Science\\
        The Johns Hopkins University \\
        Baltimore, MD 21218--2686, USA}

\date{\ }

\input{table.tex}

\newtheorem{example}{Example}[section]
\newtheorem{property}{Property}[section]
\newtheorem{definition}{Definition}[section]
\newtheorem{theorem}{Theorem}[section]
\newtheorem{lemma}{Lemma}[section]
\newtheorem{corollary}{Corollary}[section]
\newtheorem{remark}{Remark}[section]

\newcommand{\DoubleSpace}{\edef\baselinestretch{1.4}\Large\normalsize}
\newcommand{\QED}{\ \ \ \rule{2mm}{3mm}\\}
\newcommand{\arrow}[1]{\vec{\bf #1}}

%\DoubleSpace
\setlength{\oddsidemargin}{0pt}
\setlength{\evensidemargin}{0pt}
\setlength{\headsep}{0pt}
\setlength{\topmargin}{0pt}
\setlength{\textheight}{8.75in}
\setlength{\textwidth}{6.5in}


\begin{document}
\maketitle

% --------------------------------------------------------------------
%                          Introduction
% --------------------------------------------------------------------

\section{Introduction}
\label{section:intro}

     Natural quadrics (the cylinder, cone, and sphere)
have been implemented in many solid modeling systems for
a long time.  However, the cylinder and cone have non-positive
curvature and have certain limitations in constructing more complicated solids.
In terms of simplicity and expressive power, the revolute quadrics, which
consist of revolute (prolate or oblate) ellipsoids, revolute hyperboloids
(of one or two sheets) and revolute elliptic paraboloids, are natural
extensions since they are generated by revolving a conic, the simplest curve
beyond the linear world.  Although revolute quadrics had been one of the major
topics in descriptive
geometry~\cite{bereis:1964,schmid:1912,slaby:1966,strubecker:1958}, there is a
surprising lack of literature on revolute quadrics from the CAGD point of view.
In the past decade, it seems that Goldman~\cite{goldman:1983a} is the only
paper dedicated to this kind of surface.

     Computing the intersection of a plane and a revolute quadric is a simple
and practical problem.  The planar section is a basic tool in descriptive
geometry, drafting, computer aided design (CAD) and geometric
modeling~\cite{adams:1988,ding-davis:1987,hosaka:1992,ryan:1991}.
Moreover, a sequence of parallel sections is
a common aid in the visualization of a surface.

     Many general procedures for computing the intersection of two surfaces are
available.  But these general procedures could be very inefficient when they
are applied to special surfaces.  Thus, for the intersection of two special
surfaces, exploring the hidden structures of the given surfaces and then using
these special structures to find the intersection is not only efficient
but robust.  Using a classic result, the Dandelin sphere, the present
authors~\cite{johnstone-shene:1991} and Miller and
Goldman~\cite{miller-goldman:1991a} propose a very simple
geometric method for computing the intersection of a plane and a cone.
However, the Dandelin sphere technique is not feasible for the revolute
quadric surfaces since, in some cases, spheres inscribed in a revolute quadric
and tangent to the given plane do not exist, or if they do exist, they are not
tangent at the foci of the intersection conic.  Therefore, we must find a
way to bypass the use of Dandelin spheres, yet keep the simple
geometric concept intact.

     The main achievement of this paper is a set of simple closed forms for
the intersection of a plane and a revolute quadric surface.  Since a plane and
a revolute quadric intersect in a conic, by closed forms, we mean a set of
formul\ae\ giving the center (or the vertex), the axis lengths and axis
directions of the intersection conic.  We will show how to find a local
coordinate system to simplify computation and a technique of reducing our
problem to a plane so that all computations can be performed easily.

     In the following section, representations of revolute quadrics and
conic sections are presented.  Section~\ref{section:preliminaries} collects
some useful results from coordinate geometry.
In Section~\ref{section:coordinate-system}, we will discuss how to construct
a local coordinate system to simplify our computation.  A general computation
procedure is also presented at the end of this section.
Section~\ref{section:hy1} begins our computation with the
revolute hyperboloid of one sheet.  We choose this surface as our first step
simply because it is the most complicated one.  The computation for
other revolute quadric surfaces is similar.  Section~\ref{section:hy2},
Section~\ref{section:ellipsoid} and Section~\ref{section:elliptic-paraboloid}
cover the hyperboloid of two sheets, the ellipsoid and the elliptic
paraboloid respectively.  Finally, Section~\ref{section:conclusion} gives our
conclusion.

% --------------------------------------------------------------------
%                          Representations
% --------------------------------------------------------------------

\section{Surface Representation}
\label{section:representation}

     There are three popular representations of quadric surfaces: the implicit
form ($f(x,y,z)=0$), the parametric form ($x=x(s,t),y=y(s,t),z=z(s,t)$),
and the geometric form (see Definition~\ref{defn:geom} below).
It is difficult to extract the geometry ({\em e.g.}, axis, center, axis length)
of a surface from the implicit form,
although some operations such as point classification can
be done easily.  Furthermore, many computations require complicated coordinate
transformations that make the final result unreliable.  The parametric form
suffers similar problems.  Goldman~\cite{goldman:1983b} points out that from
an algebraic model ({\em i.e.}, an implicit or parametric form) it is
impossible to ascertain with any certainty the type of surface being modeled.
Wilson's experiments reported in \cite{wilson:1987} justify this observation.
Therefore, we will choose a geometric form
to represent revolute quadric surfaces.

     The lack of literature on revolute quadrics makes comparisons
difficult.  However, from Goldman's and Wilson's work, one can expect
that methods
based on geometric representations like the one presented in this paper would
be more reliable than those using transformations and algebraic manipulations.

     A surface of revolution is generated by revolving a curve, the
{\em meridian curve}, about a line,
the {\em axis of revolution}~\cite{rogers-adams:1990}.
For revolute quadrics, the meridian curve is a conic and the axis of revolution
is an axis of the conic.
There are five revolute quadric
surfaces: the prolate ellipsoid, oblate ellipsoid, hyperboloid of two sheets,
hyperboloid of one sheet, and elliptic paraboloid (Figure~\ref{fig:revQ},
respectively).
In fact, there are three more revolute quadric surfaces, the sphere, the
cylinder and the cone.  However, since the plane section of these surfaces has
been fully discussed in Johnstone and Shene~\cite{johnstone-shene:1991}, we
will not repeat it here.  Note that the technique presented in this paper also
works for these simpler surfaces.  In the rest of this paper, for simplicity,
the word ``revolute'' will usually be omitted.
\begin{figure}
\vspace{16.5cm}
\caption{The Five Revolute Quadric Surfaces}
\label{fig:revQ}
\end{figure}

\begin{remark} \rm
     In the remainder of this paper, two types of vectors will be used.
A {\em position vector} specifies a point in space such as the center of
a conic, while a {\em direction vector} gives only a direction such as the
axis direction and hence is positionless.  All position vectors,
also referred to as {\em points},  will be
denoted by upper case like $\arrow{B}$, while direction vectors will be
denoted by
lower case like $\arrow{u}$.  Note also that all direction vectors are
assumed to be of unit length. $\Box$
\end{remark}

\begin{definition}
\label{defn:geom}
     In general, four parameters are required to uniquely determine a revolute
quadric surface.  If the meridian conic is central ({\em i.e.}, an ellipse or a
hyperbola), these parameters are the center ($\arrow{V}$),
the semi-axis lengths of the meridian conic ($a$ and $b$ for major and minor
axis, respectively),
and the direction of the axis of revolution ($\arrow{\ell}$).
If the meridian conic is a parabola, three parameters are enough: the vertex
($\arrow{V}$), the focal length ($f$, from vertex to focus),
and the direction of the axis of revolution ($\arrow{\ell}$, where we
assume $\arrow{\ell}$ points towards the opening of the
paraboloid).\footnote{In fact, two parameters (the vertex and focus) would be
enough since the direction of the axis of revolution
is given by the normalized vector $\stackrel{\longrightarrow}{VF}$, where
$V$ and $F$ are the vertex and the focus respectively.  However, we prefer the
three-parameter specification since it makes the focal length explicit.}
The above representation of a revolute quadric by the center, axis of
revolution, and axis lengths of its meridian conic is called the
{\bf geometric form} of the quadric.
\end{definition}

     A prolate ellipsoid ${\cal Q}_{EP}(\arrow{V},\arrow{\ell},a,b)$
({\em resp.}, oblate ellipsoid ${\cal Q}_{EO}(\arrow{V},\arrow{\ell},a,b)$)
is generated by revolving an ellipse about its major ({\em resp.}, minor) axis.
A hyperboloid of two sheets ${\cal Q}_{H2}(\arrow{V},\arrow{\ell},a,b)$
({\em resp.}, hyperboloid of one sheet
${\cal Q}_{H1}(\arrow{V},\arrow{\ell},a,b)$)
is generated by revolving
a hyperbola about its major ({\em resp.}, minor) axis.
Finally, an elliptic paraboloid ${\cal Q}_P(\arrow{V},\arrow{\ell},f)$
is generated by revolving a parabola about its (only) axis.

Notice that $\arrow{V}+t\arrow{\ell}$ is the axis of revolution
and is always an axis of the meridian conic.
If a plane through $\arrow{V}+t\arrow{\ell}$ is taken to be the $xy$-plane,
with $\arrow{\ell}$ the positive $x$-axis,
and $\arrow{V}$ the origin, the meridian conic (and the revolute quadric
surface) is in normal form.
Table~\ref{tbl:forms} summarizes these normal forms.
\begin{table}
\caption{Normal Forms of Meridian Conics and Revolute Quadrics}
\label{tbl:forms}
$$
\BeginTable
     \OpenUp22
     \BeginFormat
          |5 c m         | c m                | c m                  |5
     \EndFormat
     \_5
     | \JustCenter \it Notation | \JustCenter \it Meridian Conic |
          \JustCenter \it Revolute Quadric | \\ \_3
     | {\cal Q}_{EP}(\arrow{V},\arrow{\ell},a,b) |
          \frac{x^2}{a^2}+\frac{y^2}{b^2}=1, a>b |
          \frac{x^2}{a^2}+\frac{y^2}{b^2}+\frac{z^2}{b^2}=1, a>b | \\ \_
     | {\cal Q}_{EO}(\arrow{V},\arrow{\ell},a,b) |
          \frac{x^2}{b^2}+\frac{y^2}{a^2}=1, a>b |
          \frac{x^2}{b^2}+\frac{y^2}{a^2}+\frac{z^2}{a^2}=1, a>b | \\ \_
     | {\cal Q}_{H2}(\arrow{V},\arrow{\ell},a,b) |
          \frac{x^2}{a^2}-\frac{y^2}{b^2}=1      |
          \frac{x^2}{a^2}-\frac{y^2}{b^2}-\frac{z^2}{b^2}=1      | \\ \_
     | {\cal Q}_{H1}(\arrow{V},\arrow{\ell},a,b) |
          -\frac{x^2}{b^2}+\frac{y^2}{a^2}=1     |
          -\frac{x^2}{b^2}+\frac{y^2}{a^2}+\frac{z^2}{a^2}=1     | \\ \_
     | {\cal Q}_{P}(\arrow{V},\arrow{\ell},f)    |
          y^2=4fx                                |
          y^2+z^2=4fx                            | \\ \_5
\EndTable
$$
\end{table}

     A plane is fully determined if we know a base point $\arrow{B}$ on it and
a vector $\arrow{o}$ giving the normal direction.  Therefore, we will use
${\cal P}(\arrow{B},\arrow{o})$ to denote the plane with base point
$\arrow{B}$ and normal vector $\arrow{o}$.

The intersection of a plane and a revolute quadric is a conic, and
we also choose a geometric form to represent this conic.
The {\em geometric form} of a central conic is
the center $\arrow{C}$, the major axis and minor axis
directions $\arrow{u}$ and $\arrow{v}$, and the semi-major and semi-minor
axis lengths $u$ and $v$.  Thus an ellipse has equation
$\arrow{C}+(u\cos\!\theta\arrow{u}+ v\sin\!\theta\arrow{v})$, or
$\arrow{C}+\left(u\frac{1-\lambda^2}{1+\lambda^2}\arrow{u}+
v\frac{2\lambda}{1+\lambda^2}\arrow{v}\right)$
if rational parameterization is required, where $\lambda$ is a parameter.
A hyperbola has equation
$\arrow{C}+(u\cosh\!\theta\arrow{u}+v\sinh\!\theta\arrow{v})$, or
$\arrow{C}+\left(u\frac{\lambda^2+1}{2\lambda}\arrow{u}+
v\frac{\lambda^2-1}{2\lambda}\arrow{v}\right)$.
For a parabola, the geometric form consists of its vertex
$\arrow{V}$, a direction vector $\arrow{u}$ pointing toward the opening of the
parabola, a direction vector $\arrow{v}$ giving the direction of the directrix,
and a focal length $f$.  Hence, the equation of the parabola is
$\arrow{V}+\left(\frac{\lambda^2}{4f}\arrow{u}+\lambda\arrow{v}\right)$,
where $\lambda$ is a parameter.

% --------------------------------------------------------------------
%                           Preliminaries
% --------------------------------------------------------------------

\section{Preliminaries}
\label{section:preliminaries}

     In this section, we will collect some useful results to be used in the
following sections.  We will introduce the concept of similar or homothetic
conics, a method to compute the straight lines on a hyperboloid of one sheet,
to compute the intersection lines when the plane is tangent
to a hyperboloid of one sheet,
and a theorem about the focal length of the intersection parabola of a plane
and a circular cone.

     Two conics in space are {\em similar}, or {\em homothetic}, if one
of the conics can be transformed to the other by translation and scaling all
coordinates by the same non-zero factor.  That is, if $C_1$ and $C_2$ are
two conics, they are similar if and only if there exists a linear
transformation ${\cal T}$ defined as follows:
\[ {\cal T} : \left[ \begin{array}{c}
                    x^\prime \\ y^\prime \\ z^\prime
              \end{array} \right] =
       \left[ \begin{array}{ccc}
                    \rho &   0   &   0    \\
                      0  & \rho  &   0    \\
                      0  &   0   & \rho
              \end{array} \right] \cdot
       \left[ \begin{array}{c}
                    x \\ y \\ z
              \end{array} \right] +
       \left[ \begin{array}{c}
                    a \\ b \\ c
              \end{array} \right],\ \ \ \ \rho\neq 0  \]
such that $C_2={\cal T}(C_1)$.  The following is a very important result
from quadric surface theory.
For a proof, see Snyder and Sisam~\cite{snyder-sisam:1914}.

\begin{lemma}
\label{lemma:similar}
     All parallel sections of a quadric surface are similar.  The centers, if
they exist, of these parallel sections are collinear.
\end{lemma}

     The hyperboloid of one sheet is a ruled surface, that is, it can be
generated by sweeping a line.  These lines on the surface are called
{\em rulings}.  Furthermore, the hyperboloid of one sheet is doubly ruled.
In other words, for each point on the surface there passes exactly two rulings.
If the equation of a revolute hyperboloid of one sheet
is $-\frac{x^2}{b^2}+\frac{y^2}{a^2}+\frac{z^2}{a^2}=1$, the circle on the
$yz$-plane can be parameterized by $(0,a\cos\theta,a\sin\theta)$ and the two
rulings meeting at this point are $(0,a\cos\theta,a\sin\theta)+
t\left(1,\mp\frac{a}{b}\sin\theta,\pm\frac{a}{b}\cos\theta\right)$
(Snyder and Sisam~\cite{snyder-sisam:1914}).  Therefore, we have the following
lemma.

\begin{lemma}
\label{lemma:rulings}
     If a hyperboloid of one sheet has equation $-\frac{x^2}{b^2}+
\frac{y^2}{a^2}+\frac{z^2}{a^2}=1$, then the rulings through any point
on the circle $(0,a\cos\!\theta,a\sin\!\theta)$ are
\[ (0,a\cos\theta,a\sin\theta)+
     t\left(1,\mp\frac{a}{b}\sin\theta,\pm\frac{a}{b}\cos\theta\right). \]
\end{lemma}

     The plane spanned by the rulings through any point on a hyperboloid of
one sheet is the tangent plane of the surface.  In fact, any tangent
plane of a hyperboloid of one sheet intersects the surface in a pair of
intersecting lines meeting at the tangent point.  Determining this pair
of tangent lines is not difficult.  Since all parallel planes intersect the
hyperboloid of one sheet in similar hyperbolas, their asymptotes form two
intersecting planes with the intersection line  containing the centers
(Lemma~\ref{lemma:similar}).
Hence, the directions of the asymptotes of the intersection hyperbola are
independent of the position of the plane.  Therefore, if we know a plane is
tangent to a hyperboloid of one sheet at a point $P$, the intersection lines
can be computed by the following three steps:
(1) compute a parallel section (which is a hyperbola),
(2) determine the asymptotes of this intersection hyperbola, and
(3) the intersection lines are the lines through $P$ with the same direction
vectors as the asymptotes.
We will use this property in the proof of Theorem~\ref{thm:hy-1}.
For a geometric treatment of this property, see
Drew~\cite[page 119]{drew:1875}.

\begin{lemma}
\label{lemma:asymptotes}
     If a plane is tangent to a hyperboloid of one sheet at a point $P$, the
intersection is a pair of intersecting lines meeting at $P$. The line equations
can be determined by computing the asymptotes of the intersection hyperbola of
a parallel section and using the asymptotes' directions as the directions of
the intersection lines.
\end{lemma}

     Consider an elliptic paraboloid.  It is not difficult to see that a plane
intersects the surface in a parabola if and only if this plane is parallel to
the axis of revolution.  The intersection parabola is similar to the surface's
meridian parabola and thus has the same focal length.

\begin{lemma}
\label{lemma:parabola-focal-length}
     If a plane intersects an elliptic paraboloid in a parabola, then
the intersection parabola is congruent\footnote{Two objects are
{\em congruent} if, and only if one object can be transformed by rigid motions
(translations and rotations) to the other.} to the meridian and hence its
focal length is the same as the meridian parabola.
\end{lemma}

     The following lemma will be used to compute the focal length of an
intersection parabola.  For a proof, see Johnstone and
Shene~\cite{johnstone-shene:1991}.

\begin{lemma}
\label{lemma:cone-para-focal-length}
     Suppose plane ${\cal P}$ intersects a circular cone ${\cal C}$ in a
parabola.  Let the distance from the cone's vertex to the intersection point
of ${\cal P}$ and the cone's axis be $d$.  Then the focal length of the
parabola is $f=\frac{d}{2}\frac{\sin^2\alpha}{\cos\alpha}$, where $\alpha$ is
the cone angle.
\end{lemma}

% --------------------------------------------------------------------
%            Choose an appropriate coordinate system
% --------------------------------------------------------------------

\section{Choosing a Coordinate System}
\label{section:coordinate-system}

     A properly chosen coordinate system can greatly simplify our computation.
In this section,  we will present such a system for our subsequent
calculations.

     Suppose a plane ${\cal P}$ with base point $\arrow{B}$ and normal vector
$\arrow{o}$, and a revolute quadric ${\cal Q}$ with axis
$\arrow{V}+t\arrow{\ell}$ are given, where $\arrow{V}$ is the center or the
vertex of the meridian conic.
Based on the relation of $\arrow{o}$ and $\arrow{\ell}$, we have three cases:
(1) $\arrow{o}\ ||\ \arrow{\ell}$, (2) $\arrow{o}\perp\arrow{\ell}$, and
(3) otherwise.  For the first case, ${\cal P}$ is perpendicular to the
axis of revolution.  The intersection is either empty, a point, or a circle.
For the second case, ${\cal P}$ is parallel to the axis of revolution.  The
intersection is either empty, a point, two intersecting lines, or a conic
similar to the meridian conic.  We will discuss these results in later
sections, since they are easy to compute.  In the rest of this section,
we will assume the
third case, the only case in which an appropriately chosen coordinate system is
important.

     The construction of the coordinate system consists of three steps.
The axis of revolution is taken to be the $x$-axis.  The $z$-axis is the
line through $\arrow{V}$ and perpendicular to both $\arrow{\ell}$ and
$\arrow{o}$.  The $y$-axis is determined with the right-hand rule.
This system is called the {\em global coordinate system}.
This construction makes the given plane ${\cal P}$ parallel to the $z$-axis
and perpendicular to the $xy$-plane.  Now since the $xy$-plane cuts
${\cal Q}$ into two symmetric parts and is perpendicular to ${\cal P}$,
${\cal P}\cap{\cal Q}$ is also symmetric about the $xy$-plane and therefore,
the intersection line of ${\cal P}$ and the $xy$-plane
($\arrow{D}+t\arrow{m}$ below) is either the major or the minor axis of
${\cal P}\cap{\cal Q}$.

     Define
\begin{equation}
\label{eqn:vector-a}
      \arrow{a} = \left\{ \begin{array}{ccc}
                    \arrow{o} & & \mbox{if $\arrow{\ell}\cdot\arrow{o}>0$}\\
                              & & \\
                   -\arrow{o} & & \mbox{otherwise}
                    \end{array} \right.
\end{equation}
\begin{equation}
\label{eqn:vector-n}
\arrow{n}=\frac{\arrow{\ell}\times\arrow{a}}{|\arrow{\ell}\times\arrow{a}|}
\end{equation}
\begin{equation}
\label{eqn:vector-k}
\arrow{k}=\frac{\arrow{n}\times\arrow{\ell}}{|\arrow{n}\times\arrow{\ell}|}.
\end{equation}
The definition of $\arrow{a}$ simply reverses the direction of $\arrow{o}$,
if necessary, so that the angle between $\arrow{\ell}$ and $\arrow{a}$ is less
than 90 degrees.
If $\arrow{\ell}$, $\arrow{k}$ and $\arrow{n}$ are taken to be the $x$-, $y$-
and $z$-axis respectively, then they form a right-handed coordinate system
with $\arrow{a}$ lying in the quadrant between the positive $x$ and the
positive $y$ axes.  That is, the slope of $\arrow{a}$ in the new coordinate
system is positive.  This system is called the {\em local coordinate system}
(Figure~\ref{fig:vectors-a-n-k}).
Note that since $\arrow{a}\perp\arrow{n}$, $\arrow{n}$ is
parallel to ${\cal P}$.
\begin{figure}
\vspace{5.5cm}
\caption{Vectors $\arrow{\ell}, \arrow{k}$ and $\arrow{n}$}
\label{fig:vectors-a-n-k}
\end{figure}

     Consider the vector $\arrow{m}$ defined as follows:
\begin{equation}
\label{eqn:vector-m}
\arrow{m}=\frac{\arrow{n}\times\arrow{a}}{|\arrow{n}\times\arrow{a}|}.
\end{equation}
Since $\arrow{m}\perp\arrow{a}$, $\arrow{m}$ is parallel to ${\cal P}$.
Let $\arrow{D}=\arrow{V}+d\arrow{\ell}$ be the point of
intersection of the axis of revolution and
${\cal P}$.  The intersection line of ${\cal P}$ and the plane through
$\arrow{V}$ and spanned by $\arrow{\ell}$ and $\arrow{k}$ is simply
$\arrow{D}+t\arrow{m}$, where $t$ is the line parameter.

     The slope and  equation of $\arrow{D}+t\arrow{m}$ in the local
coordinate system can be determined easily.
Since $\arrow{\ell}\cdot\arrow{a}=\cos\theta>0$, where $\theta$ is the acute
angle between $\arrow{\ell}$ and $\arrow{a}$, $\arrow{a}$ has slope
$\frac{\sin\theta}{\cos\theta}=
\sqrt{1-(\arrow{\ell}\cdot\arrow{a})^2}/(\arrow{\ell}\cdot\arrow{a})$.
Since $\arrow{m}\perp\arrow{a}$, the slope of $\arrow{D}+t\arrow{m}$, $s$,
must be
\begin{equation}
\label{eqn:slope-s}
     s=-\frac{\arrow{\ell}\cdot\arrow{a}}
           {\sqrt{1-(\arrow{\ell}\cdot\arrow{a})^2}} < 0.
\end{equation}
Since $\arrow{D}+t\arrow{m}$ passes through $(d,0,0)$ of the local coordinate
system, its equation is
\begin{equation}
\label{eqn:d+tm}
     y = s(x-d).
\end{equation}
Therefore, we have the following lemma.

\begin{lemma}
\label{lemma:vectors}
     Suppose ${\cal P}(\arrow{B},\arrow{o})$ and ${\cal Q}$ are a plane and
a revolute quadric with axis $\arrow{V}+t\arrow{\ell}$, where $\arrow{V}$ is
the center or the vertex of the meridian conic, and $\arrow{o}$ and
$\arrow{\ell}$ are neither perpendicular nor parallel to each other.
Let $\arrow{D}=\arrow{V}+d\arrow{\ell}$ be the intersection point
of $\arrow{V}+t\arrow{\ell}$ and ${\cal P}$.  In the local coordinate system,
where $\arrow{V},\arrow{\ell},\arrow{k}$ and $\arrow{n}$ are the coordinate
origin and the positive directions of the $x$-, $y$- and $z$-axis respectively,
the slope of $\arrow{a}$ and $\arrow{D}+t\arrow{m}$, the intersection line of
${\cal P}$ and the $xy$-plane, are
\[
\frac{\sqrt{1-(\arrow{\ell}\cdot\arrow{a})^2}}{\arrow{\ell}\cdot\arrow{a}}>0
\ \ \ \ \mbox{and}\ \ \ \
s=-\frac{\arrow{\ell}\cdot\arrow{a}}{\sqrt{1-(\arrow{\ell}\cdot\arrow{a})^2}}<0
\]
respectively, and the equation of $\arrow{D}+t\arrow{m}$ is $y=s(x-d)$.
\end{lemma}

     In the following, we will call the $xy$-plane the {\em axial plane} and
the meridian conic that lies on the $xy$-plane the {\em profile conic}
respectively.   In the local coordinate system, if it exists, the
{\em squared height} at any point of the axial plane is the square
of the $z$ coordinate to the surface.  Note that if the line through a point
and perpendicular to the $xy$-plane intersects the surface in real points,
the squared height at this point is positive;
otherwise, the $z$ coordinate at this point is imaginary and hence
its squared height is negative.
Recall that the line $\arrow{D}+t\arrow{m}$ is either the major
or minor axis of the intersection conic.
The other axis (if it exists) is consequently the line perpendicular to the
axial plane and through the center or vertex of the intersection conic.

     Once the local coordinate system is set up, the computation of the
intersection curve is straightforward.  The procedure consists of the
following five steps:
\begin{enumerate}
     \item Compute the intersection of the profile conic and the line
          $\arrow{D}+t\arrow{m}$.  If we have two distinct real or conjugate
          complex roots, $\arrow{D}+t\arrow{m}$ is one of the axes of the
          intersection conic.  If we have a double root, the given plane
          is tangent to the surface.  If there is only one real root (the other
          is at infinity), the intersection conic is a parabola.
       (The latter case is not handled by the following procedure.
       We will handle this case separately.)
     \item Let the two intersection points be $\arrow{P}_1$ and $\arrow{P}_2$,
          and let $\arrow{C} = \frac{1}{2}(\arrow{P}_1 + \arrow{P}_2)$,
          the midpoint.\footnote{Although the intersection points may be
          complex, their midpoint is always real.}
          $\arrow{C}$ is the center of the intersection conic.
     \item Compute the squared height at the center.  This value is called the
          squared semi-axis length along $\arrow{n}$.
     \item Let $m^2 = \frac{1}{4} |\arrow{P}_2 - \arrow{P}_1|^2$.
          This is called the squared semi-axis length along $\arrow{m}$.
     \item Finally, if the squared semi-axis lengths are both positive, the
          square root of the larger ({\em resp.}, smaller) is the semi-major
          ({\em resp.}, semi-minor) axis length.  The intersection, in this case
          is an ellipse.  If one of the squared semi-axis lengths is
          negative,\footnote{It is possible for a squared semi-axis length
          to be negative, because the two intersection points could be complex.
       For example, see the proof of Theorem~\ref{thm:hy-1}.}
       but not both, the axis ($\arrow{m}$ or $\arrow{n}$) corresponding to the
          negative length is the minor axis.  The square roots of the absolute
          values give the semi-axis lengths.
\end{enumerate}
Squared lengths are used in order to avoid complex number operations.
For a central conic in normal form $\frac{x^2}{A}+\frac{y^2}{B}=1$,
by letting $x=0$, we have the squared height at the center, $y^2=B$.
Since the normal form has semi-axis lengths
$\sqrt{|A|}$ and $\sqrt{|B|}$,
$A>0$ and $B>0$ ({\em resp.}, $A\cdot B<0$) indicates an ellipse
({\em resp.}, a hyperbola).


% --------------------------------------------------------------------
%                      Hyperboloid of One Sheet
% --------------------------------------------------------------------

\section{Hyperboloid of One Sheet}
\label{section:hy1}

     In this section, we will focus on the computation of the intersection
curve of a plane and a hyperboloid of one sheet.  We start our discussion with
the hyperboloid of one sheet because this surface has the most
complicated intersection types.  The intersection curve can be of any
conic type, as well as two intersecting or two parallel lines.
The following lemma considers the two simple cases:
(1) $\arrow{o}\ ||\ \arrow{\ell}$ and
(2) $\arrow{o}\perp\arrow{\ell}$.

\begin{lemma}
\label{lemma:h1-para&perp}
     Suppose ${\cal P}(\arrow{B},\arrow{o})$ and
${\cal Q}_{H1}(\arrow{V},\arrow{\ell},a,b)$ are a plane and a hyperboloid of
one sheet.  Then we have:
\begin{enumerate}
     \item $\arrow{o}\ ||\ \arrow{\ell}$:
          Let $\arrow{D}=\arrow{V}+d\arrow{\ell}$ be the intersection
          point of ${\cal P}$ and the axis of revolution.
          The intersection
          is a circle in ${\cal P}$ with center $\arrow{D}$ and radius
          $\frac{a}{b}\sqrt{b^2+d^2}$.
     \item $\arrow{o}\perp\arrow{\ell}$: Let $\arrow{E}=\arrow{V}+e\arrow{o}$
       be the intersection point of
          plane ${\cal P}$ and the line $\arrow{V}+t\arrow{o}$.
          Then we have:
     \begin{enumerate}
          \item $|e|>a$: The intersection curve is a hyperbola with center
               $\arrow{E}$, major and minor axis directions $\arrow{\ell}$
       and $\frac{\arrow{\ell}\times\arrow{o}}{|\arrow{\ell}\times\arrow{o}|}$,
               and semi-major and semi-minor axis lengths
               $\frac{b}{a}\sqrt{e^2-a^2}$ and $\sqrt{e^2-a^2}$.
          \item $|e|=a$: The intersection curve is a pair of intersection lines
               with equations $\arrow{E}+t\left(b\arrow{\ell}\pm
           a\frac{\arrow{\ell}\times\arrow{o}}{|\arrow{\ell}\times\arrow{o}|}
               \right)$.
          \item $|e|<a$: The intersection curve is a hyperbola with center
               $\arrow{E}$, major and minor axis directions
         $\frac{\arrow{\ell}\times\arrow{o}}{|\arrow{\ell}\times\arrow{o}|}$
               and $\arrow{\ell}$, and semi-major and semi-minor axis
               lengths $\sqrt{a^2-e^2}$ and
               $\frac{b}{a}\sqrt{a^2-e^2}$.
     \end{enumerate}
\end{enumerate}
\end{lemma}
{\bf Proof:} Omitted. \QED

     We have another simple case: $\arrow{V}\in{\cal P}$ and $|s|=\frac{a}{b}$,
where $s$ is the slope of the intersection line of ${\cal P}$ and the axial
plane in the local coordinate system.  For this case, the intersection is
a pair of parallel lines.

\begin{lemma}
\label{lemma:hy1-equal-slope}
     Suppose ${\cal P}(\arrow{B},\arrow{o})$ and
${\cal Q}_{H1}(\arrow{V},\arrow{\ell},a,b)$ are a plane and a hyperboloid of
one sheet.  If $\arrow{V}\in{\cal P}$ and $|s|=\frac{a}{b}$, where $s$ is the
slope of the intersection of ${\cal P}$ and the axial plane in the local
coordinate system, then the intersection is a pair of parallel lines
$(\arrow{V}\pm a\arrow{n}-b\arrow{m})+t\arrow{m}$.
\end{lemma}
{\bf Proof:}  In the local coordinate system, the hyperboloid of one sheet
has equation $-\frac{x^2}{b^2}+\frac{y^2}{a^2}+\frac{z^2}{a^2}=1$.
The two rulings passing through the point
$(0,a\cos\theta,a\sin\theta)$ are $(0,a\cos\theta,a\sin\theta)+t
\left(1,\mp\frac{a}{b}\sin\theta,\pm\frac{a}{b}\cos\theta\right)$.
Thus, the rulings through
$(0,0,a)$ are $(0,0,a)+t\left(1,\mp\frac{a}{b},0\right)$.  Similarly, the
rulings through $(0,0,-a)$ are $(0,0,-a)+t\left(1,\mp\frac{a}{b},0\right)$.
Since $s=-\frac{a}{b}<0$, ${\cal P}$ contains
$(0,0,\pm a)+t\left(1,-\frac{a}{b},0\right)$.  Converting this equation to
the global coordinate system gives the desired result. \QED

     For the general case, our computation follows exactly the procedure
discussed at the end of Section~\ref{section:coordinate-system}.

\begin{theorem}
\label{thm:hy-1}
     Suppose ${\cal P}(\arrow{B},\arrow{o})$ and
${\cal Q}_{H1}(\arrow{V},\arrow{\ell},a,b)$ are a plane and a hyperboloid of
one sheet.  Suppose $\arrow{\ell}$ and $\arrow{o}$ are neither parallel nor
perpendicular to each other.  Let $\arrow{D}=\arrow{V}+d\arrow{\ell}$
be the intersection point of ${\cal P}$ and the axis of revolution.
Let $s$ be the slope of the line $\arrow{D}+t\arrow{m}$ in the local
coordinate system, and $\arrow{V}\not\in{\cal P}$ if
$|s|=\frac{a}{b}$.  Let $\Theta=s^2(b^2+d^2)-a^2$.  Then we have:
\begin{enumerate}
     \item $|s|<\frac{a}{b}$: The intersection is a hyperbola, possibly
          degenerating to a pair of intersecting lines, with center
          $\arrow{V}+\frac{db^2s^2}{b^2s^2-a^2}\arrow{\ell}+
          \frac{sda^2}{b^2s^2-a^2}\arrow{k}$.  The directions and lengths of
          the semi-major and semi-minor axes are computed as follows:
     \begin{enumerate}
          \item $\Theta<0$: The major and minor axis directions are
               $\arrow{n}$ and $\arrow{m}$, and their corresponding semi-axis
               lengths are $\frac{a}{\sqrt{|b^2s^2-a^2|}}\sqrt{|\Theta|}$ and
               $\frac{ab}{|b^2s^2-a^2|}\sqrt{(1+s^2)|\Theta|}$.
          \item \label{ref:enum-case} $\Theta=0$: The intersection is a pair of
               intersecting lines with equation
\[             \left(\arrow{V}-\frac{b^2}{d}\arrow{\ell}-
               \frac{a^2}{sd}\arrow{k}\right)+t\left(b\sqrt{1+s^2}\arrow{m}\pm
               \sqrt{|b^2s^2-a^2|}\arrow{n}\right). \]
          \item $\Theta>0$: The major and minor axis directions are
               $\arrow{m}$ and $\arrow{n}$, and their corresponding semi-axis
               lengths are $\frac{ab}{|b^2s^2-a^2|}\sqrt{(1+s^2)|\Theta|}$ and
               $\frac{a}{\sqrt{|b^2s^2-a^2|}}\sqrt{|\Theta|}$.
     \end{enumerate}
     \item $|s|=\frac{a}{b}$: The intersection is a parabola with vertex
          $\arrow{V}+\frac{d^2-b^2}{2d}\arrow{\ell}+
          \frac{a}{2db}(d^2+b^2)\arrow{k}$, focal length
          $f=\frac{|d|}{2}\frac{a^2}{b\sqrt{a^2+b^2}}$, directrix direction
          $\arrow{n}$, and major axis direction
          $\arrow{m}$ ({\em resp.}, $-\arrow{m}$) if $d<0$ ({\em resp.}, $d>0$).
     \item $|s|>\frac{a}{b}$: In this case, $\Theta>0$ and the intersection is
          an ellipse with major and minor axis directions $\arrow{m}$ and
          $\arrow{n}$, and semi-major and semi-minor axis lengths
          $\frac{ab}{b^2s^2-a^2}\sqrt{(1+s^2)|\Theta|}$ and
          $\frac{a}{\sqrt{b^2s^2-a^2}}\sqrt{|\Theta|}$.
\end{enumerate}
\end{theorem}
{\bf Proof:}  In the local coordinate system, ${\cal Q}_{H1}$ has equation
$-\frac{x^2}{b^2}+\frac{y^2}{a^2}+\frac{z^2}{a^2}=1$ and the profile hyperbola
on the $xy$-plane is $-\frac{x^2}{b^2}+\frac{y^2}{a^2}=1$.
The line $y=s(x-d)$ intersects this
profile hyperbola at two points, which are solutions of the following
quadratic equation obtained by plugging $y=s(x-d)$ into
the equation of the profile hyperbola:
\begin{equation}
\label{eqn:hy1-2degree}
     (b^2s^2-a^2)x^2-2db^2s^2x+b^2(d^2s^2-a^2)=0.
\end{equation}
Assuming $|s|\neq\frac{a}{b}$, the two roots are
\begin{equation}
\label{eqn:hy1-roots}
     x = \frac{db^2s^2\pm ab\sqrt{\Theta}}{b^2s^2-a^2}.
\end{equation}
The corresponding $y$-coordinates can be calculated with $y=s(x-d)$.  Thus,
the midpoint of these two intersection points is
\[ \left(\frac{db^2s^2}{b^2s^2-a^2},\frac{sda^2}{b^2s^2-a^2},0\right), \]
while the squared distance between them is
$4\frac{a^2b^2}{(b^2s^2-a^2)^2}(1+s^2)\Theta$.  Hence, the squared semi-axis
length along the $\arrow{m}$ direction is
\begin{equation}
\label{eqn:hy1-m-length}
     \frac{a^2b^2}{(b^2s^2-a^2)^2}(1+s^2)\Theta.
\end{equation}
The squared semi-axis length along the $\arrow{n}$ direction is the squared
height to the surface at the center, which is obtained by
plugging the coordinates of
the center computed above into the surface equation:
\begin{equation}
\label{eqn:hy1-n-length}
     \frac{a^2}{b^2s^2-a^2}\Theta.
\end{equation}

     Now if the squared semi-axis length along the $\arrow{m}$ direction is
positive ($\Theta>0$), we have two subcases:  $b^2s^2-a^2>0$ and
$b^2s^2-a^2<0$.  For the first case, the squared semi-axis length along the
$\arrow{n}$ direction is also positive and the intersection is an
ellipse (Figure~\ref{fig:hy1-intersections}(a)).  Moreover, it is not difficult
to verify that the $\arrow{m}$ semi-axis length is larger than the $\arrow{n}$
semi-axis length.  Therefore, $\arrow{m}$ is the major axis direction, while
$\arrow{n}$ gives the minor axis direction.
If $b^2s^2-a^2<0$, the squared semi-axis length
along the $\arrow{n}$ direction is negative.
The intersection curve is a hyperbola
(Figure~\ref{fig:hy1-intersections}(b)).
The axis with positive squared semi-axis length is the major axis,
while the other one is the minor axis.
\begin{figure}
\vspace{9.5cm}
\caption{Various Intersection Types of a Hyperboloid of One Sheet}
\label{fig:hy1-intersections}
\end{figure}

     If the squared semi-axis length along the $\arrow{m}$ direction is
negative ($\Theta<0$), the other axis must be positive; otherwise, we will
have an imaginary conic.  Thus, the squared semi-axis length along the
$\arrow{n}$ direction must be positive ($b^2s^2-a^2<0$), and
the intersection is a
hyperbola with $\arrow{n}$ and $\arrow{m}$ as its major and minor axis
direction respectively (Figure~\ref{fig:hy1-intersections}(c)).

     Let us examine the asymptotes of the intersection hyperbola in the plane
spanned by $\arrow{m}$ and $\arrow{n}$.
The semi-axis lengths along
these two directions are
\[ m=\frac{ab}{|b^2s^2-a^2|}\sqrt{(1+s^2)|\Theta|}\ \ \ \ \ \mbox{and}\ \ \ \ \
   n=\frac{a}{\sqrt{|b^2s^2-a^2|}}\sqrt{|\Theta|}. \]
The slopes of the asymptotes with origin the center, $\arrow{m}$
and $\arrow{n}$ the $x$- and $y$-axis are
\[   \frac{n}{m} = \pm \frac{1}{b}\sqrt{\frac{|b^2s^2-a^2|}{1+s^2}}. \]
Note that this slope is independent of $d$.  The asymptotes' equations are
\[ t\left(b\sqrt{1+s^2}\arrow{m}\pm\sqrt{|b^2s^2-a^2|}\arrow{n}\right). \]

     Now suppose $\Theta=0$.  The line $y=s(x-d)$ is tangent to the profile
hyperbola and  ${\cal P}$ is tangent to ${\cal Q}_{H1}$
(Figure~\ref{fig:hy1-intersections}(d)).  By plugging $\Theta=0$ into equation
(\ref{eqn:hy1-roots}) we have the tangent point
$\left(-\frac{b^2}{d},-\frac{a^2}{sd},0\right)$.  Converting to the global
coordinate system, it becomes $\arrow{V}-\frac{b^2}{d}\arrow{\ell}-
\frac{a^2}{sd}\arrow{k}$.  From Lemma~\ref{lemma:asymptotes} the two
intersection lines are

\[ \left(\arrow{V}-\frac{b^2}{d}\arrow{\ell}-\frac{a^2}{sd}\arrow{k}\right)
     +t\left(b\sqrt{1+s^2}\arrow{m}\pm\sqrt{|b^2s^2-a^2|}\arrow{n}\right). \]

     Finally, we have to deal with the $|s|=\frac{a}{b}$ case.  In this case,
the leading coefficient of equation (\ref{eqn:hy1-2degree}) is zero.  The
only intersection point of the line $y=s(x-d)$ and the profile hyperbola is
$\left(\frac{d^2-b^2}{2d},-\frac{a(d^2+b^2)}{2bd},0\right)$.  This point is the
vertex of the intersection parabola (Figure~\ref{fig:hy1-intersections}(e)).
Note that if $d<0$ the major axis direction of the parabola is $\arrow{m}$;
otherwise, it is $-\arrow{m}$.  The only remaining factor is the focal length
of this parabola.

     As the profile hyperbola is revolved to generate the hyperboloid
of one sheet, its asymptotes generate a cone, the {\em asymptote cone}.
Since $y=s(x-d)$ is parallel to one of the asymptotes, ${\cal P}$ intersects
the hyperboloid and the asymptote cone in two parabolas with the same major
axis.  Because the asymptote cone never intersects the hyperboloid, the two
intersection parabolas do not intersect either.  This implies that they
are similar; otherwise, they must intersect in exactly two
points.\footnote{This can be shown as follows.  Without loss of generality,
suppose one of the two parabolas has the origin as its vertex.  Their
equations are $y^2=4px$ and $y^2=4q(x+h)$.  Solving for $x$ gives
$(p-q)x=qh$.  Therefore, the equation has no solution if and only if $p=q$.
That is, they are similar.} Hence, the focal length of the parabola from the
asymptote cone is the focal length of the parabola from the hyperboloid.
By Lemma~\ref{lemma:cone-para-focal-length}, the focal length is
$f=\frac{|d|}{2}\frac{\sin^2\alpha}{\cos\alpha}$, where $\alpha$ is the cone
angle.  From the slopes of the asymptotes, we have $\tan\alpha=\frac{a}{b}$.
Therefore, after eliminating $\alpha$,
$f=\frac{|d|}{2}\frac{a^2}{b\sqrt{a^2+b^2}}$. \QED

% --------------------------------------------------------------------
%                   Hyperboloid of Two Sheets
% --------------------------------------------------------------------

\section{Hyperboloid of Two Sheets}
\label{section:hy2}

     The computation for the hyperboloid of two sheets is simpler than the
hyperboloid of one sheet since we do not have to worry about the intersecting
lines and parallel lines cases (case~\ref{ref:enum-case} of
Theorem~\ref{thm:hy-1} and Lemma~\ref{lemma:hy1-equal-slope}, respectively).
Moreover, the intersection hyperbola always has
$\arrow{m}$ as its major axis direction.  As in the last section, we start
with some simple cases.

\begin{lemma}
\label{lemma:hy2-vector-para&perp}
     Suppose ${\cal P}(\arrow{B},\arrow{o})$ and
${\cal Q}_{H2}(\arrow{V},\arrow{\ell},a,b)$ are a plane and a hyperboloid of
two sheets.
\begin{enumerate}
     \item $\arrow{o}\ ||\ \arrow{\ell}$:
          Let $\arrow{D}=\arrow{V}+d\arrow{\ell}$ be the intersection point of
          ${\cal P}$ and the axis of revolution.  Then we have:
     \begin{enumerate}
          \item $|d|<a$: ${\cal P}$ and ${\cal Q}_{H2}$ are disjoint.
          \item $|d|=a$: ${\cal P}$ and ${\cal Q}_{H2}$ are tangent at
               $\arrow{D}$.
          \item $|d|>a$: ${\cal P}\cap{\cal Q}_{H2}$ is a circle in ${\cal P}$
               with center $\arrow{D}$ and radius $\frac{b}{a}\sqrt{a^2+d^2}$.
     \end{enumerate}
     \item $\arrow{o}\perp\arrow{\ell}$:
          Let $\arrow{E}=\arrow{V}+e\arrow{o}$ be the intersection
          point of ${\cal P}$ and $\arrow{V}+t\arrow{o}$.
          Then ${\cal P}\cap{\cal Q}_{H2}$ is a hyperbola with center
          $\arrow{E}$, major and minor axis directions $\arrow{\ell}$ and
          $\frac{\arrow{\ell}\times\arrow{o}}{|\arrow{\ell}\times\arrow{o}|}$,
          semi-major axis length $\frac{a}{b}\sqrt{b^2+e^2}$ and semi-minor
          axis length $\sqrt{b^2+e^2}$.
\end{enumerate}
\end{lemma}
{\bf Proof:} Omitted. \QED

\begin{theorem}
\label{thm:hy2-general}
     Suppose ${\cal P}(\arrow{B},\arrow{o})$ and
${\cal Q}_{H2}(\arrow{V},\arrow{\ell},a,b)$ are a plane and a hyperboloid of
two sheets. Suppose $\arrow{o}$ and $\arrow{\ell}$ are neither parallel nor
perpendicular to each other.  Let $\arrow{D}=\arrow{V}+d\arrow{\ell}$ be the
intersection point of ${\cal P}$ and the axis of revolution.
Let $s$ be the slope of the line $\arrow{D}+t\arrow{m}$ in the local coordinate
system.  Let $\Theta=b^2+(d^2-a^2)s^2$.  Then we have:
\begin{enumerate}
     \item $|s|\neq\frac{b}{a}$:
     \begin{enumerate}
          \item $\Theta<0$: ${\cal P}$ and ${\cal Q}_{H2}$ are disjoint.
          \item $\Theta=0$: ${\cal P}$ and ${\cal Q}_{H2}$ are tangent
               at a point $\arrow{V}-\frac{a^2}{d}\arrow{\ell}-\frac{b}{ds}
               \arrow{k}$.  Note that $\Theta=0$ and $|s|\neq\frac{b}{a}$
               imply $|d|\neq 0$.
          \item $\Theta>0$: ${\cal P}\cap{\cal Q}_{H2}$ is an ellipse
               ({\em resp.}, hyperbola) if $|s|>\frac{b}{a}$ ({\em resp.},
               $|s|<\frac{b}{a}$) with center
               $\arrow{V}-\frac{da^2s^2}{b^2-s^2a^2}\arrow{\ell}
                    -\frac{sdb^2}{b^2-s^2a^2}\arrow{m}$, major axis and minor
               axis directions $\arrow{m}$ and $\arrow{n}$, and semi-major
               and semi-minor axis lengths
               $\frac{ab}{|b^2-s^2a^2|}\sqrt{(1+s^2)\Theta}$ and
               $\frac{b}{\sqrt{|b^2-s^2a^2|}}\sqrt{\Theta}$.
     \end{enumerate}
     \item $|s|=\frac{b}{a}$:  The intersection is a parabola with vertex
          $\arrow{V}+\frac{a^2+d^2}{2d}\arrow{\ell}-
               \frac{b(a^2-d^2)}{2ad}\arrow{k}$, focal length
          $f=\frac{|d|}{2}\frac{b^2}{a\sqrt{a^2+b^2}}$, directrix direction
          $\arrow{n}$, and major axis direction $\arrow{m}$ ({\em resp.},
          $-\arrow{m}$) if $d<0$ ({\em resp.}, $d>0$).  If $d=0$, ${\cal P}$ and
          ${\cal Q}_{H2}$ do not intersect.\footnote{More precisely,
          ${\cal P}$ and ${\cal Q}_{H2}$ are tangent to each other at the point
          at infinity along the direction of one of the asymptotes.}
\end{enumerate}
\end{theorem}
{\bf Proof:}  The intersection point of the line $y=s(x-d)$ and the profile
hyperbola $\frac{x^2}{a^2}-\frac{y^2}{b^2}=1$ can be found by plugging the
former into the latter.  This gives the following quadratic equation:
\begin{equation}
\label{eqn:hy2-2degree}
     (b^2-a^2s^2)x^2+2da^2s^2x-a^2(b^2+d^2s^2)=0.
\end{equation}
Assuming $|s|\neq\frac{b}{a}$, the roots of this equation are
\[ x=\frac{-da^2s^2\pm ab\sqrt{\Theta}}{b^2-a^2s^2}. \]
If $\Theta<0$, $y=s(x-d)$ does not intersect the profile hyperbola and hence
${\cal P}$ and ${\cal Q}_{H2}$ are disjoint (Figure~\ref{fig:hy2}(a)).
Note that $\Theta<0$ implies $b^2-a^2s^2<0$ ({\em i.e.}, $|s|>\frac{b}{a}$).
\begin{figure}
\vspace{9.5cm}
\caption{Various Intersection Types of a Hyperboloid of Two Sheets}
\label{fig:hy2}
\end{figure}

     If $\Theta\geq 0$, the midpoint of the two intersection points is the
center of the intersection conic.  Thus, using the local coordinate system,
the center is
\[ \left(-\frac{da^2s^2}{b^2-a^2s^2},-\frac{sdb^2}{b^2-a^2s^2},0\right). \]
Converting it to with the global coordinate system, the center is
$\arrow{V}-\frac{da^2s^2}{b^2-a^2s^2}\arrow{\ell}
-\frac{sdb^2}{b^2-a^2s^2}\arrow{k}$.  The squared semi-axis length along
$\arrow{m}$ is a quarter of the squared distance between the two
intersection points, and is equal to
$\frac{a^2b^2}{(b^2-a^2s^2)^2}(1+s^2)\Theta$.  The squared semi-axis length
along $\arrow{n}$ is the squared height from the center to the surface.
Since the surface has equation
$\frac{x^2}{a^2}-\frac{y^2}{b^2}-\frac{z^2}{b^2}=1$ in the local coordinate
system, by plugging the coordinates of the center into the surface equation and
solving for $z^2$, we have the squared height $-\frac{b^2}{b^2-a^2s^2}\Theta$.

     If $\Theta=0$, ${\cal P}$ is tangent to ${\cal Q}_{H2}$ at
$\arrow{V}+\frac{a^2}{d}\arrow{\ell}+\frac{b^2}{ds}\arrow{k}$
(Figure~\ref{fig:hy2}(b)).

     If $\Theta>0$, $y=s(x-d)$ intersects the profile hyperbola in two real
points and hence the squared semi-axis length along $\arrow{m}$ is always
positive.  If $|s|<\frac{b}{a}$, the squared semi-axis length along $\arrow{n}$
is negative and the intersection must be a hyperbola with $\arrow{m}$ and
$\arrow{n}$ as its major and minor axis directions (Figure~\ref{fig:hy2}(c)).
The semi-axis length can be computed by taking the square root of the absolute
value of the corresponding squared length.  If $|s|>\frac{b}{a}$, both squared
semi-axis lengths are positive and hence the intersection is an ellipse with
major and minor axis directions $\arrow{m}$ and $\arrow{n}$
(Figure~\ref{fig:hy2}(d)).  The semi-axis length can be computed similarly.

     Now if $|s|=\frac{b}{a}$, the leading coefficient of equation
(\ref{eqn:hy2-2degree}) is zero and the root is $x=\frac{a^2+d^2}{2d}$
(Figure~\ref{fig:hy2}(e)).  Therefore the intersection point, which is also
the vertex of the intersection parabola, is
$\arrow{V}+\frac{a^2+d^2}{2d}\arrow{\ell}-\frac{b(a^2-d^2)}{2ad}\arrow{k}$.
Plane ${\cal P}$ intersects the surface and its asymptote cone in
non-intersecting parabolas and thus have focal length
$f=\frac{|d|}{2}\frac{\sin^2\alpha}{\cos\alpha}$, where $\alpha$ is the
angle of the asymptote cone.  Therefore, $\tan\alpha=\frac{b}{a}$.  Hence, we
have $f=\frac{|d|}{2}\frac{b}{a\sqrt{a^2+b^2}}$.  The major axis direction
can be determined as in the last section.  \QED

% --------------------------------------------------------------------
%                  Prolate and Oblate Ellipsoids
% --------------------------------------------------------------------

\section{Ellipsoids}
\label{section:ellipsoid}

     Although there are two kinds of revolute ellipsoids, their computation is
simpler than the hyperboloids as we shall see in this section.  Consider
the prolate ellipsoid ${\cal Q}_{EP}(\arrow{V},\arrow{\ell},a,b)$.  Cases
$\arrow{\ell}\ ||\ \arrow{o}$ and $\arrow{\ell}\perp\arrow{o}$ are simple.

\begin{lemma}
     Suppose ${\cal P}(\arrow{B},\arrow{o})$ and
${\cal Q}_{EP}(\arrow{V},\arrow{\ell},a,b)$ are a plane and a prolate
ellipsoid.  Then we have:
\begin{enumerate}
     \item $\arrow{\ell}\ ||\ \arrow{o}$:
          Let $\arrow{D}=\arrow{V}+d\arrow{\ell}$ be the intersection
          point of ${\cal P}$ and the axis of revolution.  Then, we have:
     \begin{enumerate}
          \item $|d|>a$: ${\cal P}$ and ${\cal Q}_{EP}$ are disjoint.
          \item $|d|=a$: ${\cal P}$ and ${\cal Q}_{EP}$ are tangent at
               $\arrow{D}$.
          \item $|d|<a$: The intersection is a circle in ${\cal P}$ with
               center $\arrow{D}$ and radius $\frac{b}{a}\sqrt{a^2-d^2}$.
     \end{enumerate}
     \item $\arrow{\ell}\perp\arrow{o}$:
          Let $\arrow{E}=\arrow{V}+e\arrow{o}$ be the intersection
          point of ${\cal P}$ and the line $\arrow{V}+t\arrow{o}$.
          Then, we have:
     \begin{enumerate}
          \item $|e|>b$: ${\cal P}$ and ${\cal Q}_{EP}$ are disjoint.
          \item $|e|=b$: ${\cal P}$ and ${\cal Q}_{EP}$ are tangent at
               $\arrow{E}$.
          \item $|e|<b$: The intersection is an ellipse with center
               $\arrow{E}$, major axis and minor axis directions $\arrow{\ell}$
       and $\frac{\arrow{\ell}\times\arrow{o}}{|\arrow{\ell}\times\arrow{o}|}$,
               and semi-major and semi-minor axis lengths
               $\frac{a}{b}\sqrt{b^2-e^2}$ and $\sqrt{b^2-e^2}$.
     \end{enumerate}
\end{enumerate}
\end{lemma}
{\bf Proof:}  Omitted. \QED

\begin{theorem}
\label{thm:prolate}
     Suppose ${\cal P}(\arrow{B},\arrow{o})$ and
${\cal Q}_{EP}(\arrow{V},\arrow{\ell},a,b)$ are a plane and a prolate
ellipsoid.  Suppose $\arrow{\ell}$ and $\arrow{o}$ are neither parallel nor
perpendicular to each other.  Let $\arrow{D}=\arrow{V}+d\arrow{\ell}$ be the
intersection point of ${\cal P}$ and the axis of revolution.
Let $s$ be the slope of the
line $\arrow{D}+t\arrow{m}$ in the local coordinate system.
Let $\Theta=b^2+(a^2-d^2)s^2$.  Then we have:
\begin{enumerate}
     \item $\Theta<0$: ${\cal P}$ and ${\cal Q}_{EP}$ are disjoint.
     \item $\Theta=0$: ${\cal P}$ and ${\cal Q}_{EP}$ are tangent at
          $\arrow{V}+\frac{a^2}{d}\arrow{\ell}-\frac{b^2}{ds}\arrow{k}$.
     \item $\Theta>0$: The intersection is an ellipse with center at
          $\arrow{V}+\frac{da^2s^2}{b^2+a^2s^2}\arrow{\ell}
               -\frac{sdb^2}{b^2+a^2s^2}\arrow{k}$, major axis and minor axis
          directions $\arrow{m}$ and $\arrow{n}$, and semi-major and semi-minor
          axis lengths $\frac{ab}{b^2+a^2s^2}\sqrt{(1+s^2)\Theta}$ and
          $\frac{b}{\sqrt{b^2+a^2s^2}}\sqrt{\Theta}$.
\end{enumerate}
\end{theorem}
{\bf Proof:}  On the $xy$-plane the profile ellipse is
$\frac{x^2}{a^2}+\frac{y^2}{b^2}=1, a>b$.  The intersection of the profile
ellipse and $y=s(x-d)$ can be found by plugging the latter into the profile
ellipse's equation.  Thus, we have
\begin{equation}
\label{eqn:pro-2degree}
     (b^2+a^2s^2)x^2-2da^2s^2x+a^2(d^2s^2-b^2)=0.
\end{equation}
This equation has roots as follows:
\[ x=\frac{da^2s^2\pm ab\sqrt{\Theta}}{b^2+a^2s^2}. \]
>From the above equation, if $\Theta<0$, there will be no real intersection
point (Figure~\ref{fig:EP}(a)); if $\Theta=0$, the
intersection point is $\left(\frac{a^2}{d},-\frac{b^2}{ds},0\right)$
(Figure~\ref{fig:EP}(b)).  Converting it to the global coordinate system, the
tangent point is $\arrow{V}+\frac{a^2}{d}\arrow{\ell}-\frac{b^2}{ds}\arrow{k}$.
\begin{figure}
\vspace{5cm}
\caption{Various Intersection Types of a Prolate Ellipsoid}
\label{fig:EP}
\end{figure}

     Suppose $\Theta>0$.  The center is the midpoint of the two intersection
points (Figure~\ref{fig:EP}(c)).  Some simple calculation gives
$\left(\frac{da^2s^2}{b^2+a^2s^2},-\frac{sdb^2}{b^2+a^2s^2},0\right)$.
Converting to the global coordinate system delivers the desired result.
The semi-axis length along $\arrow{m}$ is half of the distance between the two
intersection points,  $\frac{ab}{b^2+a^2s^2}\sqrt{(1+s^2)\Theta}$.
Finally, the semi-axis length along $\arrow{n}$ is the height from the center
to the surface.  Since the surface has equation
$\frac{x^2}{a^2}+\frac{y^2}{b^2}+\frac{z^2}{b^2}=1$ in the local
coordinate system, by plugging the coordinates of the center into the this
equation and solving for $z$, we have
$z=\frac{b}{\sqrt{b^2+a^2s^2}}\sqrt{\Theta}$. It is not difficult to verify
that the semi-axis length along $\arrow{m}$ is larger than the semi-axis
length along $\arrow{n}$.  Therefore, $\arrow{m}$ ({\em resp.}, $\arrow{n}$) is
the major ({\em resp.}, minor) axis direction.  \QED

     For an oblate ellipsoid, no additional work is necessary since by
interchanging $a$ and $b$, the major and minor axes, related lengths
will establish  the desired result immediately.  The reason
is that in the calculation of the above theorem we do not use any specific
property of the major and minor axes of the profile ellipse, except
the axis lengths $a$ and $b$.  Hence, for $\arrow{\ell}\ ||\ \arrow{o}$ and
$\arrow{\ell}\perp\arrow{o}$, we have the following lemma.

\begin{lemma}
     Suppose ${\cal P}(\arrow{B},\arrow{o})$ and
${\cal Q}_{EO}(\arrow{V},\arrow{\ell},a,b)$ are a plane and an oblate
ellipsoid.  Then we have:
\begin{enumerate}
     \item $\arrow{\ell}\ ||\ \arrow{o}$:
          Let $\arrow{D}=\arrow{V}+d\arrow{\ell}$ be the intersection
          point of ${\cal P}$ and the axis of revolution.
          Then, we have:
     \begin{enumerate}
          \item $|d|>b$: ${\cal P}$ and ${\cal Q}_{EO}$ are disjoint.
          \item $|d|=b$: ${\cal P}$ and ${\cal Q}_{EO}$ are tangent at
               $\arrow{D}$.
          \item $|d|<b$: The intersection is a circle in ${\cal P}$ with
               center $\arrow{D}$ and radius $\frac{a}{b}\sqrt{b^2-d^2}$.
     \end{enumerate}
     \item $\arrow{\ell}\perp\arrow{o}$:
          Let $\arrow{E}=\arrow{V}+e\arrow{o}$ be the intersection
          point of ${\cal P}$ and the line $\arrow{V}+t\arrow{o}$.
          Then, we have:
     \begin{enumerate}
          \item $|e|>a$: ${\cal P}$ and ${\cal Q}_{EO}$ are disjoint.
          \item $|e|=a$: ${\cal P}$ and ${\cal Q}_{EO}$ are tangent at
               $\arrow{E}$.
          \item $|e|<a$: The intersection is an ellipse with center
               $\arrow{E}$, major axis and minor axis directions
          $\frac{\arrow{\ell}\times\arrow{o}}{|\arrow{\ell}\times\arrow{o}|}$
               and $\arrow{\ell}$,
               and semi-major and semi-minor axis lengths $\sqrt{a^2-e^2}$
               and $\frac{b}{a}\sqrt{a^2-e^2}$.
     \end{enumerate}
\end{enumerate}
\end{lemma}

     The same argument as presented in Theorem~\ref{thm:prolate}
gives the following theorem.

\begin{theorem}
\label{thm:oblate}
     Suppose ${\cal P}(\arrow{B},\arrow{o})$ and
${\cal Q}_{EO}(\arrow{V},\arrow{\ell},a,b)$ are a plane and an oblate
ellipsoid.  Suppose $\arrow{\ell}$ and
$\arrow{o}$ are neither parallel nor perpendicular to each other.
Let $\arrow{D}=\arrow{V}+d\arrow{\ell}$ be the intersection point of ${\cal P}$
and the axis of revolution.  Let $s$ be the slope of the
line $\arrow{D}+t\arrow{m}$ in the local coordinate system.
Let $\Theta=a^2+(b^2-d^2)s^2$.  Then we have:
\begin{enumerate}
     \item $\Theta<0$: ${\cal P}$ and ${\cal Q}_{EO}$ are disjoint.
     \item $\Theta=0$: ${\cal P}$ and ${\cal Q}_{EO}$ are tangent at
          $\arrow{V}+\frac{b^2}{d}\arrow{\ell}-\frac{a^2}{ds}\arrow{k}$.
     \item $\Theta>0$: The intersection is an ellipse with center at
          $\arrow{V}+\frac{db^2s^2}{a^2+b^2s^2}\arrow{\ell}
               -\frac{sda^2}{a^2+b^2s^2}\arrow{k}$, major axis and minor axis
          directions $\arrow{n}$ and $\arrow{m}$, and semi-major and semi-minor
          axis lengths $\frac{a}{\sqrt{a^2+b^2s^2}}\sqrt{\Theta}$ and
          $\frac{ab}{a^2+b^2s^2}\sqrt{(1+s^2)\Theta}$.
\end{enumerate}
\end{theorem}

% --------------------------------------------------------------------
%                       Elliptic Paraboloid
% --------------------------------------------------------------------

\section{Elliptic Paraboloid}
\label{section:elliptic-paraboloid}

     In this section, we will compute the intersection of a plane ${\cal P}$
and a revolute elliptic paraboloid ${\cal Q}_P(\arrow{V},\arrow{\ell},f)$,
where $\arrow{V}$, $\arrow{\ell}$ and $f$ are the vertex, the major axis
direction pointing toward the opening of the paraboloid, and the focal length
of the profile parabola respectively. We still have two simple cases.

\begin{lemma}
\label{lemma:para-para&perp}
     Suppose ${\cal P}(\arrow{B},\arrow{o})$ and
${\cal Q}_P(\arrow{V},\arrow{\ell},f)$ are a plane and an elliptic paraboloid.
Then we have:
\begin{enumerate}
     \item $\arrow{o}\ ||\ \arrow{\ell}$:
          Let $\arrow{D}=\arrow{V}+d\arrow{\ell}$ be the intersection
          point of ${\cal P}$ and the axis of revolution.
          Then we have:
     \begin{enumerate}
          \item $d<0$: ${\cal P}$ and ${\cal Q}_P$ are disjoint.
          \item $d=0$: ${\cal P}$ and ${\cal Q}_P$ are tangent at $\arrow{V}$.
          \item $d>0$: The intersection is a circle in ${\cal P}$ with center
               $\arrow{D}$ and radius $2\sqrt{fd}$.
     \end{enumerate}
     \item $\arrow{o}\perp\arrow{\ell}$:
          Let $\arrow{E}=\arrow{V}+e\arrow{o}$ be the intersection point of
          ${\cal P}$ and the line $\arrow{V}+t\arrow{o}$.
          Then, the intersection
          is a parabola with vertex
          $\arrow{V}+\frac{e^2}{4f}\arrow{\ell}+e\arrow{o}$, major axis
          direction $\arrow{\ell}$, directrix direction
          $\frac{\arrow{\ell}\times\arrow{o}}{|\arrow{\ell}\times\arrow{o}|}$,
          and focal length $f$.
\end{enumerate}
\end{lemma}
{\bf Proof:} Omitted. \QED

     The following theorem computes the intersection curve for the
general case.

\begin{theorem}
\label{thm:elliptic-paraboloid}
     Suppose ${\cal P}(\arrow{B},\arrow{o})$ and
${\cal Q}_P(\arrow{V},\arrow{\ell},f)$ are a plane and an elliptic paraboloid.
Suppose $\arrow{\ell}$ and $\arrow{o}$ are neither parallel nor perpendicular
to each other.  Let $\arrow{D}=\arrow{V}+d\arrow{\ell}$ be the intersection
point of ${\cal P}$ and the axis of revolution.
Let $s$ be the slope of the line $\arrow{D}+t\arrow{m}$ in the local
coordinate system.  Let $\Theta=f(f+ds^2)$.  Then we have:
\begin{enumerate}
     \item $\Theta<0$: ${\cal P}$ and ${\cal Q}_P$ are disjoint.
     \item $\Theta=0$: ${\cal P}$ and ${\cal Q}_P$ are tangent at
          $\arrow{V}+\frac{f}{s^2}\arrow{m}+\frac{2f}{s}\arrow{k}$.
     \item $\Theta>0$: The intersection is an ellipse with center
          $\arrow{V}+\frac{2f+ds^2}{s^2}\arrow{\ell}+\frac{2f}{s}\arrow{k}$,
          major axis and minor axis directions $\arrow{m}$ and $\arrow{n}$,
          and semi-major and semi-minor axis lengths
          $\frac{4}{s^2}\sqrt{(1+s^2)\Theta}$ and $\frac{2}{|s|}\sqrt{\Theta}$.
\end{enumerate}
\end{theorem}
{\bf Proof:}  On the $xy$-plane, the profile parabola is $y^2=4fx$.
The intersection points of $y=s(x-d)$ and the profile parabola can be solved
by plugging the former into the latter.  This gives
\begin{equation}
\label{eqn:ellp-para-2degree}
     s^2x^2-2(2f+ds^2)x+d^2s^2=0.
\end{equation}
Solving for $x$ we have
\[ x = \frac{1}{s^2}(2f+ds^2\pm 2\sqrt{\Theta}).  \]
If $\Theta<0$, there will be no real solution and ${\cal P}$ and
${\cal Q}_P$ are disjoint (Figure~\ref{fig:para}(a)).
If $\Theta=0$, equation (\ref{eqn:ellp-para-2degree}) gives the
tangent point of ${\cal P}$ and ${\cal Q}_P$ (Figure~\ref{fig:para}(b)).
\begin{figure}
\vspace{6cm}
\caption{Various Intersection Types of an Elliptic Paraboloid}
\label{fig:para}
\end{figure}

     Now if $\Theta>0$, the midpoint of the two intersection points is
$\left(\frac{2f+ds^2}{s^2},\frac{2f}{s},0\right)$ in the local coordinate
system.  Converting it to the global coordinate system we have the desired
result (Figure~\ref{fig:para}(c)).  The axis length along $\arrow{m}$ is half
of the distance between the two intersection points and is equal to
$\frac{4}{s^2}\sqrt{(1+s^2)\Theta}$.  The axis length along $\arrow{n}$
is the height from the center to ${\cal Q}_P$.  Since the
surface has equation $y^2+z^2=4fx$ in the local coordinate system, by plugging
in the center coordinates into the surface equation and solving for $z$ we have
$z=\frac{2}{|s|}\sqrt{\Theta}$.  It is obvious that the semi-axis length
along $\arrow{m}$ is larger than the semi-axis length along $\arrow{n}$.
Therefore, $\arrow{m}$ gives the major axis direction.  \QED

% --------------------------------------------------------------------
%                           Conclusion
% --------------------------------------------------------------------

\section{Conclusion}
\label{section:conclusion}

     Using an appropriately chosen local coordinate system, we have
successfully developed a simple geometric method to compute the intersection
conic of a plane and a revolute quadric surface.  It is interesting to note
that all parallel sections can be obtained easily since the center is a linear
function of $d$ (the distance from the intersection point of the axis of
revolution and the given plane), the axis lengths are the square roots of
quadratic functions of $d$, and the axis directions are independent to $d$.
(If the intersection conic is a parabola, the vertex is a quadratic function
of $d$, the focal length is a linear function of $d$.)  Thus, once a plane
intersection is computed, all other parallel intersections can be obtained
immediately by varying $d$.  Compare this result with the one given in
Johnstone and Shene~\cite{johnstone-shene:1991}.

     Although pure geometric methods may have disadvantages in computing
plane intersection of higher degree surfaces, they are powerful tools for those
surfaces with simple geometric definitions such as quadrics and tori.
We are currently working on extending the techniques presented here to these
surfaces.

% --------------------------------------------------------------------
%                           BIBLIOGRAPHY
% --------------------------------------------------------------------

\newpage
\begin{thebibliography}{99}

\bibitem{adams:1988}
     J. A. Adams and L. M. Billow,
     {\em Descriptive Geometry and Geometric Modeling: A Basis for Design},
     Holt, Rinehart and Winston, New York (1988).

\bibitem{bereis:1964}
     R. Bereis,
     {\em Darstellende Geometrie I},
     Akademie-Verlag, Berlin (1964).

\bibitem{ding-davis:1987}
     Q. Ding and B. J. Davis,
     {\em Surface Engineering Geometry for Computer Aided Design and
          Manufacture},
     Ellis Horwood, Chichester, West Sussex, England (1987).

\bibitem{drew:1875}
     W. H. Drew,
     {\em A Geometrical Treatise on Conic Sections},
     fifth edition,
     Macmillan and Co. (1875).

\bibitem{goldman:1983a}
     R. N. Goldman,
     Quadrics of Revolution,
     {\em IEEE Computer Graphics and Applications},
     {\bf 3}(2), pp. 68--76 (1983).

\bibitem{goldman:1983b}
     R. N. Goldman,
     Two Approaches to a Computer Model for Quadric Surfaces,
     {\em IEEE Computer Graphics and Applications},
     {\bf 3}(6), pp. 21--24 (1983).

\bibitem{hosaka:1992}
     M. Hosaka,
     {\em Modeling of Curves and Surfaces in CAD/CAM},
     Springer-Verlag, New York (1992).

\bibitem{johnstone-shene:1991}
     J. K. Johnstone and C.-K. Shene,
     Computing the Intersection of a Plane and a Natural Quadric,
     {\em Computers \& Graphics},
     {\bf 16}(2), pp. 179--186 (1992).

\bibitem{miller-goldman:1991a}
     J. R. Miller and R. N. Goldman,
     Using Tangent Balls to Find Plane Sections of Natural Quadric Surfaces,
     {\em IEEE Computer Graphics and Applications},
     {\bf 12}(2), pp. 68--82 (1992).

\bibitem{rogers-adams:1990}
     D. F. Rogers and J. A. Adams,
     {\em Mathematical Elements for Computer Graphics},
     second edition,
     McGraw-Hill, New York (1990).

\bibitem{ryan:1991}
     D. Ryan,
     {\em CAD/CAE Descriptive Geometry},
     CRC Press, Boca Raton (1991).

\bibitem{schmid:1912}
     T. Schmid,
     {\em Darstellende Geometrie}, 2 volumes,
     G. J. G\"{o}schen'sche Verlagshandlung,
     Berlin and Leipzig (1912--1923).

\bibitem{slaby:1966}
     S. M. Slaby,
     {\em Fundamentals of Three-Dimensional Descriptive Geometry},
     Harcourt, Brace \& World, Inc., New York (1966).

\bibitem{snyder-sisam:1914}
     V. Snyder and C. H. Sisam,
     {\em Analytic Geometry of Space},
     Henry Hold and Company, New York (1914).

\bibitem{strubecker:1958}
     K. Strubecker,
     {\em Vorlesungen \"{u}ber Darstellende Geometrie},
     Vandenhoeck \& Ruprecht, G\"{o}ttingen (1958).

\bibitem{wilson:1987}
     P. R. Wilson,
     Conic Representations for Shape Description,
     {\em IEEE Computer Graphics and Applications},
     {\bf 7}(4), pp. 23--40 (1987).

\end{thebibliography}

% ------------------------------ THE END -----------------------------

\end{document}

