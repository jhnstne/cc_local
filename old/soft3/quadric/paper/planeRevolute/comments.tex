CK, here are some comments on the plane-revolute paper.
	I also have a number of changes that are best described
	by faxing you a copy of my annotated copy of the old paper
	along with the new paper.
	I will call you this afternoon to find out how best to fax to you.

	Since I had already made most of the changes to the file
	you sent me earlier, I am sending you an altered version
	of the *old* file.
	That is, I am not using the file that you sent me today.
	
	I have not addressed the two issues of Referee 2:
	(1) advantages of the method or (2) relevance to computer graphics.
	Would you please add something and then I can comment?
	Your notes of this morning's mail could be incorporated:
	((1) robustness (2) parallel sections)
	but, for 1, you should also say that other methods are not written up
	in the literature, 
	and, for 2, I would also say that plane sections are fundamental 
	for the definition of solids, which is needed in computer
	graphics.

p. 1, references on descriptive geometry: Couldn't an Engligh book also
	be included?

p. 4, Rmk 2.1: in Farin, position vectors and direction vectors are 
	{\em points} and vectors.  Why don't you use that idea?

p. 6, Lemma 3.4 (previous sentence): two parabolas do not necessarily have
	the same focal length even if they are similar, if the similarity
	involves scaling.  Therefore, state that intersection parabola
	and meridian parabola are actually *identical*?

p. 9, step 5: 	Need to elaborate on computation of squared semi-axis lengths.
	Most importantly, elaborate on computation of squared height (at center).
	(In order to explain negative value, and also to show how if surface
	is not above center.)
	A hint is given on p. 12 top, but more is needed.
	(Note: height is defined initially as the absolute value of the
	z-coordinate.  This does not seem appropriate given that it can be
	imaginary.)

p. 9, step 1: state the section where parabola case will be given.

p. 10, proof of Lemma 5.2: `Since s = -a/b ' is wrong.
	Would work if you replace -a/b by s in the next sentence
	and remove `since s = -a/b < 0' statement.

p. 13, just above Section 6: focal lengths of similar parabolas are the same?
	wouldn't distance between focus and vertex stretch as parabola does?
	this is the same concern as on p. 6, Lemma 3.4

p. 18: why don't you collapse Lemma 7.2 with Lemma 7.1?
	Simply say `Suppose P(B,o) is a plane and Q_E(V,l,a,b) 
	is a prolate ellipsoid (or Q_E(V,l,b,a) is an oblate ellipsoid).'
	Then replace all Q_EP by Q_E in the lemma and proof.

	Similarly could collapse Thm 7.2 with Thm 7.1.



It would be nice if $\theta$ was given some interpretation
	(e.g., Thms 5.1, 6.1, 7.1, 8.1, etc.)

In Section 4, I would have introduced what the axial plane is and
	then said that we will create a coordinate system where the
	axial plane is the xy-plane.  This would better
	motivate the early development of the section.
	This might be too much of a change to make now.