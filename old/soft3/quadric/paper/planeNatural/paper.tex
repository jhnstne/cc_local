\documentstyle [12pt]{article}

\newif\ifFull
%\Fullfalse
\Fulltrue

\def\thefootnote{\fnsymbol{footnote}}

\title{
     Computing the Intersection of a Plane \\
     and a Natural Quadric\footnotemark[1]}

\author{John K. Johnstone\ \ \ \ Ching-Kuang Shene\\ \\
          Department of Computer Science\\
          The Johns Hopkins University \\
          Baltimore, Maryland 21218  USA}

\date{\ }

\newtheorem{example}{Example}[section]
\newtheorem{property}{Property}[section]
\newtheorem{definition}{Definition}[section]
\newtheorem{theorem}{Theorem}[section]
\newtheorem{lemma}{Lemma}[section]
\newtheorem{corollary}{Corollary}[section]

\newcommand{\DoubleSpace}{\edef\baselinestretch{1.4}\Large\normalsize}
\newcommand{\QED}{\ \ \ \rule{2mm}{3mm}\\}
\newcommand{\arrow}[1]{\vec{\bf #1}}

\DoubleSpace
\setlength{\oddsidemargin}{0pt}
\setlength{\evensidemargin}{0pt}
\setlength{\headsep}{0pt}
\setlength{\topmargin}{0pt}
\setlength{\textheight}{8.75in}
\setlength{\textwidth}{6.5in}


\begin{document}
\maketitle
\footnotetext[1]{This work was supported by 
          National Science Foundation grant IRI--8910366.}

\def\thefootnote{\arabic{footnote}}
\setcounter{footnote}{0}

% --------------------------------------------------------------------
%                            INTRODUCTION
% --------------------------------------------------------------------

\section{Introduction}

Solid models are commonly composed of planes and quadric surfaces,
and their construction requires the intersection of these surfaces
\cite{HOF89,MANT88}.
Rather than using all quadrics, 
many solid modelers restrict to the use of spheres, cylinders,
and cones (the {\em natural quadrics}).
Therefore, the intersection of planes and natural quadrics is a very important
operation.

Plane sections are also a basic tool in descriptive geometry, drafting,
and computer-aided design \cite{adams:1988}.
A sequence of parallel sections is an aid in the visualization of a surface.
% It is also used to generate a cutter contour path for numerically-controlled
% (NC) machining \cite{chang:1989}.

Despite its importance, there is a surprising lack of literature 
on the computation of plane sections of natural quadrics, 
undoubtedly largely because the commercial solid modeler 
has little motivation to publish.
% e.g., conversation with voelcker and his student
Note that one should not be content with
more general methods of intersection for this
problem (e.g., the use of elimination), since they are overly complicated.
In this paper, we propose to fill this gap in the literature,
and provide new insight into this important problem.

During our recent work on the intersection of quadric surfaces 
\cite{shene:1991}, 
a method arose for the intersection of planes and natural quadrics.
The method is simple and offers new ideas on plane sections.
With no added computation, it also delivers all parallel plane sections,
parameterized by the position of the plane.
An algorithm for intersecting a plane and natural quadric that is quite
similar has recently been developed by Miller and Goldman \cite{MiGo91}.

The main thrust of the paper is the computation of the focal length
of the intersection conic.
We show how this computation can be performed,
using a restriction to a plane (different from the given plane).
We should note that, since our aim is to show the concept,
we have not optimized any of the formulae in this paper for implementation.

In the following section, 
we show how to describe a conic by its axes and center (or vertex).
In Section~\ref{section:determining-axes}, 
we show how to identify the major and minor axes of the intersection conic.
Section~\ref{section:cone} 
presents the computation of plane sections of a cone,
while the less interesting cases of sphere and cylinder are discussed
in Section~\ref{section:simple}.
In Section~\ref{ss:focal},
we give the general method of computing the focal length,
using the theory of Dandelin spheres, incircles, and excircles.
Section~\ref{section:ellipse} 
deals with elliptic and hyperbolic conic sections,
Section~\ref{section:parabola} with the parabola,
and Section~\ref{section:vertex-on-plane} 
with linear intersections (arising when the plane contains 
the vertex of the cone).


% --------------------------------------------------------------------
%                      DESCRIPTION OF A CONIC
% --------------------------------------------------------------------

\section{The Description of a Conic}

A central conic (i.e., an ellipse or hyperbola) 
is uniquely determined by the directions of
its major and minor axes, the lengths of its semi--major and semi--minor 
axes, and its center.
A parabola is uniquely determined by its major axis, 
directrix, vertex, and focal length.
Note that the ellipse has the equation $(m\cos\phi\arrow{m}+n\sin\phi\arrow{n}
)+\arrow{C}$ and the hyperbola has the equation 
$(m\cosh\phi\arrow{m}+n\sinh\phi\arrow{n})+\arrow{C}$,
where $\arrow{m},\arrow{n},\arrow{C}$ and $\phi$ are the major
axis direction, the minor axis direction, the center and a parameter.
For a parabola, if the direction of its major axis is from the vertex to the
focus, the equation is $\left(\frac{t^2}{4f}\arrow{m}+t\arrow{n}\right)+
\arrow{C}$, where $\arrow{m},\arrow{n},\arrow{C},f$ and $t$ are the major
axis direction, the directrix direction, the vertex, the focal length,
and a parameter.

In this paper, the length of the semi--minor axis, $n$, is computed 
in terms of the length of the semi--major axis, $m$, and the focal length, $f$.
For an ellipse, since $f=\sqrt{m^2-n^2}$, 
the semi--minor axis length is $n=\sqrt{m^2-f^2}$;  
for a hyperbola, the focal length is
$f=\sqrt{m^2+n^2}$ and thus $n=\sqrt{f^2-m^2}$ \cite{salmon:1954}.  % page 177

{\bf Notation:}
A plane through
$\arrow{B}$ and perpendicular to $\arrow{O}$ (its orientation) is denoted by
${\cal P}(\arrow{B},\arrow{O})$, the sphere with center $\arrow{C}$ and radius
$r$ by ${\cal S}(\arrow{C},r)$, the cylinder with axis $\arrow{V}+t\arrow{A}$
and radius $r$ by ${\cal Z}(\arrow{V},\arrow{A},r)$, and the right cone
with vertex $\arrow{V}$, axis $\arrow{V}+t\arrow{A}$ and cone angle $\alpha$
by ${\cal C}(\arrow{V},\arrow{A},\alpha)$.  Note that vectors $\arrow{m},
\arrow{n}, \arrow{O}$ and $\arrow{A}$ give directions, not positions, while
$\arrow{B},\arrow{V}$ and $\arrow{C}$ give positions.
We assume that all vectors giving directions are of unit length.
Usually, position vector $\arrow{X}$ is also denoted by $X$ in discussing the
geometric point $X$.

     $\stackrel{\longleftrightarrow}{AB}, \stackrel{\longrightarrow}{AB},
\overline{AB}$ and $|\overline{AB}|$ are the line, the half ray, the segment,
and the length of the segment determined by two distinct points $A$ and $B$.

% --------------------------------------------------------------------
%             DIRECTIONS OF THE MAJOR AND MINOR AXES
% --------------------------------------------------------------------

\section{Directions of the Major and Minor Axes}
\label{section:determining-axes}

A result in Shene and Johnstone~\cite{shene:1991} 
yields a simple way to determine the direction of the 
major and minor axis of the intersection conic.  

\begin{lemma}
\label{lemma:perpendicular-axis}
    Let ${\cal P}$ be a plane intersecting a cone/cylinder in a conic 
$C$.  Then the plane ${\cal E}$ determined by the major axis of $C$ and the 
axis of the natural quadric is perpendicular to ${\cal P}$
(Figure~\ref{fig:perpendicular-stuff}).
\end{lemma}

\begin{figure}
\vspace{4cm}
\caption{Relation of plane ${\cal P}$ and plane ${\cal E}$}
\label{fig:perpendicular-stuff}
\end{figure}

\begin{corollary}
\label{lemma:major-minor-axes}
    Let ${\cal P}(\arrow{B},\arrow{O})$ be a plane 
and ${\cal Q}$ be a cone or cylinder with axis $\arrow{Q}+t\arrow{A}$,
$\arrow{A}\neq\pm\arrow{O}$.
The directions of the minor
and major axes of the intersection conic are $\arrow{A}\times\arrow{O}$ and 
$\arrow{O}\times(\arrow{A}\times\arrow{O})$, respectively.  
\end{corollary}
{\bf Proof:}
The minor axis is perpendicular to the major axis and thus perpendicular
to ${\cal E}$ of Lemma~\ref{lemma:perpendicular-axis}.
Thus, it is perpendicular to the axis $\arrow{A}$ of the natural quadric.
Since the minor axis of the conic $C$ lies in ${\cal P}$,
it is perpendicular to $\arrow{O}$.
\QED

Note that the cross products in Corollary~\ref{lemma:major-minor-axes} are 
well defined, because the case $\arrow{A}=\pm\arrow{O}$ is easy to handle
(Lemma~\ref{lemma:parallel-1}).
The plane ${\cal E}$ of Lemma~\ref{lemma:perpendicular-axis} is important.
Since it contains the cone's axis and the conic's major axis,
we call it the {\bf axial plane}.
The axial plane is trivially computed, since its normal is the minor axis 
$\arrow{A}\times\arrow{O}$ (the axial plane contains $\arrow{A}$
and is perpendicular to ${\cal P}$).
All computations of the intersection conic will be performed in this plane.
That is, we can compute the major axis and its length, the focal length,
and the center or vertex, all while restricting to the axial plane,
since the axial plane contains the major axis and (thus) the foci and center.

% --------------------------------------------------------------------
%               Choosing the sign of the direction
% --------------------------------------------------------------------

\subsection{Choosing the sign of the direction}
\label{section:sign}

There are two choices for the directions of the major and minor axes:
$\pm \arrow{A}\times\arrow{O}$ and
$\pm \arrow{O}\times(\arrow{A}\times\arrow{O})$.
The choice is important because we will use the angle between the 
cone/cylinder axis and the plane's normal 
(called $\theta$ below) in calculations, 
and this angle is affected by the choice of direction.
We make the choice as follows.

     Assuming it exists, let $\arrow{T}$ be the intersection of ${\cal P}$,
the given plane, and the axis of the natural quadric: 
$\arrow{T}={\cal P}\cap (\arrow{V}+t\arrow{A})$ for some suitable $t$.  
Define $\arrow{a}$ as follows:
\[ \arrow{a} = \left\{
                    \begin{array}{lll}
                         \arrow{A} & & \mbox{if $t\geq 0$,} \\
                         -\arrow{A} & & \mbox{if $t< 0$.}
                    \end{array} \right. \]
We will use $\arrow{a}$ instead of $\arrow{A}$ in all of our calculations.

     Consider the orientation vector $\arrow{O}$ of plane ${\cal P}$.
The inner product $\arrow{a}\cdot\arrow{O}$ gives the cosine of the smaller
angle $\phi$ between $\arrow{a}$ and $\arrow{O}$.  $\phi$ is an acute angle
if and only if $\arrow{a}\cdot\arrow{O}>0$.  Let
\[ \arrow{o} = \left\{
                    \begin{array}{lll}
                     \arrow{O} & & \mbox{if $\arrow{a}\cdot\arrow{O}\geq 0$,}\\
                     -\arrow{O} & & \mbox{if $\arrow{a}\cdot\arrow{O}< 0$.}
                    \end{array} \right. \]
We will use $\arrow{o}$ instead of $\arrow{O}$ in all of our calculations.
Note that $\arrow{a}$ always points toward the plane ${\cal P}$ and 
the angle $\theta$ between $\arrow{a}$ and $\arrow{o}$ is always
an acute angle with $\cos\theta=\arrow{a}\cdot\arrow{o}\geq 0$.

     In what follows, unit length vectors $\arrow{m}$ and $\arrow{n}$ denote
the directions of the major and the minor axes.  That is, if they exist, 
we have the following
\[     \arrow{m} = \frac{\arrow{o}\times(\arrow{a}\times\arrow{o})}
                        {|\arrow{o}\times(\arrow{a}\times\arrow{o})|}
     \ \ \ \ \mbox{and}\ \ \ \  
     \arrow{n} = \frac{\arrow{a}\times\arrow{o}}
                        {|\arrow{a}\times\arrow{o}|}.   \]
Otherwise, $\arrow{m}$ and $\arrow{n}$ will be defined explicitly.


% --------------------------------------------------------------------
%                              CONES
% --------------------------------------------------------------------

\section{Plane/Cone Intersection}
\label{section:cone}

\subsection{How to compute the focal length}
\label{ss:focal}

Consider the intersection of a plane with a right cone, which
is the most difficult case.
In the last century, Dandelin~\cite{dandelin:1822} discovered that if a 
sphere is inscribed in the cone and tangent to the plane, the tangent point 
on the plane is one of the foci of the intersection conic 
(Figure~\ref{dandelin-sphere}).\footnote{The foci--directrix--eccentricity 
     relation was found by Morton~\cite{morton:1830} a few years later.
     For other properties derived from the Dandelin
     sphere and a proof of Dandelin's theorem in English,
     the reader is referred to
     Drew~\cite{drew:1875}, Hilbert and Cohn--Vossen~\cite{hilbert:1983}
     and Macaulay~\cite{macaulay:1895}.}
This sphere is called a {\em Dandelin} (or {\em focal}) sphere. 
We will use it to compute the focal length of the intersection.
\begin{figure}
\vspace{3.25in}
\caption{The Dandelin Sphere}
\label{dandelin-sphere}
\end{figure}

Let plane ${\cal P}$ intersect cone ${\cal C}$ in a central conic.  
(The parabola is treated separately in Section~\ref{section:parabola}.)
There will be one Dandelin sphere tangent to ${\cal P}$ at each focus point.  
Consider the axial plane, as shown in Figure~\ref{fig:tangent-points}.
It intersects ${\cal C}$ in a pair of intersecting lines, ${\cal P}$ in a 
line, and the two Dandelin spheres in two circles.  
Therefore we have a triangle and two circles tangent to all three sides.
Let $M$ be the midpoint of $\overline{RS}$ in Figure~\ref{fig:tangent-points}.
If the intersection conic is an ellipse, the focal length is 
$|\overline{MF_1}|=|\overline{MF_2}|$ (Figure~\ref{fig:tangent-points}(a)).
If it is a hyperbola, the focal length is $|\overline{MF_3}|=|\overline{MF_4}|$
(Figure~\ref{fig:tangent-points}(b)).

     From the following classical result, we derive a simple way to compute
the focal lengths $|\overline{MF_1}|$ and $|\overline{MF_3}|$.
(For a proof of the following lemma, 
see Court~\cite{court:1925},  Davis~\cite{davis:1949} or
Johnson~\cite{johnson:1929}.)

\begin{lemma}
\label{lemma:tangent-points}
     Given a triangle $\bigtriangleup VRS$, let the incircle and the excircle
in $\angle RVS$ be tangent to $\overline{RS}$ at $F_1$ and $F_2$ respectively
(Figure~\ref{fig:tangent-points}).  Then 
$2|\overline{SF_1}|=|\overline{RS}|+(|\overline{VS}|-|\overline{VR}|)$ and
$2|\overline{SF_2}|=|\overline{RS}|+(|\overline{VR}|-|\overline{VS}|)$.  
Similarly, if the excircle in $\angle VRS$ is tangent to 
$\stackrel{\longleftrightarrow}{RS}$ at $F_3$, then
$2|\overline{SF_3}|=(|\overline{VR}|+|\overline{VS}|)-|\overline{RS}|$.
\end{lemma}
\begin{figure}
\vspace{6cm}
\caption{Tangent Points of Incircle and Excircles: (a) Ellipse (b) Hyperbola}
\label{fig:tangent-points}
\end{figure}

The following corollary shows that either focus can be chosen.

\begin{corollary}
$|\overline{SF_1}|=|\overline{RF_2}|$; and, if $F_4$ is the tangent point of
the excircle in $\angle VSR$ on $\stackrel{\longleftrightarrow}{RS}$,
$|\overline{SF_3}|=|\overline{RF_4}|$.
\end{corollary}
{\bf Proof:} Simple. \QED

We now have a way of computing the focal length $|\overline{MF_1}|$ or
$|\overline{MF_3}|$.

\begin{lemma}
\label{lemma:focal-length}
     Let $F_1,F_2,F_3$ and $F_4$ be as in Lemma~\ref{lemma:tangent-points} 
and let $M$ be the midpoint of $\overline{RS}$.
Then $|\overline{MF_1}|=|\overline{MF_2}|
=\frac{1}{2}|\ |\overline{VR}|-|\overline{VS}|\ |$ and
$|\overline{MF_3}|=|\overline{MF_4}|
=\frac{1}{2}(|\overline{VR}|+|\overline{VS}|)$.
\end{lemma}
{\bf Proof:} From Lemma~\ref{lemma:tangent-points}, 
$|\overline{SF_1}|=
\frac{1}{2}(|\overline{RS}|+(|\overline{VS}|-|\overline{VR}|))$.

If $|\overline{VR}|\geq|\overline{VS}|$, the order of $F_1$ and $F_2$ on
$\overline{RS}$ is $R,F_2,F_1,S$, that is, 
$|\overline{MF_1}|\leq|\overline{MS}|=\frac{1}{2}|\overline{RS}|$.
Therefore, $|\overline{MF_1}|=\frac{1}{2}|\overline{RS}|-
|\overline{SF_1}|=\frac{1}{2}(|\overline{VR}|-|\overline{VS}|)$.

     If $|\overline{VR}|<|\overline{VS}|$, the order becomes $R,F_1,F_2,S$
and we have $|\overline{MF_1}|=|\overline{SF_1}|-|\overline{MS}|=
|\overline{SF_1}|-\frac{1}{2}|\overline{RS}|=\frac{1}{2}(|\overline{VS}|-
|\overline{VR}|)$.  
This proves the first identity.  The second can be proven the same way.  \QED

Based on the following lemma, we will distinguish three cases.

\begin{lemma}
\label{lemma:classification}
Let ${\cal P}(\arrow{B},\arrow{O})$ be a plane and 
let ${\cal C}(\arrow{V}, \arrow{A},\alpha)$ be a right cone, 
$V \not \in P$.\footnote{$V \not \in P$ guarantees that the 
     intersection is a conic.
     The simple case $V\in{\cal P}$ will be discussed in 
     Section~\ref{section:vertex-on-plane}.}
Recall that $\theta$
is the acute angle between $\arrow{a}$ and $\arrow{o}$ and 
$\cos\theta=\arrow{a}\cdot\arrow{o}\geq 0$.

     The intersection of ${\cal P}$ and ${\cal C}$
is an ellipse, a parabola, or a hyperbola if,
respectively, $\cos\theta>\sin\alpha\ (\alpha+\theta<\frac{\pi}{2})$, 
$\cos\theta=\sin\alpha\ (\alpha+\theta=\frac{\pi}{2})$, or 
$\cos\theta<\sin\alpha\ (\alpha+\theta>\frac{\pi}{2})$.
\end{lemma}
{\bf Proof:}  Note that $\theta$ is an acute angle. 
In order to have a parabolic intersection, 
$\alpha+\theta=\frac{\pi}{2}$ must hold.  This is equivalent to
$\sin\alpha=\cos\theta$.
$\alpha+\theta<\frac{\pi}{2}$ (resp., $\alpha+\theta>\frac{\pi}{2}$)
implies the intersection curve is an ellipse (resp., a hyperbola).
If $\alpha+\theta<\frac{\pi}{2}$,
we can assume $\alpha+\theta+\epsilon=\frac{\pi}{2}$ for some suitable
$0<\epsilon<\frac{\pi}{2}$.  Then $\sin\alpha=\cos(\theta+\epsilon)<\cos\theta$
because $\cos\theta$ is a decreasing function in $[0,\pi]$.  Therefore 
the curve is an ellipse if and only if $\sin\alpha<\cos\theta$.  The proof for
the other case is the same.  \QED

% ********************************************************************
%                        ELLIPSE & HYPERBOLA
% ********************************************************************

\subsection{Ellipse ($\cos\theta>\sin\alpha$) and 
          Hyperbola ($\cos\theta<\sin\alpha$)}
\label{section:ellipse}

     Assuming $\cos\theta\neq 0$ and $\cos\theta\neq 1$, the cone's axis
always intersects the given plane at some point $\arrow{T}$ and the vector
$\arrow{m}$, which is the normalized vector of 
$\arrow{o}\times(\arrow{a}\times\arrow{o})$, is well defined and gives the 
direction of the major axis.
Let $d=|\overline{VT}|$, the distance from $\arrow{V}$
to $\arrow{T}$.

     Using this construction, we shall prove the following main theorem of this
section.

\begin{theorem}
\label{theorem:cone-ellipse}
     Let ${\cal P}(\arrow{B},\arrow{O})$ be a plane and 
${\cal C}(\arrow{V},\arrow{A},\alpha)$ a cone, such that
$\cos\theta \neq \sin\alpha$, $\cos\theta\neq 0$, $\cos\theta\neq 1$, 
and $\arrow{V}\not\in{\cal P}$.  
Let $\arrow{T}={\cal P} \cap (\arrow{V}+t\arrow{A})$ and
$d = |\overline{VT}|$.

     If $\cos\theta > \sin\alpha$, ${\cal P}\cap{\cal C}$ is an ellipse, 
otherwise it is a hyperbola.  Let $\sigma$ be the sign of 
$\cos\theta - \sin\alpha$.  That is, $\sigma=1$ if it is an ellipse; otherwise
$\sigma=-1$.  The characteristics of the intersection are:
\begin{enumerate}
     \item Semi--major axis length : 
     $m=\sigma\frac{d\sin\alpha}{2}\left[ \frac{1}{\cos(\alpha+\theta)}+
           \frac{1}{\cos(\alpha-\theta)}\right]$;
     \item Semi--minor axis length : $n=\sqrt{\sigma(m^2-f^2)}$, 
     where $f$ is the focal length \\
     $f = \sigma\frac{d\cos\theta}{2}\left[ \frac{1}{\cos(\alpha+\theta)}-
             \frac{1}{\cos(\alpha-\theta)}\right]$;
     \item Center: $\arrow{C}=\arrow{V}+d\left[\arrow{a}+
             \frac{\sin\alpha}{2}\left( \frac{1}{\cos(\alpha+\theta)}-
             \frac{1}{\cos(\alpha-\theta)}\right) \arrow{m}\right]$.
\end{enumerate}
The directions of the major and minor axes are given in 
Corollary~\ref{lemma:major-minor-axes}.
\end{theorem}
{\bf Proof:}  Let the line $\arrow{T}+t\arrow{m}$, which is the intersection of
the given plane and the axial plane, intersect the cone, along the direction 
of $\arrow{m}$, first at $\arrow{R}$ and then at $\arrow{S}$ 
(Figure~\ref{fig:ellipse}(a)).  
\ifFull
Consider the ellipse case.
\else
We shall only give the proof for the ellipse,
since the proof for the hyperbola is analogous (see \cite{JS91}).
\fi
In $\bigtriangleup VRT$, we have 
$\angle VRT=\frac{\pi}{2}-(\alpha-\theta)$,
$\angle VTR=\frac{\pi}{2}-\theta$.  The law of sines gives
\[ \frac{d}{\sin\left[\frac{\pi}{2}-(\alpha-\theta)\right]} =
   \frac{|\overline{VR}|}{\sin\left(\frac{\pi}{2}-\theta\right)} =
   \frac{|\overline{RT}|}{\sin\alpha}. \]
Hence we have
\[ |\overline{VR}|=d\frac{\cos\theta}{\cos(\alpha-\theta)}\ \ \ \ \mbox{and}
     \ \ \ \ |\overline{RT}|=d\frac{\sin\alpha}{\cos(\alpha-\theta)}. \]
Similarly, for $\bigtriangleup VST$, we have
\[ |\overline{VS}|=d\frac{\cos\theta}{\cos(\alpha+\theta)}\ \ \ \ \mbox{and}
     \ \ \ \ |\overline{ST}|=d\frac{\sin\alpha}{\cos(\alpha+\theta)}. \]
The semi--major axis length is
\[ m=\frac{1}{2}|\overline{RS}|
  =\frac{1}{2}(|\overline{RT}| + |\overline{ST}|)
  =\frac{d\sin\alpha}{2}\left[ \frac{1}{\cos(\alpha+\theta)}+
                    \frac{1}{\cos(\alpha-\theta)}\right].  \]
Recall that $\arrow{a}$ lies in the right angle between $\arrow{o}$ and 
$\arrow{m}$ (Figure~\ref{fig:ellipse}(a)).  Since $\arrow{o}$ points to the 
left of $\arrow{a}$, $\angle VTS>\angle VTR$ and 
hence $|\overline{VS}|>|\overline{VR}|$.  
Therefore the focal length (Lemma~\ref{lemma:focal-length}) is
\[ f= \frac{1}{2}(|\overline{VS}| - |\overline{VR}|)
  = \frac{d\cos\theta}{2}\left[ \frac{1}{\cos(\alpha+\theta)}-
                                 \frac{1}{\cos(\alpha-\theta)}\right]. \]

\begin{figure}
\vspace{7cm}
\caption{Computing One of the Foci: (a) Ellipse  (b) Hyperbola}
\label{fig:ellipse}
\end{figure}

     The center $\arrow{C}$ is the midpoint of $\overline{RS}$.  In order not
to calculate $\arrow{R}$ and $\arrow{S}$, we will determine $\arrow{C}$
using its relative position to $\arrow{T}$.  We will prove $\arrow{C}$ lies in
the segment $\overline{TS}$ and compute 
$|\overline{CT}|=|\arrow{C}-\arrow{T}|$.  Then $\arrow{C}$ is located 
along the direction $\arrow{m}$ with
positive distance $|\overline{CT}|$ from $\arrow{T}$.  
Since $\arrow{a}$ is the angle bisector of $\angle RVS$ and
$|\overline{VS}|>|\overline{VR}|$, we have
$\frac{|\overline{VR}|}{|\overline{VS}|}=
\frac{|\overline{TR}|}{|\overline{TS}|}$ and 
therefore $|\overline{TS}|>|\overline{TR}|$.  Since
\[     |\overline{TC}| = |\overline{RC}|-|\overline{RT}| 
                     = m-|\overline{RT}| 
                     = \frac{d\sin\alpha}{2}\left[
                         \frac{1}{\cos(\alpha+\theta)}-
                         \frac{1}{\cos(\alpha-\theta)}\right],   \]
the center $\arrow{C}$ is
\[  \arrow{C}=\arrow{T}+|\overline{TC}|\arrow{m}=\arrow{V}+d\arrow{a}+
          \frac{d\sin\alpha}{2}\left[ \frac{1}{\cos(\alpha+\theta)}-
             \frac{1}{\cos(\alpha-\theta)}\right] \arrow{m}. \]
This establishes the ellipse case.
\ifFull

The computation for the hyperbola is entirely analogous to that for the
ellipse.  Let the Dandelin sphere  be 
tangent to 
$\arrow{T}+t\arrow{m}$ at $\arrow{F}$ (Figure~\ref{fig:ellipse}(b)). 
We are going to compute $|\overline{SF}|$.  
From $\bigtriangleup RVT$, using the law of sines, we have
\[ \frac{d}{\sin\left(\alpha+\theta-\frac{\pi}{2}\right)}=
     \frac{|\overline{VR}|}{\sin\left(\frac{\pi}{2}-\theta\right)} = 
     \frac{|\overline{RT}|}{\sin(\pi-\alpha)}. \]
Hence we have
\[ |\overline{RT}|=-d\frac{\sin\alpha}{\cos(\alpha+\theta)}\ \ \ \ \mbox{and}
     \ \ \ \ |\overline{RV}|=-d\frac{\cos\theta}{\cos(\alpha+\theta)}. \]
Similarly, from $\bigtriangleup SVT$, we get
\[ |\overline{ST}|=d\frac{\sin\alpha}{\cos(\alpha-\theta)}\ \ \ \ \mbox{and}
     \ \ \ \ |\overline{SV}|=d\frac{\cos\theta}{\cos(\alpha-\theta)}. \]
The semi--major axis length is
\[    m=\frac{1}{2}|\overline{RS}|=\frac{1}{2}(|\overline{RT}|-|\overline{ST}|)
     =-\frac{d\sin\alpha}{2}\left[
                    \frac{1}{\cos(\alpha+\theta)} +
                    \frac{1}{\cos(\alpha-\theta)} \right],  \]
while the focal length (Lemma~\ref{lemma:focal-length}) is
\[     f=-\frac{d\cos\theta}{2}\left[
                    \frac{1}{\cos(\alpha+\theta)} -
                    \frac{1}{\cos(\alpha-\theta)} \right]. \]
The center $\arrow{C}$ is the midpoint of $\overline{RS}$.  
Since $|\overline{CT}|=|\overline{CS}|+|\overline{ST}|=m+|\overline{ST}|$,
we have
\[ |\overline{CT}|=-\frac{d\sin\alpha}{2}\left[
        \frac{1}{\cos(\alpha+\theta)}-\frac{1}{\cos(\alpha-\theta)}\right]. \]
Therefore, the center is
\[  \arrow{C} = \arrow{V}+d\arrow{a}-|\overline{CT}|\arrow{m} 
              = \arrow{V}+d\arrow{a}+\frac{d\sin\alpha}{2}\left[
        \frac{1}{\cos(\alpha+\theta)}-\frac{1}{\cos(\alpha-\theta)}\right]
          \arrow{m}.   \]
\fi
\QED

     Note that, if the orientation of ${\cal P}$ is fixed, $m, n, f$ and 
$\arrow{C}$ are linear functions of $d$.  Hence  $d\geq 0$ serves as a 
parameter parameterizing all parallel planar sections.

     $\cos \theta = 0$ is a special case for the hyperbola.  In this case, the
given plane does not intersect the cone's axis and thus $\arrow{T}$ is not
well defined.  However, this simplifies the computation a lot.

\begin{theorem}
\label{theorem:cone-hy-special}
     For a given plane ${\cal P}(\arrow{B},\arrow{O})$ and a given cone
${\cal C}(\arrow{V},\arrow{A},\alpha)$.  If $\cos\theta= 0$ and 
$V\not\in{\cal P}$, the line $\arrow{V}+t\arrow{O}$ intersects ${\cal P}$ at
$\arrow{V}+d\arrow{O}$ and thus $|d|$ is the distance from $\arrow{V}$ to
${\cal P}$.  The intersection conic
is a hyperbola with the following characteristics:
\begin{enumerate}
     \item Major axis direction : $\arrow{A}$;
     \item Semi--major axis length : $m=\frac{|d|}{\tan\alpha}$;
     \item Minor axis direction : $\arrow{A}\times\arrow{O}$;
     \item Semi--minor axis length : $n=|d|$;
     \item Center               : $\arrow{V}+d\arrow{O}$.
\end{enumerate}
\end{theorem}
{\bf Proof:} Let $\arrow{C}$ be $\arrow{V}+d\arrow{O}$, the intersection
point of $\arrow{V}+t\arrow{O}$ and the given plane ${\cal P}$.
     Let the line, which is the intersection of plane ${\cal P}$ and the 
axial plane,
intersect the 
cone at $\arrow{R}$ and $\arrow{S}$.  
Let the Dandelin sphere be tangent to ${\cal P}$ at 
$\arrow{F}$ as shown in Figure~\ref{fig:hy-special}.  Also let the center of
the Dandelin sphere be $\arrow{D}$.  
Note that $\arrow{C}$ is the midpoint of $\overline{RS}$, the focal length is
$f=|\overline{CF}|$, and the radius of the Dandelin sphere is
$|\overline{FD}|=|\overline{CV}|=|d|$.  Since $CVDF$ is a rectangle, we have
$f=|\overline{CF}|=|\overline{VD}|$.  Using $\bigtriangleup VDG$,
$f=|\overline{VD}|=\frac{|d|}{\sin\alpha}$.  Using $\bigtriangleup VCS$,
the semi--major axis length is $m=|\overline{CS}|=\frac{|d|}{\tan\alpha}$.
Therefore the semi--minor axis length is $n=\sqrt{f^2-m^2}=|d|$.
\QED

\begin{figure}
\vspace{5.5cm}
\caption{Computing One of the Foci of the Intersection Hyperbola--Special Case}
\label{fig:hy-special}
\end{figure}

     Again, the intersection hyperbola is parameterized linearly by $d$.
The final special case is $\cos\theta=1$ (i.e., ${\cal P}$ is
perpendicular to $\arrow{A}$).

\begin{lemma}
\label{lemma:parallel-1}
     Let ${\cal P}(\arrow{B},\arrow{O})$ be a plane that is perpendicular to
the axis of the cone ${\cal C}(\arrow{V},\arrow{A},\alpha)$.
That is, $\arrow{A} = \pm\arrow{O}$.  Let $\arrow{V}+t\arrow{A}$ intersect
${\cal P}$ at $\arrow{C}=\arrow{V}+d\arrow{A}$ (i.e., $|d| = |\overline{VC}|$).
The intersection is a circle with center $\arrow{C}$ and radius
$|d| \tan \alpha$.
\end{lemma}
{\bf Proof:} Simple. \QED

\clearpage

% ********************************************************************
%                            PARABOLA
% ********************************************************************

\subsection{Parabola ($\cos\theta=\sin\alpha$)}
\label{section:parabola}

\begin{theorem}
\label{theorem:parabola}
Let ${\cal P}(\arrow{B},\arrow{O})$ be a plane and
${\cal C}(\arrow{V},\arrow{A},\alpha)$ be a cone, such that
$\cos\theta=\sin\alpha$ and $\arrow{V}\not\in{\cal P}$.
Let $\arrow{V}+t\arrow{A}$ intersect
${\cal P}$ at $\arrow{T}=\arrow{V}+d\arrow{A}$ (i.e., $d = |\overline{VT}|$).
Then the intersection is a parabola with the following characteristics:
\begin{enumerate}
     \item Major axis direction: $\arrow{m}$;
     \item Directrix direction: $\arrow{n}$;
     \item Focal length: $f=\frac{d}{2}\frac{(1-\cos^2\alpha)}{\cos\alpha}$;
     \item Vertex: 
          $\arrow{R}=\arrow{V}+d\left[
          \arrow{a}-\frac{1}{2\cos\alpha}\right]\arrow{m}$.
\end{enumerate}
\end{theorem}     
{\bf Proof:}
Let the parabola's major axis intersect the cone at 
$\arrow{R}$ (Figure~\ref{fig:parabola}).  
The unique Dandelin sphere ${\cal S}(\arrow{D},r)$
is tangent to ${\cal P}$ at the focus $\arrow{F}$ of the parabola.
Let $d=|\overline{VT}|$.  Note that $|\overline{VD}|=\frac{d}{2}$,
and $r=|\overline{VD}|\sin\alpha=\frac{d}{2}\sin\alpha$.  
Thus, the focal length is 
$f=|\overline{RF}|=r\tan\alpha=\frac{d}{2}\frac{\sin^2\alpha}{\cos\alpha}$.
To compute the vertex of the parabola, $\arrow{R}$, 
note that $|\overline{RV}|=\frac{d}{2\cos\alpha}$.
Because $\angle RVD=\angle RTD=\alpha$, $|\overline{RT}|=|\overline{RV}|$ and 
$\arrow{R}=\arrow{T}-|\overline{RT}|\arrow{m}=
\arrow{T}-\frac{d}{2\cos\alpha}\arrow{m}$.
\QED

\begin{figure}
\vspace{5.5cm}
\caption{Computing the Focus of the Intersection Parabola}
\label{fig:parabola}
\end{figure}

The intersection parabola is again parameterized linearly by $d$.

% ********************************************************************
%                      VERTEX ON THE PLANE
% ********************************************************************

\subsection{Vertex on Plane}
\label{section:vertex-on-plane}

  The only remaining case of plane/cone intersection is $\arrow{V}\in{\cal P}$.
Depending on the position of the plane, we have a single point, a tangent line
or two intersecting lines.  Intuitively, as a plane
moves parallel toward the cone vertex, the intersection ellipse, parabola
and hyperbola reduces to a point, a line and two intersecting lines.  
Using this fact, for $\arrow{V}\in{\cal P}$, we have a point, a line or two 
intersecting lines if and only if $\cos\theta>\sin\alpha$, 
$\cos\theta=\sin\alpha$ and $\cos\theta<\sin\alpha$.   
If $\cos\theta<\sin\alpha$, the intersection point is $\arrow{V}$.
If $\cos\theta=\sin\alpha$, the line is  $\arrow{V}+t\arrow{m}$ 
(Figure~\ref{fig:parabola}).  

     The case of $\cos\theta<\sin\alpha$ is more complicated and requires 
further elaboration.  Suppose ${\cal P}$ passes through the cone's vertex 
$\arrow{V}$.  Construct an arbitrary plane ${\cal P}^\prime$ not through 
$\arrow{V}$ and parallel to ${\cal P}$.  ${\cal P}^\prime$ intersects the 
cone in a hyperbola with semi--major axis length $m$ and semi--minor axis 
length $n$ (Figure~\ref{fig:vtx-on-plane}).
The asymptotes of this hyperbola are
$\arrow{C}+t(\arrow{m}\pm\frac{n}{m}\arrow{n})$, where $\arrow{m},\arrow{n}$
and $\arrow{C}$ are the directions of the major and the minor axes and the 
center.  By a theorem from classical conic section theory,
the asymptotes are parallel to the intersection lines made by plane
${\cal P}$ (see, for example, Drew~\cite[page 119]{drew:1875}).
Therefore the intersection lines have equation
$\arrow{V}+t(\arrow{m}\pm\frac{n}{m}\arrow{n})$.
Our remaining job is to determine the value of $\frac{n}{m}$.

\begin{figure}
\vspace{4cm}
\caption{Determine the Direction of the Two Intersection Lines}
\label{fig:vtx-on-plane}
\end{figure}

     Suppose $\cos\theta\neq 0$.  Then using Theorem~\ref{theorem:cone-ellipse}
to compute the semi--major axis length
$m$ and the focal length $f$, we have:
\[   \frac{n}{m} = \sqrt{\frac{f^2}{m^2}-1}=
     \sqrt{\left(\frac{\sin\theta}{\cos\alpha}\right)^2-1}.  \]
If $\cos\theta=0$, the intersection lines lie on the plane
containing the cone's axis and hence the equation is
$\arrow{V}+t(\arrow{a}\pm\tan\alpha\arrow{n})$.

\begin{theorem}
\label{theorem:cone-line}
The intersection of the plane ${\cal P}(\arrow{B},\arrow{O})$ and the cone
${\cal C}(\arrow{V},\arrow{A},\alpha)$, where \\$\arrow{V}\in{\cal P}$,
is one of the following:
\begin{enumerate}
     \item $\cos\theta>\sin\alpha$ : the point $\arrow{V}$;
     \item $\cos\theta=\sin\alpha$ : the line $\arrow{V}+t\arrow{m}$ ;
     \item $0<\cos\theta<\sin\alpha$ : the two intersecting lines 
          $\arrow{V}+t(\arrow{m}\pm\frac{n}{m}\arrow{n})$,
  where $\frac{n}{m} = \sqrt{\left(\frac{\sin\theta}{\cos\alpha}\right)^2-1}$;
     \item $\cos\theta=0$ : the two intersecting lines 
          $\arrow{V}+t(\arrow{a}\pm\tan\alpha\arrow{n})$.
\end{enumerate}
\end{theorem}
     

% --------------------------------------------------------------------
%                     REMAINING SIMPLE CASES
% --------------------------------------------------------------------

\section{The Sphere and Cylinder}
\label{section:simple}

Plane sections of the sphere and cylinder are straightforward to compute.

\begin{lemma}
\label{lemma:sphere}
     Let ${\cal P}(\arrow{B},\arrow{O})$ be a plane and 
${\cal S}(\arrow{C},r)$ be a sphere.  Let the line $\arrow{C}+t\arrow{O}$
intersect ${\cal P}$ at $\arrow{C}+d\arrow{O}$.
If $|d| < r$, the intersection is a circle with radius $\sqrt{r^2-d^2}$ 
and center $\arrow{C}+d\arrow{O}$.
Otherwise, the intersection is empty or a point.
\end{lemma}
{\bf Proof:} Simple (Figure~\ref{fig:sphere}). \QED

\begin{figure}
\vspace{3.5cm}
\caption{Plane--sphere intersection}
\label{fig:sphere}
\end{figure}


\begin{theorem}
\label{theorem:cylinder}
Let ${\cal P}(\arrow{B},\arrow{O})$ be a plane and 
${\cal Z}(\arrow{V},\arrow{A},r)$ be a cylinder whose axis $\arrow{A}$ is
not parallel or perpendicular to the plane (i.e., $0 < \cos \theta < 1$).
The intersection is an ellipse with
\begin{enumerate}
     \item Semi--major axis length    : $\frac{r}{\cos\theta}$;
     \item Semi--minor axis length    : $r$;
     \item Center                     : 
               $\arrow{C} = {\cal P}\cap (\arrow{V}+t\arrow{A})$.
\end{enumerate}
The directions of its major and minor axes are given by 
Corollary~\ref{lemma:major-minor-axes}.
\end{theorem}
{\bf Proof:} The center must clearly lie on the axis, by symmetry.
The minor axis, through $\arrow{C}$ and normal to the axial plane 
(Figure~\ref{fig:cylinder}), must have height $r$.
From Figure~\ref{fig:cylinder}, it is clear that the major axis $\overline{RC}$
is of length $\frac{r}{\cos\theta}$.
\QED

\begin{figure}
\vspace{5cm}
\caption{Plane--cylinder intersection}
\label{fig:cylinder}
\end{figure}

\begin{lemma}
     Let ${\cal P}(\arrow{B},\arrow{O})$ be a plane 
and ${\cal Z}(\arrow{V},\arrow{A},r)$ be a cylinder whose axis is parallel
to the plane.
Let $\arrow{V}+t\arrow{O}$ intersect ${\cal P}$ at 
$\arrow{T}=\arrow{V}+d\arrow{O}$.
If $|d| \leq r$, the intersection is the lines
$\arrow{T} \pm \sqrt{r^2-d^2}(\arrow{A}\times\arrow{O})+t\arrow{A}$.
Otherwise, it is empty.

Let ${\cal P}(\arrow{B},\arrow{O})$ and ${\cal Z}(\arrow{V},\arrow{A},r)$
be a cylinder whose axis is perpendicular to the plane (i.e., 
$\arrow{A} = \pm\arrow{O}$).  Let the axis $\arrow{V}+t\arrow{A}$ intersect
${\cal P}$ at $\arrow{C}$.  The intersection is a circle with center 
$\arrow{C}$ and radius $r$.
\end{lemma}
{\bf Proof:} Simple (Figure~\ref{fig:cylinder-line}). \QED

\begin{figure}
\vspace{4.5cm}
\caption{Plane--cylinder intersection, parallel to axis}
\label{fig:cylinder-line}
\end{figure}

% --------------------------------------------------------------------
%                           CONCLUSION
% --------------------------------------------------------------------

\section{Conclusion}
\label{section:conclusion}

     Although computing the intersection of a plane and a natural quadric is
a simple task and is a widely used operation in computer-aided geometric
design, very little literature is
available.  Using the Dandelin sphere technique, we have successfully
developed a simple method to compute the intersection conic.
Moreover, except for the plane--sphere case,
the resulting equation of the intersection conic is linearly parameterized
by a parameter with simple meaning.  Hence by varying this parameter, our
algorithm delivers all parallel plane sections immediately.
We are presently investigating whether the techniques of this paper 
can be generalized to revolutionary quadrics, or even general quadrics.


% --------------------------------------------------------------------
%                           REFERENCES
% --------------------------------------------------------------------

\begin{thebibliography}{999}

\bibitem{adams:1988}
     James Alan Adams and Leon M. Billow,
     {\em Descriptive Geometry and Geometric Modeling : A Basis for Design},
     Holt, Rinehart and Winston, New York, 1988.

% \bibitem{chang:1989}
%      Chao--Hwa Chang and Michel A. Melkanoff,
%      {\em NC Machine Programming and Software Design},
%      Prentice--Hall, Englewood Cliffs, New Jersey, 1989.

\bibitem{court:1925}
     Nathan Altshiller--Court,
     {\em College Geometry},
     Johnson Publishing Co., Richmond, Virginia, 1925.

\bibitem{dandelin:1822}
     G. P. Dandelin,
     M\'{e}moire sur quelques propri\'{e}t\'{e}s remarquables de la Focal
     Parabolique,
     {\em Nouveaux M\'{e}moires de l'Acad\'{e}mie Royale des Sciences et
     Belles--lettres de Bruxelles},
     Vol. 2 (1822), pp. 171--202.

\bibitem{davis:1949}
     David R. Davis,
     {\em Modern College Geometry},
     Addison--Wesley, 1949.

\bibitem{drew:1875}
     William H. Drew,
     {\em A Geometrical Treatise on Conic Sections},
     fifth edition,
     Macmillan and Co., 1875.

\bibitem{hilbert:1983}
     D. Hilbert and S. Cohn--Vossen,
     {\em Geometry and the Imagination},
     translated by P. Nememyi,
     Chelsea, New York, 1983.

\bibitem{HOF89}
     Christoph M. Hoffmann,
     {\em Geometric and Solid Modeling},
     Morgan Kaufmann, San Mateo, California, 1989.

\bibitem{johnson:1929}
     Roger A. Johnson,
     {\em Modern Geometry},
     Houghton Mifflin Co., New York, 1929.

\bibitem{JS91}
     John K. Johnstone and C.-K. Shene,
     {\em Computing the intersection of a plane and a natural quadric},
     Technical Report 91-04, Dept. of Computer Science, Johns Hopkins 
     University, April 1991 (revised June 1991).

\bibitem{macaulay:1895}
     Francis S. Macaulay,
     {\em Geometrical Conics},
     Cambridge University Press, London, 1895.

\bibitem{MANT88}
     M. {M\"{a}ntyl\"{a}},
     {\em An Introduction to Solid Modeling},
     Computer Science Press, Rockville, Maryland, 1988.

\bibitem{MiGo91}
     James R. Miller and Ronald N. Goldman,
     Using Tangent Balls to Find Plane Sections of Natural Quadrics,
     {\em IEEE Computer Graphics and Applications}, 
     Vol. 12, No. 2, March, 1992, pp. 68--82.

\bibitem{morton:1830}
     Pierce Morton,
     On the Focus of a Conic Section,
     {\em Transactions of the Cambridge Philosophical Society},
     Vol. 3 (1827--1830), pp. 185--191.

\bibitem{salmon:1954}
     George Salmon,
     {\em A Treatise on Conic Sections},
     sixth edition,
     reprinted by Chelsea, New York, 1954.

\bibitem{shene:1991}
     Ching-Kuang Shene and John K. Johnstone,
     On the Planar Intersection of Natural Quadrics,
     {\em ACM Symposium on Solid Modeling Foundations and
     CAD/CAM Applications}, 
     June, 1991, pp. 233--242.

\end{thebibliography}

\end{document}
