\documentstyle [12pt]{article}

\newtheorem{example}{Example}[section]
\newtheorem{property}{Property}[section]
\newtheorem{definition}{Definition}[section]
\newtheorem{theorem}{Theorem}[section]
\newtheorem{lemma}{Lemma}[section]
\newtheorem{corollary}{Corollary}[section]

\newcommand{\DoubleSpace}{\edef\baselinestretch{1.4}\Large\normalsize}
\newcommand{\QED}{\ \ \ \rule{2mm}{3mm}\\}
\newcommand{\arrow}[1]{\vec{\bf #1}}

%\DoubleSpace
\setlength{\oddsidemargin}{0pt}
\setlength{\evensidemargin}{0pt}
\setlength{\headsep}{0pt}
\setlength{\topmargin}{0pt}
\setlength{\textheight}{8.75in}
\setlength{\textwidth}{6.5in}


\begin{document}

\def\thefootnote{\arabic{footnote}}
\setcounter{footnote}{0}

     A central conic (i.e., an ellipse or hyperbola) 
is uniquely determined by the directions of
its major and minor axes, the lengths of its semi--major and semi--minor 
axes, and its center.
In this paper, the length of the semi--minor axis, $n$, is computed 
in terms of the length of the semi--major axis, $m$, and the focal length,
$f$.
For an ellipse, since $f=\sqrt{m^2-n^2}$, 
the semi--minor axis length is $n=\sqrt{m^2-f^2}$;  
for a hyperbola, the focal length is
$f=\sqrt{m^2+n^2}$ and thus $n=\sqrt{f^2-m^2}$.
A parabola is uniquely determined by its directrix, vertex, and focal
length.

{\bf Notation:}
A plane through
$\arrow{B}$ and perpendicular to $\arrow{O}$ (its orientation) is denoted by
${\cal P}(\arrow{B},\arrow{O})$, the sphere with center $\arrow{C}$ and radius
$r$ by ${\cal S}(\arrow{C},r)$, the cylinder with axis $\arrow{V}+t\arrow{A}$
and radius $r$ by ${\cal C}(\arrow{V},\arrow{A},r)$, and the right cone
with vertex $\arrow{V}$, axis $\arrow{V}+t\arrow{A}$ and cone angle $\alpha$
by ${\cal C}(\arrow{V},\arrow{A},\alpha)$.  Note that vectors $\arrow{u},
\arrow{v}, \arrow{O}$ and $\arrow{A}$ give directions, not positions.
We assume that all vectors are of unit length.

% --------------------------------------------------------------------
%             DIRECTIONS OF THE MAJOR AND MINOR AXES
% --------------------------------------------------------------------

\section{Directions of the Major and Minor Axes}
\label{section:determining-axes}

A result in Shene and Johnstone~\cite{shene:1991} 
yields a simple way to determine the direction of the 
major and minor axis of the intersection conic.  

\begin{lemma}
\label{lemma:perpendicular-axis}
    Let ${\cal P}$ be a plane intersecting a cone/cylinder in a conic 
$C$.  Then the plane ${\cal E}$ determined by the major axis of $C$ and the 
axis of the natural quadric is perpendicular to ${\cal P}$.
\end{lemma}

\begin{corollary}
\label{lemma:major-minor-axes}
    Let ${\cal P}(\arrow{B},\arrow{O})$ be a plane 
intersecting a cone/cylinder ${\cal Q}$ (with axis $\arrow{Q}+t\arrow{A}$,
$\arrow{A}\neq\arrow{O}$) in a conic $C$.
The directions of the minor
and major axes are $\arrow{A}\times\arrow{O}$ and 
$\arrow{O}\times(\arrow{A}\times\arrow{O})$, respectively.  
\end{corollary}
{\bf Proof:}
The minor axis is perpendicular to the major axis and thus perpendicular
to ${\cal E}$ of Lemma~\ref{lemma:perpendicular-axis}.
Thus, it is perpendicular to the axis $\arrow{A}$ of the natural quadric.
Since the minor axis of $C$ lies in ${\cal P}$,
it is perpendicular to $\arrow{O}$.
\QED

Note that the cross products in
Corollary~\ref{lemma:major-minor-axes} are well defined, because the case
$\arrow{A} \neq \arrow{O}$ is dealt with separately in 
Section~\ref{sec:special}.
The plane ${\cal E}$ of Lemma~\ref{lemma:perpendicular-axis} is important.
Since it contains the cone's axis and the conic's major axis,
we call it the {\bf axial plane}.
The axial plane is trivially computed, since its normal is the minor axis 
$\arrow{A}\times\arrow{O}$ (the axial plane contains $\arrow{A}$
and is perpendicular to ${\cal P}$).
All computations of the intersection conic will be performed in this plane.
That is, we can compute the major axis and its length, the focal length,
and the center or vertex, all while restricting to the axial plane,
since the axial plane contains the major axis and (thus) the foci and center.

\subsection{Choosing the sign of the direction}

There are two choices for the directions of the major and minor axes:
$\pm \arrow{A}\times\arrow{O}$ and
$\pm \arrow{O}\times(\arrow{A}\times\arrow{O})$.
The choice is important because we will use the angle between the 
cone/cylinder axis and the plane's normal 
(called $\theta$ below) in calculations, 
and this angle is affected by the choice of direction.
We make the choice as follows.

     Assuming it exists, let $\arrow{T}$ be the intersection of ${\cal P}$ 
and the axis of the natural quadric: 
$\arrow{T}={\cal P}\cap \arrow{V}+t\arrow{A}$ for some suitable $t$.  
Define $\arrow{a}$ as follows:
\[ \arrow{a} = \left\{
                    \begin{array}{lll}
                         \arrow{A} & & \mbox{if $t\geq 0$,} \\
                         -\arrow{A} & & \mbox{if $t< 0$.}
                    \end{array} \right. \]
We will use $\arrow{a}$ instead of $\arrow{A}$ in all of our calculations.

     Consider the orientation vector $\arrow{O}$ of plane ${\cal P}$.
The inner product $\arrow{a}\cdot\arrow{O}$ gives the cosine of the smaller
angle $\phi$ between $\arrow{a}$ and $\arrow{O}$.  $\phi$ is an acute angle
if and only if $\arrow{a}\cdot\arrow{O}>0$.  Let
\[ \arrow{o} = \left\{
                    \begin{array}{lll}
                     \arrow{O} & & \mbox{if $\arrow{a}\cdot\arrow{O}\geq 0$,}\\
                     -\arrow{O} & & \mbox{if $\arrow{a}\cdot\arrow{O}< 0$.}
                    \end{array} \right. \]
We will use $\arrow{o}$ instead of $\arrow{O}$ in all of our calculations.
Note that $\arrow{a}$ always points toward the plane ${\cal P}$ and 
the angle $\theta$ between $\arrow{a}$ and $\arrow{o}$ is always
an acute angle with $\cos\theta=\arrow{a}\cdot\arrow{o}\geq 0$.

\section{Plane/Cone Intersection}
\label{section:cone}

\subsection{How to compute the focal length}

Consider the intersection of a plane with a right cone, which
is the most difficult case.
In the last century, Dandelin~\cite{dandelin:1822} discovered that if a 
sphere is inscribed in the cone and tangent to the plane, the tangent point 
on the plane is one of the foci of the intersection conic 
(Figure~\ref{dandelin-sphere}).\footnote{The foci--directrix--eccentricity 
	relation was found by Morton~\cite{morton:1830} a few years later.
	For other properties derived from the Dandelin
	sphere and a proof of Dandelin's theorem in English,
	the reader is referred to
	Drew~\cite{drew:1875}, Hilbert and Cohn--Vossen~\cite{hilbert:1983}
	and Macaulay~\cite{macaulay:1895}.}
This sphere is called a {\em Dandelin} (or {\em focal}) sphere. 
We will use it to compute the focal length of the intersection.
\begin{figure}
\vspace{4.5cm}
\caption{The Dandelin Sphere}
\label{dandelin-sphere}
\end{figure}

Let plane ${\cal P}$ intersect cone ${\cal C}$ in a central conic.  
(The parabola is treated separately in Section~\ref{section:parabola}.)
There will be one Dandelin sphere tangent to ${\cal P}$ at each focus point.  
Consider the axial plane, as shown in Figure~\ref{fig:tangent-points}.
It intersects ${\cal C}$ in a pair of intersecting lines, ${\cal P}$ in a 
line, and the two Dandelin spheres in two circles.  
Therefore we have a triangle and two circles tangent to all three sides.
Let $M$ be the midpoint of $BC$ in Figure~\ref{fig:tangent-points}.
If the intersection conic is an ellipse, 
the focal length is $|\overline{MT_1}|$ (Figure~\ref{fig:tangent-points}(a)).
If it is a hyperbola, 
the focal length is $|\overline{MT_3}|$ (Figure~\ref{fig:tangent-points}(b)).

From the following classical result, we derive a simple way to compute
the focal lengths $|\overline{MT_1}|$ and $|\overline{MT_3}|$.
(For a proof of the following lemma, 
see Court~\cite{court:1925},  Davis~\cite{davis:1949} or
Johnson~\cite{johnson:1929}.)

\begin{lemma}
\label{lemma:tangent-points}
     Given a triangle $\bigtriangleup ABC$, let the incircle and the excircle
in $\angle BAC$ be tangent to $\overline{BC}$ at $T_1$ and $T_2$ respectively
(Figure~\ref{fig:tangent-points}).  Then 
$2|\overline{CT_1}|=|\overline{BC}|+(|\overline{AC}|-|\overline{AB}|)$ and
$2|\overline{CT_2}|=|\overline{BC}|+(|\overline{AB}|-|\overline{AC}|)$.  
Similarly, if the excircle in $\angle ABC$ is tangent to 
$\stackrel{\longleftrightarrow}{BC}$ at $T_3$, then
$2|\overline{CT_3}|=|\overline{AC}|+(|\overline{AB}|-|\overline{BC}|)$.
\end{lemma}
\begin{figure}
\vspace{6cm}
\caption{Tangent Points of Incircle and Excircles: (a) Ellipse (b) Hyperbola}
\label{fig:tangent-points}
\end{figure}

% The following corollary shows that either focus can be chosen.

\begin{corollary}
$|\overline{CT_1}|=|\overline{BT_2}|$; and, if $T_4$ is the tangent point of
the excircle in $\angle ACB$ on $\stackrel{\longleftrightarrow}{BC}$,
$|\overline{CT_3}|=|\overline{BT_4}|$.
\end{corollary}
{\bf Proof:} Simple. \QED

\noindent We now have a way of computing the focal length, as follows.

\begin{lemma}
\label{lemma:focal-length}
     Let $T_1,T_2,T_3$ and $T_4$ be as in Lemma~\ref{lemma:tangent-points} 
and let $M$ be the midpoint of $\overline{BC}$.
Then $|\overline{MT_1}|=|\overline{MT_2}|
=\frac{1}{2}|\ |\overline{AB}|-|\overline{AC}|\ |$ and
$|\overline{MT_3}|=|\overline{MT_4}|
=\frac{1}{2}(|\overline{AB}|+|\overline{AC}|)$.
\end{lemma}
{\bf Proof:} From Lemma~\ref{lemma:tangent-points}, 
$|\overline{CT_1}|=
\frac{1}{2}(|\overline{BC}|+(|\overline{AC}|-|\overline{AB}|))$.

If $|\overline{AB}|\geq|\overline{AC}|$, the order of $T_1$ and $T_2$ on
$\overline{BC}$ is $B,T_2,T_1,C$, that is, 
$|\overline{MT_1}|\leq|\overline{MC}|=\frac{1}{2}|\overline{BC}|$.
Therefore, $|\overline{MT_1}|=\frac{1}{2}|\overline{BC}|-
|\overline{CT_1}|=\frac{1}{2}(|\overline{AB}|-|\overline{AC}|)$.

     If $|\overline{AB}|<|\overline{AC}|$, the order becomes $B,T_1,T_2,C$
and we have $|\overline{MT_1}|=|\overline{CT_1}|-|\overline{MC}|=
|\overline{CT_1}|-\frac{1}{2}|\overline{BC}|=\frac{1}{2}(|\overline{AC}|-
|\overline{AB}|)$.  
This proves the first identity.  The second can be proven the same way.  \QED

Based on the following lemma, we will distinguish three cases.

\begin{lemma}
\label{lemma:classification}
Let ${\cal P}(\arrow{B},\arrow{O})$ be a plane and 
let ${\cal C}(\arrow{V}, \arrow{A},\alpha)$ be a right cone, 
$V \not \in P$.\footnote{$V \not \in P$ guarantees that the 
	intersection is a conic.
	The simple case $V\in{\cal P}$ will be discussed in 
	Section~\ref{section:vertex-on-plane}.}
Recall that $\theta$
is the acute angle between $\arrow{a}$ and $\arrow{o}$ and 
$\cos\theta=\arrow{a}\cdot\arrow{o}\geq 0$.

     The intersection of ${\cal P}$ and ${\cal C}$
is an ellipse, a parabola, or a hyperbola if,
respectively, $\cos\theta>\sin\alpha$, $\cos\theta=\sin\alpha$, or 
$\cos\theta<\sin\alpha$.
\end{lemma}
{\bf Proof:}  Note that $\theta$ is an acute angle. 
In order to have a parabolic intersection, 
$\alpha+\theta=\frac{\pi}{2}$ must hold.  This is equivalent to
$\sin\alpha=\cos\theta$.
$\alpha+\theta<\frac{\pi}{2}$ (resp., $\alpha+\theta>\frac{\pi}{2}$)
implies the intersection curve is an ellipse (resp., a hyperbola).
If $\alpha+\theta<\frac{\pi}{2}$,
we can assume $\alpha+\theta+\epsilon=\frac{\pi}{2}$ for some suitable
$0<\epsilon<\frac{\pi}{2}$.  Then $\sin\alpha=\cos(\theta+\epsilon)<\cos\theta$
because $\cos\theta$ is a decreasing function in $[0,\pi]$.  Therefore 
the curve is an ellipse if and only if $\sin\alpha<\cos\theta$.
\QED

\subsection{Ellipse ($\cos\theta>\sin\alpha$)}
\label{section:ellipse}

Let $\arrow{T}$ be the intersection of the plane with the cone's axis.
Let the line $\arrow{T}+t\arrow{o}\times(\arrow{a}\times\arrow{o})$
intersect the cone, along the direction 
$\arrow{o}\times(\arrow{a}\times\arrow{o})$, first at $\arrow{R}$ and then at
$\arrow{S}$ (Figure~\ref{fig:ellipse}).  
Let $d=|\overline{VT}|$.
In $\bigtriangleup VRT$, we have $\angle VRT=\frac{\pi}{2}-(\alpha-\theta)$,
$\angle VTR=\frac{\pi}{2}-\theta$.  The law of sines gives
\[ \frac{d}{\sin\left[\frac{\pi}{2}-(\alpha-\theta)\right]} =
   \frac{|\overline{VR}|}{\sin\left(\frac{\pi}{2}-\theta\right)} =
   \frac{|\overline{RT}|}{\sin\alpha}. \]
Hence we have
\[ |\overline{VR}|=d\frac{\cos\theta}{\cos(\alpha-\theta)}\ \ \ \ \mbox{and}
     \ \ \ \ |\overline{RT}|=d\frac{\sin\alpha}{\cos(\alpha-\theta)}. \]
Similarly, for $\bigtriangleup VST$, we have
\[ |\overline{VS}|=d\frac{\cos\theta}{\cos(\alpha+\theta)}\ \ \ \ \mbox{and}
     \ \ \ \ |\overline{ST}|=d\frac{\sin\alpha}{\cos(\alpha+\theta)}. \]
The semi--major axis length is
\begin{equation}
\label{eqn:RS}
 m=\frac{1}{2}|\overline{RS}|
  =\frac{1}{2}(|\overline{RT}| + |\overline{ST}|)
  =\frac{d\sin\alpha}{2}\left[ \frac{1}{\cos(\alpha+\theta)}+
                    \frac{1}{\cos(\alpha-\theta)}\right],
\end{equation}
and the focal length (Lemma~\ref{lemma:focal-length}) is
\begin{equation}
\label{eqn:f}
 f= \frac{1}{2}(|\overline{VS}| - |\overline{VR}|)
  = \frac{d\cos\theta}{2}\left[ \frac{1}{\cos(\alpha+\theta)}-
                                 \frac{1}{\cos(\alpha-\theta)}\right].
\end{equation}

\begin{figure}
\vspace{5cm}
\caption{Computing One of the Foci of the Intersection Ellipse}
\label{fig:ellipse}
\end{figure}

     Finally, the center $\arrow{C}$ of the ellipse is
the midpoint of $\overline{RS}$:
\begin{equation}
\label{eqn:center}
     \arrow{C}=\arrow{R}+\frac{1}{2}(\arrow{S}-\arrow{R}).
\end{equation}

\begin{theorem}
\label{theorem:cone-ellipse}
Let ${\cal P}(\arrow{B},\arrow{O})$ be a plane and 
${\cal C}(\arrow{V},\arrow{A},\alpha)$ a cone, such that
$\cos\theta>\sin\alpha$, $\cos\theta\neq 1$, and $\arrow{V}\not\in{\cal P}$.
Let $T := {\cal P} \cap \arrow{A}$,
$d := {\rm dist}(V,T)$,
$\cos \theta := a \cdot o$, and
let $R$ and $S$ be the intersections of the plane with the two lines 
of the cone in the axial plane.
The intersection of ${\cal P}$ and ${\cal C}$
is an ellipse with the following characteristics:
\begin{enumerate}
     \item Semi--major axis length : 
	$m=\frac{d\sin\alpha}{2}\left[ \frac{1}{\cos(\alpha+\theta)}+
           \frac{1}{\cos(\alpha-\theta)}\right]$;
     \item Semi--minor axis length : $n=\sqrt{m^2-f^2}$, 
	where $f$ is the focal length \\
	$f = \frac{d\cos\theta}{2}\left[ \frac{1}{\cos(\alpha+\theta)}-
             \frac{1}{\cos(\alpha-\theta)}\right]$;
     \item Center: $\arrow{C}=\arrow{R}+\frac{1}{2}(\arrow{S}-\arrow{R})$.
\end{enumerate}
The directions of the major and minor axes are given in 
Corollary~\ref{lemma:major-minor-axes}.
\end{theorem}

     Note that, if the orientation of ${\cal P}$ is fixed, $m, f$ and $n$
are linear functions of $d$.  Hence  $d\geq 0$ serves as a parameter
parameterizing all parallel planar sections.

\begin{thebibliography}{999}

\bibitem{court:1925}
     Nathan Altshiller--Court,
     {\em College Geometry},
     Johnson Publishing Co., Richmond, Virginia, 1925.

\bibitem{dandelin:1822}
     G. P. Dandelin,
     M\'{e}moire sur quelques propri\'{e}t\'{e}s remarquables de la Focal
     Parabolique,
     {\em Nouveaux M\'{e}moires de l'Acad\'{e}mie Royale des Sciences et
     Belles--lettres de Bruxelles},
     Vol. 2 (1822), pp. 171--202.

\bibitem{davis:1949}
     David R. Davis,
     {\em Modern College Geometry},
     Addison--Wesley, 1949.

\bibitem{davis:1967}
     Harry F. Davis,
     {\em Introduction to Vector Analysis},
     second edition,
     Allyn and Bacon, Boston, 1967.

\bibitem{drew:1875}
     William H. Drew,
     {\em A Geometrical Treatise on Conic Sections},
     fifth edition,
     Macmillan and Co., 1875.

\bibitem{hilbert:1983}
     D. Hilbert and S. Cohn--Vossen,
     {\em Geometry and the Imagination},
     translated by P. Nememyi,
     Chelsea, New York, 1983.

\bibitem{johnson:1929}
     Roger A. Johnson,
     {\em Modern Geometry},
     Houghton Mifflin Co., New York, 1929.

\bibitem{macaulay:1895}
     Francis S. Macaulay,
     {\em Geometrical Conics},
     Cambridge University Press, London, 1895.

\bibitem{morton:1830}
     Pierce Morton,
     On the Focus of a Conic Section,
     {\em Transactions of the Cambridge Philosophical Society},
     Vol. 3 (1827--1830), pp. 185--191.

\bibitem{shene:1991}
     Ching-Kuang Shene and John K. Johnstone,
     On the Planar Intersection of Natural Quadrics,
     to appear in {\em ACM Symposium on Solid Modeling Foundations and
     CAD/CAM Applications}, June, 1991.

\end{thebibliography}

\end{document}
