%
% Authors:      Joon-Kyung Seong, Gershon Elber, John K.~Johnstone, 
%                    and Myung-Soo Kim
%               Computer Science Dept.
%               SNU/Technion/UAB
% Date: October 23 2003
%

\documentclass[11pt]{article}          % LaTeX 2e

\usepackage{epsfig}

\setlength{\oddsidemargin}{0.10in}    % was -0.4
\setlength{\evensidemargin}{0.10in}   % was -0.4
% \setlength{\headsep}{-0.43in}       % commented out
\setlength{\topmargin}{0in}           % was 0.3
\setlength{\textheight}{9.0in}        % was 9.5
\setlength{\textwidth}{6.5in}
\setlength{\parskip}{2.0mm}                  % added
\setlength{\baselineskip}{1.7\baselineskip}  % added

\newtheorem{corollaryenv}{Corollary}
\newenvironment{corollary}{\begin{quote}\begin{corollaryenv}}%
                           {\end{corollaryenv}\end{quote}}

\newtheorem{definitionenv}{Definition}
\newenvironment{definition}{\begin{quote}\begin{definitionenv}}%
                           {\end{definitionenv}\end{quote}}

\newtheorem{lemmaenv}{Lemma}
\newenvironment{lemma}{\begin{quote}\begin{lemmaenv}}%
                           {\end{lemmaenv}\end{quote}}

\newtheorem{algenv}{Algorithm}[section]
\newenvironment{algorithm}{\begin{quote}\begin{algenv}}%
                          {\end{algenv}\end{quote}}

\newenvironment{Algorithm}[2]%
        {\begin{table}[#1]\renewcommand{\baselinestretch}{1.0}%
        \begin{algorithm}\begin{tt}\label{#2}\begin{tabbing}%
        \hspace{1em}\=\hspace{1em}\=\hspace{1em}\=\hspace{1em}\=%
        \hspace{1em}\=\hspace{1em}\=\hspace{1em}\=\hspace{1em}\=%
        \hspace{1em}\=\hspace{1em}\=\hspace{1em}\=\\}%
        {\end{tabbing}\end{tt}\end{algorithm}%
        \renewcommand{\baselinestretch}{1.0}\end{table}}

%
% Allow more freedom for Figures.
%
\renewcommand{\topfraction}{0.90}
\renewcommand{\bottomfraction}{0.90}
\renewcommand{\textfraction}{0.05}
\renewcommand{\floatpagefraction}{0.90}
\renewcommand{\dbltopfraction}{0.90}
\renewcommand{\dblfloatpagefraction}{0.90}
\setcounter{topnumber}{4}
\setcounter{bottomnumber}{2}
\setcounter{totalnumber}{4}

\def\binom#1#2{{#1\choose #2}}
\newcommand{\choosein}[2]{\binom{#2}{#1}}
\newcommand{\Bezier}{B\'{e}zier}
\newcommand{\NURB}{NURBS}
\newcommand{\Bspline}{B-spline}
%\newcommand{\prbox}{\hfill\rule{1.5ex}{1.5ex}}
\newcommand{\prbox}{\rule{1.5ex}{1.5ex}}
\newcommand{\eqnref}[1]{(\ref{#1})}
\newcommand{\eqnrange}[2]{(\ref{#1}--\ref{#2})}
\newcommand{\CH}{{\cal C}{\cal H}}
\newcommand{\inner}[2]{\left<{#1}, {#2} \right>}
\newcommand{\twovert}[2]{\begin{array}{c} \mbox{\vspace{0.07in}} \\
                                          \mbox{\normalsize ${#1}$} \\[-0.05in]
                                          \mbox{\small ${#2}$}
                         \end{array}}
\newcommand{\ddcrv}{\frac{d^2C(t)}{dt^2}}
\newcommand{\dcrv}{\frac{dC(t)}{dt}}
\newcommand{\ducrv}{\frac{dC_1(u)}{du}}
\newcommand{\dvcrv}{\frac{dC_2(v)}{dv}}
\newcommand{\dSdu}{\frac{\partial S(u, v)}{\partial u}}
\newcommand{\dSxdu}{\frac{\partial s_x(u, v)}{\partial u}}
\newcommand{\dSydu}{\frac{\partial s_y(u, v)}{\partial u}}
\newcommand{\dSzdu}{\frac{\partial s_z(u, v)}{\partial u}}
\newcommand{\dSdv}{\frac{\partial S(u, v)}{\partial v}}
\newcommand{\dSxdv}{\frac{\partial s_x(u, v)}{\partial v}}
\newcommand{\dSydv}{\frac{\partial s_y(u, v)}{\partial v}}
\newcommand{\dSzdv}{\frac{\partial s_z(u, v)}{\partial v}}
\newcommand{\aprxsrf}[2]{({#1}$\leadsto${#2})}
\newcommand{\Reals}{\mbox{\normalsize{\sl I\mbox{\hspace{-0.03in}}R}}}
\newcommand{\vertspace}{\mbox{\LARGE $^{}_{}$ \hspace{-2mm}}}

%%%%%%%%%%%%%%%%%%%%%%%%%%%%%%%%%%%%%%%%%%%%%%%%%%%%%%%%%%%%%%%%%%%%%%%%%%%%%
\begin{document}

\title{\bf The Convex Hull of Freeform Surfaces}

\author{\begin{tabular}{cccc}
           {\em Joon-Kyung Seong} & {\em Gershon Elber} & 
              {\em John K.~Johnstone} & {\em Myung-Soo Kim} \\
           {\small SNU, Korea} & 
              {\small Technion, Israel} &
              {\small UAB, USA} &
              {\small SNU, Korea} \\
%           {\small Seoul National University} & 
%              {\small Technion, Israel Inst.\ of Technology} & 
%              {\small University of Alabama at Birmingham} &
%              {\small Seoul National University} \\
%           {\small Seoul 151-742, South Korea} & 
%              {\small Haifa 32000, Israel} & 
%              {\small Birmingham, AL 35294-1170} &
%              {\small Seoul 151-742, South Korea} \\
           {\small {\tt swallow@3map.snu.ac.kr}} & 
              {\small {\tt gershon@cs.technion.ac.il}} & 
              {\small {\tt jj@cis.uab.edu}} & 
              {\small {\tt mskim@cse.snu.ac.kr}}
        \end{tabular}}

\date{}

%
% Double spacing this time.
%
% \setlength{\baselineskip}{1.0\baselineskip}
%

\maketitle

\begin{abstract}
\noindent
   We present an algorithm for computing the convex hull of
   freeform rational surfaces.  The convex hull problem is reformulated
   as one of finding the zero-sets of polynomial equations;
   using these zero-sets
   we characterize developable surface patches and planar patches
   that belong to the boundary of the convex hull.
\end{abstract}

\noindent
{\bf Key Words:}
Convex hull, common tangent, zero-set finding,
freeform rational surface, B-spline, symbolic computation


%%%%%%%%%%%%%%%%%%%%%%%%%%%%%%%%%%%%%%%%%%%%%%%%%%%%%%%%%%%%%%%%%%%%%%%%%%%%%

\section{Introduction}

Computing the convex hull of a freeform surface
is a challenging task in geometric modeling.
Because of the difficulty in computing the exact convex hull
of a spline surface, the convex hull of its control points is usually
used as a simple, yet rough, approximation to the convex hull.
When a tighter bound is needed for the convex hull, we can subdivide
the surface into smaller pieces and then union the convex hulls of
the control points of these small pieces. 
This simple approach will require a large number of subdivisions 
until the approximating convex hull converges to the exact convex hull
within a certain reasonable bound.  In this paper, we present
an algorithm that computes the convex hull of a rational surface
without resorting to a polygonal approximation of the given surface
or the subdivisions.

The convex hull computation has applications in many important
geometric problems such as interference checking in motion planning
and object culling in graphics rendering.  Interference between two convex
objects is considerably easier to test than between two non-convex objects.
Consequently, convex hulls of general non-convex objects (or their rough
approximations such as spheres and axis-aligned bounding boxes) are often used
in a simple first test for checking the interference between the two
original objects.   (If there is no interference between the convex hulls,
it is guaranteed that the original objects have no interference.)
These applications motivated the development of many efficient
convex hull algorithms in computational geometry~\cite{Prep85}.
The previous work, however, has been mostly limited to computing
the convex hull of discrete points, polygons, and
polyhedra~\cite{Graham83,Lee83,Prep85}.

There are a few previous algorithms that can compute the convex hulls
of rational curves in the plane.  Some of these algorithms are quite
theoretical~\cite{Kim91,Souvaine90,Schaeffer87} in the sense that
there are certain delicate issues that must be resolved for the efficiency
and robustness of these algorithms to hold in actual implementation.
More practical approaches were taken in two recent results reported
by Elber et al.~\cite{Elber2001b} and Johnstone~\cite{Johnstone2001}.
The convex hull algorithm we present in this paper is based on
extending our recent result~\cite{Elber2001b} to the three-dimensional case.

Given a point $S(u,v)$ on a rational surface, let $T(u,v)$ denote
the tangent plane of the surface at the point $S(u,v)$.
For the point $S(u,v)$ to be on the boundary of the convex hull,
the surface $S$ should be completely contained in one side of
the tangent plane $T(u,v)$.  Thus the Gaussian curvature of $S(u,v)$ 
must be non-negative.  For the sake of simplicity of presentation,
we assume that the Gaussian curvature is positive
on this surface point $S(u,v)$ and moreover the surface $S$ contains
no line passing through the point $S(u,v)$).
% which means
%that the tangent plane $T(u,v)$ intersects the surface $S$
%at no other point $S(s,t)$, for $(s,t) \neq (u,v)$.

Let $D$ denote the region in the $uv$-domain where the surface
patch $S(u,v)$ belongs to the boundary of the convex hull.
In this paper, we show that the boundary of this region $D$
can be computed in terms of the zero-set of three equations in
four variables $u,v,s,t$, and sometimes in terms of the zero-set of
two equations in three variables $u,v,t$.  The boundary curves
of the surface $S$ may also contribute to the convex hull of $S$.
Let $I$ denote the parameter interval for a boundary curve segment
that appears on the boundary of the convex hull of the surface $S$.
The end points of the interval $I$ can be computed
by solving three equations in three variables,
and sometimes by solving two equations in two variables.
The boundary of the convex hull may contain some triangles,
each of which is obtained from a tangent plane touching at three 
different points of the surface.  These planes can be computed
by solving a system of six equations in six variables.

The rest of this paper is organized as follows.
In Section~2, we reduce the convex hull problem to
computing the zero-sets of polynomial equations.
The convex hull of rational space curves is needed
to deal with the boundary curves of a rational surface.
This is an important problem by itself;
Section~3 thus presents an algorithm for computing
the convex hull of a rational space curve.
Section~4 computes tritangent planes by solving
a system of six equations in six variables.  
Section~5 discusses how to combine all these components
to construct the boundary of the convex hull.
Finally, in Section~6, we conclude this paper.

%%%%%%%%%%%%%%%%%%%%%%%%%%%%%%%%%%%%%%%%%%%%%%%%%%%%%%%%%%%%%%%%%%%%%%%%%%%%%

\section{The Convex Hull of Freeform Surfaces}
\label{sec-ch-surface}

In this section, we consider the convex hull of freeform surfaces.
Although we will use one non-convex surface in the following discussion
for clarity,  one may consider the second surface point $S(s,t)$
as a point on another surface.

Let $S(u,v)$ be a regular $C^1$-continuous rational surface. 
Consider the tangent plane of $S$ at $S(u,v)$ as a moving plane
while continuously touching the surface tangentially.
Then, any surface point $S(u,v)$
such that the surface is completely contained in one side of
the tangent plane is on the boundary of the convex hull of
the surface $S$. On the other hand, if the tangent plane at $S(u,v)$ 
intersects the surface at any other surface point $S(s,t)$, then
the surface point $S(u,v)$ cannot be on the boundary of
the convex hull of the surface $S$.
Note, however, that one must be careful of common bitangent planes because 
a surface point at which the tangent plane is also tangent to some
other surface point could be on the boundary of the convex hull. 

The tangent plane of $S$ at $S(u,v)$ contains
another surface point $S(s,t)$ if and only if 
\begin{eqnarray*}
   {\mathcal F}(u,v,s,t) &=& \inner{S(u,v)-S(s,t)}{N(u,v)}\\
   &=& |S(u,v)-S(s,t)\ \ S_u(u,v)\ \ S_v(u,v)|\\
   &=& 0,
\end{eqnarray*}
where $N(u,v) = S_u(u,v) \times S_v(u,v)$ is the normal vector field 
of the surface $S$ and $| \cdot |$ denotes the determinant of
a matrix consisting of three column vectors.
Then, the set of surface points, for which 
the tangent plane intersects the surface $S$ at no other points, 
is defined as follows:
\begin{eqnarray*}
\CH{}^o(S) = \left\{~S(u,v)~\left|~ 
{\cal F}(u,v,s,t) \neq 0,~\forall (s,t) \neq (u,v) \right. \right\}.
\end{eqnarray*}
This set $\CH{}^o(S)$ is clearly a subset of the boundary of the convex hull
of $S$ and it is also a subset of the surface $S$ itself:
\begin{equation}
    \CH{}^o(S) \subset \CH{}(S) \cap S.
\label{eqn-cho-ch}
\end{equation}
The difference $(\CH{}(S) \cap S) \setminus \CH{}^o(S)$ contains some extra
points such as: (i) the boundary curve of each connected surface patch of
$\CH{}^o(S)$ and, sometimes, (ii) the boundary curve of $S$ itself if $S$ is 
not a closed surface.  (For now, we assume that each tangent plane of $S$ is
tangent to $S(u,v)$ at no more than two surface points;
thus all isolated points and isolated curve segments of
$\CH{}(S) \cap S$ must be on the boundary curve of $S$.)

As mentioned above, the zero-set of 
${\mathcal F}(u,v,s,t) = 0 \wedge (u,v) \neq (s,t)$ in
the $uvst$-domain cannot contribute to the boundary of the convex hull
of the surface $S(u,v)$. That is, if the point $(u,v)$ in the parametric
domain falls into the projection of this zero-set, then the corresponding
surface point $S(u,v)$ cannot be on the boundary of the convex hull 
of $S$. The boundary of the `uncovered' region of the $uv$-plane
(under this projection) is characterized as the projection of the 
$st$-silhouette curves (along the $st$-direction) of the zero-set. This means
that the $s$-partial derivative and the $t$-partial derivative must 
simultaneously vanish along the silhouette curve, which can be 
characterized as the
intersection of the following three hypersurfaces in the $uvst$-space:
\begin{eqnarray}
   {\cal F}(u,v,s,t) &=& 0, \label{eqn-f-uvst} \\
   {\cal F}_s(u,v,s,t) &=& 0, \label{eqn-s-partial}\\
   {\cal F}_t(u,v,s,t) &=& 0. \label{eqn-t-partial}
\end{eqnarray}
One should consider only the solutions satisfying $(u,v) \neq (s,t)$
from the above equations. Furthermore, if $(u,v)$ and $(s,t)$ are
on the same surface, only $(u,v) > (s,t)$ should be considered, in
lexicographic order.

Now, consider Equations (\ref{eqn-s-partial}) and (\ref{eqn-t-partial})
differently:
\begin{eqnarray*}
{\cal F}_s(u,v,s,t) &=& \frac{\partial}{\partial s} 
\inner{S(u,v)-S(s,t)}{N(u,v)}, \\
&=& - \inner{S_s(s,t)}{N(u,v)} = 0,\\
{\cal F}_t(u,v,s,t) &=& - \inner{S_t(s,t)}{N(u,v)} = 0.
\end{eqnarray*}
Therefore, Equations (\ref{eqn-s-partial}) and (\ref{eqn-t-partial})
characterize the condition that the tangent plane at the surface point
$S(u,v)$ is also tangent to the surface at the other point $S(s,t)$.

Having three Equations (\ref{eqn-f-uvst}), (\ref{eqn-s-partial}), and
(\ref{eqn-t-partial}) in four variables, one gets a univariate curve
as the simultaneous solution in the $uvst$-space. This solution curve can 
be parametrized by a variable $\alpha$:
\[
   (u(\alpha),v(\alpha),s(\alpha),t(\alpha)).
\]
Denote by $T_\alpha$ the common bitangent plane of $S$
at $S(u(\alpha),v(\alpha))$ and $S(s(\alpha),t(\alpha))$.
Then, the surface patch that bounds the convex hull of the surface $S$ can
be constructed by connecting the corresponding surface points
$S(u(\alpha),v(\alpha))$ and $S(s(\alpha),t(\alpha))$ by line segments.
All these line segments correspond to the common bitangent plane
$T_\alpha$. Consequently, the surface patch that bounds the convex hull
of $S$ is the envelope surface of the tangent planes $T_\alpha$, which is
a developable surface~\cite{Aumann91,docarmo,JW99,Pottmann}.  We consider only solution 
points which satisfy $\inner{N(u,v)}{N(s,t)} > 0$. Redundant solutions
$(u,v,s,t)$, where $\inner{N(u,v)}{N(s,t)} < 0$, should be purged away. 
Figure~\ref{fig-ch-surface-1}(a) shows a developable surface patch 
that bounds the convex hull of a B-spline surface;
Figure~\ref{fig-ch-surface-1}(b) shows a similar surface patch
that bounds two convex surfaces.


\begin{figure}
    \begin{tabular}{cc}
    \psfig{width=2.7in,figure={figures/ch-two-1.ps}} & 
    \psfig{height=2.7in,figure={figures/ch-two-2.ps}} \\
    {\large (a)}  &  {\large (b)} \\ \\
    \begin{picture}(0,0)
        \put(20,80){\large $S(u,v) = S(s,t)$}
	\put(240,75){\large $S(s,t)$}
	\put(190,230){\large $S(u,v)$}
    \end{picture}
    \end{tabular}
\vskip -0.5in
    \caption{(a) A developable surface patch on the boundary of
    the convex hull of a rational surface $S(u,v)$ and (b)
    a similar patch that bounds two rational surfaces $S(u,v)$ and $S(s,t)$.}
    \label{fig-ch-surface-1}
\vskip 0.37in
\end{figure}

In the above discussion, we assumed that both the surface points
$S(u,v)$ and $S(s,t)$ are interior points of the surface $S$.
When one of the two points is located on the boundary curve of $S$,
the above characterization needs some refinements.
We assume that the second point $S(s_0,t)$ is located
on the boundary curve of $S$ with a fixed parameter $s=s_0$.
Then, the above characterization can be modified as follows
\begin{eqnarray}
   {\cal F}(u,v,s_0,t) &=& 0, \label{eqn-ch-bndry}\\
   {\cal F}_t(u,v,s_0,t) &=& 0. \nonumber
\end{eqnarray}
The zero-set of Equation (\ref{eqn-ch-bndry}) generates a silhouette curve 
in the $uvt$-space, which can be parametrized by a variable $\alpha$:
\[
   (u(\alpha),v(\alpha),t(\alpha)).
\]
The bounding developable surface patch can be constructed
by connecting the corresponding surface points
$S(u(\alpha),v(\alpha))$ and $S(s_0,t(\alpha))$ by line segments.

The curve $(u(\alpha),v(\alpha))$ outlines the boundary
on the surface patch $S(u,v)$ that belongs to the boundary of
the convex hull of the surface $S$.  The end points of
the curve segment $S(s_0,t)$ that belong to the boundary
of the convex hull are computed by solving the following
three equations in three variables:
\begin{eqnarray*}
   {\cal F}(u,v,s_0,t) &=& 0,\\
   {\cal F}_u(u,v,s_0,t) &=& 0,\\
   {\cal F}_v(u,v,s_0,t) &=& 0.
\end{eqnarray*}

The second surface point may be a corner point $S(s_0,t_0)$ as well.
In this case, one needs to solve the following bivariate equation:
\[
   {\cal F}(u,v,s_0,t_0) = 0,
\]
which characterizes the tangent planes from $S(s_0,t_0)$
to the freeform surface $S(u,v)$.
The zero-set generates a curve in the $uv$-plane,
which can be parametrized by a variable $\alpha$:
\[
   (u(\alpha),v(\alpha)).
\]
The bounding developable surface patch is constructed
as a conical surface that has its apex at the corner point
$S(s_0,t_0)$ and is generated by the curve
$S(u(\alpha),v(\alpha))$ on the surface $S$.

The convex hull of a freeform surface also contains
the convex hull of its boundary space curves.
The approach to computing the convex hull of a rational space curve
is slightly different from the above three cases.
There is no unique tangent plane at a point on a space curve.
The tangent planes form a one-parameter family of planes,
each of which contains the tangent line of the space curve.
Representing all tangent planes at a curve point requires
one additional parameter for the rotation about the tangent line.
Moreover, computing the convex hull of a space curve is 
an interesting problem by itself.
The next section is thus devoted to this special case.

%%%%%%%%%%%%%%%%%%%%%%%%%%%%%%%%%%%%%%%%%%%%
\section{The Convex Hull of a Space Curve}
\label{sec-space-curve}

In this section, we assume that the rational space curve $C(t)$ is
non-planar; that is, its torsion is non-zero at all curve points
except at finitely many inflection points.
(The convex hull of a planar curve can be computed using the results
of Elber et al.~\cite{Elber2001b} and Johnstone~\cite{Johnstone2001}.)
The convex hull of a freeform space curve $C(t)$ is characterized
by the planes that are tangent to the curve at two different locations.

Assume that there is a plane tangent to the curve $C$ at two different
locations $C(s)$ and $C(t)$.  Then the three vectors of
$C'(s)$, $C'(t)$, and $C(s)-C(t)$ are parallel to the tangent plane.
Hence, the determinant of a matrix consisting of these three vectors 
must vanish:
\begin{eqnarray*}
{\cal G}(s,t) &=& |C(s)-C(t)\ \ C'(s)\ \ C'(t)|\\
&=& \inner{C(s)-C(t)}{C'(s) \times C'(t)}\\
&=& 0,
\end{eqnarray*}
which is a necessary condition for the existence of
a common tangent plane at the two curve points $C(s)$ and $C(t)$.
Once again, we need to consider only the solutions satisfying $s \neq t$.

The zero-set of the above equation can be locally parametrized 
by $t$ (or by $s$): ${\cal G}(s(t),t) = 0$
(or ${\cal G}(s,t(s)) = 0$).
By connecting the corresponding curve points $C(s(t))$ and $C(t)$
by line segments, we can construct the surfaces
that bound the convex hull of the space curve $C(t)$. Again, these
surfaces are developable surfaces because all the line segments 
connecting the points $C(s(t))$ and $C(t)$ are from the common 
tangent planes and so the surfaces are envelope surfaces of these
common tangent planes~\cite{Pottmann}.
Figure~\ref{fig-ch-curve1}(a) shows a space curve and
Figure~\ref{fig-ch-curve1}(b) shows its convex hull bounded by
developable surface patches.

We now consider how to compute the interior points of
the curve segment $C(t)$ that belong to the boundary of
the convex hull.
Let $T(t)$ denote the common tangent plane
at two corresponding curve locations $C(s(t))$ and $C(t)$.
This plane may not bound the convex hull of the curve $C$
if the curve $C$ intersects this plane transversally
at a third point $C(u)$, where $ u \neq s(t)$ and $u \neq t$.
This condition can be formulated as follows
\begin{eqnarray*}
   {\cal H}(u,s,t)
   &=& |C(u)-C(t)\ \ C'(s)\ \ C'(t)|\\
   &=& \inner{C(u)-C(t)}{C'(s) \times C'(t)}\\
   &=& 0,
\end{eqnarray*}
but
\begin{eqnarray*}
   {\cal H}_u(u,s,t)
   &=& |C'(u)\ \ C'(s)\ \ C'(t)|\\
   &=& \inner{C'(u)}{C'(s) \times C'(t)}\\
   &\neq& 0,
\end{eqnarray*}
where ${\cal H}_u$ is the $u$-partial of ${\cal H}$.

The redundancy of the solution $(s,t)$
changes only through the configurations where
three curve points $C(u),C(s),C(t)$ admit a common tangent plane.
The condition is characterized by the following three equations:
\begin{eqnarray*}
{\cal G}(s,t) &=& 0,\\
{\cal H}(u,s,t) &=& 0,\\
{\cal H}_u(u,s,t) &=& 0.
\end{eqnarray*}

\begin{figure}
    \begin{tabular}{cc}
    \psfig{width=2.5in,figure={figures/curve1.ps}} & 
    \psfig{width=2.5in,figure={figures/ch-curve1.ps}} \\  
    {\large (a)}  &  {\large (b)} \\
    \end{tabular}
    \caption{$C(t)$ is a rational space curve with non-zero torsion:
    (a) shows a rational space curve $C(t)$; and
    (b) shows the developable surface patches
    that bound the convex hull of $C(t)$.}
\vskip 0.37in
    \label{fig-ch-curve1}
\end{figure}

In the above discussion, we assumed that both the curve points
$C(s)$ and $C(t)$ are interior points of the curve $C$.
When one of the two points is an end point $C(s_0)$ of the curve $C$,
we need a modification to this approach.
Let $T(t)$ denote the tangent plane from a fixed point $C(s_0)$ to
the curve $C(t)$.  Assume that the curve $C$ transversally intersects
the tangent plane $T(t)$ at a third point $C(u)$, $u \neq t$.  Then,
we have the following condition:
\begin{eqnarray*}
   {\cal K}(u,t)
   &=& |C(u)-C(s_0)\ \ C(t)-C(s_0)\ \ C'(t)|\\
   &=& \inner{C(u)-C(s_0)}{(C(t)-C(s_0)) \times C'(t)}\\
   &=& 0,
\end{eqnarray*}
but
\begin{eqnarray*}
   {\cal K}_u(u,t)
   &=& |C'(u)\ \ C(t)-C(s_0)\ \ C'(t)|\\
   &=& \inner{C'(u)}{(C(t)-C(s_0)) \times C'(t)}\\
   &\neq& 0.
\end{eqnarray*}

The redundancy of the solution $t$
changes only through the configurations where
three curve points $C(s_0),C(u),C(t)$ admit a common tangent plane.
The condition is characterized by the following two equations:
\begin{eqnarray*}
{\cal K}(u,t) &=& 0,\\
{\cal K}_u(u,t) &=& 0.
\end{eqnarray*}

Figure~\ref{fig-ch-curve2} shows a rational space curve $C(t)$
and its convex hull which is bounded by two conical developable surfaces,
where each surface has its apex at an end point of the curve and
is generated by the curve $C(t)$.

\begin{figure}
\bigskip
\bigskip
\begin{center}
    \begin{tabular}{c}
    \psfig{width=2.7in,figure={figures/ch-curve2.ps}} \\
    %\begin{picture}(0,0)
    %    \put(70,60){\large $C(t)$}
    %\end{picture}
    \end{tabular}
    \caption{$C(t)$ is a rational space curve; its convex hull 
	is bounded by two conical developable surface patches.}
    \label{fig-ch-curve2}
\end{center}
\vskip 0.37in
\end{figure}

%%%%%%%%%%%%%%%%%%%%%%%%%%%%%%%%%%%%%%%%%%%%%%%%%%%%%%%%%%%%%%%%%%%%%%%%%%%%%
\section{Tritangent Planes}
\label{sec-tri-tangencies}

In Section \ref{sec-ch-surface}, we considered bitangent planes.
In this section, we consider how to deal with tritangent planes.
The bitangent condition prescribes a curve in the parametric domain.
The case of tritangency results in a solution with zero dimension. 
To form the complete convex hull, one needs to combine 
these tritangent planes with the bitangent developable surfaces which
bound the convex hull.

We can extend the three equations (\ref{eqn-f-uvst})--(\ref{eqn-t-partial})
from Section \ref{sec-ch-surface} to the tritangent condition. 
Let the three tangent points be $S(u,v), S(s,t)$ and $S(m,n)$.
Then, the three equations (\ref{eqn-f-uvst})--(\ref{eqn-t-partial}) 
constrain the tangent plane at $S(u,v)$ to be tangent to 
surface $S(s,t)$. We now add three more equations to constrain 
this tangent plane to be tangent to the third surface point 
$S(m,n)$ as well.  Consequently, we have six equations in six variables:
\begin{eqnarray}
   {\cal F}(u,v,s,t) &=& \inner{S(u,v)-S(s,t)}{N(u,v)} = 0, \label{eqn-f-uvst-new} \\
   {\cal F}_s(u,v,s,t) &=& -\inner{S_s(s,t)}{N(u,v)} = 0, \label{eqn-s-partial-new}\\
   {\cal F}_t(u,v,s,t) &=& -\inner{S_t(s,t)}{N(u,v)} = 0, \label{eqn-t-partial-new} \\
{\cal F}(u,v,m,n) &=& \inner{S(u,v)-S(m,n)}{N(u,v)} \label{eqn-g-uvmn} = 0,\\
%&=& 0, \nonumber \\
{\cal F}_m(u,v,m,n) &=& -\inner{S_m(m,n)}{N(u,v)} = 0, \label{eqn-m-partial}\\
{\cal F}_n(u,v,m,n) &=& -\inner{S_n(m,n)}{N(u,v)} = 0. \label{eqn-n-partial}
\end{eqnarray}
These six equations (\ref{eqn-f-uvst-new})--(\ref{eqn-n-partial})
in six variables have a zero-dimensional solution or a finite set of points.
Examples for common tritangent planes to three surfaces are shown in
Figure~\ref{fig-three-tangent}.

\begin{figure}
    \begin{tabular}{cc}
    \psfig{width=2.7in,figure={figures/ch-three-tang.ps}} & 
    \psfig{width=2.7in,figure={figures/ch-three-trim.ps}} \\
    {\large (a)}  &  {\large (b)}
    \end{tabular}
    \caption{(a) All the common tritangent planes to three surfaces. 
        (b) One developable 
	surface which bounds first two convex parts of the surface and 
	the planes (triangles) which are common tangent to the surface.
        }
    \label{fig-three-tangent}
\vskip 0.37in
\end{figure}

%-----------------------------------------------------------------------%
\section{Trimming and Combining the Convex Hull}
\label{sec-trim-combine}

As in the case of Section~\ref{sec-ch-surface}, the tritangent planes
include some redundant solutions.
The redundancy can be eliminated using the following three conditions:
\begin{eqnarray}
& & \inner{N(u,v)}{N(s,t)} > 0 \label{eqn-trim1} \\
& & \inner{N(u,v)}{N(m,n)} > 0 \label{eqn-trim2} \\
& & \inner{S(x,y) - S(u,v)}{N(u,v)} \neq 0,\ 
\forall (x,y) \neq (u,v), (s,t), (m,n). \label{eqn-trim3}
\end{eqnarray}
The first two Equations (\ref{eqn-trim1}) and (\ref{eqn-trim2}) enforce 
the tritangent planes to have their normal vectors in the same direction. 
Then, Equation (\ref{eqn-trim3}) constrains the planes
not to pass through the interior of the convex hull.
%Equations (\ref{eqn-trim1}) and (\ref{eqn-trim2}) assume that
%the input surfaces are properly oriented.

To complete the convex hull construction, one needs to compute 
bitangency for each pair of surface parts: the bitangency of $S(u,v)$ and 
$S(s,t)$, the bitangency of $S(s,t)$ and $S(m,n)$, and the bitangency
of $S(m,n)$ and $S(u,v)$. Not all the bitangent developable surfaces
contribute to the boundary of the final convex hull. One needs to trim out
certain parts of these developable surfaces according to
the common tritangent planes to the three surface parts.

Because all the developable surface patches, which form the convex hull of 
two surface parts as mentioned in Section \ref{sec-ch-surface}, bound
the convex hull of each pair of surface parts, the boundary curves of these 
developable surfaces always intersect with tritangent planes
at the corresponding tritangent points (see Figure~\ref{fig-three-tangent}(b)).
In fact, we can also determine the trimming line of each developable surface
as an edge of an adjacent tritangent plane.
As all the developable surfaces are generated by a moving bitangent plane 
between two surfaces, the trimmed developable surface is continuously 
connected to a tritangent plane.
Figures \ref{fig-ch-three1} and \ref{fig-ch-three2} show
two examples of the convex hull.

\begin{figure}
\begin{center}
    \begin{tabular}{cc}
    \psfig{width=3.2in,figure={figures/s.ps}}
    \psfig{width=2.7in,figure={figures/ch-three-1.ps}} \\
    \end{tabular}
    \caption{(a) A surface with three convex parts;
        and (b) its convex hull.}
    \label{fig-ch-three1}
\end{center}
\vskip 0.2in
\end{figure}

\begin{figure}
\begin{center}
    \begin{tabular}{c}
    \psfig{width=3.6in,figure={figures/ch-three-2.ps}} \\
    \end{tabular}
    \caption{The convex hull of three ellipsoids.}
    \label{fig-ch-three2}
\end{center}
\vskip 0.37in
\end{figure}

%%%%%%%%%%%%%%%%%%%%%%%%%%%%%%%%%%%%%%%%%%%%%%%%%%%%%%%%%%%%%%%%%%%%%%%%%%%%%
\section{Conclusion}
\label{sec-conclusion}
In this paper, we have presented an algorithm for computing
the convex hull of freeform surfaces.
The problem was first reformulated as a zero-set finding problem;
and it is then solved by computing the zero-set and
projecting the solutions onto a proper subspace.

%%%%%%%%%%%%%%%%%%%%%%%%%%%%%%%%%%%%%%%%%%%%%%%%%%%%%%%%%%%%%%%%%%%%%%%%%%%%%
\section*{ACKNOWLEDGMENTS}

We would like to thank the anonymous referees for useful comments.
We are appreciative of the stimulating research environment
provided by the Dagstuhl Seminar on Geometric Modeling.
% JJ
This work was supported in part by the National Science Foundation under 
grant CCR-0203586, in part by the Israeli Ministry of Science
Grant No.~01--01--01509, and in part by
the Korean Ministry of Science and Technology (MOST) under the
Korean-Israeli binational research grant.

%%%%%%%%%%%%%%%%%%%%%%%%%%%%%%%%%%%%%%%%%%%%%%%%%%%%%%%%%%%%%%%%%%%%%%%%%%%%%

\bibliographystyle{acm}
\begin{thebibliography}{99}

\bibitem{Aumann91}
G.~Aumann.
\newblock Interpolation with Developable B\'ezier Patches.
\newblock {\em Computer Aided Geometric Design\/},
Vol.~8, pp.~409--420, 1991.

\bibitem{Kim91}
C.~Bajaj and M.-S.~Kim.
\newblock Convex Hulls of Objects Bounded by Algebraic Curves.
\newblock {\em Algorithmica\/}, Vol.~6, pp.~533--553, 1991.

\bibitem{Souvaine90}
D.~Dobkin and D.~Souvaine.
\newblock Computational Geometry in a Curved World.
\newblock {\em Algorithmica\/}, Vol.~5, No.~3, pp.~421--457, 1990.

\bibitem{docarmo}
M.P.~do Carmo.
\newblock {\em Differential Geometry of Curves and Surfaces}.
\newblock Prentice-Hall, Inc., Upper Saddle River, New Jersey 07458, 1976.

\bibitem{Elber93}
G.~Elber and E.~Cohen.
\newblock Second-Order Surface Analysis Using Hybrid Symbolic and
Numeric Operators.
\newblock ACM Transactions on Graphics, Vol. 12, No.2, April 1993, 
pp. 160--178.

\bibitem{Elber2001a}
G.~Elber and M.-S.~Kim.
\newblock Geometric Constraint Solver
Using Multivariate Rational Spline Functions.
\newblock {\em Proc.~of ACM Symposium on Solid Modeling and Applications\/},
Ann Arbor, MI, June 4--8, 2001.

\bibitem{Elber2001b}
G.~Elber, M.-S.~Kim, and H.-S.~Heo.
\newblock The Convex Hull of Rational Plane Curves.
\newblock {\em Graphical Models\/}, Vol.~63, No.~3, pp.~151--162, 2001.

\bibitem{Graham83}
R.~Graham and F.~Yao.
\newblock Finding the Convex Hull of a Simple Polygon.
\newblock {\em J.~of Algorithms\/}, Vol.~4, No.~4, pp.~324--331, 1983.

% \bibitem[Hartshorne 77]{hartshorne}
%\bibitem{hartshorne}
%R.~Hartshorne.
%{\em Algebraic Geometry}. Springer-Verlag, New York, 1977.

\bibitem{Johnstone2001}
J.~Johnstone.
\newblock A Parametric Solution to Common Tangents.
\newblock {\em Proc.~of Int'l Conf.~on Shape Modeling and Applications\/},
Genova, Italy, May 7--11, 2001, 240--249.

% \bibitem[Johnstone 02]{jj02}
%\bibitem{jj02}
%J.~Johnstone.
%The Tangential Curve, 2002.

% \bibitem[Johnstone 03]{jj03tangsurf}
%\bibitem{jj03tangsurf}
%J.~Johnstone.
%The Bezier tangential surface system: a dual representation of tangent space,
%2003.

%\bibitem{jj03}
%% JJ
%J.~Johnstone.  Kernels from Hulls, 2003.

\bibitem{JW99}
B.~J\"uttler and M.~Wagner.
\newblock Rational Motion-based Surface Generation.
\newblock {\em Computer-Aided Design\/},
Vol.~31, No.~3, pp.~203--214, 1999.

\bibitem{Lee83}
D.T.~Lee.
\newblock On Finding the Convex Hull of a Simple Polygon.
\newblock {\em Int'l J.~Computer and Information Sciences\/},
Vol.~12, No.~2, pp.~87--98, 1983.

\bibitem{Pottmann}
H.~Pottmann and J.~Wallner.
\newblock {\em Computational Line Geometry\/}.
\newblock Springer-Verlag, Berlin, 2001.

\bibitem{Prep85}
F.~Preparata and M.~Shamos.
\newblock {\em Computational Geometry: An Introduction\/}.
\newblock Springer-Verlag, New York, NY, 1985.

\bibitem{Schaeffer87}
A.~Sch\"affer and C.~Wan Wyk.
\newblock Convex Hulls of Piecewise-Smooth Jordan Curves.
\newblock {\em J.~of Algorithms\/}, Vol.~8, No.~1, pp.~66--94, 1987.

\end{thebibliography}

%%%%%%%%%%%%%%%%%%%%%%%%%%%%%%%%%%%%%%%%%%%%%%%%%%%%%%%%%%%%%%%%%%%%%%%%%%%%%

\end{document}
