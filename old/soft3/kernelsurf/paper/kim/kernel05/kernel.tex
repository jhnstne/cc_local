% ----------------------------------------------------------------
% Israel-Korea2005 Paper *****************************************
% ----------------------------------------------------------------
\documentclass{acmsiggraph}

\usepackage{times}
%\usepackage{ikconf_2005}



% if you use PostScript figures in your article
% use the graphics package for simple commands
\usepackage{graphics}
% or use the graphicx package for more complicated commands
\usepackage{graphicx}
% or use the epsfig package if you prefer to use the old commands
\usepackage{epsfig}

% The amssymb package provides various useful mathematical symbols
\usepackage{amssymb}

\begin{document}

\newtheorem{definitionenv}{Definition}
\newenvironment{definition}{\begin{quote}\begin{definitionenv}}%
                           {\end{definitionenv}\end{quote}}
\newenvironment{defn2}{\begin{quote}\begin{definitionenv}}%
                           {\end{definitionenv}\end{quote}}

\newtheorem{propositionenv}{Proposition}
\newenvironment{proposition}{\begin{quote}\begin{propositionenv}}%
                           {\end{propositionenv}\end{quote}}

\newtheorem{theoremenv}{Theorem}
\newenvironment{theorem}{\begin{quote}\begin{theoremenv}}%
                           {\end{theoremenv}\end{quote}}

\newtheorem{lemmaenv}{Lemma}
\newenvironment{lemma}{\begin{quote}\begin{lemmaenv}}%
                           {\end{lemmaenv}\end{quote}}

\newtheorem{corollaryenv}{Corollary}
\newenvironment{corollary}{\begin{quote}\begin{corollaryenv}}%
                           {\end{corollaryenv}\end{quote}}

\newtheorem{observationenv}{Observation}
\newenvironment{observation}{\begin{quote}\begin{observationenv}}%
                           {\end{observationenv}\end{quote}}

\newenvironment{proof}{\par\smallskip\noindent{\bf Proof.}\
\ignorespaces}{\stopproof\ignorespaces\bigbreak}

\newenvironment{notepar}{\par\smallskip\noindent{}\
\ignorespaces}{\stopproof\ignorespaces\bigbreak}

%\newcommand{\cal}{\mathcal}
\newcommand{\CH}{{\cal C}{\cal H}}
\newcommand{\inner}[2]{\left<{#1}, {#2} \right>}

\def\stopproof{\qquad\square}
\def\square{\vbox{\hrule height.2pt\hbox{\vrule width.2pt height5pt
\kern5pt \vrule width.2pt} \hrule height.2pt}}

\newcommand{\Comment}[1]{\relax}  % makes a "comment" (not expanded)
\newcommand{\QED}{\vrule height 1.4ex width 1.0ex depth -.1ex\ \vspace{.3in}} % square box
\newcommand{\lyne}[1]{\mbox{$\stackrel{\textstyle \leftrightarrow}{#1}$}}
\newcommand{\seg}[1]{\mbox{$\overline{#1}$}}
\newcommand{\prf}{\noindent{{\bf Proof}:\ \ \ }}
\newcommand{\choice}[2]{\mbox{\footnotesize{$\left( \begin{array}{c} #1 \\ #2 \end{array} \right)$}}}      
\newcommand{\scriptchoice}[2]{\mbox{\scriptsize{$\left( \begin{array}{c} #1 \\ #2 \end{array} \right)$}}}
\newcommand{\tinychoice}[2]{\mbox{\tiny{$\left( \begin{array}{c} #1 \\ #2 \end{array} \right)$}}}


\title{The Kernel of Freeform Surfaces\\[-0.07in]
{\normalsize (Extended Abstract)}}

\author{Gershon Elber$^1$, Joon-Kyung Seong$^2$, John K.~Johnstone$^3$, 
Wenping Wang$^4$, Myung-Soo Kim$^5$\\
$^1$Technion -- Israel Institute of Technology, Israel\\
$^2$University of Utah, USA\\
$^3$University of Alabama at Birmingham, USA\\
$^4$University of Hong Kong, China\\
$^5$Seoul National University, Korea}

\maketitle

\section*{Abstract}

We present algorithms for computing the kernel of a closed freeform 
rational surface.  The kernel computation is reformulated 
as a problem of finding the zero-sets of polynomial equations; 
using these zero-sets we characterize developable surface patches 
and planar patches that belong to the boundary of the kernel.
Using a plane-point duality, this paper also explores a duality
relationship between the kernel and the convex hull.
\vskip 1em {\bf Keywords:} 
Kernel, gamma-kernel, zero-set finding, freeform rational surface, B-spline,
symbolic computation, convex hull, duality


%----------------------------------------------------------------------%
% main text
\section{Introduction}
\label{sec-introduction}

The kernel of a closed surface is the set of points
that can see every point of the surface.
It is a convex connected subset of the interior of the volume
bounded by the surface.
Figure~\ref{fig-kernel-crv} shows a simple two-dimensional example
where the dark area represents the kernel of a closed freeform curve
in the plane.
As one can see from this example, the boundary of a kernel consists
of some segments of the curve and tangent line segments
at certain inflection points of the curve.
For a closed freeform surface, the kernel is bounded by some patches
of the surface and tangent developable surface patches at parabolic points.
The kernel is an important structure to understand in developing
a theory of visibility~\cite{Prep85}.  For example, interpreting the surface 
as the boundary of a room, a point of the kernel would be an ideal
location to place a light, camera, or guard,
since the entire room would be visible.

The interior of the kernel of a freeform closed surface $S(u,v)$
is characterized by the set of points that are never contained
in the tangent planes of the surface.  The containment of
a point $P = (x,y,z)$ in the tangent plane of $S(u,v)$ can be
formulated as a polynomial equation in five variables:
\[
\inner{(x,y,z)-S(u,v)}{N(u,v)} = 0,
\]
where $N(u,v)$ is the normal vector of $S(u,v)$.
The projection of the zero-set of this equation
(onto the $xyz$-space) covers all non-interior points of the kernel.
Thus the interior of the kernel is the volume
that is uncovered by the projection of the zero-set.

The duality between planes and points in the three-dimensional space
implies a similar duality relationship between the kernel and
the convex hull.  This paper explores the details of the duality
and presents an algorithm for computing the kernel of a closed surface
using the convex hull of its dual tangential surface.

The computation of kernels from convex hulls offers a different point of
view on the computation of the kernel, allowing all attention
to be placed on the development of an algorithm
for the efficient and robust computation of the convex hull.
We can appeal to the Voronoi diagram and Delaunay triangulation
for a comparable example~\cite{Prep85}.  There are algorithms for the computation of
each of these structures, but many still prefer to always compute
the Voronoi diagram by computing the Delaunay triangulation and then dualizing.

\begin{figure}
    \begin{center}
    \psfig{width=1.7in,figure={figures/kernel-crv-1.ps}}
    \end{center}
\vskip -0.37in
    \caption{The kernel of a planar curve.}
    \label{fig-kernel-crv}
\end{figure}

The rest of this paper is organized as follows.
In Section 2, we present algorithms for computing the kernels
of freeform closed curves and surfaces.
The duality relationship between the kernel and the convex hull
is explored in depth in Section 3.
Finally, in Section~4, we conclude this paper.


%----------------------------------------------------------------------%
\section{The Kernel of a Freeform Curve and a Freeform Surface}
\label{sec-kernel}

In this section, we present an algorithm for computing the kernel of
a freeform surface, which is closely related to the convex hull 
computation~\cite{Elber2001b,seong2004}.
For the clarity of explanation, we first consider the kernel of
a freeform closed curve and generalize its construction to the case of
computing the kernel of a freeform closed surface.
There is a dual relationship between the kernel and the convex hull.
More details will be discussed in the next section
(see also \cite{jj03}).

\subsection{The Kernel of a Freeform Curve}
\label{subsec-kernel-curve}

The kernel of a closed curve in the plane is the set of points
that can see every point of the curve.
%A self-intersecting curve always has an empty kernel
%since no point can see every point in the neighborhood of
%a self-intersection point.
Let $C(t) = (x(t),y(t))$ be a $C^1$-continuous planar rational curve;
and $P$ be a point in the plane.
Assume that $P$ is located on the tangent line of the curve at $C(t)$:
\[
\inner{P-C(t)}{N(t)} = 0,
\]
where $N(t) = (y'(t),-x'(t))$ is the normal vector of $C(t)$.

If the curve $C$ has non-zero curvature at the tangential point $C(t)$,
the point $P$ cannot see at least a certain part of the curve
unless $P$ is the tangent point $C(t)$ itself.
%
% JJ: I didn't understand this next sentence.
%
% MSKIM: What about this change?
%
Moreover, the point $P$ is contained in the swept region of nearby tangent lines.
Thus, all points $P$ (except $C(t)$) on the tangent line belong to
the exterior of the kernel in the sense of the point-set topology.
Now, consider a tangent line at an inflection point $C(t)$.
In this case, some points on the tangent line may see
every point of the curve.  Thus, these points may be included
in the boundary of the kernel, but not in its interior.
(Note that they are infinitely close to the exterior of the kernel,
i.e.~the swept area of tangent lines at non-inflection points.)

Now consider a point $P$ that is never covered by a tangent line
of the curve:
\begin{eqnarray}
\inner{P - C(t)}{N(t)} \neq 0,\quad \forall\ t.
\label{def-krnl-crv}
\end{eqnarray}
The point $P$ should be in the kernel of the curve $C$.
Otherwise, there exists a curve point $C(t_1)$ that $P$ cannot see.
In other words, a certain segment of $C$ blocks the sight of $C(t_1)$
from the view point $P$.  Now consider the line connecting $P$ and
$C(t_1)$ and rotate the line about the point $P$.  Since the curve
is closed, there is a curve point $C(t)$ where the line touches
tangentially.  Then the point $P$ is contained in the tangent line
at this point $C(t)$, which is a contradiction to (\ref{def-krnl-crv}).

The set of points $P$ satisfying Equation (\ref{def-krnl-crv}) is
an open planar region since its complement, the swept area of
all tangent lines, is a closed set.  (Here, `open' and `closed'
are interpreted in the sense of the point-set topology.)
Note that this argument may not hold in general
for an unbounded curve $C(t)$ such as a hyperbola.

Now, it is clear that each interior point $P$ of the kernel 
must satisfy condition (\ref{def-krnl-crv}).
Otherwise, point $P$ is on a tangent line.
%
% JJ: I didn't understand this next sentence
%
% MSKIM: I slightly modified it.
%
Then, as discussed above, point $P$ is either in the exterior
or on the boundary of the kernel.
This is a contradiction to that $P$ is an interior point of the kernel.
Consequently, (\ref{def-krnl-crv}) is a characterizing condition
for the interior of the kernel.  The boundary of the kernel
consists of some segments of the curve $C(t)$ and some segments
of tangent lines at certain inflection points.  All uncovered points
in the sweep of tangent lines belong to the interior of the kernel.
 
Consider the following single equation in three variables, 
\begin{eqnarray}
{\cal T}(x, y, t) = \inner{(x, y) - C(t)}{N(t)} = 0. \label{eqn-kernel-crv}
\end{eqnarray}
A plane point $P_0 = (x_0, y_0)$ is not included in the interior of
the kernel if and only if there exists
$t_0$ such that ${\cal T}(x_0, y_0, t_0) = 0$. Having one equation in 
three variables, the zero-set of ${\cal T}(x, y, t) = 0$ is a bivariate 
surface in the $xyt$-space. 
Figure~\ref{fig-kernel-1}(a) illustrates one example of the zero-set for the 
planar curve shown in white.  Now, by projecting the zero-set along
the $t$-direction onto the $xy$-plane, the uncovered region in the $xy$-plane 
is the region of the $xy$-plane that satisfies (\ref{def-krnl-crv})
and hence is in the interior of the kernel of $C(t)$
(see Figure \ref{fig-kernel-1}(b)).

\begin{figure}
    \begin{tabular}{cc}
    \psfig{width=1.47in,figure={figures/kernel-zero-1.ps}} & 
    \psfig{width=1.47in,figure={figures/kernel-zero-2.ps}} \\
    {\large (a)}  &  {\large (b)}
    \end{tabular}
    \caption{(a) The zero-set of Equation (\ref{eqn-kernel-crv}) 
	for the white-colored planar curve. 
	(b) Top (along the $t$-direction) view of the 
	zero-set. The uncovered region of this zero-set is
	the kernel of the planar curve.}
    \label{fig-kernel-1}
%\vskip 0.3in
\end{figure}

\begin{figure}
    \begin{tabular}{cc}
    \mbox{\hspace{-0.1in}}
    \psfig{width=1.7in,figure={figures/kernel-r-1.ps}} & 
    \mbox{\hspace{-0.2in}}
    \psfig{width=1.7in,figure={figures/kernel-r-2.ps}} \\[-0.2in]
    {\large (a)}  &  {\large (b)}
    \end{tabular}
\vskip -0.1in
    \caption{Two $\gamma$-kernels: (a) $\gamma = 20^\circ$ and
	(b) $\gamma = 5^\circ$.  (a) Note that the regular kernel of
        a (convex) ellipse is the whole region surrounded by the ellipse;
	and (b) compare this with Figure~\ref{fig-kernel-crv}(a)
        which is the regular kernel of the same curve.}
    \label{fig-r-kernel}
%\vskip 0.37in
\end{figure}


\subsection{The Gamma Kernel of a Freeform Curve}
\label{subsec-gamma-kernel-curve}

It is also interesting to consider an angle-variant of the kernel.
Instead of considering the region uncovered by the tangent lines,
we may consider a similar region uncovered by rotated lines
that make angle $\gamma$ with the tangent lines.
This generates a convex region that is contained in the kernel.
The $\gamma$-kernel can also be defined as follows:

\begin{definitionenv}
Let $C(t)$ be a $C^1$-continuous planar curve.  A point $P$ is
in the $\gamma$-kernel of $C(t)$ if $P$ has a line of sight with every point 
of $C(t)$ that makes at least angle $\gamma$ with $C'(t)$,
the tangent of $C(t)$.
\end{definitionenv}
If $\gamma = 0$, then the $\gamma$-kernel is reduced to the regular kernel 
of $C(t)$. The $\gamma$-kernel of $C(t)$ can be computed by the zero-set 
of the following equation:
\begin{eqnarray*}
{\cal T}_1(x, y, t) = \inner{(x,y) - C(t)}{R_{\pm\gamma} [N(t)]} = 0, 
\end{eqnarray*}
where $R_{\pm\gamma} [\ ]$ is the rotation operation by angle $\pm\gamma$.
Figure~\ref{fig-r-kernel} shows two simple examples of $\gamma$-kernel.




\subsection{The Kernel of a Freeform Surface}
\label{subsec-kernel-surface}

We consider the problem of computing the kernel of a freeform closed surface.
In the surface case, we employ the following multivariate equation:
\begin{eqnarray}
{\cal L}(x,y,z,u,v) = \inner{(x,y,z) - S(u,v)}{N(u,v)} = 0,
\label{eqn-kernel-srf}
\end{eqnarray}
where $N(u,v) = S_u(u,v) \times S_v(u,v)$ is the normal vector field of the 
surface $S(u,v)$.   Consider the projection of the zero-set of
the above equation onto the $xyz$-space.
The uncovered convex volume under this projection is
the interior of the kernel of the surface $S$.

Similarly to the case of freeform closed curves in the plane,
the kernel of a freeform closed surface is bounded
by some patches of the surface and 
certain developable surface patches that are generated as
envelopes of tangent planes at parabolic points of the surface $S(u,v)$.
The reason can be explained (rather informally) as follows.

Assume that a point $P$ is located on the tangent plane of $S$ at
an elliptic point $S(u,v)$.  The point $P$ may not see
certain part of $S$ unless it is the tangent point $S(u,v)$ itself.
Thus, all points $P$ (except $S(u,v)$) on the tangent plane belong to
the exterior of the kernel.  (Note that some elliptic points
on the convex part of the surface may appear on the boundary of the kernel
as one can see at the bottom part of Figure~\ref{fig-kernel-srf}(c).)

Now, consider a tangent plane at a hyperbolic point $S(u,v)$.
The tangent plane intersects the surface in a curve
that has a self-intersection at the point $S(u,v)$.
Thus each point $P$ on the tangent plane may not see
some part of the surface $S$.  Consequently, all points
on such a tangent plane belong to the exterior of the kernel.

In the case of parabolic points, the tangent planes
at these points do a similar role as the tangent lines
at inflection points in the curve case.
The envelope surfaces of tangent planes along the parabolic lines
may generate some patches on the boundary of the kernel.

Figure \ref{fig-kernel-srf} shows three examples of the kernel of a 
freeform surface. These examples were computed by deriving the parabolic
lines of the surfaces, following~\cite{Elber93}, and computing the 
envelope of the tangent planes moving along the parabolic lines
(see also \cite{JW99}).

\begin{figure}
    \begin{tabular}{ccc}
    \mbox{\hspace{-0.2in}}
    \psfig{width=1.2in,figure={figures/kernel-srf-1.ps}} &
    \mbox{\hspace{-0.3in}}
    \psfig{width=1.2in,figure={figures/kernel-srf-2.ps}} & 
    \mbox{\hspace{-0.2in}}
    \psfig{width=1.2in,figure={figures/kernel-srf-3.ps}}\\
    {\large (a)}  &  {\large (b)} &  {\large (c)}
    \end{tabular}
    \caption{Three examples of computations of the kernel 
	(dark region) of freeform surfaces. Please note that 
	magenta-colored curves are 
	parabolic lines of the surface.}
    \label{fig-kernel-srf}
\vskip 0.2in
\end{figure}


%----------------------------------------------------------------------%
\section{Duality between the Convex Hull and the Kernel}
\label{sec-dual}

In this section, we explore the dual relationship between the kernel and convex hull
for closed surfaces.
This relationship can be used to compute kernels from convex hull algorithms.

We begin by reviewing some theory of duality.
There is a geometric duality between points and planes in 3-space~\cite{Pottmann}.
Classically, the plane $a_1x + a_2y + a_3z + a_4=0$ is dual to the point
$(a_1,a_2,a_3,a_4)$
in projective 3-space
(or equivalently, the point
$(\frac{a_1}{a_4},\frac{a_2}{a_4},\frac{a_3}{a_4})$ in Cartesian 3-space),
although many other point-plane dual maps can be defined.
Using geometric duality, the tangent space of a surface $S(u,v)$ can be
mapped
to a surface $S^*(u,v)$ in dual space, by dualizing the tangent plane at
$S(u,v)$ to the point $S^*(u,v)$.
$S^*$ can be called a {\em tangential surface}, as it represents a
tangent space.
This mapping to dual space is complicated by the fact that, for a closed
surface with empty kernel,
some tangent planes will inevitably map to infinity.\footnote{One
solution to this problem is presented in \cite{jj04tangsurf},
where three cooperating dualities, called a tangential surface system,
are used to represent a tangent space robustly.}
However, when the kernel of $S$ is not empty and a single point of this
kernel is known,
a simpler dual representation of $S$'s tangent space is possible.
By translating the kernel point to the origin,
a single tangential surface is enough (Figure~\ref{fig:peartangdsurf}).

\begin{lemma}
\label{lem:drobust}
If the origin lies inside the kernel of $S$,
a single tangential surface of $S$ is a robust finite representation of
the tangent space of $S$.
\end{lemma}
\prf
Planes through the origin are mapped to infinity by the above duality.
Since the origin lies inside the kernel,
no tangent plane will pass through the origin and no tangent plane will map to infinity.
\QED

\begin{figure}
\begin{center}
%\includegraphics*[scale=.18]{img/peartangdsurf.ps}
\includegraphics*[scale=.4]{img/peartangdsurf.ps}
\end{center}
\vskip -0.2in
\caption{The tangential surface of a pear}
\label{fig:peartangdsurf}
\end{figure}

%%%%%%%%%%%%%%%%%%%%%%%%%%%%%%%%%%%%%%%%%%%%%%%%%%%%%%%%%%%%%%%%

\subsection{A Reduction to the Convex Hull}
\label{sec:reduction}

We are now ready to establish the dual relationship between kernel and convex hull
for a closed $C^1$-continuous surface $S$.
The general structure of the dual relationship will be discussed in this section,
while Section~\ref{sec:cusp} will study its details and provide a complete algorithm.

Suppose that a single point of the kernel has been computed and the surface $S$
has been translated so that this point is at the origin.
Section~\ref{sec:firstpt} will treat the computation of a single point of the kernel.

Using the argument of Section~\ref{subsec-kernel-curve},
an interior point of the kernel can be characterized as a point that is covered by
no tangent plane.
This can be reinterpreted in dual space.

\begin{defn2}
A plane $P^*$ in dual space is {\bf free} if $P^*$ does not intersect the tangential surface $S^*$.
\end{defn2}

\begin{lemma}
\label{lem:dualkernelchar}
$P$ is an interior point of the kernel of $S$ if and only if 
its dual plane $P^*$ is free.
\end{lemma}
\prf
The statement 'no tangent plane of $S$ intersects $P$' in primal space
is equivalent to the statement 'no point of $S^*$ lies on $P^*$' in dual space.
\QED

Lemma~\ref{lem:dualkernelchar} identifies kernel points
with free planes in dual space.
Since the free planes of a surface bound its convex hull,
this immediately suggests a relationship between the kernel and convex hull.

\begin{theorem}
\label{thm:kernelhull}
If the origin lies inside the kernel of $S$,
the kernel of $S$ is dual to the convex hull of the tangential surface $S^*$.
More accurately, the boundary of the kernel is dual to the family of planes whose
envelope is the convex hull.
\end{theorem}
\prf
Since the origin lies inside the kernel, $S^*$ is a closed, finite surface
and its convex hull is well defined.
Since kernel points of $S$ correspond to free planes,
boundary points of the kernel correspond to free tangent planes of $S^*$
(free planes that are just about to intersect $S^*$).
The free tangent planes of a surface $D$ bound its convex hull: the point $P \in D$ lies
on the boundary of the convex hull of $D$ if and only if the tangent plane at $P$ 
does not intersect $D$.
Therefore, the boundary points of the kernel of $S$ correspond to the boundary points
of the convex hull of $S^*$.\footnote{Typically, points in primal space would be associated with planes in dual space,
	but we can also associate points with points:
	the point $p \in S$ in primal space is associated
	with the point $q \in S^*$ whose tangent plane dualizes to $p$.
	This is particularly easy when there is a parameterization, since $p$ and $q$ 
	have the same parameters.}
\QED

%%%%%%%%%%%%%%%%%%%%%%%%%%%%%%%%%%%%%%%%%%%%%%%%%%%%%%%%%%%%%%%%

\subsection{The First Point of the Kernel}
\label{sec:firstpt}

The reduction of kernel to convex hull relies on the prior knowledge
of a single point of the kernel.
This point is used to adjust the surface before dualization, ensuring
the finiteness of the tangential surface.
We now discuss the computation of this first point of the kernel.

Suppose that $S$ is concave.
There is no loss of generality in considering concave surfaces,
since the kernel of a convex surface $S$ is simply the interior of $S$.
Since we shall see that the tangential surface $S^*$ of
a concave surface $S$ is also concave, $S^*$ has bitangent planes.
Therefore, bitangent planes can be used in our construction. % of the candidate set.

The search for a kernel point will be reduced to the search for a free plane
(Lemma~\ref{lem:dualkernelchar}).
This moves the search to dual space, consistent with keeping the computation
of the kernel in dual space.
To make the search for a free plane tractable, a candidate set will be found.
A {\bf candidate set} is a finite set of planes in dual space
that necessarily contains a free plane (whenever the kernel is not empty).
We can test each candidate until a free plane is found.
If no free plane is found amongst the candidates,
the kernel must be empty.
Otherwise, the dual of the free plane is a kernel point.
Since there are a finite number of candidates, this is a valid solution.

The challenge is to restrict the candidates for a free plane from
all planes (a triply-infinite system) to a manageable finite set.
There are three degrees of freedom to the search for a free plane.
Two of these degrees of freedom can be removed by restricting the search
to free bitangent planes.
Consider a free plane in dual space.
While preserving its freedom, we can move this free plane
until it touches the surface once, generating a tangent plane,
then pivot until it touches the surface again, generating a bitangent plane.
Sometimes, this process can be continued one more level,
moving the bitangent plane about the surface until it becomes a
tritangent plane.
However, in some cases, there may be no tritangent planes
(consider an hourglass-shaped surface).
Thus, we must handle both cases, a surface with tritangent planes
and a surface with only one-parameter families of bitangent planes.

The candidate set is built as follows.
Consider a one-parameter family of bitangent planes of $S^*$.
If some of these planes are free and some are not free,
there will be a free tritangent plane at the boundary
between free and nonfree planes of the family.
Call a family of planes {\em homogeneous} if every plane is
free or every plane is not free.
We conclude that either all of the families of bitangent planes
are homogeneous or some tritangent plane is free.
This shows that the set consisting of all tritangent planes
and one (arbitrary) representative from each bitangent family
is a candidate set.
Notice that it is a finite set and contains a free plane when anyone exists.
By testing this candidate set, a free plane, and hence a kernel point, can be found.
This defines a dual approach to finding the first point of the kernel.

To get a robust seed point for the kernel,
all of the free planes in the candidate set should be found,
yielding a set of kernel points, and the seed kernel point 
should be defined to be the centroid of this set.
This works since the kernel is convex, and moves the point from the boundary
into the interior of the kernel.

Since there is no longer any way to guarantee the finiteness of the tangential surface
in mapping to dual space (Lemma~\ref{lem:drobust}),
the computation of the first kernel point must use the tangential surface system 
for robustness (see the preamble to Section~\ref{sec-dual}),
by testing for freedom in all three cooperating dual spaces.

%%%%%%%%%%%%%%%%%%%%%%%%%%%%%%%%%%%%%%%%%%%%%%%%%%%%%%%%%%%%%%%%%%%%%%%%%%%%%%%%

\subsection{The Hull's Structure}
\label{sec:cusp}

We now examine the structure of the convex hull of a tangential
surface $S^*$ through an analysis of its bitangent and tritangent planes,
which are necessarily degenerate when the surface $S$ has a nontrivial kernel.
This leads to a more precise understanding of the 
dual relationship between kernel and hull, culminating in an
algorithm for the construction of a kernel from a convex hull.

\begin{defn2}
A {\bf bitangent plane} of the surface $S^*$ is a plane that
is tangent to $S^*$ at two or more distinct points.
This plane is {\bf conventional} if the surface has unique, well-defined 
tangent planes at the points of tangency,
otherwise it is {\bf degenerate}.
\end{defn2}

\begin{lemma}
If $S$ has a nonempty kernel, all bitangent and tritangent planes 
of the tangential surface $S^*$ are degenerate.
\end{lemma}
\prf
Let $S$ be a surface with nonempty kernel.
$S$ has no self-intersections, otherwise the kernel is empty
since no point can see through a self-intersection.
But conventional bitangent and tritangent planes of $S^*$ are associated
with double and triple self-intersections of $(S^*)^* = S$ \cite{jj04tangsurf}.
Thus, no bitangent and tritangent planes of $S^*$ are conventional.
\QED

In order to understand degenerate tangent planes, we must first understand cusps.
Consider a concave
surface $S$ with nonempty kernel.
$S$ will contain parabolic points.
% JJ: I don't need this statement and so I'm removing it to avoid confusing the issue
% after all, the kernel of a concave surface is the intersection of the inside
% halfspaces of the tangent planes of its parabolic points
% and those of some of its elliptic points.
Since the tangent planes at parabolic points of $S$ are dual to cusps on $S^*$,
the tangential surface $S^*$ will contain cusps 
(Figures~\ref{fig:peartangdsurf} and \ref{fig:toptangdsurf}).
In general, these cusps are not isolated: they gather into closed curves on the surface.
If a point $P$ lies on a curve of cusps $C$,
the {\em tangent line} of $P$ is the tangent of $C$ at $P$.
Any plane through a cusp $P$ that contains the tangent line of $P$ may be considered
a degenerate tangent plane of the surface,
which leads to a degenerate case of the bitangent and tritangent plane.\footnote{Upon entering a cusp $P$, 
	the tangent plane reverses direction, sweeping out all of 3-space 
	by rotating about the tangent line of $P$,
	until it is ready to leave $P$ on the other side in an opposite orientation.
        This explains why any plane through a cusp's tangent line may be considered a tangent plane.}

\begin{figure}
\begin{center}
%\includegraphics*[scale=.18]{img/toptangdsurf.ps}
\includegraphics*[scale=.4]{img/toptangdsurf.ps}
\end{center}
\vskip -0.2in
\caption{The tangential surface of a top}
\label{fig:toptangdsurf}
\end{figure}

\begin{defn2}
A {\bf degenerate bitangent plane} is 
(1) a plane that contains the tangent lines of two distinct cusps or
(2) a tangent plane (at some noncusp point) that contains the tangent line of a cusp. 
A {\bf degenerate tritangent plane} is 
a plane that contains the tangent line of three cusps, 
a tangent plane that contains the tangent lines of two cusps,
or a bitangent plane that contains the tangent line of one cusp.\footnote{The
	third case will not participate in the hull of $S^*$, since
	it implies a self-intersection in $S$.}
\end{defn2}

The structure of the hulls of our tangential surfaces can now be clarified.
The tangential surface $S^*$ of a surface $S$ with nonempty kernel 
will have a finite number of one-parameter families of degenerate bitangent planes
and a finite number of degenerate tritangent planes.
To create the convex hull, a surface is capped by bitangent developables
(the envelopes of these bitangent plane families) and tritangent planes.
The bitangent developables
are of two types, matching the two types of degenerate bitangent planes: 
a lofting between two cusp curves (type 1) and
a lofting between a cusp curve and a curve on the surface (type 2).

Using this deeper understanding of the hull's structure,
we can now describe how to compute the kernel directly from the hull.
The patches of the hull that lie on the surface $S^*$
cause no trouble: they correspond directly
to the patches of the kernel that lie on the surface $S$.
It is the translation of the patches of the hull that do not lie on the surface 
(the bitangent developables and tritangent planes)
that requires some elaboration.
Notice that each bitangent developable $D^*$ in the hull of $S^*$ 
is associated with a curve $D$ in primal space, since a family of planes dualizes to a curve.

We now have the following algorithm for computing the kernel of $S$ 
from the convex hull of $S^*$.

\begin{enumerate}
\item Compute the convex hull of $S^*$.
\item For each patch of the hull that lies on $S^*$, add the associated patch of $S$
	to the kernel.
\item For each bitangent developable $D^*$ of the hull, compute the associated curve $D$ in primal space
  (see below).
\item If two bitangent developables $D_1^*$ and $D_2^*$ of the hull meet at a cusp curve, 
  add the lofting between their associated curves $D_1$ and $D_2$ to the kernel.\footnote{
    % This lofting is well defined, since $D_1$ and $D_2$ 
    % will necessarily have the same parameterization.
    Each line of this lofting is generated by a tangent plane in dual space pivoting 
    about a point of the cusp curve, from one developable to the other developable.}
\item If a bitangent developable $D^*$ neighbours a tritangent plane $T$ on the hull,
add the lofting between the associated curve $D$ and the tritangent plane's
associated point $T^*$ to the kernel.  This lofting between $D$ and 
$T^*$ is a conical surface.
\end{enumerate}

The curve $D$ dual to a bitangent developable $D^*$ is easily
computable, as follows.
Suppose that the developable $D^*$ is a lofting between $C^*_1(t)$ and $C^*_2(t)$,
where $C^*_1(t)$ is a cusp curve if $D^*$ is of type 1 and
a curve of tangency with $S^*$ if $D^*$ is of type 2,
while $C^*_2(t)$ is a cusp curve.
$D$ is the dual of the bitangent planes along $C^*_1(t)$.
For type 1 developables, the bitangent plane at $C^*_1(t)$ is the plane through
$C^*_1(t)$ and the tangent line of $C^*_2(t)$.
For type 2 developables, the bitangent plane at $C^*_1(t)$ is the same as the tangent plane
at $C^*_1(t)$.
$D$ lies in the interior of $S$ for type 1 developables,
and on $S$ for type 2 developables.

This concludes the reduction of kernel to convex hull.
The structural relationship between the kernel and hull that is revealed
by this study provides insight into these two fundamental geometric structures.

%-------------------------------------------------------------------------%
\section{Conclusion}
\label{sec-conclusion}
In this paper, we have presented algorithms for computing
the kernel of freeform surfaces.
The problem was first reformulated as a zero-set finding problem;
and it is then solved by computing the zero-set and
projecting the solutions onto a proper subspace.
We have also shown that there is an interesting geometric duality
between the kernel and the convex hull.


%------------------------------------------------------------------------%
\section*{ACKNOWLEDGMENTS}

This work was supported in part by the National Science Foundation under 
grant CCR-0203586, in part by the Israeli Ministry of Science
Grant No.~01--01--01509, and in part by
the Korean Ministry of Information and Communication (MIC) under
the Program of IT Research Center on CGVR.


\bibliographystyle{acmsiggraph}
\begin{thebibliography}{\protect\citename{Magnenat-Thalmann et~al\mbox{.}
  }1988}

% \bibitem[Names(Year)]{label} or \bibitem[Names(Year)Long names]{label}.
% (\harvarditem{Name}{Year}{label} is also supported.)
% Text of bibliographic item

\bibitem[\protect\citename{do Carmo }1976]{docarmo}
{\sc do Carmo, M.P. 1976.}
\newblock {\em Differential Geometry of Curves and Surfaces}.
\newblock Prentice-Hall, New Jersey, 1976.

\bibitem[\protect\citename{Elber and Cohen }1993]{Elber93}
{\sc Elber, G., and Cohen, E. 1993.}
\newblock Second-Order Surface Analysis Using Hybrid Symbolic and
Numeric Operators.
\newblock ACM Transactions on Graphics, Vol. 12, No.2, 1993, 
pp.~160--178.

\bibitem[\protect\citename{Elber and Kim }2001]{Elber2001a}
{\sc Elber, G., and Kim, M.-S. 2001.}
\newblock Geometric Constraint Solver
Using Multivariate Rational Spline Functions.
\newblock {\em Proc.~of ACM Symposium on Solid Modeling and Applications\/},
Ann Arbor, MI, June 4--8, 2001.

\bibitem[\protect\citename{Elber et~al\mbox{.} }2001]{Elber2001b}
{\sc Elber, G., Kim, M.-S., and Heo, H.-S. 2001.}
\newblock The Convex Hull of Rational Plane Curves.
\newblock {\em Graphical Models\/}, Vol.~63, No.~3, pp.~151--162, 2001.

% \bibitem[Hartshorne 77]{hartshorne}
%\bibitem[\protect\citename{Hartshorne }1997]{hartshorne}
%{\sc R.~Hartshorne.}
%{\em Algebraic Geometry}. Springer-Verlag, New York, 1977.

\bibitem[\protect\citename{Johnstone }2001]{Johnstone2001}
{\sc Johnstone, J. 2001.}
\newblock A Parametric Solution to Common Tangents.
\newblock {\em Proc.~of Int'l Conf.~on Shape Modeling and Applications\/},
Genova, Italy, May 7--11, 2001, pp.~240--249.

\bibitem[\protect\citename{Johnstone }2003]{jj03}
% JJ
{\sc Johnstone, J. 2003.}  
\newblock Kernels from Hulls, 
\newblock {\em Proc.~of the 41st ACM Southeast Conference}, 
Savannah, Georgia, 2003, pp.~354--358.

% \bibitem[Johnstone 04]{jj04tangsurf}
\bibitem[\protect\citename{Johnstone }2004]{jj04tangsurf}
{\sc Johnstone, J. 2004.}
\newblock The Bezier Tangential Surface System: 
a Robust Dual Representation of Tangent Space,
\newblock {\em Computing\/}, Vol.~72, Nos.~1--2, pp.~105--115, 2004.

\bibitem[\protect\citename{J\"uttler and Wagner }1999]{JW99}
{\sc J\"uttler, B., and Wagner, M. 1999.}
\newblock Rational Motion-based Surface Generation.
\newblock {\em Computer-Aided Design\/},
Vol.~31, No.~3, pp.~203--214, 1999.

\bibitem[\protect\citename{Pottmann and Wallner }2001]{Pottmann}
{\sc Pottmann, H., and Wallner, J. 2001.}
\newblock Computational Line Geometry.
\newblock Springer--Verlag Berlin Heidelberg, pp.~400--410, 2001.

\bibitem[\protect\citename{Preparata and Shamos }1985]{Prep85}
{\sc Preparata, F., and Shamos, M. 1985.}
\newblock {\em Computational Geometry: An Introduction\/}.
\newblock Springer-Verlag, New York, 1985.

\bibitem[\protect\citename{Seong et~al\mbox{.} }2004]{seong2004}
{\sc Seong, J.-K., Elber, G., Johnstone, J., and Kim, M.-S. 2004.}
\newblock The Conex Hull of Freeform Surfaces.
\newblock {\em Computing\/}, Vol.~72, Nos.~1--2, pp.~171--183, 2004.

\end{thebibliography}

%%%%%%%%%%%%%%%%%%%%%%%%%%%%%%%%%%%%%%%%%%%%%%%%%%%%%%%%%%%%%%%%%%%%%%%%%%%%%


\end{document}
