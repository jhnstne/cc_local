%\documentclass{elsart}

% Use the option doublespacing or reviewcopy to obtain double line spacing
\documentclass[doublespacing]{elsart}

% the natbib package allows both number and author-year (Harvard)
% style referencing;
%\usepackage{natbib}

% if you use PostScript figures in your article
% use the graphics package for simple commands
\usepackage{graphics}
% or use the graphicx package for more complicated commands
\usepackage{graphicx}
% or use the epsfig package if you prefer to use the old commands
\usepackage{epsfig}

% The amssymb package provides various useful mathematical symbols
\usepackage{amssymb}

\begin{document}

\newtheorem{definitionenv}{Definition}
\newenvironment{definition}{\begin{quote}\begin{definitionenv}}%
                           {\end{definitionenv}\end{quote}}
\newenvironment{defn2}{\begin{quote}\begin{definitionenv}}%
                           {\end{definitionenv}\end{quote}}

\newtheorem{propositionenv}{Proposition}
\newenvironment{proposition}{\begin{quote}\begin{propositionenv}}%
                           {\end{propositionenv}\end{quote}}

\newtheorem{theoremenv}{Theorem}
\newenvironment{theorem}{\begin{quote}\begin{theoremenv}}%
                           {\end{theoremenv}\end{quote}}

\newtheorem{lemmaenv}{Lemma}
\newenvironment{lemma}{\begin{quote}\begin{lemmaenv}}%
                           {\end{lemmaenv}\end{quote}}

\newtheorem{corollaryenv}{Corollary}
\newenvironment{corollary}{\begin{quote}\begin{corollaryenv}}%
                           {\end{corollaryenv}\end{quote}}

\newtheorem{observationenv}{Observation}
\newenvironment{observation}{\begin{quote}\begin{observationenv}}%
                           {\end{observationenv}\end{quote}}

\newenvironment{proof}{\par\smallskip\noindent{\bf Proof.}\
\ignorespaces}{\stopproof\ignorespaces\bigbreak}

\newenvironment{notepar}{\par\smallskip\noindent{}\
\ignorespaces}{\stopproof\ignorespaces\bigbreak}

\newcommand{\cal}{\mathcal}
\newcommand{\CH}{{\cal C}{\cal H}}
\newcommand{\inner}[2]{\left<{#1}, {#2} \right>}

\def\stopproof{\qquad\square}
\def\square{\vbox{\hrule height.2pt\hbox{\vrule width.2pt height5pt
\kern5pt \vrule width.2pt} \hrule height.2pt}}

\newcommand{\Comment}[1]{\relax}  % makes a "comment" (not expanded)
\newcommand{\QED}{\vrule height 1.4ex width 1.0ex depth -.1ex\ \vspace{.3in}} % square box
\newcommand{\lyne}[1]{\mbox{$\stackrel{\textstyle \leftrightarrow}{#1}$}}
\newcommand{\seg}[1]{\mbox{$\overline{#1}$}}
\newcommand{\prf}{\noindent{{\bf Proof}:\ \ \ }}
\newcommand{\choice}[2]{\mbox{\footnotesize{$\left( \begin{array}{c} #1 \\ #2 \end{array} \right)$}}}      
\newcommand{\scriptchoice}[2]{\mbox{\scriptsize{$\left( \begin{array}{c} #1 \\ #2 \end{array} \right)$}}}
\newcommand{\tinychoice}[2]{\mbox{\tiny{$\left( \begin{array}{c} #1 \\ #2 \end{array} \right)$}}}

%\newtheorem{theorem}{Theorem}	
%\newtheorem{rmk}[theorem]{Remark}
%\newtheorem{example}[theorem]{Example}
%\newtheorem{conjecture}[theorem]{Conjecture}
%\newtheorem{claim}[theorem]{Claim}
%\newtheorem{notation}[theorem]{Notation}
%\newtheorem{lemma}[theorem]{Lemma}
%\newtheorem{corollary}[theorem]{Corollary}
%\newtheorem{defn2}[theorem]{Definition}
%\newtheorem{observation}[theorem]{Observation}
% \input{header}

%\setlength{\oddsidemargin}{0pt}
%\setlength{\topmargin}{0in}	% should be 0pt for 1in
%\setlength{\textheight}{8.6in}
%\setlength{\textwidth}{6.875in}
%\setlength{\columnsep}{5mm}	% width of gutter between columns

\begin{frontmatter}

% Title, authors and addresses

% use the thanksref command within \title, \author or \address for footnotes;
% use the corauthref command within \author for corresponding author footnotes;
% use the ead command for the email address,
% and the form \ead[url] for the home page:
% \title{Title\thanksref{label1}}
% \thanks[label1]{}
% \author{Name\corauthref{cor1}\thanksref{label2}}
% \ead{email address}
% \ead[url]{home page}
% \thanks[label2]{}
% \corauth[cor1]{}
% \address{Address\thanksref{label3}}
% \thanks[label3]{}

\title{The Convex Hulls and Kernels\\ of Freeform Surfaces}

% use optional labels to link authors explicitly to addresses:
% \author[label1,label2]{}
% \address[label1]{}
% \address[label2]{}

\author{}

\address{}

\begin{abstract}
% Text of abstract

\noindent 
% JJ
We present algorithms for computing the convex hull and kernel of
freeform rational surfaces.  The convex hull problem is reformulated
as one of finding the zero-sets of polynomial equations;
using these zero-sets
we characterize developable surface patches and planar patches
that belong to the boundary of the convex hull.
The kernel computation can also be reduced to a zero-set finding problem.
% JJ
Using a plane-point duality, this paper explores a 
duality relationship between the kernel and the convex hull.

\end{abstract}

\begin{keyword}
Convex hull, kernel, duality, common tangent, zero-set finding,
freeform rational surface, B-spline, symbolic computation
\end{keyword}

\end{frontmatter}

%----------------------------------------------------------------------%
% main text
\section{Introduction}
\label{sec-introduction}

Computing the convex hull of a freeform surface
is a challenging task in geometric modeling.
Because of the difficulty in computing the exact convex hull
of a spline surface, the convex hull of its control points is usually
used as a simple, yet rough, approximation to the convex hull.
When a tighter bound is needed for the convex hull, we can subdivide
the surface into smaller pieces and then union the convex hulls of
the control points of these small pieces. 
This simple approach will require a large number of subdivisions 
until the approximating convex hull converges to the exact convex hull
within a certain reasonable bound.  In this paper, we present
an algorithm that computes the convex hull of a rational surface
without resorting to a polygonal approximation of the given surface
or the subdivisions.

The kernel of a closed surface is the set of interior points
that can see every point of the surface.
It is a convex connected subset of the surface's interior.
The kernel is an important structure to understand in developing a theory of visibility.
For example, interpreting the surface as the boundary of a room,
a point of the kernel would be an ideal location to place a light, camera, or robot,
since the entire room would be visible.

The convex hull computation has applications in many important
geometric problems such as interference checking in motion planning
and object culling in graphics rendering.  Interference between two convex
objects is considerably easier to test than between two non-convex objects.
Consequently, convex hulls of general non-convex objects (or their rough
approximations such as spheres and axis-aligned bounding boxes) are often used
in a simple first test for checking the interference between the two
original objects.   (If there is no interference between the convex hulls,
% JJ
it is guaranteed that the original objects have no interference.)
These applications motivated the development of many efficient
convex hull algorithms in computational geometry~\cite{Prep85}.
The previous work, however, has been mostly limited to computing
the convex hull of discrete points, polygons, and
polyhedra~\cite{Graham83,Lee83,Prep85}.

There are a few previous algorithms that can compute the convex hulls
of rational curves in the plane.  Some of these algorithms are quite
theoretical~\cite{Kim91,Souvaine90,Schaeffer87} in the sense that
there are certain delicate issues that must be resolved for the efficiency
and robustness of these algorithms to hold in actual implemention.
More practical approaches were taken in two recent results reported
by Elber et al.~\cite{Elber2001b} and Johnstone~\cite{Johnstone2001}.
The convex hull algorithm we present in this paper is based on
extending our recent result~\cite{Elber2001b} to the three-dimensional case.

Given a point $S(u,v)$ on a rational surface, let $T(u,v)$ denote
the tangent plane of the surface at the point $S(u,v)$.
For the point $S(u,v)$ to be on the boundary of the convex hull,
the surface $S$ should be completely contained in one side of
the tangent plane $T(u,v)$.  Thus the Gaussian curvature of $S(u,v)$ 
must be non-negative.  For the sake of simplicity of presentation,
we assume that the Gaussian curvature is positive
on this surface point $S(u,v)$ (i.e., the surface $S$ contains
no line passing through the point $S(u,v)$).
% which means
%that the tangent plane $T(u,v)$ intersects the surface $S$
%at no other point $S(s,t)$, for $(s,t) \neq (u,v)$.

% JJ
Let $D$ denote the region in the $uv$-domain where the surface
patch $S(u,v)$ belongs to the boundary of the convex hull.
In this paper, we show that the boundary of this region $D$
can be computed in terms of the zero-set of three equations in
four variables $u,v,s,t$, and sometimes in terms of the zero-set of
two equations in three variables $u,v,t$.  The boundary curves
of the surface $S$ may also contribute to the convex hull of $S$.
Let $I$ denote the parameter interval for a boundary curve segment
that appears on the boundary of the convex hull of the surface $S$.
The end points of the interval $I$ can be computed
by solving three equations in three variables,
and sometimes by solving two equations in two variables.
The boundary of the convex hull may contain some triangles,
each of which is obtained from a tangent plane touching at three 
different points of the surface.  These planes can be computed
by solving a system of six equations in six variables.

The interior of the kernel of a freeform closed surface $S(u,v)$
is characterized by the set of points that are never contained
in the tangent planes of the surface.  The containment of
a point $P = (x,y,z)$ in the tangent plane of $S(u,v)$ can be
formulated as a polynomial equation in five variables:
\[
\inner{(x,y,z)-S(u,v)}{N(u,v)} = 0,
\]
where $N(u,v)$ is the normal vector of $S(u,v)$.
The projection of the zero-set of this equation
(onto the $xyz$-space) covers all non-interior points of the kernel.
Thus the interior of the kernel is the volume
that is uncovered by the projection of the zero-set.

The duality between planes and points in the three-dimensional space
implies a simular duality relationship between the kernel and
the convex hull.  This paper explores the details of the duality
and presents an algorithm for computing the kernel of a closed surface
using the convex hull of its dual tangential surface.

The computation of kernels from convex hulls offers a different point of
view on the computation of the kernel, allowing all attention
to be placed on the development of an algorithm
for the efficient and robust computation of the convex hull.
We can appeal to the Voronoi diagram and Delaunay triangulation
for a comparable example.  There are algorithms for the computation of
each of these structures, but many still prefer to always compute
the Voronoi diagram by computing the Delaunay triangulation and then dualizing.

The rest of this paper is organized as follows.
In Section~2, we reduce the convex hull problem to
computing the zero-sets of polynomial equations.
The convex hull of rational space curves is needed
to deal with the boundary curves of a rational surface.
This is an important problem by itself;
Section~3 thus presents an algorithm for computing
the convex hull of a rational space curve.
Section~4 computes tri-tangent planes by solving
a system of six equations in six variables.  
Section~5 discusses how to combine all these components
to construct the boundary of the convex hull.
In Section 6, we present algorithms for computing the kernels
of freeform closed curves and surfaces.
The duality relationship between the kernel and the convex hull
is explored in depth in Section 7.
Finally, in Section~8, we conclude this paper.

%----------------------------------------------------------------------%
% JJ
\section{The Convex Hull of Freeform Surfaces}
\label{sec-ch-surface}

In this section, we consider the convex hull of freeform surfaces.
Although we will use one concave surface in the following discussion
for clarity,  one may consider the second surface point $S(s,t)$
as a point on another surface.

Let $S(u,v)$ be a regular $C^1$-continuous rational surface. 
Consider the tangent plane of $S$ at $S(u,v)$ as a moving plane
while continuously touching the surface tangentially.
Then, any surface point $S(u,v)$
such that the surface is completely contained in one side of
the tangent plane is on the boundary of the convex hull of
the surface $S$. On the other hand, if the tangent plane at $S(u,v)$ 
intersects the surface at any other surface point $S(s,t)$, then
the surface point $S(u,v)$ cannot be on the boundary of
the convex hull of the surface $S$.
Note, however, that one must be careful of common bi-tangent planes because 
% JJ
a surface point at which the tangent plane is also tangent to some
other surface point could be on the boundary of the convex hull. 

The tangent plane of $S$ at $S(u,v)$ contains
another surface point $S(s,t)$ if and only if 
\begin{eqnarray*}
   {\mathcal F}(u,v,s,t) &=& \inner{S(u,v)-S(s,t)}{N(u,v)}\\
   &=& |S(u,v)-S(s,t)\ \ S_u(u,v)\ \ S_v(u,v)|\\
   &=& 0,
\end{eqnarray*}
where $N(u,v) = S_u(u,v) \times S_v(u,v)$ is the normal vector field 
of the surface $S$. Then, the set of surface points, for which 
the tangent plane intersects the surface $S$ at no other points, 
is defined as follows:
\begin{eqnarray*}
\CH{}^o(S) = \left\{~S(u,v)~\left|~ 
{\cal F}(u,v,s,t) \neq 0,~\forall (s,t) \neq (u,v) \right. \right\}.
\end{eqnarray*}
This set $\CH{}^o(S)$ is clearly a subset of the boundary of the convex hull
of $S$ and it is also a subset of the surface $S$ itself:
\begin{equation}
    \CH{}^o(S) \subset \CH{}(S) \cap S.
\label{eqn-cho-ch}
\end{equation}
The difference $(\CH{}(S) \cap S) \setminus \CH{}^o(S)$ contains some extra
points such as: (i) the boundary curve of each connected surface patch of
$\CH{}^o(S)$ and, sometimes, (ii) the boundary curve of $S$ itself if $S$ is 
not a closed surface.  (For now, we assume that each tangent plane of $S$ is
% JJ
tangent to $S(u,v)$ at no more than two surface points;
thus all isolated points and curve segments of
$\CH{}(S) \cap S$ must be on the boundary curve of $S$.)

As mentioned above, the zero-set of 
${\mathcal F}(u,v,s,t) = 0 \wedge (u,v) \neq (s,t)$ in
the $uvst$-domain cannot contribute to the boundary of the convex hull
% JJ
of the surface $S(u,v)$. That is, if the point $(u,v)$ in the parametric
domain falls into the projection of this zero-set, then the corresponding
surface point $S(u,v)$ cannot be on the boundary of the convex hull 
of $S$. The boundary of the 'uncovered' region of the $uv$-plane
(under this projection) is characterized as the projection of the 
$st$-silhouette curves (along the $st$-direction) of the zero-set. This means
that the $s$-partial derivative and the $t$-partial derivative must 
simultaneously vanish along the silhouette curve, which can be 
characterized as the
intersection of the following three hypersurfaces in the $uvst$-space:
\begin{eqnarray}
   {\cal F}(u,v,s,t) &=& 0, \label{eqn-f-uvst} \\
   {\cal F}_s(u,v,s,t) &=& 0, \label{eqn-s-partial}\\
   {\cal F}_t(u,v,s,t) &=& 0. \label{eqn-t-partial}
\end{eqnarray}
One should consider only the solutions satisfying $(u,v) \neq (s,t)$
from the above equations. Furthermore, if $(u,v)$ and $(s,t)$ are
on the same surface, only $(u,v) > (s,t)$ should be considered, in
lexicographic order.

Now, consider Equations (\ref{eqn-s-partial}) and (\ref{eqn-t-partial})
differently:
\begin{eqnarray*}
{\cal F}_s(u,v,s,t) &=& \frac{\partial}{\partial s} 
\inner{S(u,v)-S(s,t)}{N(u,v)}, \\
&=& - \inner{S_s(s,t)}{N(u,v)} = 0,\\
{\cal F}_t(u,v,s,t) &=& - \inner{S_t(s,t)}{N(u,v)} = 0.
\end{eqnarray*}
Therefore, Equations (\ref{eqn-s-partial}) and (\ref{eqn-t-partial})
characterize the condition that the tangent plane at the surface point
$S(u,v)$ is also tangent to the surface at the other point $S(s,t)$.

Having three Equations (\ref{eqn-f-uvst}), (\ref{eqn-s-partial}), and
(\ref{eqn-t-partial}) in four variables, one gets a univariate curve
as the simultaneous solution in the $uvst$-space. This solution curve can 
be parameterized by a variable $\alpha$:
\[
   (u(\alpha),v(\alpha),s(\alpha),t(\alpha)).
\]
Denote by $T_\alpha$ the common bi-tangent plane of $S$
at $S(u(\alpha),v(\alpha))$ and $S(s(\alpha),t(\alpha))$.
Then, the surface patch that bounds the convex hull of the surface $S$ can
be constructed by connecting the corresponding surface points
$S(u(\alpha),v(\alpha))$ and $S(s(\alpha),t(\alpha))$ by line segments.
All these line segments correspond to the common bi-tangent plane
$T_\alpha$. Consequently, the surface patch that bounds the convex hull
of $S$ is the envelope surface of the tangent planes $T_\alpha$, which is
a developable surface~\cite{docarmo}.  We consider only solution 
points which satisfy $\inner{N(u,v)}{N(s,t)} > 0$. Redundant solutions
$(u,v,s,t)$, where $\inner{N(u,v)}{N(s,t)} < 0$, should be purged away. 
Figure~\ref{fig-ch-surface-1} shows a developable surface patch 
that bounds the convex hull of a B-spline surface.

\begin{figure}
    \begin{tabular}{cc}
    \psfig{width=2.7in,figure={figures/ch-two-1.ps}} & 
    \psfig{height=2.7in,figure={figures/ch-two-2.ps}} \\
    {\large (a)}  &  {\large (b)} \\
    \begin{picture}(0,0)
        \put(20,80){\large $S(u,v) = S(s,t)$}
	\put(240,75){\large $S(s,t)$}
	\put(190,230){\large $S(u,v)$}
    \end{picture}
    \end{tabular}
\vskip -0.5in
    \caption{\textsf{(a) A developable surface patch on the boundary of
    the convex hull of a rational surface $S(u,v)$ and (b)
    a similar patch between two rational surfaces $S(u,v)$ and $S(s,t)$.}}
    \label{fig-ch-surface-1}
\vskip 0.2in
\end{figure}

In the above discussion, we assumed that both the surface points
$S(u,v)$ and $S(s,t)$ are interior points of the surface $S$.
When one of the two points is located on the boundary curve of $S$,
the above characterization needs some refinements.
We assume that the second point $S(s_0,t)$ is located
on the boundary curve of $S$ with a fixed parameter $s=s_0$.
Then, the above characterization can be modified as follows
\begin{eqnarray}
   {\cal F}(u,v,s_0,t) &=& 0, \label{eqn-ch-bndry}\\
   {\cal F}_t(u,v,s_0,t) &=& 0. \nonumber
\end{eqnarray}
The zero-set of Equation (\ref{eqn-ch-bndry}) generates a silhouette curve 
in the $uvt$-space, which can be parameterized by a variable $\alpha$:
\[
   (u(\alpha),v(\alpha),t(\alpha)).
\]
The bounding developable surface patch can be constructed
by connecting the corresponding surface points
$S(u(\alpha),v(\alpha))$ and $S(s_0,t(\alpha))$ by line segments.

The curve $(u(\alpha),v(\alpha))$ outlines the boundary
on the surface patch $S(u,v)$ that belongs to the boundary of
the convex hull of the surface $S$.  The end points of
the curve segment $S(s_0,t)$ that belong to the boundary
of the convex hull are computed by solving the following
three equations in three variables:
\begin{eqnarray*}
   {\cal F}(u,v,s_0,t) &=& 0,\\
   {\cal F}_u(u,v,s_0,t) &=& 0,\\
   {\cal F}_v(u,v,s_0,t) &=& 0.
\end{eqnarray*}

The second surface point may be a corner point $S(s_0,t_0)$ as well.
In this case, one needs to solve the following bivariate equation:
\[
   {\cal F}(u,v,s_0,t_0) = 0,
\]
which characterizes the tangent planes from $S(s_0,t_0)$
to the freeform surface $S(u,v)$.
The zero-set generates a curve in the $uv$-plane,
which can be parameterized by a variable $\alpha$:
\[
   (u(\alpha),v(\alpha)).
\]
The bounding developable surface patch is constructed
as a conical surface that has its apex at the corner point
$S(s_0,t_0)$ and is generated by the curve
$S(u(\alpha),v(\alpha))$ on the surface $S$.

The convex hull of a freeform surface also contains
the convex hull of its boundary space curves.
The approach to computing the convex hull of a rational space curve
is slightly different from the above three cases.
There is no unique tangent plane at a point on a space curve.
The tangent planes form a one-parameter family of planes,
each of which contains the tangent line of the space curve.
Representing all tangent planes at a curve point requires
one additional parameter for the rotation about the tangent line.
Moreover, computing the convex hull of a space curve is 
an interesting problem by itself.
The next section is thus devoted to this special case.

%----------------------------------------------------------------------%
\section{The Convex Hull of a Space Curve}
\label{sec-space-curve}

In this section, we assume that the rational space curve $C(t)$ is
non-planar; that is, its torsion is non-zero at all curve points.
(The convex hull of a planar curve can be computed using the results
of Elber et al.~\cite{Elber2001b} and Johnstone~\cite{Johnstone2001}.)
The convex hull of a freeform space curve $C(t)$ is characterized
by the planes that are tangent to the curve at two different locations.

Assume that there is a plane tangent to the curve $C$ at two different
locations $C(s)$ and $C(t)$.  Then the three vectors of
$C'(s)$, $C'(t)$, and $C(s)-C(t)$ are parallel to the tangent plane.
Hence, the determinant of a matrix consisting of these three vectors 
must vanish:
\begin{eqnarray*}
{\cal G}(s,t) &=& |C(s)-C(t)\ \ C'(s)\ \ C'(t)|\\
&=& \inner{C(s)-C(t)}{C'(s) \times C'(t)}\\
&=& 0,
\end{eqnarray*}
which is a necessary condition for the existence of
a common tangent plane at the two curve points $C(s)$ and $C(t)$.
Once again, we need to consider only the solutions satisfying $s \neq t$.

The zero-set of the above equation can be locally parameterized 
by $t$ (or by $s$): ${\cal G}(s(t),t) = 0$
(or ${\cal G}(s,t(s)) = 0$).
By connecting the corresponding curve points $C(s(t))$ and $C(t)$
by line segments, we can construct the surfaces
that bound the convex hull of the space curve $C(t)$. Again, these
surfaces are developable surfaces because all the line segments 
connecting the points $C(s(t))$ and $C(t)$ are from the common 
tangent planes and so the surfaces are envelope surfaces of these
common tangent planes~\cite{Pottmann}.
Figure~\ref{fig-ch-curve1}(a) shows a space curve and
Figure~\ref{fig-ch-curve1}(b) shows its convex hull bounded by
developable surface patches.

We now consider how to compute the interior points of
% JJ
the curve segment $C(t)$ that belong to the boundary of
the convex hull.
Let $T(t)$ denote the common tangent plane
at two corresponding curve locations $C(s(t))$ and $C(t)$.
This plane may not bound the convex hull of the curve $C$
if the curve $C$ intersects this plane transversally
at a third point $C(u)$, where $ u \neq s(t)$ and $u \neq t$.
This condition can be formulated as follows
\begin{eqnarray*}
   {\cal H}(u,s,t)
   &=& |C(u)-C(t)\ \ C'(s)\ \ C'(t)|\\
   &=& \inner{C(u)-C(t)}{C'(s) \times C'(t)}\\
   &=& 0,
\end{eqnarray*}
but
\begin{eqnarray*}
   {\cal H}_u(u,s,t)
   &=& |C'(u)\ \ C'(s)\ \ C'(t)|\\
   &=& \inner{C'(u)}{C'(s) \times C'(t)}\\
   &\neq& 0,
\end{eqnarray*}
where ${\cal H}_u$ is the $u$-partial of ${\cal H}$.

The redundancy of the solution $(s,t)$
changes only through the configurations where
three curve points $C(u),C(s),C(t)$ admit a common tangent plane.
The condition is characterized by the following three equations:
\begin{eqnarray*}
{\cal G}(s,t) &=& 0,\\
{\cal H}(u,s,t) &=& 0,\\
{\cal H}_u(u,s,t) &=& 0.
\end{eqnarray*}

\begin{figure}
    \begin{tabular}{cc}
    \psfig{width=2.5in,figure={figures/curve1.ps}} & 
    \psfig{width=2.5in,figure={figures/ch-curve1.ps}} \\  
    {\large (a)}  &  {\large (b)} \\
    \end{tabular}
    \caption{\textsf{$C(t)$ is a rational space curve with non-zero torsion:
    (a) shows a rational space curve $C(t)$; and
    (b) shows the developable surface patches
    that bound the convex hull of $C(t)$.}}
\vskip 0.2in
    \label{fig-ch-curve1}
\end{figure}

In the above discussion, we assumed that both the curve points
$C(s)$ and $C(t)$ are interior points of the curve $C$.
When one of the two points is an end point $C(s_0)$ of the curve $C$,
we need a modification to this approach.
Let $T(t)$ denote the tangent plane from a fixed point $C(s_0)$ to
the curve $C(t)$.  Assume that the curve $C$ transversally intersects
the tangent plane $T(t)$ at a third point $C(u)$, $u \neq t$.  Then,
we have the following condition:
\begin{eqnarray*}
   && {\cal K}(u,t)\\
   &=& |C(u)-C(s_0)\ \ C(t)-C(s_0)\ \ C'(t)|\\
   &=& \inner{C(u)-C(s_0)}{(C(t)-C(s_0)) \times C'(t)}\\
   &=& 0,
\end{eqnarray*}
but
\begin{eqnarray*}
   && {\cal K}_u(u,t)\\
   &=& |C'(u)\ \ C(t)-C(s_0)\ \ C'(t)|\\
   &=& \inner{C'(u)}{(C(t)-C(s_0)) \times C'(t)}\\
   &\neq& 0.
\end{eqnarray*}

The redundancy of the solution $t$
changes only through the configurations where
three curve points $C(s_0),C(u),C(t)$ admit a common tangent plane.
The condition is characterized by the following two equations:
\begin{eqnarray*}
{\cal K}(u,t) &=& 0,\\
{\cal K}_u(u,t) &=& 0.
\end{eqnarray*}

Figure~\ref{fig-ch-curve2} shows a rational space curve $C(t)$
and its convex hull which is bounded by two conical developable surfaces,
where each surface has its apex at an end point of the curve and
is generated by the curve $C(t)$.

\begin{figure}
\bigskip
\bigskip
\begin{center}
    \begin{tabular}{c}
    \psfig{width=2.7in,figure={figures/ch-curve2.ps}} \\
    %\begin{picture}(0,0)
    %    \put(70,60){\large $C(t)$}
    %\end{picture}
    \end{tabular}
    \caption{\textsf{$C(t)$ is a rational space curve; its convex hull 
	is bounded by two conical developable surface patches.}}
    \label{fig-ch-curve2}
\end{center}
\vskip 0.2in
\end{figure}

%----------------------------------------------------------------------%
\section{Tri-Tangent Planes}
\label{sec-tri-tangencies}

In Section \ref{sec-ch-surface}, we considered bi-tangent planes.
In this section, we consider how to deal with tri-tangent planes.
% JJ
The bi-tangent condition prescribes a curve in the parametric domain.
The case of tri-tangency results in a solution with zero dimension. 
To form the complete convex hull, one needs to combine 
these tri-tangent planes with the bi-tangent developable surfaces which
bound the convex hull.

We can extend the three Equations (\ref{eqn-f-uvst})--(\ref{eqn-t-partial})
from Section \ref{sec-ch-surface} to the tri-tangent condition. 
Let the three tangent points be $S(u,v), S(s,t)$ and $S(m,n)$.
Then, the three Equations 
(\ref{eqn-f-uvst}), (\ref{eqn-s-partial}), and (\ref{eqn-t-partial}) 
constrain the tangent plane at $S(u,v)$ to be tangent to 
surface $S(s,t)$. We now add three more equations to constrain 
this tangent plane to be tangent to the third surface point 
$S(m,n)$ as well.  Consequently, we have six equations in six variables:
\begin{eqnarray}
   {\cal F}(u,v,s,t) &=& 0, \label{eqn-f-uvst-new} \\
   {\cal F}_s(u,v,s,t) &=& 0, \label{eqn-s-partial-new}\\
   {\cal F}_t(u,v,s,t) &=& 0. \label{eqn-t-partial-new} \\
{\cal F}(u,v,m,n) &=& \inner{S(u,v)-S(m,n)}{N(u,v)} \label{eqn-g-uvmn} = 0,\\
%&=& 0, \nonumber \\
{\cal F}_m(u,v,m,n) &=& 0, \label{eqn-m-partial}\\
{\cal F}_n(u,v,m,n) &=& 0. \label{eqn-n-partial}
\end{eqnarray}
% JJ
These six equations (\ref{eqn-f-uvst-new})--(\ref{eqn-n-partial})
in six variables have a zero dimensional solution or a finite set of points.
An example for common tri-tangent planes to three surfaces is shown in
Figure~\ref{fig-three-tangent}(a).

\begin{figure}
    \begin{tabular}{cc}
    \psfig{width=2.7in,figure={figures/ch-three-tang.ps}} & 
    \psfig{width=2.7in,figure={figures/ch-three-trim.ps}} \\
    {\large (a)}  &  {\large (b)}
    \end{tabular}
    \caption{\textsf{(a) All the common tri-tangent planes to three surfaces. 
	Note
	that surface's topology don't need to be same. (b) One developable 
	surface which bounds first two convex parts of the surface and 
	the plane (triangle) which is common
	tangent to the surface. The boundary of the developable
	surface exactly intersect with the end point of the common 
	tri-tangent triangle.}}
    \label{fig-three-tangent}
\vskip 0.2in
\end{figure}

%-----------------------------------------------------------------------%
\section{Trimming and Combining the Convex Hull}
\label{sec-trim-combine}

As in the case of Section~\ref{sec-ch-surface}, the tri-tangent planes
include some redundant solutions.
The redundancy can be eliminated using the following three conditions:
\begin{eqnarray}
& & \inner{N(u,v)}{N(s,t)} > 0 \label{eqn-trim1} \\
& & \inner{N(u,v)}{N(m,n)} > 0 \label{eqn-trim2} \\
% JJ
& & \inner{S(x,y) - S(u,v)}{N(u,v)} \neq 0,\ 
\forall (x,y) \neq (u,v), (s,t), (m,n). \label{eqn-trim3}
\end{eqnarray}
The first two Equations (\ref{eqn-trim1}) and (\ref{eqn-trim2}) enforce 
the tri-tangent planes to have their normal vectors in the same direction. 
Then, Equation (\ref{eqn-trim3}) constrains the planes
not to pass through the interior of the convex hull.
%Equations (\ref{eqn-trim1}) and (\ref{eqn-trim2}) assume that
%the input surfaces are properly oriented.

To complete the convex hull construction, one needs to compute 
bi-tangency for each pair of surface parts: the bi-tangency of $S(u,v)$ and 
$S(s,t)$, the bi-tangency of $S(s,t)$ and $S(m,n)$, and the bi-tangency
of $S(m,n)$ and $S(u,v)$. Not all the bi-tangent developable surfaces
contribute to the boundary of the final convex hull. One needs to trim out
certain parts of these developable surfaces according to
the common tri-tangent planes to the three surface parts.

Because all the developable surface patches, which form the convex hull of 
two surface parts as mentioned in Section \ref{sec-ch-surface}, bound
the convex hull of each pair of surface parts, the boundary curves of these 
developable surfaces always intersect with tri-tangent planes
at the corresponding surface points (see Figure~\ref{fig-three-tangent}(b)).
In fact, we can always determine the trimming line of each developable surface.
As all the developable surfaces are generated by a moving bi-tangent plane 
between two surfaces, the trimmed developable surface is continuously 
connected to a tri-tangent plane.
Figures \ref{fig-ch-three1} and \ref{fig-ch-three2} show
two examples of the convex hull.

\begin{figure}
\begin{center}
    \begin{tabular}{cc}
    \psfig{width=3.2in,figure={figures/s.ps}}
    \psfig{width=2.7in,figure={figures/ch-three-1.ps}} \\
    \end{tabular}
    \caption{\textsf{(a) A surface with three convex parts;
        and (b) its convex hull.}}
    \label{fig-ch-three1}
\end{center}
\end{figure}

\begin{figure}
\begin{center}
    \begin{tabular}{c}
    \psfig{width=3.6in,figure={figures/ch-three-2.ps}} \\
    \end{tabular}
    \caption{\textsf{The convex hull of three ellipsoids.}}
    \label{fig-ch-three2}
\end{center}
\bigskip
\bigskip
\end{figure}


%----------------------------------------------------------------------%
\section{The Kernel of a Freeform Curve and a Freeform Surface}
\label{sec-kernel}

In this section, we present an algorithm for computing the kernel of
a freeform surface, which is closely related to the convex hull computation.
For the clarity of explanation, we first consider the kernel of
a freeform curve and generalize its construction to the case of
computing the kernel of a closed surface.
There is a dual relationship between the kernel and the convex hull.
More details will be discussed in the next section
% JJ
(see also Johnstone \cite{jj03}).

\subsection{The Kernel of a Freeform Curve}
\label{subsec-kernel-curve}

The kernel of a closed curve in the plane is the set of points
that can see every point of the curve.
%A self-intersecting curve always has an empty kernel
%since no point can see every point in the neighborbood of
%a self-intersection point.
Let $C(t) = (x(t),y(t))$ be a $C^1$-continuous planar rational curve;
and $P$ be a point in the plane.
Assume that $P$ is located on the tangent line of the curve at $C(t)$:
\[
\inner{P-C(t)}{N(t)} = 0,
\]
where $N(t) = (y'(t),-x'(t))$ is the normal vector of $C(t)$.

If the curve $C$ has non-zero curvature at the tangential point $C(t)$,
the point $P$ cannot see at least a certain part of the curve
unless $P$ is the tangent point $C(t)$ itself.
At an inflection point, some points on the tangent line may see
every point of the curve, at least in the local neighborhood of
the tangent point $C(t)$.  Since there are only a finite number of
% JJ
such tangent lines and they cannot form a planar region,
% JJ
tangent lines at inflection points may contribute to
the boundary of the kernel, but not to the interior of the kernel.

Now consider a point $P$ that is never covered by a tangent line
of the curve:
\begin{eqnarray}
\inner{P - C(t)}{N(t)} \neq 0,\quad \forall\ t.
\label{def-krnl-crv}
\end{eqnarray}
The point $P$ should be in the kernel of the curve $C$.
Otherwise, there exists a curve point $C(t_1)$ that $P$ cannot see.
In other words, a certain segment of $C$ blocks the sight of $C(t_1)$
from the view point $P$.  Now consider the line connecting $P$ and
$C(t_1)$ and rotate the line about the point $P$.  Since the curve
is closed, there is a curve point $C(t)$ where the line touches
tangentially.  Then the point $P$ is contained in the tangent line
at this point $C(t)$, which is a contradiction.

Now, assume that $P$ is a point in the interior of the kernel.
Then this point is never on the tangent line of the curve; and
% JJ
the point $P$ satisfies Equation (\ref{def-krnl-crv}),
which is a characterizing condition for the interior of the kernel.
 
Consider the following single equation in three variables, 
\begin{eqnarray}
{\cal T}(x, y, t) = \inner{(x, y) - C(t)}{N(t)} = 0. \label{eqn-kernel-crv}
\end{eqnarray}
A plane point $P_0 = (x_0, y_0)$ is not included in the interior of
the kernel if and only if there exists
$t_0$ such that ${\cal T}(x_0, y_0, t_0) = 0$. Having one equation in 
three variables, the zero-set of ${\cal T}(x, y, t) = 0$ is a bivariate 
surface in the $xyt$-space. 
Figure~\ref{fig-kernel-1}(a) illustrates one example of the zero-set for the 
planar curve shown in white.  Now, by projecting the zero-set along
the $t$-direction onto the $xy$-plane, the uncovered region in the $xy$-plane 
% JJ
is the region of the $xy$-plane that satisfies (\ref{def-krnl-crv})
and hence is in the interior of the kernel of $C(t)$
(see Figure \ref{fig-kernel-1}(b)).
Another example is shown in Figure~\ref{fig-kernel-crv}.
As one can see from these examples, the boundary of a kernel consists
of some segments of the curve and tangent line segments
at certain inflection points of the curve.

\begin{figure}
    \begin{tabular}{cc}
    \psfig{width=2.7in,figure={figures/kernel-zero-1.ps}} & 
    \psfig{width=2.7in,figure={figures/kernel-zero-2.ps}} \\
    {\large (a)}  &  {\large (b)}
    \end{tabular}
    \caption{\textsf{(a) The zero-set of Equation (\ref{eqn-kernel-crv}) 
	for the white-colored planar curve. 
	(b) Top (along the $t$-direction) view of the 
	zero-set. The uncovered region of this zero-set is
	the kernel of the planar curve.}}
    \label{fig-kernel-1}
\vskip 0.3in
\end{figure}

\begin{figure}
    \begin{tabular}{c}
    \psfig{width=4.3in,figure={figures/kernel-crv-1.ps}} \\
%    \psfig{width=2.7in,figure={figures/kernel-open-crv.ps}} \\
%   {\large (a)}  &  {\large (b)}
    \end{tabular}
    \caption{\textsf{The kernel of a planar curve.}}
%   (b) The kernel of an open curve.}}
    \label{fig-kernel-crv}
\end{figure}

\subsection{The Gamma Kernel of a Freeform Curve}
\label{subsec-gamma-kernel-curve}

It is also interesting to consider an angle-variant of the kernel.
Instead of considering the region uncovered by the tangent lines,
we may consider a similar region uncovered by rotated lines
that make angle $\gamma$ with the tangent lines.
This generates a convex region that is contained in the kernel.
The $\gamma$-kernel can also be defined as follows:

\begin{definitionenv}
Let $C(t)$ be a $C^1$-continuous planar curve.  A point $P$ is
in the $\gamma$-kernel of $C(t)$ if $P$ has a line of sight with every point 
of $C(t)$ that makes at least angle $\gamma$ with $C'(t)$,
the tangent of $C(t)$.
\end{definitionenv}
If $\gamma = 0$, then the $\gamma$-kernel is reduced to the regular kernel 
of $C(t)$. The $\gamma$-kernel of $C(t)$ can be computed by the zero-set 
of the following equation:
\begin{eqnarray*}
{\cal T}_1(x, y, t) = \inner{(x,y) - C(t)}{R_{\pm\gamma} [N(t)]} = 0, 
\end{eqnarray*}
where $R_{\pm\gamma} [\ ]$ is the rotation operation by degree $\pm\gamma$.
Figure~\ref{fig-r-kernel} shows two simple examples of $\gamma$-kernel.


\begin{figure}
    \begin{tabular}{cc}
    \psfig{width=2.7in,figure={figures/kernel-r-1.ps}} & 
    \psfig{width=2.7in,figure={figures/kernel-r-2.ps}} \\
    {\large (a)}  &  {\large (b)}
    \end{tabular}
    \caption{\textsf{Two $\gamma$-kernels. (a) $\gamma = 20^\circ$. Note that
	the regular kernel of a (convex) ellipse is the whole region surrounded
	by the ellipse. 
	(b) $\gamma = 5^\circ$. Compare this with Figure 
	\ref{fig-kernel-crv}~(a) which is the regular kernel of the same 
	curve.}}
    \label{fig-r-kernel}
\vskip 0.2in
\end{figure}

\subsection{The Kernel of a Freeform Surface}
\label{subsec-kernel-surface}

We consider the problem of computing the kernel of a freeform closed surface.
In the surface case, we employ the following multivariate equation:
\begin{eqnarray}
{\cal L}(x,y,z,u,v) = \inner{(x,y,z) - S(u,v)}{N(u,v)} = 0,
\label{eqn-kernel-srf}
\end{eqnarray}
where $N(u,v) = S_u(u,v) \times S_v(u,v)$ is the normal vector field of the 
surface $S(u,v)$.   Consider the projection of the zero-set of
the above equation onto the $xyz$-space.
The uncovered volume under this projection is the kernel of the surface $S$.

Similarly to the case of freeform closed curves in the plane,
the kernel of a free-form closed surface is bounded
by some patches of the surface and 
% JJ
some developable surface patches that are generated by
tangent planes at parabolic points of the surface $S(u,v)$.

Assume that a point $P$ is on the tangent plane of $S$ at
an elliptic point $S(u,v)$.  The point $P$ may not see
certain part of $S$ unless it is the tangent point $S(u,v)$ itself.
% JJ
Elliptic points may not contribute to the boundary of the kernel.
Moreover, for the case of a tangent plane at a hyperbolic point $S(u,v)$,
it is clear that each point $P$ on the tangent plane may not see
some part of the surface $S$.  Consequently, no hyperbolic points
can contribute to the boundary of the kernel.  
In the case of parabolic points, the envelope surface of tangent planes
along the parabolic lines may contribute to the boundary of the kernel.

% JJ
Figure \ref{fig-kernel-srf} shows three examples of the kernel of a 
free-form surface. These examples were computed by deriving the parabolic
lines of the surfaces, following \cite{Elber93}, and computing the 
envelope of the tangent planes moving along the parabolic lines.

\begin{figure}
    \begin{tabular}{ccc}
    \psfig{width=1.8in,figure={figures/kernel-srf-1.ps}} & 
    \psfig{width=1.8in,figure={figures/kernel-srf-2.ps}} & 
    \psfig{width=1.8in,figure={figures/kernel-srf-3.ps}}\\
    {\large (a)}  &  {\large (b)} &  {\large (c)}
    \end{tabular}
    \caption{\textsf{Three examples of computations of the kernel 
	(dark region) of free-form surfaces. Please note that 
	magenta-colored curves are 
	parabolic lines of the surface.}}
    \label{fig-kernel-srf}
\vskip 0.2in
\end{figure}


%----------------------------------------------------------------------%
\section{Duality between the Convex Hull and the Kernel}
\label{sec-dual}

In this section, we explore the dual relationship between the kernel and convex hull
for closed surfaces.
This relationship can be used to compute kernels from convex hull algorithms.

%%%%%%%%%%%%%%%%%%%%%%%%%%%%%%%%%%%%%%%%%%%%%%%%%%%%%%%%%%%%%%%%

We begin by developing some theory of duality.
There is a geometric duality between points and planes in 3-space \cite{hartshorne}.
Classically, the plane $a_1x + a_2y + a_3z + a_4=0$ is dual to the point $(a_1,a_2,a_3,a_4)$ 
in projective 3-space 
(or equivalently, the point $(\frac{a_1}{a_4},\frac{a_2}{a_4},\frac{a_3}{a_4})$ in Cartesian 3-space),
although many other point-plane dual maps can be defined.
Using geometric duality, the tangent space of a surface $S(u,v)$ can be mapped 
to a surface $S^*(u,v)$ in dual space, by dualizing the tangent plane at $S(u,v)$ to the point $S^*(u,v)$.
We call $S^*$ a {\em tangential surface}, as it represents a tangent space.\footnote{The tangential surface
  of this section is equivalent to the tangential $a_4$-surface of \cite{jj03tangsurf}.}
This mapping to dual space is complicated by the fact that, for a closed surface with empty kernel,
some tangent planes will inevitably map to infinity.
To solve this problem, 
three cooperating dualities are needed to represent a tangent space robustly (Figure~\ref{fig:peartangsystem}).
The reader is referred to \cite{jj03tangsurf} for the complete details
on this representation, called the tangential surface system,
and a computational mechanism for tangential surfaces.
However, when the kernel of $S$ is not empty and a single point of this
kernel is known,
a simpler dual representation of $S$'s tangent space is possible.
By translating the kernel point to the origin,
a single tangential surface is enough (Figure~\ref{fig:peartangdsurf}).

\begin{figure}[h]
\begin{center}
\includegraphics*[scale=.36]{img/peartangsystem.ps}
\end{center}
\caption{The tangential surface system of a pear}
\label{fig:peartangsystem}
\end{figure}

\begin{lemma}
\label{lem:drobust}
If the origin lies inside the kernel of $S$,
the tangential surface of $S$ is a robust finite representation of the tangent space of $S$.
\end{lemma}
\prf
Planes through the origin are mapped to infinity by the above duality.
Since the origin lies inside the kernel,
no tangent plane will pass through the origin and no tangent plane will map to infinity.
\QED

\begin{figure}
\begin{center}
\includegraphics*[scale=.36]{img/peartangdsurf.ps}
\end{center}
\caption{The tangential surface of a pear}
\label{fig:peartangdsurf}
\end{figure}

%%%%%%%%%%%%%%%%%%%%%%%%%%%%%%%%%%%%%%%%%%%%%%%%%%%%%%%%%%%%%%%%

\subsection{A reduction to the convex hull}
\label{sec:reduction}

We are now ready to establish the dual relationship between kernel and convex hull
for a closed $C^1$-continuous surface $S$.
The general structure of the dual relationship will be discussed in this section,
while Section~\ref{sec:cusp} will study its details and provide a complete algorithm.

Suppose that a single point of the kernel has been computed and the surface $S$
has been translated so that this point is at the origin.
Section~\ref{sec:firstpt} will treat the computation of a single point of the kernel.

Using the argument of Section~\ref{subsec-kernel-curve},
an interior point of the kernel can be characterized as a point that is covered by
no tangent plane.
This can be reinterpreted in dual space.

\begin{defn2}
A plane $P^*$ in dual space is {\bf free} if $P^*$ does not intersect the tangential surface $S^*$.
\end{defn2}

\begin{lemma}
\label{lem:dualkernelchar}
$P$ is an interior point of the kernel of $S$ if and only if 
its dual plane $P^*$ is free.
\end{lemma}
\prf
The statement 'no tangent plane of $S$ intersects $P$' in primal space
is equivalent to the statement 'no point of $S^*$ lies on $P^*$' in dual space.
\QED

Lemma~\ref{lem:dualkernelchar} identifies kernel points
with free planes in dual space.
Since the free planes of a surface bound its convex hull,
this immediately suggests a relationship between the kernel and convex hull.

\begin{theorem}
\label{thm:kernelhull}
If the origin lies inside the kernel of $S$,
the kernel of $S$ is dual to the convex hull of the tangential surface $S^*$.
More accurately, the kernel is dual to the family of planes whose
envelope is the convex hull.
\end{theorem}
\prf
Since the origin lies inside the kernel, $S^*$ is a closed, finite surface
and its convex hull is well defined.
Since kernel points of $S$ correspond to free planes,
boundary points of the kernel correspond to free tangent planes of $S^*$
(free planes that are just about to intersect $S^*$).
The free tangent planes of a surface $D$ bound its convex hull: the point $P \in D$ lies
on the boundary of the convex hull of $D$ if and only if the tangent plane at $P$ 
does not intersect $D$.
Therefore, the boundary points of the kernel of $S$ correspond to the boundary points
of the convex hull of $S^*$.\footnote{Typically, points in primal space would be associated with planes in dual space,
	but we can also associate points with points:
	the point $p \in S$ in primal space is associated
	with the point $q \in S^*$ whose tangent plane dualizes to $p$.
	This is particularly easy when there is a parameterization, since $p$ and $q$ 
	have the same parameters.}
\QED

%%%%%%%%%%%%%%%%%%%%%%%%%%%%%%%%%%%%%%%%%%%%%%%%%%%%%%%%%%%%%%%%

\subsection{The first point of the kernel}
\label{sec:firstpt}

The reduction of kernel to convex hull relies on the prior knowledge
of a single point of the kernel.
This point is used to adjust the surface before dualization, ensuring
the finiteness of the tangential surface.
We now discuss the computation of this first point of the kernel.

Suppose that $S$ is concave.
There is no loss of generality in considering concave surfaces,
since the kernel of a convex surface $S$ is simply the interior of $S$.
Since we shall see that the tangential surface $S^*$ of
a concave surface $S$ is also concave, $S^*$ has bitangent planes.
Therefore, bitangent planes can be used in our construction. % of the candidate set.

The search for a kernel point will be reduced to the search for a free plane
(Lemma~\ref{lem:dualkernelchar}).
This moves the search to dual space, consistent with keeping the computation
of the kernel in dual space.
To make the search for a free plane tractable, a candidate set will be found.
A {\bf candidate set} is a finite set of planes in dual space
that necessarily contains a free plane (whenever the kernel is not empty).
We can test each candidate until a free plane is found.
If no free plane is found amongst the candidates,
the kernel must be empty.
Otherwise, the dual of the free plane is a kernel point.
Since there are a finite number of candidates, this is a valid solution.

The challenge is to restrict the candidates for a free plane from
all planes (a triply-infinite system) to a manageable finite set.
There are three degrees of freedom to the search for a free plane.
Two of these degrees of freedom can be removed by restricting the search
to free bitangent planes.
Consider a free plane in dual space.
While preserving its freedom, we can move this free plane
until it touches the surface once, generating a tangent plane,
then pivot until it touches the surface again, generating a bitangent plane.
Sometimes, this process can be continued one more level,
moving the bitangent plane about the surface until it becomes a
tritangent plane.
However, in some cases, there may be no tritangent planes
(consider an hourglass-shaped surface).
Thus, we must handle both cases, a surface with tritangent planes
and a surface with only one-parameter families of bitangent planes.

The candidate set is built as follows.
Consider a one-parameter family of bitangent planes of $S^*$.
If some of these planes are free and some are not free,
there will be a free tritangent plane at the boundary
between free and nonfree planes of the family.
Call a family of planes {\em homogeneous} if every plane is
free or every plane is not free.
We conclude that either all of the families of bitangent planes
are homogeneous or some tritangent plane is free.
This shows that the set consisting of all tritangent planes
and one (arbitrary) representative from each bitangent family
is a candidate set.
Notice that it is a finite set and contains a free plane when anyone exists.
By testing this candidate set, a free plane, and hence a kernel point, can be found.
This defines a dual approach to finding the first point of the kernel.

To get a robust seed point for the kernel,
all of the free planes in the candidate set should be found,
yielding a set of kernel points, and the seed kernel point 
should be defined to be the centroid of this set.
This works since the kernel is convex, and moves the point from the boundary
into the interior of the kernel.

Since there is no longer any way to guarantee the finiteness of the tangential surface
in mapping to dual space (Lemma~\ref{lem:drobust}),
the computation of the first kernel point must use the tangential surface system 
for robustness (see the preamble to Section~\ref{sec-dual}),
by testing for freedom in all three cooperating dual spaces.

%%%%%%%%%%%%%%%%%%%%%%%%%%%%%%%%%%%%%%%%%%%%%%%%%%%%%%%%%%%%%%%%%%%%%%%%%%%%%%%%

\subsection{The hull's structure}
\label{sec:cusp}

We now examine the structure of the convex hull of a tangential
surface $S^*$ through an analysis of its bitangent and tritangent planes,
which are necessarily degenerate when the surface $S$ has a nontrivial kernel.
This leads to a more precise understanding of the 
dual relationship between kernel and hull, culminating in an
algorithm for the construction of kernel from convex hull.

\begin{defn2}
A {\bf bitangent plane} of the surface $S^*$ is a plane that
is tangent to $S^*$ at two or more distinct points.
This plane is {\bf conventional} if the surface has unique, well-defined 
tangent planes at the points of tangency,
otherwise it is {\bf degenerate}.
\end{defn2}

\begin{lemma}
If $S$ has a nonempty kernel, all bitangent and tritangent planes 
of the tangential surface $S^*$ are degenerate.
\end{lemma}
\prf
Let $S$ be a surface with nonempty kernel.
$S$ has no self-intersections, otherwise the kernel is empty
since no point can see through a self-intersection.
But conventional bitangent and tritangent planes of $S^*$ are associated
with double and triple self-intersections of $(S^*)^* = S$ \cite{jj02,jj03tangsurf}.
Thus, no bitangent and tritangent planes of $S^*$ are conventional.
\QED

In order to understand degenerate tangent planes, we must first understand cusps.
Consider a concave
surface $S$ with nonempty kernel.
$S$ will contain parabolic points:
after all, the kernel of a concave surface is the intersection of the inside
halfplanes of the tangent planes of its parabolic points.
Since the tangent planes at parabolic points of $S$ are dual to cusps on $S^*$,
the tangential surface $S^*$ will contain cusps 
(Figures~\ref{fig:peartangsystem}-\ref{fig:toptangdsurf}).
In general, these cusps are not isolated: they gather into closed curves on the surface.
If a point $P$ lies on a curve of cusps $C$,
the {\em tangent line} of $P$ is the tangent of $C$ at $P$.
Any plane through a cusp $P$ that contains the tangent line of $P$ may be considered
a degenerate tangent plane of the surface,
which leads to a degenerate case of the bitangent and tritangent plane.\footnote{Upon entering a cusp $P$, 
	the tangent plane reverses direction, sweeping out all of 3-space 
	by rotating about the tangent line of $P$,
	until it is ready to leave $P$ on the other side in an opposite orientation.
        This explains why any plane through a cusp's tangent line may be considered a tangent plane.}

\begin{figure}
\begin{center}
\includegraphics*[scale=.36]{img/toptangdsurf.ps}
\end{center}
\caption{The tangential surface of a top}
\label{fig:toptangdsurf}
\end{figure}

\begin{defn2}
A {\bf degenerate bitangent plane} is 
(1) a plane that contains the tangent line of two distinct cusps or
(2) a tangent plane (at some noncusp point) that contains the tangent line of a cusp. 
A {\bf degenerate tritangent plane} is 
a plane that contains the tangent line of three cusps, 
a tangent plane that contains the tangent lines of two cusps,
or a bitangent plane that contains the tangent line of one cusp.\footnote{The
	third case will not participate in the hull of $S^*$, since
	it implies a self-intersection in $S$.}
\end{defn2}

The structure of the hulls of our tangential surfaces can now be clarified.
The tangential surface $S^*$ of a surface $S$ with nonempty kernel 
will have a finite number of one-parameter families of degenerate bitangent planes
and a finite number of degenerate tritangent planes.
To create the convex hull, a surface is capped by bitangent developables
(the envelopes of these bitangent plane families) and tritangent planes.
The bitangent developables
are of two types, matching the two types of degenerate bitangent planes: 
a lofting between two cusp curves (type 1) and
a lofting between a cusp curve and a curve on the surface (type 2).

Using this deeper understanding of the hull's structure,
we can now describe how to compute the kernel directly from the hull.
The patches of the hull that lie on the surface $S^*$
cause no trouble: they correspond directly
to the patches of the kernel that lie on the surface $S$.
It is the translation of the patches of the hull that do not lie on the surface 
(the bitangent developables and tritangent planes)
that requires some elaboration.
Notice that each bitangent developable $D^*$ in the hull of $S^*$ 
is associated with a curve $D$ in primal space, since a family of planes dualizes to a curve.

We now have the following algorithm for computing the kernel of $S$ 
from the convex hull of $S^*$.

\begin{enumerate}
\item Compute the convex hull of $S^*$.
\item For each patch of the hull that lies on $S^*$, add the associated patch of $S$
	to the kernel.
\item For each bitangent developable $D^*$ of the hull, compute the associated curve $D$ in primal space
  (see below).
\item If two bitangent developables $D_1^*$ and $D_2^*$ of the hull meet at a cusp curve, 
  add the lofting between their associated curves $D_1$ and $D_2$ to the kernel.\footnote{
    % This lofting is well defined, since $D_1$ and $D_2$ 
    % will necessarily have the same parameterization.
    Each line of this lofting is generated by a tangent plane in dual space pivoting 
    about a point of the cusp curve, from one developable to the other developable.}
\item If a bitangent developable $D^*$ neighbours a tritangent plane on the hull,
	add the lofting between the associated curve $D$ and the tritangent's 
	associated point to the kernel.
\end{enumerate}

$D$ is easily computable.
Suppose that the developable $D^*$ is a lofting between $C_1(t)$ and $C_2(t)$,
where $C_1(t)$ is a cusp curve if $D^*$ is of type 1 and
a curve of tangency with $S^*$ if $D^*$ is of type 2,
while $C_2(t)$ is a cusp curve.
$D$ is the dual of the bitangent planes along $C_1(t)$.
For type 1 developables, the bitangent plane at $C_1(t)$ is the plane through
$C_1(t)$ and the tangent line of $C_2(t)$.
For type 2 developables, the bitangent plane at $C_1(t)$ is the tangent plane
at $C_1(t)$.
$D$ lies in the interior of $S$ for type 1 developables,
and on $S$ for type 2 developables.

This concludes the reduction of kernel to convex hull.
The structural relationship between the kernel and hull that is revealed
by this study provides insight into these two fundamental geometric structures.

\section{Acknowledgements}

We would like to thank Wenping Wang for alerting us
to a dual relationship between the kernel and the convex hull.
We are appreciative of the stimulating research environment
provided by the Dagstuhl Seminar on Geometric Modeling.
This work was supported by the National Science Foundation under 
grant CCR-0203586.

%----------------------------------------------------------------------%
% The Appendices part is started with the command \appendix;
% appendix sections are then done as normal sections
% \appendix

% \section{}
% \label{}

% Bibliographic references with the natbib package:
% Parenthetical: \citep{Bai92} produces (Bailyn 1992).
% Textual: \citet{Bai95} produces Bailyn et al. (1995).
% An affix and part of a reference:
%   \citep[e.g.][Ch. 2]{Bar76}
%   produces (e.g. Barnes et al. 1976, Ch. 2).

\begin{thebibliography}{}

% \bibitem[Names(Year)]{label} or \bibitem[Names(Year)Long names]{label}.
% (\harvarditem{Name}{Year}{label} is also supported.)
% Text of bibliographic item

\bibitem{Kim91}
C.~Bajaj and M.-S.~Kim.
\newblock Convex Hulls of Objects Bounded by Algebraic Curves
\newblock {\em Algorithmica\/}, Vol.~6, pp.~533--553, 1991.

\bibitem{Souvaine90}
D.~Dobkin and D.~Souvaine.
\newblock Computational Geometry in a Curved World.
\newblock {\em Algorithmica\/}, Vol.~5, No.~3, pp.~421--457, 1990.

\bibitem{docarmo}
M.P.~do Carmo.
\newblock {\em Differential Geometry of Curves and Surfaces}.
\newblock Prentice-Hall, Inc., Upper Saddle River, New Jersey 07458, 1976.

\bibitem{Elber93}
G.~Elber and E.~Cohen.
\newblock Second-Order Surface Analysis Using Hybrid Symbolic and
Numeric Operators.
\newblock ACM Transactions on Graphics, Vol. 12, No.2, April 1993, 
pp. 160--178.

\bibitem{Elber2001a}
G.~Elber and M.-S.~Kim.
\newblock Geometric Constraint Solver
Using Multivariate Rational Spline Functions.
\newblock {\em Proc.~of ACM Symposium on Solid Modeling and Applications\/},
Ann Arbor, MI, June 4--8, 2001.

\bibitem{Elber2001b}
G.~Elber, M.-S.~Kim, and H.-S.~Heo.
\newblock The Convex Hull of Rational Plane Curves.
\newblock To appear in {\em Graphical Models\/}, 2001.

\bibitem{Graham83}
R.~Graham and F.~Yao.
\newblock Finding the Convex Hull of a Simple Polygon.
\newblock {\em J.~of Algorithms\/}, Vol.~4, No.~4, pp.~324--331, 1983.

% \bibitem[Hartshorne 77]{hartshorne}
\bibitem{hartshorne}
R.~Hartshorne.
{\em Algebraic Geometry}. Springer-Verlag, New York, 1977.

\bibitem{Johnstone2001}
J.~Johnstone.
\newblock A Parametric Solution to Common Tangents.
\newblock {\em Proc.~of Int'l Conf.~on Shape Modeling and Applications\/},
Genova, Italy, May 7--11, 2001, 240--249.

% \bibitem[Johnstone 02]{jj02}
\bibitem{jj02}
J.~Johnstone.
The Tangential Curve, 2002.

% \bibitem[Johnstone 03]{jj03tangsurf}
\bibitem{jj03tangsurf}
J.~Johnstone.
The Bezier tangential surface system: a dual representation of tangent space,
2003.

\bibitem{jj03}
J.~Johnstone.
From Kernels to Hulls, 2003.

\bibitem{Lee83}
D.T.~Lee.
\newblock On Finding the Convex Hull of a Simple Polygon.
\newblock {\em Int'l J.~Computer and Information Sciences\/},
Vol.~12, No.~2, pp.~87--98, 1983.

\bibitem{Pottmann}
H. Pottmann and J. Wallner.
\newblock Computational Line Geometry.
\newblock Springer--Verlag Berlin Heidelberg, pp. 400--410, 2001.

\bibitem{Prep85}
F.~Preparata and M.~Shamos.
\newblock {\em Computational Geometry: An Introduction\/}.
\newblock Springer-Verlag, New York, NY, 1985.

\bibitem{Schaeffer87}
A.~Sch\"affer and C.~Wan Wyk.
\newblock Convex Hulls of Piecewise-Smooth Jordan Curves.
\newblock {\em J.~of Algorithms\/}, Vol.~8, No.~1, pp.~66--94, 1987.

\end{thebibliography}

%%%%%%%%%%%%%%%%%%%%%%%%%%%%%%%%%%%%%%%%%%%%%%%%%%%%%%%%%%%%%%%%%%%%%%%%%%%%%

\end{document}
\end{thebibliography}

\end{document}

