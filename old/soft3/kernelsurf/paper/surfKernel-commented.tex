\documentclass[12pt]{article}
\usepackage[pdftex]{graphicx}
\usepackage{times}
\input{header}

\newif\ifJournal
\Journalfalse
\newif\ifTalk
\Talkfalse

\setlength{\oddsidemargin}{0pt}
\setlength{\topmargin}{0in}	% should be 0pt for 1in
	% \setlength{\headsep}{.5in}
	% \setlength{\textheight}{8.875in}
\setlength{\textheight}{8.6in}
\setlength{\textwidth}{6.875in}
\setlength{\columnsep}{5mm}	% width of gutter between columns

	% \DoubleSpace

% \title{}
% \author{J.K. Johnstone}
%	Geometric Modeling Lab\\
%	Computer and Information Sciences\\
%	University of Alabama at Birmingham\\
%	University Station, Birmingham, AL, USA 35294}

\begin{document}
% \maketitle

%%%%%%%%%%%%%%%%%%%%%%%%%%%%%%%%%%%%%%%%%%%%%%%%%%%%%%%%%%%%%%%%

\section{Computing the kernel of a surface from the convex hull}

There is a dual relationship between the kernel and convex hull
for closed surfaces.
This relationship can be used to compute kernels from convex hull algorithms,
as we shall explore in this section.
	% much as Voronoi diagrams are computed from Delaunay triangulation algorithms.
	% Kernels and convex hulls are related by a geometric (plane-point) duality,
	% whereas Voronoi diagrams and Delaunay triangulations are related by a graph (face-vertex) duality.
The {\bf kernel} of a closed surface $S$ is the locus of points that can see every
point of $S$:
\[
\mbox{kernel}(S) = \{ P \in R^3: \seg{PQ} \ \cap \ S = \emptyset 
		\hspace{.25in} \forall \ Q \in S \}
\]
It is a convex connected subset of the surface's interior.
The kernel is an important structure to understand in developing a theory of visibility.
	% A theory of visibility can be used to solve many problems in lighting, rendering, and motion planning.
For example, interpreting the surface as the boundary of a room,
a point of the kernel would be an ideal location to place a light, camera, or robot,
since the entire room would be visible.

The computation of kernels from hulls
offers a different point of view on the computation of the kernel,
allowing all attention to be placed on the development of an algorithm
for the efficient and robust computation of the convex hull.
We can appeal to the Voronoi diagram and Delaunay triangulation for a comparable example.
There are algorithms for the computation of each of these structures,
but many still prefer to always compute the Voronoi diagram by computing
the Delaunay triangulation and then dualizing.

%%%%%%%%%%%%%%%%%%%%%%%%%%%%%%%%%%%%%%%%%%%%%%%%%%%%%%%%%%%%%%%%

\subsection{Duality and tangent space}
\label{sec:duality}

We must begin by developing some theory of duality.
There is a geometric duality between points and planes in 3-space \cite{hartshorne}.
Classically, the plane $ax+by+cz+d=0$ is dual to the point $(a,b,c,d)$ in projective 3-space
(or equivalently, the point $(\frac{a}{d},\frac{b}{d},\frac{c}{d})$ in Cartesian 3-space),
although many other point-plane dual maps can be defined.
Using geometric duality, the tangent space of a surface $S(t)$ can be mapped 
to a surface $S^*(t)$ in dual space, by dualizing the tangent plane at $S(t)$ to the point $S^*(t)$.
We call $S^*(t)$ a tangential surface, as it represents a tangent space.
This mapping to dual space is complicated by the fact that, for a closed surface,
some tangent planes will inevitably map to infinity.
	% regardless of the duality that is used.
To solve this problem, 
three cooperating dualities are needed to represent a tangent space robustly.
The reader is referred to \cite{jj02tangsurf} for the complete details
on this representation, called the tangential surface system 
(Figure~\ref{fig:peartangsystem}).
%
\Comment{
\begin{defn2}
The plane $ax+by+cz+d=0$ is {\bf a-dual} to the point $(\frac{d}{a},\frac{b}{a},\frac{c}{a})$,
{\bf b-dual} to the point $(\frac{a}{b},\frac{d}{b},\frac{c}{b})$, and
{\bf c-dual} to the point $(\frac{a}{c},\frac{b}{c},\frac{d}{c})$.
The point $(a,b,c)$ is {\bf a-dual} to the plane $x+by++cz+a=0$
{\bf b-dual} to the plane $ax+y+cz+b=0$, and
{\bf c-dual} to the plane $ax+by+z+c=0$.
\end{defn2}

\begin{defn2}
\label{defn:tssystem}
$S^*_a(t)$ is the {\bf tangential a-surface} of the surface $S(t)$ if $S^*_a(t)$ is the a-dual
of the tangent plane at $S(t)$.
$S^*_b(t)$ is the {\bf tangential b-surface} of the surface $S(t)$ if $S^*_b(t)$ is the b-dual
of the tangent plane at $S(t)$.
$S^*_c(t)$ is the {\bf tangential c-surface} of the surface $S(t)$ if $S^*_c(t)$ is the c-dual
of the tangent plane at $S(t)$.
The combination of 
the tangential a-surface $S^*_a$ within $(-\infty,\infty) \times [-1,1] \times [-1,1]$,
the tangential b-surface $S^*_b$ within $(-1,1) \times (-\infty,\infty) \times [-1,1]$, and
the tangential c-surface $S^*_c$ within $(-1,1) \times (-1,1) \times (-\infty,\infty)$
is called the {\bf tangential surface system} ($S_a^*, S_b^*, S_c^*$) of $S$.
See Figure~\ref{fig:peartangsystem}.
The tangential surface system is a robust representation of the tangent space of $S$.
\end{defn2}
}
%
However, when a single point of $S$'s kernel is known,
another dual representation of $S$'s tangent space is possible.\footnote{Since many
	surfaces have empty kernels, this is not a general solution to
	the representation of tangent spaces.}
After translating the kernel point to the origin,
the tangential d-surface alone is a robust representation of the tangent space of $S$.

\begin{defn2}
The plane $ax+by+cz+d=0$ is {\bf d-dual} to the point $(\frac{a}{d},\frac{b}{d},\frac{c}{d})$.
The point $(a,b,c)$ is {\bf d-dual} to the plane $ax+by+cz+1=0$.
$S^*(t)$ is the {\bf tangential d-surface} of the surface $S(t)$ if $S^*(t)$ is the d-dual
of the tangent plane at $S(t)$ (Figure~\ref{fig:peartangdsurf}).
\end{defn2}

\begin{lemma}
\label{lem:drobust}
If the origin lies inside the kernel of $S$,
the tangential d-surface of $S$ is a robust representation of the tangent space of $S$.
\end{lemma}
\prf
Planes through the origin are mapped to infinity by the d-duality.
Since the origin lies inside the kernel,
no tangent plane will pass through the origin (Lemma~\ref{lem:primalkernelchar} below), 
so no tangent plane will map to infinity.
\QED

%%%%%%%%%%%%%%%%%%%%%%%%%%%%%%%%%%%%%%%%%%%%%%%%%%%%%%%%%%%%%%%%

\subsection{A reduction to the convex hull}
\label{sec:reduction}

We are now ready to establish the dual relationship between kernel and convex hull
for a closed surface $S$.
The general structure of this relationship will be discussed in this section,
while Section~\ref{sec:cusp} will study its details and provide a complete algorithm.
In this section, it is assumed that a single point of the kernel has been computed and the surface $S$
has been translated so that this point is at the origin.
This makes the tangential d-surface $S^*$ into a closed finite surface
and a robust representation for the tangent space (Lemma~\ref{lem:drobust}).
Section~\ref{sec:firstpt} will treat the computation of a single point of the kernel.

A point of the kernel is a point that sees the entire surface.
An alternative characterization is available for surfaces,
as observed by Gershon Elber \cite{elber02}:
a point inside the kernel is not hit by any tangent plane.\footnote{A boundary point
	of the kernel is hit either by its own tangent plane or by the tangent plane 
	of a parabolic point.}

\begin{lemma}
\label{lem:primalkernelchar}
The point $P$ lies strictly inside the kernel of a smooth closed surface $S$ if and only if 
no tangent plane of $S$ intersects $P$.
\end{lemma}
\prf
If $P$ does not lie in the kernel,
then $P$ does not see some point $Q \in S$.
That is, $\seg{PQ}$ hits the surface in a second point $R$.
Since the surface is closed, 
as the line $\lyne{PQR}$ is rotated about $P$,
it will eventually become tangent to the surface.
	% (as the two points of intersection coalesce as the line leaves the surface)
The tangent plane at this point contains $P$.
Thus, if $P$ does not lie in the kernel, then it is hit by some tangent plane.
	% As a limit case of $\seg{PQ} \cap S \neq \emptyset$,
A point of the kernel boundary, as a limit of this argument, will also be hit by some tangent plane.
This shows that if no tangent plane of $S$ intersects $P$, 
$P$ must lie in the interior of the kernel.

Conversely, if the tangent plane at $Q \in S$ intersects $P$, then $P'Q$ will intersect $S$ twice for some point $P'$ in 
the neighbourhood of $P$, meaning $P'$ is not in the kernel.
Thus, if $P$ lies in the interior of the kernel, no tangent plane of $S$ can intersect $P$.
\QED

	% This membership test for a kernel point
	% involves the testing of an infinite number of tangent planes, which is impractical.
	% However, by mapping to dual space, it becomes a simple test: a single plane-surface intersection.

\noindent This characterization of a kernel point can be reinterpreted in dual space.

\begin{defn2}
A plane $P^*$ in dual space is {\bf free} if $P^*$ does not intersect the tangential surface $S^*$.
\end{defn2}

\begin{corollary}
\label{cor:dualkernelchar}
The point $P$ lies strictly inside the kernel of $S$ if and only if 
the plane $P^*$ is free.
\end{corollary}
\prf
The statement 'no tangent plane of $S$ intersects $P$' in primal space
is equivalent to the statement 'no point of $S^*$ lies on $P^*$' in dual space.
\QED

Corollary~\ref{cor:dualkernelchar} identifies kernel points in primal space
with free planes in dual space.
Since the free planes of a surface bound its convex hull,
this immediately suggests a relationship between the kernel and convex hull.

\begin{theorem}
\label{thm:kernelhull}
If the origin lies inside the kernel of $S$,
the kernel of $S$ is dual to the convex hull of the tangential d-surface $S^*$.
More accurately, the kernel is dual to the family of planes whose
envelope is the convex hull.
\end{theorem}
\prf
The convex hull is well defined, since $S^*$ is a closed, finite surface.
Since kernel points of $S$ correspond to free planes,
boundary points of the kernel correspond to free tangent planes of $S^*$
(free planes that are just about to intersect $S^*$).
The free tangent planes of a surface $D$ bound its convex hull: the point $P \in D$ lies
on the boundary of the convex hull of $D$ if and only if the tangent plane at $P$ 
does not intersect $D$.
% In particular, the envelope of a surface's free tangents planes define its convex hull.
Therefore, the boundary points of the kernel of $S$ correspond to the boundary points
of the convex hull of $S^*$.\footnote{Typically, points in primal space would be associated with planes in dual space,
	but we can also associate points with points:
	the point $p \in S$ in primal space is associated
	with the point $q \in S^*$ whose tangent plane dualizes to $p$.
	This is particularly easy when there is a parameterization, since $p$ and $q$ 
	have the same parameters.}
% In particular, patches of the hull that lie on $S^*$ correspond to patches
% of the kernel that lie on $S$; and 
% linear (ruled surface) components of the hull that span concavities and lie outside $S^*$
% correspond to the linear components of the kernel inside $S$.
\QED

We delay a detailed treatment of how the kernel is computed from the hull
until Section~\ref{sec:cusp}, after the hull of the tangential surface is better understood.

%%%%%%%%%%%%%%%%%%%%%%%%%%%%%%%%%%%%%%%%%%%%%%%%%%%%%%%%%%%%%%%%

\subsection{The first point of the kernel}
\label{sec:firstpt}

The reduction of kernel to convex hull relies on the prior knowledge
of a single point of the kernel.
This point is used to adjust the surface before dualization, ensuring
the finiteness of the tangential d-surface.
This section discusses the computation of this first point of the kernel.

The search for a kernel point will be reduced to the search for a free plane,
using the association of Corollary~\ref{cor:dualkernelchar}.
This moves the search to dual space, consistent with keeping the computation
of the kernel in dual space.
To make the search for a free plane tractable, a candidate set will be found.
A {\bf candidate set} is a finite set of planes in dual space
that necessarily contains a free plane (whenever the kernel is not empty).
We can test each candidate until a free plane is found.
If no free plane is found amongst the candidates,
the kernel must be empty.
Otherwise, the dual of the free plane is a kernel point,
which can be used to seed the algorithm of Section~\ref{sec:reduction}.
Since there are a finite number of candidates, this is a valid solution.

The challenge is to restrict the candidates for a free plane from
all planes (a triply-infinite system) to a manageable finite set.
There are three degrees of freedom to the search for a free plane.
Two of these degrees of freedom can be removed by restricting the search
to free bitangent planes.
% To reduce these degrees of freedom and make the search more manageable,
% we restrict our search to free tritangent planes.
Consider a free plane in dual space.
While preserving its freedom, we can move this free plane
until it touches the surface once, generating a tangent plane,
then pivot until it touches the surface again, generating a bitangent plane.
% The rest of the section details the search for a free bitangent plane.
% In other words, if any plane in dual space is free,
% one of the bitangent planes of $S^*$ is free.
Sometimes, this process can be continued one more level,
moving the bitangent plane about the surface until it becomes a
tritangent plane.
However, in some symmetrical cases, there are no tritangent planes.
Thus, we must handle both cases, a surface with tritangent planes
and a surface with only one-parameter families of bitangent planes.
% In certain cases, the bitangent plane can be moved along the surface
% until reaching a tritangent plane.
% However, in some symmetrical cases, no tritangent planes exist.
% then pivot yet again until it touches the surface a third time,
% generating a tritangent plane.
% {\bf THIRD PIVOTING IS NOT ALWAYS POSSIBLE.}

The candidate set is built as follows.
Consider a one-parameter family of bitangent planes of $S^*$.
If some of these planes are free and some are not free,
there will be a free tritangent plane at the boundary
between free and nonfree planes of the family.
Call a family of planes {\bf homogeneous} if every plane is
free or every plane is not free.
We conclude that either all of the families of bitangent planes
are homogeneous or some tritangent plane is free.
This shows that the set consisting of all tritangent planes
and one (arbitrary) representative from each bitangent family
is a candidate set.
Notice that it is a finite set and contains a free plane when anyone exists.
By testing this candidate set, a free plane, and hence a kernel point, can be found.
This defines a dual approach to finding the first point of the kernel.

To get a robust seed point for the kernel,
all of the free planes in the candidate set should be found,
yielding a set of kernel points, and the seed kernel point 
should be defined to be the centroid of this set.
This works since the kernel is convex, and moves the point from the boundary
into the interior of the kernel.

\Comment{
	Notice that a plane may be fully defined by two points of tangency,
	like a tangent plane that straddles the concavity of an hourglass figure.
	That is, although the bitangent plane indeed has another degree of freedom
	(generating a family of bitangent planes),
	this family wraps around the surface, never reaching a tritangency.
}

	% The following lemma applies either to a-dual space 
	% (and the tangential a-surface $S^*$), to b-dual space
	% (and the tangential b-surface $S^*$), or to c-dual space
	% (and the tangential c-surface $S^*$).

Since there is no longer any way to guarantee the finiteness of the tangential surface
in mapping to dual space (unlike Section~\ref{sec:reduction}),
the computation of the first kernel point must use the tangential surface system 
for robustness (Section~\ref{sec:duality}),
by testing for freedom in all three cooperating dual spaces.

\Comment{
A version of the kernel characterization of Corollary~\ref{cor:dualkernelchar},
adapted to the tangential a-surface, b-surface, and c-surface of a tangential surface system,
is given below.
Using this criterion, a candidate set is created in each dual space (a, b, and c),
and then tested for freedom across all three dual spaces.

\begin{defn2}
The {\bf plane system} of a point $P$ is the combination of the a-dual, b-dual and c-dual planes of $P$,
$(P^*_a, P^*_b, P^*_c)$.
In a-dual space, the a-dual $P^*_a$ is {\bf free} if it does not intersect the tangential a-surface 
of $S$ inside $(-\infty,\infty) \times [-1,1] \times [-1,1] \subset R^3$.
In b-dual space, the b-dual $P^*_b$ is {\bf free} if it does not intersect the tangential b-surface
of $S$ inside $(-1,1) \times (-\infty,\infty) \times [-1,1]$.
In c-dual space, the c-dual $P^*_c$ is {\bf free} if it does not intersect the tangential c-surface
of $S$ inside $(-1,1) \times (-1,1) \times (-\infty,\infty)$.
The plane system $(P^*_a, P^*_b, P^*_c)$ is {\bf free} if 
all of $P^*_a$, $P^*_b$ and $P^*_c$ are free (in their respective dual spaces).
\end{defn2}

\begin{lemma}
$P$ lies strictly inside the kernel of $S$ if and only if
its plane system $(P^*_a,P^*_b,P^*_c)$ is free.
\end{lemma}
}

%%%%%%%%%%%%%%%%%%%%%%%%%%%%%%%%%%%%%%%%%%%%%%%%%%%%%%%%%%%%%%%%%%%%%%%%%%%%%%%%

\subsection{The hull's structure}
% Cusps on the tangential surface}
\label{sec:cusp}

This section examines the structure of the convex hull of a tangential
surface $S^*$, through an analysis of its bitangent and tritangent planes
which are necessarily degenerate when the surface $S$ has a nontrivial kernel.
% This structure is important in the computation of the convex hull.
% as well as the computation of the first kernel point
This leads to a more precise understanding of the 
dual relationship between kernel and hull, culminating in an
algorithm for the construction of kernel from convex hull.

	% The development of this section applies equally well to any type of tangential surface: a, b, c, or d.
	% We restrict our attention to the tangential surfaces $S^*$ of 
	% concave surfaces with nonempty kernel.
	% If the kernel of $S$ is empty, any candidate set can be chosen and the search for a free plane
	% is uninteresting.
	% and any search for a free plane in dual space will be fruitless
	% if the surface has an empty kernel,

\begin{defn2}
A {\bf bitangent plane} of the surface $S^*$ is a plane that
is tangent to $S^*$ at two or more distinct points.
This plane is {\bf conventional} if the surface has unique, well-defined 
tangent planes at the points of tangency,
otherwise it is {\bf degenerate}.
\end{defn2}

\begin{lemma}
\label{lem:degenerate}
If $S$ has a nonempty kernel,
all bitangent and tritangent planes of the tangential surface $S^*$ are 
degenerate.
\end{lemma}
\prf
Let $S$ be a surface with nonempty kernel.
$S$ has no self-intersections, otherwise the kernel is empty
since no point can see through a self-intersection.
But conventional bitangent and tritangent planes of $S^*$ are associated
with double and triple self-intersections of $(S^*)^* = S$ \cite{jj02,jj02tangsurf}.
Thus, no bitangent and tritangent planes of $S^*$ are conventional.
\QED

% \begin{defn2}
% An {\bf asymptotic direction} at a point $P$ of the surface $S$ is a direction
% at which the second fundamental form is zero \cite{struik}.
% The normal section of the surface at $P$ in this asymptotic direction has
% a point of inflection.
	% \footnote{A linear normal section is a degenerate case.}
	% Struik, Lectures on Classical Differential Geometry, p. 77-78
% A {\bf parabolic point} is a point of the surface with exactly one asymptotic
% direction.
% Equivalently, the Gaussian curvature at a parabolic point is zero.
% A {\bf parabolic point} is a point of a surface with zero Gaussian curvature.
% \end{defn2}

In order to understand degenerate bitangent planes, we must understand cusps.
Consider a concave\footnote{There is no loss of generality in only 
	considering concave surfaces,
	since the kernel of a convex surface $S$ is simply the interior of $S$.}
surface $S$ with nonempty kernel.
% We first establish that the tangential surfaces of surfaces with
% nonempty kernels will contain cusps.
$S$ will contain parabolic points (points with zero Gaussian curvature \cite{struik}):
after all, the kernel of a concave surface is the intersection of the inside
halfplanes of the tangent planes of its parabolic points.
Since the tangent planes at parabolic points of $S$ are dual to cusps on $S^*$,
the tangential surface $S^*$ will contain cusps 
(Figures~\ref{fig:peartangsystem}-\ref{fig:toptangdsurf}).
% \footnote{Consider
%	entering a parabolic point along its asymptotic direction:
%	the tangent plane reverses its direction of rotation.
%	In dual space, this corresponds to a point reversing its direction at a cusp.}
In general, these cusps are not isolated: they gather into closed curves on the surface.
Thus, one can talk about the tangent line of a cusp.

\begin{defn2}
\label{defn:tangline}
Let $P$ be a cusp of $S^*$ that lies on a curve of cusps $C$.
The {\bf tangent line} of $P$ is the tangent of $C$ at $P$.
\end{defn2}

Any plane through a cusp $P$ that contains the tangent line of $P$ can be considered
a tangent plane of the surface.
To see this, consider a cusp $P$.
Upon entering $P$, the tangent plane reverses direction, stopping to sweep
out the entire 3-space by rotating about the tangent line of $P$,
until it is ready to leave $P$ on the other side in an opposite orientation.

This leads to a degenerate case of the bitangent plane.

\begin{defn2}
A {\bf degenerate bitangent plane} is 
(1) a plane that contains the tangent line of two distinct cusps or
(2) a tangent plane (at some noncusp point) that contains the tangent line of a cusp. 
\end{defn2}

It can be shown that, if the surface contains two or more cusp curves,
a typical cusp point has a degenerate bitangent plane of the first type.
That is, given the tangent line of a cusp, 
it is possible to find a second tangent line that lies in the same plane.

\Comment{
	Argument why a plane can contain the tangent line of two cusps.
	Let $P$ be a cusp on the cusp curve $C_1$.
	Let $C_2$ be another cusp curve.
	There exists a tangent of $C_2$ coplanar with the tangent at $P$.
	(That is, there exists a bitangent plane through $P$.)
	Consider the Gauss sphere, now used to analyze tangents rather than normals.
	Draw the unit tangents of $C_2$ on the Gauss sphere.
	This is a closed curve $G(C_2)$ that straddles both hemispheres.
	The lines parallel to $P$'s tangent line define a great circle on
	the Gauss sphere.
	This great circle must intersect $G(C_2)$,
	at which point we have a tangent that is parallel, and hence coplanar, with
	$P$'s tangent line.
}

\begin{defn2}
A {\bf degenerate tritangent plane} is 
a plane that contains the tangent line of three cusps or
a tangent plane that contains the tangent lines of two cusps.
\end{defn2}

The structure of the hulls of our tangential surfaces can now be clarified.
A tangential surface will have a finite number of one-parameter families of degenerate bitangent planes
and a finite number of degenerate tritangent planes.
To create the convex hull, a surface is capped by bitangent developables and tritangent planes.
The bitangent developables
are of two types, matching the two types of degenerate bitangent planes: 
a lofting between two cusp curves (type 1) and
a lofting between a cusp curve and a curve on the surface (type 2).

	% Theorem~\ref{thm:kernelhull} leads directly to the kernel
	% if the family of planes that generate the hull is known.
	% But we need to elaborate on the computation of the kernel
	% if only the hull is known (that is, the envelope of these planes).
Using this deeper understanding of the hull's structure,
we now describe how to compute the kernel directly from the hull.
Of course, the patches of the hull that lie on the surface $S^*$
cause no trouble: they correspond directly
to the patches of the kernel that lie on the surface $S$.
It is the translation of the patches of the hull that do not lie on the surface 
(the bitangent developables and tritangent planes)
that requires some elaboration.

Each bitangent developable $D^*$ in the hull of $S^*$ 
is associated with a curve $D$ in primal space.\footnote{Envelopes
	of planes dualize to curves.}
If two developables $D_1^*$ and $D_2^*$ of the hull meet at a cusp curve, 
the kernel will contain a lofting between the associated curves 
$D_1$ and $D_2$ in primal space 
(which is well defined, since $D_1$ and $D_2$ 
will necessarily have the same parameterization).\footnote{Each line 
	of this lofting is generated by a tangent plane in dual space pivoting 
	about a point of the cusp curve,
	from one developable to the other developable.}

This development is summed up in the following algorithm for computing the kernel of $S$ 
from the convex hull of $S^*$.

\begin{enumerate}
\item Compute the convex hull of $S^*$.
\item For each patch of the hull that lies on $S^*$, add the associated patch of $S$
	to the kernel.
%	associated means 'with the same parameters'
\item For each bitangent developable $D^*$ of the hull, compute the associated curve $D$ in primal space.
	$D$ is the dual of the bitangent planes along either boundary of $D^*$.
\item For each pair of neighbouring bitangent developables on the hull,
	% necessarily along a cusp curve
	add the lofting between their associated curves to the kernel.
\item For each bitangent developable that neighbours a tritangent plane,
	% necessarily along a cusp curve
	add the lofting between the developable's associated curve and the tritangent's 
	associated point to the kernel.
\end{enumerate}

$D$ is easily computable.
Suppose that the developable $D^*$ is the lofting between $C_1(t)$ and $C_2(t)$,
where $C_1(t)$ is a cusp curve if $D^*$ is of type 1 and
a curve of tangency with $S^*$ if $D^*$ is of type 2,
while $C_2(t)$ is always a cusp curve.
For type 1 developables, the bitangent plane at $C_1(t)$ is the plane through
$C_1(t)$ and the tangent line of $C_2(t)$.
For type 2 developables, the bitangent plane at $C_1(t)$ is the tangent plane
at $C_1(t)$, so $D$ is the image of $C_1(t)$.\footnote{A curve in primal
	space and a curve in dual space are images if they share the same
	parameters on their respective surfaces.}
$D$ lies in the interior of $S$ for type 1 developables,
and on $S$ for type 2 developables.

\Comment{
All that remains is to compute the curve $D$ in primal space associated
with a bitangent developable $D^*$ in dual space.
Each developable is the envelope of a plane family.
By understanding this plane family, we can easily dualize the bitangent developable.
If the developable $D^*$ is of type 1, 
let $C_1(t)$ and $C_2(t)$ be the cusp curves bounding $D^*$
such that $D^*(s,t) = (1-s)C_1(t) + sC_2(t)$, $s \in [0,1]$.
The plane family that generates $D^*$ consists of
the degenerate bitangent planes of type 1 along $C_1$ (or $C_2$).
Since the lofting between $C_1$ and $C_2$ is already known,
this is easily computed:
$\{P(t): P(t) \mbox{ is a plane through $C_1(t)$ and the tangent line of $C_2(t)$}\}$.
$D$ is the dual of this plane family,
which will lie in the interior of $S$.
If the developable $D^*$ is of type 2, the plane family that generates the developable
is the family of tangent planes along the curve of tangency of $D^*$ with $S^*$.
Thus, $D$ is the image of this curve of tangency.
(That is, these curves share the same parameters on their respective surfaces.)
}

This concludes the reduction of kernel to convex hull.

\begin{figure}[h]
\begin{center}
\includegraphics*[scale=.36]{img/peartangsystem.jpg}
\end{center}
\caption{The tangential surface system of a pear}
\label{fig:peartangsystem}
\end{figure}

\clearpage

\begin{figure}
\begin{center}
\includegraphics*[scale=.36]{img/peartangdsurf.jpg}
\end{center}
\caption{The tangential d-surface of a pear}
\label{fig:peartangdsurf}
\end{figure}

\begin{figure}
\begin{center}
\includegraphics*[scale=.36]{img/toptangdsurf.jpg}
\end{center}
\caption{The tangential d-surface of a top}
\label{fig:toptangdsurf}
\end{figure}

\clearpage

% In primal space, these correspond to the intersections of
% the envelopes of parabolic tangent planes with the surface or with each other.

	% A {\bf cusp} of the surface $S$ is a point whose incoming and outgoing
	% tangent planes have opposite orientations: $t^+ = -t^-$ \cite{millmanParker}.
	% p. 71, 5.5b

	% Consider cusps on curves.
	% At a cusp, the curve's tangent reverses direction, stopping to sweep
	% out the entire plane at the cusp.
	% Consequently, the tangent at a cusp is degenerate 
	% and any line through a cusp can be considered a tangent to the cusp.
	% Similarly, cusps on surfaces lead to degenerate tangent planes.

%%%%%%%%%%%%%%%%%%%%%%%%%%%%%%%%%%%%%%%%%%%%%%%%%%%%%%%%%

\Comment{
We are looking for a free plane and, to reduce the degrees of freedom
of the search, that lies multiply tangent to the tangential surface.
Multiple tangencies must involve a degenerate cusp, since the primal
surface contains no self-intersections.
So we must understand how a plane is tangent to the surface at a cusp $P$.
There is more freedom to a 'tangent plane' at a cusp.
We should rather think of being tangent to the cusp curve through $P$.
This would seem to be equivalent to a plane that contains the tangent
of the cusp curve through $P$.
We must argue why two tangents of two different cusp curves must align.

We should look instead for bitangencies.
Degenerate bitangencies are tangent planes that are also tangent to 
cusp curves (see pear duality).
This is feasible!
We have removed two of the three degrees of freedom (by looking
for a free degenerate bitangent plane rather than a free plane).
The remaining degree of freedom may be inevitable,
as witnessed by the top.
That is, there may be no distinguished plane in the 1-dimensional family
of bitangent planes that define the hull.
}

%%%%%%%%%%%%%%%

\Comment{
% This lemma allows the search for a free plane to be restricted
% to a search for a free cusp tritangent,
% reducing its domain from a triply infinite family of planes to
% a finite set of tritangents.
% % Cusp tritangents can be used by computing the cusps and 
% % then the tangents through the cusps.
% % \cite{jj01acmse} shows how to compute the tangents through a point.

A cusp of $S^*$ dualizes to the tangent plane of a parabolic point of $S$.\footnote{A parabolic
	point involves a reversal of the tangent plane's direction, 
	dual to the cusp's reversal of direction.}
Thus, a plane through the tangent line of two cusps
dualizes to the intersection of the envelopes of two families of parabolic tangent planes.
A tangent plane through a cusp dualizes 
the intersection of the surface with an envelope of a family of parabolic tangent planes.

In conclusion, a single point of the kernel can be found by calculating
a small finite set of candidate points, either through the computation
in dual space of cusp tritangents or the computation in primal space of
---.
Of these candidates, the planes (or the duals of the points)
that are free correspond to kernel points.
If none are free, the kernel is empty.
Since the kernel points found by this algorithm
will inherently lie on the boundary of the kernel,
their sample mean is a more robust choice for the first point of the kernel.
(Recall that the kernel is convex.)
}

%%%%%%%%%%%%%%%

\Comment{
Consider what we did to find a free line in 2-space.
Lines in 2-space are a 2-dimensional family.
We first restricted to tangent lines of the curve, a 1-dimensional family,
then to bitangents, a 0-dimensional family,
all by smoothly translating and rotating a free line in free space
(no constraints) to a free line that just touches the curve (1 constraint)
to a free line that touches the curve twice (2 constraints).

Let's try the same thing with surfaces.
Planes in 3-space are a 3-dimensional family, so we need to add 3 constraints.
Restrict to tangent planes of the surface, a 2-dimensional family
(ignoring cusps).
Can you find a tangent plane that touches 2 cusps?

\begin{lemma}
If any plane in dual space is free,
one of the cusp tritangents of $S^*$ is free.
\end{lemma}
\prf
We can assume that the kernel of $S$ is not empty,
otherwise no planes in dual space are free.
Since a surface with a self-intersection has an empty kernel,
we can also assume that $S$ has no self-intersections.
But conventional bitangent and tritangent planes of $S^*$ are associated
with self-intersections of $(S^*)^*$, and $(S^*)^* = S$.
Thus, all of the tritangent planes of the tangential surface
are cusp tritangents.
See Figure~\ref{}.

Consider a free plane in dual space.
While preserving its freedom, we can move this free plane
until it touches the surface once,
then pivot until it touches the surface again, generating a bitangent plane,
then pivot yet again until it touches the surface a third time,
generating a tritangent plane.
This pivoting is always possible because we are pivoting about cusps.
%
%	This will not necessarily work for conventional bitangents.
%	But it does when we incorporate cusps, as follows.
%	Push the free line until it touches a cusp (what if it can't reach a cusp?).
%	Then rotation is free about this cusp, until we become tangent to the curve.
%	(The problem with rotation about a non-cusp is that you are moving along
%	the curve in order to maintain tangency and you may go to infinity
%	before a second point of tangency is found.)
%
%	Notice that $S^*$ will have a (cusp) bitangent, since 
%	$S$ has an inflection point and $S^*$ has a cusp.
\QED
}

% \subsection{Benefits of reduction}

% We have presented a method for reducing the computation of the kernel
% to the computation of a convex hull in dual space.
	% We do not argue that this method is faster than the computation of the kernel
	% purely in primal space.\footnote{In primal space, the kernel of $S$ is 
	% the intersection of the inside halfplanes of the tangent planes of parabolic points of $S$.}

% {\bf I assume that you establish that the kernel is the intersection
% of the inside halfplanes of the parabolic points' tangent planes?}

% {\bf Interesting future research question (not for this paper):
% convex hulls with cusps.}

% What are the tritangent caps of a convex hull?

	% The seed points probably do not compute the entire kernel, like with the curve,
	% since they are only finite in number (and the kernel is probably defined
	% by more than its corners).  On second thought, yes they may still define.

\section{Acknowledgements}

We thank Wenping Wang for alerting us to a dual relationship between the kernel
and the convex hull.
We are appreciative of the stimulating research environment provided by the Dagstuhl
Seminar on Geometric Modeling.

\bibliographystyle{plain}
\begin{thebibliography}{Johnstone 02a}

\bibitem[Elber 02]{elber02}
Elber, G. (2002)
The kernel of freeform planar parametric curves.
Presented at Dagstuhl 2002.

\bibitem[Hartshorne 77]{hartshorne}
Hartshorne, R. (1977) 
Algebraic Geometry.
Springer-Verlag (New York).

\bibitem[Johnstone 01]{jj01}
Johnstone, J. (2001)
A Parametric Solution to Common Tangents.
International Conferenced on Shape Modelling and Applications (SMI2001),
Genoa, Italy, IEEE Computer Society, 240--249.

\bibitem[Johnstone 02a]{jj02}
Johnstone, J. (2002)
The tangential curve.

\bibitem[Johnstone 02b]{jj02tangsurf}
Johnstone, J. (2002)
% A robust dualization of tangent space:
The Bezier tangential surface system.

\bibitem[Struik 1961]{struik}
Struik, D. (1961)
Lectures on Classical Differential Geometry.
Dover (New York).

\end{thebibliography}

\ifJournal
Our kernet set is sharp, in the sense that an upperbound is sharp:
there is no smaller kernel set that would always define the kernel.
There are examples where the intersections of inflection tangents
define the kernel (if the convex hull is built out of lines between cusps,
such as a 3-cusped concave triangle) and other examples where the
intersections of inflection tangents with the curve define the kernel
(if the tangential curve has one cusp only),
so both components of the kernel set are necessary in some cases.
\fi

\ifTalk
In our reduction of the kernel to the convex hull,
we need to find a seed point in the kernel.
The algorithm for querying a point's inclusion in a star-shaped
polygon \cite[p. 44]{preparata85} 
also relies on finding a seed point in the kernel to
act as the origin.
\fi

\ifJournal
Possible extension to open curves:

Our method won't work as is on an open concave curve,
because it assumes that K defines a closed set
and open curves will have open kernels that go off to infinity.
However, it could probably be easily changed to work:
at least to define the kernel inside the convex hull of
the open curve.
For example, add the line segment between the endpoints as
a pseudo-inflection tangent.
\fi

\end{document}

%%%%%%%%%%%%%%%%%%%%%%%%%%%%%%%%%%%%%%%%%%%%%%%%%%%%%%%%%%%%%%%%%%%%%%%%%%%%%

\begin{lemma}
The kernel is convex and connected.
\end{lemma}

{\bf No need to compute bitangents in this algorithm, and thus no need
for the following discussion.}

\subsection{Bitangents}

Tangential curves are excellent at computing bitangents
of curves.

The bitangents of a curve will be important tools in this paper.
In \cite{jj01}, we showed how to compute the bitangents of a smooth curve
efficiently using a dual representation of the tangent space of the curve, 
called the tangential curve.

We use a dual space to compute bitangents.
Bitangents of $C$ are self-intersections of the tangential curve system of $C$
(i.e., self-intersections of the clipped tangential a-curve $C_a^*$
and self-intersections of the clipped tangential b-curve $C_b^*$).

Bitangents of a plane curve \cite{jj01,jj02}.

Convex hull of closed plane curve \cite{jj02,elber01}.

%%%%%%%%%%%%%%%%%%%%%%%%%%%%%%%%%%%%%%%%%%%%%%%%%%%%%%%%%%%%%%%%%%%%%%%%%%%%%

The kernel is robustly computed using tangential c-curves,
although bitangents are robustly computed using tangential a-curves and b-curves,
packaged into the tangential curve system.
The c-curve is appropriate after the seed point of the kernel point
has been moved to the origin, because the tangent space of the curve 
will not intersect the origin and so the tangential c-curve is finite
(whereas the a-curve and b-curve are inherently infinite since a closed
curve must have horizontal and vertical tangents).
Even before the seed point is calculated, the c-curve works better
than the a-curve or b-curve.
%
We guarantee that the tangent space of the curve does not intersect the origin.
	% since we shall compute a convex hull in dual space, which is inherently
	% closed and therefore inherently straddles both a-dual and b-dual spaces,
	% which complicates the hull computation.

\begin{rmk}
Once we find a point of the kernel and translate it to the origin,
we can exclusively use tangential c-curves for any problem (e.g., bitangent),
since they have no problems with infinity, and ignore the subtlety of
clipping and tangential curve systems (Section~\ref{sec:duality}).
\end{rmk}

