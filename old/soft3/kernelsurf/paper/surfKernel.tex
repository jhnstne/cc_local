\documentclass[12pt]{article}
\usepackage[pdftex]{graphicx}
\usepackage{times}

\newcommand{\Comment}[1]{\relax}  % makes a "comment" (not expanded)
\newcommand{\QED}{\vrule height 1.4ex width 1.0ex depth -.1ex\ \vspace{.3in}} % square box
\newcommand{\lyne}[1]{\mbox{$\stackrel{\textstyle \leftrightarrow}{#1}$}}
\newcommand{\seg}[1]{\mbox{$\overline{#1}$}}
\newcommand{\prf}{\noindent{{\bf Proof}:\ \ \ }}
\newcommand{\choice}[2]{\mbox{\footnotesize{$\left( \begin{array}{c} #1 \\ #2 \end{array} \right)$}}}      
\newcommand{\scriptchoice}[2]{\mbox{\scriptsize{$\left( \begin{array}{c} #1 \\ #2 \end{array} \right)$}}}
\newcommand{\tinychoice}[2]{\mbox{\tiny{$\left( \begin{array}{c} #1 \\ #2 \end{array} \right)$}}}

\newtheorem{theorem}{Theorem}	
\newtheorem{rmk}[theorem]{Remark}
\newtheorem{example}[theorem]{Example}
\newtheorem{conjecture}[theorem]{Conjecture}
\newtheorem{claim}[theorem]{Claim}
\newtheorem{notation}[theorem]{Notation}
\newtheorem{lemma}[theorem]{Lemma}
\newtheorem{corollary}[theorem]{Corollary}
\newtheorem{defn2}[theorem]{Definition}
\newtheorem{observation}[theorem]{Observation}
% \input{header}

\setlength{\oddsidemargin}{0pt}
\setlength{\topmargin}{0in}	% should be 0pt for 1in
\setlength{\textheight}{8.6in}
\setlength{\textwidth}{6.875in}
\setlength{\columnsep}{5mm}	% width of gutter between columns

\begin{document}

%%%%%%%%%%%%%%%%%%%%%%%%%%%%%%%%%%%%%%%%%%%%%%%%%%%%%%%%%%%%%%%%

\section{Computing the kernel of a surface from the convex hull}

There is a dual relationship between the kernel and convex hull
for closed surfaces.
This relationship can be used to compute kernels from convex hull algorithms,
as we shall explore in this section.
The {\bf kernel} of a closed surface $S$ is the locus of points that can see every
point of $S$:
\[
\mbox{kernel}(S) = \{ P \in R^3: \seg{PQ} \ \cap \ S = \emptyset 
		\hspace{.25in} \forall \ Q \in S \}
\]
It is a convex connected subset of the surface's interior.
The kernel is an important structure to understand in developing a theory of visibility.
For example, interpreting the surface as the boundary of a room,
a point of the kernel would be an ideal location to place a light, camera, or robot,
since the entire room would be visible.

The computation of kernels from hulls
offers a different point of view on the computation of the kernel,
allowing all attention to be placed on the development of an algorithm
for the efficient and robust computation of the convex hull.
We can appeal to the Voronoi diagram and Delaunay triangulation for a comparable example.
There are algorithms for the computation of each of these structures,
but many still prefer to always compute the Voronoi diagram by computing
the Delaunay triangulation and then dualizing.

%%%%%%%%%%%%%%%%%%%%%%%%%%%%%%%%%%%%%%%%%%%%%%%%%%%%%%%%%%%%%%%%

\subsection{Duality and tangent space}
\label{sec:duality}

We must begin by developing some theory of duality.
There is a geometric duality between points and planes in 3-space \cite{hartshorne}.
Classically, the plane $ax+by+cz+d=0$ is dual to the point $(a,b,c,d)$ in projective 3-space
(or equivalently, the point $(\frac{a}{d},\frac{b}{d},\frac{c}{d})$ in Cartesian 3-space),
although many other point-plane dual maps can be defined.
Using geometric duality, the tangent space of a surface $S(u,v)$ can be mapped 
to a surface $S^*(u,v)$ in dual space, by dualizing the tangent plane at $S(u,v)$ to the point $S^*(u,v)$.
We call $S^*$ a tangential surface, as it represents a tangent space.
This mapping to dual space is complicated by the fact that, for a closed surface with empty kernel,
some tangent planes will inevitably map to infinity.
To solve this problem, 
three cooperating dualities are needed to represent a tangent space robustly.
The reader is referred to \cite{jj03tangsurf} for the complete details
on this representation, called the tangential surface system 
(Figure~\ref{fig:peartangsystem}).
However, when the kernel of $S$ is not empty and a single point of this
kernel is known,
a simpler dual representation of $S$'s tangent space is possible.\footnote{Since many
	surfaces have empty kernels, this is not a general solution to
	the representation of tangent spaces.}
After translating the kernel point to the origin,
the tangential d-surface alone is a robust representation of the tangent space of $S$.

\begin{defn2}
The plane $ax+by+cz+d=0$ is {\bf d-dual} to the point $(\frac{a}{d},\frac{b}{d},\frac{c}{d})$.
The point $(a,b,c)$ is {\bf d-dual} to the plane $ax+by+cz+1=0$.
$S^*$ is the {\bf tangential d-surface} of the surface $S$ if $S^*(u,v)$ 
is the d-dual of the tangent plane at $S(u,v)$ (Figure~\ref{fig:peartangdsurf}).
\end{defn2}

\begin{figure}[h]
\begin{center}
\includegraphics*[scale=.36]{img/peartangsystem.jpg}
\end{center}
\caption{The tangential surface system of a pear}
\label{fig:peartangsystem}
\end{figure}

\begin{figure}
\begin{center}
\includegraphics*[scale=.36]{img/peartangdsurf.jpg}
\end{center}
\caption{The tangential d-surface of a pear}
\label{fig:peartangdsurf}
\end{figure}

\begin{lemma}
\label{lem:drobust}
If the origin lies inside the kernel of $S$,
the tangential d-surface of $S$ is a robust representation of the tangent space of $S$.
\end{lemma}
\prf
Planes through the origin are mapped to infinity by the d-duality.
Since the origin lies inside the kernel,
no tangent plane will pass through the origin (Lemma~\ref{lem:primalkernelchar} below), 
so no tangent plane will map to infinity.
\QED

%%%%%%%%%%%%%%%%%%%%%%%%%%%%%%%%%%%%%%%%%%%%%%%%%%%%%%%%%%%%%%%%

\subsection{A reduction to the convex hull}
\label{sec:reduction}

We are now ready to establish the dual relationship between kernel and convex hull
for a closed surface $S$.
The general structure of this relationship will be discussed in this section,
while Section~\ref{sec:cusp} will study its details and provide a complete algorithm.
In this section, it is assumed that a single point of the kernel has been computed and the surface $S$
has been translated so that this point is at the origin.
This makes the tangential d-surface $S^*$ into a closed finite surface
and a robust representation for the tangent space (Lemma~\ref{lem:drobust}).
Section~\ref{sec:firstpt} will treat the computation of a single point of the kernel.

A point of the kernel is a point that sees the entire surface.
An alternative characterization is available for surfaces,
as observed by Gershon Elber \cite{elber02}:
a point inside the kernel is not hit by any tangent plane.\footnote{A boundary point
	of the kernel is hit either by its own tangent plane or by the tangent plane 
	of a parabolic point.}

\begin{lemma}
\label{lem:primalkernelchar}
The point $P$ lies strictly inside the kernel of a smooth closed surface $S$ if and only if 
no tangent plane of $S$ intersects $P$.
\end{lemma}
\prf
If $P$ does not lie in the kernel,
then $P$ does not see some point $Q \in S$.
That is, $\seg{PQ}$ hits the surface in a second point $R$.
Since the surface is closed, 
as the line $\lyne{PR}$ is rotated about $P$,
it will eventually become tangent to the surface.
The tangent plane at this point contains $P$.
Thus, if $P$ does not lie in the kernel, then it is hit by some tangent plane.
A point of the kernel boundary, as a limit of this argument, will also be hit by some tangent plane.
This shows that if no tangent plane of $S$ intersects $P$, 
$P$ must lie in the interior of the kernel.

Conversely, if the tangent plane at $Q \in S$ intersects $P$, then $P'Q$ will intersect $S$ twice for some point $P'$ in 
the neighbourhood of $P$, meaning $P'$ is not in the kernel.
Thus, if $P$ lies in the interior of the kernel, no tangent plane of $S$ can intersect $P$.
\QED

\noindent This characterization of a kernel point can be reinterpreted in dual space.

\begin{defn2}
A plane $P^*$ in dual space is {\bf free} if $P^*$ does not intersect the tangential surface $S^*$.
\end{defn2}

\begin{corollary}
\label{cor:dualkernelchar}
The point $P$ lies strictly inside the kernel of $S$ if and only if 
the plane $P^*$ is free.
\end{corollary}
\prf
The statement 'no tangent plane of $S$ intersects $P$' in primal space
is equivalent to the statement 'no point of $S^*$ lies on $P^*$' in dual space.
\QED

Corollary~\ref{cor:dualkernelchar} identifies kernel points in primal space
with free planes in dual space.
Since the free planes of a surface bound its convex hull,
this immediately suggests a relationship between the kernel and convex hull.

\begin{theorem}
\label{thm:kernelhull}
If the origin lies inside the kernel of $S$,
the kernel of $S$ is dual to the convex hull of the tangential d-surface $S^*$.
More accurately, the kernel is dual to the family of planes whose
envelope is the convex hull.
\end{theorem}
\prf
The convex hull is well defined, since $S^*$ is a closed, finite surface.
Since kernel points of $S$ correspond to free planes,
boundary points of the kernel correspond to free tangent planes of $S^*$
(free planes that are just about to intersect $S^*$).
The free tangent planes of a surface $D$ bound its convex hull: the point $P \in D$ lies
on the boundary of the convex hull of $D$ if and only if the tangent plane at $P$ 
does not intersect $D$.
Therefore, the boundary points of the kernel of $S$ correspond to the boundary points
of the convex hull of $S^*$.\footnote{Typically, points in primal space would be associated with planes in dual space,
	but we can also associate points with points:
	the point $p \in S$ in primal space is associated
	with the point $q \in S^*$ whose tangent plane dualizes to $p$.
	This is particularly easy when there is a parameterization, since $p$ and $q$ 
	have the same parameters.}
\QED

We delay a detailed treatment of how the kernel is computed from the hull
until Section~\ref{sec:cusp}, after the hull of the tangential surface is better understood.

%%%%%%%%%%%%%%%%%%%%%%%%%%%%%%%%%%%%%%%%%%%%%%%%%%%%%%%%%%%%%%%%

\subsection{The first point of the kernel}
\label{sec:firstpt}

The reduction of kernel to convex hull relies on the prior knowledge
of a single point of the kernel.
This point is used to adjust the surface before dualization, ensuring
the finiteness of the tangential d-surface.
This section discusses the computation of this first point of the kernel.

For the remainder of the section, we assume that $S$ is concave.
There is no loss of generality in only considering concave surfaces,
since the kernel of a convex surface $S$ is simply the interior of $S$.
% [if S is convex, then no parabolic points so no cusps, so obvious that S is convex]
% We can also assume that $S$ has a nonempty kernel, otherwise any
% candidate set will work.
We shall see in Section~\ref{sec:cusp} that the tangential surface $S^*$ of
a concave surface $S$ is also concave.
In particular, $S^*$ has bitangent planes.
Therefore, bitangent planes can be used in our construction. % of the candidate set.

The search for a kernel point will be reduced to the search for a free plane,
using the association of Corollary~\ref{cor:dualkernelchar}.
This moves the search to dual space, consistent with keeping the computation
of the kernel in dual space.
To make the search for a free plane tractable, a candidate set will be found.
A {\bf candidate set} is a finite set of planes in dual space
that necessarily contains a free plane (whenever the kernel is not empty).
We can test each candidate until a free plane is found.
If no free plane is found amongst the candidates,
the kernel must be empty.
Otherwise, the dual of the free plane is a kernel point,
which can be used to seed the algorithm of Section~\ref{sec:reduction}.
Since there are a finite number of candidates, this is a valid solution.

The challenge is to restrict the candidates for a free plane from
all planes (a triply-infinite system) to a manageable finite set.
There are three degrees of freedom to the search for a free plane.
Two of these degrees of freedom can be removed by restricting the search
to free bitangent planes.
Consider a free plane in dual space.
While preserving its freedom, we can move this free plane
until it touches the surface once, generating a tangent plane,
then pivot until it touches the surface again, generating a bitangent plane.
Sometimes, this process can be continued one more level,
moving the bitangent plane about the surface until it becomes a
tritangent plane.
However, in some cases, there may be no tritangent planes.
Thus, we must handle both cases, a surface with tritangent planes
and a surface with only one-parameter families of bitangent planes.

The candidate set is built as follows.
Consider a one-parameter family of bitangent planes of $S^*$.
If some of these planes are free and some are not free,
there will be a free tritangent plane at the boundary
between free and nonfree planes of the family.
Call a family of planes {\em homogeneous} if every plane is
free or every plane is not free.
We conclude that either all of the families of bitangent planes
are homogeneous or some tritangent plane is free.
This shows that the set consisting of all tritangent planes
and one (arbitrary) representative from each bitangent family
is a candidate set.
Notice that it is a finite set and contains a free plane when anyone exists.
By testing this candidate set, a free plane, and hence a kernel point, can be found.
This defines a dual approach to finding the first point of the kernel.

To get a robust seed point for the kernel,
all of the free planes in the candidate set should be found,
yielding a set of kernel points, and the seed kernel point 
should be defined to be the centroid of this set.
This works since the kernel is convex, and moves the point from the boundary
into the interior of the kernel.

Since there is no longer any way to guarantee the finiteness of the tangential surface
in mapping to dual space (unlike Section~\ref{sec:reduction}),
the computation of the first kernel point must use the tangential surface system 
for robustness (Section~\ref{sec:duality}),
by testing for freedom in all three cooperating dual spaces.

%%%%%%%%%%%%%%%%%%%%%%%%%%%%%%%%%%%%%%%%%%%%%%%%%%%%%%%%%%%%%%%%%%%%%%%%%%%%%%%%

\subsection{The hull's structure}
\label{sec:cusp}

This section examines the structure of the convex hull of a tangential
surface $S^*$, through an analysis of its bitangent and tritangent planes
which are necessarily degenerate when the surface $S$ has a nontrivial kernel.
This leads to a more precise understanding of the 
dual relationship between kernel and hull, culminating in an
algorithm for the construction of kernel from convex hull.

\begin{defn2}
A {\bf bitangent plane} of the surface $S^*$ is a plane that
is tangent to $S^*$ at two or more distinct points.
This plane is {\bf conventional} if the surface has unique, well-defined 
tangent planes at the points of tangency,
otherwise it is {\bf degenerate}.
\end{defn2}

\begin{lemma}
\label{lem:degenerate}
If $S$ has a nonempty kernel,
all bitangent and tritangent planes of the tangential surface $S^*$ are 
degenerate.
\end{lemma}
\prf
Let $S$ be a surface with nonempty kernel.
$S$ has no self-intersections, otherwise the kernel is empty
since no point can see through a self-intersection.
But conventional bitangent and tritangent planes of $S^*$ are associated
with double and triple self-intersections of $(S^*)^* = S$ \cite{jj02,jj03tangsurf}.
Thus, no bitangent and tritangent planes of $S^*$ are conventional.
\QED

In order to understand degenerate bitangent planes, we must understand cusps.
Consider a concave
surface $S$ with nonempty kernel.
$S$ will contain parabolic points (points with zero Gaussian curvature \cite{struik}):
after all, the kernel of a concave surface is the intersection of the inside
halfplanes of the tangent planes of its parabolic points.
Since the tangent planes at parabolic points of $S$ are dual to cusps on $S^*$,
the tangential surface $S^*$ will contain cusps 
(Figures~\ref{fig:peartangsystem}-\ref{fig:toptangdsurf}).
In general, these cusps are not isolated: they gather into closed curves on the surface.
Thus, one can talk about the tangent line of a cusp.

\begin{defn2}
\label{defn:tangline}
Let $P$ be a cusp of $S^*$ that lies on a curve of cusps $C$.
The {\bf tangent line} of $P$ is the tangent of $C$ at $P$.
\end{defn2}

\begin{figure}
\begin{center}
\includegraphics*[scale=.36]{img/toptangdsurf.jpg}
\end{center}
\caption{The tangential d-surface of a top}
\label{fig:toptangdsurf}
\end{figure}

Any plane through a cusp $P$ that contains the tangent line of $P$ may be considered
a tangent plane of the surface.\footnote{Upon entering a cusp $P$, 
	the tangent plane reverses direction, sweeping out all of 3-space 
	by rotating about the tangent line of $P$,
	until it is ready to leave $P$ on the other side in an opposite orientation.}
This leads to a degenerate case of the bitangent plane.

\begin{defn2}
A {\bf degenerate bitangent plane} is 
(1) a plane that contains the tangent line of two distinct cusps or
(2) a tangent plane (at some noncusp point) that contains the tangent line of a cusp. 
\end{defn2}

It can be shown that, if the surface contains two or more cusp curves,
a typical cusp point has a degenerate bitangent plane of the first type.
That is, given the tangent line of a cusp, 
it is possible to find a second tangent line that lies in the same plane.

\begin{defn2}
A {\bf degenerate tritangent plane} is 
a plane that contains the tangent line of three cusps, 
a tangent plane that contains the tangent lines of two cusps,
or a bitangent plane that contains the tangent line of one cusp.\footnote{The
	third case will not participate in the hull of $S^*$, since
	it implies a self-intersection in $S$.}
\end{defn2}

The structure of the hulls of our tangential surfaces can now be clarified.
A tangential surface will have a finite number of one-parameter families of degenerate bitangent planes
and a finite number of degenerate tritangent planes.
To create the convex hull, a surface is capped by bitangent developables
(the envelopes of bitangent plane families) and tritangent planes.
The bitangent developables
are of two types, matching the two types of degenerate bitangent planes: 
a lofting between two cusp curves (type 1) and
a lofting between a cusp curve and a curve on the surface (type 2).

Using this deeper understanding of the hull's structure,
we now describe how to compute the kernel directly from the hull.
The patches of the hull that lie on the surface $S^*$
cause no trouble: they correspond directly
to the patches of the kernel that lie on the surface $S$.
It is the translation of the patches of the hull that do not lie on the surface 
(the bitangent developables and tritangent planes)
that requires some elaboration.

Since a family of planes dualizes to a curve,
each bitangent developable $D^*$ in the hull of $S^*$ 
is associated with a curve $D$ in primal space.
If two developables $D_1^*$ and $D_2^*$ of the hull meet at a cusp curve, 
the kernel will contain a lofting between the associated curves 
$D_1$ and $D_2$ in primal space 
(which is well defined, since $D_1$ and $D_2$ 
will necessarily have the same parameterization).\footnote{Each line 
	of this lofting is generated by a tangent plane in dual space pivoting 
	about a point of the cusp curve,
	from one developable to the other developable.}

This development is summed up in the following algorithm for computing the kernel of $S$ 
from the convex hull of $S^*$.

\begin{enumerate}
\item Compute the convex hull of $S^*$.
\item For each patch of the hull that lies on $S^*$, add the associated patch of $S$
	to the kernel.
\item For each bitangent developable $D^*$ of the hull, compute the associated curve $D$ in primal space.
	$D$ is the dual of the bitangent planes along either boundary of $D^*$.
\item For each pair of neighbouring bitangent developables on the hull,
	add the lofting between their associated curves to the kernel.
\item For each bitangent developable that neighbours a tritangent plane,
	add the lofting between the developable's associated curve and the tritangent's 
	associated point to the kernel.
\end{enumerate}

$D$ is easily computable.
Suppose that the developable $D^*$ is the lofting between $C_1(t)$ and $C_2(t)$,
where $C_1(t)$ is a cusp curve if $D^*$ is of type 1 and
a curve of tangency with $S^*$ if $D^*$ is of type 2,
while $C_2(t)$ is a cusp curve.
For type 1 developables, the bitangent plane at $C_1(t)$ is the plane through
$C_1(t)$ and the tangent line of $C_2(t)$.
For type 2 developables, the bitangent plane at $C_1(t)$ is the tangent plane
at $C_1(t)$, so $D$ is the image of $C_1(t)$.\footnote{A curve in primal
	space and a curve in dual space are images if they share the same
	parameters on their respective surfaces.}
$D$ lies in the interior of $S$ for type 1 developables,
and on $S$ for type 2 developables.

This concludes the reduction of kernel to convex hull.
The structural relationship between the kernel and hull that is revealed
by this study is an important insight into these two fundamental geometric structures.

\section{Acknowledgements}

We thank Wenping Wang for alerting us to a dual relationship between the kernel
and the convex hull.
We are appreciative of the stimulating research environment provided by the Dagstuhl
Seminar on Geometric Modeling.
This work was supported by the National Science Foundation under 
grant CCR-0203586.


\bibliographystyle{plain}
\begin{thebibliography}{99}

% \bibitem[Elber 02]{elber02}
\bibitem{elber02}
Elber, G. (2002)
The kernel of freeform planar parametric curves.
Presented at Dagstuhl 2002.

% \bibitem[Hartshorne 77]{hartshorne}
\bibitem{hartshorne}
Hartshorne, R. (1977) 
Algebraic Geometry.
Springer-Verlag (New York).

% \bibitem[Johnstone 01]{jj01}
\bibitem{jj01}
Johnstone, J. (2001)
A Parametric Solution to Common Tangents.
International Conferenced on Shape Modelling and Applications (SMI2001),
Genoa, Italy, IEEE Computer Society, 240--249.

% \bibitem[Johnstone 02]{jj02}
\bibitem{jj02}
Johnstone, J. (2002)
The tangential curve.

% \bibitem[Johnstone 03]{jj03tangsurf}
\bibitem{jj03tangsurf}
Johnstone, J. (2003)
The Bezier tangential surface system: a dual structure for visibility analysis.

% \bibitem[Struik 1961]{struik}
\bibitem{struik}
Struik, D. (1961)
Lectures on Classical Differential Geometry.
Dover (New York).

\end{thebibliography}

\end{document}

