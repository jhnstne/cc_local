\documentclass[12pt]{article}
\usepackage{times}
\usepackage[pdftex]{graphicx}
\makeatletter
\def\@maketitle{\newpage
 \null
 \vskip 2em                   % Vertical space above title.
 \begin{center}
       {\Large\bf \@title \par}  % Title set in \Large size. 
       \vskip .5em               % Vertical space after title.
       {\lineskip .5em           %  each author set in a tabular environment
        \begin{tabular}[t]{c}\@author 
        \end{tabular}\par}                   
  \end{center}
 \par
 \vskip .5em}                 % Vertical space after author
\makeatother

% default values are 
% \parskip=0pt plus1pt
% \parindent=20pt

\newcommand{\SingleSpace}{\edef\baselinestretch{0.9}\Large\normalsize}
\newcommand{\DoubleSpace}{\edef\baselinestretch{1.4}\Large\normalsize}
\newcommand{\Comment}[1]{\relax}  % makes a "comment" (not expanded)
\newcommand{\Heading}[1]{\par\noindent{\bf#1}\nobreak}
\newcommand{\Tail}[1]{\nobreak\par\noindent{\bf#1}}
\newcommand{\QED}{\vrule height 1.4ex width 1.0ex depth -.1ex\ \vspace{.3in}} % square box
\newcommand{\arc}[1]{\mbox{$\stackrel{\frown}{#1}$}}
\newcommand{\lyne}[1]{\mbox{$\stackrel{\leftrightarrow}{#1}$}}
\newcommand{\ray}[1]{\mbox{$\vec{#1}$}}          
\newcommand{\seg}[1]{\mbox{$\overline{#1}$}}
\newcommand{\tab}{\hspace*{.2in}}
\newcommand{\se}{\mbox{$_{\epsilon}$}}  % subscript epsilon
\newcommand{\ie}{\mbox{i.e.}}
\newcommand{\eg}{\mbox{e.\ g.\ }}
\newcommand{\figg}[3]{\begin{figure}[htbp]\vspace{#3}\caption{#2}\label{#1}\end{figure}}
\newcommand{\be}{\begin{equation}}
\newcommand{\ee}{\end{equation}}
\newcommand{\prf}{\noindent{{\bf Proof}:\ \ \ }}
\newcommand{\choice}[2]{\mbox{\footnotesize{$\left( \begin{array}{c} #1 \\ #2 \end{array} \right)$}}}      
\newcommand{\scriptchoice}[2]{\mbox{\scriptsize{$\left( \begin{array}{c} #1 \\ #2 \end{array} \right)$}}}
\newcommand{\tinychoice}[2]{\mbox{\tiny{$\left( \begin{array}{c} #1 \\ #2 \end{array} \right)$}}}
\newcommand{\ddt}{\frac{\partial}{\partial t}}
\newcommand{\Sn}[1]{\mbox{{\bf S}$^{#1}$}}
\newcommand{\calP}[1]{\mbox{{\bf {\cal P}}$^{#1}$}}

\newtheorem{theorem}{Theorem}	
\newtheorem{rmk}[theorem]{Remark}
\newtheorem{example}[theorem]{Example}
\newtheorem{conjecture}[theorem]{Conjecture}
\newtheorem{claim}[theorem]{Claim}
\newtheorem{notation}[theorem]{Notation}
\newtheorem{lemma}[theorem]{Lemma}
\newtheorem{corollary}[theorem]{Corollary}
\newtheorem{defn2}[theorem]{Definition}
\newtheorem{observation}[theorem]{Observation}

% \font\timesr10
% \newfont{\timesroman}{timesr10}
% \timesroman


\newif\ifCommentary             % verbose version (adds nuance to the stark version)
\Commentaryfalse
\newif\ifCompanion              % material for companion paper
\Companionfalse
\newif\ifSurrounding            % if covering 'A surrounding L' case
\Surroundingtrue
\newif\ifJournal                % journal paper
\Journaltrue
\newif\ifTalk                   % comments for an oral presentation
\Talktrue
\newif\ifComment                % large-scale comments
\Commentfalse
\newif\ifOld                    % old versions
\Oldfalse

\newcommand{\hero}{star}

\setlength{\headsep}{.5in}
\markright{\today \hfill}
\pagestyle{myheadings}

\DoubleSpace

\setlength{\oddsidemargin}{0pt}
\setlength{\topmargin}{0in}	% should be 0pt for 1in
\setlength{\textheight}{8.6in}
\setlength{\textwidth}{6.875in}
\setlength{\columnsep}{5mm}	% width of gutter between columns

% -----------------------------------------------------------------------------

\title{The visible and the invisible} % Col 1:16
% Analyzing visibility I:\\ two curves.}
% Understanding the structure of invisibility\\I: Two closed curves.}
% Scenes of two closed curves.}
% Visibility in smooth flatland
% the visibility of a single smooth object in a smooth scene
% An analysis of invisibility for a smooth scene in flatland}
% The invisibility of a smooth object in flatland}
% Invisibility: an analysis of a smooth scene in flatland
% Where is a smooth object invisible? An analysis in flatland
% The visibility of a smooth object in flatland
% On the visibility of a smooth object
% \ifTalk
% Taking umbrage. (Good conference title?)
% \fi
% what an object sees
% Research paper: May 2003}
% Visibility analysis of a smooth curve I: 
%       Local umbra and penumbra: 
%       a single curved object lit by a curved area light
% The discontinuity mesh of a smooth scene of curves
% Global illumination from a smooth area light:\\the discontinuity mesh 
%        of a smooth scene in 2-space
% The geometry of shadow:\\the umbra cast by a curved light source\\
%        in a scene of curved obstacles
% Visibility in flatland.
% Shadows in flatland
% Bitangency in a scene of smooth curves,\\with applications to visibility
% Tangential curves for bitangency, with applications to smooth versions 
%        of the convex hull and visibility graph, and anticipations of 
%        shading from smooth lights in 3d	

\author{J.K. Johnstone\thanks{Mailing address: Department of Computer and 
    Information Sciences, UAB, Birmingham, AL 35294-1170.  This work
    was partially supported by the National Science Foundation under grant CCR-0203586.}}


\begin{document}
\maketitle

\vspace{.5in}

\ifTalk

\begin{quote}
For by him were all things created, that are in heaven, and that are in earth, 
visible and invisible (Colossians 1:16)
\end{quote}

\begin{quote}
The lamp of the body is the eye.  (Matthew 6:22)
\end{quote}

\begin{quote}
\ldots but everything exposed by the light becomes visible \ldots (Ephesians 5, 13)
\end{quote}

\begin{quote}
\ldots I have heard\\
The cock, that is the trumpet to the morn, \\
Doth with his lofty and shrill-sounding throat\\
Awake the god of day (Hamlet, I.I.154-157)
\end{quote}

\begin{quote}
Over the land there lies a long shadow\\
westward reaching wings of darkness\\
(The Return of the King, J.R.R. Tolkien, p. 812, the words of Malbeth the Seer)
\end{quote}

\begin{quote}
Harry Potter's invisibility cloak.
\end{quote}

\begin{quote}
Perseus and the Nymph's helmet of invisibility in the Medusa tale.
\end{quote}
\fi

\ifCommentary
A technical report on the umbra cast by a single object (the visibility of an object 
in a scene of two objects), both back and front; with accompanying software.
\fi

\clearpage

% -----------------------------------------------------------------------------

\begin{abstract}

This paper considers the visibility of a particular object, called the \hero, 
in a scene of smooth objects.
The viewpoints that do not see the \hero\ define two-dimensional cells in 
free space.\footnote{A viewpoint does not see the \hero\ if it does not see any point 
  of the \hero.}
We wish to build these cells.
The question 'where is the \hero\ invisible?' reduces to the question
'where does the visibility of the \hero\ qualitatively change?'.
These qualitative changes in visibility are called visual events, some of which define
the boundaries of the invisible cells.

Visibility and invisibility are important issues in occlusion culling and shadow
computation for computer graphics.
If the camera does not see an object, there is no need to insert that object into
the graphics pipeline, which saves computation.
If the light does not see a point, 
that point lies in the umbra and is shaded accordingly.
(The light acts as the \hero\ in this application; the invisible cells encode the umbra.)
Visibility and invisibility are also useful concepts in radiation therapy and 
surveillance.
A radiation source should be located in a cell that does not see 
a healthy organ (here the organ is the \hero).
In general, we may wish to verify that a particular location of a sensitive object
(the \hero) is safe from a dangerous source (which it is if the source lies in the
cells of invisibility).
In surveillance of a valued object (the \hero) by a flock of moving cameras, 
we want to guarantee that at any time, some camera lies outside the area of invisibility.

A novel aspect of this paper is its treatment of arbitrary smooth scenes in 2-space.
An object in the scene is represented by a closed curve in 2-space.
Each curve is arbitrary: there is no restriction such as convexity.
To model our concept of a distinct object in a scene,
each curve is simple (no self-intersections) and closed,
and the interiors of two curves are disjoint.
This model for an object differs from earlier work on polygonal or convex scenes.
For certain applications, it is useful to interpret a curve in the scene
as the footprint of a 2.5-dimensional extruded surface.

Related work on visibility 
includes art gallery theorems (but these work with polygons and polyhedra, 
and consider a viewpoint that sees everything); visibility skeletons (again linear
polygons and polyhedra); aspect graphs (this work is more descriptive and 
less algorithmic); and counting bitangents (in computational geometry; 
again not algorithmic; a complexity result).

\clearpage

{\bf Context/setting}: A collection of objects bounded by closed curves in 2-space.

{\bf Cast}: One of the objects is distinguished as the subject of our visibility analysis.

{\bf Issue/conflict}: Where is the distinguished object invisible? 
  Equivalently (and more practically), where can the camera be 
  and not see the distinguished object? 
  Equivalently (and also practically), what is the umbra of the
  scene when the distinguished object is interpreted as a light and every object is
  interpreted as the footprint of an infinite 2.5D curved extrusion?

{\bf New issue}: the distinguished object is an arbitrary closed curve (not polygonal or convex)

{\bf Sub-issue/conflict}: Where are the visual events for hard visibility?

{\bf Subsub-issue}: Where are the direct (or type 1) visual events for hard visibility?
Where are the indirect (or type 2) visual events for hard visibility?

The definition of visual events has three stages:
\begin{enumerate}
\item define the bitangents involved in visual events (a filtering process)
\item define the part of each bitangent that contributes to a visual event
\item form invisible regions from the visual events
\end{enumerate}
\end{abstract}

\vspace{1in}

{\bf rondo names (workstations)}: alabama, canada, visibility, mississippi, abracadabra

\ifOld
Old version of abstract:
\begin{abstract}
We are interested in the visibility analysis of a scene of smooth objects.
In this paper, we consider the foundational case of a scene of [two] arbitrary 
closed curves in 2-space.
We characterize the points that do not see one of the objects in the scene
(which collectively define the umbral cells of this object), 
    % if the object is the light, this is the umbra;
    % even in this special case, cells are more finely delineated than the entire umbra
through a study of the visual events at the boundaries of these cells.
We next characterize the points that only partially see one of the objects 
(which collectively define the penumbral cells of this object),
again through a study of the associated visual events.
As an illustration of the theory,
the progression of the visual events and cells is studied as one curve surrounds another
in a timeseries.
\end{abstract}
\fi

\clearpage

% -----------------------------------------------------------------------------

Preparatory tasks: {\bf data collection for 2/2.5D visibility}

\begin{itemize}
\item get test data, in the form of {\bf fonts} (a natural curved data source) which are also
      needed for the font rendering project
\item get test data, in the form of {\bf smooth architectural footprints} 
  (blueprints of cityscape?) which are also needed for an architecture project;
  2.5D examples are ideal applications, for 2.5D lighting 
  and for motion in 3D upon a floor
\item get test data, in the form of {\bf leaves} (from leaf jpgs)
\item one approach to gathering curve data is to collect polygonal data,
      and then interpolate the vertices by a cubic spline; the {\bf panel} 
      would be a good source for this.
\item Nazca monkey is an ultimate complex example
\item need test data with concavities and one object surrounding or at least overlapping
      another, to illustrate the full complexity of visual events
\item add the capacity to include linear segments 
      in an interpolation (necessary for fonts and architecture)
\item clean up and {\bf release software} that, given a Bezier curve (indirectly through a set of points
      to interpolate), generates (and stores?) {\bf all bitangents},
      using tangential curves
\end{itemize}

Tasks once theory is developed:

\begin{itemize}
\item paint scene with umbra and penumbra
\end{itemize}

Preparatory tasks for 3D visibility:

\begin{itemize}
\item use surfaces of revolution
\item download {\bf polyhedral meshes} and get parsers
\item smooth the meshes: option 1 is triangular patches (difficult); 2 is subdivision surfaces (but doesn't generate Bezier surface); totally different option is to develop
  visibility theory for meshes interpreted smoothly (like {\bf discrete differential geometry})
\item compute {\bf tangential surfaces}
\item compute {\bf bitangent developables}
\end{itemize}

\clearpage

% -------------------------------------------------------------------------------------

\section{Introduction}

Visibility analysis is a fertile area of research in computer graphics
and computational geometry, because of its applications to the computation of 
shadows,             % dynamic shadows
global illumination,
occlusion culling,   % less so because of the advantages of conservative visibility
camera control (where it can be used to plan camera motions that maintain visibility
  or invisibility, {\bf a separate paper}),
motion planning (where visual events form a good basis for path skeletons, 
since shortest paths follow visual events, {\bf a separate paper}), and
surveillance (i.e., planning locations for optimal visibility, such as 
the work on art gallery theorems in computational geometry \cite{orourke}).

This paper begins a formal study of visibility in smooth scenes,
extending previous analyses of polyhedral scenes 
\cite{durandthesis,discontinuity-mesh} to the analysis of smooth scenes.
Each object in the scene is bounded by an {\em arbitrary} closed curve or surface.
Our approach is more algorithmic than the work in the vision literature 
on aspect graphs \cite{aspectgraph}
and the recent work in the computational geometry literature 
on counting and analyzing the complexity of bitangents,
free bitangents, and visibility structures \cite{aggarwal,hazel}.

We are interested in analyzing the visibility of a single object in the scene,
which will be called the {\bf distinguished object}.
When does this object become visible?  
How can free space be decomposed into regions that do not see this object?
%
[We begin modestly with the basic but intriguing case of two curves.
{\bf Future papers} will expand on this work to deal with scenes of arbitrarily many 
closed curves, scenes with closed surfaces, and partial visibility.]

Throughout the paper, uppercase characters refer to the curves in the scene (e.g., A, B).
These are closed, simple curves whose interiors do not intersect.
A is reserved for the distinguished object whose visibility is being analyzed.
We are interested in finding the visual events that reveal A,
and then using these visual events to compute the cells where A is not seen at all.

ADD FIGURE OF OAK LEAF AND ITS INVISIBILITY CELLS.

\clearpage

\subsection{The first line of sight past an obstacle}

We need to understand how to look past, or beyond, an object B.
Each object of the scene blocks certain regions from sight.
Given a viewpoint P and an object B, we need to understand the region of space
whose visibility is blocked by B, or equivalently the first lines of sight
that see past B.
After all, the star may be hiding in this invisible region behind B.

\begin{defn2}
Let $B$ be an object defined as the interior of a simple closed curve $B'$.
Let $P$ be a viewpoint outside $B$ [must distinguish between P in convex hull and outside].
The {\bf region blocked by $B$} from viewpoint $P$
The {\bf cone of invisibility} defined by $B$ from viewpoint $P$
The {\bf first line of sight past $B$} from $P$.
\end{defn2}

1) the conoid region blocked
2) the first line of sight that sees past B

Figure FOO7

\clearpage

START HERE

\subsection{Approaches}

\subsubsection{Local}

{\bf 

Good clean problem for WSCG (moved to doiseea.tex).
Also a component of testing whether a bitangent defines a visual event.

Given a point P, a scene and a distinguished object A (the star),
what region is invisible from P, and is A invisible from P (i.e., does it completely lie
in the invisible region)?

to determine if A is visible from P:
polygons $\rightarrow$
curves   $\rightarrow$
tangents through P $\rightarrow$
visual tangents through P ('extremal' and 'piercing') $\rightarrow$
invisible cells (1 per object), or invisible cones (except be careful of front umbra) 
$\rightarrow$
visible cone for star A (same as invisible cone for object B, but with different 
                         interpretation) $\rightarrow$
sort of visual bitangents about P, and plane sweep to determine areas of coverage (number
of blocking objects per region): 
   are A's visual bitangents in a region of positive coverage?
   and once blocking objects are determined, do A's visual tangents 
   intersect these objects before they reach points of tangency: if yes, A is invisible.
[this could be a paper]

reminiscent of visibility graph computation by duality and line arrangements

impractical for computing locus of viewpoints that don't see A
(since there are an infinite number of viewpoints to test)
but OK for interactive local test

also useful background for understanding when A can peek out from behind B

From a local point of view (camera is standing at symbolic P, what does it see now?),
lines of sight past B and umbra are interesting: does A lie in the region behind B?

Equivalent to shadow computation of point light P (but now smooth objects).

For efficiency (in the many cases when P does see A),
we could first test if P's line of sight to A along the tangents to A are free;
this would avoid the large test if probability is high that P sees A.
}

\clearpage

\subsubsection{Global: object blockage}

Given an object B and the star object A, 
what viewpoints in 2-space are fully blocked from A by B?

bitangents of A and B, bitangents of B and C
extremal bitangents, piercing bitangents: potential visual events
region blocked by B

option 1: start with the region blocked by B and expand by taking other objects C 
          into account; in particular, use the approach of the Eurographics submission,
	  which can be interpreted as follows:
	  starting from a (potential) direct visual event of A and B,
	  if an object C interferes, then sweep in B's tangent space, looking for
	  a true indirect visual event; and sweep in A's tangent space, looking for
	  a true direct visual event

	  the advantage of this approach is cohesion: we are growing a region;
	  at all times we have a conservative estimate of the invisible region

	  still benefits from an understanding of direct vs. indirect visual events
	  and an understanding of the requirements of a true direct visual event
	  and a true indirect visual event

	  problem: how would you ever sweep to the correct visual event in FOOBAR14?
	  it seems that we have to identify visual events by applying all constraints
	  rather than sweeping, but then how do we organize visual events into cells?

option 2: classify all bitangents as visual events or not
          then define cells from the halfplanes of visual events
	  problem: difficult to build cells as a postprocessing step

process for option 1:

curves -->
bitangents -->
potential visual events (w.r.t. the two curves that define the bitangent) -->
true visual events (using sweep process) -->
invisible cells

\clearpage

{\bf 
idea (9/16/05):
\begin{itemize}
\item
compute all visual events by filtering bitangents through constraints
  (e.g., bitangent must be free between points of tangency, must not intersect star, ...);
\item
build cells associated with each object B (start with direct visual events built from
  bitangents of star and B, then clip by all indirect visual events);
\item
combine overlapping cells (if one of B's visual events intersects C, then collapse with
  C's cell)
\end{itemize}
}

\clearpage

precursors:
what is a visual event? (direct vs. indirect, constraints on each)

2 papers on visibility in early fall:
\begin{itemize}
\item {\bf what is a visual event?}
\item visibility of an object from a point P
\end{itemize}

\noindent 
3 papers on quaternion spline in fall:
\begin{itemize}
\item generalized cross product
\item rational map to sphere
\item emptiest point on sphere
\end{itemize}

\noindent 
3 papers on camera control in fall:
\begin{itemize}
\item joystick GUI control
\item collision correction
\item orientation constraints
\end{itemize}

\noindent 
3 projects in fall:
\begin{itemize}
\item curve editor
\item Cg shader
\item mesh parsing
\end{itemize}

WSCG: what is a visual event?

SoCG: visibility from P or global object visibility

GI:   orientation constraints (or WAFR)

Curves and Surfaces: curve editor

CCCG: generalized cross product or rational map to sphere or emptiest point on sphere

WAFR: orientation constraints

ACMSE: collision correction

game conference: joystick GUI control

journal: 3d kernel

Lei and David: software release and Tomb Raider

David: motion GUI, Cg; 

Lei: ruled surface GUI, visibility

David: bounce ideas of various papers off him

Michael and Aqueasha: choose a project

\clearpage

THEN CONTINUE HERE

\subsection{Visual event}

{\bf 
We could characterize a visual event as follows. (A = star, B = blocking object)

\begin{itemize}
\item bitangent [peeking out: first becoming visible]
\item free bitangent [visible]
\item outer bitangent (both objects on same side) [characteristic of direct ve]
\item does not intersect A first in either direction [A is not visible along line of sight]
\item A is not visible from P = point of tangency with B
  [for testing visibility of A from P in the forward direction, past A, we can rotate
   to look for A, since visibility of A from the rotated point of tangency implies
   visibility from the original point of tangency; however, 
   when testing visibility of A from P in the negative direction, past B, we cannot just
   rotate to look for A, since visibility of A from the rotated point of tangency does
   not necessarily imply visibility from the original point of tangency; so we need
   to test visibility of A from P directly]
\item visual event occurs behind B (at most until first intersection with an object)
\end{itemize}

key operations to optimize and make robust: 
find first intersection of line (bitangent) with scene;
find which side of a tangent an object lies (where the object is tangent);
visibility of A from P (yes/no)

DOES P SEE A?
}

\clearpage

\subsection{}

From a global perspective (where we are trying to understand visibility in the entire 
scene),
the abstract visual event is of more interest (at what bitangents does A appear?)
(after having established that bitangents are interesting through the thought experiment
of A peeping out behind B at a bitangent)
or the umbra cast by an entire object B, which then has to be expanded by other objects.
The latter seems less productive since it involves addition of objects one at a time.
The former's abstraction may be more powerful.
An examination of the two approaches on Figure 3 of visual event page would be useful
in analyzing their relative advantages.

\subsubsection{Global: visual event}

\clearpage

\section{Visual events}
\label{sec:ve}

\begin{defn2}
The point p {\bf sees the point} q if the line segment \seg{pq} does not intersect an
object in the scene.
$p$ {\bf sees the object} A if it sees some point of A.
   % Thus, a point does not see A if it does not see any point of A.
   % that is, for every point $q \in A$, the line segment \seg{pq}
   % intersects some object B of the scene, $A \neq B$.
\end{defn2}

\begin{defn2}
A {\bf visual event} for A is a point at which the visibility of A 
   % abruptly (if restricting to hard visual events)
qualitatively changes.
A {\bf hard visual event} is a point at which A first becomes visible,
and a {\bf soft visual event} is a point at which A first becomes 
entirely visible.\footnote{A visual event 
  is usually associated with a curve
  or surface, such as an EEE event surface \cite{teller92}.
  We use a finer unit, for additional flexibility in a cluttered environment.
  For example, several disconnected segments of a line 
  can be visual events in a complicated scene (CAN IT REALLY?).}
\end{defn2}

Visual events are important because they bound the umbra and penumbra.

\begin{defn2}
The {\bf umbra} of a scene (with respect to the distinguished object A)
is the subset of free space that does not see A.
The {\bf penumbra} of a scene is the subset of free space that sees part, 
but not all, of A.
An {\bf umbral (resp., penumbral) cell} is a connected component of the umbra 
(resp., penumbra).
\end{defn2}

Visual events in a scene of closed curves occur at bitangents \cite{durandthesis}.
%
% \begin{lemma}
% Visual events in a scene of closed curves occur at bitangents.
% \end{lemma}
% \prf
%
The issue of which bitangents, and which parts of these bitangents, 
is the topic of this paper.

% Note that we don't need to distinguish between the front and back umbra at this stage,
% and perhaps forever, because of the use of visual events that blur this distinction.

\begin{defn2}
A {\bf bitangent} of A and B is a line that is tangent to both A and B.
A {\bf self-bitangent} of B is a line that is tangent to B in two distinct points.
\end{defn2}

\subsection{}

To understand invisibility of the \hero, 
we must understand when the \hero\ can become visible.

Suppose that the \hero\ is presently invisible.
The \hero\ can become visible in two different ways: looking past a blocking object
and looking through a hole between two blocking objects.

Develop the idea of 'extremal tangent of C from a point P' (one that contains the entire
curve C on one side).
This captures looking past C from P.

[Consider hard visibility.]
It is important to distinguish between direct and indirect visual events.
Starting from a point that does not see A (P in Figure~\ref{fig:directvisualevent}),
imagine moving past a blocking object B, looking for A.
The line of sight past B is a tangent sweeping in the
tangent space of B, searching for A over the horizon
(Figure~\ref{fig:directvisualevent}).
If the line of sight to A remains clear, 
A will be seen % (causing an abrupt, qualitative change in visibility)
when the tangent first touches A,
at which time it will be bitangent to A and B
(the darker line in Figure~\ref{fig:directvisualevent}).
We call this a {\bf direct visual event}.
If the line of sight past B to A is blocked by an intervening object C
(Figure~\ref{fig:indirectvisualevent}),
A will first be seen when the sweeping tangent sees past C, at which time
it is bitangent to B and C (Figure~\ref{fig:indirectvisualevent}).
We call this an {\bf indirect visual event}
(since it does not involve a bitangent with A).
% With an indirect visual event, A is seen through a hole between B and C.
% \QED
In other words, in analyzing the visibility of the object A,
direct visual events occur at bitangents of A and B,
as A appears over the horizon of B,
while indirect visual events occur at bitangents of B and C,
as A appears through a hole bounded by B and C.
% \footnote{In unusual cases, C may be the same curve as B.}

% tangentialCurve -l ../../umbra/data/vg9.pts
\begin{figure}[h]
\begin{center}
\includegraphics*[scale=.25]{img/twoClosedCurve/directVisualEvent.jpg}
\end{center}
\caption{A direct visual event.  As a tangent sweeps in B's tangent space, a visual
  event is found at a bitangent. (ADD A,B notation, and P)}
\label{fig:directvisualevent}
\end{figure}

% Figure fig:indirectvisualevent here
% tangentialCurve -l ../../umbra/data/globalumb2.pts
\begin{figure}[h]
\begin{center}
\includegraphics*[scale=.25]{img/twoClosedCurve/indirectVisualEvent.jpg}
\end{center}
\caption{An indirect visual event.  Since B's tangent is blocked by C
         before it reaches A, A is first seen (by a viewer behind B) when crossing a
	 bitangent of B and C. (ADD A,B,C notation).}
\label{fig:indirectvisualevent}
\end{figure}

For the present purposes, we assume that no object in the scene surrounds another.
This is the common case, and simplifies the theory considerably.

\begin{defn2}
A {\bf surrounds} B if the convex hull of A contains all of B (FIGURE).
\end{defn2}

The type of bitangent involved in a direct visual event is quite different
from the type involved in an indirect visual event:
the former involve supporting (or outer) bitangents,
while the latter involve separating (or inner) bitangents, 
which we now define.

\begin{defn2}
\label{defn:extremal}
A bitangent of A and B is {\bf extremal} if it does not intersect A or 
B.\ifCommentary\footnote{Clearly, points 
  of tangency are not counted as intersections.}\fi
\label{defn:outerbitang}
An extremal bitangent T of A and B is {\bf outer} if A and B lie on the same side of T,
and {\bf inner} if A and B lie on opposite sides of T.
\end{defn2}

\clearpage

\subsection{Anatomy of a visual event}

Visual events involve only part of a bitangent: 
we must develop a terminology for these parts.

\begin{defn2}
\label{defn:seg}
Consider a bitangent of A and B.
The {\bf A-early segment} is the infinite ray starting
at the point of bitangency with A and moving away from B.
The {\bf B-early segment} is the infinite ray starting
at the point of bitangency with B and moving away from A.
The {\bf middle segment} is the segment between the points of tangency.
% A {\bf prefix} of an A-early (resp., B-early) segment is a subsegment starting 
% at its point of tangency with A (resp., B).
% If one of the curves, say B, is distinguished, the B-early segment is also called
% the {\bf late segment}.
% The {\bf late segment} of a bitangent of A and L is the infinite ray starting at the
% point of bitangency with A and moving away from 
% L\ifCommentary (Figure~\ref{fig:outer}b)\fi.
% The asymmetry of this definition relies on the distinguished nature of L.
\end{defn2}

The B-early segment of a bitangent may define a visual event (caused by B blocking A),
but only part of it, the B-prefix, will participate. 

\begin{defn2}
\label{defn:prefix}
Consider the B-early segment of a bitangent of A and B.
The {\bf B-prefix} is the subsegment
from the point of tangency with B until the first intersection with an object (if any).
% A, B, or another visual event (if any).
% Note: this is only a good definition when A and B are the only curves in the scene.
% When general scenes are considered, we are just looking for potential visual events
% and it is better to define 'until the first intersection with an object (if any)',
% which defines a superset of the visual event.
% or {\bf the first intersection with a visual event V such that the remaining bitangent of A and B
% lies outside V} (if any).
% (The {\bf inside of a visual event} is ---.)
\end{defn2}

Another necessary condition for a visual event is an unblocked
line of sight between A and B.

\begin{defn2}
A bitangent is {\bf free} if its middle segment does not intersect, 
or even become tangent to, any object of the scene.
We also say that a free bitangent of A and B {\bf sees} A and B.
\end{defn2}

% Figure fig:bitang here

Some additional definitions are necessary later.

\begin{defn2}
Let T be a bitangent of A and B, such that A and B lie on the same side of T
in the neighbourhood of their points of tangency.
The {\bf inside} of T is the side that contains A and B in the local neighbourhood.
The {\bf outside} of T is the opposite side.
\end{defn2}

\begin{defn2}
Let C be a closed curve and P a point outside C.
A tangent of C through P is {\bf extremal} if it does not intersect C.
C will have two extremal tangents through P, which span C.
\end{defn2}

May not be necessary (details of tangency vs actual intersection dealt with as a comment):
\begin{defn2}
A line segment {\bf weakly intersects} a curve A if it either intersects A 
or is tangent to A.
A line segment {\bf strongly intersects} a curve A if it intersects A transversally
(not just tangentially touching).
\end{defn2}

The paper is organized as follows.
In the next section, we analyze the conditions that must be satisfied by a bitangent 
that is a visual event.
   % and distinguish two types of visual event.
   % one involving the appearance of A past a single object, 
   % and the other involving the appearance of A through a hole between two objects.
Once the conditions for a visual event are defined, 
we establish the connection between visual events and umbral cells.
Finally, to illustrate the analytical tools for visibility that have been developed, 
to more fully understand visibility for the seminal case of a scene of two curves, 
and to explore the dynamic nature of visibility,
we consider visibility in a scene of two curves A and B,
throughout a time-series as B gradually surrounds A.

\clearpage

% -------------------------------------------------------------------------------------

\section{Bitangents that define visual events} 
% and thus outer 
% A taxonomy of visual events
\label{sec:ve}

% Visibility analysis in smooth flatland involves bitangents.

% -------------------------------------------------------------------------------------

A visual event occurs somewhere along a free bitangent,
but this is not the only condition that must be satisfied by a visual event.
We now analyze the necessary and sufficient conditions for a visual event.
   % it is only necessary if you only test freedom wrt A and B, making it a local test
We begin with direct visual events.
% e.g., a free bitangent between A and B that later intersects A is not a visual event

{\bf Develop the constraints one by one, using illustrative examples.}

Suppose that T is a bitangent of A and B that contains a direct visual event.
Recall the definition of a direct visual event in terms of a tangent sweeping 
in the tangent space of B (Section~\ref{sec:ve}).
Since we are looking past B to see A, 
which has just appeared over the horizon,
there must be a clear line of sight to A: 
\begin{itemize}
\item T is free
\end{itemize}
Figure~\ref{fig:free} shows that we must also disallow points of tangency on the middle segment.

% figure 'star' in notes: illustrating freedom constraint
% umbra ../data/2circ.pts
% umbra ../data/3circ.pts
\begin{figure}
\begin{center}
\includegraphics*[scale=.25]{img/twoClosedCurve/2circ.jpg}
\includegraphics*[scale=.25]{img/twoClosedCurve/3circ.jpg}
\end{center}
\caption{}
\label{fig:free}
\end{figure}

Since A has just appeared over the horizon of B,
\begin{itemize}
\item A and B must lie on the same side of T in the neighbourhood
of their points of tangency with B.
\end{itemize}
Figure~\ref{} shows that this is only a local constraint 
(figure square0 of B crossing T, A crossing T).

By the definition of a visual event, A must not have been visible immediately
before the visual event.
This has several ramifications.
Let $P_A$ and $P_B$ be T's points of tangency with A and B, respectively.
We are interested in the visibility of A along the segment of T
immediately behind B (the B-prefix of T), or equivalently immediately behind $P_B$.
The first constraint speaks to the visibility of A looking forward along the bitangent,
the second to the visibility of A looking backwards along the bitangent,
and the third to the visibility of A looking up.
Consider the visibility forward first.
The freedom constraint (above) makes it impossible for A to be already visible between
$P_A$ and $P_B$, but to ensure that A is not already visible behind $P_A$, it is 
necessary that A is not the first object encountered by T:
\begin{itemize}
\item A is not the first object intersected by the A-early segment of T.
\end{itemize}

Figure triangle (with A just peeking over the bitangent, but still blocked by C).

Looking backward, A must also be invisible, 
which means that A cannot be the first object encountered by T:
\begin{itemize}
\item A is not the first object intersected by the B-early segment of T.
\end{itemize}
Figure -- shows that T cannot even be {\em tangent} to the B-prefix of T.

Figure 'fig:backwards' showing this.

If the A-early and B-early segments of T do not intersect A at all, 
these constraints are enough.
In this case, A lies below the bitangent and some part of the B-prefix will not see A,
being blocked by B both forward and backward.
(See Theorem~\ref{thm:visualseg} for how much of the B-prefix is a visual event.)
% NO: For example, whenever A does not surround B, the above conditions are sufficient
% to distinguish a direct visual event.
However, more testing is necessary when the A-early or B-early segments intersect A.
Consider Figures~\ref{fig:triangle, full version} or \ref{fig:square}, for example, 
where the bitangent does not define a visual event even though it satisfies the above 
constraints.
In this case, A may already be visible from above the bitangent.
This is the third case of visibility-testing.

Figure 'fig-square' and 'fig-triangle, both abridged and full versions'.

We can test whether A is already visible as follows.
A sweep is made across blocking objects, looking for A.
A ray L is swept, starting at the A-early segment of T, which we know is blocked from A by an 
above constraint, and then at the B-early segment of T, which we also know is blocked from A.
A sweep stops once a visible part of A is found (A is already visible)
or once we sweep past all of A (A is not already visible).
Each sweep is implemented as follows (the parameter L is a ray starting at $P_B$):
\begin{itemize}
\item Algorithm 'sweep(L)'
\item sweep direction about $P_B$ = from L towards the outside of T
\item while (L hits some C before A \&\& L eventually hits A)
\begin{itemize}
\item L = extremal tangent of C through $P_B$ (the first one found by rotating in 
  the sweep direction)
\end{itemize}
\item if (L hits A first) return (true) else return (false) 
\end{itemize}
The first sweep is sweep(L), where L is the A-early segment of T.
The second sweep is sweep(L), where L is the B-early segment of T.


START HERE


Care must be taken with points of tangency with A.
These are legal in some cases and illegal in others.
The full analysis is tedious but rather straightforward.






\subsection{Direct visual events}

         % a {\bf direct visual bitangent}.
         % A bitangent that defines a direct visual event (participates in)
         % will be called a {\bf directly visual} bitangent.

A visual event in a smooth scene will occur at a bitangent;
and, by its very nature, a direct visual event will occur at an outer bitangent
(Figure~\ref{fig:directvisualevent}).
To determine which outer bitangents qualify as direct visual events,
we observe that, as a visual event is crossed, 
A must become visible where it was not before.
The filter therefore has two components: 
A must be visible and A must not already have been visible.
For A to have become visible along the bitangent, 
the bitangent must be free (i.e., no object intersects
the middle segment, where it would block the line of sight).
For A to not already have been visible requires a subtler test.
If A wraps around to enter the bitangent's outside halfspace (Figure),
we must guarantee that it is always blocked from the proposed visual event.
Suppose that an early segment of the bitangent hits A: then we must guarantee
that there is an earlier intersection with another object that will block A.
We must then sweep away from the bitangent and check that other objects continue
to block A until we have passed A (Figure).
Notice that this second test is only necessary if A is concave, and the sweeping
portion of the test is only necessary if A can almost surround another object
(since A could not wrap around a blocking object to peek out the other side
unless it almost surrounds that object).

Another way that A could already be visible is if it is already visible on B's early
segment.
This can occur even if A does not intersect the bitangent (Figure 9 on white pad's notes).

\begin{theorem}
\label{thm:directve}
   % Suppose that A does not surround B.
   % SHOULD BE OK WITHOUT THIS RESTRICTION: if A surrounds B, then any bitangent
   % of A and B will violate condition 2.
An outer bitangent T of A and B defines a direct visual event for A if and only if
          % singular noun is better, although less precise
\begin{enumerate}
\item {\bf (define outer, and then remove this condition)} A and B lie on the same side of T in the neighbourhood of their points of tangency,
\item T is a free bitangent,
% (the middle segment of T does not touch any object in the scene, even tangentially),
\item A is not the first object intersected by the A-early or B-early segment of T,
\Comment{
\item the A-early segment of T does not strongly intersect A,\footnote{Of course, points of tangency with A are allowed, even two or more points of tangency.}
      at least until after it strongly intersects some object $C \neq A$ (C may be B),
% transversally,\footnote{The purpose of the adjective 'transversally' is to ignore points of tangency.} 
\item the B-early segment of T does not weakly intersect A, %, or even come tangent to A, 
      at least until after it strongly intersects some object $C \neq A$ (C may be B), and
}
\item A is not already visible along the B-prefix of T.  Equivalently,
      the function 'seeAalready(T,A)' returns false, where this function is 
      defined as follows:
\begin{itemize}
\item L = B-early segment of T; $P_B$ = point of tangency of T with B
\item DIR = a direction of rotation about $P_B$, from L towards the outside of T
\item while (the ray L hits some C before A \&\& L eventually hits A)
\begin{itemize}
\item L = ray of next extremal tangent of C through $P_B$, rotating in direction DIR
\end{itemize}
\item if (L hits A first) return (true) else return (false) 
\end{itemize}
% need algorithm test, since A may wrap around behind B and become visible again, destroying this visual event: see figure box{1} and box{2}
\end{enumerate}
The direct visual events along this bitangent lie on the B-prefix of T
(see Theorem~\ref{thm:visualseg}).
% These conditions are also sufficient if the scene has only two curves.
% For general scenes, it may not be the entire prefix if another object
% is encountered.
\end{theorem}
\prf
Suppose that T is a bitangent of A and B that defines a direct visual event for A.
Recall the definition of a direct visual event in terms of a tangent sweeping 
in the tangent space of B (Section~\ref{sec:ve}).
Since we are looking past B to see A, 
which has just appeared over the horizon,
there must be a clear line of sight to A: T is free.
Figure --- shows that we must also disallow points of tangency on the middle segment.
Since A has just appeared, 
A and B must lie on the same side of the bitangent in the neighbourhood
of their points of tangency with B.
% This establishes that the visual event must be an outer visual bitangent.
By the definition of a visual event, A must not have been visible immediately
before the visual event.
This has several ramifications.
% The bitangent must be extremal with respect to A.
Looking forward along the bitangent from $P_B$ to $P_A$
(where $P_A$ and $P_B$ are the points of tangency with A and B, respectively),
the bitangent must not intersect A, except at points of tangency.
{\bf Also, another component of A must not be visible: --- as in Figure-square.}
And looking backward, A must be invisible, at least in the immediate neighbourhood
of $P_B$, which means that A cannot be the first object that intersects the B-early
segment.
Figure -- shows that T cannot even be {\em tangent} to the B-prefix of T.
Since one must be looking past B towards A,
the visual event begins at the point of tangency with B
and extends away from A.
It cannot continue past an intersection with another object, which will newly block A.
Therefore, visual events are restricted to the prefix of the B-early segment.
It may be further restricted, as described in Theorem~\ref{thm:visualseg}.

Now suppose that T is a bitangent that satisfies the three conditions of the theorem.
There is a visual event along part of the B-prefix of T,
since A is visible past B (freeness) and was not visible already (other conditions).
We shall analyze below how far visual events extend along the prefix,
based on interactions with other visual bitangents (Theorem~\ref{thm:visualseg}).
\QED

ADD FIGURE BOX[1], BOX[2] AND BOX[3] SHOWING HOW A ALREADY BE VISIBLE EVEN IF FIRST 4
CONDITIONS ARE MET, AND THUS MOTIVATING THE ALGORITHM ADDITION.

\Comment{
For general scenes (more than 2 curves),
it is not worthwhile to construct the exact visual event at this time, 
since this involves a global analysis based on other visual events.
For example, it is not enough to simply define the visual event as the prefix 
of the B-early segment up to the first encounter with another visual event 
(Figure of globalumb4.pts: {\bf this works for the local umbra of B in globalumb4.pts,
but consider the two disjoint components of the visual event of B and L in globalumb4.pts,
so it does not work for the global umbra cast by A}).
Therefore, we only define the bitangent involved in a direct visual event
and delay the construction of the visual event to a later stage.
}

Figure~\ref{fig:directve} shows some examples of direct visual events.
% general scenes: the bitangents associated with
% direct visual events, and the prefix of their B-early segments, which is known to %
% contain the visual event.
% illustrating the necessity of the constraints of Theorem~\ref{thm:directve} 
% as well as their sharpness (i.e., no other local constraints can be added).

% Figure fig:directve here

% Figure fig:dve here 

% Figure fig:directumbrae here

ADD FIGURES REFERENCED IN PROOF.

ADD FIGURE of 2d/scene/data/lightSceneCornellLip2, which illustrates multiple segments
of visual events along a single visual bitangent.

% -------------------------------------------------------------------------------------

\clearpage

\subsection{Indirect visual events}

A different set of constraints is satisfied by a bitangent that defines
an indirect visual event. % an {\bf indirect visual bitangent}

\begin{theorem}
\label{thm:indirectve}
A bitangent T of B and C defines an indirect visual event for A if and only if
\begin{enumerate}
\item T is a free bitangent,
\item the B-early segment of T does not intersect A, or even come tangent to A, 
      until after it intersects some object $C \neq A$ (C may be B), 
\item the C-early segment of T does intersect A (transversally, not tangentially), 
      before it intersects, or is tangent to, any other object, and
\item B and C lie on opposite sides of T in the neighbourhood of their points of tangency
      (MOVE UP?).
\end{enumerate}
The indirect visual events along this bitangent lie on the B-prefix of T
(see Theorem~\ref{thm:visualseg}).
         % The visual event extends along the B-early segment
         % from its point of tangency with B until its first intersection with B or C 
         % (if any) or until another visual event?? (if any).
\end{theorem}
\prf
Suppose that T is a bitangent of B and C that defines an indirect visual event for A.
Indirect visual events occur as one looks past an object (say B) and then another object
(say C) to view A through a hole between B and C (Section~\ref{sec:ve}).
Imagine a tangent sweeping on B until it sees past the intervening object C.
The visual event occurs at a bitangent of B and C.
B and C must lie on opposite sides of the bitangent,
otherwise they do not define a hole (and A was already visible before the visual event,
as can be seen by rotating the tangent away from C).
Since this bitangent must see A, and we are not dealing with a direct visual event,
the bitangent must intersect A transversally on the C-early segment.
The B-early segment cannot intersect A, otherwise A was already visible (looking behind)
before this visual event.
   % The B-early segment cannot intersect A, otherwise A is already visible
   % on both sides of the B-early segment.
Since the line of sight to A must be clear, T is free of intersections with other objects
both on the middle segment and on the C-early segment up to the first
intersection with A.
The visual event clearly begins at the point of tangency with B and extends away from C
along the B-early segment, stopping at the first intersection with an object.

Now suppose that the bitangent T satisfies the conditions of this theorem.
There is a visual event along the B-prefix of T,
  % immediately behind B,
since A is visible past B and C (first three conditions) and was not visible already
(second and fourth conditions).
\QED

A special case of the indirect visual event arises from self-bitangents of B.
All of the conditions of Theorem~\ref{thm:indirectve} are equivalent 
except the two points of tangency are now both on B.
% earlier called B-inner
This is the only type of indirect visual event in a scene of two curves.

Figure~\ref{fig:indirectve} shows some examples of indirect visual events,
illustrating the necessity and sharpness of the constraints of the following theorem.

% Figure fig:indirectve here

% Figure fig:ive here

% Figure fig:indirectumbrae here

COMPARE THIS TO DISCONTINUITY MESH DEFINITIONS OF VISUAL EVENTS.

\clearpage

\begin{defn2}
The bitangents that satisfy the conditions of Theorem~\ref{thm:directve}
will be called {\bf direct visual bitangents},
since they define direct visual events.
The bitangents that satisfy the conditions of Theorem~\ref{thm:indirectve}
will be called {\bf indirect visual bitangents},
since they define indirect visual events.
Collectively they will be called visual bitangents.
\end{defn2}

           % HOW COULD A VISUAL BITANGENT NOT DEFINE A VISUAL EVENT?
           % IT CAN IF WE RESTRICT TO 'FREE WRT A AND B ONLY'.
           % THIS MAKES IT A LOCAL TEST.
           % BUT WE CAN ALWAYS MAKE THIS RESTRICTION AFTER THE FACT USING ABOVE THEOREMS.

\begin{theorem}
\label{thm:visualseg}
% Let T be a bitangent satisfying the conditions of Theorem~\ref{thm:directve}.
The visual events associated with a visual bitangent T lie on % the segments of 
the B-prefix of T % that lie 
between the $(2i)^{th}$ and $(2i+1)^{st}$ intersections with the B-prefixes
of other visual bitangents.
(Here the beginning of the prefix is considered the 0th intersection,
and the end of the prefix, which may be a point at infinity, 
is considered the last intersection.)
\end{theorem}
\prf
START HERE
Consider the visual event along a B-prefix of a bitangent of A and B.
The visual event cannot continue past another visual event, 
which will newly reveal A in another direction.
...
\QED

\begin{defn2}
The segments of Theorem~\ref{thm:visualseg} will be called the 
{\bf visual segments} of the visual bitangent.
\end{defn2}

BUILD EXAMPLE FOR THEOREM~\ref{thm:visualseg} 
FOR BOTH DIRECT (A BETTER ONE) AND INDIRECT VISUAL EVENTS.

\clearpage

% -------------------------------------------------------------------------------------

\section{The structure of visual events}
\label{sec:structureve}

Even in a scene of two curves,
there can be arbitrarily many direct visual events
and arbitrarily many indirect visual events (Figures~\ref{fig:directve}b 
and \ref{fig:indirectve}c).
We want to add some structure to the collection of visual events.
Our first observation is that all but a few of the direct visual bitangents 
between A and B will intersect B.
For example, note that most of the bitangents in Figure~\ref{fig:directve}b intersect B.

ADD DEFINITION OF 'A surrounds B' HERE OR EVEN ABOVE (need it in proof of Lemma 13).

ADD DEFINITION OF 'chordal region' HERE OR EVEN ABOVE (need it in proof of case 2 in Lemma 13).

ADD DEFINITION OF 'inside and outside of bitangent of A and B (such that A and B lie on same side)'
(need it in proof of case 2).

ADD EXPLICIT STATEMENT of the set of 3 weak conditions for a direct visual event.

In analyzing the bitangents of A and B to prove this result, 
it is not interesting to enforce freedom of the bitangents 
with respect to all objects in the scene, since this is independent of A and B.
We instead analyze a relaxed version of the conditions of Theorem~\ref{thm:directve}, 
where the first condition ('the middle segment of T does not intersect
{\em any} object') is replaced by 'the middle segment of T does not intersect A or B'.
These are called the {\bf weak} conditions for a direct visual event.

We start with a special case.

\begin{lemma}
\label{lem:convex}
Let B and C be two convex objects that do not contain each other.
Exactly two bitangents satisfy the weak conditions of a direct visual bitangent.
\end{lemma}
\prf
If B does not intersect C, B and C have 4 bitangents:
two are supporting (A and B on the same side)
and two are separating (A and B on opposite sides).
The two supporting bitangents satisfy the weakened version of Theorem~\ref{thm:directve}.
The two separating bitangents violate condition 3 of Theorem~\ref{thm:directve}.

As B and C intersect, the two separating bitangents are lost,
but the two supporting bitangents remain.
\QED

We now limit the number of non-intersecting bitangents.

\begin{lemma}
\label{lem:atmost2}
Let $D$ be the bitangents of A and B that satisfy the weak conditions for a direct 
visual event.
The number of bitangents in D whose B-early segments do not intersect B 
(call this number $\Gamma$)
is at most 2.
\end{lemma}
\prf
{\bf Case 1: Suppose that A and B do not surround each other.}
Let $T \in D$ such that the B-early segment of T does not intersect B.
Since $T \in D$, we know that T does not intersect A (by condition 2 of a direct visual bitangent).
Therefore, A lies entirely on one side of T, the same side that B lies on (by condition 3
of a direct visual bitangent).
B is also sealed off entirely on one side of T, which can be seen as follows.
B cannot cross T on its B-early segment (by assumption),
on its middle segment (by condition 1 of a direct visual bitangent, 
even in its weakened state), or
on the A-early segment (otherwise B surrounds A, contradicting our assumption: 
see Figure~\ref{scratch paper}).
Since neither A nor B intersects T,
T is not only a bitangent of A and B, but a bitangent of convexHull(A) and convexHull(B).
These convex hulls do not contain each other, since A and B do not surround each other.
By Lemma~\ref{lem:convex}, only two bitangents of ch(A) and ch(B) 
satisfy the weak conditions.
This shows $\Gamma = 2$ when A and B do not surround each other.

% The inner/outer bitangents of two curves are equivalent to the bitangents of the
% convex hull of A and the convex hull of B.)

{\bf Case 2: Suppose that B surrounds A.}
Since A and B do not intersect, A must lie in a concavity of B.
Let $T_1$ and $T_2$ be two bitangents that satisfy the weak conditions
for a direct visual event, whose B-early segments do not intersect B.
If either of these bitangents does not exist, we are done.
Suppose that a third bitangent T satisfies the weak conditions for a direct visual event,
whose B-early segment does not intersect B.
For the B-early segment of T to be free of intersections with B, it must
leave through the hole in B's concavity (since T certainly enters the concavity, as 
it lies tangent to A, and only one of the early segments can leave through the concavity's
hole).
Therefore, B's point of tangency with T cannot lie inside $T_1$ or inside $T_2$.
Therefore, it lies in the region R defined by the intersection of the 
outside of $T_1$ and $T_2$.
%
FIGURE ON YELLOW PAD
%
But this is a contradiction: none of B can lie in the region R, 
since B cannot cross its boundary, as it does not intersect
the middle or B-early segments of $T_1$ or $T_2$ (and, of course, B has a component
not in R).
Therefore, a third bitangent T cannot exist.

% -------------------------------

\Comment{
Let $T_i \in D$ such that the B-early segment of T does not intersect B.
Since $T_i$ does not intersect A (condition 2 of Theorem~\ref{thm:directve}),
the argument is unaffected by replacing A by its convex hull.
Therefore, we assume without loss of generality that A is a convex object.
}

\Comment{
Consider the self-bitangent T of B that spans the concavity that holds A,
with points of tangency $P_1 \in B$ and $P_2 \in B$.
T defines a chordal region that contains A (the concavity).
Since $T_i$ is tangent to A, it must enter and exit the finite chordal region.
$T_i$ intersects B as it leaves through the curved boundary of the concavity.
$T_i$'s point of tangency with B must lie inside the chordal region,
since all of B lies inside T (in that T defines part of the convex hull: IS THIS NECESSARY?)
and the middle segment of $T_i$ must not intersect B.
Since all of $T_i$'s intersections with B must lie on the A-early segment,
$T_i$'s point of tangency with B is closer to T than its point of tangency with A,
and $T_i$ must exit the chordal region through T.
}

\Comment{
B must connect the point of tangency of $T_1$ with B
and the point of tangency of $T_2$ with B (define these points),
passing through region 2 (the region that lies inside both bitangents).
Suppose that there is a third bitangent $T_3$ that satisfies the weak conditions
for a direct visual event and whose B-early segment does not intersect B.
For $T_3$ to have a free middle segment, its point of tangency with B
must lie in region X (FIGURE).
But then its B-early segment intersects B, a contradiction.
We conclude that $T_3$ does not exist, and that there are at most two satisfying
bitangents.
}
\Comment{
Moving in the tangent space of B from T, 
we can sweep into the concavity until we hit A in two different ways:
one starting from $P_1$, yielding the bitangent $T_1$, and the other starting from $P_2$,
yielding the bitangent $T_2$ (FIGURE).
[Could these two bitangents be the same?]
$T_1$ and $T_2$ satisfy the weak conditions and their B-early segments do not intersect B 
(FALSE: see counterexample on p. 40, Belle).
%
Notice that $T_1$ (resp., $T_2$) defines a chordal region of B that lies inside the
concavity and encloses A (FIGURE).
Since $T_3$ is tangent to A, it must enter and exit this chordal region,
crossing $T_1$ (resp., $T_2$) and B inside the concavity.
%
Since the B-early segment of $T_3$ is free of intersections with B,
its point of tangency with B must lie 'outside' $T_1$ (define outside).
By the same argument, $T_3$ must also have a point of tangency with B outside $T_2$.
However, B cannot enter the region outside both $T_1$ and $T_2$, since B cannot cross
the B-early segment of $T_1$ or $T_2$. 
}

% -------------------------------

{\bf Case 3: Suppose that A surrounds B.}
% note: there can still be visual events by interacting with other objects C
\QED

\clearpage

The following lemma particularizes Lemma~\ref{lem:atmost2}.

\begin{lemma}
Let $D$ be the bitangents of A and B that satisfy the weak conditions for a direct visual event.
Let $\Gamma$ be the number of bitangents in D whose B-early segments do not intersect B 
($\mbox{early}_B(T) \cap B = \emptyset$).
\begin{enumerate}
\item
If neither A nor B surrounds the other, $\Gamma = 2$.
\item
If A encases B or B encases A, $\Gamma = 0$.
\item 
If neither A nor B encases the other,
exactly two bitangents satisfy $\mbox{early}(T) \cap B = \emptyset$ (?).
\item 
If A surrounds B but does not encase it, $1 \geq \Gamma \leq 2$.
\item 
If B surrounds A but does not encase it, $1 \geq \Gamma \leq 2$.
\end{enumerate}
\end{lemma}
\prf
\begin{enumerate}
\item
\item
One violates the freedom condition and the other violates the 'B-early segment does not
intersect B' condition.
\item
But one of these may not be free wrt B.
\item
\item
\end{enumerate}
\QED

\clearpage

\begin{lemma}
\label{lem:Aouter}
\ifCommentary
Let A and B be closed curves whose interiors have no intersection.
\fi
Suppose that B surrounds A.
A and B have at most two A-outer bitangents.  (OLD PROOF.)
% \item If B surrounds but does not encase A, 
%      then A and B have two A-outer bitangents. % we mean 'exactly two'
% \item If A surrounds but does not encase B, 
%      then A and B have two B-outer bitangents.
\end{lemma}
\prf
Suppose that B surrounds A: that is, A lies inside the convex hull of B.
Since A does not intersect B, it must lie in a concavity of B.
Let T be the bitangent of B that caps this concavity and belongs to the convex hull of B, 
with points of bitangency $P_B$ and $P_{B'}$.
If we sweep in the tangent space of B,
starting at $P_B$ and moving into the concavity, until the tangent touches A,
we shall find a bitangent T of A and B.
T may be an A-outer bitangent: 
both A and B lie on the same side of T at their points of bitangency,
and T does not intersect A, since it is the limit of a sweep for which both of these 
properties are always true; and
since A lies in a concavity, T will certainly intersect B after it touches A
(i.e., on the A-early segment).
The deciding factor is the status of the middle and B-early segments of T,
which may or may not intersect B (Figure of time series).
By a similar argument, an analogous sweep in the tangent space of B, starting at $P_{B'}$ 
rather than $P_B$ and again moving into the concavity, may generate a second
A-outer bitangent.

We now argue that there are no other A-outer bitangents.
Consider an A-outer bitangent of A and B.
% Its point of tangency with B cannot lie inside the concavity housing A,
% YES IT CAN.
% If its point of tangency with B lies outside the concavity that contains A, ...
% Therefore, the point of tangency with B must lie at the entry to the concavity.
% We have found the only two A-outer bitangents that satisfy this constraint.

THIS SEARCH TECHNIQUE MAY FAIL: SEE [BELLE P. 40].
BUT WE CAN PROVE AN UPPERBOUND BASED ON THE CONCAVITY ALONE, WITHOUT SWEEPING.

Suppose that B surrounds A: that is, A lies inside the convex hull of B.
Since A does not intersect B, it must lie in a concavity of B.
Let T be an A-outer bitangent, tangent to B at $P_B$.
Since A lies in a concavity of B, T will certainly intersect B after it touches A
(on the A-early segment): let $K \in B$ be the first such intersection.
Consider the curve segment \arc{P_B K}.
Figure on bottom of [p. 40, BS].
Since this curve segment surrounds A, any other tangent of A will intersect it, and thus B.
In particular, any other A-outer bitangent will intersect B.
But A-outer bitangents can only intersect B on their A-early segments.
Thus, the A-early segment of other A-outer bitangents must point into T's inside halfspace,
implying that the B-early segment points into T's outside halfspace.
Therefore, the point of tangency on B of any other A-outer bitangent must lie
in T's outside halfspace.
Suppose that there is a second A-outer bitangent U.
By the same argument, the point of tangency on B of a third A-outer bitangent
must lie in T's outside halfspace and U's outside halfspace (*).

Now note that their B-early segments do not intersect B.
% these A-outer bitangents must leave through the concavity's hole,
Think of the implication of this: the intersection of T's outside halfspace 
and U's outside halfspace is a region partially bounded by these B-early segments.
Since B is connected and cannot cross into this region, no part of B lies in this region.
By (*), there can be no third A-outer bitangent.
\QED

\clearpage

(When neither A nor B surrounds the other, these two bitangents also enjoy freedom
of the entire bitangent line; after they surround but don't encase, one of
the early segments intersects one of the objects.
However, this is irrelevant now that our analysis does not need to distinguish
between outer and A-outer bitangents.)

Section 7.8 (\ref{sec:noaouter}) on p. 25.
If B encases A, A and B have one 'consistent' piercing bitangent (Belle, p. 23).

{\bf Why is this important?}
Limit of two is important for defining visual events that end other visual events
like in Figure~\ref{fig:directve}c.

See 'A and B have at most 2 A-outer bitangents when B surrounds A' proof 
in Section~\ref{sec:BsurroundA} below.

\clearpage

Our next observation is the analogous one for indirect visual events:
all but a few of the indirect visual bitangents
between B and C (where the C-early segment hits A) will intersect B (is this true?).
See Figure~\ref{fig:indirectve}c.

\begin{theorem}
Similar bound for indirect visual events.
\end{theorem}

\subsection{Indirect visual events}

Develop a similar count for indirect visual events that do not intersect B.

Also filter down to the bitangents that satisfy indirect visual event conditions.
There could be many (p. 51 of Belle).
At most one of these (?) does not intersect B and this one defines an infinite
umbral region bounded by the B-early segment, a curve segment of B, and
a non-intersecting direct visual bitangent (see above paragraph; prove that there
is always a non-intersecting direct visual accompanying a non-intersecting indirect visual.)

% Let T be a direct bitangent.
% If the B-early segment of T intersects B,
% T defines a closed umbral region bounded by the prefix of the B-early segment 
% (a line segment) and 
% the curve segment of B from T's point of 
% tangency with B, moving away from A, to T's first intersection with B.
% % moving into the bitangent.

% --------------------------------------------

% \begin{itemize}
% \item direct bitangents whose B-early segment intersects B
% \item other direct bitangents (limited number)
% \item indirect bitangents whose B-early segment intersects B (p. 51, Belle)
% \item other indirect bitangents (surroundingSpiral)
% \end{itemize}

\clearpage

% ------------------------------------------------------------------------------------

\section{Raw of 'Building invisibility cells'}

(This section will eventually be ignored, 
but it has good ideas for next section, so keep until
incorporated into next section.)

Present counting argument for 'visual bitangents that hit B' first.

We want to build invisibility cells, using the visual events.
Visibility cells are bounded by visual events (visual bitangents).

{\bf intersects B}
Each visual bitangent that intersects B defines an invisibility cell.
(Could it define more than one?)
The typical case is illustrated by Figure~\ref{fig:directve}b.
Defn: If T is a visual bitangent, define cell(T) to be the 
invisibility cell(s) bounded by T.
{\bf direct that intersects B}
For a direct visual bitangent T that intersects B, cell(T) = the chordal region 
defined by T's B-prefix.
Proof: cannot see A again anywhere along the B-prefix.
{\bf indirect that intersects B}
For an indirect visual bitangent T between B and C that intersects B, 
cell(T) = the chordal region intersected with the inside halfspace of C's direct v.e.
bitangent on 'other side of C'.
Proof: A is blocked by C until it reaches over C at a direct v.e.
This is its first opportunity to see A.  If it misses this opportunity, it is lost
because ---.

CHANGE 'visual bitangent' to 'v.e. bitangent'

% -------------------------------------------------------------------------------------

\section{Invisibility cells}
% Counting visual events

   % Definition of umbra bounded by (1) direct and (2) indirect visual events.
   % How to define the umbra from the visual events.

% START EDITING HERE.

We wish to associate an invisibility cell with each visual event.
In all but a few cases, these visual events are easily associated with invisibility cells.
In particular, the association is simple whenever the visual bitangent intersects B
(either the direct visual bitangent of A and B, or the indirect visual bitangent of
B and C where the C-early segment intersects A).

Suppose that the visual event is stopped by B.

Consider a direct visual event, defined by a bitangent that satisfies
the conditions of Theorem~\ref{thm:directve}, say a bitangent T of A and B.
The visual event is the prefix of T's B-early segment. % Theorem~\ref{thm:directve}
Suppose that the B-early segment of T intersects B (FIGURE),
before it intersects any other visual event. % this last condition is only necessary
   % when 3 or more curves in the scene.
   %
   % implying that the visual event starts and stops at B.
   % The easiest case, when B stops the visual event, is also the most common.
Then the visual event is a chord of B from the point of tangency with B
to the first intersection with B,
and defines two chordal regions (Definition~\ref{defn:chordalregion}).\footnote{The 
  B-early segment does not intersect A by Theorem~\ref{thm:directve}.
  It does not intersect another visual event because ---.}
   % The prefix of T's B-early segment (Definition~\ref{defn:prefix}), which begins
   % at T's point of tangency with B, must stop at the intersection with B,
   % since A cannot intersect the B-early segment (Theorem~\ref{thm:directve})
   % and all visual events are blocked from this early segment by B.

\begin{defn2}
\label{defn:chordalregion}
Let B be a simple closed curve with the closed interior $\hat{B}$.
   % $\Re^2 - \hat{B}$ is a connected region.
Two distinct points $P,Q \in B$
define two curve segments $\arc{PQ} \subset B$ and $\arc{QP} \subset B$
that, each combined with the chord \seg{PQ}, 
decompose $\Re^2 - \hat{B}$ into two disjoint regions.
These two regions may be identified with the side of \seg{PQ} that they lie on
(in the immediate neighbourhood of \seg{PQ})
and will be called the {\bf inside} and {\bf outside chordal region} of \seg{PQ}.
\end{defn2}

% Figure fig:chordalregion here

The associated invisibility region is the chordal region 
that lies (in the neighbourhood of the chord) 
on the same side as A (Figure~\ref{fig:directumbrae}).
Notice that this characterization works both for the typical case of
Figure~\ref{fig:directve}a and the more unusual case of Figure~\ref{fig:directve}d
(where B surrounds A and the chordal region is flipped inside out).

Proof?

The same simple construction applies to certain indirect visual events.
Consider an indirect visual event, defined by a bitangent that satisfies
the conditions of Theorem~\ref{thm:indirectve},
say a bitangent T of B and C such that the C-early segment intersects A.
The visual event is the prefix of T's B-early segment.
Suppose that the B-early segment of T intersects B,
so that the visual event is a chord of B (FIGURE).
The associated invisibility region is the chordal region 
that lies locally on the opposite side from C, using the point of tangency with C to
gauge C's side (Figure~\ref{fig:indirectumbrae}).

We have shown that the invisibility region associated with a visual event is simple
to determine whenever the visual event is stopped by B 
(i.e., the B-early segment of the associated bitangent intersects B).
   % \footnote{Of course, their 
   %  A-early segment is allowed to intersect B, as in Figure~\ref{fig:surround}.}
How many visual events remain?
Fortunately, in the case of direct visual events, at most two.
   % How many visual events satisfy this extra condition?
   % Fortunately, in the case of direct visual events, at most two violate it.
   % All but at most two direct visual events are stopped by B,
Figure~\ref{fig:directve}b is a good example.

\clearpage

% However, before we continue, % analyze direct and indirect visual events more deeply,
% we dispense with a special case.

% NOT REALLY NECESSARY, SINCE CONDITION 2 of theorem 10 would rule out visual events
% for this case anyway (bitangent must intersect A).

\begin{defn2}
A {\bf surrounds} B if all of B lies inside the convex hull of A.
\end{defn2}

% Figure fig:surround here

\Comment{
\begin{lemma}
\label{lem:asurroundb}
Consider a scene of two curves.
If A surrounds B, every point of free space sees A.
There are no visual events for A (Figure~\ref{fig:AsurroundB}).
CLEARLY WRONG: Think of a B that has a concavity and lies inside a concavity of A.
B's concavity can easily define an invisibility cell and a visual event.
Yes, this is a visual event that intersects B.
% in a scene of 3 curves, there could be visual events for example if C encases B
% and C lies between A and B
\end{lemma}

% dynamicLocalBackUmbra -A ../data/surroundingSpiralOther.pts
\begin{figure}
\begin{center}
\includegraphics*[scale=.25]{img/twoClosedCurve/AalmostsurroundsB1.jpg}
\includegraphics*[scale=.25]{img/twoClosedCurve/AalmostsurroundsB2.jpg}
% \includegraphics*[scale=.25]{img/twoClosedCurve/AalmostsurroundsB3.jpg}
\includegraphics*[scale=.25]{img/twoClosedCurve/AsurroundsB.jpg}
\end{center}
\caption{FALSE: As A wraps around B (a-c), the visual events and umbra of A eventually disappear.
         Once A surrounds B (c), every point sees A. FALSE (see above): can have a visual
	 event in a concavity of B.
	 The first two figures show the umbra of A.}
\label{fig:AsurroundB}
\end{figure}
}

\begin{defn2}
\label{defn:encase}
B {\bf encases} A if every tangent of A intersects B in both directions.
\end{defn2}

% Figure fig:encase here

Encasing is a stronger condition than surrounding (Figure~\ref{fig:surroundEncase}):
if B encases A, then B surrounds A.
Notice that if B encases A, no point outside B's convex hull can see A.

   % \begin{lemma} NOT A USEFUL CHARACTERIZATION
   % B encases A if and only if A and B have no A-extremal bitangents.
   % \end{lemma}

\ifCommentary
Encasing another curve is stronger than surrounding it.
Correspondingly, encasing curves more stringently restrict the region that 
sees the encased curve (Lemma~\ref{lem:countOuter}).
\fi

\begin{theorem}
More detailed analysis of how many of these rare visual events (Lemma below),
comparing encasing and surrounding.
\end{theorem}

{\bf How to define invisibility regions for these 'rare' cases.}
See 'Remaining umbra' section below.

{\bf Time-series analysis (p. 38 Belle)}

{\bf Treatment of more curves.  For example, a third curve that interferes with the middle segment of the visual event's bitangent.  Need to debug the outer/inner sweep of the
Eurographics paper.}

{\bf Parallelism of the computation of visual events}

{\bf 3D analysis}

We now ask the question: 
how are the umbral regions of the non-intersecting visual events defined?


If there are two of these 'non-intersecting visual bitangents', they together define
an umbral region (closed if they intersect, open if they don't).
If there is one, it combines with an 'indirect visual bitangent' to form an umbral region
(see surroundingSpiral.pts once a self-inner bitangent exists.)
If there is none, there is no associated umbral region.

\clearpage

% umbra ../data/woak.pts
\begin{figure}[h]
\begin{center}
\ 
\end{center}
\caption{(a) A scene of two curves A and B.
         (b) The bitangents of A and B.
         (c) The self-bitangents of B.
         (d) The A-early segment of a bitangent.
         (e) The middle segment of this bitangent.
         (f) The B-early segment of this bitangent.
         (g) The B-prefix of this bitangent.
         (h) A free bitangent.}
\label{fig:bitang}
\end{figure}

% umbra -P 2 ../data/vg9New.pts (has concave A, unlike umbra1.pts)
% umbra -P 2 ../data/tsint.pts 
% umbra -P 2 ../data/dveIntersecting.pts
% umbra -P 2 ../data/surroundingSpiralEnvelope.pts
% umbra -P 2 ../data/vg10new.pts
%
\begin{figure}[h]
\begin{center}
\includegraphics*[scale=.15]{img/twoClosedCurve/directBitangVg9New.jpg}
\includegraphics*[scale=.15]{img/twoClosedCurve/directBitangTsint.jpg}
\includegraphics*[scale=.15]{img/twoClosedCurve/directBitangIntersecting.jpg}
\includegraphics*[scale=.15]{img/twoClosedCurve/directBitangSurroundingSpiral.jpg}
\includegraphics*[scale=.15]{img/twoClosedCurve/directBitangVg10new.jpg}
\end{center}
\caption{Bitangents that define direct visual events.  
         (a-c) Conventional (d-e) Surrounding [ADD A NOTATION THROUGHOUT]}
\label{fig:directve}
\end{figure}

\begin{figure}[h]
\begin{center}
\includegraphics*[scale=.15]{img/twoClosedCurve/dveVg9New.jpg}
\includegraphics*[scale=.15]{img/twoClosedCurve/dveTsint.jpg}
\includegraphics*[scale=.15]{img/twoClosedCurve/dveIntersecting.jpg}
\includegraphics*[scale=.15]{img/twoClosedCurve/dveSurroundingSpiral.jpg}
\includegraphics*[scale=.15]{img/twoClosedCurve/dveVg10new.jpg}
\end{center}
\caption{The direct visual events associated with Figure~\ref{fig:directve}.}
\label{fig:dve}
\end{figure}

% START HERE: fix software for encasing case: only one bitangent.

\begin{figure}
\begin{center}
\includegraphics*[scale=.15]{img/twoClosedCurve/umbraVg9New.jpg}
\includegraphics*[scale=.15]{img/twoClosedCurve/umbraTsint.jpg}
\includegraphics*[scale=.15]{img/twoClosedCurve/umbraIntersecting.jpg}
% \includegraphics*[scale=.15]{img/twoClosedCurve/umbraSurroundingSpiral.jpg}
\includegraphics*[scale=.15]{img/twoClosedCurve/umbraVg10new.jpg}
\end{center}
\caption{The umbrae associated with Figure~\ref{fig:directve}.}
\label{fig:directumbrae}
\end{figure}

% umbra ../data/globalumb4.pts
% umbra -P 2 ../data/surroundingSpiralIndirect.pts
% umbra -P 2 ../data/surroundingSpiralBelle51.pts
% general scenes: globalumb2; globalumb4 again; umbra1withB; vg90?; vg900new
%
\begin{figure}[h]
\begin{center}
\includegraphics*[scale=.15]{img/twoClosedCurve/indirectBitangGlobalUmb4.jpg}
\includegraphics*[scale=.15]{img/twoClosedCurve/indirectBitangSurrSpiral.jpg}
\includegraphics*[scale=.15]{img/twoClosedCurve/indirectBitangp51.jpg}
\end{center}
\caption{Bitangents that define indirect visual events. (ADD A notation)}
\label{fig:indirectve}
\end{figure}

\begin{figure}[h]
\begin{center}
\includegraphics*[scale=.15]{img/twoClosedCurve/iveGlobalUmb4.jpg}
\includegraphics*[scale=.15]{img/twoClosedCurve/iveSurrSpiral.jpg}
\includegraphics*[scale=.15]{img/twoClosedCurve/ivep51.jpg}
\end{center}
\caption{The indirect visual events associated with Figure~\ref{fig:indirectve}.}
\label{fig:ive}
\end{figure}

\begin{figure}
\begin{center}
% \includegraphics*[scale=.15]{img/twoClosedCurve/umbraGlobalUmb4.jpg}
% \includegraphics*[scale=.15]{img/twoClosedCurve/umbraSurrSpiral.jpg}
% \includegraphics*[scale=.15]{img/twoClosedCurve/umbrap51.jpg}
\end{center}
\caption{The umbrae associated with Figure~\ref{fig:indirectve}. (CREATE IMAGES)}
\label{fig:indirectumbrae}
\end{figure}

\begin{figure}[h]
\begin{center}
\ 
\end{center}
\caption{The (a) inside and (b) outside chordal region associated with the given chord
         (USE SPIRAL and a bitangent)}
\label{fig:chordalregion}
\end{figure}

% dynamicLocalBackUmbra -P 2 -a ../data/surroundingSpiral.pts 
\begin{figure}[h]
\begin{center}
\includegraphics*[scale=.25]{img/twoClosedCurve/surround.jpg}
\end{center}
\caption{A surrounds B {\bf (mark A and B)}}
\label{fig:surround}
\end{figure}

% dynamicLocalBackUmbra -P 2 -a ../data/surroundingSpiral.pts 
\begin{figure}[h]
\begin{center}
\includegraphics*[scale=.25]{img/twoClosedCurve/encase.jpg}
\end{center}
\caption{B encases A}
\label{fig:encase}
\end{figure}

\clearpage

% -------------------------------------------------------------------------------------

\section{Umbra}

\subsection{Remaining umbra}

\begin{defn2}
The {\bf inside halfspace of a visual event} involving B is the halfspace that contains B,
in the neighbourhood of the visual event's point of tangency with B.
\end{defn2}

Consider the direct and indirect visual events that 'do not intersect B'.
The additional umbra cast by B will be defined by the intersection 
of the inside halfplanes of the visual events involving B (either direct visual
events from A and B or indirect visual events from B and C)
and the visual events involving C where C interferes with B.
[See globalumb4.]

Umbra = intersection of halfplanes as defined above, plus the chord halfplane of B
from its direct bitangents?

-----

Claim: if C interferes with a direct visual event of B, this visual event is replaced
       by the indirect visual event of B and C and the direct visual event of C on
       the opposite side.
       When B does not surround A, it has two extremal direct visual events.

% ----------------------------------------------

\clearpage

% -------------------------------------------------------------------------------------

\section{Time series}

Full analysis of timeseries as B surrounds A, illustrating the theory.
0) counting of types of visual bitangents
1) empty if A surrounds B
2) defined by two direct visual events initially
3) defined by one direct and one indirect visual event next
4) defined by one direct visual event once A's tangents all intersect B twice (encased?)

Is this distinction of cases useful? Yes, as an illustration of the theory.

Animation of this timeseries.

Distinction between many B-outer and AB-outer(?) whereas at most 2 A-outer(?).

\ifCommentary
A time-series as on p. 38 of BellSouth workbook [BS],
illustrating the many cases of the back umbra cast by a single curve.

The analysis of each class of umbral region will have three parts:
1) define the components of this umbral region (the bitangents);
2) count the components (are there enough?);
3) define the umbral region, using these components.
\fi

\section{More curves}

Now consider the addition of a third curve.
How might this expand the existing umbral regions?
This gets into inner sweeps.

\section{Parallelism}

What is the complexity of computing visual events brute-force?
How much can we improve this using growth from local to global umbra?
How much can we parallelize using growth from local to global umbra (challenging to do
more parallelization than considering each visual event independently)?

Consider efficiency of point location in each cell of the decomposition, as opposed
to brute-force local computation from each viewpoint.

\section{3D}

Generalization to 3D visual events.

% -------------------------------------------------------------------------------------

\clearpage

\section{A further classification}

We can classify direct visual bitangents based on their intersection with B.
Recall that the only restriction on intersections with B is that they not occur
on the middle segment of the bitangent.
(Use Figure on p. 42 of Belle, using Graphics/MoreGraphics/Xfig or a Windows app.)

\begin{defn2}
A direct visual bitangent T of A and B is 
\begin{itemize}
\item {\bf outer} if T does not intersect B.
\item {\bf A-outer} if the A-early segment of T intersects B
      but the B-early segment does not.
\item {\bf B-outer} if the B-early segment of T intersects B
      but the A-early segment does not.
\item {\bf AB-outer} if the A-early and B-early segments of T both intersect B.
\end{itemize}
\end{defn2}



\clearpage

\subsection{Indirect visual events (OLD)}

A certain type of self-bitangent may also be a visual event for A.

\begin{defn2}
A self-bitangent T of B is {\bf self-inner} if
\begin{itemize}
\item the middle segment of T does not intersect A or B (clear line of sight)
\item one of the B-early segments intersects A;
      and if it also intersects B, it does so after it intersects A, moving away from
      the point of tangency (so a potential visual event for A)
% note: need not intersect B after A; it could instead intersect B on the other B-early segment
\item the other B-early segment does not intersect A
      % it definitely may intersect B
\item B lies on opposite sides of T in the neighbourhood of its points of tangency
      (potential visual event)
% \item one of the B-early segments of T intersects A (so a potential visual event for A)
% \item the segment from the point of tangency with B to the intersection with A
%       does not intersect B (clear line of sight)
% \item one of the B-early segments of T intersects B (?)
\end{itemize}
\end{defn2}

Self-inner bitangents define visual events.

\begin{theorem}
Let T be a self-inner bitangent of B.
Let S be the B-early segment of T that does not intersect A.
% Its non-A-intersecting B-early segment of a self-inner bitangent of B,
Some prefix of S defines a visual event for the invisibility of A.
In particular, the visual event extends along the B-early segment
from its point of tangency with B until its first intersection with B (if any).
\end{theorem}
\prf
This is the second type of visual event described in the proof of 
Theorem~\ref{thm:directve},
arising from A appearing through a hole in B.
Consequently, it arises from a self-bitangent of B, with B on opposite sides
at the points of tangency.
Imagine looking through the hole defined by B.
To make a visual event, A must appear at this bitangent on the other side of the hole,
so a B-early segment S of the bitangent must intersect A.
Of course, A must not already be visible before the hole, so
the other B-early segment must not intersect A (looking in the other direction).
Finally, the middle segment of the bitangent must be free of intersections with B
(otherwise we cannot see through the hole) and free of intersections with A
(otherwise A is already visible).
For the same reason, the B-early segment must not intersect B before it intersects A
(otherwise we cannot see A through the hole).
\QED

[Arbitrarily many of these too, akin to AB-outer and B-outer bitangents;
self-inner should be motivated as cousins of the latter.]

\clearpage

\begin{lemma}
Bound on number of outer visual bitangents.
\end{lemma}

We now classify the outer visual bitangents based on their intersection with B.
(Use Figure on p. 42 of Belle, using Graphics/MoreGraphics/Xfig or a Windows app.)

\begin{defn2}
An outer visual bitangent T of A and B is 
\begin{itemize}
\item {\bf outer} if T does not intersect B.
\item {\bf A-outer} if the A-early segment of T intersects B
      but the B-early segment does not.
\item {\bf B-outer} if the B-early segment of T intersects B
      but the A-early segment does not.
\item {\bf AB-outer} if the A-early and B-early segments of T both intersect B.
\end{itemize}
\end{defn2}

{\bf Examples.}
(1) Various phases of B-surrounding-A timeseries.
(2) Then repeat of this time-series example, with classification into outer, A-outer, etc.
(3) tsint example of B-outer bitangents (p. 46 of Belle).

\begin{lemma}
We are interested in the inside of outer and A-outer bitangents.
We are interested in the inside of B-outer and AB-outer bitangents until their
first intersection with B.
There is a limit on the number of outer and A-outer bitangents.
There is no limit on the number of B-outer and AB-outer bitangents.
How to construct umbra for outer and A-outer bitangents.
Each B-outer and AB-outer bitangent defines a component of the umbra(?).
\end{lemma}

The sequence of outer visual bitangents as B surrounds A and A surrounds B.

\clearpage

\subsection{Dynamic analysis}

As B wraps around A (time-series), 
the umbra cast by B grows and, more interestingly, the bitangents that bound this umbra
qualitatively change:
from two bitangents that define a halfspace containing A and B (2 outer bitangents),
to two bitangents that relax this condition to allow some intersections with B 
(2 A-outer bitangents), 
to one of these A-outer bitangents along with a bitangent that separates B (one A-outer and one self-inner),
to a single bitangent that pierces B (one piercing bitangent).
We now strive to rigorously define these bitangent types and to 
characterize the transitions between these classes,
thereby characterizing the umbra (full invisibility) of A.

{\bf We need to intuitively introduce the families of umbra arising as B wraps around A,
using descriptions and images, BEFORE the hardcore theory is presented, otherwise
the reader will get lost and bored with the detail of the definitions and the lemmas.}

The umbra will be bound by one or two bitangents throughout all of the phases.
Call a bitangent --- if A and B lie on the same side of the bitangent in the neighbourhood
of their points of tangency.
Throughout most of the phases (all but the strange transitional phase 3), 
the bitangents that define the umbra will not intersect A.
However, as B wraps around A, their relationship with B changes:
no intersections with B in phase 1,
only intersections with B {\em before} A in phase 2, and
only intersections with B {\em before} A and {\em before} B in phase 4.
There is also a transitional phase between phases 2 and 3 where the bitangent
is a self-bitangent of B, only intersects B before A, but intersects A.
%
During this transitional phase, the self-inner bitangent transitions from the A-outer
to the A-piercing bitangent as the self-inner bitangent rolls over A [moving in B's
tangent space], pushed down by B, until it reaches the other side of A and
coalesces with the other A-outer bitangent [and now continues moving in A's tangent 
space].
%
We can distinguish between these different phases as follows.
When B lies in the convex hull of A, there is no umbra [phase 0].
As A and B separate, so that neither is contained in the convex hull of the other,
we enter phase 1.
As A enters the convex hull of B, we enter phase 2.
As the number of A-outer bitangents drops from 2 to 1, we enter phase 3.
As the number of A-outer bitangents drops from 1 to none, or equivalently
when every tangent of A's tangent space has intersections with B both before and after A,
we enter the final phase 4.

\clearpage

\begin{defn2}
B {\bf surrounds} A if all of A lies inside the convex hull of B.
\end{defn2}

\begin{lemma}
Suppose that B surrounds A. 
B can have many self-inner bitangents, each of which defines a region of invisibility
(from the B-early segment that doesn't intersect A until the first intersection with B,
if any).
Figure of multiple invisibility regions generated by self-inner bitangents, 
on p. 45 of Belle.
\end{lemma}

START HERE
Consider the umbra cast by B as B progressively wraps around A.
If neither B nor A surrounds the other, we will show that they have two outer bitangents.
SHOW THIS BELOW.
As B wraps around A, it eventually surrounds A, at exactly the moment when the two
outer bitangents coalesce into one outer tritangent.
After this point, A and B only have A-outer bitangents, starting with two.
As B wraps further around A, it eventually crosses one of these A-outer bitangents.
This removes the A-outer bitangent that was crossed, 
since it now violates the second condition of a visual bitangent.
However, at exactly the same moment that one of the outer bitangents disappears, 
a self-inner bitangent is created that becomes a visual event,
replacing the visual event at the outer bitangent.
SHOW THAT OUTER BITANGENT IS A VISUAL EVENT?
The self-inner bitangent remains a visual event until it rotates past A.
At precisely the moment that the self-inner bitangent disappears, 
the A-outer bitangent becomes an AB-outer bitangent.

As the curve B surrounds A, the visual events transform from outer bitangents, to
A-outer bitangents, to 

The different stages are defined as follows.
Neither surrounding: 2 outer bitangents; perhaps some B-outer bitangents
B surrounds A, and 2 A-outer bitangents.
B surrounds A, one A-outer bitangent and one self-inner bitangent.
B surrounds A, one AB-outer bitangent.
That is, self-inner is a visual event while it intersects A.
All outer visual bitangents mark visual events too.
There are two of them in the first two phases and one of them in the last two phases.
The switch from two to one outer visual bitangent is marked by a tritangent to
B, B and A (i.e., when an inner bitangent of B coalesces with an outer visual bitangent
of A and B).

{\bf Note that the umbra is always bounded by B-early segments.}

Conversely, as A wraps around B, the two outer bitangents disappear exactly 
when A surrounds B (when B completely enters the convex hull of A).
When A surrounds B, B does not cast any umbra.

Throughout this section, $P_A \in A$ and $P_B \in B$ refer to the points of 
bitangency on a bitangent of A and B.
If there is more than one point of tangency with A or B,
$P_A$ and $P_B$ should be chosen to make \seg{P_A P_B} as long as possible.

\clearpage

\subsection{No outer}

\begin{lemma}
If A surrounds B, the back umbra cast by B is empty ({\bf develop an example}).
\end{lemma}

\clearpage

\subsection{Two outer; neither surrounds the other}

\begin{defn2}
\label{defn:extremal}
A bitangent of A and B is {\bf extremal} if it does not intersect A or 
B.\ifCommentary\footnote{Clearly, points of tangency are not counted as intersections.}\fi
\end{defn2}

\begin{defn2}
\label{defn:outerbitang}
An extremal bitangent T of A and B is {\bf outer} if A and B lie on the same side of T,
and {\bf inner} if A and B lie on opposite sides of T.
\end{defn2}

\ifCommentary
% outer to supporting? not as suggestive a term, so no.
% inner to separating? 
In the literature, inner and outer bitangents may be called
separating and supporting bitangents, respectively.
We find the terms 'inner' and 'outer' more suggestive for the work of this paper,
especially later when these bitangents are incorporated into sweeps.
Note that each outer bitangent of A and B defines a halfspace containing A and B.

{\em Figure of bitangents, extremal bitangents, outer bitangents and inner bitangents 
in the typical case (perhaps Figure~\ref{fig:outer} and~\ref{fig:innersweep}b-c).}
\fi

\vspace{.3in}

\begin{lemma}
\label{lem:twoOuter}
\ifCommentary
Let A and B be closed curves whose interiors have no intersection.
\fi
Suppose that neither A nor B surrounds the other.
In (2-3), suppose that A is the distinguished object.
\begin{enumerate}
\item A and B have exactly two outer bitangents.
\item The B-early segments of the two outer bitangents of A and B mark visual events 
      for A.
      A appears/disappears as these segments are crossed.
\item
\label{lem:localbackumbra}
      The B-early segments of the two outer bitangents of A and B, 
      and the relevant curve segment of B between them,
      bound the back umbra cast by B.
      The back umbra is the region that lies inside both outer bitangents
      \ifCommentary(Figure~\ref{fig:backumbra}).\else.\fi
\end{enumerate}
\end{lemma}
\prf
Proof of (1).
Two convex closed nonintersecting regions will have four bitangents,
two outer and two inner.
As the convex regions intersect, the two inner bitangents are lost. % used in Lemma~\ref{lem:twoInner} below
The inner/outer bitangents of two curves are equivalent to the bitangents of the
convex hull of A and the convex hull of B.
% That is, compute the convex hull of A and the convex hull of B, then take
% bitangents: these are the inner and outer bitangents of A and B.
\ifCommentary (Figure~\ref{fig:outer}a).\fi

Proof of (3).
Clear.
Note that A is completely invisible from a point P iff the visibility cone (bounded
by the outer bitangents from P to A) is completely blocked.
% It is well known that bitangents define visual events \cite{}.
% Consider a point that lies between the late segments of the two outer bitangents.
% It cannot see L because ---.
\QED

\ifCommentary
See umbra -P 0 ../data/outPiercingSweep.pts.
\fi

\vspace{.3in}

\ifCommentary
	% umbra data/umbra/umbra1.pts > foobar &
\begin{figure}[h]
\begin{center}
\includegraphics*[scale=.15]{img2/umbra1back.jpg}
\end{center}
\caption{The back umbra of A.}
\label{fig:backumbra}
\end{figure}
\fi

\clearpage

\subsection{B surrounds A}
\label{sec:BsurroundA}

\ifCommentary
If B surrounds A, the definition of outer bitangent must be generalized,
since every bitangent will necessarily intersect B.
% If a curve surrounds another curve, the definition of outer bitangent must be generalized,
% since every bitangent will necessarily intersect the surrounding curve.
The appropriate generalization is to allow intersections on the 'other' side of A.
% the surrounded object.
\fi

\begin{defn2}
A bitangent of A and B is {\bf A-extremal} if 
it does not intersect A at all and
it does not intersect B except possibly on the A-early segment.
That is, intersections with B are allowed behind A, closer to $P_A$ than to $P_B$.
\end{defn2}

% We first restrict to bitangents that might be involved in a visual event
% arising from the appearance/disappearance of A.

% \begin{defn2}
% A bitangent of A and B is {\bf A-visual} if the entire line \lyne{P_A P_B} does not
% intersect A and the line segment \seg{P_A P_B} does not intersect A or B,
% % where $P_A$ and $P_B$ are the points of bitangency with A and B.\footnote{If there is
% %   more than one point of tangency with A (resp., B), 
% %   choose the furthest from B (resp., A).}
% \end{defn2}

% An extremal bitangent (Definition~\ref{defn:extremal}) is an A-visual bitangent that 
% does not intersect B.
% A-extremal bitangents relax this restriction on intersections.

% \begin{defn2}
% An A-visual bitangent is {\bf A-extremal} if \ray{P_A P_B} does not intersect B.
% That is, intersections with B are allowed behind A (closer to $P_A$ than $P_B$).
% \end{defn2}

% \begin{defn2}
% An A-visual bitangent is {\bf B-piercing} if \ray{P_A P_B} does intersect B.
% \end{defn2}

\begin{defn2}
An A-extremal bitangent T is {\bf A-outer} if A and B lie on the same side of T 
in the neighbourhood of $P_B$,
and {\bf A-inner} if A and B lie on opposite sides of T in the neighbourhood of $P_B$.
% The generalization of extremal bitangents to A-extremal bitangents is necessary
% to work with objects that surround other objects.
\end{defn2}

\ifCommentary
{\em Figure of surrounding object, A-extremal bitangents, A-outer bitangents and A-inner bitangents (perhaps Figure~\ref{fig:backsurrounding}b and~\ref{fig:G2}c may be relevant).}

The complementary generalization of the extremal bitangent, where intersections occur
closer to $P_B$ than to $P_A$, is also important.
They play a role in the definition of both the front and back umbra.
\fi

% \vspace{.3in}

\begin{lemma}
\label{lem:Aouter}
\ifCommentary
Let A and B be closed curves whose interiors have no intersection.
\fi
Suppose that B surrounds A.
\begin{enumerate}
\item A and B have at most two A-outer bitangents.
% \item If B surrounds but does not encase A, 
%      then A and B have two A-outer bitangents. % we mean 'exactly two'
% \item If A surrounds but does not encase B, 
%      then A and B have two B-outer bitangents.
\end{enumerate}
\end{lemma}
\prf
Suppose that B surrounds A: that is, A lies inside the convex hull of B.
Since A does not intersect B, it must lie in a concavity of B.
Let T be the bitangent of B that caps this concavity and belongs to the convex hull of B, 
with points of bitangency $P_B$ and $P_{B'}$.
If we sweep in the tangent space of B,
starting at $P_B$ and moving into the concavity, until the tangent touches A,
we shall find a bitangent T of A and B.
T may be an A-outer bitangent: 
both A and B lie on the same side of T at their points of bitangency,
and T does not intersect A, since it is the limit of a sweep for which both of these 
properties are always true; and
since A lies in a concavity, T will certainly intersect B after it touches A
(i.e., on the A-early segment).
The deciding factor is the status of the middle and B-early segments of T,
which may or may not intersect B (Figure of time series).
By a similar argument, an analogous sweep in the tangent space of B, starting at $P_{B'}$ 
rather than $P_B$ and again moving into the concavity, may generate a second
A-outer bitangent.

We now argue that there are no other A-outer bitangents.
Consider an A-outer bitangent of A and B.
% Its point of tangency with B cannot lie inside the concavity housing A,
% YES IT CAN.
% If its point of tangency with B lies outside the concavity that contains A, ...
% Therefore, the point of tangency with B must lie at the entry to the concavity.
% We have found the only two A-outer bitangents that satisfy this constraint.

THIS SEARCH TECHNIQUE MAY FAIL: SEE [BELLE P. 40].
BUT WE CAN PROVE AN UPPERBOUND BASED ON THE CONCAVITY ALONE, WITHOUT SWEEPING.

Suppose that B surrounds A: that is, A lies inside the convex hull of B.
Since A does not intersect B, it must lie in a concavity of B.
Let T be an A-outer bitangent, tangent to B at $P_B$.
Since A lies in a concavity of B, T will certainly intersect B after it touches A
(on the A-early segment): let $K \in B$ be the first such intersection.
Consider the curve segment \arc{P_B K}.
Figure on bottom of [p. 40, BS].
Since this curve segment surrounds A, any other tangent of A will intersect it, and thus B.
In particular, any other A-outer bitangent will intersect B.
But A-outer bitangents can only intersect B on their A-early segments.
Thus, the A-early segment of other A-outer bitangents must point into T's inside halfspace,
implying that the B-early segment points into T's outside halfspace.
Therefore, the point of tangency on B of any other A-outer bitangent must lie
in T's outside halfspace.
Suppose that there is a second A-outer bitangent U.
By the same argument, the point of tangency on B of a third A-outer bitangent
must lie in T's outside halfspace and U's outside halfspace (*).

Now note that their B-early segments do not intersect B.
% these A-outer bitangents must leave through the concavity's hole,
Think of the implication of this: the intersection of T's outside halfspace 
and U's outside halfspace is a region partially bounded by these B-early segments.
Since B is connected and cannot cross into this region, no part of B lies in this region.
By (*), there can be no third A-outer bitangent.
\QED

Notice that, if neither A nor B surrounds the other (the case of the previous section),
there are exactly two A-outer bitangents, both of which are also outer bitangents.

\clearpage

\subsection{Two A-outer}

\begin{lemma}
Suppose that B surrounds A, and they have exactly two A-outer bitangents.
\begin{enumerate}
\item % When there are exactly two A-outer bitangents,
      The B-early segments of these bitangents mark visual events for A.
      As these segments are crossed, A appears/disappears.
\item 
\label{lem:localbackumbrasurrounding}
      % When there are exactly two A-outer bitangents,
      The B-early segments of these bitangents,
      and the relevant curve segment of B between them,
      bound the back umbra cast by B.
      % the back umbra cast by B is bounded by the late segments
      % of the two A-outer bitangents of A and B and the relevant curve segment of B 
      % between them.
      The back umbra is the region that lies inside {\em either} bitangent.
      \ifCommentary (yielding a much larger umbra than Lemma~\ref{lem:localbackumbra}) \fi
      \ifCommentary(Figure~\ref{fig:backsurrounding}c).\fi
      % The effect is a much larger umbra.
\end{enumerate}
\end{lemma}

\ifCommentary
	% umbra -w data/umbra/vg10.pts > foobar &
\begin{figure}[h]
\begin{center}
\includegraphics*[scale=.15]{img2/vg10.jpg}
\includegraphics*[scale=.15]{img2/vg10outer.jpg}
\includegraphics*[scale=.15]{img2/vg10back.jpg}
\end{center}
\caption{Defining the back umbra of A when A surrounds L. (a) A surrounds L. (b) The two L-outer bitangents of A and L. (c) The back umbra of A.}
\label{fig:backsurrounding}
\end{figure}
\fi

\clearpage

\subsection{One A-outer}

\begin{defn2}
A self-bitangent T of B is {\bf extremal} if 
\begin{itemize}
\item the middle segment of T does not intersect A or B
\item one or both of the early segments of T intersects B [not necessary if we restrict
      our attention to inner, since this is inherent to inner selfbitangents]
\item an early segment of T intersects A before it intersects B
      [otherwise, why would this self-bitangent be interesting as a visual event for A?]
\end{itemize}
An extremal self-bitangent T of B is {\bf outer} if B lies on the same side of T
in the neighbourhood of its points of tangency and {\bf inner} if 
B lies on opposite sides of T in the neighbourhood of its points of tangency.
\end{defn2}

\vspace{.5in}

\begin{lemma}
Suppose that B surrounds A and they have exactly one A-outer bitangent $T_1$.
\begin{enumerate}
\item B has exactly one inner self-bitangent $T_2$ 
  % this self-bitangent enters the concavity of B that contains A
\item The B-early segments of $T_1$ and $T_2$ mark visual events for A.
      As these segments are crossed, A appears/disappears.
\item The B-early segments of $T_1$ and $T_2$, and the relevant curve segment of B,
      bound the back umbra cast by B.
      The umbra is the region inside either bitangent.
\end{enumerate}
\end{lemma}
\prf
\QED

\clearpage

\subsection{No A-outer}
\label{sec:noaouter}

START HERE

\begin{defn2}
B {\bf encases} A if every tangent of A intersects B in both directions.
\end{defn2}

\begin{lemma}
B encases A if and only if A and B have no A-extremal bitangents.
\end{lemma}

\ifCommentary
{\em Figure of encasing curve, from bottom of p. 23 of Bellsouth workbook}
\fi

\ifCommentary
Encasing another curve is stronger than surrounding it.
Correspondingly, encasing curves more stringently restrict the region that 
sees the encased curve (Lemma~\ref{lem:countOuter}).
\fi

\begin{defn2}
Classically, closed curves are oriented so that the inside is always to the left.
Call this direction around the curve the {\bf positive direction}.
% Given a point P on a closed curve B, the {\bf left-handed direction} from P
% is the direction from P that would be travelled to keep B on the left.
% An A-foo bitangent is {\bf A-consistent} if the positive direction 
% from the first piercing point on B is consistent with the inside of A.
\end{defn2}

\begin{defn2}
\label{defn:piercing1}
A bitangent T of A and B is {\bf A-foo} if 
it does not intersect A at all,
its middle segment does not intersect B but its A-early and B-early segments do,
$P_B$ is not an inflection point, 
and the positive direction 
from the first intersection of the B-early segment with B is 
consistent with the inside of A.
\end{defn2}

\ifCommentary
{\em Figure of A-consistent bitangent and not A-consistent bitangent, from bottom of p. 23 of Bellsouth workbook}
\fi

\begin{defn2}
An A-foo bitangent T of A and B is {\bf outer}
if A and B lie on the same side of T in the neighbourhood of $P_B$, and {\bf inner} 
if A and B lie on opposite sides of T in the neighbourhood of $P_B$.
COLLAPSE ALL INNER/OUTER DEFINITIONS IN A GENERIC DEFINITION.
\end{defn2}

\ifCommentary
{\em Figure of two curves and their A-foo bitangents, perhaps from 
Figure~\ref{fig:piercing}).}
\fi

\begin{defn2}
 % For piercing bitangents, there is a third bitangential segment.
 % used to be called intermediate, but we don't want to confuse it with middle segment,
 % which now replaces our earlier term 'early segment'
The segment of an A-foo bitangent between its point of tangency with B and 
the B-early segment's first intersection with B is called
the {\bf bridge segment} (Figure~\ref{fig:piercing}c).
\end{defn2}

% Bitangents that contain intersections between their points of tangency
% will never contribute to visual events.
% Since we are interested in visual events that arise from the appearance/disappearance
% of A (we are analyzing the visibility of the distinguished object A),
% we are only interested in bitangents that do not intersect A.
% However, bitangents with intersections with B outside this interior segment remain of interest.
% These bitangents can be organized by the position of their intersections.

% The asymmetry between A and L in conditions 2-6 is resolvable by the distinguished
% nature of L.
% Condition (1) makes T extremal with respect to L so that it can be a visibility boundary,
% while (2) makes T piercing.
% \ifSurrounding
% Due to the care taken with conditions 2 and 4, 
% no adjustment needs to be made for surrounding objects.
% \fi

\vspace{.3in}

\begin{lemma}
Suppose that B surrounds A and they have no A-outer bitangents.
\begin{enumerate}
\item A and B have exactly one outer A-foo bitangent(?).
\item The back umbra is bounded by the bridge segment of the outer A-foo bitangent and 
      the relevant curve segment of B.
      It is the region locally inside the bitangent.
\end{enumerate}
\end{lemma}

\vspace{.3in}

\clearpage

It is important to establish the number of outer bitangents, 
which are fundamental to visibility.
% We end this section by analyzing the number of extremal bitangents between
% two objects.

\begin{lemma}
\label{lem:countOuter}
Let A and B be closed curves whose interiors have no intersection.
\begin{itemize}
\item If B encases A, 
      then A and B have one A-consistent bitangent(?).
\item If A encases B, 
      then A and B have one B-consistent bitangent(?).
\item If B surrounds but does not encase A, 
      then A and B have two A-outer bitangents. % we mean 'exactly two'
\item If A surrounds but does not encase B, 
      then A and B have two B-outer bitangents.
\item If neither A nor B surrounds the other, then $A$ and $B$ have two outer bitangents.
\end{itemize}
In short, A and B have exactly two bitangents that are A-outer, B-outer or outer
when they are not encasing, and one outer piercing bitangent of the consistent variety otherwise.
\end{lemma}

SHOW HOW BACK UMBRA ARISES DIRECTLY FROM THESE OUTER, A-OUTER OR CONSISTENT BITANGENTS.

IT MAY BE BETTER TO GO WITH THE LAYERED APPROACH, COMPLETELY 
INTRODUCING EACH CLASS OF BITANGENT THROUGH COUNTING PROOF AND USE IN BACK UMBRA 
BEFORE PRESENTING NEXT CLASS.
THIS WILL MOTIVATE THE SPECIAL CASES BETTER.

\clearpage

\begin{lemma}
Let A and B be closed curves.
\begin{itemize}
\item If their convex hulls do not intersect,
A and B have two inner bitangents. % again, we mean exactly two
\item If their convex hulls intersect,
A and B have two bitangents that are either A-inner or B-inner.
\end{itemize}
In short, A and B have exactly two bitangents that are A-inner, B-inner, or inner.
\end{lemma}
\prf
See umbra -P 0 ../data/outPiercingSweep.pts.
\QED
%-----------------------------------

\clearpage

\ifCommentary
	% umbra data/umbra/umbra1.pts > foobar &
\begin{figure}
\begin{center}
\includegraphics*[scale=.15]{img2/umbra1outer.jpg}
\includegraphics*[scale=.15]{img2/umbra1late.jpg}
\end{center}
\caption{(a) The two outer bitangents of A and L. (b) The late segment of a bitangent.}
\label{fig:outer}
\end{figure}

	% umbra data/umbra/umbra1.pts > foobar &
\begin{figure}
\begin{center}
\includegraphics*[scale=.15]{img2/umbra1piercing.jpg}
\includegraphics*[scale=.15]{img2/umbra1outerpiercing.jpg}
\includegraphics*[scale=.15]{img2/umbra1interm.jpg}
\end{center}
\caption{(a) The piercing bitangents of A and L.
         (b) The outer piercing bitangents. 
         (c) The bridge segment of a piercing bitangent.}
\label{fig:piercing}
\end{figure}
\fi

\section{The front umbra}

Define piercing bitangents.


% ----------------------------------------------------------------------------------------

\end{document}

\begin{figure}[h]
\begin{center}
\includegraphics*[scale=.15]{img/twoClosedCurve/---.jpg}
\end{center}
\caption{}
\label{fig:}
\end{figure}

\clearpage

Structure of figures:

front and back umbra: umbra1
global back umbra:  cornell scene and globalumb3
global front umbra: cornell scene

Visibility analysis = set of objects that are visible from a given point.
Umbra = locus of points that do not see any of the light.

visibility analysis = analysis of which objects in the smooth scene are seen (at all) and unseen (completely)
A-umbra = locus of points in the smooth scene that do not see (any of) A [A is completely unseen; A is hiding; where the pipeline can avoid rendering A.]
    Responsibility-centered visibility [caters to divide-and-conquer parallelization;
      possibly to dynamic scenes?]
    Locus of points in scene that are blocked from A by B alone.
    Locus of points in scene that are blocked from A by B and some other objects.
       Locus of pts that lie behind B and are blocked from A by B and other objects.
       Locus of pts that lie in front of B and are blocked from A by B and other objects.
A-penumbra = locus of points in the smooth scene that do not see all of A
    This is not an important distinction for occlusion culling: if any of an object is
    seen, it will need to be rendered.
    But, it is an important distinction for exact shadow computation.

dynamic computation through responsibility-centered computing:
   If you know that a point is blocked from A due to $\{B_i\}$, then if one of $B_i$ is moved
   or removed, we know that we have to recompute the visibility of this point,
   otherwise not.
       Every object points to regions that it has affected [hopefully, 
	 this is a significant minority of the whole scene].
       When an object is moved, the regions that it has affected are recomputed.

The visibility analysis can be done through computation of the A-umbra of all objects A
in the scene.  These A-umbrae may be merged into a global map with regions labeled
with the visible objects, or an individual point may be queried about which objects it sees: it sees A iff it does not lie in the A-umbra (a point location question).

\clearpage
% \fi

\begin{abstract}
Proofs (for SoCG).\\
Sweeping in tangent space of surface. Are there other 2d sweeps?

In discontinuity mesh work (Fiume/Stewart),
given a viewpoint, the umbra and penumbra of the polyhedral scene is determined.
In aspect graph work,
the polyhedral scene is decomposed into homogeneous visibility regions: if the viewpoint
is placed in a cell, the same objects are visible throughout the cell (a stricter 
equivalence is actually defined).
We are doing the same, but keep track of information differently.
The visibility of each object is stored separately.
This is equally powerful to the combined data structure and has the advantage
that if the visibility of only one object is required (e.g., the light if computing shadows), less computation is required.
The scene is decomposed into homogeneous visibility regions with respect to a 
single object: if the viewpoint is placed in a cell, the viewpoint maintains the 
same visibility (either total or partial) of the distinguished object throughout the cell.

We analyze the visibility of an object in a smooth two-dimensional scene.
The novel issue is a practical algorithm for the generalization of visibility analysis 
from polygonal environments to smooth environments: 
each object is defined by arbitrary curves.
We also compute the global structure, akin to aspect graphs.
2-space is divided into regions from which the distinguished object is completely visible,
partially visible and completely invisible.
(Partial visibility is new.)
No restrictions on the relationship of the distinguished object to the surrounding scene 
are made.
Proofs of the claims are made.
The work is integrated into the existing visibility literature more fully.
(More elaborate examples are included.)
(2.5D work is expanded?)
(We give a fuller treatment of incremental analysis?)
(A full software release is provided?)

{\bf How does this relate to Cohen-Or's hierarchy of occlusion culling, 
and to shadow computation?}

Consider a distinguished object L in a scene of smooth objects in 2-space,
where each object is bounded by a closed curve.
This paper answers the following question: where is L invisible?
% We analyze the invisibility of a smooth object L within a smooth 2d scene,
% in which each object is bounded by a closed curve.
\ifJournal
  Space is decomposed into regions that see all of L, regions that see part of L,
  and regions that see none of L.
\else
Space is decomposed into regions that see none of L and regions that see part or all of L,
\fi
yielding the global structure of L's invisibility.
An analysis of object invisibility is useful in occlusion culling, shadow computation, 
and motion planning.

Visibility analysis among smooth objects has received little attention, 
as most papers on visibility analysis have concentrated upon polygonal and polyhedral 
scenes.
The algorithmic construction in flatland of this paper 
builds a foundation for practical smooth visibility analysis.
% If L is interpreted as a smooth area light source, it computes the umbra cast by L.
% It may be used to plan motions through the scene that remain completely hidden
% from L, since the global structure of invisibility is known.
\ifJournal
If L is an object of focus, this decomposition can be used to plan movements
% such as camera movements (mention in full paper)
through the scene that maintain complete visibility of the object, since the
global structure of visibility is known.
If L is an object to be avoided, such as an enemy or radiation source,
this decomposition may be used to plan movements through the scene that 
remain completely hidden from L, since the global structure of invisibility is known.
\fi

% {\bf Avoid the use of 'bitangent' in the abstract.}
The computation of object visibility depends on the structured filtering of 
potential visual events down to true visual events.
A taxonomy of bitangents that contribute to visual events is developed.
The regions that do not see L are found by sweeping a tangent about
certain objects in the scene.
A divide and conquer approach, conducive to parallelization,
is used to compute the effect of every object
on the distinguished object's visibility.

\ifJournal
WE ARE IGNORING PENUMBRA IN CONFERENCE PAPER (CONCENTRATING ON INVISIBILITY RATHER
THAN VISIBILITY)

IGNORE SURROUNDING CASE TOO IN CONFERENCE PAPER.

IGNORE TOUCHING CASE TOO.
(1) When obstacles touch,
global umbrae of obstacles do not cover entire umbra of scene.
However, the infinitesimally close scene with the objects not touching is correct.
Moreover, the penumbrae may cover the requisite global umbra.
(It certainly does for touching circles.)
(2) Touching curves don't have inner bitangents.
These two facts (1) and (2) may be related: 
if the inner bitangents disappear, we may use a different definition of local umbra.
See circleUmbraProblem (touching) and circleUmbraProblem2 (barely not touching)
for insights.
\fi

\end{abstract}

% Contributions:  This paper contributes to Eurographics' 
% focus area of visibility and shadows,
% by building a foundation for practical visibility analysis in curved environments.

% Benefits:  
% The analysis of a smooth scene broadens the scope of visibility analysis,
% offering the potential of improved accuracy and the ability to deal directly
% with a smooth scene rather than discretizing it.
% A parallelizable method for filtering potential visual events down to actual visual events
% is presented.

% {\bf Old abstract:}
% The umbra cast by a smooth area light source in a 2d scene of smooth
% objects is studied.
% A different approach is taken from previous work in lighting.
% First, the umbra is decomposed into (possibly overlapping) subumbrae cast
% by each object, and these regions are fully defined geometrically.
% Second, the area light source and scene of objects are both smooth,
% bounded by freeform parametric curves.
% Since the geometric analysis of smooth environments is challenging,
% the analysis is restricted to 2d and 2$\frac{1}{2}$d environments.


% full definition of the region, not point-by-point query

% \newpage

% ----------------------------------------------------------------------------------------

\clearpage

\section{Introduction}
\label{sec:intro}

In this paper, we analyze the invisibility of a distinguished object L 
in a scene of smooth objects.
In particular, we compute the region of free space 
from which the distinguished object L is not visible.
\ifJournal
  Space is decomposed into regions that see all of L, regions that see part of L,
  and regions that see none of L.
\else
  % Space is decomposed into regions that see none of L and regions that see part 
  % or all of L.
\fi
% --------------------------------------
% APPLICATIONS
This type of visibility analysis of a single object has many applications.
% (1)
It reveals camera locations from which L need not be rendered (occlusion culling).
% (2)
If L is interpreted as a smooth area light, it computes the umbra
\ifJournal
and penumbra
\fi
cast by L (shadow computation).
% If the identified object is interpreted as a light, the analysis reveals
% the umbra and penumbra cast by this light.
% (2b): not immediate enough for conference; better to implement first too
\ifJournal
If L is interpreted as an area light and 
the umbra is computed with respect to the permanent objects in the scene
(e.g., furniture or buildings),
other objects that move across the scene (e.g., people or cars) will move into
and out of the precomputed umbra, indicating how they should be lit.
  % This is another good application to illustrate the method since it differs from 
  % traditional umbral computation.
\fi
% (3)
  % It reveals locations from which the distinguished object is completely shielded.
If L is interpreted as an undesirable object, 
\ifJournal
such as sound crew's equipment,
\fi
the analysis reveals camera locations that avoid seeing the undesirable object
(camera placement).
  % sound technician's boom
\ifJournal
  or safe locations that are blocked from a radiation source.
\fi
  % where to stand to avoid being seen
% (4)
\ifJournal
  If L is an object of focus, this decomposition may be used to plan movements
  % such as camera movements (mention in full paper)
  through the scene that maintain complete visibility of the object, since the
  global structure of visibility is known.
\else
  % If L is an object to be avoided,
  % such as an enemy or radiation source,
Moreover, the decomposition may be used to plan entire motions through the scene that 
remain completely hidden from L, since the global structure of invisibility is known.
\fi

\ifComment
  Good because it uses complete visibility, which is more subtle than partial visibility,
  and it doesn't compete with the speed of rendering as does 1.
  Planning a path as in the next application is a good generalization too.
  Probably better than the umbral application since it emphasizes the general
  visibility interpretation rather than the object as light specialization.}
\fi

\ifJournal
[Only when penumbra is included]
% where to stand to be seen
It reveals locations from which the distinguished object is completely accessible, 
with no intervening object blocking the view,
such as seating locations that see the entire stage,
camera locations that see the face of an actor,
painter nozzle locations that see the object to be painted,
or X-ray locations that see the tumor.
\fi

% There are two interpretations of the problem, matching the two problem definitions.
% In one interpretation, the camera moves around the free part of the scene,
% and it knows what amount of the specified object it sees.
% In another interpretation, the camera is the object (akin to stereo eyes),
% and it is known exactly which points in the scene are completely or partially
% visible to the camera.
% The first interpretation matches the object visibility problem,
% while the second interpretation matches the shadow problem.

% --------------------------------------
% STUDY THE SMOOTH PROBLEM

The vast majority of previous work in visibility analysis
has been carried out in scenes of polygon meshes.
There has been some work on smooth scenes but this has not been especially algorithmic.
The purpose of this paper is to develop practical methods for the analysis
of smooth nonlinear scenes.
Since visibility analysis is expensive when applied to large scenes,
we are sensitive to the construction of efficient algorithms that are inherently
parallelizable.
The consideration of smooth objects broadens the scope of the analysis and
opens the way for new strategies.
The visibility analysis of a smooth scene offers the potential of improved accuracy.
It gives the ability to deal directly with a smooth scene rather than discretizing it
into a mesh, which is more natural, more accurate and more efficient for scenes that
are defined initially by smooth primitives such as NURBS.
Smooth scenes have more complex primitives but less of them, which may
actually improve the efficiency of visibility analysis.
% Although the complexity of each primitive operation is increased, the
% number of operations is reduced.

% --------------------------------------
% STUDY THE 2D PROBLEM

This paper studies smooth object invisibility in the simplified context of flatland,
which has notable precedents (e.g., \cite{heckbert92a}).
We study the invisibility of a distinguished object L in a scene of smooth objects in
2-space, where each object is bounded by a closed curve.
This allows the development of a practical foundation for smooth visibility
and facilitates the understanding of key issues, such as locality of influence and
the influence of one object on another, in a clearer context.
Recent visibility analysis also shows the
application of 2d analysis to 3d analysis \cite{leyvand03}.
%
% For prototype purposes, this problem is posed in 2D.
% each object is defined by a closed freeform curve.
%
A scene in flatland models a three-dimensional scene where
the objects are infinite in one dimension,
which realistically models some natural applications (such as a cityscape of buildings).
It yields a solution from the perspective of a person or camera at groundlevel
(Figure~\ref{fig:cityscape}).
% extruded footprints
% Imagine a person walking amongst tall extruded skyscrapers, looking straight forward:
% a solution of the object visibility problem realistically models
% the visibility of this scene from the person's perspective.

In developing a theory for the invisibility of a smooth object L,
we borrow the term 'umbra' from shadow computation:
the umbra
is the set of viewpoints in the scene from which L is completely invisible.
% it is useful to interpret L as a light,
% whether or not the application is shadow computation.
This allows the familiar idea of an object casting a shadow to
add intuitive insight to several constructs that must be developed.
% Since the region of invisibility from the light is called the umbra,
% we borrow this terminology for our problem (Definition~\ref{defn:umbra}).
However, note that the theory works equally well in
broader contexts of visibility analysis including
occlusion culling and motion planning.
% The following definition captures our problem.

The visibility of a scene will be analyzed using a divide and conquer strategy,
one object at a time.
This may appear impossible, since visibility is defined by the 
interaction of many objects.
However, a divide and conquer strategy can still be used, in which one object at a time
dominates, as follows.
First, the umbra cast by each object in isolation (the local umbra) is computed.
Next, each object's umbra is expanded and refined 
by taking other objects into account (the global umbra).
Finally, the umbra of the scene is the union of the global umbrae
of each object in the scene.

% We shall talk of the umbra cast by an object A,
% referring to the points that are fully invisible from L 
% because of the blocking action of A.

%%%%%%%%%%%%%%%%%%%%%%%%%%%%%%%%%%%%%%%%%%%%%%%%%%%%%%%%%%%%%%%%%%%%%%%%%%%%%

% P is in the local umbra of A if all rays from P to L are blocked by A.

	% umbra -r 1.2 data/umbra/woak.pts > foobar &
        % umbra -r 3 -P 14 -c .03 -S ../../scene/data/buildingSceneClean
\begin{figure}
\begin{center}
\includegraphics*[scale=.25]{img2/vg900globalall.jpg}
\includegraphics*[scale=.25]{img2/buildingSceneCleangbu.jpg}
\end{center}
\caption{Analyzing the invisibility of the object L in a scene.
         The umbra of each object is drawn in 
         a different color.  In the bottom figure, 
	 local and global umbra are drawn in different shades.}
% USE MORE CONCAVE LIGHTS IN EXAMPLES
\label{fig:impressive}
\end{figure}

	% umbra -r 1.2 ../data/woak.pts > foobar & 
\begin{figure}
\begin{center}
\includegraphics*[scale=.15]{img2/woakBitang.jpg}
\includegraphics*[scale=.15]{img2/woaku.jpg}
\includegraphics*[scale=.15]{img2/woakbu.jpg}
\includegraphics*[scale=.15]{img2/woakfu.jpg}
\end{center}
\caption{(a) The bitangents of A and L. (b) The umbra of A. (c) The back umbra of A. (d) The front umbra of A.}
\label{fig:woakeg}
\end{figure}

	% umbra -r 1.2 data/umbra/woak.pts > foobar &
\begin{figure}
\begin{center}
% \includegraphics*[scale=.15]{img2/hud.jpg}
\includegraphics*[scale=.2]{img2/hudbu.jpg}
\includegraphics*[scale=.2]{img2/hudfu.jpg}
\end{center}
\caption{An architectural example: the footprint of a Marcel Breuer building. (a) The back umbra of A. (b) The front umbra of A.}
\label{fig:hudeg}
\end{figure}

% The subject of this paper is a computation of the umbra of a smooth scene.

\ifJournal
  % Two points see each other if their line of sight is unblocked.
  % A point sees an object if the point sees some point of the object.  % Mordor
  % An object sees a point if some point of the object sees the point.
  A point of the umbra sees no point of L.
  A point of the penumbra sees some points of L and is blocked from other points of L
  by an object $A \neq L$.
  A point of the illumina sees all points of L except for those that are blocked by L itself
  % The only points of L that are not seen by a point of the illumina are those that are blocked by L itself
  % For a point of the illumina, the only points that it does not see are those that are blocked by L itself
  (i.e., points on the other side of L).
  % no points of L are blocked by another object $A \neq L$.
\fi

%%%%%%%%%%%%%%%%%%%%%%%%%%%%%%%%%%%%%%%%%%%%%%%%%%%%%%%%%%%%%%%%%%%%%%%%%%%%%

\begin{defn2}
\label{defn:localglobal}
Consider a scene of objects, with a distinguished object L.
\ifJournal
  \begin{itemize}
  \item The {\bf L-umbra} is the locus of points from which L is completely invisible.
  \item The {\bf L-penumbra} is the locus of points from which L is partially visible.
  \item The {\bf L-illumina}\footnote{We are unaware of any term for this type of region, so
    a new term is introduced, derived from the Latin for 'within the light'.}
    % just as umbra is from the Latin for shadow and 'pen' is from the Latin for almost. 
  is the locus of points from which L is completely visible
  (i.e., no object blocks the view of L).
  \end{itemize}
  When L is clear from context, these terms 
  will be shortened to umbra, penumbra and illumina.
\fi

\ifJournal
  We begin with a definition of the three regions of object visibility.
\fi

\begin{itemize}
\item
The {\bf umbra} of the scene (with respect to L)
is the set of viewpoints from which L is completely invisible (Figure~\ref{fig:impressive}).
\item
The {\bf local umbra of an object A} is 
% the umbra that would be cast by A if it was the only object in the scene (besides L).
the umbra 
cast by A when A is imagined to be the only object in the scene, besides L (Figure~\ref{fig:woakeg}).
\item
The {\bf global umbra of A} is the umbra 
cast by A when all objects in the scene are taken into account.
\ifJournal
The {\bf local and global penumbra of A} are defined analogously.
\fi
For P to be a point of the global umbra of A, 
A must contribute in some way to the blocking of L from P,
although this blocking is often achieved through the cooperation of A with other objects
in the scene (Figure~\ref{fig:globalcornelllip}).
% although perhaps in concert with other objects of the scene.
% \item
% The {\bf local umbra of a scene with respect to L} 
% is the union of the local umbrae of the objects in the scene  (Figure~\ref{}).
% \item
% The {\bf global umbra of a scene with respect to L} 
% is the union of the global umbrae of the objects in the scene (Figure~\ref{}).
\end{itemize}
\end{defn2}

% Since we shall call the distinguished object L throughout the paper,
% we refer to this region simply as the umbra of the scene.
% The goal of this paper may be rephrased as the computation of the umbra of a smooth scene
% with respect to one of the objects in this scene.

The decomposition of the region where L is invisible
into possibly overlapping subumbrae cast by each object has the advantage
of simplicity, parallelizability, and incremental computation.
The concentration on one object at a time is
an important simplification to the structure of the computation,
since it allows one object to temporarily dominate all of the other objects.
In the computation of the umbra cast by an object A, 
all objects are of interest only through their relationship
to A or L, rather than their mutual interrelationships.
% This dominance of one object yields important simplifications when reasoning about visibility. 
Another benefit is parallelizability:
the computation of the umbra of an object A may be performed at the same
time as the computation of the umbra of an object B.
This is an important factor in a large scene.
% This keeps less balls up in the air at once.
% Thus, we only need to consider n-to-1 relationships, not n-to-n relationships.
A third benefit of computing invisibility object by object
is that a point will know which object(s) is
responsible for blocking it from L.
This allows efficient incremental updates to the invisibility structure upon the addition
or removal of an object.

It is well known that visual events occur at bitangents,
the lines that are tangent to at least two objects in the scene \cite{durand00a}.
The computation of object visibility can be viewed as the structured computation
of visual events, by filtering 
potential visual events down to true visual events (Figure~\ref{fig:woakeg}).
Which bitangents and which parts of these bitangents contribute to the umbra?
A taxonomy of bitangents that contribute to visual events is developed in this paper.
\ifJournal
An {\bf umbral bitangent} is a bitangent that contributes to the boundary of the umbra.
A {\bf penumbral bitangent} is a bitangent that contributes to the boundary of the penumbra.
\fi
% Only a few of the bitangents of a scene are important to visibility 
An algorithmic strategy of our computation of object invisibility is sweeping.
The local and global umbrae are found by sweeping a tangent about
certain objects in the scene, always guided by the dominating object A
and the distinguished object L.

The rest of the paper is structured as follows.
Section~\ref{sec:lit} reviews related literature.
Section~\ref{sec:ve} introduces our taxonomy of bitangents that contribute
to visual events. 
This section is a general reference for terms used in the later algorithmic development.
Section~\ref{sec:single} covers the foundational case of computing 
invisibility in a scene of two objects.
Sections~\ref{sec:scene}-\ref{sec:globalfront} generalize to arbitrary scenes.
Section~\ref{sec:divide} explores the advantages of our divide-and-conquer approach.
Finally, Section~\ref{sec:conclusions} offers some thoughts about the invisibility problem
and directions for future work.

\ifSurrounding
\else
% Say that A surrounds B if B lies entirely inside the convex hull of A.
For conciseness and clarity, we assume in the rest of the paper that 
L does not lie entirely inside the convex hull of any object of the scene
and no object lies entirely inside the convex hull of L.
That is, no object surrounds another, which is the typical case.
The generalization of the theory to cover surrounding objects is straightforward,
mostly involving generalizations of the definition of different
categories of bitangent, and is covered in a technical report.
\fi

% See Remark 26 in Section 20 below (near the end) 
% about TREATING TOUCHING OBSTACLES CAREFULLY.
If two objects touch, then they must be treated as a single object in the theory.
Care must be taken not to treat them as two individual objects if they touch,
as illustrated in Figure~\ref{fig:touch}.

% umbra -P 9 ../data/circleUmbraProblem.pts
% umbra -P 9 ../data/circleUmbraProblem2.pts
\begin{figure}
\begin{center}
\includegraphics*[scale=.15]{img2/circleAlmostTouching.jpg}
\includegraphics*[scale=.15]{img2/circleTouching.jpg}
\end{center}
\caption{If two objects touch, they must be treated as a single object.
         (a) As two objects approach, the umbra is computed correctly.
         (b) When they touch, they must be treated as a single object.}
\label{fig:touch}
\end{figure}

% For simplicity in drawing the regions of visibility,
% we enclose the scene of objects in a room.
% None of the theory depends upon this enclosing room.

% \clearpage

\subsection{Motivation}

Suppose that we know how to compute the L-umbra cast by an object A [L-occlusion of A?].
Then we can perform visibility analysis as follows.
The L-umbra [L-occlusion?] of all objects with respect to the distinguished object L
defines the regions occluded from L.
The union of the occluded regions for all objects L defines the visibility of the scene.

L-umbra does not have the right connotations!
area of scene where L must be rendered [is at least partially visible] = 
                                                                  complement of L-umbra
Visibility analysis of a scene = union {objects L in scene} L-umbra 
L-umbra                        = union {objects A $\neq$ L in scene} L-umbra cast by A
PVS of scene at P = all objects L such that P lies in the complement of their L-umbra
   - this can be precomputed
but the complexity of aspect graphs is known to be too large
we aren't doing much better

umbra cast by the light L = L-umbra

% ----------------------------------------------------------------------------------------

\clearpage

\section{Related work}
\label{sec:lit}

% NSF review of literature and discussion in its text (Visibility Analysis subsection).
% Do we want to stress the sophisticated aspect graph literature (as we do below)?

The vast literature on visibility analysis is well reviewed in \cite{durand00a}.
We can only highlight a few issues here.
Recently, there has been considerable work on pushing visibility analysis,
and shadow computation in particular,
to interactive speeds on polygon meshes \cite{assarsson03,bala03,sen03},
including huge polygon meshes \cite{unc03}.
An orthogonal direction to improve visibility and shadow computation is to higher
fidelity smooth models, which is the approach taken in this paper.
We are particularly 
motivated by the call in \cite{hart99} to consider visibility for non-polygonal surfaces.

Another direction to push visibility analysis is towards incremental, parallelizable
computation, motivated by the expense of visibility analysis
and its need for interactivity.
The algorithm developed in this paper lends itself to parallelization and efficient
incremental computation through its divide-and-conquer approach, treating the influence
of one object at a time.
This issue is discussed more fully in Section~\ref{sec:divide}.
% The divide-and-conquer approach taken by this paper lends itself to incremental
% computation and parallelization (Section~\ref{sec:divide}).

Object visibility for polyhedral meshes 
has been widely studied in the context of general visibility analysis,
global illumination, shadow computation, and radiosity.
Constructs that have been developed include the 
aspect graph \cite{gigus90}, 
discontinuity mesh \cite{drettakis94,heckbert92a,lischinski92,stewart94}, and
visibility skeleton and complex \cite{durand96,durand97a,pocchiola96}.
% This theory is again developed for polygons and polyhedra.
% introduced by Heckbert for radiosity, used by Lischinski
% downplay discontinuity mesh: Durand quote: 'prone to robustness problems and 
% irregular meshes induced are hard to handle'
% Fiume \cite{} and Stewart \cite{} have developed discontinuity meshes
% that analyze visibility amongst polyhedra.
% Durand \cite{} has developed a visibility skeleton for polygons and polyhedra.

% such as Teller \cite{soda hall SIGGRAPH paper}.
The analysis of visibility in smooth scenes %, and especially its global structure,
  % theory for scenes of smooth objects
has begun, with a considerable vision literature on aspect graphs 
of smooth surfaces (e.g., \cite{chen91,eggert93,koenderink76,petitjean92,ponce90},
including solids of revolution, algebraic surfaces, quadrics and parametric surfaces).
However, most of this work is qualitative
and at an abstract, topological, or nonalgorithmic level.
The algorithm of this paper has been implemented and works well.
There has also been some work on
visibility complexes of smooth convex objects \cite{durand97b} and
the introduction of Bezier curves into visibility complexes \cite{baciu00}.
	% koenderink76, koenderink79: definition of aspect graph
	% gigus90,gigus91: polyhedra
	% sri89: curved objects
	% eggert89,eggert93: solids of revolution
	% kriegman89,kriegman90: solids of revolution
	% ponce90: parametric surfaces
	% chen91: quadrics
	% rieger93: algebraic surfaces
	% petitjean92, etc.: algebraic surfaces
% Bitangency structures are not actually computed
% and most methods are not implemented.

The literature on shadows is well reviewed in \cite{wooReview}.
The entire motivation for shadow maps \cite{williams78}
was the computation of shadows of smooth objects,
but this is done at the expense of an image space algorithm prone to aliasing,
and a point light.
Most work on area light sources has concentrated on convex lights (and again polyhedral
environments), while this paper allows the area light to have an arbitrary shape.

Most shadow algorithms only compute the intersection of the umbra with the receivers 
(the objects and the walls).
We analyze the entire area of umbra, since our goal is not just to render the scene,
but to understand all points
that are invisible to L for applications such as occlusion culling and motion planning.
% discontinuity mesh also does the following:
% Our point of view takes a more global and eager (in the technical sense) 
% perspective, analyzing the entire scene at once, 
% rather than lazily (in the technical sense) evaluating visibility only from the active point.

Work on potentially visible sets and conservative visibility
has been applied to occlusion culling \cite{durand00b,leyvand03,teller91,teller93},
again for polyhedral environments.
% The occlusion culling problem in a smooth scene defined by implicit surfaces 
% has been studied using polars by -- and Sederberg \cite{sederberg}.

% Application 1 ({\bf occlusion culling}: camera locations where object need not be rendered) is illustrated
% by Sederberg's SIGGRAPH paper on rendering, using polars.

% One major issue is that most attention has focussed on scenes of polygons and polyhedra.
% A novelty of this paper is the visibility analysis of a smooth object in a smooth scene.


% Since the literature on visibility analysis is vast 
% (and well reviewed in \cite{durand00a}),
% Durand {\bf review} of visibility.
% we can only discuss a few issues 
% and highlight some differences with the existing literature.

% We are not aware of any work on a geometric (region-based) definition
% of the shadow cast by a {\bf smooth} area light source
% in a scene of smooth objects.


\Comment{
Drettakis and Fiume \cite{drettakis94}
build a discontinuity mesh for a polygonal light and polyhedral scene.

Define the {\bf discontinuity mesh}, referring to 94 SIGGRAPH papers.
Since we concentrate on invisibility in this paper,
we have not yet built the complete discontinuity mesh 
for a smooth scene of curves in 2-space.
However, the theory developed in this paper paves the way for the complete
discontinuity mesh.
This discontinuity mesh is a fundamental tool for visibility analysis in a smooth scene.
For example, it can be used to compute soft shadows in a scene of 
smooth objects and smooth area lights.
And it can be used for occlusion culling in a scene of smooth objects.
The curve discontinuity mesh is similar to the aspect graph
of a smooth scene in 2-space.

Work on {\bf discontinuity meshes} also analyzes visibility from a global perspective
rather than the present position of the camera.

Note: Drettakis/Fiume-style {\bf discontinuity mesh}
in 3D only meshes the {\bf receivers} (object boundaries),
not the entire space.  This is motivated by radiosity applications.
However, for full visibility information, we want a meshing of the entire space, which
is the approach we take in this paper.
This is more analogous to Teller-style visibility analysis for walkthroughs.
For example, we want to be able to move an object through the scene and determine that it 
is in shadow as it enters a region of umbra.  Or determine that the viewer is in umbra as it
moves through a cell.

{\bf Fiume and Stewart in SIGGRAPH 94} address application 5 (umbra/penumbra) 
in a polyhedral scene.
The light is restricted to an area light source.
As always with lighting analysis, the problem is to define
the boundary of the umbra/penumbra on the receivers.
This is weaker than computing the boundary of the regions in 3-space 
(not just on the receivers) that define the umbra/penumbra.
{\bf (Give an example that shows the difference.)}
}

% Work on shadows from an area light source.
\ifTalk
The {\bf area light} problem is a generalization of the point light problem.
As the light shrinks to a point, the area light becomes a point light
and inner/outer bitangents become tangents through the point.
\fi

\ifJournal
Although it is well understood that bitangents define visual events,
an algorithm to systematically and actively filter these bitangents to the boundaries
of visibility regions has not been developed.
\fi

% {\bf Pointwise} definition of umbra and penumbra is simple (see appendix).

% How does antipenumbra paper relate? it doesn't as it computes visibility through a portal

% Artistic literature on shadow and lighting.

\ifJournal
Bitangents are required throughout the paper, since they define umbral boundaries.
We have developed efficient algorithms for computing bitangents
using intersection in dual space \cite{jj01}.
\fi

% \clearpage

% -------------------------------------------------------------------------------------

\clearpage

\section{Surface analogue}

\begin{defn2}
Let A and B be closed surfaces.
P is a {\bf bitangent plane} of A and B if it is tangent to both A and B.
D is a {\bf bitangent developable} of A and B if it is the envelope of a one-parameter
family of bitangent planes of A and B.
\end{defn2}

\begin{defn2}
The bitangent developable D of A and B is {\bf extremal} if it does not intersect A or B.
The extremal bitangent developable D of A and B is {\bf outer}
if A and B lie on the same side of all its constituent bitangent planes,
% e.g., cylinder between 2 spheres
and {\bf inner} if A and B lie on opposite sides of all its constituent bitangent planes.
% cone of 2 spheres
\end{defn2}

\begin{defn2}
The bitangent developable D of A and B is {\bf A-extremal} if it does not
intersect A and all intersections of D with B lie on A's side of the tangent plane [is this well-defined?]?
\end{defn2}

\begin{defn2}
The bitangent developable D of A and B is {\bf B-piercing} 
if it does not intersect A and it has some intersections with B, but none between
A and B and none behind A.
\end{defn2}

\clearpage

\subsection{2.5D analysis, using 2D analysis}

\subsection{2 closed surfaces}

bitangent developable

A-early patch; middle patch; B-early patch

bitangent developable is visual if it does not intersect A; the middle patch does not
intersect B.

Outer visual if A and B lie on the same side of T

visual event is a prefix of the B-early patch, from its curve of tangency with B
until first curve of intersection with B (if any) or another visual event.

Realistically, are we interested in this type of decomposition?

\clearpage

\section{Bitangents that may define visual events: the penumbral case} % and thus inner

\begin{lemma}
\label{lem:twoInner}
\ifCommentary
Let A and B be closed curves whose interiors have no intersection.
\fi
If the convex hulls of A and B do not intersect,
then A and B have exactly two inner bitangents.
\end{lemma}
\prf
See the proof of Lemma~\ref{lem:twoOuter}.
\QED


\section{Casting umbra}

\begin{defn2}
Consider a scene of smooth curves, with one distinguished curve L.
The {\bf umbra} of the scene is the locus of points in free space 
from which no part of L is visible.
(The umbra will be called the {\bf L-umbra} in those rare cases when L is not understood 
from context.)
   [The {\bf L-umbra of a scene} is the locus of points in the scene from which the 
    distinguished object L is occluded (completely invisible).]
Every object (curve) A in the scene causes some occlusion.
The umbra can be decomposed into various overlapping regions, based on
the objects that are responsible for the occlusion of L.
The {\bf umbra cast by A} is the subset of the umbra % area of occlusion from L 
for which A is (at least partially) responsible.\footnote{This 
  notion of responsibility will become clearer
  as we develop this idea throughout the paper.}
The umbra cast by A may be split into two parts: the front and back umbra,
based on its location relative to A.
The precise definition of these two areas is best left to the construction phase?
The front umbra is due to concavities of A that cast umbra 
The {\bf front umbra cast by A} is the subset of the umbra cast by A 
that lies between L and A, caused by an overhanging concavity of A.
A line of sight from L to a point P of the front umbra will hit A both before and after P.
The {\bf back umbra cast by A} is the remaining subset that lies behind A.
\end{defn2}

There is a subtle difference between the computation of the front and back umbra,
which necessitates our distinction between these two regions.

There are certain advantages to analyzing visibility using this divide-and-conquer
approach, including parallelization and dynamism.
We are computing the area of the scene that is occluded because of A, for each object A.
Thus, if an object is not visible from a certain viewpoint, 
we will know exactly what set of objects are responsible for this occlusion.
The first advantage is that 
the occlusion due to an object A may be computed in parallel with the occlusion 
due to an object B, so the expense of visibility analysis may be parallelized.S
The second advantage is that it is valuable to assign responsibility for occlusion:
for example, if an object A moves, only those occlusion cells that involve A need
to be recomputed.

% ----------------------------------------------------------------------------------------

\clearpage

\section{A foundational case}

See Lemma~\ref{lem:localbackumbra} and Figure~\ref{fig:backumbra}.

See Lemma~\ref{lem:localbackumbrasurrounding} and Figure~\ref{fig:backsurrounding}.

% The definition of a visibility zone is tantamount to the defn of a special bitangent.
% local back: outer bitangent
% local front: inner bitangent

% -------------------------------------------------------------------------------------

\clearpage

\section{The effect of other objects}

When other objects are added to the scene,
the back umbra cast by a single object A will often be influenced by these other objects,
which conspire with A to occlude L from a larger region 
(Figure~\ref{fig:globalcornelllip}).
Our computation of the back umbra cast by A in a complicated scene
will depend on the computation of the back umbra cast by A in a scene consisting 
only of A and L, by expanding the latter umbra in a series of sweeps.

\begin{defn2}
The {\bf local umbra cast by A} is the umbra cast by A when A and L are the only
objects in the scene.
The {\bf global umbra cast by A} is the umbra cast by A in a general scene.
For P to be a point of the global umbra cast by A, 
A must be responsible in some way for the blocking of L from P,
although this blocking is often achieved through the cooperation of A with other objects
in the scene.
% although perhaps in concert with other objects of the scene.
\end{defn2}

\begin{lemma}
The global back umbra of A is larger than the local back umbra of A if and only if
an object B intersects the middle segment of an outer bitangent of A and L
(Figure~\ref{fig:crossingouter}).
\end{lemma}
\prf
INCOMPLETE.
The local back umbra of A is defined by the outer bitangents of A and L.
An object B will not affect the umbra cast by A if it lies to the left, behind, or to the right of A.
Even an object that lies entirely between the outer bitangents of A and L
does not affect A's back umbra all by itself, although
it may do so by teaming up with other objects
(Figure~\ref{fig:innersweep}).
Another object B in the scene can expand A's back umbra in only one way:
by interacting with the bitangents that define the umbra, before they reach A.
\QED


        % umbra -S -r 1.5 data/scene/lightSceneCornellLip2Minus > foobar &
        % umbra -S -r 1.5 data/scene/lightSceneCornellLip2 > foobar &
\begin{figure}[h]
\begin{center}
\includegraphics*[scale=.15]{img2/cornelllip2local.jpg}
\includegraphics*[scale=.15]{img2/cornelllip2global.jpg}
\includegraphics*[scale=.15]{img2/cornelllip2globalall.jpg}
\end{center}
\caption{Other objects may influence the umbra cast by A.
  (a) The back umbra cast by A in a scene of two objects. 
  (b) The back umbra cast by A in a scene of three objects. 
  (c) The entire back umbra of this scene.}
\label{fig:globalcornelllip}
\end{figure}

% a simple example of how another object B in the
% scene may expand the back umbra cast by A, by conspiring with A to block more of L.
  % When is the global back umbra of A larger than the local back umbra of A?
  % Figure~\ref{fig:localvsglobalpure} is a pure expression of 
  % the expansion from the local umbra to the global umbra due to the presence of 
  % other objects.
% The local umbra may be computed by the methods of Section~\ref{sec:single}.
% The next section begins our study of the global umbra.

	% umbra data/umbra/globalumb1.pts > foobar &
	% umbra data/umbra/globalumb2.pts > foobar &
\begin{figure}
\begin{center}
\includegraphics*[scale=.10]{img2/gu1outer.jpg}
\includegraphics*[scale=.10]{img2/gu2outer.jpg}
\end{center}
\caption{The back umbra of A expands when an object intersects the 
         middle segment of an outer bitangent of A and L}
\label{fig:crossingouter}
\end{figure}

We shall assume in the sequel that some object B intersects the middle segment of an outer bitangent of A and L
(otherwise the global back umbra is equivalent to the local back umbra).

The global back umbra of an object A is computed by refining its local back umbra,
using two sweeps.
% The local umbra is refined to incorporate the context of neighbouring objects using two sweeps.
The first sweep potentially expands the umbra and is computed by sweeping a tangent of A.
The second sweep potentially shrinks the umbra and is computed by sweeping a tangent of L.
The next section discusses the first sweep, called the inner sweep.

% During the computation of the global umbra, it will again become clear
% that the computation of object visibility
% in a smooth environment is largely a search for the relevant bitangents 
% in the massive collection of bitangents between scene objects.
% These relevant bitangents will be called umbral and penumbral bitangents.

% -------------------------------------------------------------------------------------

An algorithm for the inner sweep of a bitangent T is as follows.
This algorithm is applied to each outer bitangent of A and L in turn.

\vspace{.1in}

\centerline{innersweep(T,A)}

\begin{enumerate}
\item while an object B interferes with T on the segment between A and L
\begin{enumerate}
% \item CANDIDATE = \{inner bitangents of A/B and A/L\}
% \item T = closest bitangent of CANDIDATE, in sweeping on A towards L
\item T = closest inner bitangent of A and B, in sweeping on A towards L
\item if T has swept past L (i.e., past an inner bitangent of A and L), return (T)

% \item T = inner bitangent of A and B (the closest from the present T 
%         in sweeping towards L)
% \item if T has swept past L (i.e., past an inner bitangent of A and L)
% \begin{enumerate}
% \item  T = this inner bitangent of A and L
% \item  return
% done = true
% \end{enumerate}
\end{enumerate}
\item return (T)
\end{enumerate}

\ifJournal
Figure~\ref{fig:localvsglobalpure}b illustrates a global back umbra computed 
using inner sweeps to refine the local back umbra.
(GET A MORE ELABORATE EXAMPLE, PERHAPS ONE THAT DOES NOT REQUIRE AN OUTER SWEEP.)
\fi

	% umbra data/umbra/globalumb3.pts
\begin{figure}[b]
\begin{center}
\includegraphics*[scale=.15]{img/globalumb3innersweep1.jpg}
\includegraphics*[scale=.15]{img/globalumb3innersweep2.jpg}
\includegraphics*[scale=.15]{img/globalumb3innersweep3.jpg}
\end{center}
\caption{Three stages of an inner sweep
	 (a) T = outer bitangent of A and L
	 (b) T = inner bitangent of A and B
	 (c) T = inner bitangent of A and C}
\label{fig:innersweep}
\end{figure}

\clearpage

% 0000000000000

The algorithm for the outer sweep of a bitangent T is as follows.
This algorithm is applied to each outer bitangent of A and L in turn.

% \item[iii.]	If angle(H) 'exceeds' (in the direction of current travel) angle(G), return (NULL).

\vspace{.2in}

\centerline{outersweep (T,A)}

\begin{enumerate}
% \item T = outer bitangent of A and L
\item G = innersweep(T,A)
% \item sweep direction = away from A
\item while an object B interferes with T on the segment between L and G
\begin{enumerate}
% \item S = \{outer bitangents of B and L\}
\item T = closest outer bitangent of B and L, in sweeping on L away from A
\item if T has swept past G (i.e., T does not intersect the late segment of G), 
      return($\emptyset$)
\end{enumerate}
% \item clip G by T [LEAVE THIS TO POSTPROCESSING]
\item return(T)
\end{enumerate}

	% umbra data/umbra/globalumb7.pts
\begin{figure}[b]
\begin{center}
\includegraphics*[scale=.15]{img/globalumb7outersweep1.jpg}
\includegraphics*[scale=.15]{img/globalumb7outersweep2.jpg}
\includegraphics*[scale=.15]{img/globalumb7outersweep3.jpg}
\end{center}
\caption{Three stages of an outer sweep of T 
         (a) T = outer bitangent of L and A
	 (b) T = outer bitangent of L and B
	 (c) T = outer bitangent of L and C}
\label{fig:outersweep}
\end{figure}

	% umbra data/umbra/globalumbouter.pts
\begin{figure}
\begin{center}
\includegraphics*[scale=.1]{img2/guoInnertang.jpg}
\includegraphics*[scale=.1]{img2/guoOutertang.jpg}
\includegraphics*[scale=.1]{img2/guoglobal.jpg}
\end{center}
\caption{The necessity of the outer sweep.
%  L may reappear on the other side of an interfering object B. 
        (a) The visibility boundary G defined by an inner sweep. 
        (b) The visibility boundary defined by an outer sweep
        (c) The global umbra of A.
            Note the clipping of G as L reappears to view.}
\label{fig:needoutersweep}
\end{figure}

\ifJournal
The output of the outer sweep algorithm outersweep(T,A)
is a new boundary of the global umbra,
which starts after its intersection with innersweep(T,A).
See Figures~\ref{fig:needoutersweep}c and \ref{fig:globalcornelllip}c.
\fi

% \clearpage

Finally, we have an algorithm to compute the boundary of the 
global back umbra of an object A.
The outer sweep is only necessary if the inner sweep is nontrivial.

\begin{enumerate}
\item boundary = $\emptyset$
\item for each outer bitangent T of A and L
\begin{enumerate}
\item G = \{innersweep(T,A)\}
\item if $\mbox{G} \neq \{\mbox{T}\}$, then H = \{outersweep(T,A)\}, else H = $\emptyset$
\item boundary = boundary $\cup$ G $\cup$ H
\end{enumerate}
\end{enumerate}

The boundary also contains the curve segment 
of A that lies between the bitangents defined by inner sweeps and on the other side from L.

EMPHASIZE THIS PARAGRAPH.
In short, 
the global back umbra of an object 
is bounded by (at most) four bitangents and a curve segment.
Two of the bitangents are inner, generated by inner sweeps,
and two of the bitangents are outer, generated by outer sweeps.
One or both of the outer bitangents may be excluded, if they do not clip anything.
All of the sweeps start from outer bitangents of A and L.

Figure~\ref{fig:globalback} offers other examples of the global back umbra.

	% umbra -f .7 -P 14 ../data/umbra1withB4.pts
	% umbra data/umbra/vg900new.pts > foobar & : crashes
	% umbra data/umbra/vg900.pts > foobar &
\begin{figure}
\begin{center}
\includegraphics*[scale=.15]{img2/umbra1withBgbu.jpg}
% \includegraphics*[scale=.15]{img2/vg900newumbra.jpg}
\includegraphics*[scale=.15]{img2/vg900global.jpg}
\end{center}
\caption{More examples of the global back umbra}
%         (a) ALSO SHOWS POTENTIAL MERGING OF FRONT AND BACK UMBRA;
% Case 3 of the global front umbra of A: an outer piercing bitangent of A and L is replaced by an inner bitangent of A and L, and this component of the global front umbra is part of the global back umbra}
\label{fig:globalback}
\end{figure}

\clearpage

\subsection{The inner sweep}
\label{sec:innersweep}

START HERE

% In this section, we begin to show how to compute the global back umbra from the 
% local back umbra.
% The next section finishes this computation.
% Recall that the local back umbra of A is bounded by the outer bitangents of A and L
% (Lemma~\ref{lem:localbackumbra}).

A boundary of invisibility must itself see L.
When there are no other objects in the scene, an outer bitangent of A and L sees L.
However, as other objects are added to the scene, this property is destroyed
(Figure~\ref{fig:innersweep}a).
The purpose of the first sweep is to move the outer bitangent until its sight of L 
is restored,
by sweeping it past interfering objects (Figure~\ref{fig:innersweep}).
This sweep is performed in the tangent space of A, towards the inside of L.
% The bitangents that define the global back umbra may be found by refining
% the bitangents that define the local back umbra, until they see L.

The continuous sweep is replaced by a discrete sweep, as follows.
The sweep will always stop at a bitangent, since bitangents define visual events.
In this particular case, the sweeping tangent must stop at an inner bitangent
of A and the interfering object B (Figure~\ref{fig:innersweep}b-c).
We jump directly to the next bitangent candidate and test for completion.
The sweep is complete if the new bitangent sees L:
i.e., its segment between A and L has no intersections with other objects
(Figure~\ref{fig:innersweep}c but not Figure~\ref{fig:innersweep}b).
If the new bitangent still does not see L, because of the blocking of a new object C,
the sweep continues recursively.
The sweep can also be stopped if it has swept past L, marked by
sweeping past an inner bitangent of A and L.
Notice that the umbra of A will never extend past the inner bitangent of A and L,
since beyond this bitangent A can have no possible effect on the visibility of L.

% {\bf At first time that we stop at light:
% explain that we stop at light because maximal umbra of A 
% is defined by inner bitangents with L.}

\ifJournal
  In this case, the global umbra of A is set to the penumbra of A,
  which is declared to be the maximal size of the global umbra.\footnote{The only
    points that can be blocked from L by A are the points inside the
    penumbra of A.  The responsibility for blocking L is rightly passed
    to another object when the point leaves A's penumbra.}
\fi

Since the sweep is towards the interior of L and repeatedly 
replaces the present bitangent by an inner bitangent,
it is called an {\bf inner sweep}.

% -------------------------------------------------------------------------------------

\clearpage

\subsection{The outer sweep}

Inner sweeps begin the refinement from local to global back umbra,
and outer sweeps complete the job.
Each inner sweep defines a visibility boundary on one side of a collection of
blocking objects between A and L (Figure~\ref{fig:innersweep}), % ~\ref{fig:globalcornelllip}
while each outer sweep defines a visibility boundary on the other side (Figure~\ref{fig:outersweep}). % \ref{fig:globalcornelllip}

% Although G marks a boundary of the umbra,
% L will often become visible again at some point of G,
% on the other side of the blocking objects.
% The purpose of the second sweep is to find this point of reappearance
% of L and clip G there.

% by moving some umbral bitangents of A.
% A second phase of refinement is required to add umbral bitangents 
% that clip the umbral bitangents generated by the inner sweep.

% This section describes this second sweep.

Consider an outer bitangent T of A and L.
The inner sweep moves T so that it sees L past any interfering objects B,
yielding the bitangent G (Figure~\ref{fig:needoutersweep}a).
% = innersweep(T,A).
% The inner sweep finds a bitangent T at which L becomes visible again after an interfering object B.
Although G is a visibility boundary of L near A,
as we leave A on G, 
L may eventually become visible again from the other side of blocking object B
(Figure~\ref{fig:needoutersweep}b).  % Figure~\ref{fig:globalcornelllip}, 
\ifTalk
A sunrise to match the sunset.
\fi
% However, there is another visibility boundary of L on the other side of B
A second sweep must find this other visibility boundary and use it to clip G
(Figure~\ref{fig:needoutersweep}c).

While the inner sweep moves in the tangent space of A searching for L,
the outer sweep moves in the tangent space of L, searching for G
(Figure~\ref{fig:outersweep}).
% A tangent of L is swept across all objects that block it from G.
As with the inner sweep, the tangent is initialized to an outer bitangent of A and L,
% (i.e., a boundary of the local umbra),
but we now sweep in the opposite direction, towards the outside of A.
% When L's tangent sees G, 
% it is used to clip G, becoming another boundary of the global umbra.
The sweep can be stopped as soon as the tangent passes G,
in which case G need not be refined.
We again leverage the fact that changes of visibility
must occur at discrete bitangent events to convert the continuous sweep into a discrete sweep.
Since the sweep is towards the outside of A and repeatedly
replaces the present bitangent by an outer bitangent,
it is called an {\bf outer sweep}.

% Let T be an outer bitangent of A and L such that an object B intersects the middle segment 
% of T.\footnote{If no such bitangent exists, 
%   the global umbra is the same as the local umbra,
%   and neither inner nor outer sweep is necessary.}
% T is repeatedly replaced by the outer bitangent of B and L,
% until the middle segment of this outer bitangent is free of objects B (Figure~\ref{fig:outersweep}).

% The inner sweep expands the umbra and the outer sweep shrinks it.
% This potentially clips each umbral bitangent generated by the inner sweep.

% ---------------------------------------------------------------------------------------

\clearpage

\section{The front umbra: foundational case}

% A scene of two objects
% The local umbra
% The umbra/penumbra cast by a single object
% Object visibility in a scene of two objects
\label{sec:single}

[
  The computation of visibility is better, or more constructively,
  approached as the computation of blocking.
  When L is alone in the scene, everything is visible and nothing is blocked.
  As objects are added to the scene, how much more of L
  becomes blocked and therefore invisible?
  Seen from this perspective, the base case for object visibility is
  the scene composed of two objects, the object of interest L and a single blocking object A
  This section considers this base case.
]

\begin{defn2}
\label{defn:piercing1}
Let A be the distinguished object.
A bitangent T of A and B is {\bf piercing}
% \item T is a bitangent of A and B with points of tangency $P_A \in A$ and $P_B \in B$,
% if \lyne{P_A P_B} does not intersect A and \seg{P_A P_B} does not intersect B,
% but \ray{P_A P_B} does intersect B.
if it does not intersect A at all,
it intersects B on its B-early segment, but nowhere else, and
% That is, intersections with B are allowed (actually, required) behind B, closer
% to $P_B$ than to $P_A$.
$P_B$ is not an inflection point.
\end{defn2}

We call them piercing bitangents because they pierce B after their point of bitangency
with B.
(Imagine lines of sight emanating from A along the bitangents.)

Our analysis of invisibility begins by
computing where L is invisible in a scene of only two smooth objects A and L.
The general problem is solved by refining this base case.
% This section deals with the base case of a scene of two objects, L and A.
The region of invisibility in a scene of two objects comes in two flavors:
the {\bf back umbra} cast behind A and 
the {\bf front umbra} cast by A onto itself, which lies between L and A
(Figure~\ref{fig:woakeg}).
% P lies in the back umbra (resp., front umbra) of A if a line from P to L 
% hits A in only one direction (resp., in both directions) from P.
\ifJournal
The penumbra comes in two analogous flavors.
\fi
The back umbra is associated with outer bitangents and the front umbra
with outer piercing bitangents.

% In a practice that shall become familiar in the definition of other visibility zones,
% the definition of the back umbra depends on the definition of a special
% type of bitangent that bounds this visibility zone.

% ---------------------------------------------------------------------------------------

% -------------------------------------------------------------------------------------

% \clearpage

The front umbra of A is composed of zero or more components (Figures~\ref{fig:woakeg}d and \ref{fig:frontumbra}).
Each outer piercing bitangent of A 
defines an overhang that blocks L from more of the scene.
Each component of the front umbra is bounded by the bridge segment of
an outer piercing bitangent of A and L and
the relevant curve segment of A.

	% umbra data/umbra/umbra1.pts > foobar &
\begin{figure}[h]
\begin{center}
\includegraphics*[scale=.15]{img2/umbra1front.jpg}
\end{center}
\caption{The front umbra of A}
\label{fig:frontumbra}
\end{figure}

\ifSurrounding
Figure~\ref{fig:frontsurrounding} illustrates the case when A surrounds L, which follows
the same construction.
\fi

% -------------------------------------------------------------------------------------

\ifJournal
To review, the visibility of a single object in a scene of two objects
can be decomposed into five zones (Figure~\ref{fig:umbra2}):
(1) the back umbra, defined by the late segments of outer bitangents;
(2) the back penumbra, defined by the late segments of inner bitangents;
(3) the front umbra, defined by the bridge segments of outer piercing bitangents; 
(4) the front penumbra, defined by the bridge segments of inner piercing bitangents; and
(5) the illumina, defined by the rest of the scene's free space.

	% umbra data/umbra/umbra1.pts > foobar &
\begin{figure}[h]
\begin{center}
\includegraphics*[scale=.15]{img2/umbra1umbra.jpg}
\end{center}
\caption{The zones of visibility of the light}
\label{fig:umbra2}
\end{figure}
\fi

% -------------------------------------------------------------------------------------

\clearpage

\section{A scene of many objects}
% The influence of other objects}
\label{sec:scene}

Now that object invisibility is understood for a scene of two objects,
we can consider general scenes of many objects,
and transition from the local umbra to the global umbra
(Definition~\ref{defn:localglobal}).
Recall that we are still computing only the umbra cast by a single object in the scene, 
using a divide-and-conquer approach.
When there are many objects in a scene, 
the umbra cast by a single object may be influenced by other objects in the scene,
which conspire with A to hide more regions from L
% making the global umbra of A larger than the local umbra of A
(Figure~\ref{fig:globalcornelllip}).

% -------------------------------------------------------------------------------------

% -------------------------------------------------------------------------------------

\section{The global front umbra}
\label{sec:globalfront} 

The construction of the global front umbra of an object A
has many structural similarities to the construction of its global back umbra.
Both find the global umbra by refining a local umbra using two sweeps,
% the first sweeping on A and the second not,
the purpose of which is to restore visibility past blocking objects.
However, there are some subtle differences between the front and back umbra.
First, inner/outer bitangents are replaced by inner/outer piercing bitangents
and the front umbra may have several components.
The major change is in the second sweep, which begins by sweeping on L but later pivots
and sweeps on a blocking object B.
% We must also consider the possibility that the front umbra will grow into the back umbra.
% No we can ignore this case and allow front and back umbrae to overlap if necessary: see below.

Recall that each component of the local front umbra of A is defined by
an outer piercing bitangent T of A and L (Section~\ref{sec:single}).
The associated component of the global front umbra of A can be found as follows.
The {\bf inner piercing sweep} refines T by skirting around objects that interfere 
with its middle segment, sweeping in the tangent space of A towards L
(Figure~\ref{fig:cornell1sceneB}).
% During this process, T is replaced by inner bitangents.
The {\bf outer piercing sweep} refines T by skirting around interfering objects 
in the opposite direction, beginning by sweeping in the tangent space of L away from A
(Figure~\ref{fig:globalfrontumbra}c).
% During this process, T is replaced by outer bitangents.
An inner piercing sweep and an outer piercing sweep are applied to every outer 
piercing bitangent T of A and L,
generating the bitangents that form the boundary of the global front umbra of A.

Figure~\ref{fig:cornell1sceneJ2} % (OR FIGURE~\ref{fig:cornell1sceneJ}c).
illustrates the importance of computing the global front umbra.
% this example also shows that we should add the inner bitangent of A and L as an
% umbral bitangent
The global back umbra of all objects in the scene is shown on the left,
and the global front umbra of A on the right.
Notice that the global front umbra defines some previously undiscovered regions
of invisibility.
% Notice that the area added by the inner piercing sweep is enough,
% when added with the umbra of B, to fill in the entire inner region.
% Alone it is not.

        % umbra -f .5 ../data/cornell1sceneJ2.pts
\begin{figure}
\begin{center}
\includegraphics*[scale=.15]{img2/cornell1sceneJ2lfu2.jpg}
\includegraphics*[scale=.15]{img2/cornell1sceneJ2gfu2.jpg}
% \includegraphics*[scale=.25]{img2/cornell1sceneJ2gfu.jpg}
% \includegraphics*[scale=.25]{img2/cornell1sceneJ2gfu3.jpg}
\end{center}
\caption{Necessity of piercing sweeps to refine the local front umbra.
%         The global front umbra is necessary.
         The local front and back umbra are not enough to define
	 the regions of invisibility in the scene.
         (a) The local front and back umbra of the scene.
         (b) The global front umbra of A.
         Note that the global front umbra finds regions of invisibility
	 missed by the local front and back umbra.}
\label{fig:cornell1sceneJ2}
\end{figure}

\ifJournal
        % umbra ../data/cornell1sceneJ.pts
        % umbra -f 1 ../data/cornell1sceneK.pts
\begin{figure}
\begin{center}
\includegraphics*[scale=.15]{img2/cornell1sceneJinner.jpg}
% \includegraphics*[scale=.15]{img2/cornell1sceneKinner.jpg}
% \includegraphics*[scale=.15]{img2/--.jpg}
\includegraphics*[scale=.15]{img2/cornell1sceneJBumbra.jpg}
\end{center}
\caption{ALTERNATIVE EXAMPLE FOR NECESSITY OF GFU, WITH NO PROBLEMS RENDERING
         BACK UMBRA OF B}
%  Another example of the necessity of the global front umbra.
%         (a) Need to stop sweep at inner bitangent with light
%         (b) Need to stop sweep at inner piercing bitangent with light}
%         (c) Just the global umbra of A does not cover the umbra of the scene.
%         (d) Just the global umbra of B does not cover it.
%         (e) But together they do.
\label{fig:cornell1sceneJ}
\end{figure}
\fi

% ADD SOME FIGURES OF GLOBAL FRONT UMBRA.

% Figures: umbra -f .5 ../data/cornell1sceneJ2.pts (illustrates clipping by outer piercing sweep bitangent)


\subsection{The inner piercing sweep}

\ifJournal
Consider Figure~\ref{fig:globalfrontumbra}.
In the absence of B, A's front umbra is bounded by an outer piercing bitangent T.
However, B blocks L further and enlarges the umbra of A.
We can find where L reappears around B by moving in A's tangent space towards L,
starting at the outer piercing bitangent T.
The tangent will see past B at some inner bitangent,
typically at an inner piercing bitangent of A and B (Figure~\ref{fig:cornell1sceneB}b).
When we sweep to the end of B, L may still be blocked by another object C, 
in which case the sweep must continue recursively (as did the inner and outer sweeps).

Finally, L may never reappear during the inner piercing sweep, 
in which case we stop the global front umbra's growth as we sweep past L,
at an inner or inner piercing bitangent of A and L 
(Figure~\ref{fig:cornell1sceneJ}).
\fi

The algorithm for the inner piercing sweep of a bitangent T is 
identical to the inner sweep (Section~\ref{sec:innersweep}), 
except that piercing bitangents must also be considered.
\ifJournal
Interference is tested with the middle segment rather than the segment between A and L,
which is subtly different than the inner sweep.
\fi
This algorithm is applied to each outer piercing bitangent of A and L in turn.

\vspace{.1in}

\centerline{innerPiercingSweep(T,A)}

\begin{enumerate}
% \item initialT = T
\item while an object B interferes with T on the middle segment between A and L
\begin{enumerate}
     % \item $S = \{\mbox{inner and inner piercing bitangents of A and B or A and L}\}$
\item T = closest inner (or inner piercing) bitangent of A and B, 
      in sweeping on A towards L 
\item if T has swept past L (i.e., past an inner or inner piercing bitangent of A and L),
      return (T) % which we want to become a part of the global front umbra boundary
        % for example, consider cornell1sceneH if B was taller
% \item if T is an inner bitangent of A and B (we have a merge of front and back umbra),
%       return ($\emptyset$): BETTER SOLUTION THAN FOLLOWING OLD SOLUTION
\end{enumerate}
\item return (T)
\end{enumerate}

        % umbra -f .5 data/umbra/cornell1sceneJ2.pts
\begin{figure}
\begin{center}
\includegraphics*[scale=.15]{img2/cornell1sceneJ2lfu.jpg}
\includegraphics*[scale=.15]{img2/cornell1sceneJ2ips.jpg}
\includegraphics*[scale=.15]{img2/cornell1sceneJ2ops.jpg}
\includegraphics*[scale=.15]{img2/cornell1sceneJ2gfu.jpg}
\end{center}
\caption{Computing the global front umbra.
        (a) A component of the local front umbra of A.
        (a) The refinement of this component by an inner piercing sweep.
        (b) The refinement of this component by an outer piercing sweep.
	(c) The global front umbra of A.
%          PREFER TO USE cornell1sceneE.pts
% NEED TO DEBUG SCENEE BEFORE WE CAN USE IT: NOT A HIGH PRIORITY
% A good example for a typical outer piercing sweep
% is cornell1sceneE.pts (see cornell1sceneEbitang.jpg).
}
\label{fig:globalfrontumbra}
\end{figure}

\ifJournal
\begin{figure}
\begin{center}
\includegraphics*[scale=.25]{img2/cornell1sceneEbitang.jpg}
\end{center}
\caption{USE THIS EXAMPLE BUT EXPAND TO BE LIKE NEXT FIGURE.}
%         Illustration of sweeping to an inner bitangent of A and B.
%         Case 0 when outer piercing sweep saves the day and stops front umbra
%         before it merges with back umbra.
\end{figure}
\fi

% MORAL: ALWAYS SIMPLY RETURN THE BITANGENT YOU FIND IN THE INNER PIERCING SWEEP.
% YOU CAN'T TELL IF THERE IS A MERGING WITH THE BACK UMBRA 
% UNTIL THE OUTER PIERCING SWEEP,
% SINCE THE OUTER PIERCING SWEEP MAY CUT IT OFF BEFORE IT MERGES (BY CUTTING
% IT BY A PIERCING BITANGENT)

% OLD VERSION:
% \begin{enumerate}
% \item if T is an inner bitangent of A and B (we have a merge of front and back umbra)
% \begin{enumerate}
% \item NO:
%       if no inner sweep reached T during the creation of the global back umbra of A
% \begin{enumerate}
% \item U = closest outer bitangent of A and L 
%       (in sweeping on L away from A starting from initialT)
% \item V = innerSweep(U)
% \item replace V by T as a boundary of the global back umbra of A
% \end{enumerate}
% \item return ($\emptyset$)
% \end{enumerate}
% \end{enumerate}

The result of an inner piercing sweep becomes a boundary of the global front umbra
(Figure~\ref{fig:globalfrontumbra}a-b).

\ifJournal
Typically, the outer piercing bitangent T is replaced
by an inner piercing bitangent of A and B, which is the first
opportunity to see past B to L (Figure~\ref{fig:case1}).
% This is analogous to the inner sweep's move to an inner bitangent of A and B.
% Unlike the inner sweep however, where an inner bitangent of A and B always exists, 
However, A and B may have no inner piercing bitangents (Figure~\ref{fig:case2}).
Therefore, the search for the next bitangent must be broadened to include 
the inner piercing bitangents of A and L (the next best choice as shown in Figure~\ref{fig:case2},
but possibly also nonexistent) and
the inner bitangents of A and L (which are guaranteed to exist).
% \footnote{One might think that 
%  inner piercing bitangents of A and L should also be considered.
%  But if A and B do not have an inner piercing bitangent,
%  then any inner piercing bitangent of A and L must intersect B
%  (if it did not, then Figure~\ref{fig:case2} shows you that an inner piercing bitangent
%  of A and B would also exist).
%  Therefore, we can ignore inner piercing bitangents of A and L.
%  We only need to consider inner bitangents of A and L.}
%  STILL WANT: choosing inner bitangent of A and L in Figure of case 2 makes too big an umbra.
These bitangents also guarantee that the sweep does not continue past L.
Reaching a bitangent of L during the sweep is a sign that the sweep has past L and may stop.
% Reaching an inner bitangent of A and L is a sign that the front umbra has merged into the
% back umbra (Figure~\ref{fig:case3}) (and can safely be ignored?)
\fi

% \clearpage

        % umbra -c .02 data/umbra/cornell1sceneB.pts
	% umbra data/umbra/vg900new.pts
	% umbra data/umbra/vg900newest.pts
\begin{figure}
\begin{center}
\includegraphics*[scale=.15]{img2/cornell1sceneBlfu.jpg}
\includegraphics*[scale=.15]{img2/cornell1sceneBgfu.jpg}\
\includegraphics*[scale=.15]{img2/vg900newgfu.jpg}
\includegraphics*[scale=.15]{img2/vg900newestgfu.jpg}
\end{center}
\caption{Inner piercing sweeps.}
%         (a) A component of the local  front umbra of A.
%         (b) The refinement of this component through an inner piercing sweep,
%             moving in the tangent space of A, producing a component of the global
%	    front umbra of A.}
%         (c) Case 1 of the global front umbra of A: (a) a component of the local front umbra, bounded by an outer piercing bitangent of A and L; (b) an outer piercing bitangent of A and L is replaced by an inner piercing bitangent of A and B}
%         (d) Case 2 of the global front umbra of A: an outer piercing bitangent of A and L is replaced by an inner piercing bitangent of A and L}
\label{fig:cornell1sceneB}
% \label{fig:case1} for (c)
% \label{fig:case2} for (d)
\end{figure}


\subsection{The outer piercing sweep}

% Note: we don't care if the front umbra merges with the back umbra or not,
% since we will compute the front umbra independently from the back umbra,
% allowing them to overlap if that is the way it happens (just like we allow
% the umbra of A to overlap with the umbra of B without concern).

Like the outer sweep, the outer piercing sweep 
% The second refinement of T 
calculates the reappearance of L on the other side of the interfering objects,
which may trim the recently refined bitangent G that bounds the front umbra.
\ifJournal
We can find where L reappears around B on the other side by moving in L's tangent space away
from A.
The tangent will see past B at some outer bitangent,
typically at an outer bitangent of B and L (Figure~\ref{fig:cornell1sceneB}).
This bitangent may or may not (Figure~\ref{fig:cornell1sceneB}) 
clip the expanded front umbra.
\fi
% If the tangent sweeps past G before it sweeps past the interfering objects,
% the sweep is stopped.
%
% Change on 1/20/04:
% After sweeping past first object B,
% if an object C still intersects sweeping bitangent, then pivot this bitangent
% around B.
%
However, unlike the outer sweep, it is
possible for L to become visible through a hole in the blocking objects,
even though from the perspective of the sweeping tangent on L the objects
continuously block visibility (Figure~\ref{fig:pivotOutPiercingSweep}).
For this reason, we sweep on L past the first interfering object, but
if a second interfering object is found,
we begin sweeping in the tangent space of the first blocking object rather than L.
Also, if there is a choice, 
we must choose the interfering object closest to A.
% These two alterations distinguish the outer piercing sweep from the outer sweep.

     % umbra -P 5 -e .0001 -f .7 -c .03 ../data/outPiercingSweep4.pts
\begin{figure}
\begin{center}
\includegraphics*[scale=.15]{img2/outPiercingSweep4.jpg}
\end{center}
\caption{It is possible to grow the umbra too large if the outer piercing sweep remains
         in the tangent space of L throughout.
	 In this example, the sweep would continue past A and include all of the
	 region in front of A in the umbra, which is incorrect.
         After sweeping past the first object in the tangent space of L,
         the sweep must switch to the tangent space of the first interfering object.}
\label{fig:pivotOutPiercingSweep}
\end{figure}


% The general algorithm for the outer piercing sweep of T is as follows:
% while an object B interferes with T, sweep past B.
% Let FIRSTB be the first interfering object.
% The sweep should move in the tangent space of L the first time, and then
% in the tangent space of FIRSTB in subsequent steps.

The algorithm for the outer piercing sweep of a bitangent T is as follows.
% again incorporating recursive applications of the sweep when other objects become involved.
It is applied to each outer piercing bitangent of A and L in turn.

\vspace{.2in}

\centerline{outerPiercingSweep(T,A)}

\begin{enumerate}
\item G = innerPiercingSweep (T,A); firstTime = true
%     if G = $\emptyset$, return [nothing to clip]?
% \item OBJ = L
\item while an object B interferes with T on the segment between L and G
(if more than one object interferes, choose the one closest to A)
\begin{itemize}
\item if (firstTime)
\begin{itemize}
\item FIRSTB = B; firstTime = false
\item T = closest outer bitangent of B and L, 
      in sweeping on L away from A
      % no need for outer piercing bitangent: it's still an outer bitangent if it hits A
\end{itemize}
\item else
\begin{itemize}
\item T = closest inner bitangent of B and FIRSTB, 
      in sweeping on FIRSTB, continuing in the same direction as the first sweep
\end{itemize}
\item if T has swept past G, return ($\emptyset$)
% \item if (firstTime) CANDIDATE = \{outer bitangents of A/L and B/L\}
                      % to stop once you see past B and to stop sweeping past the light
% \item else CANDIDATE = \{inner bitangents of B/OBJ\}
% \item T = closest bitangent of CANDIDATE from the present T, sweeping in the tangent space
%           of OBJ (starting in the direction away from A and continuing in this direction
%	  in subsequent sweeps)
%\item if (firstTime) FIRSTB = B
\end{itemize}


\Comment{
% 1/25/04: it's easier not to worry about this merging case, and allow overlap of front and back umbra;
% I think that there is merging iff both G and T do not intersect A, in which case
% the front umbra can be ignored as it is encompassed by the back umbra, but I have no proof yet
\item if G does not intersect A 
% [i.e., G is an inner bitangent of A and B; inner piercing sweep swept past A]
      and T does not intersect A,
% [outer piercing sweep does not cut G off before it reaches the back umbra], 
we may have a merging of front and back umbra, so we
may want to ignore this front umbra component or add a boundary to the global back umbra.

{\bf Comments about this merging step of the algorithm:}
If the inner piercing sweep sweeps past A (i.e., G is an inner bitangent of A and B),
then the front umbra may potentially merge into the back umbra.
If the outer piercing sweep stops at a piercing bitangent (a bitangent that hits A),
then it will cut off the front umbra before it can merge with the back umbra.
However, if neither the bitangent generated by the inner piercing sweep
nor the bitangent generated by the outer piercing sweep hits A,
then the front umbra has grown into the back umbra.
}

\item return (T)
\end{enumerate}

Note that it is possible for the front umbra to grow into the back umbra
(Figure~\ref{fig:globalback}a).
It is simplest to allow overlap between the front and back umbra in this case,
rather than computing the exact merging of the front and back umbra.
% After all, we already have overlap of umbrae in our divide-and-conquer approach,


% There is no need to worry about merging front and back umbra.
% If we sweep to an inner bitangent of A and B (i.e., the inner piercing sweep
% has swept past A), then mark this component of the front umbra as special
% and one of two cases occurs:
% \begin{itemize}
% \item
% 0) 
% if the outer piercing sweep stops (completely) at a piercing bitangent, 
% it will cut off the front umbra before it merges with the back umbra,
% and we keep the umbra component like a normal umbra component**;
% % if the global back umbra was not refined, then B must not intersect an outer
% % bitangent of A and L and the outer piercing sweep must stop before it reaches the
% % outer bitangent of A and L, and in particular, must stop at a piercing position,
% % so it trims the front umbra component before it merges with the back umbra;
%
% \item
% 1) if the outer piercing sweep stops at a non-piercing bitangent,
% the front umbra merges with the back umbra and care must be taken.
% (Perhaps this component should be ignored or a new boundary added.)
% % In this case, an obstacle must block the outer bitangent of A and L
% % (otherwise the outer piercing sweep would have stopped before it reached this
% % outer bitangent, necessarily at a piercing bitangent).
% % Thus, the back umbra will have been refined nontrivially from its local back umbra.

% % If B intersects an outer bitangent of A and L,
% % then the refinement of the global back umbra will have swept to merge with the front
% % umbra and we can ignore the present front umbra component.
% \end{itemize}

\Comment{
Unnecessary addition to inner piercing sweep:
In some situations, L will reappear at an inner bitangent $\cal{I}$ of A and B 
(Figures~\ref{fig:cornell1sceneD} and \ref{fig:cornell1sceneE}), 
rather than an inner piercing bitangent.
This signals that the global front umbra grows into the global back umbra.
% In this case, the global front umbra merges with the global back umbra
% and $\cal{I}$ becomes a boundary of the back umbra.
% (instead of the front umbra),
The global back umbra may have already absorbed the front umbra
(Figure~\ref{fig:cornell1sceneD}) 
if an inner sweep reached the same inner bitangent $\cal{I}$ during the creation of the
global back umbra (Section~\ref{sec:innersweep}),
and we are done.
However, in other cases work is still necessary (Figure~\ref{fig:cornell1sceneE})
and the bitangent $\cal{I}$ must replace a boundary of the present global back umbra.

Since there is more than one boundary bitangent
% 2, 3, or 4, depending on the success of inner and outer sweeps
% bitangents bounding the back umbra, 
we must be careful to replace the correct one.\footnote{The 
    bitangent to be replaced can be of various flavors.
    For example, in Figure~\ref{fig:cornell1sceneE}, it is an outer bitangent of A and L,
    whereas in Figure~\ref{fig:cornell1sceneH}, it is an inner bitangent of A and C.}
Each of the bitangents bounding the back umbra 
was refined from an outer bitangent of A and L,
so choosing the correct bitangent is equivalent 
to choosing the correct outer bitangent from which it was refined.
% so our replacement dilemma can be reduced to choosing the correct outer bitangent of 
% A and L.
This is done as follows.
Let T be the outer piercing bitangent that initiated the inner piercing sweep,
% The present inner piercing sweep began at an outer piercing bitangent T of A and L, 
and suppose it is tangent to A at $P_A$.
We choose the outer bitangent of A and L that is reached first when sweeping L's tangent,
starting from T and locally moving away from A's inside at $P_A$.
The bitangent refined from this outer bitangent {\bf in inner or outer sweep?}
is the one that is replaced by $\cal{I}$ (Figure~\ref{fig:cornell1sceneH}.)
After all, we want to adjust the umbra on the same side of A as $\cal{I}$.
By sweeping a tangent of L from the starting point of the inner piercing sweep of 
this side, we shall reach the starting point of the inner sweep on this side.
The associated boundary is the one that should be replaced.
}

% \centerline{oldouterPiercingSweep(T,A)}
%
% \begin{enumerate}
% \item G = innerPiercingSweep (T,A)
% \item if G = $\emptyset$, return [nothing to clip]
% % \item sweep direction = away from A
% \item while an object B interferes with the segment of T between L and G
% % recall that this means the segment between A and L
% \begin{enumerate}
% \item S = \{outer bitangents of B and L\} $\cup$ % to stop once you see past B
%           \{outer bitangents of A and L\}        % to stop sweeping past the light
% \item T = closest bitangent of S from the present T, in sweeping on L away from A
% \item [new] 
%       if T intersects A, ---
% \item [new]
%       if T does not intersect A (after it leaves B), may have merged with back umbra
% \item if T is an outer bitangent of A and L, return ($\emptyset$) [we have swept past A: this condition implies the next one, but is easier to test; if we are willing to incur the more expensive intersection test,
% we could remove outer bitangents of A and L from the candidate set]
% \item if T has swept past G (i.e., T does not intersect the bridge segment of G),
%       return ($\emptyset$)
% \end{enumerate}
% \item return (T)
% \end{enumerate}

% ---------------------------------------------------------------------------------

% \clearpage

% \subsection{Defining the global front umbra}

The global front umbra of A is defined by the bitangents returned by the inner 
and outer piercing sweeps.
Each outer piercing bitangent T of A and L generates a component
of the global front umbra of A.
% unless the inner piercing sweep swept this bitangent until it merged with the back umbra.
The component is bounded by innerpiercingsweep(T,A),
outerpiercingsweep(T,A), and the relevant curve segment of A.

% \begin{lemma}
% Let the local front umbra of A be bounded by the set S of outer piercing bitangents.
% The global front umbra of A is the region bounded by the bitangents 
% $\{\mbox{piercing-sweep}(T,A): T \in S\}$.
% Once again, the bridge segments of these bitangents are used.
% \end{lemma}

\Comment{
Consider the point P in Figure~\ref{vg900newest, P in the seam between the back penumbra and the global front umbra},
which is outside all of the shadow categories for A (global back umbra,
global back penumbra, global front umbra, global front penumbra).
It is clearly in the umbra of the scene.
However, we do not consider it to be in the umbra of A, since A has no effect
on the fact that shadow is cast upon P.
It is classified only in the umbra of B.
The umbra of the scene is the union of the umbra of the objects,
and we want to keep track of which objects have caused each point in
the umbra or penumbra.
This will allow a quick recalculation of shadow if an object is removed.
}

% There is no need for a second sweep, because the umbra is now surrounded on the back side and
% cannot be gotten at in any other way.

% \clearpage

\ifJournal
\section{Incremental updates}

See third advantage of object-by-object computation in section on Local-global umbrae.
Discuss how to update the umbra when an object is removed.

\section{A comparison with local analysis}

We analyze visibility
from a global perspective rather than from one local point 
(the present position of the camera).
Consider a solution using the latter strategy. ---

\fi

\Comment{
\section{Examples}

Consider some more elaborate examples of the umbra 
\ifJournal
and penumbra 
\fi
cast by a single object.
In Figure~\ref{fig:whiteoak}, the object A is a white oak leaf.
\ifSurrounding
In Figure~\ref{fig:cornell2}, A is the footprint of a building on the Cornell campus designed by Richard
Meier.
Notice that A surrounds L.
\fi
In Figure~\ref{fig:cornell1and2}, L and A are the footprints of neighboring 
buildings on the Cornell campus designed by Richard Meier. 
This illustrates the use of this type of visibility analysis in walkthroughs of an urban landscape.

A 2.5D cityscape is studied, where each object is defined by the extrusion of a smooth curve (each object
is a building) and movement is at street level (as a pedestrian).
This allows the examination of this challenging problem in a manageable yet practical environment
% The study will be of a scene in 2-space where the objects are defined by smooth curves 
(Figure~\ref{fig:fancyopeningexample}).

Test data: 2d cityscape.
% abstract closed curves standing in for actors and booms.
An example application would be the planning of a path for a person 
(moving in the street plane) amongst skyscrapers such that a certain building is
always in full view.

SHOW EXTRUDED VERSION OF FIGURE 1'S CITYSCAPE:
scene ../data/scene/buildingSceneClean3d.
}

% --------------------------------------------------------------------------------------

\section{The advantages of a divide-and-conquer approach}
\label{sec:divide}

The divide-and-conquer approach developed by this paper
approaches visibility analysis from a different perspective,
considering the objects in a scene one at a time
% adding objects {\bf incrementally} 
and determining how they block more and more of the distinguished object L.
% This paper has developed a divide-and-conquer approach to visibility analysis,
% for the computation of visibility in a smooth scene.
%
    % However, the problem again reduces to the computation of extremal visibility events,
    % which in this smooth case are bitangents.
This divide-and-conquer approach has several benefits, 
including incremental computation, parallelizability,
and global understanding of visibility,
which we now consider.

% We use a divide-and-conquer approach to compute the effect of
% every object to the distinguished object's invisibility.

% ---------------------------------------------------------------------------------
% incremental

The algorithm,
in which each object defines the region of invisibility for which it is responsible,
allows efficient incremental updates.
Consider the addition of an object A to the scene.
To recompute invisibility, the invisibility region defined by the new object A
must be computed and A must be incorporated into other objects' invisibility regions.
However, A can only affect the invisibility region of an object B if it lies 
between B and L and within the outer bitangents of B and L,
which greatly limits the number of candidates.
For example, if A lies on the outskirts of the scene, no other object B's region 
must be updated; or if A lies close to L in the interior of the scene, only objects B
that lie in the swath behind A must be updated.
The removal of an object has similar bounded effects.

% SHOW how we can assign obstacles to regions (SEE SOFTWARE)?

% ---------------------------------------------------------------------------------
% parallelization
The algorithm is inherently parallelizable.
The local and global umbra of each object $A_i$ may be computed 
in parallel on a different processor $P_i$.
We can even parallelize the construction of the front and back umbra
within this computation.
The bitangents of each pair of objects in the scene of $n$ objects
may be precomputed in parallel on $n^2$ processors.
Given the complexity of visibility analysis,
visibility algorithms that allow a parallel treatment are particularly attractive.
The issues of parallel computation of visibility warrant a deeper study,
which we are undertaking.

% ---------------------------------------------------------------------------------
% global understanding
Finally, the divide-and-conquer algorithm yields a deeper understanding
of the global structure of the invisibility,
% The structure of invisibility is understood at a global level,
even yielding information about which objects contribute to the blocking of each 
invisible point, a form of quantitative invisibility.

\section{Conclusions}
\label{sec:conclusions}

% Given a scene of smooth objects containing a distinguished smooth object L,
This paper has addressed an important problem in visibility analysis:
how to compute the region where L is invisible,
where L is a smooth object embedded in a scene of smooth objects.
    % Previous work has identified visibility computation with
    % the computation of extremal visibility, usually extremal lines that see the light.
Each object A is responsible for the definition of a region of invisibility,
the local region that it impacts.
The invisibility of the scene is simply the union of the invisibility cast by each object.
The invisibility region cast by A is computed in two stages.
First, it is computed as if A was the sole object in the scene.
Next, it is refined to incorporate the context of neighbouring
objects using four sweeps, two in front and two in back.
The first sweep may expand the invisibility cast by A due to objects that
lie between A and L, and is computed by sweeping a tangent of A.
The second sweep may shrink the invisibility cast by A and is computed by
sweeping a tangent of L or an interfering object.
All sweeps are discretized by observing that visual events always occur at bitangents.


% ---------------------------------------------------------------------------------
% smoothness
We have already studied the advantages of the divide-and-conquer approach
introduced by this paper (Section~\ref{sec:divide}).
The other contribution of this paper to visibility analysis is the
algorithmic treatment of smooth scenes.
The analysis of smooth scenes offers great promise, despite its challenges,
because of the large number of existing smooth models, which can be handled directly,
the reduction in cardinality of the primitives in the scene
when compared with polyhedral meshes,
and the new approaches that it opens up, such as sweeping.

% ---------------------------------------------------------------------------------
% future work
There are several directions for future work.
The extension of this work to partial visibility (penumbra) is being studied.
The analysis of smooth 3D scenes has been a goal from the beginning.
The structure of this analysis is now much clearer, learning from the divide-and-conquer
approach and structured filtering of the 2d analysis.
Moreover, the 2d analysis can be incorporated directly into the 3d analysis.
We are interested in the application of the analysis to movement in cityscapes
(Figure~\ref{fig:cityscape}).
% Visual events become surfaces and sweeping is done in ---.
% 3D (sweeping, bitangent developables).
%
% The robustness of computing the special bitangents that define visual events 
% (inner and outer bitangents, inner and outer piercing bitangents) can be improved.
%
We are also interested in comparing smooth solutions to visibility analysis with 
linear solutions,
evaluating the tradeoff between the reduced number
of primitives in smooth scenes and the added complexity of each operation on
the primitive.
A deeper analysis of the incremental behavior of the algorithm is also necessary.

% Much promise.

\begin{figure}
\begin{center}
\includegraphics*[scale=.15]{img2/cityscape2.jpg}
\includegraphics*[scale=.15]{img2/cityscape4.jpg}
\end{center}
\caption{Visibility analysis in flatland can naturally be applied to visibility
         analysis of a cityscape at sidewalk level}
\label{fig:cityscape}
\end{figure}

% ---------------------------------------------------------------------------------
% general algorithm

\Comment{
The region of invisibility carved out by each object is found by an active and organized
search for visual events.
Potential visual events (inner/outer bitangents and inner/outer piercing bitangents)
are filtered and organized into invisibility regions by sweeping.

The invisibility region defined by the object A is bounded by a collection of bitangents
between A, L, and intervening objects of the scene.
The definition of the umbra is a search for the bitangents that bound the umbra.
}


% \section{Computing bitangents}
%
% Bitangents are best computed in dual space by the intersection of line curves
% \cite{farinProj99,pottmannLineBook}.
% With a line curve, a curve is represented by its tangents.
% By interpreting the control lines (which are vectors) of a line curve
% as control points, another curve is generated dual to the original curve.
% The Bezier representation of this dual representation of a Bezier curve 
% is given in \cite{johnstoneShapeModeling}.
% The intersection points of these curves in dual space correspond to the bitangents
% of the original curves.
%
% \clearpage

\ifJournal
\section{Point location within a visibility zone}

{\em For purposes of point location}, the back umbra of A can be defined 
as a polygon as follows.
Begin with the curve segment of A that lies between the two points of
bitangency and lies on the other side from L.
Walk from A's point of tangency with $T_1$ to the intersection
of the ray defined by $T_1$ with the room, around the room to 
the intersection of the ray defined by $T_2$ with the room, and back
to the point of tangency with $T_2$,
where $T_1$ and $T_2$ are the two outer bitangents of A and L.

You can build the back umbra polygon by walking from the room
point of an outer bitangent (umbraExtreme) to its point of tangency with A,
around on a curve segment to the point of tangency of the other outer bitangent,
to its room point (umbraExtreme) and around the room polygon until the polygon
is closed.

% Our original definition, which does not generalize to surrounding objects.
% If A does not surround L, begin by intersecting the inner halfspaces
% of the outer bitangents of A and L with the room polygon.
% Then intersect this polygon with the halfspace bounded 
% by the chord of A defined by the two points of tangency of
% the outer bitangents, using the side away from L.
% Finally, replace the chord segment by the curve segment between the two
% points of tangency (the one that lies away from L).

% \clearpage
\fi

% ------------------------------------------------------------------------------------

\bibliographystyle{latex8}
% \bibliographystyle{plain}
\begin{thebibliography}{99}

\bibitem{assarsson03}    % interactive shadows for meshes: one way to push shadows
Assarsson, U. and T. Akenine-M\"{o}ller (2003)
A Geometry-based Soft Shadow Volume Algorithm using Graphics Hardware.
SIGGRAPH '03, 511--520.

\bibitem{baciu00}
Baciu, G. and C. Yu (2000)
Visibility in the Presence of Occluders with Curved Boundaries.
Pacific Graphics 2000, 105--114.

\bibitem{bala03} % also looking for interactive shadow edges for meshes
Bala, K., B. Walter and D. Greenberg (2003)
Combining Edges and Points for Interactive High-Quality Rendering.
SIGGRAPH '03, 631--640.

\bibitem{chen91}
Chen, S. and H. Freeman (1991)
On the Characteristic Views of Quadric-Surfaced Solids.
IEEE Workshop on Directions in Automated CAD-Based Vision, 34--43.

\bibitem{drettakis94}  % with Fiume on discontinuity mesh
Drettakis, G. and E. Fiume (1994)
A Fast Shadow Algorithm for Area Light Sources Using Backprojection.
SIGGRAPH '94, 223--230.

\bibitem{durand96}
Durand, F. and G. Drettakis and C. Puech (1996)
The 3D visibility complex: a new approach to the problems of accurate
visibility.
7th Eurographics Workshop on Rendering, Portugal.

\bibitem{durand97a}
Durand, F. and G. Drettakis and C. Puech (1997)
The Visibility Skeleton: A Powerful and Efficient Multi-Purpose Global
Visibility Tool.
SIGGRAPH '97, 89--100.

\bibitem{durand97b}
Durand, F. and G. Drettakis and C. Puech (1997)
The 3D Visibility Complex: a unified data structure for global
visibility of scenes of polygons and smooth objects.
9th Canadian Conference on Computational Geometry.

% \bibitem[Durand 97]{durand97}
% Durand, F. and G. Drettakis and C. Puech (1997)
% 3D Visibility made visibly simple: an introduction to the Visibility
% Skeleton.
% Video Proceedings of ACM Symposium on Computational Geometry, Nice.

\bibitem{durand00a}
Durand, F. (2000)
A Multidisciplinary Survey of Visibility.
SIGGRAPH 2000 Course Notes on Visibility: Problems, Techniques
and Applications.
Also available from http://graphics.lcs.mit.edu/~fredo/.

\bibitem{durand00b}
Durand, F. and G. Drettakis and J. Thollot and C. Puech (2000)
Conservative Visibility Preprocessing using Extended Projections.
SIGGRAPH '00, 239--248.

% \bibitem[Orti 96]{orti96}
% Orti, R. and F. Durand and S. Rivi$\grave{e}$re and C. Puech (1996)
% Using the visibility complex for radiosity computation.
% ACM Workshop on Applied Computational Geometry, Philadelphia.

% another Orti paper: see cv.gb.pdf radiosity for dynamic scenes (bottom of page)

% \bibitem[Rivi$\grave{e}$re 95]{riviere95}
% Rivi$\grave{e}$re, S. (1995)
% Topologically sweeping the visibility complex of polygonal scenes.
% 11th ACM Symposium on Computational Geometry, Vancouver, C36--37.

% \bibitem[Rivi$\grave{e}$re 97]{riviere97}
% Rivi$\grave{e}$re, S. (1997)
% Dynamic visibility in polygonal scenes with the visibility complex.
% ACM Symposium on Computational Geometry, Nice, 421--423.

\bibitem{eggert93}
Eggert, D. and K. Bowyer (1993)
Computing the Perspective Projection Aspect Graph of Solids of Revolution.
IEEE Transactions on Pattern Analysis and Machine Intelligence 15(2), 109--127.

\bibitem{elber90}
Elber, G. and E. Cohen (1990)
Hidden Curve Removal for Free Form Surfaces.
SIGGRAPH '90, 95--104.

% \bibitem{ghali}
% Ghali, S.
% Computation and Maintenance of Visibility and Shadows in the Plane.
% Technical Report.

\bibitem{gigus90}
Gigus, Z. and J. Malik (1990)
Computing the Aspect Graph for Line Drawings of Polyhedral Objects.
IEEE Transactions on Pattern Analysis and Machine Intelligence 12(2), 113--122.

\bibitem{unc03}   % hard shadows for huge models: another way to push shadows
Govindaraju, N. et. al. (2003)
Interactive Shadow Generation in Complex Environments.
SIGGRAPH '03, 501--510.

\bibitem{hart99}   % see p. 153: call for work on visibility on nonpolygonal surfaces
Hart, D., P. Dutr\'{e} and D. Greenberg (1999)
Direct Illumination with Lazy Visibility Evaluation.
SIGGRAPH '99, 147--154.

\bibitem{heckbert92a}
Heckbert, P. (1992)
Radiosity in Flatland.
Eurographics '92, Computer Graphics Forum 11(3), 181--192.

% \bibitem{heckbert92b}
% Heckbert, P. (1992)
% Discontinuity Meshing for Radiosity.
% Third Eurographics Workshop on Rendering, 203--216.

\bibitem{koenderink76}
Koenderink, J. and A. van Doorn (1976)
The Singularities of the Visual Mapping.
Biological Cybernetics 24, 51--59.

% \bibitem[Koenderink 90]{koenderink}
% Koenderink, J. (1990)
% Solid Shape.
% MIT Press (Cambridge, MA).

\bibitem{leyvand03}
Leyvand, T., O. Sorkine and D. Cohen-Or (2003)
Ray Space Factorization for From-Region Visibility.
SIGGRAPH '03, 595--604.

\bibitem{lischinski92}
Lischinski, D., F. Tampieri and D. Greenberg (1992)
Discontinuity Meshing for Radiosity.
IEEE Computer Graphics and Applications 12(6), 25--39.

\bibitem{petitjean92}
Petitjean, S. and J. Ponce and D. Kriegman (1992)
Computing Exact Aspect Graphs of Curved Objects: Algebraic Surfaces.
International Journal of Computer Vision 9(3), 231--255.

% \bibitem{petitjean96}
% Petitjean, S. (1996)
% The Enumerative Geometry of Projective Algebraic Surfaces and
% the Complexity of Aspect Graphs.
% International Journal of Computer Vision 19(3), 1--28.

\bibitem{pocchiola96}
Pocchiola, M. and G. Vegter (1996)
The Visibility Complex.
International Journal of Computational Geometry and Application 6(3),
279--308.

\bibitem{ponce90}
Ponce, J. and D. Kriegman (1990)
Computing Exact Aspect Graphs of Curved Objects: Parametric Surfaces.
Proc. of AAAI-90, 1074--1079.

\bibitem{sen03}          % also looking for interactive speeds
Sen, P., M. Cammarano and P. Hanrahan (2003)
Shadow Silhouette Maps.
SIGGRAPH '03, 521--526.

% \bibitem[Sripradisvarakul 89]{sri89}
% Sripradisvarakul, T. and R. Jain (1989)
% Generating Aspect Graphs for Curved Objects.
% Workshop on Interpretation of 3D Scenes, 109--115.

\bibitem{stewart94}
Stewart, A. J. and S. Ghali (1994)
Fast Computation of Shadow Boundaries Using Spatial Coherence and Backprojections.
SIGGRAPH '94, 231--238.

\bibitem{teller91}
Teller, S. and C. Sequin (1991)
Visibility Preprocessing for Interactive Walkthroughs.
SIGGRAPH '91, 61--67.

\bibitem[Teller 92]{teller92}
Teller, S. (1992)
Computing the Antipenumbra of an Area Light Source.
SIGGRAPH '92, 139--148.

\bibitem{teller93}
Teller, S. and P. Hanrahan (1993)
Global Visibility Algorithms for Illumination Computations.
SIGGRAPH '93, 239--246.

\bibitem{williams78}
Williams, L. (1978)
Casting Curved Shadows on Curved Surfaces.
SIGGRAPH '78, 270--274.

\bibitem{wooReview}
Woo, A., P. Poulin and A. Fournier (1990)
A Survey of Shadow Algorithms.
IEEE Computer Graphics and Applications 10(6), 13--32.

% \bibitem{jjFullJournalSmoothVisibilityInFlatland}
% Johnstone, J. (2004)
% Smooth visibility in flatland, in preparation.

\end{thebibliography}

% \clearpage

\ifJournal
	% umbra -r 1.2 data/umbra/woak.pts
\begin{figure*}
\begin{center}
\includegraphics*[scale=.15]{img2/woak.jpg}
\includegraphics*[scale=.15]{img2/woakfu.jpg}
\includegraphics*[scale=.15]{img2/woakbu.jpg}
\ifJournal
   \includegraphics*[scale=.25]{img2/woakfp.jpg}
   \includegraphics*[scale=.25]{img2/woakbp.jpg}
\fi
\end{center}
\caption{Analyzing the visibility of L.
        (a) The scene A and L.
        (b) The front umbra of A.
        (c) The back umbra of A.
\ifJournal
        (d) The front penumbra of A.
        (e) The back penumbra of A.
\fi
         }
\label{fig:whiteoak}
\end{figure*}

	% umbra -r 1.2 data/umbra/woak.pts
\begin{figure}
\begin{center}
\includegraphics*[scale=.15]{img2/woakVisibility.jpg}
\end{center}
\caption{Analyzing the visibility of an object (a) 2.5D version (b) 2D version.}
\label{fig:fancyopeningexample}
\end{figure}

	% umbra -S -r 1.2 data/scene/lightSceneHud
\begin{figure}
\begin{center}
\includegraphics*[scale=.15]{img2/hud.jpg}
\includegraphics*[scale=.15]{img2/hudbu.jpg}
\includegraphics*[scale=.15]{img2/hudfu.jpg}
\includegraphics*[scale=.15]{img2/hudbp.jpg} 
\includegraphics*[scale=.15]{img2/hudfp.jpg}
\end{center}
\caption{Analyzing the invisibility of L.
         A is the footprint of Marcel Breuer's HUD building.
        (a) The scene A and L 
        (b) The back umbra of A.
        (c) The front umbra of A.
        (d) The back penumbra of A.
        (e) The front penumbra of A.
         }
\label{fig:hud}
\end{figure}
\fi

\ifJournal
\begin{figure}
\begin{center}
% \includegraphics*[scale=.15]{img2/--.jpg}
\end{center}
\caption{An inner piercing bitangent of A and B}
\label{fig:innerpiercing}
\end{figure}
\fi

\ifSurrounding
	% umbra -w data/umbra/vg10new.pts > foobar &
\begin{figure}
\begin{center}
\includegraphics*[scale=.15]{img2/vg10newpiercing.jpg}
\includegraphics*[scale=.15]{img2/vg10newouterpiercing.jpg}
\includegraphics*[scale=.15]{img2/vg10newfront.jpg}
\end{center}
\caption{Defining the front umbra of A when A surrounds L. 
        (a) The piercing bitangents of A and L.
        (b) The outer piercing bitangents. 
        (c) The front umbra of A.}
\label{fig:frontsurrounding}
\end{figure}
\fi

\ifSurrounding
	% umbra -S -w -r 1.5 data/scene/lightSceneCornell2
\begin{figure}
\begin{center}
\includegraphics*[scale=.15]{img2/cornell2.jpg}
\includegraphics*[scale=.15]{img2/cornell2fu.jpg}
\includegraphics*[scale=.15]{img2/cornell2bu.jpg}
   \includegraphics*[scale=.15]{img2/cornell2fp.jpg}
   \includegraphics*[scale=.15]{img2/cornell2bp.jpg} 
\end{center}
\caption{Analyzing the visibility of L.
        (a) The scene A and L.
        (b) The front umbra of A.
        (c) The back umbra of A.
        (d) The front penumbra of A.
        (e) The back penumbra of A.
          }
\label{fig:cornell2}
\end{figure}
\fi

\ifJournal
	% umbra -S -r 1.2 data/scene/buildingSceneCornell1
\begin{figure}
\begin{center}
\includegraphics*[scale=.15]{img2/cornell1and2.jpg}
\includegraphics*[scale=.15]{img2/cornell1and2bu.jpg}
\ifJournal
   \includegraphics*[scale=.15]{img2/cornell1and2fp.jpg}
   \includegraphics*[scale=.15]{img2/cornell1and2bp.jpg} 
\fi
\end{center}
\caption{Analyzing the visibility of L.
        (a) The scene A and L.
        (b) The back umbra of A.
\ifJournal
        (c) The front penumbra of A.
        (d) The back penumbra of A.
\fi
         }
\label{fig:cornell1and2}
\end{figure}
\fi

\ifJournal
	% umbra data/umbra/globalumb0.pts > foobar &
	% umbra data/umbra/globalumb2.pts > foobar &
\begin{figure}
\begin{center}
\includegraphics*[scale=.15]{img2/gu0global.jpg}
\includegraphics*[scale=.15]{img2/gu2global.jpg}
\includegraphics*[scale=.15]{img2/gu2globalall.jpg}
\end{center}
\caption{A simple example of the effect of another object on an object's umbra. (a) The local umbra of A. (b) The global umbra of A. (c) The global umbra of all objects in the scene.}
\label{fig:localvsglobalpure}
\end{figure}
\fi

\ifJournal
	% umbra data/umbra/globalumbmax.pts
\begin{figure}
\begin{center}
\includegraphics*[scale=.15]{img2/gumax.jpg}
\end{center}
\caption{The maximal size of the global back umbra is the back penumbra}
\label{fig:maxumbra}
\end{figure}

% GIVE AN EXAMPLE OF SWEEPING PAST G.
\fi





\ifJournal
        % umbra -c .05 -f .3 ../data/cornell1sceneD.pts
\begin{figure}
\begin{center}
\includegraphics*[scale=.15]{img2/cornell1sceneDinner.jpg}
\includegraphics*[scale=.15]{img2/cornell1sceneDgbu.jpg}
\end{center}
\caption{(a) The local front umbra of A, with inner bitangents of the scene.
         (b) The global back umbra of A before the inner piercing sweep, already merged
             with the front umbra.}
\label{fig:cornell1sceneD}
\end{figure}
\fi

\ifJournal
        % umbra -c .05 -f .3 ../data/cornell1sceneE.pts
\begin{figure}
\begin{center}
\includegraphics*[scale=.15]{img2/cornell1sceneEinner.jpg}
\includegraphics*[scale=.15]{img2/cornell1sceneElbu.jpg}
\end{center}
\caption{(a) The local front umbra of A, with inner bitangents of the scene.
         (b) The global back umbra of A before the inner piercing sweep, not yet merged
             with the front umbra.
         (c) The global back umbra of A after the inner piercing sweep.
             Notice the merging of the front and back umbra.
             NOTICE THAT OUTER PIERCING SWEEP NECESSARILY CLIPS IT.}
\label{fig:cornell1sceneE}
\end{figure}

        % umbra -P 11 -c .05 -f .5 ../data/cornell1sceneH.pts
\begin{figure}
\begin{center}
\includegraphics*[scale=.15]{img2/cornell1sceneHinner.jpg}
\includegraphics*[scale=.15]{img2/cornell1sceneHlbu.jpg}
\end{center}
\caption{INTERESTING EXAMPLE.
         (a) The local front umbra of A, with inner bitangents of the scene.
         (b) The local back umbra of A.
         (b) The global back umbra of A before the inner piercing sweep, not yet merged
             with the front umbra.
         (c) The global back umbra of A after the inner piercing sweep.
             Notice the merging of the front and back umbra.}
\label{fig:cornell1sceneH}
\end{figure}
\fi

\ifJournal
	% umbra data/umbra/globalumb2.pts > foobar &
	% umbra data/umbra/vg900new.pts > foobar &
\begin{figure}
\begin{center}
\includegraphics*[scale=.15]{img2/gu2umbra.jpg}
\end{center}
\caption{This companion paper develops the theory of this image}
% \label{}
\end{figure}

        % umbra data/umbra/vg9.pts > foobar &
	% umbra data/umbra/vg900.pts > foobar &
\begin{figure}
\begin{center}
\includegraphics*[scale=.15]{img2/vg9global.jpg}
\includegraphics*[scale=.15]{img2/vg900global.jpg}
\includegraphics*[scale=.15]{img2/vg900globalall.jpg}
\end{center}
\caption{ANOTHER EXAMPLE (POSSIBLY USEFUL): The addition of objects can affect the global umbra: (a) The local umbra of A. (b) The global umbra of A. (c) The global umbra of all objects in the scene.}
\label{fig:globalvg900}
\end{figure}
\fi

\ifJournal
\clearpage

\subsection{The local penumbra}

The computation of the penumbra cast by A---the region that sees some but not all of L---
is similar to the computation of the umbra.
The difference is that bitangents are chosen that separate A and L,
instead of bitangents that define a halfspace containing A and L.
Consider the back penumbra of A.

\begin{defn2}
An extremal bitangent of A and L is {\bf inner} if
\begin{itemize}
\item A and L lie on opposite sides of T in the neighbourhood of A's point of bitangency $P_A$.
\end{itemize}
That is, T defines a separating halfplane (Figure~\ref{fig:backpenumbra}a).
\end{defn2}

{\bf Local back penumbra.}
Unless A intersects the convex hull of L (in which case the back penumbra is empty),
A and L will have two inner bitangents.
Consider the region bounded by the late segments of the two
inner bitangents of A and L (Figure~\ref{fig:backpenumbra}b).
% too early to introduce this idea: This is the maximal extent of the back umbra of A when other obstacles are added to the scene.
The back penumbra of A is this region minus the front and back umbra of A (Figure~\ref{fig:backpenumbra}c).
This construction of the back penumbra works for surrounding objects as well 
(Figure~\ref{fig:backpenumbrasurrounding}).

	% umbra data/umbra/umbra1.pts > foobar &
\begin{figure}
\begin{center}
\includegraphics*[scale=.15]{img2/umbra1inner.jpg}
\includegraphics*[scale=.15]{img2/umbra1maxumbra.jpg}
\includegraphics*[scale=.15]{img2/umbra1bp.jpg}
\end{center}
\caption{Defining the back penumbra. 
  (a) The inner bitangents of A and L.
  (b) The region between the inner bitangents.
  (c) The back penumbra of A.}
\label{fig:backpenumbra}
\end{figure}

	% umbra -w data/umbra/vg10new.pts > foobar &
\begin{figure}
\begin{center}
\includegraphics*[scale=.15]{img2/vg10newinner.jpg}
\includegraphics*[scale=.15]{img2/vg10newmaxumbra.jpg}
\includegraphics*[scale=.15]{img2/vg10newbp.jpg}
\end{center}
\caption{Defining the back penumbra when A surrounds L.
  (a) The inner bitangents of A and L.
  (b) The region between the inner bitangents.
  (c) The back penumbra of A.}
\label{fig:backpenumbrasurrounding}
\end{figure}

{\bf Local front penumbra.}
Analogously, the front penumbra is defined by replacing the front umbra's outer piercing bitangent by
an inner piercing bitangent that separates A and L.
It is found by sweeping in the tangent space of A towards L.
If no inner piercing bitangent is found before sweeping across L
(i.e., before reaching an inner bitangent of A and L),
then the front penumbra associated with this component disappears,
merging with the back penumbra (consider cornell1sceneB.pts).

\begin{defn2}
A piercing bitangent of A and L is {\bf inner} (Figure~\ref{fig:frontpenumbra}a) if
\begin{itemize}
\item A and L lie on opposite sides of T in the neighbourhood of A's point of bitangency $P_A$.
\end{itemize}
\end{defn2}

\subsection{There is no need for a global penumbra}

{\bf Global back penumbra}

Hypothesis: the penumbra cannot be expanded by other objects, since it is already
maximal.  Thus, there is no global penumbra (or local penumbra), just a penumbra.
Yes: consider a point P outside the back penumbra.
A can have no effect on or responsibility for blocking P from L.
Under this definition (penumbra of A is region between inner bitangents),
the penumbra of the scene is already the union of the penumbrae of each object A.
There is no need to expand.

Now consider the need for a global front penumbra.
Again, there is no need.
This case has a second argument.
Suppose that you do try to compute a global version of the front penumbra, 
by sweeping past an object B that interferes with the inner piercing bitangent.
Then you see past B once you reach an inner bitangent of B and L;
but this inner bitangent bounds the penumbra of B
so the additional region that we are contemplating adding to the penumbra of A
is already captured by B's penumbra.
The more fundamental reason is, however, again that A does none of the blocking
of L for any point outside the front and back penumbra.
This relies on the same argument that we used in stopping the sweep at
an inner bitangent as we sweep past L (Section~\ref{}).

See example figures in Bellsouth notebook, page 3.

{\bf end of global back penumbra}

% START HERE

Each inner piercing bitangent defines a component of the front penumbra:
the closed area bounded by the bridge segment of this bitangent,
minus the associated component of the front umbra (Figure~\ref{fig:frontpenumbra}c and 
\ref{fig:frontpenumbrasurrounding}c).

	% umbra data/umbra/umbra1.pts > foobar &
\begin{figure}
\begin{center}
\includegraphics*[scale=.15]{img2/umbra1innerpiercing.jpg}
\includegraphics*[scale=.15]{img2/umbra1maxfront.jpg}
\includegraphics*[scale=.15]{img2/umbra1fp.jpg}
\end{center}
\caption{Defining the front penumbra. 
         (a) The inner piercing bitangents.
         (b) The region bounded by the late inner bitangents.
         (c) The front penumbra of A.}
\label{fig:frontpenumbra}
\end{figure}

	% umbra -w data/umbra/vg10new.pts > foobar &
\begin{figure}
\begin{center}
\includegraphics*[scale=.15]{img2/vg10newinnerpiercing.jpg}
\includegraphics*[scale=.15]{img2/vg10newmaxfront.jpg}
\includegraphics*[scale=.15]{img2/vg10newfp.jpg}
\end{center}
\caption{Defining the front penumbra when A surrounds L. 
         (a) The inner piercing bitangents.
         (b) The region bounded by the late inner bitangents.
         (c) The front penumbra of A.}
\label{fig:frontpenumbrasurrounding}
\end{figure}

\clearpage

\section{Incremental computation: adding an object}

\section{Incremental computation: removing an object}

\section{Future work}

Once 3d visibility analysis is analyzed, we can study practical problems
such as tumor irradiation.

{\bf X-ray application}
X-ray interpretation of application 2:
For example, if the identified object is a tumor and the other objects in the scene
are surrounding organs and bones,
the visible zone encodes ideal locations for an X-ray source to radiate the tumor.
X-ray interpretation of application 3:
For example, if the identified object is a healthy organ
and the other objects in the scene are bones that block radiation,
the invisible zone is an ideal location for an X-ray source (since the organ will
be completely blocked from the harmful radiation of the X-ray).
Once a potential location for the X-ray source is found that sees the tumor (using the
tumor as identified object) and
is shielded from a healthy organ (using the healthy organ as identified object), 
the visibility of the X-ray source can be analyzed
with respect to the entire body (using the X-ray as identified object)
to verify that no important healthy structure is
visible to radiation.

{\bf Paint gun application}
Replacing the X-ray source with a paint spray gun yields another application
(e.g., for robotic spraying of a car).

% -----------------------------------------------------------------------------

\clearpage

\section{Introduction}

PRESENT THE MATERIAL INDEPENDENTLY FROM LIGHTING AT FIRST.

We are interested in the following analytic lighting problem:
the lighting by a smooth area light source 
of a room full of smooth objects.
Previous work on lighting has primarily concentrated on the simpler problem of
polyhedral objects and point or polygonal light sources.
{\bf The umbra cast by a smooth area light in a 2d scene
of smooth objects, where smooth means bounded by $C^1$-continuous curves,
is already a very interesting problem, and is the focus of this paper.}
In this paper, we compute the umbra cast in 2-space 
% or 2$\frac{1}{2}$-space: not really since angle looking up changes visibility
when the light and the objects % $\{B_i\}_{i=1}^n$ 
in the scene are bounded by smooth closed curves,
already a challenging problem that paves the way for an attack 
on the general lighting problem in 3-space.
% This is a very challenging problem, too broad to consider immediately 
% in its fullest generality.
% Problem: computing the umbra cast by a smooth area light source in a smooth 
% scene in 2-space or 2$\frac{1}{2}$-space.
% This is already an interesting and challenging problem.
The umbra of a scene is the area that does not see any of the light.
% collection of regions in the scene
Umbral computation can be interpreted
more generally as the computation of regions that are entirely invisible
to a distinguished object, an important visibility question.
For example, if the light is reinterpreted as the camera,
the umbra defines parts of the scene that do not need to be rendered.
If the light is reinterpreted as a robot, the umbra defines parts of the
scene that do not see the robot.
Visibility in 2-space is particularly important in robotics, as the robot
moves on a 2-dimensional floor.
% The theory developed in this paper also paves the way for an attack on the 
% general lighting problem in 3-space.

{\bf See 02Dagstuhl/img/umbra (and Dagstuhl talk slides) for coherent series of images illustrating 
the local vs. global umbra, in simple circle examples.}


	\begin{defn2}
	The {\bf umbra of a scene} is the collection of regions 
	in the scene that do not see any of the light.
	\end{defn2}

The main characteristic of this paper that distinguishes it from much work on
lighting is its concentration on a smooth light and smooth objects.
Its other distinguishing characteristic is that we are interested in 
defining the global structure of the umbra.
(THIS ISN'T DISTINGUISHING: OTHER WORK ON DISCONTINUITY MESH DOES THIS.)
That is, we are interested in defining the umbra region by region,
rather than point by point.
It is popular to define lighting point by point,
answering the question 'is point P in the umbra?'.
We are interested in a deeper understanding of the structure of the umbra,
through a direct definition of the area that defines the umbra.
This is more intellectually satisfying but certainly more challenging as well.

\ifTalk
This distinction between point and regional approaches is analogous to the 
two ways for computing a tangent on a Bezier curve:
computing the tangent at a point
or computing the hodograph of the curve, representing all of the tangents.)
\fi

% ------------------------------------------------------------------------------

\clearpage

% ------------------------------------------------------------------------------

\clearpage

\section{Taking umbrage} % A divide and conquer strategy

We will compute the umbra of a scene using a divide and conquer strategy,
one object at a time.
First, we compute the umbra cast by each object in isolation
(the local umbra).
We then refine and expand the umbra of each object
by taking other objects into account (the global umbra of an object).
Finally, the umbra of the scene is defined in terms of these object umbrae.

The computation of the umbra one object at a time has the advantage
of simplicity, parallelizability, and incremental computation.
The concentration on one object at a time is
an important simplification to the structure of the computation,
since it allows one object to (temporarily) dominate all of the other objects.
That is, in the computation of the umbra cast by an object A, 
all objects are of interest only through their relationship
to A or the light, rather than their mutual interrelationships.
This dominance of one object yields important simplifications when
reasoning about lighting.
Another benefit is parallelizability:
the computation of the umbra of an object A can be performed at the same
time as the computation of the umbra of an object B.
This can be an important factor in a large scene.
% This keeps less balls up in the air at once.
% Thus, we only need to consider n-to-1 relationships, not n-to-n relationships.
A third benefit of computing the umbra object by object
is that a point of the umbra will know which objects are
responsible for its inclusion in the umbra.\footnote{The umbra cast by 
	A and the umbra cast by B may certainly overlap.}
This will allow efficient incremental updates to the umbra upon the addition
or removal of an object.


SECTION REVIEW:

Section~\ref{sec:innersweep};
Section~\ref{sec:outersweep};
While the first sweep refines the local umbra bitangent,
the second sweep refines the global umbra bitangent
just computed in the first sweep.
% ------------------------------------------------------------------------------

\clearpage

\section{Lighting}

We will divide the umbra into two regions: the umbra behind the object
and the umbra in front of the object.

Consider a single object A.

\begin{defn2}
The {\bf umbra} of a scene is the region that sees none of the light.
The {\bf penumbra} of a scene is the region that sees some, but not all,
of the light.
The umbra will be computed one object at a time.
(The resulting regions may, and usually will, overlap.)
The umbra of each object will be divided into two regions, the {\bf front} 
and {\bf back umbra}, which will be computed separately.
Intuitively, the front umbra lies in a series of pockets between the 
object and the light, while the back umbra lies behind the object
(on the opposite side of the light).
More precisely, the front umbra is defined by piercing bitangents
while the back umbra is defined by inner and outer bitangents.
The exact definition of these areas is delayed.
The computation of a front or back umbra is further refined
by first computing it locally, then globally.
The {\bf local front umbra} of an object A is the 
front umbra of A if A was the only object in the scene.
The {\bf global front umbra} of A is the front umbra of A when the effect
of all objects in the scene is considered.
It is computed by refining the local front umbra.
Similarly for local and global back umbrae.
\end{defn2}

\begin{defn2}
The {\bf discontinuity mesh} of a scene S of objects is ---.
\end{defn2}

The discontinuity mesh of a scene S is solved when S is composed of polyhedra (Fiume, Stewart).
In this paper, we study the discontinuity mesh when S is composed of smooth curves.
A related result is the visibility preprocessing of a polygonal environment of Teller and Sequin.

% ----------------------------------------------------------------------------------

\clearpage

\section{A glossary of terms}

\begin{defn2}
\label{defn:surround}
A {\bf surrounds} B if B lies entirely inside the convex hull of A.
\end{defn2}

Since there is no umbra if the light surrounds the object, 
we can assume without loss of generality 
that the light does not surround the object.
If the object surrounds the light (Figure~\ref{fig:G2}c),
the theory is more complicated.
% As we develop the theory, we shall point out how it simplifies when
% the object does not surround the light,
% which is true in the vast majority of cases.
Since the object does not surround the light in the vast majority
of cases, we postpone this unusual case until Section~\ref{sec:surround}
and assume that the object does not surround the light.

\begin{defn2}
\label{defn:outer}
Let A and B be closed curves, where neither curve surrounds the other.
Let $t$ be a tangent of A,
and let $T$ be a bitangent of A and B with points of tangency $P_A \in A$ and 
$P_B \in B$.

\begin{itemize}
\item $t$ is {\bf extremal} if $t$ does not intersect A
	(i.e., A lies in one of the halfplanes defined by $t$).
\item If $t$ is extremal, the {\bf inside} of $t$ is
	the halfplane that contains A.
% \item Case 1: A does not surround B and B does not surround A.
% \begin{itemize}
\item $T$ is {\bf extremal} if $T$ does not intersect A or B.
\begin{itemize}
\item $T$ is {\bf outer} if it is extremal and 
A and B lie on the same side of $T$ ({\bf Figure G4}).
\item $T$ is {\bf inner} if it is extremal and
A and B lie on opposite sides of $T$.
\end{itemize}
% \end{itemize}
\item
The {\bf early segment} of T is the component of T between A and B.
% It may be abbreviated early(T).
\item 
If B is the light,
the {\bf late segment} of T is the infinite ray of T starting at
the point of bitangency with A and moving away from the light 
(Figure~\ref{fig:inout}d).
\item
If B is an object that lies between A and the light L\footnote{This 
	restriction on B is necessary to distinguish A from B.}
and T intersects L,\footnote{This restriction on T is necessary
	to guarantee that the direction 'moving away from the light'
	is well defined.}
the {\bf late segment} of T is the infinite ray of T starting at the point
of bitangency with A and moving away from the light.
\end{itemize}
\end{defn2}

Notice that all of these definitions are local: they depend upon A and B
but not on any other objects.
Thus, if the scene changes, these properties only need to be recomputed
for the objects that have changed.

\begin{lemma}
Let A and L be smooth closed curves, where neither A nor L surrounds the
other.\footnote{This lemma extends to the surrounding case with an
	appropriate definition of inner and outer bitangent, as shown in
	Section~\ref{sec:surround}.}
$A$ and $L$ have exactly 2 outer bitangents and 2 inner bitangents.
\end{lemma}
\prf
This lemma relies on the fact that the inner and outer tangents are extremal.
There is an alternative definition of inner and outer bitangents
that has a more intuitive appeal: the inner and outer bitangents of A and B
are the only bitangents of the convex hull of A and the convex hull of B.
That is, compute the convex hull of A and the convex hull of B, then take
bitangents: there are only 4 bitangents (proof?) and these are the inner and
outer bitangents of A and B.
\QED

See Figure~\ref{fig:inout}.

	% umbra data/umbra/vg9.pts
\begin{figure}
\begin{center}
\includegraphics*[scale=.15]{img/vg9bitang.jpg}
\includegraphics*[scale=.15]{img/vg9inner.jpg}
\includegraphics*[scale=.15]{img/vg9outer.jpg}
\end{center}
\caption{(a) The bitangents
	 (b) The inner bitangents 
	 (c) The outer bitangents
	 (d) The late segment of an outer bitangent}
\label{fig:inout}
\end{figure}

In all figures, the light is drawn in black and the objects in grey.
In the following sections, A will be a distinguished object for which
we are computing the umbra, B another object, and L the light.

% ------------------------------------------------------------------------------

\clearpage

\section{The back umbra}
% The local and global umbra of an object}
\label{sec:localumbra}

We will consider the back umbra first.

\begin{defn2}
The {\bf local umbra of an object} $A$ is the umbra that would be
cast if $A$ were the only object in the scene.
A {\bf local umbral bitangent} of A is a bitangent 
that forms the boundary of the local umbra of A.
\end{defn2}


The local umbra cast by an object is a region bounded by the object
and two of its bitangents to the light.
Which bitangents, and which parts of these bitangents (Figure~\ref{fig:inout}a)?
\ifTalk
We are looking for bitangents that do not intersect the object or light.
However, we must relax our constraints when the object surrounds the light.
\fi
Figure~\ref{fig:G2} illustrates the answer, which is captured by the following lemma.

% ------------------------------------------------------------------------------

% It is the outer bitangents of A and L that bound the local umbra of A
% (Figure~\ref{fig:G2}).
% We capture this fact in a theorem.

\begin{theorem}
\label{thm:localumbra}
Let A be an object, L the light, and
$T_1$ and $T_2$ the late segments of their outer bitangents.
The local umbra of A is the region bounded by $T_1$, $T_2$, 
and the relevant part of A.
% and inside both $T_1$ and $T_2$ in the
% neighbourhood of their points of tangency with $A$.
\end{theorem}
\prf
Every other part of A sees some of L.
Argue that the light becomes visible as you cross these bitangents.
\QED

% Put another way, 
\begin{corollary}
All local umbra bitangents of A are outer bitangents of A and L.
\end{corollary}

	% umbra data/umbra/vg9.pts
\begin{figure}
\begin{center}
\includegraphics*[scale=.15]{img/vg9localumbra.jpg}
% \includegraphics*[scale=.15]{img/ob1-bitang.jpg}
% \includegraphics*[scale=.15]{img/ob1.jpg}
\end{center}
% \centerline{\epsfig{figure=img/ob1-bitang.ps,height=1.542in,width=1.528in}}
% 20% reduction
% \centerline{\epsfig{figure=img/ob1.ps,height=1.567in,width=1.553in}}
% 20% reduction
\caption{The back umbra of an object}
\label{fig:G2}
\end{figure}

\Comment{
	If A does not surround L,
	the local umbra is bounded if and only if the late segments intersect
	(Figures~\ref{fig:G2} and \ref{fig:globalumb3}).
	If A surrounds L,
	the local umbra is unbounded.
}

\vspace{.1in}

{\bf Algorithm:}
\begin{enumerate}
\item Find the bitangents of A and L.
\item Find the two bitangents that are L-outer.
\item Find the late segments of these outer bitangents and their insides.
\end{enumerate}

% \begin{rmk}
% The case described in this section actually subsumes the case of 
% Section~\ref{sec:notsurround}, since an outer bitangent is always
% a pseudo-outer bitangent too.
% That is, you can look for pseudo-outer bitangents
% in all cases.
% We have presented them separately
% since the definition of pseudo-outer bitangents is less intuitive
% without an earlier understanding of outer bitangents,
% and the simpler outer bitangent is sufficient for the vast majority of cases.
% \end{rmk}

\begin{rmk}
More computationally, the local umbra is defined as follows.
Let the two outer bitangents be \lyne{P_L P_A} and \lyne{Q_L Q_A},
where $P_L,Q_L \in L$ and $P_A,Q_A \in A$.
If the rays \ray{P_L P_A} and \ray{Q_L Q_A} intersect,
let $P=Q$ be this intersection;
otherwise let $P$ and $Q$ be the points 
at infinity on the rays \ray{P_L P_A} and \ray{Q_L Q_A}, respectively.

The local umbra is the region bounded by the (potentially infinite)
segments \seg{P_A P}, \seg{Q_A Q},
and the curve segment \arc{P_A Q_A} of A.
\end{rmk}

\begin{implementation}
In our implementation, we build all of the scene inside a predefined cube.
% Equivalently, one can define the scene arbitrarily and then find a bounding box.
In the test for bitangent extremality (Definition~\ref{defn:outer}),
the intersection of an infinite line or ray
with a curve then reduces to the intersection of a finite segment with a curve
(the part of the ray/line inside the box).
This is simpler and more robust.
\end{implementation}

% ------------------------------------------------------------------------------------

% \clearpage

% \section{The global umbra of an object}

\begin{defn2}
\label{defn:global}
The {\bf global umbra of an object $A$} is the umbra cast by $A$
when all other objects are taken into account.
A {\bf global umbral bitangent} of A is a bitangent 
that forms the boundary of the global umbra of A.
\end{defn2}

The local umbra captures the umbra of an object in isolation (Figure~\ref{fig:vg900}b).
However, the umbra cast by a single object can be influenced by other objects.
The global umbra expands the local umbra to consider these other objects in the scene
(Figure~\ref{fig:vg900}c).
In many cases, the global umbra is the same as the local umbra.
We shall first determine when the global umbra differs from the local umbra.
Then, we shall show how the expanded umbra is defined in the cases when
the global umbra is larger than the local umbra.

In Section~\ref{sec:localumbra}, we saw that
the local umbra is defined by outer bitangents of A.
If no object interferes with the early segments of these outer bitangents,
the global umbra is identical to the local umbra.
However, as an object B begins to interfere with these early segments,
the umbra of A is stretched by B (Figure~\ref{fig:vg900}).

\begin{lemma}
\label{lem:differs}
The global umbra of A differs from the local umbra of A
if and only if an object B intersects the early segment
of an outer bitangent of A and L.
\end{lemma}
\prf
To affect the global umbra of A,
an object must conspire with A to block more of the light.
It cannot do so if it lies behind A or to its left or right.
Even an object that lies strictly between the outer bitangents of A
does not directly affect A's umbra, although it may do so by combining
with other objects as we shall see in Section~\ref{sec:innersweep}, 
one of which must satisfy the condition of this lemma.
The discussion of the inner sweep in Section~\ref{sec:innersweep}
will clarify how the object B expands the umbra.
\QED

	% umbra data/umbra/vg900.pts
\begin{figure}
\begin{center}
\includegraphics*[scale=.15]{img/vg900outer.jpg}
\includegraphics*[scale=.15]{img/vg900localumbra.jpg}
\includegraphics*[scale=.15]{img/vg900globalumbra.jpg}
\end{center}
\caption{The crossing of an outer bitangent of A and L 
	     marks a difference between the local and global umbra
	 (a) the outer bitangents of A and L
% An object with other objects interfering with its early outer bitangent
	 (b) the local back umbra of A
 	 (c) the global back umbra of A
}
\label{fig:vg900}
\end{figure}

The global umbra is bounded by bitangents, just like the local umbra.
However, the set of bitangents that can form the 
boundary of the global umbra is richer than the local umbra.
In addition to the outer bitangents of A and L used by the
local umbra, the global umbra may be bounded by inner bitangents of A and B
(where B is an object intervening between A and L)
and outer bitangents of B and L.

% ------------------------------------------------------------------------------------

\clearpage

\section{The global back umbra, through a refinement of the local back umbra}

We define the global back umbra of an object A by dynamically expanding its local back umbra,
using two sweeps of a tangent across interfering objects.
% There are two types of refinement of the umbra.
Both sweeps start with an outer bitangent of the object A and the light L.
% which can be interpreted either as a tangent of A or as a tangent of L.
The first sweep moves this tangent smoothly within the tangent space of A.
The second sweep moves this tangent smoothly within the tangent space of L,
in the opposite direction.
Both sweeps move the tangent across all objects intervening between A and L.
The first sweep continues until the tangent sees the light (or has swept
completely past it).
The second sweep continues until the tangent sees the umbral boundary added
in the first sweep (or has swept completely past it).

These continuous tangent sweeps are transformed into discrete sweeps
by observing that the sweep must always stop at a bitangent.
This allows the sweep to jump directly to the next bitangent candidate
and test for completion.
The resulting sweep involves a finite set of jumps between bitangents,
analogous to the line sweep of computational geometry \cite{preparataShamos}.
In the first sweep,
the inner bitangents of A with a blocking object B define the possible
stopping positions of the sweeping tangent.
In the second sweep, the outer bitangents of L with a blocking object B
define the possible stopping positions.
The next two sections elaborate on these two sweeps that refine
the local umbra into the global umbra.

% duplicate of fig:vg900
	% ---
	% umbra data/globalumb2.pts
	% umbra data/globalumb2.pts
\begin{figure}
\begin{center}
\includegraphics*[scale=.15]{img/globalumb1outer.jpg}
\includegraphics*[scale=.15]{img/globalumb1gu.jpg}
\includegraphics*[scale=.15]{img/globalumb2gu.jpg}
\includegraphics*[scale=.15]{img/globalumb2bgu.jpg}
\includegraphics*[scale=.15]{img/globalumb2cgu.jpg}
% \includegraphics*[scale=.15]{img/globalumb2localumbra.jpg}
% \includegraphics*[scale=.15]{img/globalumb2gumbUnfill.jpg}
% \includegraphics*[scale=.15]{img/globalumb2globalumbra.jpg}
% \includegraphics*[scale=.15]{img/globalumb2-local.jpg}
% \includegraphics*[scale=.15]{img/globalumb2.jpg}
% \includegraphics*[scale=.15]{img/globalumb2-PO.jpg}
\end{center}
\caption{An object interfering with the outer bitangent expands the umbra}
% (a) The crossing of an outer bitangent of A and L 
% 	     marks a difference between the local and global umbra
%	 (b) Local umbra 
%	 (c-d) Global umbra}
\label{fig:globalumb2}
\end{figure}

\clearpage

\subsection{The inner sweep: looking for the light}
\label{sec:innersweep}
% expanding the local umbra while looking for the light}

The first sweep moves in the tangent space of A,
looking for the light L by sweeping across objects that block A from L
(Figure~\ref{fig:innersweep2}).
Its purpose is to refine a local umbra bitangent into a global umbra
bitangent.
Its effect is to expand the local umbra.
We call this sweep an {\bf inner sweep},
because the local umbra bitangent will be repeatedly replaced by 
inner bitangents, and because the sweep is towards the interior of the light.

	% umbra data/umbra/globalumb3.pts
\begin{figure}
\begin{center}
\includegraphics*[scale=.15]{img/globalumb3innersweep1.jpg}
\includegraphics*[scale=.15]{img/globalumb3innersweep2.jpg}
\includegraphics*[scale=.15]{img/globalumb3innersweep3.jpg}
\end{center}
\caption{An inner sweep
	 (a) T = outer bitangent of A and L
	 (b) T = inner bitangent of A and B1
	 (c) T = inner bitangent of A and B2}
\label{fig:innersweep2}
\end{figure}

Consider a local umbra bitangent T, 
which is necessarily an outer bitangent of A and L (Theorem~\ref{thm:localumbra}).
Suppose that an object B intersects T's early segment,
% the early segment of an outer bitangent T of A and L,
implying that the global umbra is different from the local umbra
(Lemma~\ref{lem:differs}).
See Figure~\ref{fig:innersweep2}a.
T will be swept in the tangent space of A, 
towards the inside of the light, until it sees the light.
This continuous sweep can be realized discretely by
repeatedly replacing T by an inner bitangent of A and B,
until there is no object B that hits this inner bitangent
between A and the light, or the sweep passes the light.
Since the inner sweep is looking for the light,
it will stop if it sweeps past the light.
% (without ever seeing it).
This is marked by sweeping past an inner bitangent of A and L (Figure~\ref{fig:maximalumbra}).
The global umbra is never larger than 
the region bounded by the late inner bitangents of A and L (but often smaller).
% This implies that the maximal size of the global umbra is the region 
% bounded by the late inner bitangents of A and L 
% Recall that the minimal size of the global umbra is the region
% bounded by the late {\em outer} bitangents of A and L.

Here is the algorithm for an inner sweep from the local umbra bitangent T of object A.

% Put another way, the global umbra of A is no larger than the region
% bounded by the late inner bitangents of A and L,
% so the expansion of the inner sweep must stop there.

\vspace{.2in}

\centerline{innersweep(T,A)}

\begin{enumerate}
\item T = outer bitangent of A and L
\item while an object B interferes with T on the segment between A and L
\begin{enumerate}
\item T = inner bitangent of A and B (the closest from the present T 
	in the sweep direction)
\item if T has swept past the light (i.e., past an inner bitangent of A and L)
\begin{enumerate}
\item  T = this inner bitangent of A and L
\item  return
% done = true
\end{enumerate}
\end{enumerate}
\end{enumerate}

If more than one object B interferes in step 2,
an arbitrary choice of B is made.

This inner sweep replaces a local umbra bitangent by a global umbra bitangent.
Once both local umbra bitangents of A have been refined
to global umbra bitangents, their late segments define new boundaries
of the global umbra.
% just as with the local umbra.
However, the second sweep may further refine this boundary.

\Comment{
\centerline{innerSweep (T, A, L)}
\begin{enumerate}
\item 
	G = T;\ \ \ done = false;\ \ \ first = true;
	% \item While the umbral bitangent T moves (has been freshly updated on previous step):
\item 	
	sweep direction = direction towards inside of light from T
\item 
	while (!done)
\begin{enumerate}
\item   if (first)
\begin{enumerate}
\item	SEG = early segment of G;\ \ \ first = false;
\end{enumerate}
\item	else
\begin{enumerate}
\item	$G_A = $ G's point of bitangency with A
\item	$G_I = G \cap L$
\item	SEG = segment between $G_A$ and $G_I$
\end{enumerate}
\item 
	$\mbox{HIT} := \{B_i \neq A : B_i \mbox{ is an object that intersects
	SEG} \}$
\item 
	if $\mbox{HIT} = \emptyset$, done = true
\item   
	else
\begin{enumerate}
\item
	choose an arbitrary $B'$ from HIT.
\item 
	\ [search for the light, around the blocker] 
	G = first A-inner bitangent of A and $B'$ that is found
	in the sweep direction.
\end{enumerate}
\end{enumerate}
\item	return(G)
\end{enumerate}
}

% (That is, any umbra outside this range is generated by a different object.)
% Outside of this range, A has no impact: it doesn't block any of the light.
% So A can be ignored outside this region.

% The sweep of A's tangent can be stopped at an inner tangent of A and L.

\ifTalk
Animation of sweep.
\fi

	% umbra data/umbra/maxumbra.pts (with 'Maximal umbra' option)
\begin{figure}
\begin{center}
\includegraphics*[scale=.15]{img/maxumbra.jpg}
\end{center}
\caption{The maximal size of the back umbra}
%  (bounded by late inner bitangents of A and L)}
% using one object B that blocks the entire light 
% and big enough that its outer bitangents with the light 
% also don't intersect the late inner of A and L
\label{fig:maximalumbra}
\end{figure}

	% umbra data/umbra/globalumb3.pts
\begin{figure}
\begin{center}
\includegraphics*[scale=.15]{img/globalumb3luWithSweep.jpg}
\includegraphics*[scale=.15]{img/globalumb3lu.jpg}
\end{center}
\caption{The umbra after refinement by the inner sweep}
\label{fig:umbraafterinnersweep}
\end{figure}

%%%%%%%%%%%%%%%%%%

\clearpage

\subsection{The outer sweep: looking for the global umbra from the light}
\label{sec:outersweep}

While the first sweep expands the umbra, the second sweep shrinks it.
Its purpose is to refine a global umbra bitangent just computed by an
inner sweep, by clipping it as it reappears to the light.
Rather than sweeping a tangent along the object,
it sweeps a tangent along the light.

	% umbra data/umbra/globalumb7.pts
\begin{figure}
\begin{center}
\includegraphics*[scale=.15]{img/globalumb7outersweep1.jpg}
\includegraphics*[scale=.15]{img/globalumb7outersweep2.jpg}
\includegraphics*[scale=.15]{img/globalumb7outersweep3.jpg}
\includegraphics*[scale=.15]{img/globalumb7gu.jpg}
\end{center}
\caption{The outer sweep of T (a) T = outer bitangent of L and A
	 (b) T = outer bitangent of L and B1
	 (c) T = outer bitangent of L and B2
	 (d) the resulting global umbra}
\label{fig:severalstages}
\end{figure}

Consider a local umbra bitangent T and the global umbra bitangent G
computed from T by the inner sweep.
Although G marks a boundary of the shadow,
the light will often become visible again at some point of G,
on the other side of the blocking objects.
The purpose of the second sweep is to find this point of reappearance
into light and clip G there.
To accomplish this, a tangent of the light L is swept across 
all objects that block it from G.
As with the inner sweep,
the tangent is initialized to the local umbra bitangent T, 
but we now sweep in the opposite direction (towards the outside of A).
At the point when L's tangent sees G, 
this tangent is added to the boundary of the global umbra, effectively clipping G.

% While the inner sweep expands the umbra by looking for the light,
% the second sweep shrinks the umbra by looking for the umbra from the light.
% The job of 
% We know that this bitangent defines a boundary of light
% visibility at $P_A$ (its point of bitangency with A) as it leaves A
% on the late segment.
% But the light will often become visible once again as the bitangent
% crosses out of darkness on the other side of the objects.
% Therefore, this bitangent needs to be clipped.

% The search for G's reentry into light must be done from the perspective of the 
% light, using a sweep of the light's tangent looking for the newly
% computed global umbra bitangent T.

As with the inner sweep, we leverage the fact that changes of visibility
must occur at discrete bitangent events
to convert the continuous sweep into a discrete sweep.
Recall that we can assume that an object B hits the early segment of 
the local umbra bitangent T,
otherwise the global umbra is the same as the local umbra
(and neither inner nor outer sweep is necessary).
T is repeatedly replaced by the outer bitangent of B and L,
until the early segment of this outer bitangent is free of objects B (Figure~\ref{fig:severalstages}).
The sweep can be stopped as soon as the sweeping tangent passes G,
since it is then guaranteed to never see G.
Since the tangent is repeatedly replaced by outer
bitangents, and it sweeps towards the outside of A,
the sweep is called an {\bf outer sweep}.
Here is the algorithm for an outer sweep from the local umbra bitangent T of object A.

% \item[iii.]	If angle(H) 'exceeds' (in the direction of current travel) angle(G), return (NULL).



\vspace{.2in}

\centerline{outersweep (T,A)}

\begin{enumerate}
\item T = outer bitangent of A and L
\item G = innersweep(T,A)
\item B = object that interferes with T on the segment between A and L
\item do
\begin{enumerate}
\item T = outer bitangent of B and L (the closest from the present T in the
	sweep direction)
\item if T has swept past G, return since G does not need to be refined
\end{enumerate}
while a new object B interferes with T on the segment between the old B and L
\item clip G by T 
\item return(T)
\end{enumerate}

The outer sweep further refines the global umbra.
The output of the outer sweep algorithm is a new boundary of the global
umbra, which should start after its intersection with 
innerSweep(T,A) (Figure~\ref{fig:severalstages}).
The {\bf inside} of this new global umbra bitangent 
is the side that contains the light L.
Its {\bf late segment} is the segment after its intersection with 
innerSweep(T,A).

% {\bf outerSweep(T,A,L)}, the output of this second sweep from T is a tangent of L.

% Notice that this may introduce a bend into the umbral boundary.
% (two linked lines)

\Comment{
\begin{figure}
\begin{center}
\includegraphics*[scale=.15]{img/globalumb7gumbWithOuter.jpg}
\includegraphics*[scale=.15]{img/globalumb7gumb.jpg}
% \includegraphics*[scale=.15]{img/globalumb4umbra.jpg}
% \includegraphics*[scale=.15]{img/globalumb7umbra.jpg}
\end{center}
\caption{The outer sweep further refines the global umbra (add a figure before these two
	 showing the global umbra after the inner sweep but before the refinement
	 of the outer sweep}
\label{fig:outersweep}
\end{figure}
}

% \clearpage

\Comment{
\centerline{outerSweep (T, A, L)}
\begin{enumerate}
\item	H = T;\ \ G = innerSweep (T,A,L);\ \ done = false;\ \ first = true;
\item	sweep direction = direction towards outside of A from T
\item   while (!done)
\begin{enumerate}
\item   if (first) 
\begin{enumerate}
\item	SEG = early segment of H; \ \ \ first = false;
\end{enumerate}
\item	else
\begin{enumerate}
\item	$H_L = $ H's point of bitangency with the light
\item	$H_I = G \cap H$
\item	SEG = segment between $H_L$ and $H_I$
\end{enumerate}
\item	HIT := $\{ B_i: B_i \mbox{ is an object that intersects SEG}\}$
\item   if $\mbox{HIT} = \emptyset$, done = true;
\item 	else
\begin{enumerate}
\item	choose an arbitrary $B'$ from HIT
\item   \ [search for G, around the blocker] H = first L-outer bitangent of L and $B'$ that is found
	in the sweep direction.
\end{enumerate}			
\end{enumerate}
\item 	return(H)
\end{enumerate}
}

% ----------------------------------------------------------------------------------

\clearpage

\section{The front umbra}
% earlier called self-umbra

The front umbra is the shadow cast by an object upon itself.

\begin{defn2}
Let A and B be closed curves, where neither curve surrounds the other.
Let $t$ be a tangent of A,
and let $T$ be a bitangent of A and B with points of tangency $P_A \in A$ and 
$P_B \in B$.

\begin{itemize}
\item T is a {\bf piercing bitangent} of A with respect to B if:
% self-shading was original term
\begin{itemize}
\item the line segment \seg{P_A P_B} is free (does not intersect A or B) 
\item the line \lyne{P_A P_B} intersects A
\item the line \lyne{P_A P_B} does not intersect B
\item	A and B lie on the same side of T in the neighbourhood of 
	the points of tangency (Figure~\ref{fig:piercing}a),
	if B is the light.
\item   A and B lie on different sides
	of T in the neighbourhood of the points of tangency
	(Figure~\ref{fig:piercing}b),
	if B is an object that lies between A and the light L
	and T intersects L.
\item $P_A$ is not an inflection point.
\end{itemize}
b makes it piercing.
c makes it extremal (otherwise it cannot be a boundary of the umbra, if B is the light, 
or a boundary to the light, if B is another object).
d and e make it a possible umbral boundary.
\item	The {\bf intermediate segment} of a piercing tangent T
	is the segment of T between its point of tangency with A and 
	its first intersection with A.
\end{itemize}
\end{defn2}

	% umbra data/umbra/vg9new.pts
	% umbra data/umbra/vg900new.pts
\begin{figure}
\begin{center}
\includegraphics*[scale=.15]{img/vg9newshadinglight.jpg}
\includegraphics*[scale=.15]{img/vg900newshadingob.jpg}
\end{center}
\caption{(a) A piercing bitangent of A with respect to the light
	 (b) A piercing bitangent of A with respect to another object}
\label{fig:piercing}
\end{figure}

\begin{figure}
\begin{center}
\includegraphics*[scale=.15]{img/vg9newpiercing.jpg}
\end{center}
\caption{This is not a piercing bitangent}
\end{figure}

The computation of the front umbra is orthogonal to the computation of the
back umbra.
We begin with the local front umbra.

\begin{lemma}
Let A be an object, L the light, and $\{T_i\}$ the piercing bitangents
of A and L.
The local front umbra of the object A is a collection of regions,
one per piercing bitangent,
each bounded by the intermediate segment of a piercing bitangent
and the associated boundary of A.
\end{lemma}

% umbra data/umbra/vg9New.pts
\begin{figure}
\begin{center}
\includegraphics*[scale=.15]{img/vg9NewFrontUmbra.jpg}
\end{center}
\caption{The front umbra}
\end{figure}

% umbra data/umbra/vg900new.pts
\begin{figure}
\begin{center}
\includegraphics*[scale=.15]{img/vg900newPiercing.jpg}
\end{center}
\caption{Sweeping to the next piercing bitangent}
\end{figure}

The global front umbra is again found by refining the local front umbra
by a sweep.
This sweep is based upon objects that interfere with the early segment
of the piercing bitangent that defines the front umbra.

Here is the algorithm for an -- sweep from the local front umbra bitangent T of object A.

\vspace{.2in}

\centerline{---sweep(T,A)}

\begin{enumerate}
\item T = piercing bitangent of A and L
\item while an object B interferes with T on its early segment (between A and L)
\begin{enumerate}
\item T = piercing bitangent of A and B (the closest from the present T 
	in the sweep direction), or if no piercing bitangents exist, then T
	is the 
\item if T has swept past the light (i.e., past --)
\begin{enumerate}
\item  T = -
\item  return
% done = true
\end{enumerate}
\end{enumerate}
\end{enumerate}

I don't think there is a need for a second sweep (an analog of the back umbra's outer sweep).

% ----------------------------------------------------------------------------------

\clearpage

\section{The global umbra of an object and a scene}

% The boundary of the global umbra is defined by refining each bitangent
% of the boundary of the local umbra, using an inner sweep,
% followed by refining this new boundary, using an outer sweep.

The global umbra of the object A is now defined by applying
an inner and outer sweep to the two bitangents of its local umbra,
generating up to four global umbra bitangents.
Let $T_1$ and $T_2$ be the two outer bitangents of A and L.
The global umbra of A is defined by A and the four bitangents
innerSweep($T_1$,A), innerSweep($T_2$,A),
outerSweep($T_1$,A), and outerSweep($T_2$,A),
lying to the inside of their late segments (Figure~\ref{}).
% In other words, the global umbra of A is the polygon defined by the intersection
% of the inside halfspaces of innerSweep($T_1$,A), innerSweep($T_2$,A),
% outerSweep($T_1$,A), and outerSweep($T_2$,A).
% (See \cite{preparataShamos} for a discussion of the
% intersection of halfspaces, including algorithms.)
The global umbra of an object is pseudo-convex: convex except possibly
for its curved boundary.
If the umbra is restricted to a polygonal room,
the room wall's halfspaces are also used to restrict the umbra.

Notice that an object B may affect the global umbra of an object A
without lying between A and the light (Figure~\ref{see drawn example}).
Nevertheless, a space-partitioning scheme (e.g., octrees) can still 
be used to constrain
the objects that need to be considered in the computation of the global
umbra of an object A.
For example, objects on the opposite side of the light from A 
do not have to be considered in the computation of the global umbra of A.
Precisely, the objects inside the following region can affect the 
global umbra of A: the region spanned by the tangents of L parallel to 
the inner bitangents of A and L.
In particular, the local umbra after the refinement of the inner sweeps
is no larger than the region spanned by the inner bitangents I of A and L,
and once an outer bitangent of B and L does not intersect
(after it sweeps past parallel with I)
it cannot affect this umbra by clipping (Figure~\ref{active region drawing}).

\ifTalk
We are computing dawns and sunsets, where the light (sun) appears or disappears
over the horizon of an object.
(just as, when flying around the Earth in an airplane, one may pass into night
but then, as you continue travelling, pass back into day).
\fi

The {\bf umbra of a scene} is the union of the global umbrae of the
objects in the scene (Figure~\ref{fig:scene1}).
% An {\bf inner umbral bitangent} of A is
% a global umbra bitangent of A generated by an inner sweep.
% An {\bf outer umbral bitangent} of A is
% a global umbra bitangent of A generated by an outer sweep.
	% % from a local umbra bitangent of A 

Figure~\ref{fig:scene2} illustrates the effect of adding an object to a scene.

% \subsubsection{Penumbrae conspiring to create umbra}

% \vspace{.5in}

\begin{rmk}
Care must be taken if objects touch.
The proposed solution inherently assumes nontouching objects.
Since the proposed solution is correct in the limit as the objects approach
but do not yet touch, the touching solution should be realized as this
limit rather than computed directly.

% NOT A PROBLEM: in the limit (when objects approach but never touch) our solution is correct.
MUST SIMPLY TREAT TOUCHING OBJECTS DIFFERENTLY.
Consider a light surrounded by circular objects, which abut, allowing no
light to pass their ring.
This appears to be a vivid example of the insufficiency of local umbrae for computing
the global umbra.
Notice that the stretching of the umbra addressed in the previous section
is not the cause of the larger global umbra, since no object intersects
another object's PO bitangents.
Instead, this is a case of the penumbrae of objects conspiring together
to create umbra.
This is a more subtle umbra to compute, using .3 + .7 = 1 ideas,
rather than 1 + anything else = 1.
\end{rmk}

	% umbra data/umbra/vg9001.pts
\begin{figure}
\begin{center}
\includegraphics*[scale=.15]{img/vg9001bitang.jpg}
\includegraphics*[scale=.15]{img/vg9001localumbra.jpg}
\includegraphics*[scale=.15]{img/vg9001globalumbra.jpg}
\end{center}
\caption{The global umbra of a scene: (a) its bitangents (b) its local umbrae (c) its global umbra}
\label{fig:scene1}
\end{figure}

% \begin{figure}[hb]
%\begin{center}
% \includegraphics*[scale=.15]{img/--}
%\end{center}
% \caption{Sometimes the global umbra is the union of the local umbrae (umbra data/globalumb1.pts)}
% \label{fig:globalumb1}
% \end{figure}

% \begin{figure}
% \begin{center}
% \includegraphics*[scale=.15]{img/globalumb3.jpg}
% \end{center}
% % \centerline{\epsfig{figure=img/globalumb3.ps,height=2.906in,width=2.911in}}
% % 40% reduction
% \caption{Pushing an umbral bitangent through several objects (game of tag)}
% \label{fig:globalumb3}
% \end{figure}

	% umbra data/umbra/ovoid2.pts
% \begin{figure}
% \begin{center}
% \includegraphics*[scale=.15]{img/ovoid2-local.jpg}
% \end{center}
% \caption{A simple local umbra: too simple?}
% \label{fig:G1}
% \end{figure}

% -----------------------------------------------------------------------------

\clearpage

\clearpage

\section{Penumbra}

The back penumbra is the difference between the maximal umbra
and the global umbra.
There are several regions of the front penumbra.
Consider the local front penumbra.
Each region of the local front penumbra is a region bounded by the intermediate segments
of an 'outer' piercing bitangent of A and L and its associated 'inner' piercing bitangent.
[need to refine definition of piercing bitangent].

\clearpage

\section{The discontinuity mesh of a point light source}

Consider the special case of a point light.
(Only in journal version.)

\section{Cell definition, point location in discontinuity mesh}

Show partition of 2-space defined by the discontinuity mesh of a single object
(showing the umbra/penumbra of every object would have overlaps, and so
difficult to show; but try it too)

Discuss point location in this discontinuity mesh.

Discuss painting this mesh with shadow.

% ------------------------------------------------------------------------------------

\section{$2 \frac{1}{2}$-D visibility}

Our umbral solution in 2-space also gives an approximate solution
to lighting and visibility in a $2 \frac{1}{2}$-D environment.
Consider flythroughs of a city defined by digital elevation maps.

% ------------------------------------------------------------------------------------

\section{Conclusions}

Remind them that this theory is also applicable to general visibility
analysis of smooth objects.

We show how the analysis of an entire scene can be broken down into
the analysis of a single object at a time.
We also show how two quite far-removed objects can affect each other,
even if their lines of sight to the light do not overlap.

If the light is interpreted as the sun and the umbra as the night,
the inner sweep may be interpreted as a search for the sunset,
and the outer sweep a search for the dawn of the light.


The inner umbral bitangents are tangent to A,
while the outer umbral bitangents are tangent to L.

Both sweeps define a global umbra bitangent G of A
through refinement of a local umbra bitangent T of A.
% (Recall that T is an outer bitangent of A and L.)

The sweeping of a tangent (and the associated widening of the umbral boundary)
can continue through several objects, like a game of tag.
% The pushing will stop when no object intersects the early segment associated
% with the umbral boundary (Figure~\ref{fig:globalumb3}).

We don't want to try to compute the entire global umbra all at once:
by breaking it up into the global umbra generated by each object,
the problem becomes tractable.

The obvious next step with this theory is the computation of the penumbra
and the extension to 3-space.

\begin{figure}
\begin{center}
\includegraphics*[scale=.15]{img/vg90umbra.jpg}
\includegraphics*[scale=.15]{img/vg900umbra.jpg}
\end{center}
\caption{The effect of adding one object: (a) global umbra before (b) global umbra after}
\label{fig:scene2}
\end{figure}

% \begin{figure}
% \begin{center}
% \includegraphics*[scale=.1]{img/globalumb2-PI.jpg}
% \end{center}
% \caption{Inner bitangents refine the local umbra into the global umbra}
% \label{fig:innerbitang}
% \end{figure}

\section{Dagstuhl Talk}

Sweeping a tangent to (i) compute the umbra and (ii) compute the convex hull.

\section{Appendix}
\label{sec:appendix}

\subsection{Software}

\begin{itemize}
\item draw B/W umbral regions for a collection of objects
\item draw B/W local penumbral regions for a collection of objects
\item draw B/W global penumbral regions for a collection of objects 
\item draw greyscale local penumbral region for one object
\item draw greyscale global penumbral regions for a collection of objects
\item find the extreme bitangents
\end{itemize}

\subsection{Defining the umbra point by point}
\label{sec:local}

It is simple to determine if a point P is in the umbral shadow cast by a 
single object A (Figure I):
P is in umbra of A iff all of the light's tangents through P (it is enough to test
the extremal two) hit A.
However, several objects can conspire to place P in umbra, although it is not
in any one object's umbra (Figure III).
Finding this type of umbral point requires more subtlety.
For example, it is not enough to demand simply that all of the light's tangents through P
hit some object (Figure II).
In effect, we need to test that there are no cracks between objects through which
the light becomes visible, by 'sweeping' a beam centered at P across the light.
The following algorithm for testing if P lies in umbra achieves this virtual sweep.

\begin{defn2}
A tangent T of the curve C through a point P is {\bf extremal}
if the entire curve C lies in the same halfplane defined by T
(i.e., T does not intersect C).
The {\bf inside halfplane} of an extremal tangent T of C through a point P is
the halfplane defined by T that contains C.
An extremal tangent is {\bf left-extremal} if its inside halfplane 
contains the point at infinity (1,0,0) along the positive x-axis,
otherwise it is {\bf right-extremal}.
\end{defn2}

Is the point P in the umbra?
\begin{description}
\item[] T $\leftarrow$ left-extremal tangent of the light through P
\item[] while (T has not swept past light \&\& 
       T hits some object A before it hits the light)
\begin{description}
\item[]	T = right-extremal tangent of A through P
\end{description}
\item P is in umbra iff T has swept past light
\end{description}

We are sweeping a line about P searching for a passageway to the light, 
leaping past objects that block the light by jumping to their right-extremal tangents, 
until we either find the light or sweep past it.
Figure IV (Figure III with sweep annotations) illustrates this algorithm.

What is the complexity of this algorithm?

% ------------------------------------------------------------------------------------

\centerline{{\bf What is new}}
\begin{enumerate}
\item implementation of algebraic geometry's dual curve that is
\begin{enumerate}
\item robust using 2 dual spaces, and
\item details worked out for Bezier curves, and
\item a primal structure in dual space, unlike dual Bezier curves
\end{enumerate}
\item dual Chaikin
\begin{enumerate}
\item 	don't even evaluate the parts of the curve unaffected by bitangents
		(really be lazy!)  (change the demo to show this)
\item 	analyze the improvement in accuracy as you move through subdivision stages
\item 	when do bad mistakes disappear and refinement only sets in?
\end{enumerate}
\item lazy subdivision idea for subdivision curves
\item application of bitangents to fundamental applications that are only known
	for polygons, not smooth curves
\begin{enumerate}
\item to smooth convex hull, extending Graham scan
\item to smooth visibility graph, improving quality of shortest paths
			[might be able to extend Welzl's algorithm]
\item to smooth lights in 2d, with penumbra and umbra calculation,
	     	anticipating a lifting to smooth lights in 3d
\begin{enumerate}
\item   software to shade the umbra/penumbra
\item 	to determine efficiently whether a curve is in umbra
\item 	to determine efficiently which parts of a curve are in penumbra
\item 	to determine their relative lighting (how much of the light they see)
\end{enumerate}
\end{enumerate}
\end{enumerate}
	     
%%%%%%%%%%%%%%%%%%%%%%%%%%%%%%%%%%%%%%%%%%%%%%%%%%%%%%%%%%%%%%%%%%%%%%%%%%%
%%%%%%%%%%%%%%%%%%%%%%%%%%%%%%%%%%%%%%%%%%%%%%%%%%%%%%%%%%%%%%%%%%%%%%%%%%%
			
\section{Introduction}

3 categories of lighting: umbra (fully shaded), penumbra (partially lit), 
	and noon (fully lit)

\section{Flawed approaches}

Don't want to cast light from each point of the light: 
	then every point of 2-space
	has an infinite number of light points casting light on it,
	for which it must sum the contributions
	Notice that every light point casts across a halfspace minus the
	areas blocked by objects or the neighbouring parts of the light (if
	light is concave)
	
On the other hand, computing the light intensity at every pixel
	in free space is also undesirable (?).
	
%%%%%%%%%%%%%%%%%%%%%%%%%%%%%%%%%%%%%%%%%%%%%%%%%%%%%%%%%%%%%%%%%%%%%%%%%%%
			
\section{General algorithm}

\subsection{Part I: Definition of global penumbra}

Instead, first decompose 2-space into global penumbra and global umbra using bitangents:
	this allows us to ignore all but the global penumbral regions;
	this is done for each object: a point is in the global penumbra
	if it is in at least one local penumbral region and it is not in any
	local umbral region, which would block it;
	this is a form of point location in cells from a line arrangement
			
local umbra 
	(from the bitangents of the light and an object A)
	defn: region of space that is entirely blocked from the light by A
	- elegant definition of umbral region: cell in a line/curve arrangement
	- hacky but simple definition of umbra: intersection of halfspaces defined
		by bitangents and objects
	- the hacky definition is simple to implement and simple for point 
		classification, although not optimal efficiency
local penumbra 
	(from the bitangents of the light and an object)
	defn: region of space that is partially blocked from the light by A
local noon
	= 2-space - (local umbra and local penumbra)
	defn: region of space that sees the entire light w.r.t. A
		(i.e., region that is not blocked from the light at all by A)
global umbra 
	= union of local umbras
	defn: region of space that is entirely blocked from the light 
	      (by one or a combination of the objects)
*global penumbra*	(the nontrivial one)
	= (union of local penumbras) - global umbra
	defn: region of space that is partially blocked from the light 
	      (by one or a combination of the objects)
global noon
	= intersection of local noons
	= 2-space - (global umbra and global penumbra)
	
%%%%%%%%%%%%%%%%%%%%%%%%%%%%%%%%%%%%%%%%%%%%%%%%%%%%%%%%%%%%%%%%%%%%%%%%%%%
			
\subsection{Part II: Definition of light intensity inside global penumbra}

for each point P in a global penumbral region,
  for each object A such that P is inside A's penumbral region,
	intersect tangent of A through P with light, yielding $L_A$
  choose L = $L_A$ that yields the smallest visible light segment
	(the larger light segments are blocked by one of the other objects)
  compute (read off) proportion of light visible from L on the
	outside of tangent
	   - this involves computing tangent PP' from light 
	     through P; then measuring visible (from P) portion of 
	     light from L to P', which is basically arc length
	     of convex hull from L to P'
  assign (paint) this intensity to every pixel (point) on 
	     the tangent from L to P and past P to the next 
	     intersection with an object

%%%%%%%%%%%%%%%%%%%%%%%%%%%%%%%%%%%%%%%%%%%%%%%%%%%%%%%%%%%%%%%%%%%%%%%%%%%
			
\subsection{Part III: Definition of light intensity inside global umbra and global noon}

similarly paint pure black shadow or pure white light
     from pixels (points) that are found in global umbra
     or global noon, by continuing line --- (how?)

%%%%%%%%%%%%%%%%%%%%%%%%%%%%%%%%%%%%%%%%%%%%%%%%%%%%%%%%%%%%%%%%%%%%%%%%%%%

\section{Examples}

Test data.
\begin{itemize}
\item bitangency: circle, ob1, ob3, vg7.pts
\item convex hull: vgraph: vg4.pts, vg1.pts (contour), vg3.pts (contour), vg4.pts (small natural)
	vg5.pts (large natural)
\item lighting: ob1, vg5.pts
\end{itemize}

% of \ifJournal at beginning of appendix
\fi
			
\end{document}
