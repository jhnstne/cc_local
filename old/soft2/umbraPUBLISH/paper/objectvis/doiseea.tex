\documentclass[10pt,twocolumn]{article}
\usepackage{times}
\usepackage[pdftex]{graphicx}
% \makeatletter
\def\@maketitle{\newpage
 \null
 \vskip 2em                   % Vertical space above title.
 \begin{center}
       {\Large\bf \@title \par}  % Title set in \Large size. 
       \vskip .5em               % Vertical space after title.
       {\lineskip .5em           %  each author set in a tabular environment
        \begin{tabular}[t]{c}\@author 
        \end{tabular}\par}                   
  \end{center}
 \par
 \vskip .5em}                 % Vertical space after author
\makeatother

% default values are 
% \parskip=0pt plus1pt
% \parindent=20pt

\newcommand{\SingleSpace}{\edef\baselinestretch{0.9}\Large\normalsize}
\newcommand{\DoubleSpace}{\edef\baselinestretch{1.4}\Large\normalsize}
\newcommand{\Comment}[1]{\relax}  % makes a "comment" (not expanded)
\newcommand{\Heading}[1]{\par\noindent{\bf#1}\nobreak}
\newcommand{\Tail}[1]{\nobreak\par\noindent{\bf#1}}
\newcommand{\QED}{\vrule height 1.4ex width 1.0ex depth -.1ex\ \vspace{.3in}} % square box
\newcommand{\arc}[1]{\mbox{$\stackrel{\frown}{#1}$}}
\newcommand{\lyne}[1]{\mbox{$\stackrel{\leftrightarrow}{#1}$}}
\newcommand{\ray}[1]{\mbox{$\vec{#1}$}}          
\newcommand{\seg}[1]{\mbox{$\overline{#1}$}}
\newcommand{\tab}{\hspace*{.2in}}
\newcommand{\se}{\mbox{$_{\epsilon}$}}  % subscript epsilon
\newcommand{\ie}{\mbox{i.e.}}
\newcommand{\eg}{\mbox{e.\ g.\ }}
\newcommand{\figg}[3]{\begin{figure}[htbp]\vspace{#3}\caption{#2}\label{#1}\end{figure}}
\newcommand{\be}{\begin{equation}}
\newcommand{\ee}{\end{equation}}
\newcommand{\prf}{\noindent{{\bf Proof}:\ \ \ }}
\newcommand{\choice}[2]{\mbox{\footnotesize{$\left( \begin{array}{c} #1 \\ #2 \end{array} \right)$}}}      
\newcommand{\scriptchoice}[2]{\mbox{\scriptsize{$\left( \begin{array}{c} #1 \\ #2 \end{array} \right)$}}}
\newcommand{\tinychoice}[2]{\mbox{\tiny{$\left( \begin{array}{c} #1 \\ #2 \end{array} \right)$}}}
\newcommand{\ddt}{\frac{\partial}{\partial t}}
\newcommand{\Sn}[1]{\mbox{{\bf S}$^{#1}$}}
\newcommand{\calP}[1]{\mbox{{\bf {\cal P}}$^{#1}$}}

\newtheorem{theorem}{Theorem}	
\newtheorem{rmk}[theorem]{Remark}
\newtheorem{example}[theorem]{Example}
\newtheorem{conjecture}[theorem]{Conjecture}
\newtheorem{claim}[theorem]{Claim}
\newtheorem{notation}[theorem]{Notation}
\newtheorem{lemma}[theorem]{Lemma}
\newtheorem{corollary}[theorem]{Corollary}
\newtheorem{defn2}[theorem]{Definition}
\newtheorem{observation}[theorem]{Observation}
\newtheorem{implementation}[theorem]{Implementation note}

% \font\timesr10
% \newfont{\timesroman}{timesr10}
% \timesroman

\makeatletter
\def\@maketitle{\newpage
 \null
 \vskip 2em                   % Vertical space above title.
 \begin{center}
       {\Large\bf \@title \par}  % Title set in \Large size. 
       \vskip .5em               % Vertical space after title.
       {\lineskip .5em           %  each author set in a tabular environment
        \begin{tabular}[t]{c}\@author 
        \end{tabular}\par}                   
  \end{center}
 \par
 \vskip .5em}                 % Vertical space after author
\makeatother

\newcommand{\SingleSpace}{\edef\baselinestretch{0.9}\Large\normalsize}
\newcommand{\DoubleSpace}{\edef\baselinestretch{1.4}\Large\normalsize}
\newcommand{\Comment}[1]{\relax}  % makes a "comment" (not expanded)
\newcommand{\Heading}[1]{\par\noindent{\bf#1}\nobreak}
\newcommand{\Tail}[1]{\nobreak\par\noindent{\bf#1}}
\newcommand{\QED}{\vrule height 1.4ex width 1.0ex depth -.1ex\ \vspace{.3in}} % square box
\newcommand{\arc}[1]{\mbox{$\stackrel{\frown}{#1}$}}
\newcommand{\lyne}[1]{\mbox{$\stackrel{\leftrightarrow}{#1}$}}
\newcommand{\ray}[1]{\mbox{$\vec{#1}$}}          
\newcommand{\seg}[1]{\mbox{$\overline{#1}$}}
\newcommand{\tab}{\hspace*{.2in}}
\newcommand{\se}{\mbox{$_{\epsilon}$}}  % subscript epsilon
\newcommand{\ie}{\mbox{i.e.}}
\newcommand{\eg}{\mbox{e.\ g.\ }}
\newcommand{\figg}[3]{\begin{figure}[htbp]\vspace{#3}\caption{#2}\label{#1}\end{figure}}
\newcommand{\be}{\begin{equation}}
\newcommand{\ee}{\end{equation}}
\newcommand{\prf}{\noindent{{\bf Proof}:\ \ \ }}
\newcommand{\choice}[2]{\mbox{\footnotesize{$\left( \begin{array}{c} #1 \\ #2 \end{array} \right)$}}}      
\newcommand{\scriptchoice}[2]{\mbox{\scriptsize{$\left( \begin{array}{c} #1 \\ #2 \end{array} \right)$}}}
\newcommand{\tinychoice}[2]{\mbox{\tiny{$\left( \begin{array}{c} #1 \\ #2 \end{array} \right)$}}}
\newcommand{\ddt}{\frac{\partial}{\partial t}}

\newtheorem{theorem}{Theorem}
\newtheorem{rmk}{Remark}
\newtheorem{example}{Example}
\newtheorem{conjecture}{Conjecture}
\newtheorem{claim}{Claim}
\newtheorem{notation}{Notation}
\newtheorem{lemma}{Lemma}
\newtheorem{corollary}{Corollary}
\newtheorem{defn2}{Definition}

\newif\ifJournal
\Journalfalse
\newif\ifVerbose
\Verbosetrue
\newif\ifComment                % large-scale comments
\Commentfalse

\newcommand{\hero}{star}

% \setlength{\headsep}{.5in}
% \markright{\today \hfill}
% \pagestyle{myheadings}

% \DoubleSpace

\setlength{\oddsidemargin}{0pt}
\setlength{\topmargin}{0in}	% should be 0pt for 1in
\setlength{\textheight}{8.6in}
\setlength{\textwidth}{6.875in}
\setlength{\columnsep}{5mm}	% width of gutter between columns

% -----------------------------------------------------------------------------

\title{Can you see it?\\Object visibility in a smooth flat scene}
% Do you see it?
% Part 1
\author{J.K. Johnstone\thanks{Department of Computer and 
    Information Sciences, UAB, Birmingham, AL 35294-1170.  This work
    was partially supported by the National Science Foundation under grant CCR-0203586.}}

\begin{document}
\maketitle

\begin{abstract}
Given a smooth object A nestled in a scene of smooth objects, and
% a certain vantage point P
a viewpoint P, we are interested in analyzing whether A is visible from P.
If A is visible, we want to provide a probe that realizes this visibility.
If A is not visible, we want to understand exactly which objects block it.
This is a smooth version of a classical problem in visibility analysis,
and a step towards the promotion of our understanding of visibility analysis,
% and represents part of a research program that endeavours to promote
which is quite thorough for linear structures,
to smooth scenes bounded by curves and surfaces.
In this paper, we explore this problem
in 2-space, where a smooth object is an arbitrary closed curve.
There are no constraints on the objects other than their smoothness and disjointness,
and no constraints on the number of objects in the scene.
Interesting cases arise when the objects intermingle densely and when objects
lie in concavities of other objects.
Since curves have no vertices, certain special tangents replace vertices
as the focus of analysis and the source for visual events.

A certain number of probes are necessary to determine object visibility,
and the challenge is to reduce this number as much as possible.
We develop three progressively smaller granularities of probes.
\end{abstract}

\Comment{
Keywords: extreme tangent; angular range of an object; piercing tangent;
concavity; extreme tangents of concavity; angular range of concavity;
inside of a tangent.
}

\Comment{
\begin{figure}
\begin{center}
\includegraphics[width=2in]{img/foo.jpg}
\includegraphics[width=2in]{img/foo.jpg}
\end{center}
\caption{}
\label{}
\end{figure}

\begin{figure}
\begin{center}
\includegraphics[width=3in]{img/foo.jpg}
\includegraphics[width=3in]{img/foo.jpg}
\end{center}
\caption{}
\label{}
\end{figure}
}

\Comment{
paper structure: (1) introduction where necessity for virtual vertices is established
(for polygons, we would probe at the vertices);
(2) establish structure of smooth curves beneficial to visibility (concavity, 
extreme tangents); (3) establish the optimal probes.
}

% Have entire paper aiming for objectVisibility algorithm in section 8
% (the smallest set of global probes).

\ifJournal
We classify tangents through the viewpoint.
We develop tests for whether a point lies in a concavity and how a single object
and a collection of objects can block another.

We first observe that tangents that pass through the viewpoint are sufficient.
We then observe that certain special tangents through the viewpoint are sufficient:
extreme and piercing tangents.
Finally, we refine these probes further, finding a subset of extreme
and piercing tangents that are sufficient.

We prove the sufficiency of the smallest set of probes.

% Venue:
% relationship to visibility graph and exploratory nature of problem suggests SoCG.
% Relationship to visibility problem suggests graphics conference.
% Relationship to smooth curves suggests geometry conference.
\fi

% ----------------------------

\section{Introduction}

% This paper studies object visibility in a smooth flat scene.
This paper solves the following instance of the visibility problem:
given a smooth object A in a flat scene of smooth objects, and a viewpoint P,
is A visible from P?
% is A completely invisible from P?
  % \ifTalk
  % this is a static or local version of a more general question that we will ask later:
  % what are *all* of the viewpoints from which A is completely invisible?
  % \fi
If the visibility of an object in a scene is well understood, many important questions
in computer graphics may be answered:
occlusion culling, shadow computation (the object is the light), 
and global illumination are applications typically cited.
Applications in other disciplines are easy
to find as well, such as camera positioning for surveillance.
  % \ifTalk
  % camera positioning for filming (with dynamic visibility analysis)
  % guard positioning for art gallery theorems (cite O'Rourke)
  % radiation therapy
  % \fi
Visibility analysis has received a great deal of attention for polyhedral scenes,
but little attention for smooth scenes.
This paper is part of a research program whose goal is to promote the
% dedicated to the promotion of
understanding of visibility analysis from piecewise linear scenes 
to smooth scenes bounded by smooth curves and surfaces,
with practical and efficient implementations.
We begin modestly with a solution 
for an arbitrary scene of smooth curves in 2-space (Figure~\ref{fig:eg19probe}a).
The rich complexity and subtlety of smooth objects makes this problem
interesting even in 2-space.
% a modest beginning dictated by the rich complexity of smooth objects.
% This paper establishes fundamental principles that can be built on.
There are no constraints on the objects other than their
smoothness and disjointness.

\ifJournal % GOOD ONE
% smoothness is not required: a sharp point would become another virtual vertex.
\fi

% example of final probes for a sophisticated scene, to give a sense that this is a
% nontrivial problem
% objectvis -R -s 0 -v -.65 -1 -z 1.8 data/eg19.pts
% objectvis -R -p -s 0 -v -.65 -1 -z 1.8 data/eg19complex.pts
% ifJournal: why does eg19complexDumpsCore.pts dump core?
\begin{figure}
\begin{center}
\includegraphics[width=2in]{img/eg19.jpg}
\includegraphics[width=2in]{img/eg19allprobe.jpg}
\includegraphics[width=2in]{img/eg19exprobe.jpg}
\includegraphics[width=2in]{img/eg19reversevisprobe.jpg}
% \includegraphics[width=2in]{img/eg19probe.jpg}
\end{center}
\caption{Testing the visibility of an object in a smooth scene.
  % Probing a scene to determine visibility. 
         (a) The smooth scene and viewpoint.
         (b) Probing at all tangents through the viewpoint.
         (c) Probing at all extreme and piercing tangents.
         (d) Probing at a certain subset of the extreme and piercing tangents.}
\label{fig:eg19probe}
\end{figure}

The primary issue in computing object visibility
is a reduction of the number of probes 
that are necessary to test the object's visibility.
%
\begin{defn2}
A {\bf probe} is the intersection of a ray from the viewpoint with the smooth scene,
to find the first object that is met by the probe.
It is the testing of a line of sight.
\end{defn2}
\noindent 
Each probe is either additional evidence of invisibility or a witness for visibility.
  % Since a point is visible if the line of sight to the point is free of collision 
  % with the scene, and an object is visible if any point of the object is visible,
  % technically
An object is invisible if every line of sight to the object is blocked.
However, with an understanding of the scene and an understanding of the structure of
visibility,
the visibility of an object A can be tested with a few, well chosen probes
(Figure~\ref{fig:eg19probe}).
  % rather than an infinite number of probes 
We consider three choices for the probe set,
progressively smaller although also progressively more complicated to select.
  % One goal is to reduce the number of probes and another is to reduce the cost
  % of each probe. 

The visibility analysis will be carried out in a smooth flat scene defined as follows.
%Here is a definition of the smooth flat scene that is studied in this visibility problem.

\begin{defn2}
A {\bf smooth closed curve} is an arbitrary simple
% \footnote{That is, no self-intersections.} 
closed $C^1$-continuous curve.
A {\bf smooth object} is an object bounded by a smooth closed curve.
A {\bf smooth flat scene} is a collection of disjoint smooth objects.
\end{defn2}

% Since real objects in graphics and robotics will be simple and closed,
% the only constraint is $C^1$-continuity.

Since an analysis of tangents is central to our method,
we need a well-defined tangent everywhere on the object,
hence a $C^1$-continuous curve.
A natural choice, used in our implementation, is to define the objects by B-splines.
There is no bound on the geometric complexity or the degree 
(if it is indeed polynomial or rational) of each object,
and collections of objects may have arbitrarily complex inter-relationships.
For example, an object may contain complex concavities,
an object may lie in the concavity of another, or the concavities of two objects
may interlock.
An object may even lie in a closed hole of another object.
% since the disjointness constraint only applies to their boundary.
Note that the objects may be interpreted as footprints of 2.5-dimensional
objects in 3-space. %, as in a cityscape.
Moreover, an analysis of 2-space can yield insights into problems in 3-space
(an excellent example of this is the occlusion culling work of Teller \cite{teller91}).

\begin{defn2}
The {\bf viewpoint} P is an arbitrary point in free space.
$x$ is {\bf visible} from $P$ if the open line segment \seg{Px} 
has no intersections or points of tangency with the scene.
\end{defn2}

% \ifJournal: treat the case of viewpoint inside convex hull: GOOD ONE
For clarity of presentation, we assume in this extended abstract that
the viewpoint lies outside the convex hull of each object.
The theory developed in this paper can be extended naturally to deal
with viewpoints inside the hull, but at the expense of some special 
cases in definitions % (e.g., extreme tangents) 
and theorems.

\subsection{Probing a smooth scene}

% virtual vertices, sufficiency of probes at tangents through the viewpoint
A fundamental problem in visibility analysis of a smooth scene is the lack
of vertices.
In a polygonal scene, all probes are at vertices, since a change of visibility
must occur at a vertex.
But a smooth object does not have vertices, so the scene does not directly 
reveal a finite candidate set of probes.
The smooth analogue to probing at all vertices of a polygonal scene
is probing at all tangents through the viewpoint of a smooth scene.
Equivalently, the virtual vertices of a smooth scene are 
the points of tangency of the tangents through the viewpoint (Figure~\ref{fig:monkey}).
% To see that this set of probes is sufficient to test smooth object visibility,
Consider a line of sight (a ray from the viewpoint) sweeping across the scene. 
% \ifTalk like a lighthouse beacon or radar.\fi
As the ray sweeps across an object B that blocks A, the first opportunity
for A to become visible is as the ray leaves B at a tangent of B (Figure~\ref{fig:appearing}).
\Comment{
  Since A may appear from behind part of B but still
  in front of other parts of B (FIGURE), 
  the ray may not leave B altogether at this first visibility; 
  nevertheless, the ray will be tangent to this front concavity of B.
}
% % That is, A may be blocked by a concavity in the front of B. 
Probing at all tangents through the viewpoint is equivalent for a smooth scene
to probing at all vertices in a polygonal scene.
% START HERE
% PROVE IT USING ANALOGY TO DISCUSSION FROM OTHER PAPER 
% ON SEEING A DIRECTLY PAST B (BITANGENT OF A AND B) AND THROUGH A GAP BETWEEN B AND C
% (BITANGENT OF B AND C): LET B BECOME A POINT.

% objectvis -s 0 -v 0 -2 ../../../interpolation/data/nazca_monkey.pts
\begin{figure}
\begin{center}
\includegraphics[width=2in]{img/eg19virtual.jpg}
\includegraphics[width=2in]{img/eg19tang.jpg}
% \includegraphics[width=2in]{img/monkeyCloseup.jpg}
\includegraphics[width=1.5in]{img/monkeyVirtualCloseup.jpg}
% \includegraphics[width=2in]{img/monkeyTang.jpg}
\end{center}
\caption{A curve must define its own virtual vertices (a), using tangents through the 
         viewpoint (b).  (c) is another example.}
\label{fig:monkey}
\end{figure}

% objectvis -s 1 -v -.65 -1 -z 1.8 -R data/eg19EF.pts
% objectvis -s 3 -v -.65 -1 -z 1.8 -R data/eg19ABCF.pts
% objectvis -s 1 -v 0 -1 -z 1.8 data/eg11.pts
\begin{figure}
\begin{center}
\includegraphics[width=1in]{img/eg19EFvisprobe.jpg}
% \includegraphics[width=2in]{img/eg19ABCFvisprobe.jpg}
\includegraphics[width=1in]{img/eg11tang.jpg}
\end{center}
\caption{An object first becomes visible at a tangent through the viewpoint.
        (a) An object appearing from behind another object.
        (b) An object appearing from behind a concavity.}
\label{fig:appearing}
\end{figure}

A disadvantage of these virtual vertices is that they must be computed,
and they are dependent on the viewpoint so they slide around the object as the
viewpoint moves.
But an advantage 
is that virtual vertices only encode interesting features for visibility.
For example, if we consider the smooth interpolant of a polygon, many of the polygonal
vertices have no smooth counterpart (Figure~\ref{fig:monkey}c).

The rest of the paper considers how we might develop a smaller sufficient probe set
for testing smooth object visibility than the tangents through the viewpoint.
% Is there a smaller probe set than the tangents through the viewpoint?
It is structured as follows.
Work related to object visibility is discussed in Section~\ref{sec:related}.
To find a smaller sufficient probe set, 
we need to better understand the structure of an object.
The visibility structure of a smooth object is analyzed in Section~\ref{sec:structure}.
This structure is then used in Section~\ref{sec:singleblock} to develop a test for when 
one object blocks another, and in Section~\ref{sec:smallerprobeset} 
to develop a smaller probe set.
Section~\ref{sec:cooperation} develops an even smaller probe set,
motivated by considering how an object can be blocked by a sequence of objects.
This set is developed algorithmically.
\ifJournal % GOOD ONE
A proof of the sufficiency of this probe set is given.
Section~\ref{sec:eg} provides examples to illustrate 
our analysis of smooth object visibility.
Section~\ref{sec:complexity} analyzes the complexity of the three probing strategies, and
\fi
Section~\ref{sec:thatsitfolks} ends with some concluding thoughts.
\ifJournal An appendix solves an important subproblem: does a point lie inside a concavity? \fi

Note that in this paper, the only tangents that are considered are
tangents through the viewpoint $P$,
and all of these tangents have an implicit direction, from $P$ to the point of tangency,
reflecting the act of viewing from $P$.
\ifJournal
For example, in the 
degenerate case where there are points of tangency in both directions from $P$
(Figure~\ref{fig:eg14}),\footnote{That is, at a bitangent that passes through $P$.}
the tangent needs to be treated as two directed rays, one in each direction from $P$.
% objectvis -l 3 -v -.36 -.2 -z 1.8 data/eg14.pts
\begin{figure}
\begin{center}
\includegraphics[width=2in]{img/eg14.jpg}
\end{center}
\caption{A directed ray is needed to distinguish these two views}
\label{fig:eg14}
\end{figure}
\fi
As a result of the directionality of every tangent,
it is well defined to talk about the first intersection or the first point of tangency.

% ----------------------------

\section{Related work}
\label{sec:related}

\Comment{
% polygonal object visibility, getting to the idea of virtual vertex
A consideration of object visibility in a polygonal scene will reveal
some of the issues that must be resolved.
% Consider object visibility in a polygonal scene.
Suppose that we are analyzing the visibility of the polygon A.
Since a change of visibility must occur at a vertex, it is sufficient to probe
at all vertices of the scene.
This can be improved by limiting to probes that lie in the angular range covered by A.
There is a sense that this set of probes is still too large, 
         % figure foo (need implementation of polygonal probing)
since some vertices clearly do not add information.
However, care must be taken in removing intermediate vertices, 
since some intermediate vertices are important.
 % figure of polygonal version of piercing vertex
There are also wasted probes when several objects overlap in a probe direction
(Figure~\ref{fig:eg18complex}a).
The optimization of probes to deal with overlapping objects has been well studied
for polygonal scenes in the visibility graph literature (e.g., \cite{welzl85}).
For example, in building the edges incident to a vertex surrounded by several rings of
objects, only the innermost ring needs to be probed.
The visibility graph method is to sweep a line across the scene, keeping track of the
visible object.
Our method of Section~\ref{sec:cooperation} can be viewed as a smooth analogue 
to this sweep.
}

% START HERE: Get a huge bibliography (one double-sided page).  
% Read or reference the associated literature on visibility graph; 
%             occlusion culling (FIND IT), 
%             visibility skeletons, discontinuity mesh (though this is a higher level
% 	    global problem), aspect graphs, shadow computation, surveillance;
% 	    our previous work on representing tangent space 
%	    and finding tangents through a point.
% Compare against other methods in thought experiment: polygonal scene, visibility 
% skeleton, aspect graph, discontinuity mesh, occlusion culling.

% visibility graph
% data structures
% occlusion culling
% shadows
% aspect graphs
% exact computation
% counting visual events
% silhouettes

An early topic of visibility analysis was the visibility graph for polygonal scenes
(e.g., \cite{welzl85}), used in shortest path motion.
Its computation shares some issues with this paper, namely a sweep across the objects
looking for visual events.
The main differences are the treatment of curved scenes and the focus
on one object, which introduces new issues and allows certain speedups (such as skipping
several visual events, and backtracking if this skipping is too aggressive).

The recent work on exact and robust computations with smooth objects 
(e.g., \cite{eigenwillig04,emiris04,lazard04}) is an invaluable complement to
the lifting of visibility analysis to smooth scenes,
since the robust tangent classification inherent to smooth visibility analysis
benefits from robust operations such as intersection.
%  it is useful for intersection and other computations to be robust
% during the tangent classification inherent in smooth visibility analysis.
%
There has recently been work on counting visual events in limited smooth scenes
(e.g., scenes of unit balls \cite{devillers02}).

% from NSF proposal: see results/NSF02---/NSFbody01.tex for references

The tangential analysis used in smooth object visibility
demands that we find tangents through the viewpoint.
The tangents of a curve that pass through a point can be found 
using intersection in dual space, as follows \cite{jj01b}.
Since a line dualizes to a point, the tangent space of a curve can be dualized to a curve.
In \cite{jj01a}, we show how to represent the tangent space of a Bezier curve robustly by
two Bezier curves in two different dual spaces.
Since a point P dualizes to a line dual(P), the tangents of the curve C that pass
through P dualize to the intersections of the curve dual(C) with dual(P).
% This is a line-curve intersection, or a simplified version of a curve-curve 
% intersection, which can be solved using Bezier subdivision.

There is a vast literature on visibility analysis, and it has received an excellent
review in Durand \cite{durand00a}.
% \ifTalk Unfortunately (or fortunately for this paper!) \fi
A vast majority of this literature is for polygonal or polyhedral scenes.
We highlight a few of the recent developments,
especially as they relate to smooth scenes and as they shed light on this paper.

% visibility skeleton: Durand and Durand thesis (DONE)
Many sophisticated data structures have been developed for visibility analysis in
polygonal and polyhedral scenes, including 
the visibility complex \cite{pocchiola96,durand96,durand97b}, % pocchiola in 2-space
and the visibility skeleton \cite{durand97a}.
Extension of this work to smooth scenes has begun
(e.g., \cite{durand97b} for smooth convex objects) but is still restricted in coverage.
We are interested in arbitrary smooth scenes in 2-space.

The vision community has some work on smooth scenes in the form of aspect graphs
\cite{koenderink76,ponce90,chen91,petitjean92,eggert93}.
However, this work is either restricted 
(for example, \cite{eggert93} applies only to solids of revolution) 
or qualitative and at an abstract, topological, and nonalgorithmic level.
% Gigus and Malik IEEE PAMI Feb. 1990 was a polyhedral aspect graph paper
We are interested in explicit algorithmic development, including an implementation.

The development of visibility analysis has been pushed forward by 
much important work on occlusion culling of polygonal scenes, such as Teller's 
preprocessing from 2d maps \cite{teller91} and antipenumbra \cite{teller92}.
Recent work on occlusion culling has concentrated on conservative methods
that compute potentially visible sets (e.g., \cite{durand00b, leyvand03}).
% Teller and Hanrahan 93 may be one of the first to use conservative techniques
We are interested in exact visibility 
despite its added demands. % added complexity
After all, if an approximate solution is sufficient,
then piecewise linear approximations to the smooth scene would also probably suffice.

Shadow computation is an important special case of visibility analysis 
(when is the light visible?),
and recent emphasis has been on soft shadows.
This work has been approximate and/or in image space, striving for interactivity 
especially in game environments.%(e.g., \cite{siggraph04,siggraph05 soft shadow volumes}).
We are interested in an object space solution.
% discontinuity mesh for global illumination \cite{Sillion book?, get classic SIGGRAPH reference}.

\Comment{
The study of silhouettes is relevant to visibility, since visual events
occur at silhouettes, and here there has been work with smooth objects, 
such as subdivision surfaces \cite{zorin00}.
}

% Our final method is a discrete form of a sweep, but it is unconventional in that
% the sweep is not monotonic: it backtracks occasionally.
 
% ----------------------------

\section{The structure of an object relative to visibility}
\label{sec:structure}

The first step towards an understanding of smooth object visibility is an understanding
of the relevant tangents of a smooth object.
These are the extreme tangents of an object,
the piercing tangents of an object,
and the extreme tangents of a concavity, which are defined in this section.
% the structure of a smooth object, as it relates to visibility.

% \subsection{Extreme tangents}

\begin{defn2}
Let $T$ be a tangent of the object $B$ through the viewpoint $P$.
$T$ is an {\bf extreme} tangent of $B$ if all of $B$ lies on one side of $T$
(Figure~\ref{fig:extreme}).\footnote{Computationally, 
  the tangent has no intersections with $B$.}
\end{defn2}

% objectvis -l 5 -v -.65 -1 -z 1.8 data/eg19A.pts > foobar [show extreme]
\begin{figure}[h]
\begin{center}
\includegraphics[width=2in]{img/eg19extreme.jpg}
\end{center}
\caption{Extreme tangents}
\label{fig:extreme}
\end{figure}

\begin{defn2}
\label{defn:behind}
The {\bf region blocked behind an object} $B$ is the region bounded by 
the two extreme tangents of $B$ and the back-facing curve segment of $B$ between them.
% This is also called an {\bf invisible zone of type 1}.
\end{defn2}

% EXAMPLE OF REGION BLOCKED BEHIND AN OBJECT (JUST CONNECTING THE PTS OF TANGENCY
% BY A STRAIGHT SEGMENT RATHER THAN A CURVE SEGMENT, IF DESIRED)

% This region is part of a cone, which may occasionally be called an {\bf invisibility cone}.
% or even invisible zone 1?

\ifJournal
An exact computation of this region is not necessary for our method
(only its angular range is necessary), but for completeness we describe
how it could be computed.
% We don't need to compute the exact boundary of the invisible region to answer
% our question of the visibility of A; however, since other applications may be interested,
% we explore this issue.
The region behind $B$ lies inside the extreme tangents of $B$
(the inside of the extreme tangent is the side that contains $B$).
% and the invisible region lies inside both extreme tangents.
The only question is the definition of the back-facing curve segment of $B$.
Let $T_1$ and $T_2$ be the two extreme tangents of B, and
$B(t_i)$ be the first point of tangency of $T_i$ with $B$.
The extreme tangents define two curve segments: 
$B[t_1,t_2]$ and $B[t_2,t_1]$.
One probe is sufficient to determine which is the desired back-facing segment:
cast a ray from $P$ towards $B$ 
% (any ray that is not extreme will do: 
(the ray midway between the extreme tangent rays of $A$ is a robust choice)
and find its first intersection with $B$.  
The front-facing segment of $B$ is the segment that contains the parameter value
of this intersection.

FIGURE OF EXACT INVISIBLE ZONE 
\fi

For an analysis of object visibility,
we do not need to compute the region exactly:
we only need to know its angular range and its relative depth.
The depth is defined in Section~\ref{sec:singleblock}.
% steradians in 3D
\Comment{
The angular range is measured about the viewpoint,
effectively defining the region in polar coordinates, 
with the viewpoint as origin.
}

% Every object B blocks a region of space behind it,
% and every concavity associated with a piercing tangent of B blocks a region of space
% behind it (but in front of other parts of B).
% These define regions of invisibility.
% The next definition captures the angular range (about the viewpoint) of 
% these regions of invisibility.

% definition of angular range for zone 1 invisibility (definition 9, first part)

\begin{defn2}
Consider the space of rays emanating from the viewpoint, identifying
a ray with its angle from the $x$-axis so that $[0,2\pi]$ encodes
all of the rays.
{\bf ray($\theta$)} represent the ray at angle $\theta$.
\end{defn2}
\begin{defn2} 
\label{defn:objectrange}
Let $T_1$ and $T_2$ be the extreme tangents of the object $B$
with angles $\theta_1 < \theta_2$.
The {\bf angular range of B} (Figure~\ref{fig:range}) is:
\begin{eqnarray*}
\mbox{range}(B) & :=[\theta_1,\theta_2] & \mbox{if ray} 
(\frac{\theta_1 + \theta_2}{2})
\mbox{ intersects } B \\
              & := [\theta_2, \theta_1] & otherwise
\end{eqnarray*}
% The cone spanned by range(B) is called the {\bf cone of B}.
\end{defn2}

In this definition and elsewhere in the paper,
we use the following interpretation for a closed interval.

% moved from STAR1
\begin{defn2}
Consider a closed interval $[b_0,b_1]$ and a subinterval $[t_i,t_j]$.
If $t_i > t_j$, $[t_i,t_j]$ represents $[t_i,b_1] \cup [b_0,t_j]$.
In the same way, if $B(t), t \in [b_0,b_1]$ is a closed curve,
the segment {\bf $B[t_i,t_j]$} 
represents the segment $B[t_i,b_1] \cup B[b_0,t_j]$ if $t_i > t_j$.
\end{defn2}

% objectvis -l 9 -v -.65 -1 -z 1.8 data/eg19A.pts > foobar [show angular range of star]
\begin{figure}
\begin{center}
\includegraphics[width=1in]{img/eg19angrange.jpg}
\end{center}
\caption{Angular range of an object}
\label{fig:range}
\end{figure}

% ----------------------------

% \subsection{Concavities and piercing tangents}

An object can hide behind another object and extreme tangents 
capture this type of blockage.
An object can also hide inside a concavity,
and piercing tangents capture this type of blockage.
We are only interested in concavities in front of the object
with respect to the viewpoint, since these are the concavities that influence visibility.

% Concavities are defined in terms of piercing tangents, which bound the 
% concavity,\footnote{This is a boundary with respect to visibility, 
%  not the typical boundary in terms of the object's convex hull.}

\begin{defn2}
Let $T$ be a tangent of the object $B$ through the viewpoint $P$,
where the point of tangency of $T$ with $B$ is not an inflection point.
  % ray(T,T') is the infinite ray of T emanating from T' that does not include P.
$T$ is a {\bf piercing} tangent of $B$ % in the direction T' from P
if
  % $PT' \cap B = \emptyset$
  % the segment of T between P and T' = pot(T) is free of intersections with B
  % [the early segment is free]
  %
  % and $ray(T,T') \cap B \neq \emptyset$.\footnote{A point of tangency of ray(T,T') 
  % with B would be enough.}
  % T intersects B past pot(T) = T' 
  % [the late segment is not free]
it intersects $B$ after its first point of tangency with $B$, but not before
(Figure~\ref{fig:piercing}).\footnote{A degenerate case
% Equivalently, $\seg{PT'}$ is free of intersection or tangency and T hits B.
  is possible,
  in which the point of intersection is replaced by another point of tangency, as follows.
  % But B must lie on the correct side of the tangent.
  $T$ is a degenerate piercing tangent of $B$ if it is tangent to $B$ twice,
  $B$ lies on the same side of $T$ at both points of tangency,
  and $T$ does not intersect $B$ before its first point of tangency 
  or between the points of tangency.}
\end{defn2}

\ifJournal
Figure of degenerate piercing tangent (EG9)
\fi

\ifJournal
The avoidance of inflection points in the definition of piercing tangent
is explained by Figure~\ref{fig:piercingdegenerate}.
\fi

% objectvis -l 5 -v -.65 -1 -z 1.8 data/eg19A.pts [show piercing]
% objectvis -l 10 -v .4 -.8 -z 1.8 data/eg8A.pts
\begin{figure}
\begin{center}
\includegraphics[width=1in]{img/eg19piercing.jpg}
\includegraphics[width=1in]{img/eg8piercing.jpg}
\end{center}
\caption{Piercing tangents}
\label{fig:piercing}
\end{figure}

\ifJournal
% objectvis -s 1 -v 0 -2 ../../data/doiseea/sneak.pts (before defn of piercing changed)
% objectvis data/eg9.pts
\begin{figure}
\begin{center}
\includegraphics[width=1in]{img/piercingInflection.jpg}
% \includegraphics[width=2in]{img/eg9.jpg}
\end{center}
\caption{(a) A piercing tangent at an inflection point should not be counted
         (b) Degenerate piercing tangent (EG9)}
\label{fig:piercingdegenerate}
\end{figure}
\fi
 
% INCLUDE EXAMPLE OF NOT PIERCING (WHERE IT HITS B BEFORE PT OF TANGENCY)
\ifJournal
The tangent in Figure~\ref{} is not piercing: EG9B.
\fi

\begin{defn2}
\label{defn:inside}
Let $T$ be a tangent of the object $B$ and $Q$ its first point of tangency with $B$
(i.e., the point of tangency closest to the viewpoint),
and assume that $Q$ is not an inflection point.
The {\bf inside} of $T$ is the halfplane that contains $B$
in the neighbourhood of $Q$.
% STAR1
\end{defn2}

\begin{defn2} 
\label{defn:concavity}
Let $T$ be a piercing tangent of the object $B$,
where $B$ is represented by the closed parametric curve $B(t), t \in I$.
The {\bf concavity} associated with $T$ (Figure~\ref{fig:concavity}) is
% T {\bf defines the concavity} 
\begin{eqnarray}
   B[t_2,t_1]  & \mbox{if } B(t_2 + \epsilon) \mbox{ lies inside T} \\
   B[t_1,t_2]  & \mbox{if } B(t_2 + \epsilon) \mbox{ lies outside T}
\end{eqnarray}
where $t_1 \in I$ is the parameter value of the first point of tangency of $T$
and $t_2 \in I$ is the parameter value of the first point of intersection of $T$ 
with $B$.\footnote{Or the parameter value of the second point of tangency, 
  if $T$ is degenerate.}
% The region bounded by a concavity and its piercing tangent is also
% called an {\bf invisible zone of type 2}.
\end{defn2}

% objectvis -v -.6 -1 -z 1.8 -l 10 data/eg3.pts
% objectvis -l 10 -v .4 -.8 -z 1.8 data/eg8A.pts
\begin{figure}
\begin{center}
\includegraphics[width=1in]{img/eg3concavity.jpg}
\includegraphics[width=1in]{img/eg8concavity.jpg}
\end{center}
\caption{(a) The concavity associated with a piercing tangent.
         (b) The concavity may extend beyond the associated piercing tangent.}
\label{fig:concavity}
\end{figure}

The concavity associated with a piercing tangent defines a region blocked from the
viewpoint.
% a repeat but worth repeating (and a slightly different take on it):
Concavities behind and on the side of an object are not covered
by this definition, since they are not seen from the viewpoint.
These concavities are included in the region behind the object, and do
not need special attention.
 
Just as we defined an angular range that spans an object, using extreme tangents,
we will define an angular range that spans a concavity.

% We have seen that extreme tangents define a region of invisibility behind an object.
% Piercing tangents and their associated concavities also define regions of invisibility.
% But to define this region of invisibility,
% we need to define the partners of a piercing tangent, 
% which are tangents through the viewpoint bounding the concavity.

\begin{defn2}
Let $B$ be an object represented by the closed parametric curve $B(t), t \in I$.
Let $T$ be a piercing tangent of $B$.
% such that $T$ defines the concavity $B(t), t \in J \subset I$.
The {\bf extreme tangents of the concavity} associated with $T$  
are the two tangents of the concavity through the viewpoint $P$
that minimize and maximize angle (Figure~\ref{fig:conextr}).\footnote{Another 
  characterization of these tangents
  is that they are the two tangents of the concavity through the viewpoint
  that lie closest to the extreme tangents of the entire object.}
\end{defn2}

% The practical way to identify the extreme tangents of a concavity
% is to consider the finite number of tangents
% of the concavity that pass through the viewpoint, 
% as well as the two extreme tangents of the object B,
% sort the angles of these tangents between the two extreme tangents, and choose
% the two tangents whose angles lie closest to the angles of the two extreme tangents.
% (We could also characterize the extreme tangents of the concavity as the only tangents
% in the candidate set such that all of the concavity lies on one side of T,
% which is consistent with the definition of extreme tangents of an object.
% However, finding the extreme tangents of the concavity using this characterization
% would require counting the number of intersections of each tangent with B,
% a more expensive test than sorting angles.)
%% The {\bf partner of T} is a tangent of the concavity $B'(t_0), t_0 \in J$ that passes
%% through the viewpoint P and maximizes the angle from T.

% objectvis -v -.6 -1 -z 1.8 -l 10 data/eg3.pts
% objectvis -l 10 -v .4 -.8 -z 1.8 data/eg8A.pts
\begin{figure}
\begin{center}
\includegraphics[width=1in]{img/eg3conextr.jpg}
\includegraphics[width=1in]{img/eg8conextr.jpg}
\end{center}
\caption{The two extreme tangents of a concavity.}
\label{fig:conextr}
\end{figure}


% The entire concavity lies between the piercing tangent T and its partner.
% The entire object lies between its two extreme tangents.

% angular range of invisible zone 2 (definition 9, second part)

\begin{defn2}
\label{defn:concavityrange}
Let $T$ be a piercing tangent of the object $B$,
and let $T_1$ and $T_2$ be the extreme tangents of the associated concavity $C$
with angles $\theta_1 < \theta_2$,
where we measure this angle using the ray from the viewpoint.
The {\bf angular range of $T$'s concavity} (Figure~\ref{fig:conrange}) is:
\begin{eqnarray*}
\mbox{range}(B,T) & := [\theta_1,\theta_2] & \mbox{if ray } 
(\frac{\theta_1 + \theta_2}{2}) \mbox{ intersects $C$} \\
              & := [\theta_2, \theta_1] & otherwise
\end{eqnarray*}
\end{defn2}

% FIGURE OF range(B,T) (EG5)

% objectvis -l 10 -v .4 -.8 -z 1.8 data/eg8A.pts
\begin{figure}
\begin{center}
\includegraphics[width=1in]{img/eg3conrange.jpg}
\includegraphics[width=1in]{img/eg8conrange.jpg}
\end{center}
\caption{The angular range of a concavity.}
\label{fig:conrange}
\end{figure}

  % A subtle issue in object visibility is the blocking of the viewpoint by a concavity.
Concavities complicate the analysis of object visibility.
In the absence of concavities, one probe to $A$ is sufficient to determine 
the visibility of $A$ relative to $B$ (Lemma~\ref{lem:extremeblock}).
This is not the case if part of $A$ lies in a concavity of $B$ 
(Figure~\ref{fig:appearing}b):
some of the probes may indicate that $A$ is blocked and others that $A$ is not.
Without a better understanding of concavities, 
a single probe to $A$ does not necessarily provide any information about the position of 
$A$ relative to $B$.
% We can only have confidence in the blocked probes if we better understand the 
% concavity

% In general, if a probe to A hits B before A, we can usually conclude that A is blocked
% by B 

% \subsection{Local probes}

Deciding whether a tangent of $B$ through the viewpoint is extreme or piercing
requires one probe using this tangent.
Since this probe only needs to compute intersections with $B$, not with the entire scene,
we call it a local probe to distinguish it from the more expensive
probes that are used to test visibility.
The probes used in visibility analysis are now called global probes.

\begin{defn2}
Let $R$ be a ray from the viewpoint, tangent to $B$.
A {\bf local probe} in the direction $R$ determines the first intersection of $R$ with $B$
(ignoring points of tangency).
A {\bf global probe} in the direction $R$ determines the first intersection of $R$ with 
any object in the scene.
A local probe involves the intersection of a line with $B$, 
while a global probe involves the intersection of a line with all objects in the scene
(or at least all objects in the scene whose angular range overlaps $R$).
\end{defn2}

\section{Testing whether one object blocks another}
\label{sec:singleblock}

To establish the usefulness of extreme and piercing tangents 
for visibility analysis,
we now use them to answer the following question:
% develop a test for a single object B blocking A.  That is, 
when is $A$ invisible due to a single object $B$?
This single-object blocking test is also a component of the final algorithm 
for object visibility (Section~\ref{sec:cooperation}), which tests whether
$A$ is invisible due to several cooperating objects.

There are two ways for an object $B$ to block an object $A$: $A$ lies behind $B$ 
or $A$ lies behind a front concavity of $B$.
$A$ must lie in the angular range of $B$ or the concavity,
but of course this is not sufficient.
% since the space in front of an object is not blocked from the viewpoint
% even though it lies in the correct angular range.
%  to determine the invisibility of A, of course,
% since A may lie in front of the object.
We must also measure the relative depth of the objects, which is done using probes.
Consider first whether $A$ lies behind $B$.

\begin{lemma}
\label{lem:extremeblock}
Let $A$ and $B$ be two objects, and
let $R$ be a ray from the viewpoint to any point of $a \in A$ 
strictly inside the extreme tangents.
% Let $R$ be a ray from P with angle in $\mbox{range}(B) - \cup_i \mbox{range}(B,T_i)$
% (that is, a ray to B that avoids all of the concavities).
  % Let $R_1$ be the ray from P midway between the extreme tangent rays of A.\footnote{We
  %   could test intersection of the extreme tangents with B, but this is less robust
  %   than testing intersection with $R_1$
  %   for boundary conditions when an extreme tangent of A is also tangent to B.}
  % The average of the extreme tangents (which are already known)
  % works well as the line of sight to A, since it is as far as possible from a
  % fragile boundary condition.
$A$ {\bf lies behind $B$}
if the following conditions are satisfied (Figure~\ref{fig:block}a):
\begin{itemize}
\item $\mbox{range}(A) \subset \mbox{range}(B)$
\item R intersects $B$ before $A$
\item $a$ does not lie in any concavity of $B$\footnote{Here as elsewhere, a concavity
    refers to a concavity associated with a piercing tangent.}
    % (associated with a piercing tangent of $B$)
\end{itemize}
\end{lemma}
\prf
If we can guarantee that one point of $A$ lies behind $B$
(i.e., in the region of Definition~\ref{defn:behind}),
then the first condition will guarantee that {\em all of $A$} lies behind $B$,
since $A$ and $B$ are disjoint.
The second and third conditions guarantee that the point $a$ lies behind $B$.
Its line of sight is blocked by $B$ and it does not lie in any concavity.
\QED

\vspace{-.3in}

% objectvis -v 0 -1 -z 1.8 -s 1 data/concavityAD.pts [behind block]
% objectvis -l 10 -v 1 -.8 -z 1.8 -s 1 data/eg8.pts [concavity block]
\begin{figure}[h]
\begin{center}
\includegraphics[width=1in]{img/concavityADblock.jpg}
\includegraphics[width=1in]{img/eg8block.jpg}
\end{center}
\caption{An object (in bold) can be blocked by another in two ways: 
         (a) an object lies behind another;
         (b) an object lies inside a concavity of another 
             (the other object does not, even though its angular range lies within
              the angular range of the concavity)}
\label{fig:block}
\end{figure}

% Testing whether a point lies inside a concavity is discussed in the appendix.

Testing whether a point lies inside a concavity 
is a fundamental operation in smooth object visibility analysis.
Let $C$ be the boundary of a concavity 
associated with a piercing tangent of the object $B$ (Definition~\ref{defn:concavity}).
The point $Q$ in free space lies inside this concavity
if and only if the segment \seg{PQ} from the viewpoint $P$
eventually hits $C$, and once it hits $C$, all future intersections with the curve $B$ 
are with $C$.
% Let $B$ be an object, $P$ the viewpoint, 
% $T$ a piercing tangent of $B$, $C$ the associated concavity as a region
% of space, and $c(t)$, $t \in I$ the boundary curve of the concavity.
% Consider a point $Q$ in free space and the ray \ray{PQ} from the viewpoint.
% $Q$ lies in the concavity $C$ if \ray{PQ} intersects the concavity's boundary $c(t)$,
% say first at $P_C$, and all intersections of \seg{P_C Q} with $B$ 
% lie on the concavity's boundary curve.
That is, once the ray from the viewpoint 
hits the concavity, it only hits the concavity from then on to $Q$ 
(Figure~\ref{fig:inconcavity}).
[A proof is provided in the appendix.]

% objectvis -v .4 -.8 -z 1.8 -s 1 data/eg8AB.pts [concavity block]
% objectvis -l 7 -s 1 -v -.65 -1 -z 1.8 data/eg19AE.pts [too much going on, otherwise a good eg]
\begin{figure}
\begin{center}
\includegraphics[width=1in]{img/eg8inconcavity.jpg}
\end{center}
\caption{Testing whether a point lies in a concavity}
\label{fig:inconcavity}
\end{figure}

Only two local probes are required for 
the single-object blockage test of Lemma~\ref{lem:extremeblock}.
The lemma's second constraint requires a local probe of $A$ and a local probe of $B$,
both by the probe $R$.
The third constraint can piggyback off of the same local probe of $B$.
% (see the appendix).
Therefore, the entire test of Lemma~\ref{lem:extremeblock} requires two local probes,
after preprocessing the scene to find its angular ranges and concavities.
Notice that finding an angular range of an object takes one local probe 
(Definition~\ref{defn:objectrange}),
finding a concavity takes one local probe (Definition~\ref{defn:concavity}),
and finding the angular range of a concavity takes one local probe (Definition~\ref{defn:concavityrange}).

\Comment{
Note that the test of Lemma~\ref{lem:extremeblock}
is much simpler than the naive interpretation of the problem as an object
location problem 'does $A$ lie completely inside the region blocked behind $B$?'
% which could be approached using curve intersections with the boundary of the region.
% This may seem like a point location problem for all of the points of $A$.
}

The test whether an object lies inside a concavity is similar to the test of 
Lemma~\ref{lem:extremeblock}.
It involves three local probes.

\begin{lemma}
\label{lem:inconcavity}
Let $A$ and $B$ be two objects.
% and let $E_1$ and $E_2$ be the first points of tangency of the extreme tangents of $A$.
$A$ {\bf lies inside the concavity} associated with the piercing tangent $T$ of $B$ 
% is blocked by the concavity
if the following conditions are satisfied (Figure~\ref{fig:block}b):
\begin{itemize}
\item $\mbox{range}(A) \subset \mbox{range}(B,T)$
\item one of the points of $A$ lies inside the concavity
\item the piercing tangent $T$ intersects $B$ before $A$
\end{itemize}
\end{lemma}
\prf
If one point lies inside the concavity, the only way for $A$ to leave the concavity
is through its mouth, where it will cross the piercing tangent.
% note that $A$ could leave through the mouth even if its two extreme points lie in
% the concavity, so testing the two extreme points is irrelevant.
\QED

The first condition technically does not add anything (that is, the second and third
conditions are necessary and sufficient), but it is a useful inexpensive
filter for many tests.
The second and third conditions can both be tested using a probe at the piercing tangent
if this probe intersects $A$, for a total of two local probes.
However, if this probe does not intersect $A$, since the cone of angles in the
angular range of $A$ is a bit generous (see Figure~\ref{fig:conrange}b), a third probe
is necessary to determine that one point of $A$ is in the concavity.

% For the object A to be invisible, it must be blocked by some object or by some
% collection of objects.  

% Every object blocks a certain amount of space from the viewpoint.
% This blockage comes in two varieties: an area behind the entire object
% and an area behind a concavity at the front of the object.
% TOO MUCH INFORMATION TOO SOON (NOT RELEVANT NOW):
% The invisible area behind the object is bounded by two tangents (called extreme)
% and a segment of the curve,
% while each invisible area in front of the object is bounded by a single tangent
% (called piercing) and a segment of the curve.

% FIGURE TO ILLUSTRATE INVISIBLE REGIONS

% The approach of the paper is as follows.
% Let P be the viewpoint and A be the distinguished object.
% Each object B of the scene blocks the viewpoint from a region behind itself,
% and each concavity of B that is visible from the viewpoint blocks
% another region behind the concavity (but in front of the rest of the object).

We have established a connection between extreme/piercing tangents and visibility
analysis, using them to define ranges, concavities, and certain probes
in testing if $A$ is invisible due to a single object.
However, in general, an object is invisible because of many objects, not just one
(Figure~\ref{fig:blockconcert}).
We next show that extreme and piercing tangents can act as a probe set for 
general object visibility.

% objectvis -v .45 -.75 -s 1 -z 1.8 data/eg15.pts 
% objectvis -v 0.5 -1 -s 1 -z 1.8 data/eg16ABD.pts
\begin{figure}[h]
\begin{center}
\includegraphics[width=1in]{img/eg15.jpg}
\includegraphics[width=1in]{img/eg16ABDblock.jpg}
\end{center}
\caption{(a) An object (in bold) blocked by many objects in concert, all blocking behind.
         (b) An object blocked by two objects, both blocking by concavity.}
\label{fig:blockconcert}
\end{figure}

% ----------------------------

\section{Probing at extreme and piercing tangents}
\label{sec:smallerprobeset}

It turns out that, to test an object's visibility,
it is sufficient to probe at all of the extreme and piercing tangents.
This probe set is a subset of the original probe set (tangents through the 
viewpoint), and potentially a large reduction in size 
(e.g., Figure~\ref{fig:eg18complex}).
Rather than establish the sufficiency of this new probe set, we will present 
an even better probe set in Section~\ref{sec:cooperation} 
(a certain subset of the extreme and piercing tangents) and establish its sufficiency.
We now formalize the two larger probe sets.

\begin{defn2}
A set of global probes is {\bf sufficient} if one of the probes in the probe set 
will see $A$ whenever $A$ is visible.
\end{defn2}

\noindent {\bf Tangent-through-viewpoint probes}: The tangents through 
the viewpoint that lie
in the angular range of an object $A$ define a sufficient probe set for testing the 
visibility of $A$.

\noindent {\bf Extreme/piercing probes}: The extreme\footnote{Here we only use the extreme tangents of an object, not the extreme
  tangents of a concavity.}
and piercing tangents that lie in the
angular range of an object $A$ define a sufficient probe set for testing the 
visibility of $A$.

\vspace{.1in}

Even though we create a smaller probe set in Section~\ref{sec:cooperation}, 
the tangent-through-viewpoint probe set
and the extreme/piercing probe set can both be useful, for their simplicity.
For example, if there are not many tangents through the viewpoint in the 
angular range of A, it may be faster to test all of them rather than 
using the more sophisticated choices.
Or if there are many tangents through the viewpoint but few extreme/piercing tangents
(because all of the complexity is on the backside of the objects away from the viewpoint),
the extreme/piercing probe set is appropriate.
In other words, we can think of the three probe sets as a progression of choices,
moving on to the next probe set only when the present probe set is large.

\ifJournal % GOOD ONE
Issue to examine in journal paper: 
Testing the visibility of {\em all} objects in the scene: 
since we can share probes, this is not much more expensive than testing the visibility
of a single object.
That is, the observation is that the tangents through the viewpoint and the extreme
and piercing tangents don't change when you consider another object A and its visibility.
\fi

\section{The smallest probe set}
\label{sec:cooperation}
% {Blocked by many cooperating objects}

\ifJournal % GOOD ONE
Proof that this is the smallest probe set.
Want to incorporate change to PROBE first: choose largest object before A, 
not first object.
\fi

This section develops a small probe set for testing object visibility,
a certain subset of the extreme and piercing tangents.
It is developed by exploring how an object can be blocked by a sequence of several 
objects, cooperating together to share the blocking task.

In general, an object $A$ is invisible because it is blocked by 
a collection of objects.
We need to probe at the boundaries between objects, where the responsibility
for blocking is handed off from one object to another.
One of the purposes of these probes is to reveal the next object in the chain.
With each probe, we are interested in the objects encountered before the probe hits $A$.
These are the candidates for the next object in the chain.
The other purpose of the probes is to ensure that $A$ does not slip between two objects in the chain.
Since the overlap between two objects is a vulnerable location for visibility,
more than one probe will sometimes be necessary at an overlap.
% two probes?

% objectvis -s 0 -v -.1 -.9 -z 1.8 -R data/eg18.pts (with backtrack call commented out)
% objectvis -s 0 -v -.1 -.9 -z 1.8 -R data/eg18long.pts 
% objectvis -s 0 -v -.1 -.9 -z 1.8 -R data/eg18.pts 
% objectvis -s 0 -v -.65 -1 -z 1.8 -R data/eg19.pts
\begin{figure}
\begin{center}
\includegraphics[width=1.5in]{img/eg18reversevisprobeWObacktrack.jpg} % was 2in
\includegraphics[width=1.5in]{img/eg18longreversevisprobe.jpg}
\includegraphics[width=1.5in]{img/eg18reversevisprobe.jpg}
\includegraphics[width=1.5in]{img/eg19ABCFreverseprobe.jpg}
\end{center}
\caption{The probes involved in testing visibility. (a) does not use backtracking. 
        (b) illustrates the need for backtracking (the witness to visibility is darkened).
        (c) adds backtracking to (a). 
        (d) illustrates the need for probes at piercing tangents.}
\label{fig:probe}
\end{figure}

Probes serve as witnesses to potentially large areas of invisibility.
For example, if a probe determines that $A$ lies inside a concavity of $B$,
then we know that the next opportunity for $A$ to become visible is as it leaves
the concavity, at a piercing tangent.
Or, if a probe determines that $A$ lies completely behind $B$,
then we know that the next opportunity for $A$ to become visible is on the other side
of $B$, at an extreme tangent of $B$, since $A$ cannot escape 
from behind $B$ before then to become visible.
Other probes are not (immediately) necessary in the intervening range.
They may become necessary later depending on the result of the 
probe at the other side of $B$.
Let us explore why these extra probes are sometimes necessary.

Suppose that the present object in the chain of blocking objects is $B$,
and the next probe $R$ (at a tangent of $B$)
reveals that the next object in the chain is $C$.
% Suppose that we have established that $B$ blocks $A$ until a certain probe
% (which will be a tangent to $B$),
% and at this probe $C$ is hit before $A$.
Care must be taken in handing off responsibility for blocking from $B$ to $C$.
Suppose that the probe $R$ hits $A$ before the point of tangency with $B$ 
(see the second and third probes of Figure~\ref{fig:probe}a).
This implies that $A$ is beginning to squeeze between $B$ and $C$,
and we must check if it continues far enough to peek out on the other side of $C$
and become visible (as it does in Figure~\ref{fig:probe}b).
In particular, we must backtrack and probe $C$ at its extremity that overlaps $B$;
and there are three possibilities:
if this probe hits $A$ first, then we have a witness to visibility;
if it does not hit $A$ or it hits $A$ after it hits $B$, 
then all is fine and we can continue forwards again;
if it hits $A$ before $B$ but hits another object $D$ first,
then $A$ has peeked out past $C$ but is blocked by $D$, 
and we must continue backtracking to see if $A$ peeks out the other side of $D$.
In theory, this backtracking may continue indefinitely.

The complete algorithm to determine the visibility of the smooth object $A$
from a viewpoint $P$ in a scene of smooth objects is given in Table~\ref{table:objvis}.
% in all its glory and all its gory detail, 
This algorithm is explained further below.
The function {\em objectVisibility} returns true if A is visible from the viewpoint P, 
and false if A is invisible.
Note that P is understood implicitly in this algorithm: all probes are from P and all
extreme/piercing tangents are computed with respect to P.
If A is visible, {\em witness} is a ray that testifies to the visibility of A.
If A is invisible, {\em chain}  is a list of the blocking objects.
Throughout the algorithm, 
RAY records the next probe (so tracking this variable is a valuable insight 
into the algorithm), 
B the next object in the chain, {\em range} the
angular range of A that is known to be blocked by other objects, and {\em chain} 
the present list of blocking objects.
% setting RANGE is also a useful comment on the progress made so far
% note: by the time we get down to the test of the a-probe lying in a concavity,
%       it is crucial that we know that B does not block A alone, otherwise the next 
%       probe would not necessarily be the associated piercing tangent: this might be
%       sweeping in the wrong direction.
%
% Probes identify certain objects and certain points of A.
% The next definition gives handles for these objects and points.
% Let R be a ray in the angular range of A (i.e., a valid probe direction).
% Recall that probes are always from the viewpoint.
{\bf probe(R)} is the first object that is hit by the ray R from the viewpoint,
and {\bf A-probe(R)} is the first point of A hit by R.

\begin{table}[h]
\caption{The object visibility algorithm}
\label{table:objvis}
\vspace{.2in}
\begin{tabular}{|l|} \hline
bool {\em objectVisibility} (A, \&{\em witness}, \&{\em chain}) \\[1em]

% \begin{description}
RAY = an extreme tangent of A; B = probe(RAY);\\
\ \ \ {\em range} = $\emptyset$; {\em chain} = \{B\};\\
%
if (B == A) \{ {\em witness} = RAY; return 1; \}\\
% \ifJournal: check if this next step is necessary \fi
if (A is blocked by B alone) return 0;\\
if (A-probe(RAY) lies in a concavity of B)\\
\ \ \ RAY = associated piercing tangent of B;\\
else RAY = extreme tangent of B in range(A);\\
{\em range}    = (angle(extreme tangent of A), angle(RAY));\\

while (RAY lies in A's angular range)\\
%  \begin{description}
\ \ \ oldB = B;  B = probe(RAY); {\em chain} = {\em chain} $\cup$ B;\\
\ \ \ if (B == A) \{ {\em witness} = RAY; return 1; \}\\
% note: a-probe cannot lie in concavity, or at least it doesn't matter: we always use extreme tangent of object
\ \ \ if (backtrack (RAY, A, B, oldB, {\em witness}, {\em chain}, {\em range}))\\
\ \ \ \ \ \ \ return 1;\\
\ \ \ RAY1 = extreme tangent of B not in {\em range};\\
\ \ \ RAY2 = closest piercing tangent of B not in {\em range};\\
\ \ \ RAY = RAY1 or RAY2, whichever extends {\em range} least;\\
\ \ \ {\em range} = (angle(extreme tangent of A), angle(RAY));\\
% \end{description}
return 0;\\
% \end{description}
\hline
\end{tabular}
\end{table}

\begin{table}[h]
\caption{The backtrack function}
\label{table:back}
\vspace{.2in}
\begin{tabular}{|l|} \hline
bool backtrack(RAY, A, newB, oldB, \&{\em witness},\\
\ \ \ \ \ \ \ \ \ \ \&{\em chain}, {\em range}) \\[1em]

if (RAY hits A before oldB) \\
\ \ \ RAY = closest extreme/piercing tangent of newB in\\
\ \ \ \ \ {\em range} (i.e., going backward) that sees A or oldB\\
\ \ \ \ \ in the scene containing only A, oldB and newB\\
% \ \ \ RAY1 = extreme tangent of newB in {\em range};\\
% \ \ \ \ \ \ \ \ (i.e., going backward);\\
% \ \ \ RAY2 = closest piercing tangent of newB in {\em range};\\
% \ \ \ RAY = RAY1 or RAY2, whichever is closer;\\
% \ifJournal: add the subtlety that we actually have to keep probing to replace NEWB
% until it actually changes: think of another piercing tangent short of the extremity
% in Figure 1 in the backtrack phase: need to find extremity past which we can see A (or oldB)
% this subtlety is too much for this paper
\ \ \ newB = probe(RAY); {\em chain} = {\em chain} $\cup$ newB;\\
\ \ \ if (newB == A) \{ {\em witness} = RAY; return 1; \}\\
\ \ \ if (newB == oldB) return 0;\\
\ \ \ return backtrack(RAY, A, newB, oldB, {\em witness},\\
\ \ \ \ \ \ \ \  {\em chain}, {\em range});\\
% \end{description}
else return 0;\\
% \end{description}
\hline
\end{tabular}
\end{table}

The algorithm can be explained as follows.
If at any time a probe sees $A$, the algorithm halts and returns this witness,
so we can concentrate on the case when a probe is blocked.
The first probe is at an extreme tangent of $A$, and determines the first blocker $B$
in the chain.
If $B$ does not block $A$ alone (Lemmas~\ref{lem:extremeblock} and \ref{lem:inconcavity}),
then we must search for the next opportunity for $A$ to move out from behind $B$,
and probe there to ensure that another object takes over the blocking responsibility.
The next opportunity for $A$ to move out from behind $B$ is at an extreme tangent
or piercing tangent: we choose the first that is met in the sweep direction
and probe to find the next blocker.
The chain of blocking objects is followed until either $A$ is visible or we establish that
the chain blocks all of $A$.
This handles the forward sweep across $A$.
The backtrack function handles the backward components of the sweep.

The function {\em backtrack} is provided in Table~\ref{table:back}.
After each probe (except the first), we must backtrack to check if $A$ 
squeezes out between the most recent two objects in the chain, as described above.
It returns true if a witness to $A$'s visibility is found
during this backtrack (and this witness is returned in {\em witness}).

\ifJournal
Trace through an example, based on EG18 (Figure~\ref{fig:eg19probe}).
Trace through another example, based on EG19 (apparently in concavities, but test does
not need to change) (Figure~\ref{fig:probe}).
\fi

% (COULD JUST INCLUDE IN JOURNAL PAPER)
\ifJournal
A proof that this algorithm detects visibility follows.
\fi

[In the appendix, we provide a proof that this algorithm 
correctly computes the visibility of the object $A$.]

\ifJournal % GOOD ONE
 % [after the proof, we can improve the algorithm by choosing the object that extends
 % the range best, rather than always the first object; but this overly complicates the
 % original presentation and the proof]
We can improve the algorithm for {\em objectvisibility} by reinterpreting
probe(RAY).
Rather than choosing the first object hit by the probe, we could technically
choose any object that is hit before A: we are just looking for an object that blocks 
the probe.
The optimal choice is the object before A that extends the covered range the most.
We prefer to use this version of the algorithm, where probe(RAY) is the object
hit by the probe RAY before A that most extends the covered range (i.e., whose clockwise
extreme tangent is furthest clockwise).
\fi

In the worst case, the probe set developed by this algorithm is no better than 
probing at all extreme and piercing tangents.
Typically, however, it will make a large reduction in the number of probes,
as seen in Figure~\ref{fig:eg18complex} (and Figure~\ref{fig:eg19probe}).

The algorithm (and its implied computation of extreme and piercing tangents,
concavities, and so on) has been fully implemented.

Care must be taken in computing intersections of a probe with the scene,
since we do not want a point of tangency to masquerade as an intersection.
\ifJournal
% (e.g., eg19ABCF and eg19EF).
Solution is to ignore intersections within some feature size of the point of tangency.
(Except with the first probe, don't ignore: count pt of tangency as first intersection 
if nothing comes before it.)
\fi
%
Also, we can improve the probing algorithm by choosing the object that extends the range
best, rather than always the first object that the probe hits.
These issues are not studied here for lack of space.

% objectvis -s 0 -v .75 -.9 -z 1.8 data/eg18complex.pts
\begin{figure}[h]
\begin{center}
\includegraphics[width=1.2in]{img/eg18complextang.jpg}
\includegraphics[width=1.2in]{img/eg18complexExtPie.jpg}
\includegraphics[width=1.5in]{img/eg18complexvisprobe.jpg}
\end{center}
\caption{The probe algorithm can eliminate the consideration of many piercing and 
         extreme tangents.
	 Finding a visible probe in this complex scene takes only two probes.
         (a) All tangents through the viewpoint.
         (b) All extreme and piercing tangents.
         (c) Probes used by the probing algorithm.}
\label{fig:eg18complex}
\end{figure}

% ----------------------------

\ifJournal % GOOD ONE
\vspace{-.2in}

\section{Examples}
\label{sec:eg}

REAL LIVE NON-TOY TEST CASES? BUILDING FOOTPRINTS IN A REAL CITY OR REAL BLOCK?
SEE VIENNA DATA
\fi

\Comment{
too fragile:
Monkey with object behind hands (nazcamonkeycomplex):
an example with one line of sight feeding through several objects, 
and several false alarms.

% objectvis -v .8 -2 should work too
% objectvis -s 0 -v .65 -1.1 -z 1.8 data/nazca_monkeycomplex.pts: some weird concavities: good test
% case for robustness [temporarily comment out
%          'preparing curve samples on each concavity' which causes problems with 
% 'Assertion `u >= this->knot[i] - smalleps && u <= this->knot[i+1] + smalleps' failed.]
\begin{figure}
\begin{center}
% \includegraphics[width=2in]{img/monkeyprobe.jpg}
\end{center}
\caption{Finding the visible probe for a complex object}
\label{}
\end{figure}
}

% ----------------------------

\Comment{
\section{Complexity}
\label{sec:complexity}

In determining the complexity of smooth object visibility analysis,
it is important to distinguish between local and global probes.
Local probes are used to filter tangents through the viewpoint down 
to extreme and piercing tangents, while global probes are used to test visibility.
Local probes involve intersection of a ray with one object,
whereas global probes involve intersection of a ray with several objects in the scene
(all objects that lie in the angular range of A).
We want to reduce the number of global probes.
The number of local
probes is much less important:
global probes are more expensive than local probes,
and global probes are more prone to recomputation within a dynamic scene where some
objects are moving (?).

How many probes could this algorithm use to diagnose visibility?
Of course, invisibility takes longer to diagnose than visibility, 
since the algorithm does not end early with a positive witness to visibility.
We are probing at one of the extreme rays of $A$ and 
one or both of the extreme rays of each of the objects in a chain of objects 
that block $A$. (AT PIERCING TANGENTS TOO.)
The worst case number of probes is therefore $2n-1$ where there are $n$ objects
in the angular range of $A$.

\ifJournal (?)
Consider the complexity of certain operations used by the object visibility algorithm.
Determining whether A lies behind an individual object 
requires one probe per object (but only for those objects whose angular range
contains A's angular range).
Determining whether A lies inside an individual concavity
requires two probes per concavity (but only for those concavities whose angular range
contains A's angular range).
Determining that A is blocked by a collection of objects 
requires two probes per object in the collection 
(but only for minimal collections of objects that together contain A's angular range).
% Determining that A is blocked by a collection of overlapping concavities
% requires one probe per concavity in the collection and an additional pair of probes 
% (but only for minimal collections of concavities that together contain A's angular range).
% Steps (7-8) may seem to subsume steps (5-6), but notice that the algorithm stops
% once it is established that A is blocked.
% Therefore, we make the simpler tests (5-6) in anticipation of sometimes avoiding 
% the more expensive tests (7-8).
\fi

The complexity of the total algorithm depends on the angular range of $A$ relative
to the angular range of the other objects and concavities.
Obviously, the complexity increases in more complex scenes with more objects and more 
concavities, and when $A$ spans a larger angle with respect to the viewpoint.
Let $n$ be the number of objects whose angular range overlaps $A$,
$c$ the number of concavities whose angular range contains $A$,
$N$ the number of objects that participate in overlapping collections of minimal size
whose angular ranges combine to contain $A$ (where an object is counted once for each
collection that includes it),
and $C$ the number of concavities that participate in overlapping collections
of minimal size whose angular ranges combine to contain $A$.
Then the complexity of this algorithm is $O(n+2c+2N+C)$ probes.
Typically, the angular range of A only overlaps a few objects and concavities
in the scene and a small number of probes are necessary.
\ifJournal % GOOD ONE
ESTABLISH SOME EXPERIMENTAL RESULTS ON NUMBERS OF PROBES.
\fi

We can analyze the complexity in terms of number of local probes and number of global
probes.
Note that the complexity is particularly attractive if the viewpoint and most objects
are static (so there are no local probes to recalculate extreme/piercing tangents),
but the star is dynamic so that there is a lot of computation
somewhere that swamps the precomputation of extreme/piercing tangents.

\ifJournal % GOOD ONE
This suggests test data of detecting visibility of a moving object.
\fi

\ifJournal % GOOD ONE
Implementation improvement:
lazily compute concavities, angular ranges, etc. on demand: only a few objects are
probed and then need more information computed.
\fi

% ----------------------------

% \begin{enumerate}
% \item Define certain concavities of each object.
%      [the visible ones that lie towards the viewpoint]
%      unnecessary for most concavities: only need for the first blocking object B
% \item Find the angular range of A, the region blocked behind each object,
%       and the region blocked by each visible concavity.
%       unnecessary for most objects, only need for A and the first blocking object B
% \item (Determine whether A is blocked behind an individual object.
%        Unnecessary)
% \item (Determine whether A is blocked by a concavity.  Unnecessary: will
%        test for blockage by the first B as part of the algorithm)
% \item Determine whether A is blocked behind a collection of overlapping objects.
% \end{enumerate}

% ----------------------------
} % \Comment{

\section{Conclusions}
\label{sec:thatsitfolks}

This paper has solved a fundamental question in the visibility analysis 
of a smooth scene: when is an object visible from a viewpoint?
A major complicating factor is that there are no restrictions on the scene.
The analysis of smooth object visibility involves the appropriate choice of probes,
which occur at certain special tangents of the scene.
The strength of the proposed method is that we skip many probes.
First, we reduce from all tangents through the viewpoint (in $A$'s range) to
extreme and piercing tangents, and then we skip many of the extreme and piercing tangents
by analyzing how one object blocks another.

We note that testing the visibility of all objects in the scene is not much more expensive
than testing the visibility of one object, since extreme and piercing tangents
can be reused.
The method is also amenable to lazy computation of extreme and piercing tangents,
which can reduce the number of local probes.

The final algorithm is a discrete sweep, with events at extreme and piercing tangents,
% However, unlike most other sweeps, 
% we do not stop at all events, and the sweep may back up. 
that leverages the coherence of invisibility to skip many events.
% if we discover that A is blocked by B, we can skip
% across to the other side of B before probing again. % (modulo the issue of backtracking)
Angular ranges are used to avoid probes in many cases 
(Lemmas~\ref{lem:extremeblock} and \ref{lem:inconcavity}, Table~\ref{table:objvis}).
The advantage of optimized object visibility analysis is felt most keenly
in scenes packed densely with many objects and in scenes with 
complicated objects (containing many tangents through the viewpoint).
\ifJournal
It is also valuable that this paper establishes that only probes at the extreme 
and piercing tangents are necessary, 
so that the above simplified algorithm is indeed possible.
\fi

% To measure object visibility, we must measure an object's depth with respect
% to the viewpoint as well as its angular range.

Aside from the obvious applications of this solution in visibility analysis for
occlusion culling, shadows and so on, a solution to this problem is a component 
of the natural next question: what are {\em all} of the points that see the object A?
This is equivalent to finding the visual events of a smooth scene,
which occur at bitangents of the scene (a bitangent is a line that is tangent 
to two objects).
Indeed, the original motivation for a study of the present problem was
that its solution is necessary for the filtering of bitangents down to visual events.
% A bitangent between A and B defines a visual event for A if the segment between
% A and B is free of intersections and the point of tangency with B does not see A.
This topic will be addressed further in future work.
% We are making progress on this problem, which involves bitangents.

% answer to the question posed in this paper is a component of recognition of visual events
% A topic for future work is the characterization of all points that see an object $A$.

% The paper has also developed a structure for a curve for use in visibility analysis,
% with applications beyond this particular problem.

\ifJournal
% actually, even beyond the journal probably 
future: can we use duality to aid the sweep of the probe algorithm? (like visibility
graph sweep uses duality)
\fi

\ifJournal % GOOD ONE: ALTHOUGH MORE A SEPARATE PAPER
\section{Future work, and application of this paper: Direct visual event}

application to computation of a direct visual event

A direct visual event for A is defined by a free bitangent between A and B if:
A and B lie on the same side and the point of tangency, and
the point of tangency with B does not see A (elsewhere).
\fi

\ifJournal % GOOD ONE
{\em We have a crude conference writeup of tangential curves in SMI01.
We should revisit the journal writeup, incorporating bitangents, tangents through a 
point, visibility graphs, shortest paths, and convex hull.
We need a journal version, both for the record and for its expansion and cleanup.
For example, this journal version would be the appropriate one to reference
in this paper as we compute special tangents through a point, and later when
we need bitangents for analyzing visual events.}

{\em probable second paper: 'Do I see you? Part 2: Visual events in a smooth flat scene'.
Given a scene of objects, an important question to answer is
when each object is completely invisible.
This establishes when the object can be culled from the graphics pipeline,
and is useful in general visibility analysis, such as the computation
of visual events (see \cite{jj05});
such as the computation of visual events (at a valid visual event for A,
A is completely invisible).}
\fi

% -------------------------------

\ifJournal % GOOD ONE (SEPARATE PAPER)
explorable issues: robust versions of line intersection and sideness;
  how far do we want to explore degenerate cases in this paper?
  proofs; [journal]
\fi

\ifJournal % GOOD ONE
\section{Robust computation}

In computing the inside of a tangent or the concavity defined by a piercing tangent,
we test where $B(t + \epsilon)$ lies relative to a line through $B(t)$.
This is a potentially fragile computation: we must choose $\epsilon$ small enough
to guarantee that $B[t,t+\epsilon]$ is all on the same side of the line and hasn't
yet crossed the line again, otherwise the test is meaningless.
The size of $\epsilon$ can be determined using arguments developed in the exact
curve work of SoCG 2004.
EXPLORE.

[For robustness, we may want to choose more central probes in general object visibility:
all of the probes are at tangents, but that's inherent to the method isn't it?.]

computation of fundamental operations (tangents through a point; and some trivial
   ones: line/curve intersection, inside/outside of tangent);
robust computation of fundamental operations (optional)
\fi

\clearpage

\section{Acknowledgements}

The Nazca monkey dataset was provided thanks to Jyh-Ming Lien.

\bibliographystyle{latex8}
% \bibliographystyle{plain}
\begin{thebibliography}{}

\bibitem[Chen 91]{chen91}
Chen, S. and H. Freeman (1991)
On the Characteristic Views of Quadric-Surfaced Solids.
IEEE Workshop on Directions in Automated CAD-Based Vision, 34--43.

\bibitem[Devillers 02]{devillers02}
Devillers, O., V. Dujmovic, H. Everett, X. Goaoc, S. Lazard, H. Na, 
S. Petitjean (2002)
The expected number of 3D visibility events is linear.
INRIA Rapport de Recherche No. 4671.

\bibitem[Durand 96]{durand96}
Durand, F. and G. Drettakis and C. Puech (1996)
The 3D visibility complex: a new approach to the problems of accurate
visibility.
7th Eurographics Workshop on Rendering, Portugal.

\bibitem[Durand 97a]{durand97a}
Durand, F. and G. Drettakis and C. Puech (1997)
The Visibility Skeleton: A Powerful and Efficient Multi-Purpose Global
Visibility Tool.
SIGGRAPH '97, 89--100.

\bibitem[Durand 97b]{durand97b}
Durand, F. and G. Drettakis and C. Puech (1997)
The 3D Visibility Complex: a unified data structure for global
visibility of scenes of polygons and smooth objects.
9th Canadian Conference on Computational Geometry.

% \bibitem[Durand 97]{durand97}
% Durand, F. and G. Drettakis and C. Puech (1997)
% 3D Visibility made visibly simple: an introduction to the Visibility
% Skeleton.
% Video Proceedings of ACM Symposium on Computational Geometry, Nice.

\bibitem[Durand 00a]{durand00a}
Durand, F. (2000)
A Multidisciplinary Survey of Visibility.
SIGGRAPH 2000 Course Notes on Visibility: Problems, Techniques
and Applications.
% Also available from http://graphics.lcs.mit.edu/~fredo/.

\bibitem[Durand 00b]{durand00b}
Durand, F. and G. Drettakis and J. Thollot and C. Puech (2000)
Conservative Visibility Preprocessing using Extended Projections.
SIGGRAPH 2000, 239--248.

\bibitem[Eggert 93]{eggert93}
Eggert, D. and K. Bowyer (1993)
Computing the Perspective Projection Aspect Graph of Solids of Revolution.
IEEE Transactions on Pattern Analysis and Machine Intelligence 15(2), 109--127.

\bibitem[Eigenwillig 04]{eigenwillig04}
Eigenwillig, A., L. Kettner, E. Sch\"omer and N. Wolpert (2004)
Complete, Exact, and Efficient Computations with Cubic Curves.
SoCG '04, 409--418.

\bibitem[Emiris 04]{emiris04}
Emiris, I., A. Kakargias, S. Pion, M. Teillaud and E. Tsigaridas (2004)
Towards an Open Curved Kernel.
SoCG '04, 438--446.

\bibitem[Hertzmann 00]{zorin00}
Hertzmann, A. and D. Zorin (2000)
Illustrating Smooth Surfaces.
SIGGRAPH 2000, 517--526.

\bibitem[Johnstone 01a]{jj01a}
Johnstone, J. (2001)
A Parametric Solution to Common Tangents.
International Conference on Shape Modelling and Applications (SMI2001),
240--249.

\bibitem[Johnstone 01b]{jj01b}
Johnstone, J. (2001)
Smooth Visibility from a Point.
39th Annual ACM Southeast Conference, 296--302.

\bibitem[Koenderink 76]{koenderink76}
Koenderink, J. and A. van Doorn (1976)
The Singularities of the Visual Mapping.
Biological Cybernetics 24, 51--59.

\bibitem[Lazard 04]{lazard04}
Lazard, S., L. Pen\~aranda and S. Petitjean (2004)
SoCG '04, 419--428.

\bibitem[Leyvand 03]{leyvand03}
Leyvand, T., O. Sorkine, and D. Cohen-Or (2003)
Ray space factorization for from-region visibility.
ACM Transactions on Graphics 22(3), 595--604.

\bibitem[Petitjean 92]{petitjean92}
Petitjean, S. and J. Ponce and D. Kriegman (1992)
Computing Exact Aspect Graphs of Curved Objects: Algebraic Surfaces.
International Journal of Computer Vision 9(3), 231--255.

\bibitem[Petitjean 96]{petitjean96}
Petitjean, S. (1996)
The Enumerative Geometry of Projective Algebraic Surfaces and
the Complexity of Aspect Graphs.
International Journal of Computer Vision 19(3), 1--28.

\bibitem[Pocchiola 96]{pocchiola96}
Pocchiola, M. and G. Vegter (1996)
The Visibility Complex.
International Journal of Computational Geometry and Application 6(3),
279--308.

\bibitem[Ponce 90]{ponce90}
Ponce, J. and D. Kriegman (1990)
Computing Exact Aspect Graphs of Curved Objects: Parametric Surfaces.
Proc. of AAAI-90, 1074--1079.

\bibitem[Teller and Sequin 91]{teller91}
Teller, S. and C. S\'{e}quin (1991)
Visibility preprocessing for interactive walkthroughs.
SIGGRAPH '91, 61--69.

\bibitem[Teller 92]{teller92}
Teller, S. (1992)
Computing the Antipenumbra of an Area Light Source.
SIGGRAPH '92, 139--148.

\bibitem[Welzl 85]{welzl85}
Welzl, E. (1985) 
Constructing the visibility graph for n line segments
in $O(n^2)$ time.
Information Processing Letters 20, 167--171.

\end{thebibliography}

% ----------------------------

\section{Appendix}
\label{sec:appendix}

\subsection{Proof of test for point in concavity}
% Does a point lie inside the concavity?
\label{sec:ptinconcavity}

We prove that the point $Q$ in free space lies inside the concavity with boundary $C$
if and only if the segment \seg{PQ} from the viewpoint $P$
eventually hits $C$, and once it hits $C$, all future intersections with the curve $B$ 
are with $C$.
% After all, once the line segment enters the concavity, the only part of the curve
% that it can hit is the concavity's boundary, unless it passes right through the
% concavity and enters the region behind the curve or in another concavity.
The argument is as follows.
If $Q$ lies in front of the curve, then there will be no intersections with
the concavity.
If $Q$ lies behind the curve or in a different concavity,
then the last intersection of \seg{PQ} with the curve will be 
outside the present concavity.
If $Q$ lies inside the concavity,
then \seg{PQ} cannot hit other parts of the curve once it enters the concavity,
which can be seen as follows.
A part of the curve that does not bound the concavity cannot enter the concavity
and cause an intersection with the line segment, since it would need to enter
the concavity through its mouth and would therefore define a new first intersection
of the piercing tangent with the curve and change the concavity 
(see Definition~\ref{defn:concavity}).
Therefore, once the ray enters the concavity, 
all intersections must be with the concavity.

Notice that this test involves only one local probe, to find the intersections with $B$.
% We conclude that one probe is sufficient to determine whether a point lies in a concavity.

\subsection{Proof of algorithm's correctness}

We now prove that the algorithm {\em objectVisibility} correctly computes 
the visibility of the object $A$.
We can assume, without loss of generality, that none of the probes reveal visibility.
(If one of the probes sees $A$, we have clearly computed $A$'s visibility correctly!)
We must show that the entire object is invisible, not just the finite number of points
at probes.
The first probe is an extreme tangent of $A$.
Suppose without loss of generality that it is the extreme tangent of $A$ further
counterclockwise (ccw). 
% the extreme tangent of $A$ that defines the last contact with $A$ as we rotate ccw
(If not, interchange all uses of clockwise and counterclockwise below.)
The first probe establishes that some object $B$ blocks $A$ at this probe.
If $B$ blocks $A$ alone (Section~\ref{sec:singleblock}), then we are done, 
so suppose it does not.
If the point of $A$ associated with the first probe lies inside a concavity of $B$, 
then $A$ must depart this concavity (otherwise $B$ would block $A$ alone) 
and so must cross the piercing tangent.
This is the next opportunity for $A$ to become visible, and defines the next probe in this
case.
If the point of $A$ associated with the first probe does not lie in a concavity of $B$,
then it must lie completely behind $B$.
Since we are at the ccw boundary of $A$, $A$ can only slip past $B$ to become visible
in the clockwise direction and, since the objects are disjoint, 
$A$ cannot move past $B$ until it passes the clockwise extreme tangent of $A$,
which defines the next probe in this case.
We have established that the range between the first two probes is blocked,
and can consider the algorithm's while loop.

When the next probe is chosen,
we will always have established that the angular range from the ccw side of $A$
to that probe is blocked.
Therefore, if a probe does not lie in $A$'s angular range, we know that
the probes have swept across $A$ and can conclude that $A$ is invisible.
The next probe will test the visibility of $A$ as it appears out of the shadow of $B$.
If this probe does not see $A$, it must be blocked by some new object newB.
Consider the region between $B$ (which we shall now call old$B$) and newB.
It is possible for $A$ to slip between oldB and newB and become visible in the 
backwards (ccw) direction (see Figure\ref{fig:probe}b): % eg18longreversevisprobe.jpg
we must verify that this does not occur.
We are now in the call to {\em backtrack}.
If $A$ begins to slip between oldB and newB, then the present probe will hit newB,
then $A$, then oldB: that is, it will hit $A$ before oldB.
The earliest chance for $A$ to slip past newB and become visible is found as follows.
%  is at the ccw
% end of newB (again because two objects are disjoint and $A$ is known to be strictly behind
% newB at present).
% Therefore, the next probe is at the ccw extreme tangent of newB.
% (or equivalently,
% the extreme tangent of newB that lies in the range already considered).
Consider the scene consisting only of newB, A and oldB.
The next position at which A can slip past newB is the first extreme or piercing
tangent of newB that sees A or oldB in this restricted scene.
This becomes the next probe.
If a new object is found that blocks $A$ at this new probe, but $A$ still lies between
this new object and oldB, then we must continue looking backwards (and add the new
object to the list of blockers).
The backtracking continues until we either establish that $A$ becomes visible
or we find a probe that only sees $A$ after oldB, not before, which establishes
that $A$ has stopped squeezing backwards in front of oldB (and has remained blocked
throughout), so it remains invisible and the forward search can continue.
This concludes the call to {\em backtrack}.
The next opportunity for $A$ to become visible past newB in the forward direction
is either at the other extreme tangent of newB % (the right one)
(where $A$ might appear from behind newB)
or at a piercing tangent of newB,
where $A$ might appear from behind a concavity of newB 
(Figure~\ref{fig:probe}d). % eg19ABCFreverseprobe.jpg
The forward search continues with a probe at this tangent.
The search continues in the same way from this new probe, until 
either a witness to $A$'s visibility is found, or the known range
of invisibility matches $A$'s range (meaning $A$ is completely invisible).
% \fi 

\end{document}

\subsection{High level view of rest of the paper}

{\bf for introduction?}:
We establish the usefulness of these tangents for analyzing visibility, by
developing a test for a single object B blocking A.
That is, when is A invisible due to a single object B?
(In general, A is invisible due to several cooperating objects.)
This test is also a component of the final algorithm for object visibility.

% ----------------------------

\section{Object visibility}
\label{sec:ov}

% regions of invisibility, rather than invisiblity zones?

% We now study the conditions on depth that must be satisfied for A to be blocked
% by B or one of its concavities.
% We shall see that only a few probes are required.
% These few probes are sufficient to locate A with respect to B.

% regions blocked by an object
There are many ways for A to be blocked from the viewpoint (Figure * illustrates
some of them), 
but there are only two fundamental ways for an object to block a region of space:
behind itself and behind a concavity (Figure **).
We discuss how to calculate the region blocked by an entire object
and the region blocked by a concavity (Section~\ref{}).
% Then A is blocked if it lies in these areas.
% We develop a simple test that involves angular ranges and a small number of probes
% (our term for testing a line of sight for blockage).
This development may be summarized as follows.
A blocked region associated with the object B (or by a concavity of B)
is bounded by certain tangents of B that pass through the viewpoint,
which define the angular range of the region.
That is, adopting a polar coordinate system centered at the viewpoint,
the blocked regions may be characterized by angular ranges.
A comparison of the angular range spanned by A with the blocked angular ranges
is enough to determine the visibility of A,
along with some probes of A to determine if A lies in front of or behind 
each blocking object.

% ----------------------------

\section{Table scraps}

Look in early stages of early papers on object visibility for discussion of 
point visibility.

\subsection{Introductory paragraph 1}

Given a scene of objects, an important question
is to determine when each object is completely invisible.
This establishes when the object can be culled from the graphics pipeline,
and is an important question to answer for general visibility analysis,
such as the computation of visual events (a potential visual event may be culled
if any part of the object of interest is already visible).
This leads to the following related question: given a viewpoint P and a certain
object A from the scene, is A completely invisible from P?
  % this is a static or local version of a more general question that we will ask later:
  % what are *all* of the viewpoints from which A is completely invisible?
This question has received little attention for scenes of smooth objects.
We consider a solution for an arbitrary scene of smooth curves in 2-space.
The only constraint on the curves is that they define closed, disjoint objects.
The curves may also be interpreted as footprints of 2.5-dimensional surfaces
in 3-space.

% Given a viewpoint in a scene of smooth objects in 2-space, what objects
% are visible?

\subsection{Introductory paragraph 2}

Given a viewpoint P, a scene of smooth objects in 2-space, and a distinguished object A
called the \hero, we want to answer two questions about visibility:
(1) what region of 2-space is invisible from P, and 
(2) is A invisible from P?
The first question is a smooth variant of the hard shadow problem,
and could be rephrased in this framework as a computation of the hard shadow
of a smooth scene from a point light source P.
The major new issue is the adaptation of the algorithm to a smooth scene, where
objects are bounded by smooth closed curves.
Another new issue is the reinterpretation of the problem from shadows to visibility,
since visibility is not 360 degrees but is limited by field of view.
The second question could also be interpreted as a shadow problem: if A is the area
light source, is P in the umbra?
However, since we do not need to compute the entire umbra of A to answer this question,
the question is more accurately a localized variant of the umbra problem.
(2) might also be interpreted as follows: does all of A lie in the region computed by (1)?
This reveals the connection between the two problems.

\subsection{Invisible patches in front of an object (invisible zone II)}

This area can also be captured by a cone (although it is a subset of this cone).
The cone is bounded by the free tangent at one boundary and the extreme
intersecting tangent at the other side of the patch.
Call this an invisible patch cone of the object B.

\subsection{Is A visible from P?}

If A lies entirely in an invisible front patch of B, 
then the extreme tangents of A intersect B between P and A as well as
after A (i.e., after the point of tangency with A).

\subsection{Table scraps from 'chain of blocking objects' section}

Suppose that we have begun to build a chain of objects that block $A$,
and we are looking to add the next object in this chain, using a new probe.
Let probe(R) be a function that, given a ray direction R in the angular range of $A$,
returns the object (if any) that is hit by R before $A$
and best extends the angular range of the blocking objects.
(That is, the union of the new object's angular range with the chain's range 
is maximized.)
This probe function will be used to find the next object in the blocking chain.
For example, probe(--) would return -- in EG18.

If we ignore the issue of backtracking and concavities, the basic algorithm
for object visibility probing is as simple as the following.
Let $T_1$ and $T_2$ be the two extreme tangents of $A$.

\begin{verbatim}
B = A
while (probe has not swept past A)
  B = probe(extreme tangent of B)
\end{verbatim}

or, more precisely:

\begin{verbatim}
B = A
probe R = $T_1$
while (probe has not swept past $T_2$)
  B = probe(R)
  R = extreme tangent of B in the sweep direction
\end{verbatim}

The assumption is that any probe that returns A will abort this routine and
return the probe as a witness to visibility.

Adding the issue of backtracking:
\begin{verbatim}
B = A
while (not swept past A)
  returnB = B = probe(positive extreme tangent of oldB = B)
  while (probe hits A before oldB)
    B = probe (negative extreme tangent of B)
  B = returnB
\end{verbatim}

Here, the positive extreme tangent of B is the extreme tangent of B that lies
in the direction of sweep, or alternatively the one that does not lie in the range
that has already been swept over.

This algorithm neglects the following detail: if the probed point of A lies in
a concavity of B, then the extreme tangent of B is replaced by a piercing tangent of B.

-----

FIGURE

A concavity blocks a region from the object's extreme tangent to the piercing tangent
while an object blocks a region behind it between its two extreme tangents.

We want to find a collection of objects (perhaps of size one) that blocks $A$,
if one exists.
First look for the first object in the chain, which blocks one side of $A$.
Choose an extreme tangent of $A$.
Consider the objects $B_i$ whose range contains the angle of this extreme tangent of $A$
(so that it potentially acts as a bookend of the blockers of $A$)
such that the probe at this angle hits $B_i$ before it hits $A$
(so that it lies in front of $A$ and can block),
and the probe point of $A$ (an extreme point) does not lie in a concavity of $B_i$
($A$ lies strictly behind $B_i$, not sort of behind at a concavity where it can 
squirm out).
We start with the object in this collection with the largest range overlap with $A$.
[Witness analysis: if there are no objects whose range contains the angle of the
extreme tangent of $A$, then the probe to the associated extreme point is a witness
to visibility.
If the probe at this angle hits $A$ first, it is a witness.]
Now consider the probe at the angle of the other extreme tangent of $A$.
If this angle lies in the range of $B_i$ and the probe at this angle hits $B_i$ before $A$
then we are done ($B_i$ blocks $A$ alone).
[We don't need to test the concavity issue at the other side, since we already have
a witness to $A$ lying strictly behind $B_i$.]
Otherwise, probe at the extremity of $B_i$ that lies in the range of $A$.
If this probe hits $A$ first, we have a witness to visibility.
If it does not hit $A$ first, it must hit some other objects before $A$.
Restrict to objects such that the probed point of $A$ does not lie in a concavity?
From these objects, choose the object $B_j$ whose range extends the present blocked
range the most.


% Suppose that we have computed the angle ranges blocked by each object $B_i$ in the scene
% and the angle range blocked by each concavity.
% A is invisible from P if the range of A is contained in the union of the ranges of $B_i$
% and the typical ray from P to A (but not a point of a concavity) intersects B before A;
% or if the range of A is contained in the union of the ranges of the concavities
% and the points of tangency of the extreme tangents of A both lie in the associated
% concavity (that contains the range of A).

% NEED MORE THAN ONE PROBE: PROBABLY ONE PER OBJECT.

% \begin{lemma}
% Let $R$ be a viewpoint ray to any point of $A$, say $a$.
% A is blocked by a collection of objects $\{B_i\}$ if 
% $range(A) \subset \cup_i range(B_i)$ and
% $R$ intersects 
% \end{lemma}

% using a southern colloquialism, it's time to piss on the fire and call the dogs
