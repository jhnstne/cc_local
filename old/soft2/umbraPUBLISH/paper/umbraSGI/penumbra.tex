\title{The smooth penumbra}
% Accumuluting energy during a sweep across an object
% Sweeping an object, accumulating energy
% Energy accumulation during an object sweep
% Painting light along a light's ruled surface to define penumbra
%	- lines colored by intensity

%%%%%%%%%%%%%%%%%%%%%%%%%%%%%%%%%%%%%%%%%%%%%%%%%%%%%%%%%%%%%%%%%%%%%%%%%%%
			
\section{Introduction}

3 categories of lighting: umbra (fully shaded), penumbra (partially lit), 
	and noon (fully lit)

%%%%%%%%%%%%%%%%%%%%%%%%%%%%%%%%%%%%%%%%%%%%%%%%%%%%%%%%%%%%%%%%%%%%%%%%%%%
			
\section{Flawed approaches}

Don't want to cast light from each point of the light: 
	then every point of 2-space
	has an infinite number of light points casting light on it,
	for which it must sum the contributions
	Notice that every light point casts across a halfspace minus the
	areas blocked by obstacles or the neighbouring parts of the light (if
	light is concave)
	
On the other hand, computing the light intensity at every pixel
	in free space is also undesirable (?).
	
%%%%%%%%%%%%%%%%%%%%%%%%%%%%%%%%%%%%%%%%%%%%%%%%%%%%%%%%%%%%%%%%%%%%%%%%%%%
			
\section{General algorithm}

\subsection{Part I: Definition of global penumbra}

Instead, first decompose 2-space into global penumbra and global umbra using bitangents:
	this allows us to ignore all but the global penumbral regions;
	this is done for each obstacle: a point is in the global penumbra
	if it is in at least one local penumbral region and it is not in any
	local umbral region, which would block it;
	this is a form of point location in cells from a line arrangement
			
local umbra 
	(from the bitangents of the light and an obstacle A)
	defn: region of space that is entirely blocked from the light by A
	- elegant definition of umbral region: cell in a line/curve arrangement
	- hacky but simple definition of umbra: intersection of halfspaces defined
		by bitangents and obstacles
	- the hacky definition is simple to implement and simple for point 
		classification, although not optimal efficiency
local penumbra 
	(from the bitangents of the light and an obstacle)
	defn: region of space that is partially blocked from the light by A
local noon
	= 2-space - (local umbra and local penumbra)
	defn: region of space that sees the entire light w.r.t. A
		(i.e., region that is not blocked from the light at all by A)
global umbra 
	= union of local umbras
	defn: region of space that is entirely blocked from the light 
	      (by one or a combination of the obstacles)
*global penumbra*	(the nontrivial one)
	= (union of local penumbras) - global umbra
	defn: region of space that is partially blocked from the light 
	      (by one or a combination of the obstacles)
global noon
	= intersection of local noons
	= 2-space - (global umbra and global penumbra)
	
%%%%%%%%%%%%%%%%%%%%%%%%%%%%%%%%%%%%%%%%%%%%%%%%%%%%%%%%%%%%%%%%%%%%%%%%%%%
			
\subsection{Part II: Definition of light intensity inside global penumbra}

for each point P in a global penumbral region,
  for each obstacle A such that P is inside A's penumbral region,
	intersect tangent of A through P with light, yielding L_A
  choose L = L_A that yields the smallest visible light segment
	(the larger light segments are blocked by one of the other obstacles)
  compute (read off) proportion of light visible from L on the
	outside of tangent
	   - this involves computing tangent PP' from light 
	     through P; then measuring visible (from P) portion of 
	     light from L to P', which is basically arc length
	     of convex hull from L to P'
  assign (paint) this intensity to every pixel (point) on 
	     the tangent from L to P and past P to the next 
	     intersection with an obstacle

%%%%%%%%%%%%%%%%%%%%%%%%%%%%%%%%%%%%%%%%%%%%%%%%%%%%%%%%%%%%%%%%%%%%%%%%%%%
			
\subsection{Part III: Definition of light intensity inside global umbra and global noon}

similarly paint pure black shadow or pure white light
     from pixels (points) that are found in global umbra
     or global noon, by continuing line --- (how?)

%%%%%%%%%%%%%%%%%%%%%%%%%%%%%%%%%%%%%%%%%%%%%%%%%%%%%%%%%%%%%%%%%%%%%%%%%%%
			
\section{Software}

	1) draw B/W umbral regions for a collection of obstacles
	2) draw B/W local penumbral regions for a collection of obstacles
	3) draw B/W global penumbral regions for a collection of obstacles 
	4) draw greyscale local penumbral region for one obstacle
	5) draw greyscale global penumbral regions for a collection of obstacles


	Find the bitangents
	Find the extreme bitangents		
