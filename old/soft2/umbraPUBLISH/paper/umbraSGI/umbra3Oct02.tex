\documentclass[12pt]{article}
\usepackage{latex8}
\usepackage{times}
\usepackage{epsfig}
\input{header}
\newif\ifTalk
\Talkfalse
\setlength{\headsep}{.5in}
\markright{Umbra: \today \hfill}
\pagestyle{myheadings}

\title{Shadows in flatland}
% Bitangency in a scene of smooth curves,\\with applications to visibility}	
% Tangential curves for bitangency, with applications to smooth versions of the convex hull and visibility graph, and anticipations of shading from smooth lights in 3d	
\author{J.K. Johnstone}

\begin{document}
\maketitle

% -----------------------------------------------------------------------------

\begin{abstract}
Bitangency is a fundamental relationship that underlies the study
of scenes of smooth objects in graphics and robotics.
Moving from the independent study of one object to the 
integrated study of several objects in a scene often requires bitangency.

We illustrate how the bitangent is used in 
lighting (building the umbra cast by a smooth light).
\end{abstract}

% -----------------------------------------------------------------------------

\section{Taking umbrage}

\begin{quote}
\ldots but everything exposed by the light becomes visible \ldots (Ephesians 5, 13)
\end{quote}

We are interested in the following analytic lighting problem:
lighting a room full of smooth obstacles by a smooth non-point light source.
This is a very challenging problem, too broad to consider in its fullest 
generality at first blush.
Consequently, in this paper, we prepare the way for an attack on the general problem
by focussing our attention on a more modest problem,
which is already challenging:
computing the umbra cast by a smooth non-point light source in a smooth 
scene in 2-space.
Both the light $L$ and the obstacles that define the scene $\{B_i\}$ 
shall be bounded by smooth closed curves.

\begin{defn2}
The {\bf umbra of a scene} is the locus of points in the scene that do not
see any of the light.
\end{defn2}

A computation of the umbra can be interpreted
more generally as the computation of regions that are entirely invisible
to a distinguished object (in our case, the light),
an important visibility question.
For example, if the light is reinterpreted as the camera,
the 'umbra' defines parts of the scene that do not need to be rendered,
being completely invisible.

We are interested in defining the global structure of the umbra.
That is, we are interested in defining the umbra region by region,
rather than point by point.
It is possible, and popular, to define lighting point by point,
which in our case would reduce to developing the test
'is point P in the umbra?'.
However, we are interested in a deeper understanding of the structure of the umbra,
through the direct definition of the regions that define the umbra.
This is both more efficient and more intellectually satisfying,
but certainly more challenging as well.
Under this approach, most of our attention shall be on the definition of the
boundaries between umbra and non-umbra.

\ifTalk
This distinction between point and regional approaches is analogous to the 
two ways for computing a tangent on a Bezier curve:
computing the tangent at a point
or computing the hodograph of the curve, representing all of the tangents.)
\fi

% ------------------------------------------------------------------------------

{\tiny
\subsection{Earlier work}

Polygonal lighting articles (I believe all of these are point-based).

Tony Woo's work on reachability analysis (using Gaussian sphere).

Artistic literature on shadow and lighting.

How does antipenumbra paper relate?
}	% end of \small

\clearpage

% ------------------------------------------------------------------------------

\subsection{A divide and conquer strategy}

We will compute the umbra of a scene using a divide and conquer strategy,
one obstacle at a time.
First, we compute the umbra cast by each obstacle in isolation
(the local umbra).
We then refine and expand the umbra of each obstacle
by taking other obstacles into account (the global umbra of an obstacle).
Finally, the umbra of the scene is defined in terms of these obstacle umbrae.

The computation of the umbra one obstacle at a time has the advantage
of simplicity and parallelizability.
The computation of the umbra of an obstacle A can be performed at the same
time as the computation of the umbra of an obstacle B, since they are
independent, so our strategy is naturally parallelizable.
This can be an important factor in a large scene.
The concentration on one obstacle at a time is
also an important simplification to the structure of the computation,
since it allows one obstacle to (temporarily) dominate all of the other obstacles.
That is, in the computation of the umbra cast by an obstacle A, 
all obstacles are of interest only through their relationship
to A or the light (such as their bitangents to A or the light)
rather than their mutual interrelationships.
This dominance of one obstacle yields important simplifications when
reasoning about lighting.
% This keeps less balls up in the air at once.
% Thus, we only need to consider n-to-1 relationships, not n-to-n relationships.

There is another serendipitous benefit to a divide and conquer strategy.
Since the umbra is computed obstacle by obstacle,
a point of the umbra will know which obstacle(s) is primarily 
responsible for its inclusion in the umbra.
This will allow efficient incremental updates to the umbra upon the addition
or removal of an obstacle.

Notice that the global umbra of A may certainly overlap the global umbra of B.
That is, the cells of the global umbra of a scene defined by the global umbrae
of obstacles are not necessarily disjoint (they do not form a perfect cover).

% ------------------------------------------------------------------------------

\subsection{A glossary of terms}

\begin{defn2}
\label{defn:outer}
Let A and B be smooth obstacles.
Let $t$ be the tangent through a point $P \in A$,
and let $T$ be a bitangent of A and B, with points of tangency $P_A \in A$ and 
$P_B \in B$.

\begin{itemize}
\item $t$ is {\bf extremal} if $t$ does not intersect A.
	% (i.e., A lies in one of the halfplanes defined by $t$).
\item $T$ is {\bf extremal} if $T$ does not intersect A or B.
\item $T$ is {\bf B-extremal} if $T$ does not intersect B
	and \ray{P_B P_A} does not intersect A
	(i.e., $T$ is allowed to intersect A before $P_B$, as in Figure~\ref{fig:G3}).
\item $T$ is {\bf outer} if it is extremal and 
A and B lie on the same side of $T$ (Figure G4).
\item $T$ is {\bf B-outer} if it is B-extremal and, in the neighbourhood of $P_A$,  
A and B lie on the same side of $T$.
\item $T$ is {\bf inner} if it is extremal and
A and B lie on opposite sides of $T$.
\item $T$ is {\bf B-inner} if it is B-extremal and, in the neighbourhood of $P_A$,
A and B lie on opposite sides of $T$.
\item
The {\bf early segment} of T is the component of T between A and B.
It may be abbreviated early(T).
\item 
If B is the light,
the {\bf late segment} of T is the infinite ray of T starting at
the point of bitangency with A and moving away from the light.
\item
If B is an obstacle that lies between A and the light\footnote{This 
	restriction on B is necessary to distinguish A from B.}
and T intersects L,\footnote{This restriction on T is necessary
	to guarantee that the direction 'moving away from the light'
	is well defined.}
the {\bf late segment} of T is the infinite ray of T starting at the point
of bitangency with A and moving away from the light.
\item
The {\bf inside} of an extremal tangent $t$ of A is
the halfplane defined by $t$ that contains A.
\end{itemize}
\end{defn2}

% ------------------------------------------------------------------------------

\subsection{The local umbra of an obstacle}

\begin{defn2}
The {\bf local umbra of an obstacle} $A$ is the umbra that {\bf would} be
cast if $A$ were the only obstacle in the scene.
The {\bf local umbra of a scene} is the union of the local umbrae of the
obstacles in the scene.
\end{defn2}

The local umbra cast by an obstacle is a region bounded by the obstacle
and two of its bitangents to the light (Figures~\ref{fig:G2} and \ref{fig:G3}).
Which bitangents, and which parts of these bitangents (Figure~\ref{fig:G2a})?

If the light surrounds the obstacle, there is no umbra.
Thus, we assume without loss of generality that the light does not
surround the obstacle.
However, the obstacle could surround the light (Figure~\ref{fig:G3}).
This case complicates the theory.
As we develop the theory, we shall point out how it simplifies when
the obstacle does not surround the light,
which is true in the vast majority of cases.

\begin{defn2}
\label{defn:surround}
A {\bf surrounds} B if B lies entirely inside the convex hull of A (Figure~\ref{fig:G3}).
\end{defn2}

% ------------------------------------------------------------------------------

We are basically looking for bitangents that do not intersect the obstacle
or light.
However, we must relax our constraints
when the obstacle surrounds the light (Figure~\ref{fig:G3}).
In this case, the desired umbral bitangents will intersect the obstacle,
but we only allow these intersections on the 'other' side
of the light.
	% (from the point of bitangency with A).
	% Between the light and obstacle, the umbral bitangent will act 
	% the same as our earlier outer bitangent.

\begin{lemma}
Let A and B be smooth closed curves, where B does not surround A.
\begin{itemize}
\item	A and B have exactly 2 B-outer bitangents and 2 B-inner bitangents.
\item   If A does not surround B,
they have exactly 2 outer bitangents and 2 inner bitangents.\footnote{That is,
	the B-outer bitangents become outer bitangents, and the B-inner 
	become inner.}
\end{itemize}
\end{lemma}
\prf
There is an alternative definition of inner and outer bitangents
that has a more intuitive appeal: the inner and outer bitangents of A and B
are the only bitangents of the convex hull of A and the convex hull of B.
That is, compute the convex hull of A and the convex hull of B, then take
bitangents: there are only 4 bitangents (proof?) and these are the inner and
outer bitangents of A and B.
\QED

The outer bitangents of A and L bound the local umbra of A.

\begin{theorem}
\label{thm:localumbra}
Let $T_1$ and $T_2$ be the late segments of the L-outer bitangents
of an obstacle A and the light L.
The local umbra of A is the region bounded by $T_1$, $T_2$, and 
the relevant part of A.
Of the two regions so defined, the local umbra is inside $T_1$ in the
neighbourhood of $T_1$ and inside $T_2$ in the neighbourhood of $T_2$.
(If the obstacle does not surround the light, we can simplify
this statement:
the local umbra is the region inside both $T_1$ and $T_2$.)
\end{theorem}
\prf
Argue that the light becomes visible as you cross these bitangents.
\QED

If A does not surround L,
the local umbra is bounded if and only if the late segments intersect
(Figures~\ref{fig:G2} and \ref{fig:globalumb3}).
If A surrounds L,
the local umbra is unbounded.

\vspace{.1in}

{\bf Algorithm:}
\begin{enumerate}
\item Find the bitangents of A and L.
\item Find the two bitangents that are L-outer.
\item Find the late segments of these outer bitangents and their insides.
\end{enumerate}

In short, in the general case when the obstacle A does not surround the light L,
the local umbra of A is computed by finding the late segments of
the outer bitangents of A and L.

% \begin{rmk}
% The case described in this section actually subsumes the case of 
% Section~\ref{sec:notsurround}, since an outer bitangent is always
% a pseudo-outer bitangent too.
% That is, you can look for pseudo-outer bitangents
% in all cases.
% We have presented them separately
% since the definition of pseudo-outer bitangents is less intuitive
% without an earlier understanding of outer bitangents,
% and the simpler outer bitangent is sufficient for the vast majority of cases.
% \end{rmk}

\begin{rmk}
More computationally, the local umbra is defined as follows.
Let the two L-outer bitangents be \lyne{P_L P_A} and \lyne{Q_L Q_A},
where $P_L,Q_L \in L$ and $P_A,Q_A \in A$.
If the rays \ray{P_L P_A} and \ray{Q_L Q_A} intersect,
let $P=Q$ be this intersection;
otherwise let $P$ and $Q$ be the points 
at infinity on the rays \ray{P_L P_A} and \ray{Q_L Q_A}, respectively.

The local umbra is the region bounded by the (potentially infinite)
segments \seg{P_A P}, \seg{Q_A Q},
and the curve segment \arc{P_A Q_A} of A.
\end{rmk}

\begin{implementation}
In our implementation, we build all of the scene inside
a predefined cube.\footnote{Equivalently, one can define the scene
	arbitrarily and then find a bounding box.}
In the test for bitangent extremality (Definition~\ref{defn:outer}),
the intersection of an infinite line or ray
with a curve then reduces to the intersection of a finite segment with a curve
(the part of the ray/line inside the box).
This is simpler and more robust.
\end{implementation}

% \begin{figure}
% \caption{A simple local umbra (G1: umbra data/ovoid2.pts)}
% \label{fig:G1}
% \end{figure}

	% umbra data/ob1.pts	[could also try vg9.pts]
\begin{figure}[hb]
\centerline{\epsfig{figure=img/ob1.ps,height=1.567in,width=1.553in}}
% 20% reduction
\caption{Local umbra}
\label{fig:G2}
\end{figure}

	% umbra data/vg10.pts
\begin{figure}[hb]
\centerline{\epsfig{figure=img/vg10.ps,height=1.567in,width=1.553in}}
% 20% reduction
\caption{Local umbra II: obstacle surrounding the light}
\label{fig:G3}
\end{figure}

	% umbra data/ob1.pts	
\begin{figure}[hb]
\centerline{\epsfig{figure=img/ob1-bitang.ps,height=1.542in,width=1.528in}}
% 20% reduction
\caption{There are many bitangents to choose from}
\label{fig:G2a}
\end{figure}

% ------------------------------------------------------------------------------------

\subsection{The global umbra of an obstacle}

\begin{defn2}
\label{defn:global}
\ \ 
\begin{itemize}
\item
The {\bf global umbra of an obstacle $A$} is the umbra cast by $A$,
taking all other obstacles into account.
\item
An {\bf umbral bitangent} of A is a bitangent 
that forms the boundary of the global umbra of A.
\end{itemize}
\end{defn2}

The local umbra captures the umbra of an obstacle in isolation.
However, as shown in Figure~\ref{fig:globalumb2}, 
the umbra cast by a single obstacle can be influenced by other obstacles.
The global umbra expands the local umbra to consider these other obstacles
in the scene.
In many cases, the global umbra is the same as the local umbra.
We shall first determine when the global umbra differs from the local umbra.
Then, we shall show how the expanded umbra is defined in the cases when
the global umbra is larger than the local umbra.

In Theorem~\ref{thm:localumbra}, we saw that
the local umbra is defined by L-outer bitangents of A.
If no obstacle interferes with the early segments of these L-outer bitangents,
then the global umbra is identical to the local umbra.
However, as an obstacle B begins to interfere with these early segments,
the umbra of A is stretched by B (Figure~\ref{fig:globalumb2}).

\begin{lemma}
The global umbra of A differs from the local umbra of A
if and only if an obstacle B intersects the early segment of an L-outer bitangent of 
A and L (Figure~\ref{fig:globalumb2PO}).
\end{lemma}
\prf
\ \ 
\QED

% ------------------------------------------------------------------------------------

\subsubsection{Refining the local umbra}

We define the global umbra by dynamically expanding the local umbra,
through the sweep of a tangent across interfering obstacles.
% There are two types of refinement of the umbra.
The boundary of the global umbra is determined by two sweeps.
Both sweeps start with an L-outer bitangent of the obstacle A and the light L.
This bitangent can be interpreted either as a tangent of A
or as a tangent of L.
The first sweep will move this bitangent smoothly within the tangent space
of A.
The second sweep will move this bitangent smoothly, in the opposite direction,
within the tangent space of L.
Both sweeps move the tangent across all obstacles intervening between A and L.
The first sweep continues until the tangent 'sees' the light (or has swept
completely past the light).
The second sweep continues until the tangent 'sees' the --- (or has swept
completely past it).

We can turn the continuous tangent sweeps into discrete sweeps,
by observing that the tangent must always stop at a bitangent.
In particular, in the first sweep,
the inner bitangents of A with the blocking obstacles define 
the discrete possibilities for stopping position of the sweeping tangent.
In the second sweep, the outer bitangents of L and the blocking obstacles
define the possible stopping positions.
This allows us to avoid a continuous tangent sweep:
we can jump directly to the next bitangent candidate,
(which is easily found, see below), and test for completion.
The resulting 'sweep' involves a finite set of jumps between bitangents.
This is analogous to the line sweep of computational geometry.

In short, the global umbra will be bounded by bitangents,
just like the local umbra.
However, the bitangents that can form the 
boundary of the global umbra is richer than the local umbra:
in addition to the outer bitangents of A and L used by the
local umbra, the global umbra may be bounded by inner bitangents of A and B
(where B is an obstacle intervening between A and L)
or outer bitangents of B and L.

% The pushing of the umbral boundary can continue through several obstacles,
% like a game of tag.
% The pushing will stop when no obstacle intersects the early segment associated
% with the umbral boundary (Figure~\ref{fig:globalumb3}).

\subsubsection{The inner sweep: looking for the light}

Both sweeps define a global umbra bitangent G of A
through refinement of a local umbra bitangent T of A.
(Recall that T is an L-outer bitangent of A and L.)
The first one sweeps in the tangent space of A, as follows.
The tangent is looking for the light, sweeping across obstacles 
that block it from L.

\begin{defn2}
A global umbra bitangent of A generated by an inner sweep from a local umbra 
bitangent of A will be called an {\bf inner umbral bitangent of A}.
\end{defn2}

\begin{defn2}
A global umbra bitangent of A generated by an outer sweep from a local umbra 
bitangent of A will be called an {\bf outer umbral bitangent of A}.
\end{defn2}

The inner umbral bitangents are tangent to A,
while the outer umbral bitangents are not.

\begin{defn2}
This first sweep will also be called an {\bf inner sweep}.
The sweep is inner because the tangent is repeatedly replaced by inner bitangents,
and also because the sweep refines a local umbra bitangent T
by sweeping into its interior side.
\end{defn2}

%
\centerline{innerSweep (T, A, L)}
\begin{enumerate}
\item 
	G = T;\ \ \ done = false;\ \ \ first = true;
	% \item While the umbral bitangent T moves (has been freshly updated on previous step):
\item 	
	sweep direction = direction towards inside of light from T
\item 
	while (!done)
\begin{enumerate}
\item   if (first)
\begin{enumerate}
\item	SEG = early segment of G;\ \ \ first = false;
\end{enumerate}
\item	else
\begin{enumerate}
\item	$G_A = $ G's point of bitangency with A
\item	$G_I = G \cap L$
\item	SEG = segment between $G_A$ and $G_I$
\end{enumerate}
\item 
	$\mbox{HIT} := \{B_i \neq A : B_i \mbox{ is an obstacle that intersects
	SEG} \}$
\item 
	if $\mbox{HIT} = \emptyset$, done = true
\item   
	else
\begin{enumerate}
\item
	choose an arbitrary $B'$ from HIT.
\item 
	\ [search for the light, around the blocker] 
	G = first A-inner bitangent of A and $B'$ that is found
	in the sweep direction.
\end{enumerate}
\end{enumerate}
\item	return(G)
\end{enumerate}

\begin{rmk}
We need a generalization of the inner bitangent, the A-inner bitangent,
to cover the case of an obstacle B that surrounds another obstacle A (Figure H1).
\end{rmk}

Since the first sweep is looking for the light,
it will not only stop when it sees the light,
but also when it sweeps past the light (without yet seeing it).
The latter condition occurs at the 
limit position of an inner bitangent of A and L.

\begin{lemma}
\label{lem:umbraupperbd}
The global umbra of A is {\bf no larger} than the region
bounded by the inner bitangents of A and L.
% (That is, any umbra outside this range is generated by a different obstacle.)
\end{lemma}
\prf
Outside of this range, A has no impact: it doesn't block any of the light.
So A can be ignored outside this region.
\QED

\vspace{-.1in}

\begin{corollary}
The sweep of A's tangent can be stopped at an inner tangent of A and L.
\end{corollary}

This leads to the following addition to the algorithm.

\begin{description}
\item[iii.] If G, the sweeping tangent of A, has jumped past the light
 	(i.e., past an inner bitangent of A and L),
	return (this inner bitangent of A and L).
%      then set it back to the inner bitangent of A and L
%     and done = true.
\end{description}

\begin{example}
Animation of sweep.
\end{example}

Once we find the two global umbra bitangents defined by the first sweep,
we only use their late segments,
just as with the local umbra.

\begin{defn2}
innerSweep(T,A,L), the output of this first sweep from T, is a tangent of A.
It is part of the boundary of the global umbra of A.
The {\bf inside of innerSweep(T,A,L)} is the side that contains A.
\end{defn2}

%%%%%%%%%%%%%%%%%%

\subsubsection{The outer sweep: looking for the global umbra bitangent}

The job of the first sweep is to refine the local umbra bitangent.
The job of the second sweep is to refine the global umbra bitangent
just computed in the first sweep.
In particular, we know that this bitangent defines a boundary of light
visibility at $P_A$ (its point of bitangency with A) and as it leaves A.
But as this bitangent continues away from A (on the late segment of the
bitangent), the light will often become visible once again as the bitangent
crosses out of darkness on the other side of the obstacles
(just as, when flying around the Earth in an airplane, one may pass into night
but then, as you continue travelling, pass back into day).

The search for this reentry into light is done from the perspective of the 
light, using a sweep of the light's tangent, looking for the newly
computed global umbra bitangent G.
When we 'see' G, the present location of L's tangent H in the sweep is used
to refine the boundary of the global umbra.
Namely, the late segment of H (its segment after the intersection with G)
becomes an umbral boundary.

The second sweep moves in the tangent space of L, as follows.
Let T be a local umbra bitangent, and let G be the global umbra bitangent
found by refining T with the first sweep.
As with the first sweep, we leverage the fact that changes of visibility
must occur at the discrete bitangent events.

\begin{defn2}
This second sweep will also be called an {\bf outer sweep}.
The sweep is outer because the tangent is repeatedly replaced by outer
bitangents, and because the sweep refines a local umbra bitangent T
by sweeping into its outside.
\end{defn2}

\centerline{outerSweep (T, A, L)}
\begin{enumerate}
\item	H = T;\ \ G = innerSweep (T,A,L);\ \ done = false;\ \ first = true;
\item	sweep direction = direction towards outside of A from T
\item   while (!done)
\begin{enumerate}
\item   if (first) 
\begin{enumerate}
\item	SEG = early segment of H; \ \ \ first = false;
\end{enumerate}
\item	else
\begin{enumerate}
\item	$H_L = $ H's point of bitangency with the light
\item	$H_I = G \cap H$
\item	SEG = segment between $H_L$ and $H_I$
\end{enumerate}
\item	HIT := $\{ B_i: B_i \mbox{ is an obstacle that intersects SEG}\}$
\item   if $\mbox{HIT} = \emptyset$, done = true;
\item 	else
\begin{enumerate}
\item	choose an arbitrary $B'$ from HIT
\item   \ [search for G, around the blocker] H = first L-outer bitangent of L and $B'$ that is found
	in the sweep direction.
\end{enumerate}			
\end{enumerate}
\item 	return(H)
\end{enumerate}

Note that H is always a tangent of the light L.
Note that SEG is always a finite segment of H, the segment between the light
and G (the bitangent computed in the inner sweep that we are 'looking for').

In the second sweep, a tangent of L is swept across the light, skipping across
obstacles that block it from G.
The tangent is initialized to a local umbra bitangent T, just as with the
first sweep, but we now sweep in the opposite direction.

\begin{defn2}
{\bf outerSweep(T,A,L)}, the output of this second sweep from T.
is a tangent of L.
The {\bf late segment of outerSweep(T,A,L)} is its segment after the intersection with 
innerSweep(T,A,L).
The {\bf inside} of outerSweep(T,A,L) is the side that contains L.
\end{defn2}

The outer sweep can be stopped as soon as the angle of the sweeping tangent H
passes the angle of the inner sweep boundary G,
since one is then guaranteed to never see G.
This adds the following step to the algorithm:

\begin{description}
\item[iii.]	If angle(H) 'exceeds' (in the direction of current travel) angle(G), return (NULL).
\end{description}

% Notice that this may introduce a bend into the umbral boundary.
% (two linked lines)

\begin{theorem}
Let A be an obstacle, L the light, and $\{T_1,T_2\}$ the two L-outer bitangents
of A and L (which define the boundaries of the local umbra).
The global umbra of A is defined by  
the four bitangents:
innerSweep($T_1$,A,L), innerSweep($T_2$,A,L),
outerSweep($T_1$,A,L), and outerSweep($T_2$,A,L).
It lies to the inside of their late segments.

In other words, the global umbra of A is the polygon defined by the intersection
of the inside halfspaces of innerSweep($T_1$,A,L), innerSweep($T_2$,A,L),
outerSweep($T_1$,A,L), and outerSweep($T_2$,A,L).
(See \cite{preparataShamos} for a discussion of the
intersection of halfspaces, including algorithms.)
\end{theorem}

\begin{rmk}
The global umbra of A restricted to a closed polygonal room P is 
the intersection of the 4 inner and outer umbral bitangent halfspaces
and the halfspaces of the room walls.
This is a closed polygon, which simplifies the intermediate and final
computation of the halfspace intersection.
\end{rmk}

\begin{rmk}
We are computing dawns and sunsets, where the light (sun) appears or disappears
over the horizon of an obstacle.
\end{rmk}

\begin{rmk}
The global umbra of an obstacle is 'pseudo-convex' (convex except for boundary of curve),
another simplifying advantage.
\end{rmk}

\begin{rmk}
Since only obstacles that lie between $A$ and the light can influence A's umbra,
we could use a space-partitioning scheme (e.g., octrees) to constrain
the obstacles that need to be considered in the computation of the global
umbra of an obstacle A.
\end{rmk}

% \subsubsection{Penumbrae conspiring to create umbra}

\begin{rmk}
NOT A PROBLEM: in the limit (when obstacles approach but never touch) our solution is correct.
MUST SIMPLY TREAT TOUCHING OBSTACLES DIFFERENTLY.
Consider a light surrounded by circular obstacles, which abut, allowing no
light to pass their ring.
This appears to be a vivid example of the insufficiency of local umbrae for computing
the global umbra.
Notice that the stretching of the umbra addressed in the previous section
is not the cause of the larger global umbra, since no obstacle intersects
another obstacle's PO bitangents.
Instead, this is a case of the penumbrae of obstacles conspiring together
to create umbra.
This is a more subtle umbra to compute, using .3 + .7 = 1 ideas,
rather than 1 + anything else = 1.
\end{rmk}

\begin{lemma}
The umbra of a scene of obstacles $\{B_i\}$ is the union of the 
global umbrae of the obstacles $\cup_i \mbox{globalUmbra}(B_i)$.
\end{lemma}

\begin{figure}[hb]
\caption{Sometimes the global umbra is the union of the local umbrae (umbra data/globalumb1.pts)}
\label{fig:globalumb1}
\end{figure}

	% umbra data/globalumb2.pts
\begin{figure}
\centerline{\epsfig{figure=img/globalumb2.ps,height=2.267in,width=2.279in}}
% 30% reduction
\caption{Global umbrae}
\label{fig:globalumb2}
\end{figure}

	% umbra data/globalumb2.pts
\begin{figure}
\centerline{\epsfig{figure=img/globalumb2-PO.ps,height=2.267in,width=2.279in}}
% 30% reduction
\caption{The crossing of an L-outer bitangent of A and L expands the global umbra}
\label{fig:globalumb2PO}
\end{figure}

\begin{figure}
\caption{Cutting off a new umbral bitangent (umbra data/globalumb4.pts)}
\label{fig:globalumb4}
\end{figure}

\begin{figure}
\centerline{\epsfig{figure=img/globalumb3.ps,height=2.906in,width=2.911in}}
% 40% reduction
\caption{Pushing an umbral bitangent through several obstacles (game of tag)}
\label{fig:globalumb3}
\end{figure}

\section{Dagstuhl Talk}

Sweeping a tangent to (i) compute the umbra and (ii) compute the convex hull.

% ------------------------------------------------------------------------------------

\section{$2 \frac{1}{2}$-D visibility}

Our umbral solution in 2-spae also solves questions of visibility of 2-1/2-D objects in 3-space.

\section{Software}

\begin{itemize}
\item draw B/W umbral regions for a collection of obstacles
\item draw B/W local penumbral regions for a collection of obstacles
\item draw B/W global penumbral regions for a collection of obstacles 
\item draw greyscale local penumbral region for one obstacle
\item draw greyscale global penumbral regions for a collection of obstacles
\item find the extreme bitangents
\end{itemize}

\section{Examples}

Test data.
\begin{itemize}
\item bitangency: circle, ob1, ob3, vg7.pts
\item convex hull: vgraph: vg4.pts, vg1.pts (contour), vg3.pts (contour), vg4.pts (small natural)
	vg5.pts (large natural)
\item lighting: ob1, vg5.pts
\end{itemize}

% ------------------------------------------------------------------------------------

\section{Conclusions}

We don't want to try to compute the entire global umbra all at once:
by breaking it up into the global umbra generated by each obstacle,
the problem becomes tractable.

% ------------------------------------------------------------------------------------

\bibliographystyle{latex8}
% \bibliographystyle{plain}
\begin{thebibliography}{99}

\bibitem{farin97}
Farin, G. (1997)
Curves and Surfaces for CAGD: A Practical Guide (4th edition).
Academic Press (New York).

\end{thebibliography}

\section{Appendix}
\label{sec:appendix}

\subsection{Defining the umbra point by point}
\label{sec:local}

It is simple to determine if a point P is in the umbral shadow cast by a 
single obstacle A (Figure I):
P is in umbra of A iff all of the light's tangents through P (it is enough to test
the extremal two) hit A.
However, several obstacles can conspire to place P in umbra, although it is not
in any one obstacle's umbra (Figure III).
Finding this type of umbral point requires more subtlety.
For example, it is not enough to demand simply that all of the light's tangents through P
hit some obstacle (Figure II).
In effect, we need to test that there are no cracks between obstacles through which
the light becomes visible, by 'sweeping' a beam centered at P across the light.
The following algorithm for testing if P lies in umbra achieves this virtual sweep.

\begin{defn2}
A tangent T of the curve C through a point P is {\bf extremal}
if the entire curve C lies in the same halfplane defined by T
(i.e., T does not intersect C).
The {\bf inside halfplane} of an extremal tangent T of C through a point P is
the halfplane defined by T that contains C.
An extremal tangent is {\bf left-extremal} if its inside halfplane 
contains the point at infinity (1,0,0) along the positive x-axis,
otherwise it is {\bf right-extremal}.
\end{defn2}

Is the point P in the umbra?
\begin{description}
\item[] T $\leftarrow$ left-extremal tangent of the light through P
\item[] while (T has not swept past light \&\& 
       T hits some obstacle A before it hits the light)
\begin{description}
\item[]	T = right-extremal tangent of A through P
\end{description}
\item P is in umbra iff T has swept past light
\end{description}

We are sweeping a line about P searching for a passageway to the light, 
leaping past obstacles that block the light by jumping to their right-extremal tangents, 
until we either find the light or sweep past it.
Figure IV (Figure III with sweep annotations) illustrates this algorithm.

What is the complexity of this algorithm?

% ------------------------------------------------------------------------------------

\centerline{{\bf What is new}}
\begin{enumerate}
\item implementation of algebraic geometry's dual curve that is
\begin{enumerate}
\item robust using 2 dual spaces, and
\item details worked out for Bezier curves, and
\item a primal structure in dual space, unlike dual Bezier curves
\end{enumerate}
\item dual Chaikin
\begin{enumerate}
\item 	don't even evaluate the parts of the curve unaffected by bitangents
		(really be lazy!)  (change the demo to show this)
\item 	analyze the improvement in accuracy as you move through subdivision stages
\item 	when do bad mistakes disappear and refinement only sets in?
\end{enumerate}
\item lazy subdivision idea for subdivision curves
\item application of bitangents to fundamental applications that are only known
	for polygons, not smooth curves
\begin{enumerate}
\item to smooth convex hull, extending Graham scan
\item to smooth visibility graph, improving quality of shortest paths
			[might be able to extend Welzl's algorithm]
\item to smooth lights in 2d, with penumbra and umbra calculation,
	     	anticipating a lifting to smooth lights in 3d
\begin{enumerate}
\item   software to shade the umbra/penumbra
\item 	to determine efficiently whether a curve is in umbra
\item 	to determine efficiently which parts of a curve are in penumbra
\item 	to determine their relative lighting (how much of the light they see)
\end{enumerate}
\end{enumerate}
\end{enumerate}
	     
%%%%%%%%%%%%%%%%%%%%%%%%%%%%%%%%%%%%%%%%%%%%%%%%%%%%%%%%%%%%%%%%%%%%%%%%%%%
%%%%%%%%%%%%%%%%%%%%%%%%%%%%%%%%%%%%%%%%%%%%%%%%%%%%%%%%%%%%%%%%%%%%%%%%%%%
			
\section{Introduction}

3 categories of lighting: umbra (fully shaded), penumbra (partially lit), 
	and noon (fully lit)

\section{Flawed approaches}

Don't want to cast light from each point of the light: 
	then every point of 2-space
	has an infinite number of light points casting light on it,
	for which it must sum the contributions
	Notice that every light point casts across a halfspace minus the
	areas blocked by obstacles or the neighbouring parts of the light (if
	light is concave)
	
On the other hand, computing the light intensity at every pixel
	in free space is also undesirable (?).
	
%%%%%%%%%%%%%%%%%%%%%%%%%%%%%%%%%%%%%%%%%%%%%%%%%%%%%%%%%%%%%%%%%%%%%%%%%%%
			
\section{General algorithm}

\subsection{Part I: Definition of global penumbra}

Instead, first decompose 2-space into global penumbra and global umbra using bitangents:
	this allows us to ignore all but the global penumbral regions;
	this is done for each obstacle: a point is in the global penumbra
	if it is in at least one local penumbral region and it is not in any
	local umbral region, which would block it;
	this is a form of point location in cells from a line arrangement
			
local umbra 
	(from the bitangents of the light and an obstacle A)
	defn: region of space that is entirely blocked from the light by A
	- elegant definition of umbral region: cell in a line/curve arrangement
	- hacky but simple definition of umbra: intersection of halfspaces defined
		by bitangents and obstacles
	- the hacky definition is simple to implement and simple for point 
		classification, although not optimal efficiency
local penumbra 
	(from the bitangents of the light and an obstacle)
	defn: region of space that is partially blocked from the light by A
local noon
	= 2-space - (local umbra and local penumbra)
	defn: region of space that sees the entire light w.r.t. A
		(i.e., region that is not blocked from the light at all by A)
global umbra 
	= union of local umbras
	defn: region of space that is entirely blocked from the light 
	      (by one or a combination of the obstacles)
*global penumbra*	(the nontrivial one)
	= (union of local penumbras) - global umbra
	defn: region of space that is partially blocked from the light 
	      (by one or a combination of the obstacles)
global noon
	= intersection of local noons
	= 2-space - (global umbra and global penumbra)
	
%%%%%%%%%%%%%%%%%%%%%%%%%%%%%%%%%%%%%%%%%%%%%%%%%%%%%%%%%%%%%%%%%%%%%%%%%%%
			
\subsection{Part II: Definition of light intensity inside global penumbra}

for each point P in a global penumbral region,
  for each obstacle A such that P is inside A's penumbral region,
	intersect tangent of A through P with light, yielding $L_A$
  choose L = $L_A$ that yields the smallest visible light segment
	(the larger light segments are blocked by one of the other obstacles)
  compute (read off) proportion of light visible from L on the
	outside of tangent
	   - this involves computing tangent PP' from light 
	     through P; then measuring visible (from P) portion of 
	     light from L to P', which is basically arc length
	     of convex hull from L to P'
  assign (paint) this intensity to every pixel (point) on 
	     the tangent from L to P and past P to the next 
	     intersection with an obstacle

%%%%%%%%%%%%%%%%%%%%%%%%%%%%%%%%%%%%%%%%%%%%%%%%%%%%%%%%%%%%%%%%%%%%%%%%%%%
			
\subsection{Part III: Definition of light intensity inside global umbra and global noon}

similarly paint pure black shadow or pure white light
     from pixels (points) that are found in global umbra
     or global noon, by continuing line --- (how?)

%%%%%%%%%%%%%%%%%%%%%%%%%%%%%%%%%%%%%%%%%%%%%%%%%%%%%%%%%%%%%%%%%%%%%%%%%%%
			
\end{document}
