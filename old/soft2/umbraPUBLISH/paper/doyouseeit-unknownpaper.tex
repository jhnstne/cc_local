\documentclass[12pt]{article}
\usepackage{times}
\usepackage[pdftex]{graphicx}
\makeatletter
\def\@maketitle{\newpage
 \null
 \vskip 2em                   % Vertical space above title.
 \begin{center}
       {\Large\bf \@title \par}  % Title set in \Large size. 
       \vskip .5em               % Vertical space after title.
       {\lineskip .5em           %  each author set in a tabular environment
        \begin{tabular}[t]{c}\@author 
        \end{tabular}\par}                   
  \end{center}
 \par
 \vskip .5em}                 % Vertical space after author
\makeatother

% default values are 
% \parskip=0pt plus1pt
% \parindent=20pt

\newcommand{\SingleSpace}{\edef\baselinestretch{0.9}\Large\normalsize}
\newcommand{\DoubleSpace}{\edef\baselinestretch{1.4}\Large\normalsize}
\newcommand{\Comment}[1]{\relax}  % makes a "comment" (not expanded)
\newcommand{\Heading}[1]{\par\noindent{\bf#1}\nobreak}
\newcommand{\Tail}[1]{\nobreak\par\noindent{\bf#1}}
\newcommand{\QED}{\vrule height 1.4ex width 1.0ex depth -.1ex\ \vspace{.3in}} % square box
\newcommand{\arc}[1]{\mbox{$\stackrel{\frown}{#1}$}}
\newcommand{\lyne}[1]{\mbox{$\stackrel{\leftrightarrow}{#1}$}}
\newcommand{\ray}[1]{\mbox{$\vec{#1}$}}          
\newcommand{\seg}[1]{\mbox{$\overline{#1}$}}
\newcommand{\tab}{\hspace*{.2in}}
\newcommand{\se}{\mbox{$_{\epsilon}$}}  % subscript epsilon
\newcommand{\ie}{\mbox{i.e.}}
\newcommand{\eg}{\mbox{e.\ g.\ }}
\newcommand{\figg}[3]{\begin{figure}[htbp]\vspace{#3}\caption{#2}\label{#1}\end{figure}}
\newcommand{\be}{\begin{equation}}
\newcommand{\ee}{\end{equation}}
\newcommand{\prf}{\noindent{{\bf Proof}:\ \ \ }}
\newcommand{\choice}[2]{\mbox{\footnotesize{$\left( \begin{array}{c} #1 \\ #2 \end{array} \right)$}}}      
\newcommand{\scriptchoice}[2]{\mbox{\scriptsize{$\left( \begin{array}{c} #1 \\ #2 \end{array} \right)$}}}
\newcommand{\tinychoice}[2]{\mbox{\tiny{$\left( \begin{array}{c} #1 \\ #2 \end{array} \right)$}}}
\newcommand{\ddt}{\frac{\partial}{\partial t}}
\newcommand{\Sn}[1]{\mbox{{\bf S}$^{#1}$}}
\newcommand{\calP}[1]{\mbox{{\bf {\cal P}}$^{#1}$}}

\newtheorem{theorem}{Theorem}	
\newtheorem{rmk}[theorem]{Remark}
\newtheorem{example}[theorem]{Example}
\newtheorem{conjecture}[theorem]{Conjecture}
\newtheorem{claim}[theorem]{Claim}
\newtheorem{notation}[theorem]{Notation}
\newtheorem{lemma}[theorem]{Lemma}
\newtheorem{corollary}[theorem]{Corollary}
\newtheorem{defn2}[theorem]{Definition}
\newtheorem{observation}[theorem]{Observation}
\newtheorem{implementation}[theorem]{Implementation note}

% \font\timesr10
% \newfont{\timesroman}{timesr10}
% \timesroman


\newif\ifJournal
\Journalfalse
\newif\ifComment                % large-scale comments
\Commentfalse

\newcommand{\hero}{star}

\setlength{\headsep}{.5in}
\markright{\today \hfill}
\pagestyle{myheadings}

\DoubleSpace

\setlength{\oddsidemargin}{0pt}
\setlength{\topmargin}{0in}	% should be 0pt for 1in
\setlength{\textheight}{8.6in}
\setlength{\textwidth}{6.875in}
\setlength{\columnsep}{5mm}	% width of gutter between columns

% -----------------------------------------------------------------------------

\title{Do you see it?\\Object visibility in a smooth flat scene}
% Part 1
\author{J.K. Johnstone\thanks{Department of Computer and 
    Information Sciences, UAB, Birmingham, AL 35294-1170.  This work
    was partially supported by the National Science Foundation under grant CCR-0203586.}}

\begin{document}
\maketitle

\begin{abstract}
Given a smooth object A nestled in a scene of smooth objects, and
a certain vantage point P, we are interested in analyzing whether A is visible.
If A is not visible, then we want to understand exactly which objects block it;
while if A is visible, we want to provide a probe from the viewpoint P that realizes
this visibility.
This is a smooth version of a classical problem in visibility analysis,
and represents part of a research program that endeavours to promote
our understanding of visibility analysis, which is quite exhaustive for linear structures,
to smooth environments bounded by curves and surfaces.
In this paper, we begin an exploration of this subject 
in 2-space, where a smooth object is an arbitrary closed curve.

Only a few probes are necessary to determine visibility.
We present three granularities of probes, each set smaller than the previous set but
also more complicated to compute.
We first observe that tangents that pass through the viewpoint are sufficient.
We then observe that certain special tangents through the viewpoint are sufficient:
extreme and piercing tangents.
Finally, we refine these probes even further, finding a subset of extreme
and piercing tangents that are sufficient.
We also prove the sufficiency.

paper structure: (1) introduction where necessity for virtual vertices is established
(for polygons, we would probe at the vertices);
(2) establish structure of smooth curves beneficial to visibility (concavity, 
extreme tangents); (3) establish the optimal probes.

APPARENTLY NASTY OBSERVATION, WHICH ACTUALLY INCREASES OUR UNDERSTANDING OF THE
ADVANTAGE OF THE METHOD: we have to (locally) probe at every tangent just in order to diagnose
which tangents are extreme and piercing. But this probe is only intersecting with that
one object (it is local): 
i.e., it is a property of the isolated object, independent of the scene.
If the object and viewpoint are static, then this cost can be considered one-time
preprocessing.
Even if the viewpoint is moving, the cost of a local probe is much less than the
cost of a global probe, so we have a win.
In contrast, the probes for visibility are against the entire scene, that is, they are 
global probes.
We can analyze the complexity in terms of number of local probes and number of global
probes.
Note that the complexity is particularly attractive if the viewpoint and object 
is static (so no local probes to recalculate extreme/piercing tangents),
but a few objects are dynamic (perhaps the star) so that there is a lot of computation
somewhere that swamps the precomputation of extreme/piercing tangents.
{\em In conclusion, argue for the minimization of global probes, even at the expense
of several local probes.  Then argue for the minimization of global probes in a dynamic
scene, where the cost of the local probes is reduced relatively still further.
This suggests test data of detecting visibility of a moving object.}

It is still worthwhile to avoid probes against the entire scene.

{\em We have a crude conference writeup of tangential curves in SMI01.
We should revisit the journal writeup, incorporating bitangents, tangents through a 
point, visibility graphs, shortest paths, and convex hull.
We need a journal version, both for the record and for its expansion and cleanup.
For example, this journal version would be the appropriate one to reference
in this paper as we compute special tangents through a point, and later when
we need bitangents for analyzing visual events.}
\end{abstract}

% probable second paper: 'Do I see you? Part 2: Visual events in a smooth flat scene'

\clearpage

% ----------------------------

\section{Introduction}

IT WOULD BE WORTHWHILE TO PRESENT THE SIMPLIFIED ALGORITHM FIRST
(JUST PROBE AT EXTREME AND PIERCING TANGENTS).
THEN ADD SOPHISTICATION OF HOW TO CHOOSE A SUBSET.
THE SIMPLIFIED ALGORITHM ALREADY HAS POWER.

This paragraph later, after initial introduction of high-level ideas:
A point is visible if the line of sight to the point is free of collision with the scene,
and an object is visible if any point of the object is visible.
Therefore, an object is invisible if every line of sight to the object is blocked.
A line of sight to the object will be called a {\bf probe}.
The implementation of a probe is the intersection of a ray from the viewpoint
with the scene of objects, from the viewpoint to the first point of intersection.
This paper shows how to test the invisibility of an object without using an infinite
number of probes, but rather a well chosen optimal subset.
({\em Review the test of invisibility for a polygon: only need to test all vertices of the
scene whose angle lies in the angular range of the object.
We need a counterpart to this algorithm for a smooth scene, where objects do not
have vertices.
In other words, we must determine a valid set of virtual vertices on the objects.})

Each probe is either additional evidence of invisibility or a witness of visibility.

We aim for a better understanding of object visibility through an
examination of smooth scenes.
If the visibility of an object in a scene is well understood, many important questions
may be answered:
  % An understanding of visibility analysis is central to computer graphics:
occlusion culling, shadow computation, and global illumination are
applications typically cited.
Applications in other disciplines are easy
to find as well, such as camera positioning for surveillance.
  % camera positioning for filming (with dynamic visibility analysis)
  % guard positioning for art gallery theorems (cite O'Rourke)
  % radiation therapy
Given a scene of objects, an important question to answer is
when each object is completely invisible.
This establishes when the object can be culled from the graphics pipeline,
and is useful in general visibility analysis, such as the computation
of visual events (see \cite{jj05}).
  % such as the computation of visual events (at a valid visual event for A,
  % A is completely invisible).
This paper solves the following instance of this problem:
given a smooth object A in a scene of smooth objects in 2-space, and a viewpoint P,
is A completely invisible from P?
  % this is a static or local version of a more general question that we will ask later:
  % what are *all* of the viewpoints from which A is completely invisible?
Visibility analysis has received little attention for scenes of smooth objects.
We consider a solution for an arbitrary scene of smooth curves in 2-space.
The only constraint on the curves is that they define closed, disjoint objects.
The curves may also be interpreted as footprints of 2.5-dimensional surfaces
in 3-space.
Arbitrary curves demand a new level of sophistication in the algorithmic development.

The object A is invisible if every line of sight to a point of A is blocked, but this 
does not lead to a practical algorithm.
We will show that testing the blockage of a small number of lines of sight is sufficient.
This requires a deeper understanding of how an object can block another object.
% Testing a line of sight will be done with a probe.
% We want to reduce to testing one or at most a few lines of sight.

{\em an obvious solution would be to check all visual events in a radial sweep about 
the viewpoint: we do more than this, right?
Would it work to simply probe at every extreme tangent and every piercing tangent
in the range of A?
Yes, if we first do steps 1-3 of the algorithm {\em objectvisibility}, 
since we probe at a subset of these.
The strength of the proposed method is that we skip probes at many of the
tangents through the viewpoint in A's range.
First, we reduce from all tangents through the viewpoint to
extreme and piercing tangents, and then we skip many of the extreme and piercing tangents
by analyzing how one object blocks another.
It is also valuable that this paper establishes that only probes at the extreme 
and piercing tangents are necessary, 
so that the above simplified algorithm is indeed possible.
CAN TRY IMPLEMENTING SIMPLIFIED ALGORITHM FIRST.
GIVE AN EXAMPLE WHERE MANY PROBES ARE SKIPPED.}

There are many ways for A to be blocked from the viewpoint (Figure * illustrates
some of them), 
but there are only two fundamental ways for an object to block a region of space:
behind itself and behind a concavity (Figure **).
We discuss how to calculate the region blocked by an entire object
and the region blocked by a concavity (Section~\ref{}).
% Then A is blocked if it lies in these areas.
% We develop a simple test that involves angular ranges and a small number of probes
% (our term for testing a line of sight for blockage).
This development may be summarized as follows.
A blocked region associated with the object B (or by a concavity of B)
is bounded by certain tangents of B that pass through the viewpoint,
which define the angular range of the region.
That is, adopting a polar coordinate system centered at the viewpoint,
the blocked regions may be characterized by angular ranges.
A comparison of the angular range spanned by A with the blocked angular ranges
is enough to determine the visibility of A,
along with some probes of A to determine if A lies in front of or behind 
each blocking object.

FIGURE * OF A BLOCKED IN VARIOUS WAYS BY THE SCENE: ONE OBJECT BEHIND (EG6), ONE OBJECT'S
CONCAVITY (EG7), SEVERAL OBJECTS BEHIND (EG15, A1), SEVERAL OBJECTS CONCAVITY (EG16, A1)
AND A NOT BLOCKED (EG11).

FIGURE ** OF REGIONS (EG4B and EG5B).

% For the object A to be invisible, it must be blocked by some object or by some
% collection of objects.  

The rest of the paper is organized as follows:
related work;
(definitions); 
definition of concavity;
definition of region behind object;
algorithm (possibly earlier);
angular ranges;
probes to answer various different types of question;
fundamental operations to implement the algorithm;
future work and conclusions.









% Every object blocks a certain amount of space from the viewpoint.
% This blockage comes in two varieties: an area behind the entire object
% and an area behind a concavity at the front of the object.
% TOO MUCH INFORMATION TOO SOON (NOT RELEVANT NOW):
% The invisible area behind the object is bounded by two tangents (called extreme)
% and a segment of the curve,
% while each invisible area in front of the object is bounded by a single tangent
% (called piercing) and a segment of the curve.

% FIGURE TO ILLUSTRATE INVISIBLE REGIONS

% The approach of the paper is as follows.
% Let P be the viewpoint and A be the distinguished object.
% Each object B of the scene blocks the viewpoint from a region behind itself,
% and each concavity of B that is visible from the viewpoint blocks
% another region behind the concavity (but in front of the rest of the object).

% ----------------------------

\section{Related work}

START HERE: read or reference the associated literature on occlusion culling, visibility
            skeletons, aspect graphs, shadow computation, surveillance

visibility graph 

Related literature: occlusion culling (FIND IT), discontinuity mesh (FIND IT),
aspect graph (FIND IT)

Get a huge bibliography (one double-sided page).

% ----------------------------

\section{Definitions} 

The viewpoint P is an arbitrary point in free space.

\begin{defn2}
$x$ is {\bf visible} from $P$ if the open line segment \seg{Px} 
has no intersections or points of tangency with the scene.
\end{defn2}

% moved from STAR1
\begin{defn2}
Consider a closed interval $[b_0,b_1]$ (often interpreted as the parameter interval
of a closed curve) and a subinterval $[t_i,t_j]$.
If $t_i > t_j$, $[t_i,t_j]$ represents $[t_i,b_1] \cup [b_0,t_j]$.
Equivalently, the segment {\bf $B[t_i,t_j]$} 
on the closed curve $B(t), t \in [b_0,b_1]$ represents the segment
$B[t_i,b_1] \cup B[b_0,t_j]$ if $t_i > t_j$.
\end{defn2}

[Should tangents be directed, since cameras have a view direction along a line?
This would also resolve the problem of two equidistant points of tangency, one
on either side of viewpoint P (in definition of T').]

Extremes of visibility and invisibility are defined by tangents through the
viewpoint P.  
In this paper, we only consider tangents through the viewpoint P,
and all of these tangents have an implicit direction, from P to the point of tangency,
reflecting the act of viewing from P.
That is, all tangents can be interpreted as rays emanating from P.
As an illustration of this principle, in the 
degenerate case where there are points of tangency in both directions from P
as in Figure EG14
(i.e., if the tangent is a bitangent of B, and P lies between the points of tangency),
the tangent needs to be treated as two directed rays, one in each direction from P.
Another result of this principle is that it is well defined to talk about
the first point of tangency: this is the first point
of tangency encountered in leaving P in the view direction.

FIGURE SHOWING NEED FOR DIRECTED RAYS (EG14)

Technically, A is blocked by B if all probes to A are blocked by B
(i.e., the probe hits B before A).\footnote{A probe to A is the casting of a ray
  from the viewpoint to a point of A, recording intersections along the way.}

\begin{defn2}
\label{defn:inside}
The {\bf inside} of the tangent T of B
is the halfplane that contains the curve B
in the neighbourhood of T's first point of tangency with B.
% STAR1
\end{defn2}

\begin{defn2}
A {\bf probe} of the scene is the shooting of a ray at the scene from the viewpoint $P$,
analyzing the order in which objects are met.
\end{defn2}

% ----------------------------

\section{Concavities}

  % A subtle issue in object visibility is the blocking of the viewpoint by a concavity.
Concavities complicate the analysis of object visibility.
Often one probe to A is sufficient to determine the visibility of A relative to B,
but this is not the case if part of A lies in a concavity of B (EG11).
Some of the probes may indicate that A is blocked and others that A is not.
Without a better understanding of concavities, 
a single probe to A does not necessarily provide any information about the position of A
relative to B.
% We can only have confidence in the blocked probes if we better understand the 
% concavity

FIGURE OF A PEEKING BETWEEN TWO CONCAVITIES (EG11)
% In general, if a probe to A hits B before A, we can usually conclude that A is blocked
% by B 

Let us begin with a definition of a concavity.
WE ARE ONLY INTERESTED IN CONCAVITIES IN FRONT OF THE OBJECT WRT THE VIEWPOINT.
Concavities are defined in terms of piercing tangents, which bound the 
concavity,\footnote{This is a boundary with respect to visibility, 
  not the typical boundary in terms of the object's convex hull.}
so we start with a definition of the piercing tangent, a certain
tangent through the viewpoint.

\begin{defn2}
Let $T$ be a tangent of the object $B$ through the viewpoint $P$.
  % ray(T,T') is the infinite ray of T emanating from T' that does not include P.
$T$ is a {\bf piercing} tangent of $B$ % in the direction T' from P
if
  % $PT' \cap B = \emptyset$
  % the segment of T between P and T' = pot(T) is free of intersections with B
  % [the early segment is free]
  %
  % and $ray(T,T') \cap B \neq \emptyset$.\footnote{A point of tangency of ray(T,T') 
  % with B would be enough.}
  % T intersects B past pot(T) = T' 
  % [the late segment is not free]
it intersects $B$ after its first point of tangency $T'$ with $B$, but not before.
% Equivalently, $\seg{PT'}$ is free of intersection or tangency and T hits B.
A degenerate case of the piercing tangent is possible,
in which the point of intersection after $T'$ is replaced by a point of tangency,
as follows.
% But B must lie on the correct side of the tangent.
$T$ is a {\bf (degenerate) piercing} tangent of $B$ if it does not intersect $B$ 
before its first point of tangency $T'$ but it is tangent to $B$ again after $T'$, 
say at $T''$,
such that $B$ lies on the same side of $T$ at both points of tangency $T'$ and $T''$
(and $T'T''$ is free of intersections).
\end{defn2}

EXAMPLE OF PIERCING (EG2 and EG8), INCLUDE NOT PIERCING (WHERE IT HITS B BEFORE PT OF TANGENCY)
EXAMPLE OF DEGENERATE PIERCING (EG9)

\begin{defn2} 
Let $T$ be a piercing tangent of the object $B$ (standard or degenerate),
where $B$ is represented by the closed parametric curve $B(t), t \in I$.
The {\bf concavity} associated with $T$ is
% T {\bf defines the concavity} 
\begin{eqnarray}
   B[t_1,t_2]  & \mbox{if } B(t_2 + \epsilon) 
   \mbox{ lies outside\footnote{See Definition~\ref{defn:inside}.} T} \\
   B[t_2,t_1]  & \mbox{if } B(t_2 + \epsilon) \mbox{ lies inside T}
\end{eqnarray}
where $t_1 \in I$ is the parameter value of the first point of tangency $T'$
and $t_2 \in I$ is the parameter value of the first point of intersection of T after $T'$.

The region bounded by a concavity and its piercing tangent is also
called an {\bf invisible zone of type 2}.
\end{defn2}

FIGURE OF CONCAVITY (EG10)

The concavity associated with a piercing tangent defines a region blocked from the
viewpoint.
Notice that concavities behind and on the side of an object are not covered
by this definition, since they are not seen from the viewpoint.
These concavities are included in the region behind the object, and do
not need special attention.

\subsection{Does a point lie inside the concavity?}
\label{sec:ptinconcavity}

A fundamental operation in the algorithmic development below
is testing whether a point lies inside a concavity.
Let $B$ be an object, $T$ a piercing tangent of $B$, $C$ the associated concavity as a region
of space, and $c(t)$, $t \in I$ the boundary curve of the concavity.
Consider a point $Q$ in free space.  Does it lie inside the concavity $C$? (FIGURE EG18)
Consider the line segment \seg{PQ} between the viewpoint and $Q$,
and let $P_B$ be the first intersection with $B$ of \seg{PQ} (moving from $P$ to $Q$).
$Q$ lies in the concavity $C$ if $P_B$ exists and 
all intersections with $B$ of \seg{P_B Q} lie on the concavity's boundary curve $c(t)$.
That is, once it hits the concavity, it only hits the concavity from then on.
% After all, once the line segment enters the concavity, the only part of the curve
% that it can hit is the concavity's boundary, unless it passes right through the
% concavity and enters the region behind the curve or in another concavity.
The argument is as follows.
If the point lies behind the curve or in a different concavity,
then the last intersection with the curve will be outside the present concavity.
If the point lies in front of the curve, then there will be no intersections with
the concavity.
On the other hand, if the point lies inside the concavity,
then it cannot hit other parts of the curve once it enters the concavity,
which can be seen as follows.
A part of the curve that does not bound the concavity cannot enter the concavity
and cause an intersection with the line segment, since it would need to enter
the concavity through its mouth and would therefore define a new first intersection
of the piercing tangent with the curve (see the definition of concavity).
We conclude that one probe is sufficient to determine whether a point lies in a concavity.

% ----------------------------

\section{The region blocked behind an object}

Not every object has a concavity, but every object has a blocked region behind the object.
We now define this other type of blocked region.
Like the concavity,
the blocked region behind an object is defined by certain 
special tangents through the viewpoint.
% The region behind an object needs to be defined, just like we defined a concavity.

\begin{defn2}
Let $T$ be a tangent of the object $B$ through the viewpoint $P$.
$T$ is an {\bf extreme} tangent of $B$ if all of $B$ lies on one side of $T$.
\end{defn2}

EXAMPLE OF EXTREME (EG1) 

\begin{defn2}
The {\bf region blocked behind an object} $B$ is the region bounded by 
the two extreme tangents of $B$ and the back-facing curve segment of $B$ between them.
This is also called an {\bf invisible zone of type 1}.
\end{defn2}

EXAMPLE OF REGION BLOCKED BEHIND AN OBJECT (JUST CONNECTING THE PTS OF TANGENCY
BY A STRAIGHT SEGMENT RATHER THAN A CURVE SEGMENT, IF DESIRED)

% This region is part of a cone, which may occasionally be called an {\bf invisibility cone}.
% or even invisible zone 1?

An exact computation of this region is not necessary for our method
(only its angular range is necessary), but for completeness we describe
how it can be computed.
% We don't need to compute the exact boundary of the invisible region to answer
% our question of the visibility of A; however, since other applications may be interested,
% we explore this issue.
The region behind $B$ lies inside the extreme tangents of $B$
(the inside of the extreme tangent is the side that contains $B$).
% and the invisible region lies inside both extreme tangents.
The only question is the definition of the back-facing curve segment of $B$.
Let $T_1$ and $T_2$ be the two extreme tangents of B, and
$B(t_i)$ be the first point of tangency of $T_i$ with $B$.
The extreme tangents define two curve segments: 
$B[t_1,t_2]$ and $B[t_2,t_1]$.
One probe is sufficient to determine which is the desired back-facing segment:
cast a ray from $P$ towards $B$ 
% (any ray that is not extreme will do: 
(the ray midway between the extreme tangent rays of $A$ is a robust choice)
and find its first intersection with $B$.  
The front-facing segment of $B$ is the segment that contains the parameter value
of this intersection.

FIGURE OF EXACT INVISIBLE ZONE 

% ----------------------------

\section{Angular ranges}

We have seen two types of invisible zone: type 1 invisible zones lie completely
behind an object and type 2 invisible zones lie behind concavities.
However, for the analysis of object visibility,
we do not need to compute each invisible zone exactly:
we only need to know its angular range.
% steradians in 3D
This angular range is measured about the viewpoint,
effectively defining the invisible zone in polar coordinates, 
with the viewpoint as origin.
The angular range is defined here and the depth in the next section.

% Every object B blocks a region of space behind it,
% and every concavity associated with a piercing tangent of B blocks a region of space
% behind it (but in front of other parts of B).
% These define regions of invisibility.
% The next definition captures the angular range (about the viewpoint) of 
% these regions of invisibility.

% definition of angular range for zone 1 invisibility (definition 9, first part)

\begin{defn2}
Consider the space of rays emanating from the viewpoint, identifying
a ray with its angle (say from the x-axis) so that $[0,2\pi]$ encodes
all of the rays.
{\bf ray($\theta$)} represent the ray at angle $\theta$.
\end{defn2}
\begin{defn2} 
Let $T_1$ and $T_2$ be the extreme tangents of the object $B$
with angles $\theta_1 < \theta_2$,
where we measure this angle using the ray from the viewpoint.
The {\bf angular range of B}, range(B), is the range
spanned by these extreme tangents:
\begin{eqnarray*}
\mbox{range}(B) & :=[\theta_1,\theta_2] & \mbox{if ray} 
(\frac{\theta_1 + \theta_2}{2})
\mbox{ intersects } B \\
              & := [\theta_2, \theta_1] & otherwise
\end{eqnarray*}
% The cone spanned by range(B) is called the {\bf cone of B}.
\end{defn2}

FIGURE OF range(B) (EG4)

We want an analogous definition of the angular range for a concavity,
but this requires two tangents through the viewpoint that span the concavity.
These tangents are defined from the concavity's piercing tangent.

% We have seen that extreme tangents define a region of invisibility behind an object.
% Piercing tangents and their associated concavities also define regions of invisibility.
% But to define this region of invisibility,
% we need to define the partners of a piercing tangent, 
% which are tangents through the viewpoint bounding the concavity.

\begin{defn2}
Let $B$ be an object represented by the closed parametric curve $B(t), t \in I$.
Let $T$ be a piercing tangent of $B$
such that $T$ defines the concavity $B(t), t \in J \subset I$.
The {\bf extreme tangents of the concavity} associated with $T$ 
are the two tangents of the concavity 
through the viewpoint P ($B'(t_0), t_0 \in J$ such that $P \in B'(t_0)$)
that minimize and maximize angle.\footnote{Another characterization of these tangents
  is that they are the two tangents of the concavity through the viewpoint
  that lie closest to the extreme tangents of the entire object.}
\end{defn2}

% The practical way to identify the extreme tangents of a concavity
% is to consider the finite number of tangents
% of the concavity that pass through the viewpoint, 
% as well as the two extreme tangents of the object B,
% sort the angles of these tangents between the two extreme tangents, and choose
% the two tangents whose angles lie closest to the angles of the two extreme tangents.
% (We could also characterize the extreme tangents of the concavity as the only tangents
% in the candidate set such that all of the concavity lies on one side of T,
% which is consistent with the definition of extreme tangents of an object.
% However, finding the extreme tangents of the concavity using this characterization
% would require counting the number of intersections of each tangent with B,
% a more expensive test than sorting angles.)
%% The {\bf partner of T} is a tangent of the concavity $B'(t_0), t_0 \in J$ that passes
%% through the viewpoint P and maximizes the angle from T.

FIGURE OF EXTREME TANGENTS OF CONCAVITY (EG10)
% FIGURE OF PARTNER OF T (EG3).

The entire concavity lies in the cone between the two extreme tangents
of the concavity.

% The entire concavity lies between the piercing tangent T and its partner.
% The entire object lies between its two extreme tangents.

% angular range of invisible zone 2 (definition 9, second part)

\begin{defn2}
Let $T$ be a piercing tangent of the object $B$,
and let $T_1$ and $T_2$ be the extreme tangents of the associated concavity
with angles $\theta_1 < \theta_2$,
where we measure this angle using the ray from the viewpoint.
The {\bf angular range of $T$'s concavity}, range($B$,$T$), 
is the range spanned by these extreme tangents:
\begin{eqnarray*}
\mbox{range}(B,T) & := [\theta_1,\theta_2] & \mbox{if ray } 
(\frac{\theta_1 + \theta_2}{2}) \mbox{ intersects the concavity} \\
              & := [\theta_2, \theta_1] & otherwise
\end{eqnarray*}
\end{defn2}

% FIGURE OF range(B,T) (EG5)
FIGURE OF range(B,T) (EG10 with shaded cone)

% ----------------------------

\section{Object visibility}
\label{sec:ov}

% regions of invisibility, rather than invisiblity zones?

We now turn our attention away from invisibility zones
to the main question of object visibility.
To measure object visibility, we must measure an object's depth with respect
to the viewpoint as well as its angular range.\footnote{This is the radial component of the polar coordinates $(r,\theta)$.}
In order for an object A to be completely blocked by another object B 
or by a concavity of B, it must lie in the angular range of B or the concavity.
But this is not sufficient, of course,
since the space in front of an object is not blocked from the viewpoint
even though it lies in the correct angular range.
%  to determine the invisibility of A, of course,
% since A may lie in front of the object.
We now study the conditions on depth that must be satisfied for A to be blocked
by B or one of its concavities.
We shall see that only a few probes are required.
These few probes are sufficient to locate A with respect to B.
DEFINE PROBE.

We first consider the invisibility zone behind an object and then the invisibility zone
inside a concavity.

\begin{lemma}
Let $A$ and $B$ be two objects, and
let $R$ be a ray from the viewpoint to any point of $A$, say $a$.
% Let $R$ be a ray from P with angle in $\mbox{range}(B) - \cup_i \mbox{range}(B,T_i)$
% (that is, a ray to B that avoids all of the concavities).
  % Let $R_1$ be the ray from P midway between the extreme tangent rays of A.\footnote{We
  %   could test intersection of the extreme tangents with B, but this is less robust
  %   than testing intersection with $R_1$
  %   for boundary conditions when an extreme tangent of A is also tangent to B.}
  % The average of the extreme tangents (which are already known)
  % works well as the line of sight to A, since it is as far as possible from a
  % fragile boundary condition.
$A$ is {\bf blocked by $B$} (and lies behind $B$)
if the following conditions are satisfied:
\begin{itemize}
\item $\mbox{range}(A) \subset \mbox{range}(B)$
\item R intersects $B$ before $A$
\item $a$ does not lie in any concavity of $B$ \footnote{That is, any concavity associated
      with a piercing tangent of $B$.}
      (see Section~\ref{sec:ptinconcavity})\footnote{These tests 
       only involve a total of two line intersections with B, combined.}
% \item the ray $R$ intersects B before A (BUT RAY MAY NOT INTERSECT A AT ALL!)
\end{itemize}
\end{lemma}
\prf
We need to guarantee that one point of $A$ lies behind $B$.
Then the first condition will guarantee that {\em all of $A$} lies behind $B$:
if one point of $A$ is behind $B$, then the entire curve is behind $B$
since its angular range is contained within $B$'s
and it cannot overlap $B$ (the curves are disjoint).

The second and third conditions guarantee that the point $a$ lies behind $B$.
Its line of sight is blocked by $B$ and it does not lie in any concavity.

INCOMPLETE
\QED

\begin{lemma}
Let $A$, represented by the curve $A(t)$, and $B$ be two objects, and
let $A(t_1)$ and $A(t_2)$ be the first points of tangency of the extreme
tangents of $A$.
$A$ {\bf lies inside the concavity} associated with the piercing tangent $T$ of $B$ 
% is blocked by the concavity
if the following conditions are satisfied:
\begin{itemize}
\item $\mbox{range}(A) \subset \mbox{range}(B,T)$
\item the points $A(t_1)$ and $A(t_2)$ both lie inside the concavity
\end{itemize}
\end{lemma}
\prf
If both extremities of A lie in the concavity, then all of A does.
It is not enough for one extremity to lie in the concavity, even though
the range of A lies in the concavity range, because the concavity cone
is a bit generous, encompassing regions that are not invisible (EG13).
INCOMPLETE
\QED

FIGURE OF A BLOCKED BY B (EG6)
FIGURE OF A BLOCKED BY CONCAVITY (EG7 and EG13)

% no longer necessary
% \begin{lemma}
% R of the previous lemma must exist.
% That is, the range of the concavities of B cannot cover the range of B.
% \end{lemma}

% FIGURE SHOWING RANGE OF B IS STRICTLY GREATER THAN RANGE OF CONCAVITIES (EG12)

% ----------------------------

\section{Blocked by many cooperating objects}

In general, an object $A$ is invisible because it is blocked by 
a collection of objects, cooperating to block $A$.
This section considers this issue.
What probes are necessary to test invisibility?
We need to probe at the boundaries between objects, where the responsibility
for blocking is handed off from one object to another.
One of the purposes of these probes is to reveal the next object in the chain;
the other is to ensure that $A$ does not slip between two objects in the chain.
Since the overlap between two objects is a vulnerable location for visibility,
two probes will sometimes be necessary at an overlap.

[The probes of the boundary between two objects are at extreme and piercing tangents.
If the present object in the chain is using a concavity to block $A$, 
the next probe is at a piercing tangent.
Otherwise, the next probe is at an extreme tangent.]
% Ones associated with overlapping objects that cover the range of $A$.
% We sweep across the objects to find these objects minimally (or could 
% we consider all unions of covering objects and pick optimal one directly?).
% HOW CAN WE FIND THE BEST COVERING?

Probes serve as witnesses to potentially large areas of invisibility.
For example, if a probe determines that $A$ lies inside a concavity of $B$,
then we know that the next opportunity for $A$ to become visible is as it leaves
the concavity, at a piercing tangent: 
other probes are not (immediately) necessary in the intervening range.
(They may become necessary later depending on the probe at the concavity's mouth.) 
As another example, if a probe determines that $A$ lies completely behind $B$,
then we know that the next opportunity for $A$ to become visible is on the other side
of $B$, at an extreme tangent of $B$, since $A$ cannot escape 
from behind $B$ before then to become visible.

With each probe, we are interested in the objects encountered before the probe hits $A$.
These are the candidates for the next object in the chain.
Care must be taken in handing off responsibility for blocking from $B$ to $C$.
Suppose that the present object in the chain of blocking objects is $B$,
and the next probe (at a tangent of $B$)
reveals that the next object in the chain is $C$.
% Suppose that we have established that $B$ blocks $A$ until a certain probe
% (which will be a tangent to $B$),
% and at this probe $C$ is hit before $A$.
Suppose also that this probe hits $A$ before the point of tangency with $B$.
This implies that $A$ is beginning to squeeze between $B$ and $C$,
and we must check if it continues far enough to peek out on the other side of $C$
and become visible.
Therefore, we must backtrack and probe $C$ at its extremity that overlaps $B$:
if this probe hits $A$ first, then we have a witness to visibility;
if it does not hit $A$ or it hits $A$ after it hits $B$, 
then all is fine and we can continue forwards again;
if it hits $A$ before $B$ but hits another object $D$ first,
then $A$ has peeked out past $C$ but is blocked by $D$, 
and we must continue backtracking to see if $A$ peeks out the other side of $D$.
In theory, this backtracking may continue indefinitely.

Trace through an example, based on EG18.

Trace through another example, based on EG19 (apparently in concavities, but test does
not need to change).

Probes identify certain objects and certain points of A.
The next definition gives handles for these objects and points.
\begin{defn2}
Let R be a ray in the angular range of A (i.e., a valid probe direction).
Recall that probes are always from the viewpoint.
{\bf probe(R)} is the first object that is hit by R.
{\bf A-probe(R)} is the first point of A hit by R.
\end{defn2}

The complete algorithm to determine the visibility of the smooth object $A$
from a viewpoint $P$ in a scene of smooth objects, 
in all its glory and all its gory detail, is as follows.
The function {\em objectVisibility} returns true if A is visible, and false
if A is invisible.
If A is visible, {\em witness} is a ray that testifies to the visibility of A.
If A is invisible, {\em chain}  is a list of the blocking objects.
RAY is the next probe, B is the next object in the chain, {\em range} is the present
angular range of A that is known to be covered, and {\em chain} is the present list of the
blocking objects.
% setting RANGE is also a useful comment on the progress made so far
% note: by the time we get down to the test of the a-probe lying in a concavity,
%       it is crucial that we know that B does not block A alone, otherwise the next 
%       probe would not necessarily be the associated piercing tangent: this might be
%       sweeping in the wrong direction.

\vspace{.2in}

bool {\em objectVisibility} (A, {\em witness}, {\em chain})

\begin{enumerate}
\item RAY = an extreme tangent of A; B = probe(RAY); {\em range} = $\emptyset$; 
      {\em chain} = \{B\};
\item if (probe(RAY) == A) \{ {\em witness} = RAY; return 1; \}
\item if (A is blocked by B alone) return 0;
\item if (A-probe(RAY) lies in a concavity of B) RAY = associated piercing tangent of B;
\item else RAY = the extreme tangent of B that lies in the angular range of A;
\item {\em range}    = (extreme tangent of A, RAY);

\item while (RAY lies in A's angular range)
  \begin{enumerate}
  \item oldB = B;  B = probe(RAY); {\em chain} = {\em chain} $\cup$ B;
  \item if (probe(RAY) == A) \{ {\em witness} = RAY; return 1; \}
  % note: a-probe cannot lie in concavity, or at least it doesn't matter: we always use extreme tangent of object
  \item if (backtrack (RAY, A, B, oldB, {\em witness}, {\em chain}, {\em range})) return 1;
  \item RAY = the extreme tangent of B that does not lie in {\em range};
  \item {\em range} = (extreme tangent of A, RAY);
  \end{enumerate}
\item return 0;
\end{enumerate}

The function {\em backtrack} checks if A 
squeezes out between the most recent two objects in the chain,
by probing backwards.
It returns true if a witness to A's visibility is found
during this backtrack (and this witness is returned in {\em witness}).
Notice that there is no extension of the chain in the original call of {\em backtrack}
from {\em objectVisibility}, but recursive calls may add to the chain.

\vspace{.2in}

bool backtrack(RAY, A, newB, oldB, \&{\em witness}, \&{\em chain}, {\em range})

\begin{enumerate}
\item 
if (RAY hits A before oldB)
\begin{enumerate}
\item {\em chain} = {\em chain} $\cup$ newB;
\item RAY = extreme tangent of newB in {\em range};  (i.e., going backward)
\item if (probe(RAY) == A) \{ {\em witness} = RAY; return 1; \}
\item return backtrack(RAY, A, probe(RAY), oldB, {\em witness}, {\em chain}, {\em range});
\end{enumerate}
\item else return 0;
\end{enumerate}

(COULD JUST INCLUDE IN JOURNAL PAPER)
% \ifJournal
A proof that this algorithm detects visibility follows.

\begin{theorem}
Algorithm {\em objectVisibility} correctly computes the visibility of the object A.
\end{theorem}
\prf  
Suppose that the virtual sweep across A is from left to right, 
starting at the leftmost extreme tangent of A.
(Here a left to right sweep is equivalent to a clockwise sweep.)
Suppose that none of the probes reveal visibility.
We must show that the entire object is invisible, not just these finite points.
The first probe, to the left extreme tangent of A, establishes that some object B
blocks A at this probe.
If B blocks A alone, then we are done, relying on the arguments from the single object
case of Section~\ref{sec:ov}; so we can assume that B does not block A alone.
If the point of A associated with the first probe lies inside a concavity of B, 
then A must depart this concavity (otherwise B would block A alone) 
and so must cross the piercing tangent.
This is the next opportunity for A to become visible, and defines the next probe in this
case.
If the point of A associated with the first probe does not lie in a concavity of B,
then it must lie completely behind B.
Since we are at the left boundary of A, A can only slip past B to become visible
towards the right and, since the objects are disjoint, 
A cannot move past B until it passes the right extreme tangent of A,
which defines the next probe in this case.
We have established that the range between the first two probes is blocked,
and can consider the algorithm's while loop.

When the next probe is chosen,
we will always have established that the angular range from the left side of A
to that probe is blocked.
Therefore, if a probe does not lie in A's angular range, we know that
the probes have swept across A and can conclude that A is invisible.
The next probe will test the visibility of A as it appears out of the shadow of B.
If this probe does not see A, it must be blocked by some new object newB.
Consider the region between B (which we shall now call oldB) and newB.
It is possible for A to slip between oldB and newB and become visible in the 
backwards (leftwards) direction (see Figure SOMEFIGURE): 
we must verify that this does not occur.
(We are now in the call to {\em backtrack}.)
If A begins to slip between oldB and newB, then the present probe will hit newB,
then A, then oldB: that is, it will hit A before oldB.
The earliest chance for A to slip past newB and become visible is at the leftmost
end of newB (again because two objects are disjoint and A is known to be strictly behind
newB at present).
Therefore, the next probe is at the left extreme tangent of newB.
% (or equivalently,
% the extreme tangent of newB that lies in the range already considered).
If a new object is found that blocks A at this new probe, but A still lies between
this new object and oldB, then we must continue looking backwards (and add the new
object to the list of blockers).
The backtracking continues until we either establish that A becomes visible
or we find a probe that only sees A after oldB, not before, which establishes
that A has stopped squeezing backwards in front of oldB (and has remained blocked
throughout), so it remains invisible and the forward search can continue.
This concludes the call to {\em backtrack}.
The forward search continues with a probe at the other extreme tangent of newB
(the right one), which is the next opportunity for A to become visible past newB
in the forward direction.
The search continues in the same way from this new probe, until 
either a witness to A's visibility is found, or the known range
of invisibility matches A's range (meaning A is completely invisible).
\QED
% \fi 

\ifJournal
 % [after the proof, we can improve the algorithm by choosing the object that extends
 % the range best, rather than always the first object; but this overly complicates the
 % original presentation and the proof]
We can improve the algorithm for {\em objectvisibility} by reinterpreting
probe(RAY).
Rather than choosing the first object hit by the probe, we could technically
choose any object that is hit before A: we are just looking for an object that blocks 
the probe.
The optimal choice is the object before A that extends the covered range the most.
We prefer to use this version of the algorithm, where probe(RAY) is the object
hit by the probe RAY before A that most extends the covered range (i.e., whose clockwise
extreme tangent is furthest clockwise).
\fi

\subsection{Complexity}

How many probes could this algorithm use to diagnose visibility?
We observe that invisibility takes longer to diagnose than visibility, 
since the algorithm does not end early with a positive witness to visibility.
We are probing at one of the extreme rays of $A$ and 
one or both of the extreme rays of each of the objects in a chain of objects 
that block $A$.
The worst case number of probes is therefore $2n-1$ where there are $n$ objects
in the angular range of $A$.

{\bf complexity analysis}
Consider the number of probes needed to compute an object's visibility.
Consider the complexity of certain operations used by the object visibility algorithm.
Determining whether A lies behind an individual object 
requires one probe per object (but only for those objects whose angular range
contains A's angular range).
Determining whether A lies inside an individual concavity
requires two probes per concavity (but only for those concavities whose angular range
contains A's angular range).
Determining that A is blocked by a collection of objects 
requires two probes per object in the collection 
(but only for minimal collections of objects that together contain A's angular range).
% Determining that A is blocked by a collection of overlapping concavities
% requires one probe per concavity in the collection and an additional pair of probes 
% (but only for minimal collections of concavities that together contain A's angular range).
% Steps (7-8) may seem to subsume steps (5-6), but notice that the algorithm stops
% once it is established that A is blocked.
% Therefore, we make the simpler tests (5-6) in anticipation of sometimes avoiding 
% the more expensive tests (7-8).

The complexity of the total algorithm depends on the angular range of A relative
to the angular range of the other objects and concavities.
Obviously, the complexity increases in more complex scenes with more objects and more 
concavities, and when A spans a larger angle with respect to the viewpoint.
The complexity can be analyzed more precisely as follows.
Let $o$ be the number of objects whose angular range contains A,
$c$ the number of concavities whose angular range contains A,
$O$ the number of objects that participate in overlapping collections of minimal size
whose angular ranges combine to contain $A$ (where an object is counted once for each
collection that includes it)
let $C$ be the number of concavities that participate in overlapping collections
of minimal size whose angular ranges combine to contain $A$,
and let $P$ be the cost of a single probe.
Then the complexity of this algorithm is $O((o+2c+2O+C)P)$.
Typically, the angular range of A only overlaps a few objects and concavities
in the scene and a small number of probes are necessary.
ESTABLISH SOME EXPERIMENTAL RESULTS ON NUMBERS OF PROBES.


START AN IMPLEMENTATION:

\ifJournal
Implementation improvement:
lazily compute concavities, angular ranges, etc. on demand: only a few objects are
probed and then need more information computed.
\fi

% ----------------------------

% \begin{enumerate}
% \item Define certain concavities of each object.
%      [the visible ones that lie towards the viewpoint]
%      unnecessary for most concavities: only need for the first blocking object B
% \item Find the angular range of A, the region blocked behind each object,
%       and the region blocked by each visible concavity.
%       unnecessary for most objects, only need for A and the first blocking object B
% \item (Determine whether A is blocked behind an individual object.
%        Unnecessary)
% \item (Determine whether A is blocked by a concavity.  Unnecessary: will
%        test for blockage by the first B as part of the algorithm)
% \item Determine whether A is blocked behind a collection of overlapping objects.
% \end{enumerate}

\section{Computing special tangents}

The tangents of a curve that pass through a point can be found 
using intersection in dual space.
Since a line dualizes to a point, the tangent space of a curve can be dualized to a curve.
In \cite{jj01}, we show how to represent the tangent space of a Bezier curve robustly by
two Bezier curves in two different dual spaces.
Since a point P dualizes to a line dual(P), the tangents of the curve C that pass
through P dualize to the intersections of the curve dual(C) with dual(P).
This is a line-curve intersection, or a simplified version of a curve-curve intersection,
which can be solved using Bezier subdivision \cite{}.
The details are reviewed in \cite{jj02}.

\section{Examples}

START HERE.
REAL LIVE NON-TOY TEST CASES? BUILDING FOOTPRINTS IN A REAL CITY OR REAL BLOCK?
SEE VIENNA DATA

\section{Conclusions}

The analysis of object visibility involves the appropriate choice of probes,
which occur at certain special tangents of the scene.
These include piercing tangents, extreme tangents of an object, and extreme
tangents of a concavity.
Concavities are computed from piercing tangents,
the angular range of an object from extreme tangents of the object,
the region blocked by an object from extreme tangents of the object,
and the region blocked by a concavity from extreme tangents of the concavity. 

\bibliographystyle{latex8}
% \bibliographystyle{plain}
\begin{thebibliography}{99}

\end{thebibliography}

\clearpage

\ifJournal
\section{Robust computation}

In computing the inside of a tangent or the concavity defined by a piercing tangent,
we test where $B(t + \epsilon)$ lies relative to a line through $B(t)$.
This is a potentially fragile computation: we must choose $\epsilon$ small enough
to guarantee that $B[t,t+\epsilon]$ is all on the same side of the line and hasn't
yet crossed the line again, otherwise the test is meaningless.
The size of $\epsilon$ can be determined using arguments developed in the exact
curve work of SoCG 2004.
EXPLORE.

[For robustness, we may want to choose more central probes in general object visibility:
all of the probes are at tangents, but that's inherent to the method isn't it?.]

computation of fundamental operations (tangents through a point; and some trivial
   ones: line/curve intersection, inside/outside of tangent);
robust computation of fundamental operations (optional)
\fi

% ----------------------------

\clearpage

\section{To do, and structure}

For complete paper:
\begin{itemize}
\item
datasets for example curves and viewpoints (tablet),
\item
implementation (tangents through a point, extreme and piercing tangents,
  line/curve intersection, inside/outside of tangent to define the concavity)
\item
implementation with David?
\item
optimally, complicated test example for illustration [journal]
\item
explorable issues: robust versions of line intersection and sideness;
  how far do we want to explore degenerate cases in this paper?
  proofs; [journal]
\end{itemize}

DO I NEED TO IMPLEMENT BEFORE PUBLISHING? (yes, to get figures)

COLLECT FIGURES

\begin{itemize}
\item invisible area behind an object
\item special tangents through the viewpoint that bound this area (extreme)
\item invisible area behind a concavity
\item special tangent through the viewpoint that closes this area (piercing)
\item concavity defined by a piercing tangent
\item special tangents through the viewpoint that bound a concavity (extreme) 
\item angle range of an object and concavity
\item criteria dictating that A is blocked by an object
\item criteria dictating that A is blocked behind a concavity
\item criteria dictating that A is blocked behind a combination of objects
\item subproblem: when does a point lie in a concavity
\item application to computation of a direct visual event
\end{itemize}

% ----------------------------

\section{Future work, and application of this paper: Direct visual event}

A direct visual event for A is defined by a free bitangent between A and B if:
A and B lie on the same side and the point of tangency, and
the point of tangency with B does not see A (elsewhere).

\clearpage

% ----------------------------

\section{Table scraps}

Look in early stages of early papers on object visibility for discussion of 
point visibility.

\subsection{Introductory paragraph 1}

Given a scene of objects, an important question
is to determine when each object is completely invisible.
This establishes when the object can be culled from the graphics pipeline,
and is an important question to answer for general visibility analysis,
such as the computation of visual events (a potential visual event may be culled
if any part of the object of interest is already visible).
This leads to the following related question: given a viewpoint P and a certain
object A from the scene, is A completely invisible from P?
  % this is a static or local version of a more general question that we will ask later:
  % what are *all* of the viewpoints from which A is completely invisible?
This question has received little attention for scenes of smooth objects.
We consider a solution for an arbitrary scene of smooth curves in 2-space.
The only constraint on the curves is that they define closed, disjoint objects.
The curves may also be interpreted as footprints of 2.5-dimensional surfaces
in 3-space.

% Given a viewpoint in a scene of smooth objects in 2-space, what objects
% are visible?

\subsection{Introductory paragraph 2}

Given a viewpoint P, a scene of smooth objects in 2-space, and a distinguished object A
called the \hero, we want to answer two questions about visibility:
(1) what region of 2-space is invisible from P, and 
(2) is A invisible from P?
The first question is a smooth variant of the hard shadow problem,
and could be rephrased in this framework as a computation of the hard shadow
of a smooth scene from a point light source P.
The major new issue is the adaptation of the algorithm to a smooth scene, where
objects are bounded by smooth closed curves.
Another new issue is the reinterpretation of the problem from shadows to visibility,
since visibility is not 360 degrees but is limited by field of view.
The second question could also be interpreted as a shadow problem: if A is the area
light source, is P in the umbra?
However, since we do not need to compute the entire umbra of A to answer this question,
the question is more accurately a localized variant of the umbra problem.
(2) might also be interpreted as follows: does all of A lie in the region computed by (1)?
This reveals the connection between the two problems.

\subsection{Invisible patches in front of an object (invisible zone II)}

This area can also be captured by a cone (although it is a subset of this cone).
The cone is bounded by the free tangent at one boundary and the extreme
intersecting tangent at the other side of the patch.
Call this an invisible patch cone of the object B.

\subsection{Is A visible from P?}

If A lies entirely in an invisible front patch of B, 
then the extreme tangents of A intersect B between P and A as well as
after A (i.e., after the point of tangency with A).

\subsection{Table scraps from 'chain of blocking objects' section}

Suppose that we have begun to build a chain of objects that block $A$,
and we are looking to add the next object in this chain, using a new probe.
Let probe(R) be a function that, given a ray direction R in the angular range of $A$,
returns the object (if any) that is hit by R before $A$
and best extends the angular range of the blocking objects.
(That is, the union of the new object's angular range with the chain's range 
is maximized.)
This probe function will be used to find the next object in the blocking chain.
For example, probe(--) would return -- in EG18.

If we ignore the issue of backtracking and concavities, the basic algorithm
for object visibility probing is as simple as the following.
Let $T_1$ and $T_2$ be the two extreme tangents of $A$.

\begin{verbatim}
B = A
while (probe has not swept past A)
  B = probe(extreme tangent of B)
\end{verbatim}

or, more precisely:

\begin{verbatim}
B = A
probe R = $T_1$
while (probe has not swept past $T_2$)
  B = probe(R)
  R = extreme tangent of B in the sweep direction
\end{verbatim}

The assumption is that any probe that returns A will abort this routine and
return the probe as a witness to visibility.

Adding the issue of backtracking:
\begin{verbatim}
B = A
while (not swept past A)
  returnB = B = probe(positive extreme tangent of oldB = B)
  while (probe hits A before oldB)
    B = probe (negative extreme tangent of B)
  B = returnB
\end{verbatim}

Here, the positive extreme tangent of B is the extreme tangent of B that lies
in the direction of sweep, or alternatively the one that does not lie in the range
that has already been swept over.

This algorithm neglects the following detail: if the probed point of A lies in
a concavity of B, then the extreme tangent of B is replaced by a piercing tangent of B.

-----

FIGURE

A concavity blocks a region from the object's extreme tangent to the piercing tangent
while an object blocks a region behind it between its two extreme tangents.

We want to find a collection of objects (perhaps of size one) that blocks $A$,
if one exists.
First look for the first object in the chain, which blocks one side of $A$.
Choose an extreme tangent of $A$.
Consider the objects $B_i$ whose range contains the angle of this extreme tangent of $A$
(so that it potentially acts as a bookend of the blockers of $A$)
such that the probe at this angle hits $B_i$ before it hits $A$
(so that it lies in front of $A$ and can block),
and the probe point of $A$ (an extreme point) does not lie in a concavity of $B_i$
($A$ lies strictly behind $B_i$, not sort of behind at a concavity where it can 
squirm out).
We start with the object in this collection with the largest range overlap with $A$.
[Witness analysis: if there are no objects whose range contains the angle of the
extreme tangent of $A$, then the probe to the associated extreme point is a witness
to visibility.
If the probe at this angle hits $A$ first, it is a witness.]
Now consider the probe at the angle of the other extreme tangent of $A$.
If this angle lies in the range of $B_i$ and the probe at this angle hits $B_i$ before $A$
then we are done ($B_i$ blocks $A$ alone).
[We don't need to test the concavity issue at the other side, since we already have
a witness to $A$ lying strictly behind $B_i$.]
Otherwise, probe at the extremity of $B_i$ that lies in the range of $A$.
If this probe hits $A$ first, we have a witness to visibility.
If it does not hit $A$ first, it must hit some other objects before $A$.
Restrict to objects such that the probed point of $A$ does not lie in a concavity?
From these objects, choose the object $B_j$ whose range extends the present blocked
range the most.


% Suppose that we have computed the angle ranges blocked by each object $B_i$ in the scene
% and the angle range blocked by each concavity.
% A is invisible from P if the range of A is contained in the union of the ranges of $B_i$
% and the typical ray from P to A (but not a point of a concavity) intersects B before A;
% or if the range of A is contained in the union of the ranges of the concavities
% and the points of tangency of the extreme tangents of A both lie in the associated
% concavity (that contains the range of A).

% NEED MORE THAN ONE PROBE: PROBABLY ONE PER OBJECT.

% \begin{lemma}
% Let $R$ be a viewpoint ray to any point of $A$, say $a$.
% A is blocked by a collection of objects $\{B_i\}$ if 
% $range(A) \subset \cup_i range(B_i)$ and
% $R$ intersects 
% \end{lemma}

\end{document}

% using a southern colloquialism, it's time to piss on the fire and call the dogs
