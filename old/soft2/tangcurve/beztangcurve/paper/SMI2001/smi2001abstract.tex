\documentclass[10pt,twocolumn]{article}
\usepackage{latex8}
\usepackage{times}
%------------------------------------------------------------------------- 
% take the % away on next line to produce the final camera-ready version 
\pagestyle{empty}
%------------------------------------------------------------------------- 
\newif\ifVideo
\Videofalse
\newif\ifTalk
\Talkfalse
\newif\ifJournal
\Journalfalse
\input{header}
\newcommand{\plucker}{Pl\"{u}cker\ }
\newcommand{\tang}{tangential curve\ }
\newcommand{\tangs}{tangential curves\ }
\newcommand{\Tang}{Tangential curve\ }
\newcommand{\atang}{tangential $a$-curve\ }
\newcommand{\btang}{tangential $b$-curve\ }
\newcommand{\ctang}{tangential $c$-curve\ }
\newcommand{\acurve}{$a$-curve\ }
\newcommand{\bcurve}{$b$-curve\ }
\newcommand{\ccurve}{$c$-curve\ }
\newcommand{\atangs}{tangential $a$-curves\ }
\newcommand{\btangs}{tangential $b$-curves\ }
\newcommand{\ctangs}{tangential $c$-curves\ }
% -----------------------------------------------------------------------------
\begin{document}

\title{A Parametric Solution to Common Tangents}

\author{J.K. Johnstone\\
	University of Alabama at Birmingham\\
	Geometric Modeling Lab\\
	Computer and Information Sciences\\
	University Station, Birmingham, AL 35294\\
	jj@cis.uab.edu\\
}

\maketitle

\begin{abstract}
We develop an efficient algorithm for the construction of common tangents
between a set of Bezier curves.
Common tangents are important in visibility, lighting, robot motion,
and convex hulls.
Common tangency is reduced to the intersection of parametric curves 
in a dual space, rather than the traditional intersection of implicit curves.
We show how to represent the tangent space of a plane Bezier curve as a 
plane rational Bezier curve in the dual space,
and compare this representation to the hodograph and the dual Bezier curve.
The detection of common tangents that map to infinity
is resolved by the use of two cooperating curves in dual space,
clipped to avoid redundancy.
We establish the equivalence of our solution in dual space to
a solution in \plucker space, where all the same issues are encountered
in a higher-dimensional context.
\end{abstract}

\end{document}

