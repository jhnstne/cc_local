\documentclass[10pt,twocolumn]{article}
\usepackage{latex8}
\usepackage{times}
%------------------------------------------------------------------------- 
% take the % away on next line to produce the final camera-ready version 
\pagestyle{empty}
%------------------------------------------------------------------------- 
\newif\ifVideo
\Videofalse
\newif\ifTalk
\Talkfalse
\newif\ifJournal
\Journalfalse
\input{header}
\newcommand{\plucker}{Pl\"{u}cker\ }
\newcommand{\tang}{tangential curve\ }
\newcommand{\tangs}{tangential curves\ }
\newcommand{\Tang}{Tangential curve\ }
\newcommand{\atang}{tangential $a$-curve\ }
\newcommand{\btang}{tangential $b$-curve\ }
\newcommand{\ctang}{tangential $c$-curve\ }
\newcommand{\acurve}{$a$-curve\ }
\newcommand{\bcurve}{$b$-curve\ }
\newcommand{\ccurve}{$c$-curve\ }
\newcommand{\atangs}{tangential $a$-curves\ }
\newcommand{\btangs}{tangential $b$-curves\ }
\newcommand{\ctangs}{tangential $c$-curves\ }
% -----------------------------------------------------------------------------
\begin{document}

\title{A Parametric Solution to Common Tangents}

\author{J.K. Johnstone\\
	University of Alabama at Birmingham\\
	Geometric Modeling Lab\\
	Computer and Information Sciences\\
	University Station, Birmingham, AL 35294\\
	jj@cis.uab.edu\\
}

\maketitle
\thispagestyle{empty}

\begin{abstract}
We develop an efficient algorithm for the construction of common tangents
between a set of Bezier curves.
Common tangents are important in visibility, lighting, robot motion,
and convex hulls.
Common tangency is reduced to the intersection of parametric curves 
in a dual space, rather than the traditional intersection of implicit curves.
We show how to represent the tangent space of a plane Bezier curve as a 
plane rational Bezier curve in the dual space,
and compare this representation to the hodograph and the dual Bezier curve.
The detection of common tangents that map to infinity
is resolved by the use of two cooperating curves in dual space,
clipped to avoid redundancy.
We establish the equivalence of our solution in dual space to
a solution in \plucker space, where all the same issues are encountered
in a higher-dimensional context.
\end{abstract}

\ifTalk
Where do you throw the ball so that it can be caught by the other curve?
Point/curve: Where do you shoot to hit the target (as you run around a track)?
Where does the spotlight illuminate a given spot?
A common tangent tells where to throw the ball from one curve so that
it can be caught by a second curve.
\fi

% -----------------------------------------------------------------------------
\section{Introduction}

% A common tangent is a tangent of two curves, thus it is natural
% to identify a common tangent with an intersection in some dual space.

The common tangents of two curves are at the heart of many problems
involving visibility.
% are fundamental in many geometric applications, particularly those that involve visibility.
Consider a collection of closed curves in the plane, interpreted as 
obstacles for a robot.
The common tangents of these curves define the limits of visibility,
and consequently the shortest paths that can be travelled by the robot
(Figure~\ref{fig:applications}).
If we instead interpret the collection of closed curves as 
an area light source and occluders,
the common tangents define the boundaries of the shadows (umbra and 
penumbra) that are cast by the light (Figure~\ref{fig:appl2}).
Common tangents are also needed for basic geometric computations with curves,
such as the convex hull.
\cite{parida95} mentions many more applications, including font design.
% bin-packing, and compaction.

\newbox\jjdushbox
\newdimen\jjdushwd
\font\jjdusha=jjdusha at 72.27truept
\setbox\jjdushbox=\vbox{\hbox{%
\jjdusha\char0\char1}}
\jjdushwd=\wd\jjdushbox
\setbox\jjdushbox=\hbox{\vbox{\hsize=\jjdushwd
\parskip=0pt\offinterlineskip\parindent0pt
\hbox{\jjdusha\char0\char1}
\hbox{\jjdusha\char2\char3}}}
\ifx\parbox\undefined
    \def\setjjdush{\box\jjdushbox}
\else
    \def\setjjdush{\parbox{\wd\jjdushbox}{\box\jjdushbox}}
\fi


\begin{figure}
% \vspace{2in}
\hspace{1in} \setjjdush
\caption{Common tangents for shortest paths}
\label{fig:applications}
% svgraph vg4b.rawctr, path only
% jjdush.gif
\end{figure}

\newbox\jjduligbox
\newdimen\jjduligwd
\font\jjduliga=jjduliga at 72.27truept
\setbox\jjduligbox=\vbox{\hbox{%
\jjduliga\char0\char1}}
\jjduligwd=\wd\jjduligbox
\setbox\jjduligbox=\hbox{\vbox{\hsize=\jjduligwd
\parskip=0pt\offinterlineskip\parindent0pt
\hbox{\jjduliga\char0\char1}
\hbox{\jjduliga\char2\char3}}}
\ifx\parbox\undefined
    \def\setjjdulig{\box\jjduligbox}
\else
    \def\setjjdulig{\parbox{\wd\jjduligbox}{\box\jjduligbox}}
\fi


\begin{figure}
\hspace{1in} \setjjdulig
\caption{Common tangents for lighting from an area light source}
\label{fig:appl2}
% svgraph vg4b.rawctr, visible common tangents on, source/destination off
% jjdulight.gif
\end{figure}

The common tangency problem is truly an intersection problem,
% The common tangency problem can be naturally posed as an intersection problem.
% The problem of common tangency is naturally posed as a problem of intersection.
since a common tangent is simultaneously a tangent of two curves.
% After all, a common tangent is a line that is simultaneously a tangent of two curves.
The standard solution reduces common tangency to the intersection of implicit curves.
We explore an alternate reduction, 
to the intersection of parametric Bezier curves.
% In this paper, we reduce the problem instead to the intersection
% of two parametric Bezier curves.
The robustness, efficiency, and simplicity of Bezier intersection
make this an attractive solution.

Let $C(s) = (c_x(s),c_y(s))$ and $D(t) = (d_x(t),d_y(t))$ be 
plane curves of degree $n$.
The standard solution \cite{bajajkim87,parida95} argues as follows.
% reduces to the intersection of two implicit curves.\footnote{An implicit 
%	curve in the plane is a curve represented as the zero set
%	of a bivariate polynomial, $f(x,y)=0$.}
Since a line between $C(s)$ and $D(t)$ is a common tangent if and only if 
it is orthogonal to the normals at both $C(s)$ and $D(t)$,
a common tangent is a parameter pair $(s,t)$ satisfying the following
system of equations:
%
\[
\begin{tabular}{c}
$(c_x(s) - d_x(t), c_y(s) - d_y(t)) \cdot (-c_y'(s), c_x'(s)) = 0$\\
$(c_x(s) - d_x(t), c_y(s) - d_y(t)) \cdot (-d_y'(t), d_x'(t)) = 0$
\end{tabular}
\]
%
This is the intersection of two implicit curves of degree $2n-1$
in parameter space:
\[
\begin{tabular}{c}
$f(s,t)=0$\\
$g(s,t)=0$
\end{tabular}
\]

\Comment{
A similar solution works for two
implicit curves, generating a more expensive system of 4 equations,
or for an implicit curve and a parametric curve, generating 3 equations
(see \cite{bajajkim87}).
}

% In this paper, we reduce the problem instead to the simpler intersection
% of two parametric Bezier curves.
In our solution, we interpret the tangent space of a curve as a family of lines
(Figure~\ref{fig:linefamily}),
and a common tangent as an intersection of the line families of two curves.
\Comment{
	This interpretation of common tangency as intersection
	will be a central theme of the paper.
}
To simplify the intersection of line families,
we map to a dual space. % where intersection is more natural.
By encoding a tangent line as a point in a dual space,
a family of tangent lines becomes a parametric curve in the dual space
(Figure~\ref{fig:1dual}).
The computation of common tangents then reduces to the intersection of 
curves in dual space (Figure~\ref{fig:commonTang}).
The representation of tangent space as a rational Bezier curve in dual space
(Section~\ref{sec:dual}) is a development of independent interest.
It can be a useful tool for the analysis of tangent spaces.

\newbox\jjdufambox
\newdimen\jjdufamwd
\font\jjdufama=jjdufama at 72.27truept
\setbox\jjdufambox=\vbox{\hbox{%
\jjdufama\char0\char1\char2}}
\jjdufamwd=\wd\jjdufambox
\setbox\jjdufambox=\hbox{\vbox{\hsize=\jjdufamwd
\parskip=0pt\offinterlineskip\parindent0pt
\hbox{\jjdufama\char0\char1\char2}
\hbox{\jjdufama\char3\char4\char5}}}
\ifx\parbox\undefined
    \def\setjjdufam{\box\jjdufambox}
\else
    \def\setjjdufam{\parbox{\wd\jjdufambox}{\box\jjdufambox}}
\fi


\begin{figure}
\hspace{.5in} \setjjdufam
\caption{A tangent space is a family of lines}
\label{fig:linefamily}
% dual ob2.rawctr, line field/one-obstacle only
% jjdufamily.gif
\end{figure}

Since our solution involves tangent spaces, dual spaces, and point 
representations of a line, there is much related work to explore.
We choose to present the new method in its entirety first, and make comparisons
afterwards, so that our method is entirely clear and an 
informed comparison can be made.

The paper is organized as follows.
Section~\ref{sec:dual} introduces the dualization of a line and 
the representation of a plane curve's tangent space 
by a plane curve in the dual space, the tangential curve.
This section also shows how to express the \tang as a rational Bezier curve.
Section~\ref{sec:infinity} deals with the issue of points at infinity:
it is inevitable that some tangents map to infinity in the dual space,
and we ensure that we find all common tangents by replacing a single
\tang by two cooperating, clipped tangential curves.
This section also includes the final version of our algorithm.
Experimental results are given in Section~\ref{sec:results}.
In the remaining sections, we relate our work to classical work on tangent
spaces, common tangents, dual spaces, and point representations of a line.
Section~\ref{sec:prevcommon} analyzes the other solutions to common tangents
and the advantages of our parametric solution in more detail.
\Comment{
	Sections~\ref{sec:hodo} and \ref{sec:plucker} explain why we don't
	use the obvious solutions of the hodograph and the \plucker representation
	of a line.
}
Section~\ref{sec:hodo} explains the inadequacy of the hodograph
for common tangents.
Section~\ref{sec:duality} discusses other work on duality, 
including the dual Bezier curve.
Finally, Section~\ref{sec:plucker} establishes the basic equivalence
of our solution with a \plucker coordinate solution.
Section~\ref{sec:conclude} wraps up with ideas for future work.
An appendix fills in some necessary technical detail on tangential curves.

\newbox\jjdudefbox
\newdimen\jjdudefwd
\font\jjdudefa=jjdudefa at 72.27truept
\setbox\jjdudefbox=\vbox{\hbox{%
\jjdudefa\char0\char1\char2}}
\jjdudefwd=\wd\jjdudefbox
\setbox\jjdudefbox=\hbox{\vbox{\hsize=\jjdudefwd
\parskip=0pt\offinterlineskip\parindent0pt
\hbox{\jjdudefa\char0\char1\char2}
\hbox{\jjdudefa\char3\char4\char5}}}
\ifx\parbox\undefined
    \def\setjjdudef{\box\jjdudefbox}
\else
    \def\setjjdudef{\parbox{\wd\jjdudefbox}{\box\jjdudefbox}}
\fi


\begin{figure}
\hspace{.2in} \setjjdudef
\caption{A curve and its tangential curve}
\label{fig:1dual}
% jjdudef.gif, based on 1duaL.gif
% note that this is an old figure that we cannot presently repeat,
% 		using tangential c-curves: treat like gold
\end{figure}

\newbox\jjduintbox
\newdimen\jjduintwd
\font\jjduinta=jjduinta at 72.27truept
\setbox\jjduintbox=\vbox{\hbox{%
\jjduinta\char0\char1\char2}}
\jjduintwd=\wd\jjduintbox
\setbox\jjduintbox=\hbox{\vbox{\hsize=\jjduintwd
\parskip=0pt\offinterlineskip\parindent0pt
\hbox{\jjduinta\char0\char1\char2}
\hbox{\jjduinta\char3\char4\char5}}}
\ifx\parbox\undefined
    \def\setjjduint{\box\jjduintbox}
\else
    \def\setjjduint{\parbox{\wd\jjduintbox}{\box\jjduintbox}}
\fi


\begin{figure}
\hspace{.2in} \setjjduint
\caption{Common tangency is intersection in dual space}
\label{fig:commonTang}
% jjduint.gif, based on commontang.rgb
% treat like gold like previous figure
\end{figure}

\ifVideo
  video: single curve and its dual (ob2.rawctr)
       tangent moving on curve with associated point moving on dual
\fi
\ifTalk
This is analogous to the appeal
of using the Fourier transform to shift to the frequency domain,
where the difficult operation of convolution 
becomes the simple operation of multiplication.
\fi
\ifVideo
  video: two curves in left window, their duals in right window:
       point moving on one of the dual curves with associated tangent
       moving on original curve, illustrating that as point nears intersection
       the tangent nears common tangency
\fi



\section{A representation of tangent space}
\label{sec:dual}

To map a tangent space to a curve, we must first be able to map a line to a point.
% \footnote{Section~\ref{sec:plucker} explains why we do not use the classical \plucker coordinates of a line.}

\begin{defn2}
\label{defn:dual}
The line $ax+by+c=0$ in 2-space is {\bf dual} to the point $(a,b,c) \in P^2$
in projective 2-space.
\end{defn2}
%
The dual point must live in projective space,
since the dual $(a,b,c)$ must be equivalent to the dual $(ka,kb,kc)$,
as the line $ax + by + c = 0$ is equivalent to the line
$kax + kby + kc = 0$, $k \neq 0$.

Since the tangent space of a plane curve is a continuous one-dimensional family 
of lines, a tangent space can now be represented by a curve in dual space.

\begin{defn2}
\label{defn:tangentialcurve}
Let $C(t)$ be a plane curve.
The {\bf tangential curve} of $C(t)$ is the curve $C^*(t) \subset P^2$,
where $C^*(t)$ is the dual of the tangent at $C(t)$.
\end{defn2}
%
Figure~\ref{fig:1dual} gives a simple example of a tangential curve, and
Figure~\ref{fig:commonTang} illustrates how the computation of 
common tangents in primal space becomes curve intersection in dual space.

We now express the tangential curve as a Bezier curve, 
first as a Bezier curve in projective space, and then as a rational Bezier curve.
We begin by developing the parametric form of the tangential curve for an arbitrary
parametric input curve, and then the Bezier form for a Bezier input curve.
This result also applies to B-spline curves and polynomial parametric curves,
which can be easily expressed as Bezier curves.
The Bezier form of the tangential curve for a rational Bezier input curve 
is a more elaborate application of the same methods.

\begin{theorem}
\label{thm:Bezier}
Let $C(t)$ be a plane Bezier curve of degree $n$
with control points $\{ b_i \}_{i=0}^n$ 
over the parameter interval $[t_1,t_2]$.
The \tang $C^*(t)$ is a Bezier curve
of degree $2n-1$ in projective 2-space with control points 
\begin{equation}
\label{eq:dualcoord}
\tiny{
b^*_k = \alpha
\left(
\begin{array}{l}
	\displaystyle{\sum_{j=\mbox{\tiny{max}}(0,k-n)}^{\mbox{\tiny{min}}(n-1,k)}}
	\tinychoice{n-1}{j} \tinychoice{n}{k-j} (b_{j,2} - b_{j+1,2})\\
	\displaystyle{\sum_{j=\mbox{\tiny{max}}(0,k-n)}^{\mbox{\tiny{min}}(n-1,k)}}
	\tinychoice{n-1}{j} \tinychoice{n}{k-j} (b_{j+1,1} - b_{j,1})\\
\displaystyle{\sum_{\begin{array}{c} \mbox{\tiny{$0 \leq i \leq n-1$}} \\ 
		       \mbox{\tiny{$0 \leq j \leq n$}} \\ 
		       \mbox{\tiny{$i+j=k$}}
      \end{array}}}
\tinychoice{n-1}{i} \tinychoice{n}{j} \beta_{i,j}
\end{array}
\right)}
\end{equation}
$k=0,\ldots,2n-1$, $b^*_k \in P^2$, 
over the parameter interval $[t_1,t_2]$, where
$\alpha = \tiny{\frac{n}{\Delta \tinychoice{2n-1}{k}}}$,
$\Delta = t_2 - t_1$, and
$\beta_{i,j} = (b_{i+1,2} - b_{i,2}) b_{j,1} - (b_{i+1,1} - b_{i,1}) b_{j,2}$.
\end{theorem}
\prf
Since the proof is rather technical, the majority of it is postponed to the appendix.
We only develop the basic parametric representation here.
Let $C(t) = (c_1(t), c_2(t))$.
The tangent and normal vectors of $C(t)$ are $(c'_1(t),c'_2(t))$
and $(-c'_2(t), c'_1(t))$, respectively.
	% since dot product encodes orthogonality.
Since the implicit equation of the line through $P$ with normal $N$ is 
$(X-P) \cdot N = 0$,
the implicit equation of the tangent of $C(t)$ is:
\[	
	(x-c_1(t), y-c_2(t))\ \  \cdot \ \ (-c'_2(t), c'_1(t)) = 0
\]
or
\[
	-c_2'(t) x + c_1'(t) y + [c_2'(t) c_1(t) - c_1'(t)c_2(t)] = 0 
\]
and the \tang of $C(t)$ is
\begin{equation}
\label{eq:dual1}
  (-c_2'(t),\ \  c_1'(t),\ \ c_2'(t) c_1(t) - c_1'(t)c_2(t))
\end{equation}
%
This is a polynomial curve in projective 2-space.
% CORRECT SO FAR, HAVING TESTED IN drawTang of BezierCurve.c++
% using drawImplicitLine
\QED

A Bezier curve in projective 2-space is equivalent to a 
rational Bezier curve in 2-space.
Weights of the rational Bezier curve correspond to the
projective coordinate of the control points of the polynomial Bezier curve.
In our notation, the point $(x_1,x_2,x_3),\ x_3 \neq 0$, of $P^2$ is equivalent 
to the point $(\frac{x_1}{x_3}, \frac{x_2}{x_3})$ of $\Re^2$.
We shall call $x_3$ the {\bf projective coordinate}.

We rephrase Theorem~\ref{thm:Bezier} accordingly.
% In translating to a rational curve, the last projective coordinate of 
% $C^*_p(t)$ is associated with the weight of the rational Bezier curve.  
% There are many ways to do this, 

\begin{corollary}
\label{thm:rationaldual}
Let $C(t)$ be a plane Bezier curve of degree $n$ 
with control points $\{ b_i \}_{i=0}^n$ over $[t_1,t_2]$.
The \tang $C^*(t)$ is a rational Bezier curve of degree $2n-1$ 
over $[t_1,t_2]$ with weights $\{w_k\}_{k=0}^{2n-1}$ where: 
\begin{equation}
\label{eq:wt}
\scriptsize{
w_k = \alpha
\sum_{\begin{array}{c}       0 \leq i \leq n-1 \\ 
			     0 \leq j \leq n \\ 
			     i+j=k
			     \end{array}}
\scriptchoice{n-1}{i} \scriptchoice{n}{j} \beta_{i,j}
}
\end{equation}
%
and control points $\{c_k\}_{k=0}^{2n-1}$ where:
\[
\scriptstyle{
	c_k = \frac{\alpha}{w_k} 
	\displaystyle{\sum_{j=\mbox{\tiny{max}}(0,k-n)}^{\mbox{\tiny{min}}(n-1,k)}}
	\tinychoice{n-1}{j} \tinychoice{n}{k-j} 
	\left( \begin{array}{c} 
	b_{j,2}   - b_{j+1,2} \\
	b_{j+1,1} - b_{j,1}
	\end{array} \right) 
}
\]
where $\alpha$ and $\beta_{i,j}$ are as in Theorem~\ref{thm:Bezier}.
\end{corollary}

\Comment{
% EVERYBODY KNOWS ABOUT INTERSECTION: AN INSULT (AND A STUPIDITY) TO INCLUDE IT 
\section{Curve intersection}
\label{sec:intersect}

Once our problem has been translated to dual space by computing tangential curves,
it reduces to curve intersection, 
a problem that has received much attention and is well understood \cite{sederberg86}.
	% This is a major advantage of the tangential curve solution to common tangency.
The standard solution to intersection is a divide-and-conquer approach using
curve subdivision and the convex hull property of Bezier curves:
if the bounding boxes of the two curves intersect, subdivide each curve
and recursively intersect the four subsegments,
halting when the bounding box is sufficiently small 
(the curve is sufficiently straight) 
to compute the intersection point directly.
Sederberg reports that this is the most stable of the most popular
intersection algorithms.
\Comment{
% the less details about intersection, the better: well understood
A bounding box can be a convex hull or, for efficiency, 
an axis-aligned box defined by the extremal coordinates of the control points.
It turns out that the axis-aligned box, although larger, yields a more
efficient intersection algorithm,
since the convex hull is more difficult to compute.
}

Self-intersections of a Bezier curve (for the common tangents of a single curve)
can be computed using the method of Lasser \cite{lasser89},
which also uses subdivision, with some added subtlety
(for example, two neighbouring
segments of a curve will necessarily have an intersection at their
common boundary that must be ignored).

\Comment{
Elaboration on self-intersections: cannot have self-intersection
within a Bezier curve segment unless its control polygon has a self-intersection
(explain why using variation-diminishing property?)
so self-intersections are computed entirely the same as intersection
of two curves, where both curves are the same.
For example, the self-intersections of the Bezier spline with segments
$S_0,\ldots,S_k$ are the union of the intersections of $S_i$ and $S_j$ for $i \neq j$,
just as the intersections of the Bezier spline $C$ with segments
$C_0,\ldots,C_m$ with the Bezier spline $D$ with segments 
$D_0,\ldots,D_n$ are the union of the intersections of $C_i$ and $D_j$ for $i \neq j$.
}

\Comment{
We shall compute intersections of dual curves in 2-space, 
rather than projective 2-space.
Fortunately, the intersections in 2-space are equivalent to the
intersections in projective 2-space.
(This may be a well-known fact of projective space.)
Consider an intersection in 2-space: $(\frac{a_1}{c_1}, \frac{b_1}{c_1})
= (\frac{a_2}{c_2}, \frac{b_2}{c_2})$.
Then, in projective space, $(a_1,b_1,c_1) = \frac{c_1}{c_2} (a_2,b_2,c_2)$,
so the associated lines in projective space are also equivalent and 
form an intersection.
Conversely, if two 'points' (lines) in projective space are equivalent
(an intersection) then the associated points in projective space
are also equivalent: if $(a_1,b_1,c_1) = k(a_2,b_2,c_2)$ ($k \neq 0$),
then $(\frac{a_1}{c_1}, \frac{b_1}{c_1}) = (\frac{ka_2}{kc_2}, \frac{kb_2}{kc_2})
 = (\frac{a_2}{c_2}, \frac{b_2}{c_2})$.
}
}

\section{Recovering intersections at infinity}
% \section{Points at infinity on the \tang}
\label{sec:infinity}

Lines $ax+by=0$ through the origin map to points at infinity
$(a,b,0)$ in dual space.\footnote{The point $(x_1,x_2,0)$ 
	is the point at infinity in the direction $(x_1,x_2)$.}
As a result, if two curves have a common tangent through the origin, 
the associated intersection point in dual space will be at infinity, 
and effectively will not be found.
In practice, all common tangents 'nearly' through the origin will also be lost.
How can we correct this?

Our solution is to replace a tangential curve by two cooperating,
clipped tangential curves, as follows.
In mapping the line $ax+by+c=0$ to the point $(a,b,c)$ in projective 2-space, 
any of the coordinates of $(a,b,c)$ may be considered the projective coordinate.
We have chosen $c$ as the projective coordinate, which results in mapping lines 
$ax+by=0$ to infinity.
If we instead choose $a$ as the projective coordinate, 
horizontal lines $by+c=0$ are mapped to infinity,
or if we choose $b$, vertical lines $ax+c=0$ are mapped to infinity.
%
\begin{defn2}
Let $\alpha$ be a plane curve.
The {\bf (tangential) \acurve} (resp., {\bf \bcurve} or {\bf $c$-curve}) 
of $\alpha$ is the image of 
the tangent space of $\alpha$ in dual space when $a$ (resp., $b$ or $c$) is chosen as the
projective coordinate in dual space.
\end{defn2}
%
Thus, the tangential curve of Corollary~\ref{thm:rationaldual}
is a $c$-curve, and horizontal tangents are mapped to infinity on the $a$-curve,
while vertical tangents are mapped to infinity on the $b$-curve.

Since a line cannot be simultaneously horizontal and vertical, 
any tangent will be a finite point either on the $a$-curve or the $b$-curve.
Thus, no common tangents associated with intersections at 
infinity will be missed if we compute both the intersections of the \atangs and
the intersections of the tangential $b$-curves.\footnote{Since a line can simultaneously be horizontal 
	and pass through the origin, or vertical and pass through the origin,
	combining the intersections of tangential $a$-curves and $c$-curves,
	or $b$-curves and $c$-curves, is not useful.
	We would still miss horizontal or vertical common tangents through the origin.}
	%	$a$-curve intersections and $b$-curve intersections
	%	are the only combination that complements one another.
This replaces the earlier intersection of $c$-curves only.
(See Figures~\ref{fig:eg1}-\ref{fig:eg4}.)
\ifTalk
The $a$-curves and $b$-curves cooperate to find all of the common tangents.
each individually has a blind spot, but together they see everything.
much like the three Fates that share an eyeball.
WANT NO MUTUALLY BLIND PARAMETER INTERVALS
We shall clip out a segment of a tangential curve about its point at infinity.
We must take care that the parameter interval of a clipped segment 
of one curve (when the curve is blind to common tangents) does not
overlap a parameter interval of a clipped segment of the other curve,
which would create a mutually blind interval where a common tangent
could be missed.
This is simply a matter of guaranteeing that the parameter intervals
of the clipped segments (blind regions) on both curves are kept disjoint.
\fi

The $a$-curves and $b$-curves still duplicate effort.
If a common tangent is found as an intersection of the $a$-curves,
we do not want it also to be found as an intersection of the $b$-curves,
for the sake of efficiency.
The solution is to make the \atang represent more vertical tangents
($ax+by+c=0$ with $|a| \geq |b|$)
while the \btang represents more horizontal tangents
($ax+by+c=0$ with $|a| \leq |b|$).
A common tangent that is more vertical will be found as an
intersection of the $a$-curves, while a common tangent that is more
horizontal will be found as an intersection of the $b$-curves,
making for a perfect division of duty.
We find the diagonal (45- and 135-degree) tangents,
which separate the more-horizontal from the more-vertical tangents,
and then clip the curves at these tangents.
Notice that this clipping is equivalent to a clipping of segments
about the points at infinity.
% For example, horizontal tangents are 
% associated with points at infinity on the $a$-curve).

\subsection{The final algorithm}

The use of 2 curves and clipping leads to the following algorithm
for computing the common tangents of $C$ and $D$, 
as illustrated by Figure~\ref{fig:eg1}:
\begin{description}
\item[(1)]	Compute the \atangs $C_a^*$ and $D_a^*$ 
		and the \btangs $C_b^*$ and $D_b^*$.
\item[(2)]  	Clip at diagonal tangents, as follows.
		Subdivide $C_a^*$ and $C_b^*$ (resp., $D_a^*$ and $D_b^*$)
		at the parameter values of the
		diagonal tangents of $C$ (resp., $D$),
		and classify the resulting segments
		as more-vertical or more-horizontal,
		by testing an interior tangent. % the midpoint in our implementation
		Clip out the more-horizontal segments from $C_a^*$ and $D_a^*$
		and the more-vertical segments from $C_b^*$ and $D_b^*$.
\item[(3)] 	Intersect the clipped $C_a^*$ and $D_a^*$ in $a$-space
		(the dual space of the $a$-curves),
		and the clipped $C_b^*$ and $D_b^*$ in $b$-space.
\item[(4)]	Map both sets of intersections in dual space to 
		common tangents in primal space.
\end{description}

The common tangents of a single curve can also be computed using this algorithm
(Figure~\ref{fig:eg4}).
The only difference is that $C$ and $D$ are the same curve,
and the intersection of step (3) is self-intersection.

In many applications (e.g., motion planning and lighting),
visible common tangents are the only common tangents of interest,
and an extra filtering step should be added to the algorithm (Figure~\ref{fig:appl2}).
A common tangent is visible if its two points of tangency
are visible (i.e., the line segment connecting them
has no interior intersections with any obstacles).
Since a common tangent is by definition close to the 
obstacle as it approaches it, and the intersection algorithm is approximate,
care must be taken in evaluating visibility.
% since it is easy to mistakenly find an apparent 
% intersection of the common tangent with an obstacle near the point of tangency,

The computation in each step of the algorithm is efficient.
Consider each step:
%
\begin{description}
\item[(1)] The appendix gives formulae for computing \atangs and $b$-curves.
\item[(2)] The diagonal tangents of the curve $\alpha$ are the 
	   tangents $ax+by+c=0$ with $|a|=|b|$.
	The parameter values of these diagonal tangents
	can be found by intersecting the tangent hodograph % $(\alpha_1'(t),\alpha_2'(t))$
	of $\alpha$ with the diagonal lines $x=\pm y$.
\item[(3)] This is traditional Bezier curve intersection \cite{sederberg86}.
	For the common tangents of a single curve, it is Bezier self-intersection
	\cite{lasser89}.
\item[(4)] Rather than using duality, the following solution is preferable
	to map the intersections back to common tangents. 
	We record each intersection as a parameter pair $(s,t)$,
	the parameter values of the intersection point with respect to each curve.
	The two endpoints $C(s)$ and $D(t)$ of the common tangent are then immediately 
	available.
	These endpoints are necessary for testing if a common tangent is visible.
\end{description}

The clipping by diagonal tangents may also be interpreted
as a clipping of the tangential curves by the lines $x = \pm 1$
(see Figures~\ref{fig:eg2}-\ref{fig:eg3}).
In particular, the diagonal lines in primal space map to the vertical lines
$x = \pm 1$ in both dual $a$-space and dual $b$-space:
$ax \pm ay + c=0$ maps to $(\frac{\pm a}{a},\frac{c}{a}) = (\pm 1,y),\ y \in \Re$.
Segments inside $x \in [-1,1]$ should be kept.
% since the segments at infinity are more horizontal in $a$-space (resp., more vertical in $b$-space).

\ifJournal
THIS IS THE FORMULA FOR C-CURVES (NOT A- or B-CURVES).
Note that if one wants to use duality
to map points of dual space to lines in primal space,
care must be taken to work in projective space.
If the intersection point is $C^*(t) = (c_1,c_2)$ and the weight function 
of the rational Bezier curve $C^*(t)$ at
this parameter value is $c_3$, the coordinates of the intersection
in projective dual space are $(c_3 c_1, c_3 c_2, c_3)$
and the associated line in primal space is $c_3 c_1 x + c_3 c_2 y + c_3 = 0$.
% Since this line is not enough for the filtering step (4)
% (its endpoints on the curves are also needed)
% and it is expensive to compute the endpoints of the tangent 
% from the implicit equation of the line, we use the parameter pair representation instead.
\fi

\subsection{The inevitability of intersections at infinity}

% Can we avoid points at infinity?
The development of this section may prompt the reader to ask if it is
possible to reduce common tangency to the intersection of plane curves
{\em without} introducing the problem of certain lines mapping to infinity.
The answer is no.
The mapping of some lines to infinity is a necessary consequence of 
stuffing the 3 dimensions of the family of lines $ax+by+c=0$
into the 2 dimensions of a projective 2-space.
	% Notice that normalizing to $x+ay+b=0$ removes horizontal lines.
We must stuff the 3 dimensions of tangent space into a 2-dimensional
dual space, since common tangency will not reduce to intersection if we represent
the tangent space as a space curve in 3 dimensions
(space curves do not intersect in general).
% we do not want to represent the tangent space as a space curve in 3 dimensions.
Moreover, this 2-dimensional dual space is inherently projective,
since $ax+by+c=0$ is equivalent to $kax+kby+kc=0$.
Thus, one of the three coordinates of a tangent line must be considered the 
projective coordinate, and
when this coordinate goes to 0, we have a point at infinity.
For example, we will later develop a \plucker version of the 
tangential curve that is also 3-dimensional
and has the same problems with points at infinity (Section~\ref{sec:plucker}).

\Comment{
The only alternative to using projective 2-space (which allows
several points represent the same line) is to normalize each line beforehand
(to unit length) and then use affine 2-space.
(That is, the {\bf unit} lines in a plane are a 2-dimensional space.)
However, this normalization is too expensive and awkward.
}

\Comment{
The solution is to clip a \tang about its points at infinity
(where intersections are not found robustly anyway),
removing some segments from each curve.
This clipping must be done with care:
	% The desire for a complete solution to the common tangents wars with the desires for an efficient solution.
the more we clip a tangential curve, the faster the intersection goes,
but the more likely it is that we will miss an intersection.
%
\Comment{
For completeness of solution, we must guarantee that every tangent in the 
tangent space maps to a non-clipped segment of either the \atang or the \btang,
which discourages clipping.
For efficiency, we prefer to minimize the regions of overlap 
of the two tangential curves (i.e., the tangents that map to non-clipped segments
on both tangential curves), which favours clipping.
After all, the shorter a curve is, the faster is intersection with that curve.
}
%
The optimal solution, leading to the most efficient complete intersection,
is for there to be no overlap between the two tangential curves associated
with a plane curve: every tangent in the tangent space should map to a clipped segment
on the $a$-curve and a non-clipped segment on the $b$-curve, or vice versa.

We define this clipping formally, as follows.
Let the $a$-curve be active (not clipped) on $t \in I_a$
and the $b$-curve be active on $t \in I_b$,
and let the curve in primal space be defined on $t \in I$.
Cooperation means that we want $I_a \cup I_b = I$
(all common tangents are found);
the avoidance of duplication means that we want $I_a \cap I_b = \emptyset$
(no common tangent is found twice).
That is, we want a disjoint cover of the interval $I$ by the 
intervals of the $a$-curve and $b$-curve.
}

% ____________________________________________________________________________

\Comment{

ERRONEOUS SECOND VERSION (FIRST VERSION WAS JUST USING SMALL EPSILON AND HOPING)
To guarantee the success of this cooperation,
we need to formalize and quantify the relationship between $a$-tangential
and $b$-tangential curves in the following theorem.

\begin{theorem}
\label{thm:cooperate}
WHOOPS!  WRONG: C/B, NOT C/A.
Let $B_k(0)$ be the square $\{ (x,y) : |x| \leq k, |y| \leq k\}$.\footnote{This
	is the $k$-ball about the origin in the $L_{\infty}$ metric.}
\begin{enumerate}
\item
If the point   $(\frac{b}{a}, \frac{c}{a})$ on the $a$-curve lies outside $B_k(0)$,
then the point $(\frac{a}{b}, \frac{c}{b})$ on the $b$-curve lies inside  $B_{\frac{1}{k}}(0)$.
\item
If the point   $(\frac{a}{b}, \frac{c}{b})$ on the $a$-curve lies outside $B_k(0)$,
then the point $(\frac{b}{a}, \frac{c}{a})$ on the $b$-curve lies inside  $B_{\frac{1}{k}}(0)$.
\end{enumerate}
\end{theorem}
\prf
Suppose $(\frac{b}{a}, \frac{c}{a})$ lies outside $B_k(0)$.
$\frac{|b|}{|a|} > k$ implies $\frac{|a|}{|b|} < \frac{1}{k}$,
and $\frac{|c|}{|a|} > k$ implies $\frac{|a|}{|c|} < \frac{1}{k}$,
so $(\frac{a}{b}, \frac{c}{b})$ lies inside $B_{\frac{1}{k}}(0)$.
Similarly for 2.
\QED

Theorem~\ref{thm:cooperate} says that as a point goes to infinity on the $a$-curve 
(where it may be clipped), the associated point on the $b$-curve cooperates 
by staying far away from infinity (where it will not be clipped).
This is a very important result, since it guarantees the safety of the
clipping of Section~\ref{sec:negative} that becomes necessary because of 
negative weights, as shown in the following corollary.

\begin{corollary}
If no clipped segments $C(t-\epsilon,t+\epsilon)$
of the $a$-curves and $b$-curves lie inside the square 
$B_1(0) = \{(x,y) : |x| \leq 1, |y| \leq 1\}$,
no common tangents will be lost.
\end{corollary}
\prf
If an intersection of the $a$-curves is missed 
(i.e., lies on a clipped segment of an $a$-curve outside $B_1(0)$), 
the associated intersection of the $b$-curves will not be missed 
(i.e., it lies inside $B_1(0)$ and not on a clipped segment of the $b$-curves).
Thus, all common tangents will be found, either as an intersection
of the unclipped portions of the $a$-curves or an intersection of the 
unclipped portions of the $b$-curves.
\QED

It is simple to guarantee that no clipped segments lie inside $B_1(0)$.
Consider the clipping of a segment $C(t_0-\epsilon,t_0+\epsilon)$, where $C(t_0)$
is a point at infinity on an $a$-curve ($w(t_0)=0$).
If $C(t_0-\epsilon,t_0+\epsilon)$ intersects $B_1(0)$ (which is highly unlikely
for the small $\epsilon$ that we use),
we will continually reduce $\epsilon$ until it does not.

For efficiency, we can go even further, as in the following corollary.
%
\begin{corollary}
\label{cor:insidesquare}
Only intersections inside the unit square $B_1(0)$ need to be found
in dual space.
\end{corollary}
\prf
By Theorem~\ref{thm:cooperate}, all intersections of the $a$-curves
outside $B_1(0)$ are associated with common tangents that will generate
intersections of the $b$-curves inside $B_1(0)$.
\QED

Corollary~\ref{cor:insidesquare} is a considerable time-saving device.

END OF ERRONEOUS 2ND VERSION.
}

\Comment{
3RD VERSION
This is simply a matter of guaranteeing that the parameter intervals
of the clipped segments (blind regions) on both curves are kept disjoint.
This can be achieved by an adaptive choice of $\epsilon$ during clipping, 
as follows:
%
\begin{itemize}
\item	Compute the zeroes of the weight function
	(projective coordinate) of {\em both} \atangs {\em and} 
	tangential $b$-curves: 
	say $t_0 < t_1 < t_2 < \ldots < t_m$.
\item	In clipping about $t_i$ by $(t_i - \epsilon_i, t_i + \epsilon_i)$,
	use $\epsilon_i = \frac{1}{3} \mbox{min} (t_i - t_{i-1}, t_{i+1}-t_i)$.
\end{itemize}

This $\epsilon$ is small enough to guarantee that the other 
tangential curve is not clipped away over this parameter interval.
That is, if there is a common tangent somewhere in this
parameter interval $(t_i - \epsilon_i, t_i + \epsilon_i)$,
it will be found on the other tangential curve
(while one tangential curve goes temporarily blind,
the other tangential curve continues to see).

\begin{example}
Since the parameter values of a horizontal and vertical tangent on a curve 
can be arbitrarily close, any fixed choice of $\epsilon$ during
clipping about a zero is doomed to miss common tangents on some curves.
The choice of $\epsilon$ must be adaptive.
\end{example}

END OF 3RD VERSION.
}

\subsection{Clipping and negative weights}
\label{sec:negative}

Looking at (\ref{eq:wt}),
it is clear that some of the weights of a \tang can be negative.
Negative weights are undesirable, 
since curve segments with negative weights
lack the convex hull property that is so useful for divide and conquer
techniques of intersection using subdivision.
Goldman and DeRose \cite{goldman86} have shown how to intersect curves without
the convex hull property, using an expanded convex hull,
but we have no need for this special intersection algorithm.

The use of clipped tangential curves in the final algorithm
% The clipping at diagonal tangents in the final algorithm 
has an added benefit:
it removes the problem of negative weights.
The clipping of more-horizontal tangents from a tangential $a$-curve,
and more-vertical tangents from a $b$-curve, is equivalent to the clipping 
of a neighbourhood about all points at infinity (zero weights).
% Recall that a horizontal tangent is a point at infinity on the $a$-curve.
This is equivalent to the clipping of all sign changes of the weight.
The weights of the remaining segments are purely positive or purely negative.  
The negative weights on a segment with purely negative weights can be easily
removed by multiplying all the weights by $-1$,
since rational Bezier curves are invariant under 
transformations $w_i = kw_i$ of the weights.
Thus, all weights remaining on the clipped tangential curves are positive.

The issue of negative weights is another indication that the use of two clipped
curves is a natural solution to the robust reduction of common tangency
to the intersection of rational Bezier curves.

\Comment{
OLD SOLUTION:
We choose a simpler solution that directly gets rid of the negative weights.
(Since this solution only becomes provably valid when combined with 
the approach of Section~\ref{sec:infinity},
it must be used with care if used to expunge negative weights
in general situations.)

It is actually the segments with both positive
and negative weights that cause problems.
A segment with purely negative weights can be easily
corrected by multiplying all the weights by $-1$
(since rational Bezier curves are invariant under 
transformations $w_i = kw_i$, $k \neq 0$ of the weights).
The weight function of a segment with both positive and negative weights
must contain a zero between consecutive weights of opposite sign, by continuity.
Since zero weights are associated with points at infinity,
a change in sign of the weight on a curve segment 
is associated with a point at infinity of the curve.\footnote{In the 
	tangential curve of Corollary~\ref{thm:rationaldual},
	zeroes of the weight function are associated with tangents $ax+by=0$
	through the origin.}
By subdividing at the zeroes of the weight function
$w(t) = \sum w_i B_i^n(t)$,
we can replace the undesirable segments
by segments with purely positive or purely negative weights.
We actually subdivide just before and after a zero,
since we don't want zero weights at the endpoints of a segment either
(which represent points at infinity), and since we cannot
compute the zeroes exactly anyway.
The segment that contains the zero weight in its interior (the zero-segment)
is then discarded.
This could discard some intersections, but this will be resolved
in Section~\ref{sec:infinity} (see Corollary~\ref{cor:insidesquare}).
% This is valid since the zero-segment, if built correctly,
% will only contain segments close to infinity, uninvolved with the
% intersections.

Our algorithm for the removal of negative weights is as follows.
Consider a curve segment C(t) of degree $n$ with weights $w_i$,
some of which are positive and some negative.
The weight function of this curve is $w(t) = \sum w_i B_i^n(t)$.
%
\begin{itemize}
\item Find zeroes $t = t_{0,0},t_{0,1},\ldots,t_{0,k}$ of the weight function
	$w(t) = \sum w_i B_i^n(t)$, $t \in [a,b]$.
	Although this is a classical solution of a univariate equation,
	we pose it as the intersection of the weight curve 
	$(t,f(t)) = \sum (a + \frac{i}{n} (b-a), w_i) B_i^n(t)$
	and the line $(t,0) = \sum (a + \frac{i}{n} (b-a), 0)B_i^n(t)$.
	This takes advantage of the Bezier nature of the weight function.
%
% Although this solution does not appear to relate to classical methods for the 
% solution of a univariate equation, it is actually very similar 
% to the classical bisection solution of a univariate equation
% because of the midpoint subdivision nature of Bezier curve intersection.
% NOT REALLY.
\item Subdivide C(t) at $t = t_{0,0} \pm \epsilon,\ldots,t_{0,k} \pm \epsilon$,
	on either side of the zero weights.
\item Discard the segments $C(t_{0,i}-\epsilon,t_{0,i}+\epsilon)$
	surrounding a zero.
\item The weights of the remaining segments are purely positive or
	purely negative.  Flip the sign of the purely negative segments.
\end{itemize}

\begin{example}
Example.
Give weights of a bad segment, zeroes, and resulting segments.
\end{example}
END OF OLD VERSION
}

\section{Results}
\label{sec:results}

We would like to illustrate the algorithm on some examples 
(Figures~\ref{fig:eg1}-\ref{fig:eg4}).
In all of these examples, on the left we show a pair of curves 
and their common tangents,
while on the right we show the dual world of 
the clipped tangential $a$-curves and $b$-curves and their intersections.
All common tangents are shown, visible and invisible.
Figure~\ref{fig:appl2} shows an example of visible common tangents only.
Figure~\ref{fig:eg1} is a simple example to introduce the method.
Figures~\ref{fig:eg2}-\ref{fig:eg3} are more challenging examples.
Notice how the common tangents in Figure~\ref{fig:eg3} are still found,
and still computed robustly, even in high curvature regions of the
curves, where the problem is fragile and ill-conditioned.
Figure~\ref{fig:eg4} shows the computation of the common tangents
between a single curve.
The curves and their tangential curves are the same as in Figure~\ref{fig:eg2}:
we have simply zoomed in on the tangential curves to better show detail
near the intersections.

\newbox\jjduegabox
\newdimen\jjduegawd
\font\jjduegaa=jjduegaa at 72.27truept
\font\jjduegab=jjduegab at 72.27truept
\setbox\jjduegabox=\vbox{\hbox{%
\jjduegaa\char0\char1\char2\char3\char4\char5}}
\jjduegawd=\wd\jjduegabox
\setbox\jjduegabox=\hbox{\vbox{\hsize=\jjduegawd
\parskip=0pt\offinterlineskip\parindent0pt
\hbox{\jjduegaa\char0\char1\char2\char3\char4\char5}
\hbox{\jjduegaa\char6\char7\jjduegab\char0\char1\char2\char3}
\hbox{\jjduegab\char4\char5\char6\char7\char8\char9}}}
\ifx\parbox\undefined
    \def\setjjduega{\box\jjduegabox}
\else
    \def\setjjduega{\parbox{\wd\jjduegabox}{\box\jjduegabox}}
\fi


\begin{figure*}
\hspace{.5in} \setjjduega
\caption{A simple example}
% dual ob2.rawctr, intersections and all common tangents
% jjduega.gif
\label{fig:eg1}
\end{figure*}

\newbox\jjduegbbox
\newdimen\jjduegbwd
\font\jjduegba=jjduegba at 72.27truept
\font\jjduegbb=jjduegbb at 72.27truept
\setbox\jjduegbbox=\vbox{\hbox{%
\jjduegba\char0\char1\char2\char3\char4\char5}}
\jjduegbwd=\wd\jjduegbbox
\setbox\jjduegbbox=\hbox{\vbox{\hsize=\jjduegbwd
\parskip=0pt\offinterlineskip\parindent0pt
\hbox{\jjduegba\char0\char1\char2\char3\char4\char5}
\hbox{\jjduegba\char6\char7\jjduegbb\char0\char1\char2\char3}
\hbox{\jjduegbb\char4\char5\char6\char7\char8\char9}}}
\ifx\parbox\undefined
    \def\setjjduegb{\box\jjduegbbox}
\else
    \def\setjjduegb{\parbox{\wd\jjduegbbox}{\box\jjduegbbox}}
\fi


\begin{figure*}
\hspace{.5in} \setjjduegb
\caption{A more complicated example}
% dual ob1.rawctr, intersections and all common tangents
% jjduegb.gif
\label{fig:eg2}
\end{figure*}

\newbox\jjduegcbox
\newdimen\jjduegcwd
\font\jjduegca=jjduegca at 72.27truept
\font\jjduegcb=jjduegcb at 72.27truept
\font\jjduegcc=jjduegcc at 72.27truept
\setbox\jjduegcbox=\vbox{\hbox{%
\jjduegca\char0\char1\char2\char3\char4\char5}}
\jjduegcwd=\wd\jjduegcbox
\setbox\jjduegcbox=\hbox{\vbox{\hsize=\jjduegcwd
\parskip=0pt\offinterlineskip\parindent0pt
\hbox{\jjduegca\char0\char1\char2\char3\char4\char5}
\hbox{\jjduegca\char6\char7\jjduegcb\char0\char1\char2\char3}
\hbox{\jjduegcb\char4\char5\char6\char7\char8\jjduegcc\char0}}}
\ifx\parbox\undefined
    \def\setjjduegc{\box\jjduegcbox}
\else
    \def\setjjduegc{\parbox{\wd\jjduegcbox}{\box\jjduegcbox}}
\fi


\begin{figure*}
\hspace{.5in} \setjjduegc
\caption{Another complicated example}
% dual ob3.rawctr, intersections and all common tangents
% jjduegc.gif
\label{fig:eg3}
\end{figure*}

\newbox\jjdusinbox
\newdimen\jjdusinwd
\font\jjdusina=jjdusina at 72.27truept
\font\jjdusinb=jjdusinb at 72.27truept
\setbox\jjdusinbox=\vbox{\hbox{%
\jjdusina\char0\char1\char2\char3\char4\char5}}
\jjdusinwd=\wd\jjdusinbox
\setbox\jjdusinbox=\hbox{\vbox{\hsize=\jjdusinwd
\parskip=0pt\offinterlineskip\parindent0pt
\hbox{\jjdusina\char0\char1\char2\char3\char4\char5}
\hbox{\jjdusina\char6\char7\jjdusinb\char0\char1\char2\char3}
\hbox{\jjdusinb\char4\char5\char6\char7\char8\char9}}}
\ifx\parbox\undefined
    \def\setjjdusin{\box\jjdusinbox}
\else
    \def\setjjdusin{\parbox{\wd\jjdusinbox}{\box\jjdusinbox}}
\fi


\begin{figure*}
\hspace{.5in} \setjjdusin
\caption{Common tangents of a single curve}
% dual ob1.rawctr, self-intersections and all common tangents between single curve
% jjdusingle.gif
\label{fig:eg4}
\end{figure*}

\Comment{
IT WILL BE DIFFICULT TO IMPLEMENT: MUST TRANSLATE EACH PIECE OF THE SPLINE
INDIVIDUALLY, BUILD POWER-BASE POLYNOMIALS, COMPUTE RESULTANTS, ETC.
THE ADVANTAGES OF THE PARAMETRIC SOLUTION ARE CLEAR WITHOUT FURTHER ANALYSIS.
Comparison of execution times for our method (parametric curves in dual space)
and the classical method (implicit curves in parametric space)
in Table~\ref{tab:speedBajaj}.

\begin{table}
\begin{tabular}{|c|c|c|}  	\hline
Data set & Time, old solution (seconds) & Time, our new solution\\ \hline
1 & ? & ? \\ \hline
2 & ? & ? \\ \hline
3 & ? & ? \\ \hline
\end{tabular}
\caption{Improved efficiency of our solution}
\label{tab:speedBajaj}
\end{table}

OVERKILL (ALTHOUGH SPEED DIFFERENCE IS NOTICEABLE); PREFER PITHINESS.
The increase in efficiency gained by clipping at diagonal tangents
is illustrated in Table~\ref{tab:speed}.
{Also show figures of difference, like in jones.rgb?}
% Comparison of execution time of dual and dual-00Aug23preDiag:
% clipped by diagonals and clipped only about zeros
% (i.e., computing all intersections)

\begin{table}
\begin{tabular}{|c|c|c|}  	\hline
Data set & Time, clipping immediately around zeroes (seconds) & Time, clipping at diagonal tangents\\ \hline
1 & ? & ? \\ \hline
2 & ? & ? \\ \hline
3 & ? & ? \\ \hline
\end{tabular}
\caption{Improved efficiency through clipping}
\label{tab:speed}
\end{table}
}

\section{Other work on common tangents}
\label{sec:prevcommon}

% MAKE IT SIMPLE: INTERSECTION OF PARAMETRIC RATHER THAN INTERSECTION OF IMPLICIT;
% SIMPLE SUBDIVISION OF PARAMETRIC INTERSECTION RATHER THAN COMPLICATED SUBDIVISION OF PARIDA.

In the introduction, we saw that the classical solution to the common tangents
of two parametric curves $C(s)$ and $D(t)$ of degree $n$
reduces to the intersection of two implicit curves $f(s,t)=0$ and $g(s,t)=0$
of degree $2n-1$.
The symbolic solution of this system using resultants is expensive,
since it isolates the coordinates of the solutions independently,
and reduces to the filtering of $O(16n^4)$ parameter pairs (all combinations
of $O(4n^2)$ $s$- and $t$-coordinates)
through the evaluation of the curves $C(s)$ and $D(t)$ and their hodographs $C'(s)$ and $D'(t)$
at each parameter pair.
The resultant solution also has difficulty with the Bezier structure 
of the curves.
A numerical solution using techniques of Nishita et. al. \cite{nishita90}
or Sherbrooke and Patrikalakis \cite{sherbrooke93} is more attractive, 
where the implicit curves are expressed in the Bernstein basis and intersected
using Bezier subdivision techniques.
% Thus, both methods can take advantage of the efficiency and stability of operations 
% on Bezier curves (e.g., divide-and-conquer by subdivision and the convex-hull property).
% The clipping of the Bezier curves (all segments lie inside the strip 
% $x \in [-1,1]$) also contributes strongly to efficient intersection.

The contrast between the classical solution and our method 
is between an intersection of degree $2n-1$ implicit
curves in $(s,t)$ parameter space and the intersection of degree $2n-1$ parametric
Bezier curves in dual space.
The parametric solution in dual space is appealing,
redrafting the solution in terms of the more familiar intersection
of Bezier curves.
The intersections in dual space are significantly accelerated by 
their restriction to the range $x \in [-1,1]$.
\Comment{
The primary advantage of the implicit solution is that it involves one intersection,
whereas the proper treatment of points at infinity necessitates two intersections 
in the dual space solution, one of $a$-curves and the other of $b$-curves.
However, two intersections of Bezier curves in $x \in [-1,1]$ is still very competitive
with a single intersection of implicit curves.
}
Moreover, the parametric solution in dual space generalizes naturally
to a solution for common tangent planes of surfaces in 3-space,
reducing to the intersection of two Bezier surfaces in 3-space,
a simpler solution than the generalization of the classical solution
to the intersection of several implicit surfaces in 4-space.
\Comment{
This point of view bears considerable fruit in higher dimensions,
where our solution generalizes more naturally and efficiently to the 
calculation of common tangent planes of two surfaces than the implicit solution.
The paradigm in dual space (dualize the tangent space, intersect in dual space,
and map intersections back to common tangents) reduces common tangent planes
to the intersections of 2 surfaces in 3-space.
However, the implicit solution to common tangent planes reduces to the
intersection of 3 implicit surfaces in the 4-space $(s,t,u,v)$ 
formed by the tensor product of the 2-dimensional parameter spaces 
of the surfaces.
}
\Comment{
SUBMITTED PAPER VERSION, USING SYMBOLIC SOLUTION:
Suppose that $C$ and $D$ are Bezier splines.
Unfortunately, the implicit curve solution does not take advantage
of the Bezier structure of the original curves,
and is actually undermined by it.
$f$ and $g$ are piecewise polynomial and 
we must reduce them to their constituent polynomials in order
to apply the basic solution (see below), resulting
in many polynomial intersection pairs.

The solution for a pair of polynomials is itself rather expensive.
Suppose that we have decomposed the implicit intersection into
its constituent polynomial implicit intersections.
The solution for polynomials $f$ and $g$ is as follows.
We must compute and solve the resultant of $f$ and $g$
with respect to both $s$ and $t$, which isolates the $O(4n^2)$ $s$- and 
$t$-coordinates of the intersections.
We must then evaluate $f(s,t)$ and $g(s,t)$ for all $O(16n^4)$ parameter pairs
to filter out the true intersections.
Each evaluation of $f$ and $g$ involves the evaluation of 
the two curves $C(s)$ and $D(t)$ and their hodographs $C'(s)$ and $D'(t)$.
% Therefore, this is a time-consuming computation.
% This computation depends on the fact that $f$ and $g$ are polynomial
% (for the resultant).
}
\Comment{
The $s$-coordinates $\{s_i\}$ of the intersections are found by computing
the Sylvester resultant of the two polynomials with respect to $s$, $R_s(f,g)$,
and solving this univariate equation of degree $(2n-1)^2$ in $s$.
The $t$-coordinates $\{t_j\}$ of the intersections are found similarly, 
by solving $R_t(f,g)=0$.
This generates a superset $\{(s_i,t_j)\}$ of the intersections,
which must be filtered down to the intersections by evaluating
$f(s_i,t_j)$ and $g(s_i,t_j)$ for all pairs, saving those pairs that
satisfy both equations.
}

\Comment{
There are also advantages to working in a dual space where points directly correspond
to tangent lines (the points of the curves $f(s,t)$ and $g(s,t)$ in parameter space have
no direct correspondence with tangents).
}

% (and require translation to a polynomial basis
% from the Bernstein basis, for the calculation of resultants).
% ({\bf Or is there a resultant of Bezier polynomials?})

Another solution to common tangents has been proposed by 
Parida and Mudur \cite{parida95}, using a geometric divide-and-conquer approach.
Their solution works as follows.
The curves are first decomposed into 'C-shaped curves', convex segments
monotone in tangent direction.
Each pair of C-curves, one from each curve, is compared to determine
if the pair can define a common tangent, as follows.
The pair is first reparameterized and clipped so that the C-curves
share the same parameter interval and have the same tangent directions at the endpoints. 
If certain rejection criteria involving the tangent range are satisfied,
the pair is rejected.
If the two C-curves are pseudo-linear, a test for a common tangent is
directly applied.
Otherwise, both C-curves are subdivided at their shoulder points (the point
whose tangent is parallel to the chord of the endpoints) and the process
is repeated recursively.
The subdivision involved in this solution is more complicated than the
simple midpoint subdivision of Bezier intersection,
requiring the calculation of a shoulder point at each subdivision
and reparameterization.
Moreover, an initial decomposition into C-curves is expensive, and 
there is more subdivision.
% A great deal of decomposition would be required for complicated examples
% such as Figure~\ref{fig:result1234} and \ref{fig:result6789},
% which are much more complicated than the examples in \cite{parida95}.

% Moreover, it does not lead to a generalizable solution.

We observe that Sederberg and Nishita \cite{sederberg90} solve a special case of the common
tangent problem, showing how to compute the points of tangency of two plane Bezier curves
(i.e., an intersection point of the curves where they share the same tangent).

\section{Why not the hodograph?}
\label{sec:hodo}

Our solution works with the tangent space of a curve.
In this section, we explore why we don't use the standard representation of 
the tangent space of a curve, the hodograph.
\Comment{
	The hodograph is the locus traced by its tangent vectors moved to the origin.
	Like the tangential curve, the hodograph is a clean, concise representation 
	of the tangent space as another curve.
	Better yet, the hodograph of a Bezier curve of degree $n$
	is another Bezier curve of degree $n-1$ (rather than a rational Bezier curve
	of degree $2n-1$ for the 
	tangential curve) that is easily derived from the original curve.
}
Unfortunately, the intersection of hodographs does not yield common tangents.
The problem is that the hodograph represents a tangent vector, not a tangent line.
A point $H(t)$ of the hodograph of a curve $C(t)$ represents the tangent 
vector at $C(t)$, so the tangent line at $C(t)$ is $C(t) + sH(t)$.
There are two problems with this representation:
(1) the tangent line depends on both the hodograph and the original curve,
and
(2) many points in the hodograph's space represent the same tangent line:
    in particular, 
    all points along the same line through the origin represent the same
    direction and thus the same tangent line.
(2) means that a common tangent of two curves does not imply an intersection
of their hodographs unless the tangent vectors at either end of the common
tangent are exactly the same length.
(1) means that, conversely, an intersection of the hodographs does not
imply a common tangent, but just two parallel tangents.
Thus, hodographs are not useful for finding common tangents.

\Comment{
\begin{rmk}
Notice that a common tangent is a global property of the curves,
while a tangent is a local property,
which explains why the computation of common tangents requires more 
sophistication than the computation of tangents.
\end{rmk}

Let $C(t)$ be a plane polynomial Bezier curve of degree $n$
with control points $\{b_i\}_{i=0}^n$:
\[
C(t) = \sum_{i=0}^n b_i B_i^n(t)
\hspace{1in} t \in [0,1]
\]
where $B_i^n(t) = \tinychoice{n}{i} (1-t)^{n-i} t^i$ is the $i^{th}$ Bernstein
polynomial of degree $n$.
The {\bf hodograph} of $C(t)$ is another Bezier curve, representing
the first derivative or tangent of $C(t)$:
\[
C'(t) = (c_1'(t), c_2'(t)) = \sum_{i=0}^{n-1} n\Delta b_i B_i^{n-1}(t)
\]
where $\Delta b_i = b_{i+1} - b_i$ \cite{farin97}.
}

\section{Other work using duality}
\label{sec:duality}

Our solution involves dual space, a popular tool for geometric analysis.
In this section, we compare it to other work using duality.
The abstract idea of dualizing tangent spaces into curves
is introduced in \cite[p. 54]{hartshorne}.
Dualities between a point and a hyperplane
are often used in computational geometry
\cite{edels87,orourke94},	% [p. 214]
usually for the translation of problems on configurations of points 
to problems on finite arrangements of hyperplanes.
% Notice that here points are translated to hyperplanes,
% while we are interested in translating lines to points.

A representation for plane curves using duality
has been developed by Hoschek, called dual Bezier curves \cite{hoschek83},
where a plane curve is represented as the envelope of a set of lines
rather than as a set of points.
% a plane curve as the envelope of a line family.
A line family is represented in terms of control lines $L_i$,
% as $\sum_{i=0}^n \mbox{dual}(L_i) B_i^n(t)$
and the curve is understood to be the envelope of this line family.
A point of the curve can still be evaluated using a de Casteljau algorithm
to refine the control lines.
Pottmann gives a formula for translating between the
Bezier curve (control points) and dual Bezier curve (control lines)
representations \cite{pottmann95}. % p. 429
% The curve is represented using a control 'polygon' whose control points are duals of lines.
While the dual curve is a curve in primal space expressed in a dual control structure (control lines rather than control points)
for the representation of plane curves,
the \tang is a curve in dual space expressed in a traditional control structure
for the analysis of tangent spaces.
In particular, the dual Bezier curve is not appropriate for our intersection
problem in dual space.
% This subtle difference is important for the analysis of tangential structure.
% It reflects the different goals of the two representations:
% the goal of Hoschek's dual Bezier curve is an alternate representation for plane curves,
% while the goal of the \tang is a tool for the analysis of tangent spaces.
% \footnote{Note 
%	that the purpose of dual curves is an alternate representation of plane curves,
%	not an analysis of the tangent space of a curve.}

\Comment{
	Not after Pottmann translates back and forth:
	Hoschek's dual Bezier curve is effectively a mirror image of our tangential curve: 
	%
	% it is given a family of lines and builds a curve, while the tangential curve
	% is given a curve and builds a representation of its family of tangents.
	%
	it builds a plane curve from a family of lines 
	% (specified indirectly by a set of control lines), 
	while our \tang builds a family of tangent lines from a curve.
	PROBLEM WITH PUTTING IT THIS WAY IS THAT THE DUAL CURVE CAN BE
	REVERSED, EXPRESSING A PLANE CURVE AS A FAMILY OF LINES
}
% For example, Hoschek never builds a curve in dual space.

\ifJournal
Comparison with Zorin:
different treatment of points at infinity: 2 clipped curves, not hypercube;
we work with Bezier curves rather than subdivision surfaces
devil is in the details since idea of mapping to dual space is not new anyway
\fi

\section{Equivalence of a \plucker solution}
\label{sec:plucker}

Our solution involves the mapping of lines to points.
We have used the duality of Definition~\ref{defn:dual}.
Why don't we use the classical \plucker representation of a line by a point?
A solution to common tangency can indeed be developed using a mapping of 
tangent spaces to curves in \plucker space and intersection of these curves.
It turns out that this \plucker solution is almost entirely equivalent.
We use the dual space solution since its development is more natural
for lines in 2-space, and leads to a slightly simpler implementation.

Our dual representation of a line only works in 2-space,
since the line is a hyperplane (representable by a single implicit equation)
in 2-space but not in 3-space.
%	A line in 3-space requires two implicit equations for its representation,
% 	which moreover are not unique even in a projective sense,
%	so it cannot be represented using this simple version of duality.
However, in 2-space, 
the dual representation is a bit simpler, more intuitive, and direct than \plucker coordinates.
As a representation of any line in 3-space, \plucker coordinates are more general
and \plucker space is a higher-dimensional version of our dual space:
a 5-dimensional, rather than 2-dimensional, projective space.
In all other ways, the solution of the common tangency problem in \plucker space 
is identical to our solution in dual space.
In particular, a tangent space is again mapped to a curve in \plucker space, 
and the tangential curve in \plucker space is equivalent to the tangential curve
in dual space.
This can be seen as follows.

\begin{defn2}
The {\bf \plucker tangential curve} of $C(t)$ is the curve $C^{**}(t) \subset P^5$,
where $C^{**}(t)$ are the \plucker coordinates of the tangent at $C(t)$.
\end{defn2}

The \plucker coordinates of the line $P+uV$ are $(V,P \times V)$.
Since the tangent line at $C(t)$ is $C(t) + uC'(t)$,
the \plucker tangential curve of $C(t)$
% the image of the tangent space of the curve $C(t)$ in \plucker space 
is the curve $(C'(t), C(t) \times C'(t))$.
If $C(t) = (c_1(t),c_2(t),0)$ is a plane curve,
this \plucker tangential curve becomes
\begin{equation}
\label{eqn:plucker}
	(c'_1, c'_2, 0, 0, 0, c_1c'_2 - c'_1c_2)
\end{equation}
in projective 5-space.
Compare this to (\ref{eq:dual1}).
This establishes the essential equivalence of the \plucker solution.

\Comment{
Since a \plucker solution is identical,
the computations in the previous sections can be trivially lifted to
a solution in \plucker space, if desired.
For example, the tangential curve is still 3-dimensional,
and one coordinate must be interpreted as a projective coordinate,
leading to the same solution to intersections at infinity by trimming
two sets of intersections.
}

\ifJournal
The 3 dimensions of the \plucker tangential curve is unusual
and can be explained as follows.
In general, the image of a one-dimensional family of lines is a curve
on the Klein quadric in \plucker space, and thus lies in projective 5-space.
Only in special cases like line fields (lines that lie in a fixed plane)
and line bundles (lines that pass through a fixed point)
does this curve lie in a lower-dimensional space \cite{pottmann99}.
Fortunately, the tangent space of a plane curve is a line field.	
\fi
	% Moreover, the two plane curves lie in the same plane, since the line fields
	% of the tangent spaces are identical.
\Comment{
Curve intersection in the plane is well understood, even if this plane
resides in projective 5-space (affine 6-space).
We can ignore the presence of the higher dimensions by rotating the
two-dimensional plane of the curves to the $\{z_3=0,z_4=0,z_5=0\}$
plane and ignoring the 3rd, 4th, and 5th coordinates.
After computing the intersection using classical 2-dimensional algorithms
(Sederberg), the intersection points should be rotated back before
translating their Plucker coordinates to lines.
}
\ifTalk
% No difference here either, since we use parameter values.
Indeed, the only tangible difference between the two methods is 
the translation of a point back to a line in primal space.\footnote{The
	line associated with the point $(v,m)$ in \plucker space
	is the line through $\frac{m \times v}{\|v\|^2}$
	and $\frac{m \times v}{\|v\|^2} + v$ \cite[p. 84]{selig96}.}
\fi

\ifJournal
Additional sections: computing from a point; size of visibility graph.
\fi

\section{Conclusions}
\label{sec:conclude}

This paper explores the implementation of the following idea:
common tangency in primal space as parametric curve intersection in dual space.
The solution that we have developed is efficient, robust, and simple to implement.
The ability to efficiently compute common tangents
allows us to revisit problems in visibility, shortest path motion, and lighting
in a smooth environment composed of smooth obstacles or smooth area lights.
This work also establishes the foundations for
working with tangent spaces in a dual space, which is a powerful
paradigm for the analysis of tangent spaces of curves and surfaces.
A natural extension to
the related problem of computing the tangents of a curve through a point
% the common tangency problem when one of the curves degenerates to a point.
is addressed in \cite{jj00pole}.
Another natural extension to the calculation of common tangent planes of 
smooth surfaces is being explored.
% We are interested in the computation of common tangents of curves on surfaces,
% particularly for motion amongst obstacles on a surface.
% Finally, the \tang shows promise for the solution of other problems involving
% tangent spaces, aside from common tangents.

\section{Acknowledgements}

Thanks to the anonymous referees for their comments,
and to Helmut Pottmann for references on dual Bezier curves.

% I appreciate discussions with Helmut Pottmann.
% alerting me to Hoschek's dual Bezier curves

\bibliographystyle{latex8}
% \bibliographystyle{plain}
\begin{thebibliography}{99}

\bibitem{bajajkim87}
Bajaj, C. and M.-S. Kim (1987)
Convex hull of objects bounded by algebraic curves.
Technical Report CSD-TR-697, Computer Science, Purdue University.

\bibitem{edels87}
Edelsbrunner, H. (1987)
Algorithms in Combinatorial Geometry.
Springer Verlag (Heidelberg), Chapters 1 and 12.

\bibitem{farin97}
Farin, G. (1997)
Curves and Surfaces for CAGD: A Practical Guide (4th edition).
Academic Press (New York).

\bibitem{goldman86}
Goldman, R. and T. DeRose (1986)
Recursive subdivision without the convex hull property.
Computer Aided Geometric Design 3, 247--265.

\bibitem{hartshorne}
Hartshorne, R. (1977)
Algebraic Geometry.
Springer-Verlag (New York).

\bibitem{hoschek83}
Hoschek, J. (1983)
Dual Bezier curves and surfaces.
In {\em Surfaces in Computer Aided Geometric Design},
R. Barnhill and W. Boehm, eds.,
North Holland (Amsterdam), 147--156.

\bibitem{jj00pole}
Johnstone, J. (2001)
Smooth visibility from a point.
39th Annual ACM Southeast Conference, to appear.

\Comment{
\bibitem{kim89}
Kim, M.-S. (1989)
Motion Planning with Geometric Models.
Ph.D. thesis, Computer Science, Purdue University.
}

\bibitem{lasser89}
Lasser, D. (1989)
Calculating the self-intersections of Bezier curves.
Computers in Industry 12, 259--268.

\bibitem{nishita90}
Nishita, T. and T. Sederberg and M. Kakimoto (1990)
Ray Tracing Trimmed Rational Surface Patches.
SIGGRAPH '90, Computer Graphics 24(4), 337--345.

\bibitem{orourke94}
O'Rourke, J. (1994)
Computational Geometry in C.
Cambridge University Press (New York).

\bibitem{parida95}
Parida, L. and S. Mudur (1995)
Common tangents to planar parametric curves: a geometric solution.
Computer-Aided Design 27(1), 41--47.

\bibitem{pottmann95}
Pottmann, H. (1995)
Studying NURBS Curves and Surfaces with Classical Geometry.
In Mathematical Methods for Curves and Surfaces,
edited by M. Daehlen, T. Lyche and L. Schumaker,
413--438.

\ifJournal
\bibitem{pottmann99}
Pottmann, H. and M. Peternell and B. Ravani (1999)
An introduction to line geometry with applications.
Computer-Aided Design 31, 3--16.
\fi

\Comment{
\bibitem{rockwood90}
Rockwood, A. (1990)
Accurate display of tensor product isosurfaces.
Visualization '90, 353--360.
}

\bibitem{sederberg86}
Sederberg, T. and S. Parry (1986)
Comparison of three curve intersection algorithms.
Computer Aided Design 18, 58--63.

\bibitem{sederberg90}
Sederberg, T. and T. Nishita (1990)
Curve intersection using Bezier clipping.
Computer Aided Design 22(9), 538--549.

\bibitem{sherbrooke93}
Sherbrooke, E. and N. Patrikalakis (1993)
Computation of the solutions of nonlinear polynomial systems.
Computer Aided Geometric Design 10(5), 379--405.

\ifTalk
\bibitem{selig96}
Selig, J. (1996)
Geometrical Methods in Robotics.
Springer Verlag (New York).
\fi

\end{thebibliography}

\section{Appendix}

\Comment{
\begin{defn2}
A {\bf common tangent} of the curves C and D is a line that is tangent to both C and D.
We do not distinguish whether the line is tangent at one or more points
of each curve.
A {\bf common tangent of a single curve} C is a line that is tangent to C at two or more
distinct points.
\end{defn2}
}

% We review the relationship of projective space to rational Bezier curves.

% \begin{defn2}
% {\bf Projective 2-space} $P^2$ is the space 
% $\{(x_1,x_2,x_3) : x_i \in \Re, \mbox{ not all zero}\}$
% under the equivalence relation $(x_1,x_2,x_3) = k(x_1,x_2,x_3),\ k \neq 0 \in \Re$.
% \end{defn2}

{\bf Proof of Theorem~\ref{thm:Bezier}}:
Recall that we have established that the tangential curve of $C(t)$ is 
\begin{equation}
\label{eq:dual2}
  (-c_2'(t),\ \  c_1'(t),\ \ c_2'(t) c_1(t) - c_1'(t)c_2(t))
\end{equation}
We want to express the \tang as a Bezier curve.
\ifJournal
Distinguish the two $\Delta$'s!  One is parameter length and other
is difference of control points.
\fi
Consider the third coordinate of the tangential curve,
which can be expressed as: 
\[
\begin{array}{l}
\scriptstyle{\frac{1}{\Delta}}
\displaystyle{\sum_{i=0}^{n-1}} \scriptstyle{n \Delta b_{i,2} B_i^{n-1}(t)}
 \displaystyle{\sum_{j=0}^n}     \scriptstyle{         b_{j,1} B_j^n(t)} \\
  - 
\scriptstyle{\frac{1}{\Delta}}
 \displaystyle{\sum_{i=0}^{n-1}} \scriptstyle{n \Delta b_{i,1} B_i^{n-1}(t)}
 \displaystyle{\sum_{j=0}^n}	 \scriptstyle{         b_{j,2} B_j^n(t)}
\end{array}
\]
where $\Delta = t_2 - t_1$ is the length of the parameter interval
and $\Delta b_{i,2} = b_{i+1,2} - b_{i,2}$.
We can use the product formula
% \footnote{$B_i^m(t) B_j^n(t) = \frac{ \tinychoice{m}{i} \tinychoice{n}{j} }{ \tinychoice{m+n}{i+j} } B_{i+j}^{m+n}(t)$.}
for Bernstein polynomials \cite{farin97} to simplify this to
\[
\scriptstyle{\frac{n}{\Delta}}
\displaystyle{\sum_{i=0}^{n-1} \sum_{j=0}^n }
\frac{\tinychoice{n-1}{i} \tinychoice{n}{j}}{ \tinychoice{2n-1}{i+j} } 
\scriptstyle{ B_{i+j}^{2n-1}(t) [ \Delta b_{i,2} b_{j,1} - \Delta b_{i,1} b_{j,2} ]}.
\]
Letting $k=i+j$, this becomes
\[
\sum_{k=0}^{2n-1} B_k^{2n-1}(t) 
\sum_{\begin{array}{c} \mbox{\footnotesize{$0 \leq i \leq n-1$}} \\ 
			     \mbox{\footnotesize{$0 \leq j \leq n$}} \\ 
			     \mbox{\footnotesize{$i+j=k$}}
			     \end{array}}
\frac{n}{\Delta} \frac{ \tinychoice{n-1}{i} \tinychoice{n}{j} }{ \tinychoice{2n-1}{k} }
\beta_{i,j}
\]
The third coordinate is now expressed as a 1-dimensional Bezier curve of 
degree $2n-1$ with control points $\{b^*_{k,3} \}_{k=0}^{2n-1}$ where:
\begin{equation}
\label{eq:dualw2}
\scriptstyle{
b^*_{k,3} = \frac{n}{\Delta \tinychoice{2n-1}{k}}
\displaystyle{\sum_{\begin{array}{c} \mbox{\footnotesize{$0 \leq i \leq n-1$}} \\ 
			     \mbox{\footnotesize{$0 \leq j \leq n$}} \\ 
			     \mbox{\footnotesize{$i+j=k$}}
			     \end{array}}}
\tinychoice{n-1}{i} \tinychoice{n}{j} \beta_{i,j}.
}
\end{equation}

Now consider the first two coordinates of the \tang (\ref{eq:dual2}), 
\begin{equation}
\label{eq:dualx}
-c_2'(t) = \frac{1}{\Delta} \sum_{k=0}^{n-1} -n \Delta b_{k,2} B_k^{n-1}(t)
\end{equation}
and
\begin{equation}
\label{eq:dualy}
c_1'(t)  = \frac{1}{\Delta} \sum_{k=0}^{n-1}  n \Delta b_{k,1} B_k^{n-1}(t)
\end{equation}
These are 1-dimensional Bezier curves of degree $n-1$,
which we need to degree-elevate to degree $2n-1$
for compatibility with the third coordinate.
% Any polynomial curve can clearly be expressed as a polynomial
% curve of higher degree, simply by padding on zero terms.
% The translation of a Bezier curve from degree $n$ to $n+r>n$ (degree elevation)
% is more involved,
% since the defining control points must move.
% However, 
%
A Bezier curve of degree $n$ with control points $\{ d_k \}_{k=0}^n$ 
is degree elevated \cite{farin97} to a Bezier curve of degree $n+r$ with control points
$\{ d_k^{(r)} \}_{k=0}^{n+r}$ where:
\[
d_k^{(r)} = \sum_{j=\mbox{\footnotesize{max}}(0,k-r)}^{\mbox{\footnotesize{min}}(n,k)} 
		d_j \frac{\tinychoice{n}{j} \tinychoice{r}{k-j} }{ \tinychoice{n+r}{k} }.
\]
Consequently, the 1-dimensional Bezier curve (\ref{eq:dualx}) 
of degree $n-1$ with control points $\frac{-n}{\Delta} \Delta b_{k,2}$
is degree elevated to a 1-dimensional Bezier curve of degree $2n-1$
with control points $\{b^*_{k,1} \}_{k=0}^{2n-1}$ where:
\begin{equation}
\label{eq:1stcoord}
b^*_{k,1} = 
\sum_{j=\mbox{\footnotesize{max}}(0,k-n)}^{\mbox{\footnotesize{min}}(n-1,k)} 
	\frac{-n}{\Delta} \Delta b_{j,2}
	\frac{ \tinychoice{n-1}{j} \tinychoice{n}{k-j} }{ \tinychoice{2n-1}{k} }
\end{equation}
%
\Comment{
By expanding $\frac{ \tinychoice{n-1}{j} \tinychoice{n}{k-j} }{ \tinychoice{2n-1}{k} }$
and regrouping, we can reexpress it as 
$\frac{ \tinychoice{2n-1-k}{n-1-j} \tinychoice{k}{j} }{ \tinychoice{2n-1}{n} }$
and (\ref{eq:1stcoord}) becomes
\begin{equation}
b^*_{k,1} = 
- \frac{n}{\Delta \tinychoice{2n-1}{n}} 
\sum_{j=\mbox{\footnotesize{max}}(0,k-n)}^{\mbox{\footnotesize{min}}(n-1,k)} 
	\tinychoice{2n-1-k}{n-1-j} \tinychoice{k}{j} (b_{j+1,2} - b_{j,2}).
\end{equation}
}
%
Similarly, the Bezier curve (\ref{eq:dualy})
of degree $n-1$ with control points $\frac{n}{\Delta} \Delta b_{k,1}$ 
can be degree elevated to a Bezier curve of degree $2n-1$ with control points 
$\{b^*_{k,2} \}_{k=0}^{2n-1}$ where:
\begin{equation}
\label{eq:2ndcoord}
b^*_{k,2} = 
\sum_{j=\mbox{\footnotesize{max}}(0,k-n)}^{\mbox{\footnotesize{min}}(n-1,k)} 
	\frac{n}{\Delta} \Delta b_{j,1}
	\frac{ \tinychoice{n-1}{j} \tinychoice{n}{k-j} }{ \tinychoice{2n-1}{k} }
\end{equation}
%
Combining (\ref{eq:dualw2}), (\ref{eq:1stcoord}) and (\ref{eq:2ndcoord}),
we conclude that the \tang $C^*(t)$
is a Bezier curve of degree $2n-1$ in projective space with control points 
$\{b^*_k\}_{k=0}^{2n-1}$ as in (\ref{eq:dualcoord}).
\QED

We next give formulae for the \atangs and $b$-curves.
These are computed just like our original tangential curves,
simply interpreting the projective coordinate anew.
%
\begin{theorem}
\label{thm:rationalduala}
Let $C(t)$ be a plane Bezier curve of degree $n$ 
with control points $\{ b_i \}_{i=0}^n$ over $[t_1,t_2]$.
The \atang $C_a^*(t)$ is a rational Bezier curve of degree $2n-1$ 
over $[t_1,t_2]$ with weights $\{w_k\}_{k=0}^{2n-1}$ where: 
\begin{displaymath}
\scriptsize{
w_k = \alpha
\sum_{j=\mbox{max}(0,k-n)}^{\mbox{min}(n-1,k)} 
\tinychoice{n-1}{j} \tinychoice{n}{k-j} (b_{j,2} - b_{j+1,2})
}
\end{displaymath}
%
and control points $\{c_k\}_{k=0}^{2n-1}$ where:
\begin{displaymath}
\tiny{
c_k = \frac{\alpha}{w_k} 
\left(
\begin{array}{l}
	\displaystyle{\sum_{j=\mbox{\tiny{max}}(0,k-n)}^{\mbox{\tiny{min}}(n-1,k)}}
	\tinychoice{n-1}{j} \tinychoice{n}{k-j} (b_{j+1,1} - b_{j,1})\\
	\displaystyle{\sum_{\begin{array}{c} \mbox{\tiny{$0 \leq i \leq n-1$}} \\ 
			     \mbox{\tiny{$0 \leq j \leq n$}} \\ 
			     \mbox{\tiny{$i+j=k$}}
			     \end{array}}}
\tinychoice{n-1}{i} \tinychoice{n}{j} \beta_{i,j}
\end{array}
\right)
}
\end{displaymath}
where $\alpha$ and $\beta_{i,j}$ are as in Theorem~\ref{thm:Bezier}.
\end{theorem}

\begin{theorem}
\label{thm:rationaldualb}
Let $C(t)$ be a plane Bezier curve of degree $n$ 
with control points $\{ b_i \}_{i=0}^n$ over $[t_1,t_2]$.
The \btang $C_b^*(t)$ is a rational Bezier curve of degree $2n-1$ 
over $[t_1,t_2]$ with weights $\{w_k\}_{k=0}^{2n-1}$ where: 
\begin{displaymath}
\scriptsize{
w_k = \alpha
\sum_{j=\mbox{max}(0,k-n)}^{\mbox{min}(n-1,k)} 
\tinychoice{n-1}{j} \tinychoice{n}{k-j} (b_{j+1,1} - b_{j,1})
}
\end{displaymath}
%
and control points $\{c_k\}_{k=0}^{2n-1}$ where:
\begin{displaymath}
\tiny{
c_k = \frac{\alpha}{w_k} 
\left(
\begin{array}{l}
	\displaystyle{\sum_{j=\mbox{\tiny{max}}(0,k-n)}^{\mbox{\tiny{min}}(n-1,k)}}
	\tinychoice{n-1}{j} \tinychoice{n}{k-j} (b_{j,2} - b_{j+1,2})\\
	\displaystyle{\sum_{\begin{array}{c} \mbox{\tiny{$0 \leq i \leq n-1$}} \\ 
			     \mbox{\tiny{$0 \leq j \leq n$}} \\ 
			     \mbox{\tiny{$i+j=k$}}
			     \end{array}}}
\tinychoice{n-1}{i} \tinychoice{n}{j} \beta_{i,j}
\end{array}
\right)
}
\end{displaymath}
where $\alpha$ and $\beta_{i,j}$ are as in Theorem~\ref{thm:Bezier}.
\end{theorem}

\end{document}

