From dual.tex.

\title{Intersection of two rational curves with negative weights}

{\bf Does Section~\ref{sec:negative} generalize to the translation of 
all segments with negative weights?
Does this generalize to all intersection of rational curves?
This could be a general technique for the intersection of 
two rational curves with negative weights without Goldman/deRose's convex hull
expansion.
But be careful about the clipped intersections.

A possible solution: use different projective coordinates to map 
rational curve from projective space to affine space,
and retrieve the intersections near infinity (that may lie on a clipped
segment) through another choice of projective coordinate.

Another possible idea for intersection of rational curves with negative 
weights, using line bundles in projective space, is as follows.
This leads to 3 solutions: Goldman/DeRose, robust clipping about zero weights,
and line-bundle intersection in projective space, which could be written
up as a report, if not just for the interest of the problem
of negative weights.

The above ideas can be applied to the intersection of two line bundles
through the same point,
which gives another solution to the intersection of two plane rational
curves with negative weights.
The lines of the projective version of a rational curve 
all pass through the origin,
so the rational 'ruled surface' is actually a bundle \cite{pottmann99}.
The intersection of two bundles with the same common point
is quite different from the intersection of two ruled surfaces,
and is amenable to a simpler solution.
In particular, the intersection of two bundles with the same common point
is a collection of lines, not a collection of points.
{\bf This makes it feasible to compute the intersection in \plucker space.}
(For general ruled surfaces and general bundles,
although they have images in \plucker space,
the intersection of these images is in general empty, since the 
ruled surfaces or bundles intersect in points rather than lines.
Since our bundle intersection does consist of lines, it is 
expressible in \plucker space.)
Intersection in \plucker space is simple, 
since the lines of a bundle map to points of a plane in \plucker space. % Pottmann
**This makes the intersection of two bundles with the same common point in Euclidean space 
equivalent to the intersection of two curves in a plane in Plucker space.**

Using this idea, common tangency could be computed as follows,
although this is much too convoluted.
\begin{itemize}
\item	Compute dual curves.
\item	Intersect associated bundles in Plucker space.
\item 	Translate intersections back to lines in dual space.
\item 	Translate intersection lines in dual space back to points of common
	tangency on the original curves.
\end{itemize}
}

This strategy dovetails beautifully 
with the clipping of segments near infinity in Section~\ref{sec:negative}.
The clipping of a segment near infinity on an $a$-tangential curve
(motivated by the removal of negative weights from the tangential curve)
may remove an intersection not only at infinity but near infinity,
but the common tangent associated with this missed intersection
will be found as an intersection of the $b$-tangential curves instead.

