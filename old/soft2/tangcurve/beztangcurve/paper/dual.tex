\documentclass[11pt]{article}
\newif\ifVideo
\Videofalse
\newif\ifTalk
\Talkfalse
\input{header}
\newcommand{\plucker}{Pl\"{u}cker\ }
\newcommand{\tang}{tangential curve\ }
\newcommand{\tangs}{tangential curves\ }
\newcommand{\Tang}{Tangential curve\ }
\DoubleSpace

\setlength{\oddsidemargin}{0pt}
\setlength{\topmargin}{-.2in}	% should be 0pt for 1in
% \setlength{\headsep}{.5in}
\setlength{\textheight}{8.5in}
\setlength{\textwidth}{6.5in}
\setlength{\columnsep}{5mm}	% width of gutter between columns
\markright{The dual curve: \today \hfill}
\pagestyle{myheadings}
% -----------------------------------------------------------------------------

\title{On the common tangents of plane curves}
% Dual curves and common tangents
% A tool for tangential analysis}
% A Tangential Tool
% 
% Tangency as intersection: dual curves for tangential analysis, 
%	with applications to motion planning}
% Tangential computation with the dual curve} % for visibility and shadows
% Computing tangents using the dual curve for shortest path 
%	motion in a curved environment
\author{J.K. Johnstone\thanks{Geometric Modeling Lab, 125 Campbell, 
	Computer and Information Sciences, UAB, Birmingham, AL 35294.}}

\begin{document}
\maketitle

\begin{abstract}
We present a tool for tangential analysis, the tang,
and show how it can be used to improve the calculation
of common tangents between two curves, between a single curve, or between 
a point and a curve.
The tangential information in a single curve is well captured by a hodograph
\cite{farin97}, but hodographs do not handle 
the analysis of common tangents between curves well.
The tang also offers a superior solution to common tangents
than the classical but expensive mathematical mechanism of pole/polar theory.

The tang is a tool for studying the tangents of a curve.
Unlike the hodograph, it allows the computation of common tangents, 
and is more efficient than pole/polar theory.
We show how to compute the tang for a plane Bezier curve,
and how it can be used to compute common tangents of a single curve
and between two curves.
Common tangents are important in shortest path motion, visibility,
hull computations, and lighting.

A common tangent is a tangent of two curves, thus it is natural
to identify a common tangent with an intersection in some dual space.
We explore the creation of this dual space and translation to and from it.
\end{abstract} 

\clearpage

\section{Introduction}

A common tangent of two curves is a fundamental geometric primitive.
Consider a collection of closed curves in the plane, interpreted as 
obstacles for a point robot or as an area light source with occluders
(Figure~\ref{FIGURE OF COMPLICATED ROBOT ENVIRONMENT AND LIGHT WITH OCCLUDERS}).
% SHOW OFF THE POWER OF THE METHOD WITH A COMPLICATED EXAMPLE TO WOW EVERYONE
The common tangents of these curves define the limits of visibility,
and consequently the shortest paths that can be travelled by the robot 
and the shadows (umbra/penumbra) that are cast by the light
(Figure~\ref{FIGURE OF SEVERAL CURVES AND VISIBLE COMMON TANGENTS}).
Common tangents are also needed for basic geometric computations with curves,
such as the convex hull of a curve.

This paper studies the common tangent and develops an algorithm for its
computation.
Consider the tangent space of a curve as a family of tangent lines
(Figure~\ref{(a) FIGURE OF TANGENT BUNDLE: ALL TANGENTS OF A PLANE CURVE;
(b) EXPLICITLY SHOW INTERSECTION LINE OF TWO BUNDLES}).
A common tangent of two curves is an intersection of their line families.
The interpretation of common tangency as intersection
will be a central theme of this paper.
Since the intersection of line families is awkward,
we prefer to map to a dual space where intersection is more natural.
By encoding a tangent line as a point in a dual space,
a family of tangent lines becomes a curve in the dual space,
and a common tangent becomes an intersection of two curves in dual space.
This is preferable, since curve intersection is well understood.
This paper explores the implementation of this simple idea,
common tangency in primal space as intersection in dual space,
and the resolution of the difficulties that are encountered during the
implementation.

The paper is organized as follows.
Section~\ref{sec:dual} shows how to represent the tangent space of a plane curve
by a plane curve in a dual space, called a \tang.
We use a simpler alternative to the obvious \plucker solution,
where the dual space is 2-dimensional.
Section~\ref{sec:Bezier} expresses the \tang of a Bezier curve as another Bezier curve.
Section~\ref{sec:negative} discusses how to deal with negative weights in the 
tangential curve.
Our algorithm for common tangent computation is detailed in Section~\ref{sec:alg}.
Intersection in dual space and filtering of these intersections
is dealt with in Section~\ref{sec:intersect}.
Experimental results are given in Section~\ref{sec:results}.
Section~\ref{sec:hodo} discusses the hodograph and its inadequacy
for common tangents.
Section~\ref{sec:plucker} compares our solution with a \plucker coordinate solution,
[We may give an algorithm for pole/polar computation, common tangents
where one of the curves degenerates to a point, in Section~\ref{sec:point}
and compare to classical pole/polar theory in Section~\ref{sec:pole},
although this would make a perfectly fine separate paper]
and Section~\ref{sec:conclude} wraps up with ideas for future work.

\begin{rmk}
Notice that a common tangent is a global property of the curves,
while a tangent is a local property,
which explains why the computation of common tangents requires more 
sophistication than the computation of tangents.
\end{rmk}

%%%%%%%%%%%%%%%%%%%%%%%%%%%%%%%%%%%%%%%%%%%%%%%%%%%%%%%%%%%%%%%%%%%%%%%%%%%%%%%

\section{Definitions}
\label{sec:defn}

\begin{defn2}
A {\bf common tangent} of the curves C and D is a line that is tangent to both C and D.
We do not distinguish whether the line is tangent at one or more points
of each curve.
A {\bf common tangent of a single curve} C is a line that is tangent to C at two or more
distinct points.
\end{defn2}

\begin{defn2}
{\bf Projective 2-space} $P^2$ is the space 
$\{(x_1,x_2,x_3) : x_i \in \Re, \mbox{ not all zero}\}$
under the equivalence relation $(x_1,x_2,x_3) = k(x_1,x_2,x_3),\ k \neq 0 \in \Re$.
The point $(x_1,x_2,x_3),\ w \neq 0$ of $P^2$ is equivalent to the point
$(\frac{x_1}{x_3}, \frac{x_2}{x_3})$ of $\Re^2$.
\end{defn2}

{\bf 
Review of the theory of rational Bezier curves and their relationship
to projective space.
A rational Bezier curve can be interpreted as a polynomial Bezier in projective
space.}

%%%%%%%%%%%%%%%%%%%%%%%%%%%%%%%%%%%%%%%%%%%%%%%%%%%%%%%%%%%%%%%%%%%%%%%%%%%%%%%

\clearpage

\section{A representation of tangent space}
\label{sec:dual}

We will represent the tangent space of a plane curve as another
plane curve, called the tangential curve.
Our first necessity is the ability to represent a line in 2-space as a point.
Since the tangent space of a plane curve is a smooth one-dimensional family 
of lines,
if each tangent line can be uniquely encoded as a point, 
a tangent space can be represented by a curve.

\begin{defn2}
\label{defn:dual}
The line $ax+by+c=0$ in 2-space is {\bf dual} to the point $(a,b,c) \in P^2$
in projective 2-space.
\end{defn2}
%
% could interpret a or b as the projective coordinate, instead, if desired
%
It is appropriate to consider the dual point in projective space,
since the dual $(a,b,c)$ should be equivalent to the dual $(ka,kb,kc)$,
as the line $ax + by + c = 0$ is equivalent to the line
$kax + kby + kc = 0$, $k \neq 0$.

\begin{rmk}
The duality of Definition~\ref{defn:dual}, 
and its use for the dualization of tangent spaces as defined
below, is mentioned in \cite[p. 54]{hartshorne}.
Similar dualities between a point and a hyperplane 
are often used in computational geometry
(e.g., \cite[p. 214]{orourke94}, \cite{Edelsbrunner if we can get it}),
usually for the treatment of finite arrangements of hyperplanes.

Why don't we use the classical representation of a line by \plucker coordinates?
Because we are working in 2-space.
Our representation only works with lines in 2-space,
but it is a bit simpler, more intuitive, and direct than \plucker coordinates 
in this context.\footnote{Notice that our representation only works in 2 dimensions, 
	since the line is a hyperplane in 2-space but not in 3-space.}
%	A line in 3-space requires two implicit equations for its representation,
% 	which moreover are not unique even in a projective sense,
%	so it cannot be represented using this simple version of duality.
We will later see, incidentally, that the computations of the \tang 
for the representation of Definition~\ref{defn:dual} 
and the \plucker representation are
basically equivalent (Section~\ref{sec:plucker}).
\end{rmk}

\begin{defn2}
Let $C(t)$ be a plane curve.
The {\bf tangential curve} of $C(t)$ is the curve $C^*(t) \subset P^2$,
where $C^*(t)$ is the dual of the tangent at $C(t)$ (Figure~\ref{fig:1dual}).
\end{defn2}
%
\begin{figure}
\vspace{2.5in}
\special{psfile=/usr/people/jj/modelTR/6-dual/img/1duaL.ps 
	 voffset=-300 hoffset=-50}
\caption{A curve (left) and its \tang (right)}
% tops 1dual.rgb -m 6.5 1.5 > 1dual.ps: no longer used
% single curve and its dual, with single tangent and its dual marked
\label{fig:1dual}
\end{figure}
%%%%%%%%%%%%%%%%%%%%%%%%%%%%%%
\ifVideo
  video: single curve and its dual (ob2.rawctr)
       tangent moving on curve with associated point moving on dual
\fi
%%%%%%%%%%%%%%%%%%%%%%%%%%%%%%
%
Figure~\ref{fig:commonTang} illustrates how the computation of 
common tangents in primal space becomes the well understood problem
of curve intersection in dual space.
%
\ifTalk
This is analogous to the appeal
of using the Fourier transform to shift to the frequency domain,
where the difficult operation of convolution 
becomes the simple operation of multiplication.
\fi
%
\begin{figure}
\vspace{2.5in}
\special{psfile=/usr/people/jj/modelTR/6-dual/img/commontang.ps 
	 voffset=-300 hoffset=-50}
\caption{Common tangency as intersection in dual space}
% xv commontang.rgb translated to .ps
% no longer used: tops commonTang.rgb -m 6.5 1.5 > commonTang.ps
% common tangents of two curves, intersections of the associated dual curves
\label{fig:commonTang}
\end{figure}
%
\Comment{
video: two curves in left window, their duals in right window:
       point moving on one of the dual curves with associated tangent
       moving on original curve, illustrating that as point nears intersection
       the tangent nears common tangency
}

\clearpage

\section{Computing the \tang}
\label{sec:Bezier}

We now develop a formula for the dual of a Bezier curve.
Since B-splines can be easily expressed as Bezier curves, 
this formula will capture an important practical case for dual curves.
We shall first express the dual curve as a Bezier curve in
projective space, and then as a rational Bezier curve.

\begin{theorem}
The dual of a plane Bezier curve of degree $n$
with control points $\{ b_i \}_{i=0}^n$ over the parameter interval
$[a,b]$ is a Bezier curve
of degree $2n-1$ in projective 2-space with control points 
\begin{equation}
\label{eq:dualcoord}
b^*_k = 
\frac{n}{\Delta \choice{2n-1}{k}} 
\left(
\begin{array}{l}
	\sum_{j=\mbox{\footnotesize{max}}(0,k-n)}^{\mbox{\footnotesize{min}}(n-1,k)} 
	\choice{n-1}{j} \choice{n}{k-j} (b_{j,2} - b_{j+1,2})\\
	\sum_{j=\mbox{\footnotesize{max}}(0,k-n)}^{\mbox{\footnotesize{min}}(n-1,k)} 
	\choice{n-1}{j} \choice{n}{k-j} (b_{j+1,1} - b_{j,1})\\
\sum_{\begin{array}{c} \mbox{\footnotesize{$0 \leq i \leq n-1$}} \\ 
			     \mbox{\footnotesize{$0 \leq j \leq n$}} \\ 
			     \mbox{\footnotesize{$i+j=k$}}
			     \end{array}}
\choice{n-1}{i} \choice{n}{j}
[ (b_{i+1,2} - b_{i,2}) b_{j,1} - (b_{i+1,1} - b_{i,1}) b_{j,2} ]
\end{array}
\right)
\end{equation}
$k=0,\ldots,2n-1$, $b^*_k \in P^2$, over the parameter interval $[a,b]$, where
$\Delta = b-a$.
\end{theorem}
\prf
Let $C(t) = (c_1(t), c_2(t))$ be a plane curve.
The tangent and normal vectors of $C(t)$ are $(c'_1(t),c'_2(t))$
and $(-c'_2(t), c'_1(t))$, respectively.
	% since dot product encodes orthogonality.
Since the implicit equation of the line through $P$ with normal $N$ is 
$(X-P) \cdot N = 0$,
the implicit equation of the tangent of $C(t)$ is:
\[	
	(x-c_1(t), y-c_2(t))\ \  \cdot \ \ (-c'_2(t), c'_1(t)) = 0
\]
or
\[
	-c_2'(t) x + c_1'(t) y + [c_2'(t) c_1(t) - c_1'(t)c_2(t)] = 0 
\]
and the dual curve of $C(t)$ is
\begin{equation}
\label{eq:dual1}
  (-c_2'(t),\ \  c_1'(t),\ \ c_2'(t) c_1(t) - c_1'(t)c_2(t))
\end{equation}
%
This is a polynomial curve in projective 2-space.
% CORRECT SO FAR, HAVING TESTED IN drawTang of BezierCurve.c++
% using drawImplicitLine

Now suppose $C(t)$ is a Bezier curve over the parameter interval $[a,b]$ with control points $\{ b_i \}_{i=0}^n$.
We want to express its dual curve also as a Bezier curve.
Consider the third coordinate of the dual curve:
\[
c_2'(t) c_1(t) - c_1'(t) c_2(t) 
=  \frac{1}{\Delta} \sum_{i=0}^{n-1} n \Delta b_{i,2} B_i^{n-1}(t) 
   		    \sum_{j=0}^n              b_{j,1} B_j^n(t) 
  - 
   \frac{1}{\Delta} \sum_{i=0}^{n-1} n \Delta b_{i,1} B_i^{n-1}(t) 
		    \sum_{j=0}^n	      b_{j,2} B_j^n(t) 
\]
where $\Delta = b-a$ is the length of the parameter interval.
We can use the product 
formula\footnote{$B_i^m(t) B_j^n(t) = \frac{ \choice{m}{i} \choice{n}{j} }{ \choice{m+n}{i+j} } B_{i+j}^{m+n}(t)$.}
for Bernstein polynomials \cite{farin97} to simplify this to
\[
  \frac{n}{\Delta} \sum_{i=0}^{n-1} \sum_{j=0}^n 
  \frac{\choice{n-1}{i} \choice{n}{j}}{ \choice{2n-1}{i+j} } B_{i+j}^{2n-1}(t)
  [ \Delta b_{i,2} b_{j,1} - \Delta b_{i,1} b_{j,2} ].
\]
Letting $k=i+j$, this becomes
\[
% \label{eq:dualw1}
\sum_{k=0}^{2n-1} B_k^{2n-1}(t) 
\sum_{\begin{array}{c} \mbox{\footnotesize{$0 \leq i \leq n-1$}} \\ 
			     \mbox{\footnotesize{$0 \leq j \leq n$}} \\ 
			     \mbox{\footnotesize{$i+j=k$}}
			     \end{array}}
\frac{n}{\Delta} \frac{ \choice{n-1}{i} \choice{n}{j} }{ \choice{2n-1}{k} }
[ \Delta b_{i,2} b_{j,1} - \Delta b_{i,1} b_{j,2} ]
\]
The third coordinate is now expressed as a 1-dimensional Bezier curve of 
degree $2n-1$ with control points $\{b^*_{k,3} \}_{k=0}^{2n-1}$ where:
\begin{equation}
\label{eq:dualw2}
b^*_{k,3} = \frac{n}{\Delta \choice{2n-1}{k}}
\sum_{\begin{array}{c} \mbox{\footnotesize{$0 \leq i \leq n-1$}} \\ 
			     \mbox{\footnotesize{$0 \leq j \leq n$}} \\ 
			     \mbox{\footnotesize{$i+j=k$}}
			     \end{array}}
\choice{n-1}{i} \choice{n}{j}
[ (b_{i+1,2} - b_{i,2}) b_{j,1} - (b_{i+1,1} - b_{i,1}) b_{j,2} ].
\end{equation}

Now consider the first two coordinates of the dual curve (\ref{eq:dual1}), 
\begin{equation}
\label{eq:dualx}
-c_2'(t) = \frac{1}{\Delta} \sum_{k=0}^{n-1} -n \Delta b_{k,2} B_k^{n-1}(t)
\end{equation}
and
\begin{equation}
\label{eq:dualy}
c_1'(t)  = \frac{1}{\Delta} \sum_{k=0}^{n-1}  n \Delta b_{k,1} B_k^{n-1}(t)
\end{equation}
These are 1-dimensional Bezier curves of degree $n-1$,
which we need to degree-elevate to degree $2n-1$
for compatibility with the third coordinate.
% Any polynomial curve can clearly be expressed as a polynomial
% curve of higher degree, simply by padding on zero terms.
% The translation of a Bezier curve from degree $n$ to $n+r>n$ (degree elevation)
% is more involved,
% since the defining control points must move.
% However, 
%
A Bezier curve of degree $n$ with control points $\{ d_k \}_{k=0}^n$ 
is degree elevated \cite{farin97} to a Bezier curve of degree $n+r$ with control points
$\{ d_k^{(r)} \}_{k=0}^{n+r}$ where:
\[
d_k^{(r)} = \sum_{j=\mbox{\footnotesize{max}}(0,k-r)}^{\mbox{\footnotesize{min}}(n,k)} 
		d_j \frac{\choice{n}{j} \choice{r}{k-j} }{ \choice{n+r}{k} }.
\]
Consequently, the 1-dimensional Bezier curve (\ref{eq:dualx}) 
of degree $n-1$ with control points $\frac{-n}{\Delta} \Delta b_{k,2}$
is degree elevated to a 1-dimensional Bezier curve of degree $2n-1$
with control points $\{b^*_{k,1} \}_{k=0}^{2n-1}$ where:
\begin{equation}
\label{eq:1stcoord}
b^*_{k,1} = 
\sum_{j=\mbox{\footnotesize{max}}(0,k-n)}^{\mbox{\footnotesize{min}}(n-1,k)} 
	\frac{-n}{\Delta} \Delta b_{j,2}
	\frac{ \choice{n-1}{j} \choice{n}{k-j} }{ \choice{2n-1}{k} }
\end{equation}
%
\Comment{
By expanding $\frac{ \choice{n-1}{j} \choice{n}{k-j} }{ \choice{2n-1}{k} }$
and regrouping, we can reexpress it as 
$\frac{ \choice{2n-1-k}{n-1-j} \choice{k}{j} }{ \choice{2n-1}{n} }$
and (\ref{eq:1stcoord}) becomes
\begin{equation}
b^*_{k,1} = 
- \frac{n}{\Delta \choice{2n-1}{n}} 
\sum_{j=\mbox{\footnotesize{max}}(0,k-n)}^{\mbox{\footnotesize{min}}(n-1,k)} 
	\choice{2n-1-k}{n-1-j} \choice{k}{j} (b_{j+1,2} - b_{j,2}).
\end{equation}
}
%
Similarly, the Bezier curve (\ref{eq:dualy})
of degree $n-1$ with control points $\frac{n}{\Delta} \Delta b_{k,1}$ 
can be degree elevated to a Bezier curve of degree $2n-1$ with control points 
$\{b^*_{k,2} \}_{k=0}^{2n-1}$ where:
\begin{equation}
\label{eq:2ndcoord}
b^*_{k,2} = 
\sum_{j=\mbox{\footnotesize{max}}(0,k-n)}^{\mbox{\footnotesize{min}}(n-1,k)} 
	\frac{n}{\Delta} \Delta b_{j,1}
	\frac{ \choice{n-1}{j} \choice{n}{k-j} }{ \choice{2n-1}{k} }
\end{equation}
%
Combining (\ref{eq:dualw2}), (\ref{eq:1stcoord}) and (\ref{eq:2ndcoord}),
we conclude that the dual of the Bezier curve $C(t)$
is a Bezier curve of degree $2n-1$ in projective space with control points 
$\{b^*_k\}_{k=0}^{2n-1}$ as in (\ref{eq:dualcoord}).
\QED

A Bezier curve in projective 2-space is equivalent to a 
rational Bezier curve in 2-space (Section~\ref{sec:defn}).
Since the intersection of two curves in 2-space is simpler than the intersection
of two curves in projective 2-space, 
we will translate the dual curve to a rational Bezier curve.
% In translating to a rational curve, the last projective coordinate of 
% $C^*_p(t)$ is associated with the weight of the rational Bezier curve.  

\begin{theorem}
\label{thm:rationaldual}
The dual of a Bezier curve of degree $n$ in 2-space
with control points $\{ b_i \}_{i=0}^n$ over $[a,b]$ 
is a rational Bezier curve of degree $2n-1$ over $[a,b]$
with weights $\{w_k\}_{k=0}^{2n-1}$ where: 
\begin{equation}
\label{eq:wt}
w_k = 
\frac{n}{\Delta \choice{2n-1}{k}}
\sum_{\begin{array}{c} \mbox{\footnotesize{$0 \leq i \leq n-1$}} \\ 
			     \mbox{\footnotesize{$0 \leq j \leq n$}} \\ 
			     \mbox{\footnotesize{$i+j=k$}}
			     \end{array}}
\choice{n-1}{i} \choice{n}{j}
[ (b_{i+1,2} - b_{i,2}) b_{j,1} - (b_{i+1,1} - b_{i,1}) b_{j,2} ].
\end{equation}
%
and control points $\{c_k\}_{k=0}^{2n-1}$ where:
\[
% \label{eq:coord}
	c_k = \frac{1}{w_k} 
	\frac{n}{\Delta \choice{2n-1}{k}} 
	\sum_{j=\mbox{\footnotesize{max}}(0,k-n)}^{\mbox{\footnotesize{min}}(n-1,k)}
	\choice{n-1}{j} \choice{n}{k-j} 
	\left( \begin{array}{c} 
	b_{j,2}   - b_{j+1,2} \\
	b_{j+1,1} - b_{j,1}
	\end{array} \right) 
\]
\end{theorem}

\clearpage

\section{Expunging negative weights}
\label{sec:negative}

Looking at (\ref{eq:wt}),
it is likely that some of the weights of the \tang will be negative.
But negative weights are undesirable, 
since curve segments with negative weights
lack the convex hull property that is so useful for divide and conquer
techniques of intersection using subdivision.
Goldman and DeRose \cite{goldman86} have shown how to intersect curves without
the convex hull property, using an expanded convex hull.
However, we choose a different solution
that directly gets rid of the negative weights.

It is actually the segments with both positive
and negative weights that cause problems.
A segment with purely negative weights can be easily
corrected by multiplying all the weights by $-1$
(since rational Bezier curves are invariant under 
transformations $w_i = kw_i$, $k \neq 0$ of the weights).
The weight function of a segment with both positive and negative weights
must contain a zero between consecutive weights of opposite sign, by continuity.
Since zero weights are associated with points at infinity,
a change in sign of the weight on a curve segment 
is associated with a point at infinity of the curve.\footnote{In the 
	tangential curve of Theorem~\ref{thm:rationaldual},
	zeroes of the weight function are associated with tangents $ax+by=0$
	through the origin.}
By subdividing at the zeroes of the weight function
$w(t) = \sum w_i B_i^n(t)$,
we can replace the undesirable segments
by segments with purely positive or purely negative weights.
We actually subdivide just before and after a zero,
since we don't want zero weights at the endpoints of a segment either
(which represent points at infinity), and since we cannot
compute the zeroes exactly anyway.
The segment that contains the zero weight in its interior (the zero-segment)
is then discarded.
This is valid since the zero-segment, if built correctly,
will only contain segments close to infinity, uninvolved with the
intersections. 
{\bf [How do we guarantee this formally? I don't think
we can identify when an intersection has been missed postpriori,
so I believe we will need to prove that no intersections lie beyond
a certain distance from the origin from first principles]}.

{\bf [does this generalize to the translation of all segments with negative weights?
does this generalize to all intersection of rational curves?
This could be a general technique for the intersection of 
two rational curves with negative weights without Goldman/deRose's convex hull
expansion.]}

Our algorithm for the removal of negative weights is as follows.
Consider a curve segment C(t) of degree $n$ with weights $w_i$,
some of which are positive and some negative.
The weight function of this curve is $w(t) = \sum w_i B_i^n(t)$.
%
\begin{itemize}
\item Find zeroes $t = t_{0,0},t_{0,1},\ldots,t_{0,k}$ of the weight function
	$w(t) = \sum w_i B_i^n(t)$, $t \in [a,b]$.
	We pose this as the intersection of the weight curve 
	$(t,f(t)) = \sum (a + \frac{i}{n} (b-a), w_i) B_i^n(t)$
	and the line $(t,0) = \sum (a + \frac{i}{n} (b-a), 0)B_i^n(t)$.
	This takes advantage of the Bezier nature of the weight function.
%
% Although this solution does not appear to relate to classical methods for the 
% solution of a univariate equation, it is actually very similar 
% to the classical bisection solution of a univariate equation
% because of the midpoint subdivision nature of Bezier curve intersection.
% NOT REALLY.
\item Subdivide C(t) at $t = t_{0,0} \pm \epsilon,\ldots,t_{0,k} \pm \epsilon$,
	on either side of the zero weights.
\item Discard the segments $C(t_{0,i}-\epsilon,t_{0,i}+\epsilon)$
	surrounding a zero.
\item The weights of the remaining segments are purely positive or
	purely negative.  Flip the sign of the purely negative segments.
\end{itemize}

\begin{example}
Example.
Give weights of a bad segment, zeroes, and resulting segments.
\end{example}

\begin{rmk}
Problems with zeroes of the weight function 
(positive and negative weights) are unavoidable.
(1) Any representation of a tangent space has three coordinates
[tangent space has provably three degrees of freedom?].
(2) We want the tangential curve to be a plane curve in 2 dimensions,
since space curves don't intersect in general.
Also, any dual space for tangents is inherently projective,
since $ax+by+c=0$ is equivalent to $kax+kby+kc=0$.
The only alternative to using projective 2-space (which allows
several points represent the same line) is to normalize each line beforehand
(to unit length) and then use affine 2-space.
This normalization is too expensive.
(3) So we must interpret one of the three coordinates as a projective coordinate.

Any curve with three coordinates that is expressed as a rational plane curve,
by interpreting one of the coordinates as a projective coordinate, will
have a problem with zero weights (points at infinity).
Whichever coordinate is chosen as the projective coordinate, 
it can have zeroes, in general.
For example, we will later see that there is a \plucker version of the 
tangential curve that shares the problems with zero weights
(Section~\ref{sec:plucker}).
\end{rmk}
 
\clearpage

\section{Curve intersection}
\label{sec:intersect}

Once our problem has been translated to dual space by computing \tangs, 
it is reduced to curve intersection, 
a problem that has received much attention and is quite well understood.
	% This is a major advantage of the tangential curve solution to common tangency.
The standard solution to intersection is a divide-and-conquer approach using
curve subdivision and the convex hull property of Bezier curves:
if the bounding boxes\footnote{A bounding box can be a true convex hull
	or, for efficiency, an axis-aligned bounding box defined by the
	extremal coordinates of the control points.
	It turns out that the axis-aligned box, although larger, yields a more
	efficient intersection algorithm,
	since the convex hull is difficult to compute.}
of the two curves intersect, subdivide each curve
and recursively intersect the four subsegments,
halting when the bounding box is sufficiently small or the curve sufficiently
straight to compute the intersection point directly.
Sederberg reports that this is the most stable of the most popular
intersection algorithms.
This basic algorithm can be made more sophisticated in many ways.
We recommend the approaches of 
Sederberg \cite{sederberg86,sederberg89,sederberg90},
Rockwood \cite{rockwood90} and Natarajan \cite{nat90}.
Self-intersections of a Bezier curve can be computed using
a method of Lasser \cite{lasser89}.
Self-intersection requires some subtlety: for example, two neighbouring
segments of a curve will necessarily have an 'intersection' at their
common boundary but this should not be counted as a true intersection.
However, the basic use of subdivision and the convex hull property is the same.

\Comment{
Elaboration on self-intersections: cannot have self-intersection
within a Bezier curve segment unless its control polygon has a self-intersection
(explain why using variation-diminishing property?)
so self-intersections are computed entirely the same as intersection
of two curves, where both curves are the same.
For example, the self-intersections of the Bezier spline with segments
$S_0,\ldots,S_k$ are the union of the intersections of $S_i$ and $S_j$ for $i \neq j$,
just as the intersections of the Bezier spline $C$ with segments
$C_0,\ldots,C_m$ with the Bezier spline $D$ with segments 
$D_0,\ldots,D_n$ are the union of the intersections of $C_i$ and $D_j$ for $i \neq j$.
}

\Comment{
We shall compute intersections of dual curves in 2-space, 
rather than projective 2-space.
Fortunately, the intersections in 2-space are equivalent to the
intersections in projective 2-space.
(This may be a well-known fact of projective space.)
Consider an intersection in 2-space: $(\frac{a_1}{c_1}, \frac{b_1}{c_1})
= (\frac{a_2}{c_2}, \frac{b_2}{c_2})$.
Then, in projective space, $(a_1,b_1,c_1) = \frac{c_1}{c_2} (a_2,b_2,c_2)$,
so the associated lines in projective space are also equivalent and 
form an intersection.
Conversely, if two 'points' (lines) in projective space are equivalent
(an intersection) then the associated points in projective space
are also equivalent: if $(a_1,b_1,c_1) = k(a_2,b_2,c_2)$ ($k \neq 0$),
then $(\frac{a_1}{c_1}, \frac{b_1}{c_1}) = (\frac{ka_2}{kc_2}, \frac{kb_2}{kc_2})
 = (\frac{a_2}{c_2}, \frac{b_2}{c_2})$.
}

\clearpage

\section{Computing common tangents}
\label{sec:alg}

An algorithm for computing the common tangents between
two plane polynomial Bezier curves $C$ and $D$ is as follows.
See Figures~\ref{fig:alg1}-\ref{fig:alg5}.
\begin{description}
\item[(1)]	Compute the \tangs $C^*$ and $D^*$ (Section~\ref{sec:Bezier}).
\item[(2)]	Intersect the \tangs (Section~\ref{sec:intersect}).
\item[(3)]	Map intersections in dual space to common tangents in primal space.
% \item[(4)]	Remove lines that are not common tangents (see explanation below).
\item[(4)]	Remove common tangents whose endpoints
		are not mutually visible,\footnote{Two points 
			are mutually visible (see each other) 
			if the line segment connecting them
			has no interior intersections with any obstacles.}
		if appropriate.
		Visible common tangents are the only common tangents of interest
		in applications such as motion planning and visibility.
\end{description}

The intersections are mapped to lines in an unorthodox fashion, 
avoiding the use of duality.
We choose to record each intersection as a parameter pair (s,t),
the parameter values of the intersection point with respect to each curve.
The two endpoints of the candidate tangent are then immediately available
($C(s)$ and $D(t)$), which simplifies the filtering of step (5).
	% the line $(s,t)$ is removed if it does not agree with the tangent $C'(s)$
	% or the tangent $D'(t)$.

Duality can be used to map the intersections back to primal space,
if one takes care to work in projective space.
If the intersection point is $C^*(t) = (x(t),y(t))$ and the weight function at
this parameter value is $w(t)$, the projective coordinates of the intersection
are $(w(t)x(t),w(t)y(t),w(t))$.
Of course, the line associated with the point $(a,b,c)$ in dual space
is $ax+by+c=0$.
This line alone is not enough however if the filtering step (5) is necessary.
It is expensive to compute these endpoints if only the line is known,
which explains why we use the parameter pair representation.

During the filtering of visible tangents in step (5),
care must be taken to count only interior intersections
with the common tangent segment.
As the common tangent is by definition close to the obstacle as it
approaches it, and intersection algorithms are necessarily approximate,
it is easy to mistakenly find an apparent intersection of the 
common tangent with an obstacle near the point of tangency.

The common tangents of a single curve are computed in exactly the same way.
The only difference is that $C$ and $D$ are the same curve,
and the intersection of step (2) is self-intersection.

\Comment{
\begin{figure}[bh]
\vspace{5in}
\special{psfile=/usr/people/jj/modelTR/3-spline/img/---.ps}
\caption{Common tangents of a single curve}
% tops .rgb -m 6.5 2 > 1dual.ps
\label{fig:selfalg}
\end{figure}
}

\twocolumn

\begin{figure}
\vspace{2in}
\special{psfile=/usr/people/jj/modelTR/6-dual/img/alg1.ps hoffset=-50}
\caption{Two curves (left) and their duals (right)}
% tops alg1.rgb -m 4 1.5 > alg1.ps
% Step 1:  dual curves without reflection 
\label{fig:alg1}
\end{figure}

\begin{figure}
\vspace{2in}
\special{psfile=/usr/people/jj/modelTR/6-dual/img/alg2.ps hoffset=-50}
\caption{Intersections of dual curves}
% tops alg2.rgb -m 4 1.5 > alg2.ps
% step 2: dual curve intersections, including reflection
\label{fig:alg2}
\end{figure}

\begin{figure}
\vspace{2in}
\special{psfile=/usr/people/jj/modelTR/6-dual/img/alg3.ps hoffset=-50}
\caption{Associated tangents in original space}
% tops alg3.rgb -m 4 1.5 > alg3.ps
% step 3: associated tangent superset
\label{fig:alg3}
\end{figure}

\begin{figure}
\vspace{2in}
\special{psfile=/usr/people/jj/modelTR/6-dual/img/alg4.ps hoffset=40}
\caption{Common tangents}
% tops alg4.rgb -m 4 1.5 > alg4.ps
% step 4: filtered common tangents 
\label{fig:alg4}
\end{figure}

\begin{figure}
\vspace{2in}
\special{psfile=/usr/people/jj/modelTR/6-dual/img/alg5.ps hoffset=40}
\caption{Visible common tangents}
% tops alg5.rgb -m 4 1.5 > alg5.ps
% step 5: visible common tangents
\label{fig:alg5}
\end{figure}

\clearpage
\onecolumn

\section{Results}
\label{sec:results}



\section{\Tang vs. Hodograph}
\label{sec:hodo}

Let $C(t)$ be a plane polynomial Bezier curve of degree $n$
with control points $\{b_i\}_{i=0}^n$:
\[
C(t) = \sum_{i=0}^n b_i B_i^n(t)
\hspace{1in} t \in [0,1]
\]
where $B_i^n(t) = \choice{n}{i} (1-t)^{n-i} t^i$ is the $i^{th}$ Bernstein
polynomial of degree $n$.
The {\bf hodograph} of $C(t)$ is another Bezier curve, representing
the first derivative or tangent of $C(t)$:
\[
C'(t) = (c_1'(t), c_2'(t)) = \sum_{i=0}^{n-1} n\Delta b_i B_i^{n-1}(t)
\]
where $\Delta b_i = b_{i+1} - b_i$ \cite{farin97}.

The standard representation of the tangent space of a curve
is its hodograph, the locus traced by its tangent vectors moved to the origin
(Figure~\ref{FIGURE OF HODOGRAPH}).
However, this representation is not suited to the construction
of common tangents.
A different representation of the tangent space is needed, 
which we provide.

Why the hodograph is not appropriate:
The hodograph is a clean, concise representation of the tangent space,
since it represents a curve's tangent space as another curve
(and, better yet, in the case of a Bezier curve of degree $n$, the hodograph 
is another Bezier curve of degree $n-1$ easily derived from the original curve).
However, the hodograph represents a tangent directly as a vector (from the origin to a point
of the hodograph) and to make this into a tangent line, we need to
add this vector to the associated point on the original curve.
This indirect vector representation does not allow a common tangent to 
be computed in the natural way as an intersection of tangent spaces,
since two tangent vectors can be equivalent without the associated tangent
lines being equivalent and, more importantly, all points along a line 
through the origin represent the same tangent line (i.e, the hodograph
representation of tangent lines is many-to-one) so a tangent {\em line}
is actually not represented by a single point, but by a line,
so an intersection of hodographs will not find all common tangents,
since only one representative vector of each tangent line is represented
in the hodograph.

For this construction, we prefer a direct representation of the tangent lines,
since a common tangent is a line located in 2-space
(that happens to be tangent at two points),
rather than a representation of a vector indirectly connected to an associated point.
In the hodograph, a tangent is a vector, not a line, and its 

A common tangent is naturally an 'intersection' of two tangents from 
different parts of the curve, but the hodograph does not adapt well
to this interpretation:

\section{\Tang vs. Plucker}
\label{sec:plucker}

ALMOST EQUIVALENT, EXCEPT HIGHER DIMENSIONAL DUAL SPACE MAKES IT SLIGHTLY
	MORE DIFFICULT AND TRANSLATION OF 
	ARBITRARY LINE, NOT JUST LINES IN TWO DIMENSIONS, MAKES IT SLIGHT OVERKILL
	SINCE THE BASIC COMPUTATIONS ARE THE SAME, IT IS TRIVIAL
	TO TRANSLATE TO THE PLUCKER SOLUTION IF DESIRED.
	FOR EXAMPLE IF A PLUCKER FRAMEWORK IS ALREADY EXTANT IN ONE'S SYSTEM,
	FOR COMPATIBILITY.

another alternative could have been Plucker space representation,
	a more general representation of a line as a point
	(working in higher dimensions, 
	still reduces to intersection of two plane curves
	since the image of a line field lies in a plane;
	basically the same solution yielding the same problems encountered
	in our solution;
	our solution works since we are working in 2 dimensions
	where the implicit equation of a line is one equation, not two)


The obvious representation of a line by a point is by \plucker coordinates.
[want to confront thoughts of Plucker immediately]
The full power of \plucker coordinates, and the representation of a line
as a point in projective 5-space, is not necessary however: a simpler
representation of a line by a point works better.

\subsection{A \plucker interpretation}

We need a line representation that maps all lines well.
The {\bf Plucker} representation of lines is such a representation.
We can map the tangent space of a plane curve to Plucker space,
using the Plucker coordinates of the tangent lines.
Since the tangent space of a plane curve is a line field, 
its image in Plucker space is a plane curve.\footnote{In general,
	the image of a one-dimensional family of lines is a curve
	on the Klein quadric in Plucker space.
	Only in special cases like line fields and line bundles 
	is this a plane curve.}
Thus, just as with dual curves,
the common tangents of two plane curves can
be found through the intersection of two plane curves in image space,
but now these plane curves are curves in Plucker space
which enjoy added stability.

{\bf After further thought:}
Switching to Plucker space is not a panacea, and enjoys exactly the
same problems as our tangential curves, but in a higher-dimensional context: 
either lines through the origin or horizontal lines or vertical lines will still
be a source of instability.
One observes that any curve in projective space will have trouble with some 
regions that go to infinity: those regions that send the projective coordinate 
to zero.
For example, the Plucker curve $(V,P \times V)$ is
$(C'(t),C(t) \times C'(t)) = (c'_1,c'_2,0,0,0,c_1c'_2 - c'_1c_2)$
({\bf as it turns out almost identical to the dual curve}).
{\bf One of these coordinates must be considered the homogeneous coordinate},
and once it goes to 0 we will have a point at infinity.
Vertical lines, horizontal lines, or lines through the origin will
correspond to points at infinity depending on whether the first, second,
or sixth coordinate is chosen to be the projective coordinate, respectively.

\subsection{Plucker method}

Once we begin thinking of intersecting bundles in Plucker space, 
another, more direct approach comes to mind.
The lines in a certain plane are called a line field.
The tangents of our original plane curves form two line fields.
The lines of a line field map to a plane in Plucker space, just like
the lines of a line bundle did (Pottmann, p. 6).
And once again, the desired intersections of the line fields (the common tangents!)
are themselves lines.
Therefore, we can again intersect in Plucker space, but now using
the line fields associated with the tangents rather than the line bundles
associated with the duals of the tangents:

\begin{itemize}
\item	Intersect tangent line fields of the original curves in Plucker space.
\item   Translate intersections back to common tangents in Euclidean space.
\end{itemize}

\subsection{Computing tangent line fields in Plucker space}

Given a plane curve, we wish to express its tangent space
(a line field) as a plane Bezier curve in Plucker space.

The normalized Plucker coordinates of the line $P+tV$ ($\|V\|=1$) are
$(V,P \times V)$.  Notice that this does not depend on the choice of
point $P$.
The homogeneous Plucker coordinates of the line through $A$ and $B$,
where $A = (a_0,a) = (a_0,a_1,a_2,a_3)$, 
      $B = (b_0,b) = (b_0,b_1,b_2,b_3)$ are points in projective 3-space,
are $(a_0 b - b_0 a, a \times b)$.
We prefer the latter homogenous Plucker coordinates, since they do 
not require normalization.

Consider the tangent space of a plane curve.
Two points on the tangent line at C(t) are C(t) and C(t) + C'(t).\footnote{We
	assume that $C'(t) \neq 0$.  That is, the curve
	is regular.  A zero derivative is associated with a
	singular cusp of the curve where the tangent is undefined.}
Thus, the homogeneous Plucker tangent space of a plane curve C(t) is
$(C'(t), C(t) \times C'(t))$.
Here, $C(t)$ is interpreted as a curve in 3-space with $z=0$:
$C(t) = (c_1(t),c_2(t),0)$.
Simplifying, the Plucker tangent space is 
$(c'_1, c'_2, 0, 0, 0, c_1c'_2 - c'_1c_2)$ in projective 5-space.
See Dual curve paper derivation of Bezier form of $c_1c'_2 - c'_1c_2$
If $C(t) = \sum b_i B_i^n(t)$ is a Bezier curve, ---

\subsection{Rational curves with negative weights are bundles}

{\bf This is a very convoluted approach that doesn't yield any advantages.}

Dealing with bundles or line fields, negative weights don't seem to matter.
(But they do if you insist on working with two-dimensional plane curves
for easy intersection, in which case some coordinate must be treated
as the projective one.)
	
The lines of the projective version of a rational curve 
all pass through the origin,
so the rational 'ruled surface' is actually a bundle (Pottmann, Intro to Line 
Geometry, p. 6).
The intersection of two bundles with the same common point
is quite different from the intersection of two ruled surfaces,
and is amenable to a simpler solution.
In particular, the intersection of two bundles with the same common point
is a collection of lines, not a collection of points.
This makes it feasible to compute the intersection in Plucker space.
For general ruled surfaces and general bundles,
although they have images in Plucker space,
the intersection of these images is in general empty, since the 
ruled surfaces or bundles intersect in points rather than lines.
Since our bundle intersection does consist of lines, it is 
expressible in Plucker space.
Intersection in Plucker space is simple, 
since the lines of a bundle map to points of a plane in Plucker space
(Pottmann, p. 6).
This makes the intersection of two bundles with the same common point in Euclidean space 
equivalent to the intersection of two curves in a plane in Plucker space.
Curve intersection in the plane is well understood, even if this plane
resides in projective 5-space (affine 6-space).
We can ignore the presence of the higher dimensions by rotating the
two-dimensional plane of the curves to the $\{z_3=0,z_4=0,z_5=0\}$
plane and ignoring the 3rd, 4th, and 5th coordinates.
After computing the intersection using classical 2-dimensional algorithms
(Sederberg), the intersection points should be rotated back before
translating their Plucker coordinates to lines.

In short, common tangency can be computed as follows.
\begin{itemize}
\item	Compute dual curves.
\item	Intersect associated bundles in Plucker space.
\item 	Translate intersections back to lines in dual space.
\item 	Translate intersection lines in dual space back to points of common
	tangency on the original curves.
\end{itemize}

Common tangency in the original space now becomes ruled surface intersection
in dual 3-space.
{\bf This may motivate a return to our study of ruled surface intersection.}



\section{Computing common tangents of a point and curve}
\label{sec:point}

The problem of finding the tangents 
from a curve that meet a given point 
is classically solved using pole/polar theory,
but dual curves offer a better solution.
Tangents of a curve that meet the point $P$ become points on the dual curve
that lie on the dual line of $P$.
Once again, a tangency relationship is translated to an intersection problem,
this time an even simpler curve/line intersection.


If one of the curves degenerates to a point, dual curves can still
be used to compute common tangents.
That is, dual curves can be used to compute the tangents from a curve
that intersect a given point.
Consider a curve $C$ and a point $P$.
Moving to the dual space, the point $P$ becomes a line $P^*$
and the tangents of $C$ become points.
Consequently, tangents from $C$ through $P$ are associated
with points of $C^*$ on the line $P^*$.
The algorithm for the computation of the tangents from $C$ through $P$ is
as follows:
\begin{itemize}
\item	Compute the the dual $P^*$ and the dual curve $C^*$.
\item	Compute the intersections of the line $P^*$ and curve $C^*$.
\item	Compute the duals of these intersections,
	which are the desired tangents through $P$.
\end{itemize}

Note: the dual line must be represented as a Bezier curve in order to 
apply Bezier curve intersection algorithms, so it actually needs to be 
represented as a line segment, not an infinite line.
We need to guarantee that the chosen line segment contains all of the 
intersections with the dual curve.
There are two approaches: we could compute the intersections and verify
that there are the appropriate number of intersections, the class(?) of the 
curve.  However, some points do not generate the maximum number of 
tangents (think of a point inside a circle).
A better approach is to compute a bounding box for the dual
curves of the obstacles (ignoring the 'zero-weight' segments which would make
the box infinitely large) and build the line segment as large as this box.

Note: the point (a,b,1) in projective space and the line ax+by+cw=1 are duals.
     In this model, the dual of the origin (0,0,1) is the line w=0 at infinity!
     Thus, a pole at the origin will not be treated well in pole/polar 
     computations.
     We can avoid the origin, or translate the obstacles and pole together
     away from the origin before computing.


\Comment{
\begin{figure}[bh]
\vspace{3in}
\special{psfile=/usr/people/jj/modelTR/3-spline/img/--.ps}
\caption{Tangents through a fixed point}
% tops pointTang.rgb -m 6.5 2 > pointTang.ps
% a curve and a point, with tangents from curve that intersect point;
%	dual curve and dual line, with intersections
\label{fig:pointTang}
\end{figure}
}
%
\Comment{
video:  curve and point in left window, dual curve and dual line in right window;
	point moving on dual curve with associated tangent moving on curve,
	illustrating that as the point approaches an intersection with the line
	in dual space, the tangent meets the point in affine space
}

\section{Pole/polar theory}
\label{sec:pole}

Polar theory is the classical method for finding the tangents of a curve $C$
through a point $P$.
The {\bf first polar} of $P$ with respect to $C$ is a curve $D$ such that
the intersections of $C$ and $D$ are the points of $C$ whose tangents meet $P$.
If $C$ is the plane algebraic curve defined by the polynomial $f(x_1,x_2,x_3)=0$
of degree $n$ and $P = (p_1,p_2,p_3)$, both $C$ and $P$ in projective 2-space,
then the first polar of $P$ with respect to $C$ 
is an algebraic curve of degree $n-1$ defined by
\[
	p_1 \frac{\partial f}{\partial x_1} +
	p_2 \frac{\partial f}{\partial x_2} +
	p_3 \frac{\partial f}{\partial x_3} = 0
\]
or $P \cdot \nabla f = 0$ \cite{semple85}.	% p. 10-11
For example, in the simple case of the circle $C$, $x_1^2 + x_2^2 - x_3^2 = 0$,
the first polar of $(5,0,1)$ with respect to $C$ is the line
$5(2x_1) - 2x_3 = 0$ or $x_1 = \frac{1}{3}x_3$ (Figure~\ref{fig:polar}).

\begin{figure}
\vspace{2.5in}
%\special{psfile=/usr/people/jj/modelTR/3-spline/img/polar.ps}
\caption{The first polar of a point with respect to a circle}
% file: ---.showcase
% tops polarCircle.rgb -m 6.5 1.5 > polarCircle.ps
% unit circle and (5,0) with tangents to circle and polar line x=1/3
\label{fig:polar}
\end{figure}

An algorithm for the computation of the tangents from $C$ through $P$
using polars would be as follows:
\begin{itemize}
\item	If necessary, implicitize $C$ to determine its implicit representation $f(X)=0$.
\item	Compute the first polar of $P$ with respect to $C$, a curve $D$.
\item	Compute the intersections of the algebraic curves $C$ and $D$.
\item	Compute the tangents at these intersections, which are the desired
	tangents through $P$.
\end{itemize}

This polar solution is inferior to the proposed dual curve solution of 
Section~\ref{sec:dual}.
The dominating costs of the polar solution
are the implicitization and the algebraic curve intersection,
while the dominating cost of the dual curve solution is
the parametric line/curve intersection.
The dual curve solution is more efficient, since 
intersection of parametric Bezier curves is simpler than 
the intersection of algebraic curves,
and line/curve intersection is simpler than general curve/curve intersection.
The dual curve solution also has the benefit of compatibility with
the dual curve solution for common tangents.

\section{Conclusions and future work}
\label{sec:conclude}

The \tang may be used for further analysis of a curve's tangent space.

common tangents of *rational* Bezier plane curve

\clearpage

\bibliographystyle{plain}
\begin{thebibliography}{Lozano-Perez 83}

\bibitem[Farin 97]{farin97}
Farin, G. (1997)
Curves and Surfaces for CAGD: A Practical Guide (4th edition).
Academic Press (New York).

\bibitem[Goldman 86]{goldman86}
Goldman, R. and T. DeRose (1986)
Recursive subdivision without the convex hull property.
Computer Aided Geometric Design 3, 247--265.

\bibitem[Hartshorne 77]{hartshorne}
Hartshorne, R. (1977)
Algebraic Geometry.
Springer-Verlag (New York).

\bibitem[Hoschek 83]{hoschek83}
Hoschek, J. (1983)
Dual Bezier curves and surfaces.
In {\em Surfaces in Computer Aided Geometric Design},
R. Barnhill and W. Boehm, eds.,
North Holland (---), 147--156.

\bibitem[Johnstone 99]{jjjimbo99}
Johnstone, J. and J. Williams (1999)
A rational quaternion spline of arbitrary continuity.
Manuscript.

\bibitem[Lasser 89]{lasser89}
Lasser, D. (1989)
Calculating the self-intersections of Bezier curves.
Computers in Industry 12, 259--268.

\bibitem[Natarajan 90]{nat90}
Natarajan, B. (1990)
On computing the intersection of B-splines.
6th Annual Symposium on Computational Geometry, 157--167.

\bibitem[O'Rourke 94]{orourke94}
O'Rourke, J. (1994)
Computational Geometry in C.
Cambridge University Press (New York).

\bibitem[Rockwood 90]{rockwood90}
Rockwood, A. (1990)
Accurate display of tensor product isosurfaces.
Visualization '90, 353--360.
% comparison of curve intersection algorithms

\bibitem[Saito 93]{saito93}
Saito, T. and G. Wang and T. Sederberg (1993)
Hodographs and normals of rational curves and surfaces.
Manuscript.

\bibitem[Sederberg 86]{sederberg86}
Sederberg, T. and S. Parry (1986)
Comparison of three curve intersection algorithms.
Computer Aided Design 18, 58--63.

\bibitem[Sederberg 89]{sederberg89}
Sederberg, T., S. White, and A. Zundel (1989)
Fat arcs: a bounding region with cubic convergence.
Computer Aided Geometric Design 6, 205--218.

\bibitem[Sederberg 90]{sederberg90}
Sederberg, T. and T. Nishita (1990)
Curve intersection using Bezier clipping.
Computer Aided Design 22, 538--549.

\bibitem[Semple 85]{semple85}
Semple, J. and L. Roth (1985)
Introduction to Algebraic Geometry.
Oxford University Press (Oxford).

\end{thebibliography}

\section{Common tangents on surfaces}

We have considered common tangents of curves in 2-space.
It is also useful to compute common tangents of curves on surfaces.
This is a more challenging computation 
because tangents must be geodesics [define geodesic] and 
geodesics on a surface are typically not straight lines, and are difficult
to characterize and compute.
To compute a tangent, we must be able to compute the geodesic
from a given point in a given direction (the direction of the derivative
of the curve at the point).
[Discussion of computation of a geodesic: solution of differential equations]

Although geodesics are difficult to compute on most surfaces,
there are certain important surfaces with simpler geodesics.
The design of curves on a sphere is a very important problem,
yet the geodesics of a sphere are all great circles of the sphere.
We now consider common tangents on \Sn{2}\ and \Sn{3}, the unit spheres 
at the origin in 3-space and 4-space.

Curves on \Sn{3}, called quaternion splines, are crucial for motion planning
in computer animation and robotics.
It is natural to introduce orientation obstacles (orientations disallowed for
kinematic reasons or for their undesirability, such as an upside-down 
orientation during the motion of a cup or camera), just as we have position
obstacles.
A classical tool for obstacle avoidance in the plane is the visibility graph,
which involves the computation of common tangents of the planar obstacles.
A visibility graph for avoidance of orientation obstacles on \Sn{3}\
requires the computation of common tangents of curves on \Sn{3}.

\section{Common tangents on \Sn{2}}

A geodesic, and thus a tangent, on \Sn{2}\ is a great circle.
In order to dualize the tangent space of a curve on \Sn{2},
we need the implicit equation of a tangent.
Although the implicit equation of a circle in 3-space involves 2 equations,
a sphere and a plane, our sphere will always be \Sn{2}\ and so 
a great circle is characterized by a plane, a single linear equation.
Another serendipity is that the plane will always pass through the origin,
so its constant term disappears and our plane is defined by 3 rather than
4 coefficients.
This allows the tangent space of a curve on \Sn{2}\ to be captured just 
like the tangent space of a plane curve, by a plane rational Bezier curve 
in dual space.

\begin{defn2}
On the surface \Sn{2}, 
the great circle $\{x^2+y^2+z^2-1=0,\ ax+by+cz=0\}$ and
the point $(a,b,c) \in P^2$ are {\bf duals}.
\end{defn2}

Notice that the dual $(a,b,c)$ is the normal of the plane $ax+by+cz=0$.

\begin{defn2}
The {\bf dual of a curve $C(t)$ on \Sn{2}} is the curve $C^*(t) \subset P^2$
where the point $C^*(t)$ is the dual of the tangent circle at $C(t)$.
\end{defn2}

Let $C(t)$ be a curve on \Sn{2}.
Since the plane of the tangent circle at $C(t)$ contains the vectors $C(t)$
and $C'(t)$, it has the normal $C(t) \times C'(t)$.
Thus, the dual of the tangent circle at $C(t)$ is $C(t) \times C'(t)$.
Notice that this dual is the binormal vector at $C(t)$.\footnote{Thus, in
	computing a dual curve, we are computing a binormal curve.
	This is related to Saito/Sederberg's normal hodograph and 
	Catmull's normal surface.
	A major difference is that we compute the dual curve in projective
	space and project to a rational curve in 2-space (to allow intersection), 
	so the binormal vector cannot be read off directly.}
	
We now develop a formula for the dual of a Bezier curve on \Sn{2}.
Let $C(t)$ be a rational Bezier curve on \Sn{2}:
\[
	C(t) = \frac{\sum_{k=0}^n w_k b_k B_k^n(t)}{\sum_{k=0}^n w_k B_k^n(t)}
\]
We must deal with rational Bezier curves, since curves on \Sn{2}\ (and \Sn{3})
are not polynomial, in general \cite{jjjimbo99}.
{\bf Try to generalize dual curves in 2-space using the same method.}
The dual curve is $C(t) \times C'(t)$.
Since the dual curve is originally expressed in projective space,
we can compute $\alpha(t) C(t) \times C'(t)$, since vector length
is immaterial in projective space.
In particular, in computing the dual curve, 
we are only interested in the direction of $C(t)$ and $C'(t)$, not
their lengths.
This allows a simplification: we can replace the rational curves
$C(t)$ and $C'(t)$ by polynomial curves of the same direction.
A polynomial expression for the direction of $C(t)$ is
\[
	\sum_{k=0}^n w_k b_k B_k^n(t)
\]
We use a result of Saito, Wang and Sederberg \cite{saito93}
for a polynomial expression for the direction of $C'(t)$.

\begin{lemma}
Let $C(t) = \frac{\sum_{k=0}^n w_k b_k B_k^n(t)}{\sum_{k=0}^n w_k B_k^n(t)}$ 
be a rational Bezier curve in 3-space.
A vector in the direction $C'(t)$ is 
\[
	\sum_{k=0}^{2n-2} H_k B_k^{2n-2}(t)
\]
where
\[
	H_k = \frac{1}{\choice{2n-2}{k}}
		\sum_{i=\mbox{max}(0,k-n+1)}^{\lfloor k/2 \rfloor}
		(k-2i+1)\choice{n}{i} \choice{n}{k-i+1}
		(w_i b_{k-i+1,0} - w_{k-i+1} b_{i,0},
		 w_i b_{k-i+1,1} - w_{k-i+1} b_{i,1},
		 w_i b_{k-i+1,2} - w_{k-i+1} b_{i,2}),\ \ 
	b_i = (b_{i,0}, b_{i,1}, b_{i,2})
\]
\end{lemma}

algorithm for common tangent computation (is filtering necessary?):

\section{Common tangents on \Sn{3}}

Discuss applications to orientation obstacles.

\section{Common tangents on developable surfaces}

A geodesic on a developable surface is also tractable:
it is a rolled-up line.


\section{Future work}

Can we define a surface version of the dual curve?
This might lead to a mechanism for finding the common tangent planes 
of two surfaces.
The extension of the polar to surfaces works for tangents from a surface
to a point: the intersections of the first polar of the surface with the
surface define the points of tangency (p. 211 of Snyder and Sisam as backup
of same basic theory of Semple and Roth).

Use the surface version to compute {\bf silhouettes}.
(See Hart/Dutre/Greenberg 1999 Siggraph open problems for motivation.)

\ \ \ \ \ 3D version.
Ruled surface (or bipolar curve) that defines common tangents between two surfaces.

Extension to a surface:
tangents and common tangents *on* a surface.

Geodesics on this ruled surface.

other tangential relationships:
are tangents of the dual curve = points of the original curve? (test empirically)
proof: common tangents of dual curves must be intersections in original space, 
	by symmetry, thus tangents of dual must be points of original curve
a way to compute the envelope of a family of lines:
	translate family of lines to dual space, where it becomes a curve;
	compute the tangents of this dual curve (hodograph), 
	whose duals become the envelope curve in the original space
	not interesting in 2-space, but analogue in 3-space would be interesting:
	computing envelope of planes

(The following is a reasonable topic in the context of my work on 
swept surfaces and envelopes, but unrelated to tangential computation
and curves on surfaces.)
Curved extension of Minkowski sum (see Bajaj and Kim, Generation of Configuration
Space Obstacles: The Case of Moving Algebraic Curves
and Compliant Motion Planning with Geometric Models).
That is, for a non-point robot, the robot must first be shrunk to a point
and the obstacles expanded by their Minkowski sum with the robot
in order to apply the visibility graph algorithm.
Typically the robot and obstacles have been polygonal.
(A circular robot has also been dealt with by Yap.)
We have shown how to deal with curved obstacles.
Now we want to deal with a curved robot.

\section{Dual surfaces}

Relationship of dual surface with Catmull's normal surface and Gauss maps.

Dualism allows the tangent space of a hypersurface to be reinterpreted
as another hypersurface.

A dual relationship can also be established in 3-space between planes and
points, leading analogously to a useful tool in the analysis of tangential relationships
between surfaces in 3-space.	
In projective 3-space, the plane $p_1x + p_2y + p_3z + p_4w = 0$ 
is dual to the point $(p_1,p_2,p_3,p_4)$.
This dualism allows a two-parameter family of planes to be interpreted as a
surface.
In particular, the dual of the tangent space of a surface in 3-space is 
another surface, called the {\bf dual surface}.
Common tangency again provides an example of the use of the dual 
surface for the analysis of tangential relationships.
The common tangent planes of two surfaces are encoded by the intersection
curve of the associated dual surfaces.

Dual surfaces are the 3-dimensional analog to dual curves.
A dual relationship can be established between points and planes in 3-space:
the point $(p_1,p_2,p_3,p_4)$ of projective 3-space
is dual to the plane $p_1x_1 + p_2x_2 + p_3x_3 + p_4x_4 = 0$.
Given a parametric surface $S(u,v)$, its dual surface is the surface $S^*(u,v)$
where the point $S^*(u,v)$ is the dual to the tangent plane at $S(u,v)$.
The intersection of two dual surfaces represents the common tangent planes
of the associated surfaces.
Consequently, the common tangent planes between two surfaces $C$ and $D$ can be
computed as follows:
\begin{itemize}
\item	Compute the dual surfaces $C^*$ and $D^*$.
\item 	Compute the intersection curve of $C^*$ and $D^*$.
	Let this curve be $\alpha(t)$ where $\alpha(t) = C^*(u_t,v_t)$.
\item	$\alpha^*(t) = C(u_t,v_t)$ is the bipolar curve on the surface $C$,
	with respect to $D$.
\end{itemize}
(Define bipolar curve.)

\section{Dual curves of space curves}

We have shown how to compute the dual of a plane curve and a surface.
What about a space curve?
Unfortunately, duals only exist for objects of codimension 1.
(Define codimension.)
The tangent of a space curve is a line in 3-space, which can no longer
be expressed as a single linear equation since it is no longer a hyperplane.
Consequently, the dual of a space curve tangent, and thus of a space curve,
is undefined.

\section{}

A generalization of the definition of dualism:
A hyperplane in $n$-space can be associated with a point in projective 
$n$-space, since the implicit equation of the hyperplane has $n+1$ coefficients 
and the point has $n+1$ coordinates.
It is appropriate for the point to be in projective space,
since any nonzero multiple of the hyperplane's equation represents the same plane.
%
\begin{defn2}
A {\bf hypersurface} in $n$-space is an $(n-1)$-manifold (a surface of codimension 1).
For example, curves in 2-space and surfaces in 3-space are hypersurfaces.
A {\bf hyperplane} is a linear hypersurface.
For example, lines in 2-space and planes in 3-space are hyperplanes.
\end{defn2}
%
\begin{defn2}
The hyperplane $a_1x_1+\cdots+a_nx_n+a_{n+1}=0$ % in $n$-space
and the point $(a_1,\ldots,a_{n+1}) \in P^n$ are {\bf duals}.
\end{defn2}
%
\begin{defn2}
The {\bf dual of a hypersurface} $C(t_1,\ldots,t_{n-1})$ in $n$-space is the
hypersurface \linebreak $C^*(t_1,\ldots,t_{n-1}) \subset P^n$ where the point 
$C^*(t_1,\ldots,t_{n-1})$ is the dual 
of the tangent hyperplane at $C(t_1,\ldots,t_{n-1})$ \cite{hartshorne}. % p. 54
See Figure~\ref{fig:1dual}.
\end{defn2}


\end{document}

%%%%%%%%%%%%%%%%%%%%%%%%

SCRAPS:

The tangent space of a curve is a smooth one-dimensional family of lines.
A line can be represented by a point in \plucker space or a point in dual space,
and a continuous one-dimensional line family becomes a curve 
in the \plucker or dual space (henceforth, \plucker or dual curve).
We are interested in the representation of curve tangent spaces as
\plucker or dual curves.
We are also interested in the computation of common tangents between
plane curves through the intersection of \plucker or dual curves.

%%%%%%%%%%%%%%%%%%%%%%%%

[To get around the instability of points at infinity,
one may use two interpretations of the curve in projective space:
the first with $x_i$ as the projective coordinate and the second
with $x_j$ ($i \neq j$) as the projective coordinate.
In this scheme, a one-dimensional family of lines would be represented
by two curves in image space (e.g., dual space or Plucker space),
which are theoretically redundant but, in practice, map different parts
of the family to infinity and thus together allow robust treatment of all parts
of the family, as long as wrong solutions arising due to instabilities
near infinity can be detected.
This detection of wrong solutions is possible with common tangents, 
since it is simple to determine
if a candidate tangent is actually a common tangent.

This leads to the following algorithm:
\begin{enumerate}
\item Intersect dual curves using $x_3$ as projective coordinate.
\item Intersect dual curves using $x_2$ as projective coordinate.
\item Use all intersections as candidates.  Reject solutions that
	do not yield common tangents.
\end{enumerate}
]

[Luckily, we have a choice of which coordinate to choose as the projective one
and not all of these choices can have a zero at the same point.
Thus, for any given point, we can choose the projective coordinate to be
a nonzero one.
Or in practice, we can compute all three curves, using all choices of projective
coordinate, and perform the computations with respect to all these curves.]

