\documentclass[12pt]{article}
\usepackage[pdftex]{graphicx}
% \usepackage{epsfig}
\usepackage{latex8}
\usepackage{times}
\input{header}

\newif\ifTalk
\Talkfalse
\newif\ifJournal
\Journalfalse
\newif\ifFuture		% issues that are useful for future papers
\Futurefalse

\newcommand{\plucker}{Pl\"{u}cker\ }
\newcommand{\tang}{tangential curve\ }
\newcommand{\tangs}{tangential curves\ }
\newcommand{\Tang}{Tangential curve\ }
\newcommand{\atang}{tangential a-curve\ }
\newcommand{\btang}{tangential b-curve\ }
\newcommand{\ctang}{tangential c-curve\ }
\newcommand{\atangs}{tangential a-curves\ }
\newcommand{\btangs}{tangential b-curves\ }
\newcommand{\ctangs}{tangential c-curves\ }
\newcommand{\acurve}{a-curve\ }
\newcommand{\bcurve}{b-curve\ }
\newcommand{\ccurve}{c-curve\ }
\SingleSpace
\setlength{\headsep}{.5in}
	% \setlength{\oddsidemargin}{0pt}
	% \setlength{\topmargin}{-.2in}	% should be 0pt for 1in
\setlength{\textheight}{8.5in}
	% \setlength{\textwidth}{6.5in}
	% \setlength{\columnsep}{5mm}	% width of gutter between columns
% \markright{The tangential curve: \today \hfill}
% \pagestyle{myheadings}

% -----------------------------------------------------------------------------

\title{The tangential curve}
\author{John K. Johnstone\\
	Geometric Modeling Lab\\
	Computer and Information Sciences\\
	The University of Alabama at Birmingham\\
	University Station, Birmingham, AL 35294}

\begin{document}
\maketitle

% -----------------------------------------------------------------------------

\begin{abstract}
Many questions about lines are best referred to dual space, 
and this includes analyses of the tangent space of a curve.
We study the tangential curve, a dual image of the tangent space of
a plane curve, and the tangential curve system, a robust model of the entire tangent
space of a plane curve split across two dual spaces.
The modeling of a tangent space by a tangential curve system
is a useful tool for analyzing the interrelationships of a scene of curves, 
such as in visibility analysis or motion planning.
The tangential curve system is a pair of clipped curves in dual space, which is
often simpler to work with than a line family.

We develop our specific choice of duality and representation for dual space,
guided by behaviour at infinity and the need to dualize smooth
line families rather than finite line arrangements.
We give an algebraic specification of the tangential curve for 
rational curves, and explain the clipping process.
As an illustration of the theory, 
we develop the computational framework of Bezier tangential curves.
We exhibit the power of tangential curves by solving the bitangency problem,
which reduces to intersection in dual space.
We analyze the advantages of the tangential curve system over other 
related structures, including the hodograph, the dual Bezier curve, and a
\plucker coordinate solution.
Finally, we study the mapping of a tangential curve back to
primal space, through the reapplication of duality.
\end{abstract}

\noindent Keywords: tangent space of a plane curve, geometric duality,
		    tangential curve.

\clearpage

% \tableofcontents

% \clearpage

% \listoffigures

% \clearpage

% -----------------------------------------------------------------------------

\section{Introduction}

Our interest in this paper is in the development of 
a different representation for the
tangent space of a plane curve, the tangential curve system.
This is a dual representation, using the geometric duality between lines in 2-space and points.
This duality allows a line family in primal space, such as the tangent space of a curve,
to be interpreted as a point family in dual space.
That is, we can map the tangent space to a curve in dual space, which
we call the tangential curve.
This paper strives to understand how best to map the tangent space of a plane curve 
to a curve in dual space, and explores what it looks like once we get there.

Why do we want a different representation for the tangent space?
The present representations 
	% (the curve's derivative or hodograph)
are sufficient for the local analysis of a curve,
where we only need to compute individual tangents on individual curves,
such as for the analysis of continuity between segments or motion along a curve.
% piecewise interpolation or even normal computation on a surface for shading.
However, for the global analysis of a scene of curves,
where we need to study the interrelationships between curves and understand the
tangent space of a curve globally,
such as in visibility analysis, motion planning, and lighting,
	% For example, a bitangent of two curves is a global property of the tangent spaces
	% of the curves, not something that can be computed from local information.
	% For this type of application, 
a different representation for the tangent space of a curve is needed,
one that captures the whole tangent space in a compact form 
and encodes tangent lines rather than tangent vectors.
The tangential curve system is such a representation.

To map a tangent space to a curve, we must first be able to map a line to a point.
In Section~\ref{sec:duality}, we discuss the geometric duality between hyperplanes and points.
Our particular choice of duality and representation of dual space is explained through
a contrast with other possible choices.
In order to robustly encode all lines, we impose a dichotomy of the space of lines into two families,
and use two dualities to represent all lines.
The need for two dualities is motivated by a duality's behaviour at infinity,
which we analyze.
Our Cartesian representation of dual space, which preserves the rationality 
and 2-dimensionality of the tangential curve, 
is contrasted with other representations in dealing with points at infinity.

In Section~\ref{sec:tangcurve}, we lift this dualization of lines to the dualization of tangent spaces.
Using our two dualities, we develop a representation
for the tangent space spread across two dual spaces, as two clipped curves.
We provide an algebraic specification of the tangential curves of a rational curve.
The tangential curve can be computed for any curve format.
Section~\ref{sec:bez} considers a special case, the tangential curve of a Bezier curve,
and expresses it as a rational Bezier curve.
We show that the problematic negative weights of this rational Bezier curve
are removed by the clipping defined by a tangential curve system.

In Section~\ref{sec:bitang},
as an example of the utility of our new representation, 
we show how easily the bitangents of two curves can be computed
using the tangential curve system.
In Section~\ref{sec:comparison}, the tangential curve is compared to 
the related structures of the hodograph and the dual Bezier curve.
We also establish the basic equivalence
of our solution with a \plucker coordinate solution.
Section~\ref{sec:back} shows how a tangential curve maps back to primal space,
and Section~\ref{sec:conclude} provides some conclusions.

This paper builds upon the introduction of the tangential curve in \cite{jj01a}.
We look at the tangential curve in a more general context,
allowing it to stand alone as a theoretical development,
and expand its theory considerably.

Since we work in projective space throughout the paper,
we dispense with a definition of projective space
before we enter the body of the paper.

\begin{defn2}
\label{defn:proj}
{\bf Projective 2-space ${\bf P^2}$} is the space 
$\{(x_1,x_2,x_3) : x_i \in \Re, \mbox{ not all zero}\}$
under the equivalence relation $(x_1,x_2,x_3) = k(x_1,x_2,x_3),\ k \neq 0 \in \Re$.
The point $(x_1,x_2,x_3)$ in projective 2-space
is equivalent to the point $(\frac{x_1}{x_3},\frac{x_2}{x_3})$
in Cartesian 2-space.
	% Points of $P^2$ with $x_3=0$ are associated with points at infinity.
The point $(x_1,x_2,0)$ is the {\bf point at infinity} in the direction $(x_1,x_2)$.
The 3 coordinates in projective 2-space are called {\bf homogeneous coordinates}.
We distinguish the third homogeneous coordinate by calling it the 
{\bf projective coordinate}.
\end{defn2}

% -----------------------------------------------------------------------------

\section{Line-point duality}
\label{sec:duality}

Dualities are valuable in many disciplines.
For example, in physics it is impossible to fully understand light without
an appreciation of its dual nature as particle and wave.
% photoelectric effect: acts as particle
% 2-slit experiment:    acts as wave
In this paper, 
we work with the classical geometric duality between hyperplanes and points.
This duality is often used to replace hyperplanes (which are viewed as difficult)
by points (which are viewed as simple).
We use this same powerful idea, but in the context of a smooth family,
to replace smooth one-dimensional line families by smooth one-dimensional
point families (or curves).
This can clarify the analysis of tangent spaces,
since the tangent space of a plane curve is a continuous one-dimensional
family of lines (Figure~\ref{fig:linefamily}).

\begin{figure}[h]
\begin{center}
\includegraphics*[scale=.36]{img/jjdufamily.jpg}
\end{center}
% \centerline{\epsfig{figure=img/jjdufamily.ps,height=1.485in,width=1.590in}}
% 36% reduction
\caption{A tangent space is a family of lines}
\label{fig:linefamily}
% dual ob2.rawctr, line field/one-obstacle only
% jjdufamily.gif
\end{figure}

A geometric duality establishes a relationship between the hyperplanes 
of an $n$-space and the points of this $n$-space.
This relationship is mutual:\footnote{We shall use the notation $P^*$ for the dual of P.} 
if $P^* = Q$, then $Q^* = P$ (or equivalently, $(P^*)^* = P$).
Another fundamental property of a duality,
sometimes called the {\bf principle of duality} \cite{pedoe70}, % Pedoe, p. 277
is its preservation of incidence between linear manifolds:
\[
P \subseteq Q \Rightarrow Q^* \subseteq P^*
\]
where $P$ and $Q$ are linear manifolds (points, lines, planes, $\ldots$, hyperplanes).  % varieties
Notice that containment is reversed.
A corollary of the principle of duality is the dualization of join to intersection, and vice versa.
For example, if the join of $n$ points $P_1, \ldots, P_n$ defines the hyperplane $Q$,
then the hyperplanes $P_1^*, \ldots, P_n^*$ will intersect in the point $Q^*$.

Consider the duality  in 2-space between lines and points,
the duality of primary interest in this paper.
There are many choices for the specific point that is dual to a line.
We choose the elegant duality that maps
the line $ax+by+c=0$ to the point $(a,b,c)$ in projective 2-space \cite{hartshorne77}.
The dual point must live in projective space,
since the line $ax + by + c = 0$ is equivalent to the line
$kax + kby + kc = 0$, $k \neq 0$.

\begin{defn2}
\label{defn:dual}
The line $ax+by+c=0$ in 2-space is 
\begin{itemize}
\item	{\bf c-dual} to the point $(a,b,c) \in P^2$.
\item   {\bf a-dual} to the point $(c,b,a) \in P^2$.
\item   {\bf b-dual} to the point $(a,c,b) \in P^2$.
\end{itemize}
\end{defn2}

For example, under c-duality, $(ax+by+c=0)^* = (a,b,c)$
and $(a,b,c)^* = (ax+by+c=0)$.
We illustrate the dual nature of the maps of Definition~\ref{defn:dual} by showing
that they satisfy the principle of duality.

\begin{lemma}
The maps of Definition~\ref{defn:dual} 
satisfy the principle of duality.
\end{lemma}
\prf
Consider a-duality.\footnote{I thank Alan Sprague for pointing out 
	the asymmetry of our original choice for the a-duality in the 
	conference paper \cite{jj01a}, 	where the point dual to $ax+by+c=0$ was $(b,c,a)$.
	This asymmetry prevents it from satisfying the principle of duality.
	Interestingly, this 'duality' still works
	well for finding bitangents, its main purpose in that paper.}
Since the only linear manifolds in 2-space are points and lines,
it suffices to show that 
\[	P \in L \Rightarrow L^* \in P^*
\]
where $P$ is a point in 2-space, $L$ is a line in 2-space, 
and $P^*$ and $L^*$ are their a-duals.
Let $P=(A,B,C)$ and $L$ be the line $ax+by+c=0$.
$P^*$ is the line $Cx+By+A=0$ and $L^*$ is the point $(c,b,a)$.
If $P \in L$, then $aA+bB+cC=0$,
so $L^*$ also lies on $P^*$.
The proofs for b-duality and c-duality are analogous.
\QED

The dualities of Definition~\ref{defn:dual} 
differ only in their assignment of the third homogeneous coordinate,
which affects their behaviour at infinity.\footnote{The
	name of the duality (a-, b-, or c-) is based on this assignment.}
Lines through the origin ($ax+by=0$) are mapped to infinity by the c-duality;
horizontal lines ($by+c=0$) are mapped to infinity by the a-duality;
and vertical lines ($ax+c=0$) are mapped to infinity by the b-duality.

Behaviour at infinity would not be a concern if we modeled
the entire projective space $P^2$, including points at infinity.
A point in projective 2-space can be modeled as a line through the
origin in 3-space \cite{foley96}, which can be uniquely identified with its
point of intersection with a hemisphere of the unit sphere $S^2$ in 3-space,
so all of projective 2-space $P^2$ can be modeled by $S^2$ \cite{zorin00}.
However, we do not want to use this representation of $P^2$,
since it would force us to project 
the dual images of tangent spaces (which are curves)
to $S^2$, a nonrational operation that would destroy the rationality
of these curves (Section~\ref{sec:tangcurve}).
Instead, we shall use a Cartesian representation of dual space, 
representing $(a,b,c) \in P^2$ by the Cartesian point $(\frac{a}{c},\frac{b}{c})$
and points at infinity are indeed lost.

Since this issue is central to our later representation
of the tangent space, we restate it in a different way.
Our Cartesian representation of dual space $P^2$ has two advantages:
(1) it preserves the rationality of a tangent space under dualization and
(2) it sends the tangent space to a plane curve in 2-space.
An inherent penalty of this
stuffing of the 3 dimensions of the family of lines $ax+by+c=0$
into the 2 dimensions of the Cartesian representation
is the mapping of some lines to infinity.
One of the three coordinates of a tangent line must be considered the 
projective coordinate, and
when this coordinate goes to 0, we have a point at infinity.\footnote{We
	will later develop a \plucker version of the 
	tangential curve that has the same inherent problems with points at infinity 
	(Section~\ref{sec:plucker}).}
Therefore, our development must deal with points at infinity.

We now turn to a graceful solution of this loss of information at infinity.
In theory, lines through the origin map to infinity under c-duality.
In practice, none of the lines 'nearly' through the origin 
are robustly encoded by the c-duality.
Similarly, near-horizontal and near-vertical lines 
are badly represented by the a- and b-dualities, respectively.
However, each of the latter dualities represents half of the lines well:
the a-duality those lines far from horizontal and
the b-duality those lines far from vertical.

\begin{defn2}
\label{defn:steep}
A line $ax+by+c=0$ is {\bf steep} if $|a| \geq |b|$
and {\bf shallow} if $|a| < |b|$.
% A shallow line is more horizontal than vertical.
\end{defn2}

\begin{lemma}
\label{lem:steep}
The a-duality robustly dualizes steep lines to the strip $y \in [-1,1]$.
The b-duality robustly dualizes shallow lines to the strip $x \in (-1,1)$.
\end{lemma}
\prf
The line $ax+by+c=0$ a-dualizes to the Cartesian point $(\frac{c}{a}, \frac{b}{a})$.
If $|a| \geq |b|$, this is a point of the strip $y \in [-1,1]$.
The line $ax+by+c=0$ b-dualizes to the Cartesian point $(\frac{a}{b}, \frac{c}{b})$.
If $|a| < |b|$, this is a point of the strip $x \in (-1,1)$.
\QED

\vspace{-.2in}

\begin{corollary}
The space of all lines in 2-space can be robustly represented by
the strip $y \in [-1,1]$ in a-dual space, representing steep lines,
and the strip $x \in (-1,1)$ in b-dual space, representing shallow lines.
\end{corollary}

% The dichotomy of lines in 2-space into steep and shallow lines is a disjoint cover
% of all lines.
% This suggests representing steep lines in a-dual space and shallow lines in b-dual space.
We shall apply this principle when modeling the tangent space in the 
next section.
Notice that it is not useful to use the c-dual space in combination
with one of the other dual spaces for the representation of lines, 
since a line through the origin can also be horizontal or vertical.

Before we leave this section, we observe that there are many other line-point dualities 
that we could have chosen from.
Three other choices, popular in computational geometry,
relate the line $ax+by+1=0$ with the point $(a,b)$ \cite{chazelle85}, % also O'Rourke, p. 214, called the polar duality
	% it relates points and lines via the unit circle, analogously to inversion in a circle
the line $y = -ax + b$ with the point $(a,b)$ \cite{welzl85}, % Lemma 2, p. 170
and the line $y=2ax-b$ with the point $(a,b)$ \cite{orourke94}.	% O'Rourke, p. 214
	% another polar duality: $(r,\theta)$ \cite{ghali} % http://www.cs.ualberta.ca/~ghali/thesis/node4.html
These dualities have special geometric properties, 
which are leveraged by the algorithms that use them.\footnote{For example, 
	the second duality allows the simple analysis of lines by increasing slope
	and the third duality relates points and tangents of the parabola $y=x^2$.}
However, all of these dualities require some normalization of the
implicit equation $ax+by+c=0$ of a line before its dualization.
This normalization is not a problem when working with a finite set of lines,
as in their application to finite line arrangements in computational geometry
\cite{chazelle85,edels87,orourke94,welzl85},
but would cause trouble when working with infinite line families
such as in Lemma~\ref{lem:rattangacurve} below
(for reasons analogous to the normalization problems discussed above in the context
of a representation of dual space by $S^2$).
Fortunately, we do not need the special properties of the above dualities, 
so we can use our simpler duality that requires no normalization of the line.

Richard Patterson \cite{patterson01} and Wenping Wang \cite{wenping01}
have also explored the use of duality in geometric modeling,
for finding the cusps and inflection points of a Bezier curve and its implicitization,
and for collision detection, respectively.

We shall now turn to the representation of the tangent space of a plane curve,
using the ideas developed in this section.

% -----------------------------------------------------------------------------

\section{The tangential curve system}
\label{sec:tangcurve}

In the previous section, we explored the representation of a line
by a point in dual space, with the eventual representation of a smooth line family
always in mind.
In this section, we move on to the representation of a smooth line family,
in particular the tangent space of a curve, by a curve or set of curves in dual space.
Because of the issues surrounding the robust treatment of points at infinity,
the tangent space will be represented by two clipped curves.

\begin{defn2}
\label{defn:tangentialcurve}
Let $C(t)$ be a plane curve.
The {\bf tangential curve} of $C(t)$ is the curve $C^*(t) \subset P^2$,
where $C^*(t)$ is the dual of the tangent at $C(t)$.
The {\bf tangential a-curve} $C^*_a$ (resp., {\bf tangential b-curve} $C^*_b$)
is the tangential curve where the a-duality (resp., b-duality) of 
Definition~\ref{defn:dual} is used.
\end{defn2}

Notice that a tangential curve is a standard curve with a primal
structure.
Its only idiosyncrasy is that it lives in dual space.
However, the tangential curve is not yet a robust representation
of a curve's tangent space.
For this, we need a pair of tangential curves.

\begin{defn2}
\label{defn:tangentialsystem}
The {\bf tangential curve system} of $C(t)$ is a combination of
the tangential a-curve clipped to the strip $y \in [-1,1]$ and 
the tangential b-curve clipped to the strip $x \in (-1,1)$
(Figure~\ref{fig:tangcurve1}).
	% will use the same term for tangential surfaces of a surface?
\end{defn2}

\begin{lemma}
The tangential curve system of $C(t)$ represents the entire tangent space of $C$.
\end{lemma}
\prf
By Lemma~\ref{lem:steep}, 
steep tangents of $C$ are represented by the clipped tangential a-curve and
shallow tangents of $C$ by the clipped tangential b-curve.
Since every tangent is either steep or shallow, the two clipped curves
cooperate to represent the entire tangent space of $C$.
\QED

% -----------------------------------------------------------------------------

Figures~\ref{fig:tangcurve1}-\ref{fig:tangcurve3} illustrate various 
tangential curve systems.
Notice how the tangential curve system yields new insight into the complexity of a curve's
tangent space (compare Figures~\ref{fig:tangcurve1} and \ref{fig:tangcurve3}).

	% tangentialCurve -p -d 5 data/circleOne.pts
\begin{figure}
\begin{center}
\includegraphics*[scale=.65]{img/jjduta1.jpg}
\end{center}
% \centerline{\epsfig{figure=img/jjduta1.ps,height=3.494in,width=6.509in}}
% 65% reduction
\caption{A circle and its tangential curve system (left: $C$; top right: clipped $C^*_a$; bottom right: clipped $C^*_b$)}
\label{fig:tangcurve1}
\end{figure}

	% tangentialCurve -p data/ob1b.pts
\begin{figure}
\begin{center}
\includegraphics*[scale=.65]{img/jjduta2.jpg}
\end{center}
% \centerline{\epsfig{figure=img/jjduta2.ps,height=3.494in,width=6.509in}}
% 65% reduction
\caption{A second tangential curve system, with one tangent shown}
\label{fig:tangcurve2}
\end{figure}

	% tangentialCurve -p data/ob3a.pts
\begin{figure}
\begin{center}
\includegraphics*[scale=.65]{img/jjduta3.jpg}
\end{center}
% \centerline{\epsfig{figure=img/jjduta3.ps,height=3.494in,width=6.509in}}
% 65% reduction
\caption{A third tangential curve system, with one tangent shown}
\label{fig:tangcurve3}
\end{figure}

% -----------------------------------------------------------------------------

The tangential curves of a tangential curve system must be clipped,
either to the strip $y \in [-1,1]$ or the strip $x \in (-1,1)$.
There are two ways to clip a tangential curve, in primal or dual space.
In primal space, we can find the parameter values of the diagonal 
(45- and 135-degree) tangents and clip the tangential curve at these parameter values.
The parameter values on $C$ of these diagonal tangents can be found by intersecting
the tangent hodograph of $C$ with the diagonal lines $x = \pm y$.
In dual space, we can find the intersections of the lines $y=\pm 1$ (or $x=\pm 1$)
with the tangential curve and clip at these points.
	% This has the danger that an intersection with a segment going to 
	% infinity is difficult when using the rational Bezier tangential curve.
Both methods involve two line intersections.
We recommend clipping in primal space when dealing with Bezier tangential curves,
because their negative weights complicate clipping in dual space
(see the end of Section~\ref{sec:bez}).

Although we clip the tangential curves to prevent the coverage of the a-curve
(steep tangents) from overlapping the coverage of the b-curve (shallow tangents),
this clipping has another serendipitous effect: it leads to smaller curves
and a more efficient representation.
This is desirable, especially when these curves are intersected
as in the computation of bitangents of Section~\ref{sec:bitang}.
The shorter a curve is, the faster is intersection with that curve
when using a subdivision style of intersection.

% ----------------------------------------------------------------------------

We end this section with a more precise description of the tangential a-curve
and b-curve.

\begin{lemma}
\label{lem:rattangacurve}
Let $C(t) = (\frac{c_0(t)}{c_2(t)}, \frac{c_1(t)}{c_2(t)})$ be a rational curve.
The tangential a-curve of $C(t)$ is the rational curve
\begin{equation}
\label{eq:rattangacurve}
	(\frac{c_0 c'_1 - c'_0 c_1}{c_1 c'_2 - c'_1 c_2},
	 \frac{c'_0 c_2 - c_0 c'_2}{c_1 c'_2 - c'_1 c_2}).
\end{equation}
The tangential b-curve is
\begin{equation}
\label{eq:rattangbcurve}
	(\frac{c_1 c'_2 - c'_1 c_2}{c'_0 c_2 - c_0 c'_2},
	 \frac{c_0 c'_1 - c'_0 c_1}{c'_0 c_2 - c_0 c'_2}).
\end{equation}

\end{lemma}
\prf
The tangent vector of $C(t)$ is 
$C'(t) = \frac{1}{c_2^2} (c'_0 c_2 - c_0 c'_2, c'_1 c_2 - c_1 c'_2)$,
so its normal vector is 
\[
	(\frac{c_1 c'_2 - c'_1 c_2}{c_2^2}, 
	 \frac{c'_0 c_2 - c_0 c'_2}{c_2^2}).
\]
Since the implicit equation of the line through $P$ with normal $N$ is 
$(X-P) \cdot N = 0$,
the implicit equation of the tangent line of $C(t)$ is:
\[
	(x - \frac{c_0}{c_2}, y - \frac{c_1}{c_2}) \cdot
	(\frac{c_1 c'_2 - c'_1 c_2}{c_2^2},
	 \frac{c'_0 c_2 - c_0 c'_2}{c_2^2}) = 0
\]
or, multiplying by $c_2^2$, which is nonzero except at points at infinity,\footnote{So
	this formula will not be valid at points at infinity,
	but these will be clipped away anyway.}
\[
	(c_1 c'_2 - c'_1 c_2) x +
	(c'_0 c_2 - c_0 c'_2) y -
	\frac{c_0c_1c'_2 - c_0c'_1c_2 + c'_0c_1c_2 - c_0c_1c'_2}{c_2} = 0.
\]
This shows that the tangential a-curve of $C(t)$ is 
\[
	(c_0c'_1 - c'_0c_1, c'_0 c_2 - c_0 c'_2, c_1 c'_2 - c'_1 c_2)
\]
and the tangential b-curve is 
\[
	(c_1 c'_2 - c'_1 c_2, c_0c'_1 - c'_0c_1, c'_0 c_2 - c_0 c'_2).
\]
These are polynomial curves in projective 2-space or, equivalently, the rational
curves (\ref{eq:rattangacurve}) and (\ref{eq:rattangbcurve}) in Cartesian 2-space.
\QED

\Comment{
\begin{corollary}
\label{cor:tangacurve}
Let $C(t) = (c_0(t), c_1(t))$ be a polynomial curve.
The tangential a-curve of $C(t)$ is the rational curve
\[
	(\frac{c_0'c_1 - c_1' c_0}{c_1'}, - \frac{c_0'}{c_1'})
\]
The tangential b-curve is
\[
	(- \frac{c_1'}{c_0'}, \frac{c_1' c_0 - c_0'c_1}{c_0'})
\]
\end{corollary}
	% \prf
	% The tangent and normal vectors of $C(t)$ are $(c'_0(t),c'_1(t))$
	% and $(-c'_1(t), c'_0(t))$, respectively.
	% Since the implicit equation of the line through $P$ with normal $N$ is 
	% $(X-P) \cdot N = 0$,
	% the implicit equation of the tangent of $C(t)$ is:
	% \[	
	% 	(x-c_0(t), y-c_1(t))\ \  \cdot \ \ (-c'_1(t), c'_0(t)) = 0
	% \]
	% or
	% \[
	% 	-c_1'(t) x + c_0'(t) y + [c_1'(t) c_0(t) - c_0'(t)c_1(t)] = 0.
	% \]
	% This shows that the \atang of $C(t)$ is
	% \begin{equation}
	% \label{eq:dual1}
	%   (c_1'(t) c_0(t) - c_0'(t)c_1(t),\ \ c_0'(t),\ \ -c_1'(t))
	% \end{equation}
	% %
	% This is a polynomial curve in projective 2-space
	% or, equivalently, the rational curve (\ref{eq:tangacurve}) in Cartesian 2-space.
	% The b-curve is analogous.
		% CORRECT, having tested in drawTang of BezierCurve.c++ 
		% using drawImplicitLine
	% \QED
}

% ----------------------------------------------------------------------------

\section{Bezier tangential curves}
\label{sec:bez}

We now work out the details of the tangential curve system for a Bezier curve.
We have found this special case of a tangential curve system particularly useful,
largely because it allows the tangential curve to be integrated immediately
into existing modelers, including our research modeler.

\begin{theorem}
\label{thm:Bezier}
Let $C(t)$ be a plane Bezier curve of degree $n$
with control points $\{ b_i = (b_{i,0}, b_{i,1}) \}_{i=0}^n$ 
over the parameter interval $[t_0,t_1]$.
The \atang $C_a^*(t)$ is a rational Bezier curve
of degree $2n-1$ over $[t_0,t_1]$ with weights $\{w_k\}_{k=0}^{2n-1}$ where: 
\begin{equation}
\label{eq:wt}
w_k = 	\alpha 
\displaystyle{\sum_{i=\mbox{\tiny{max}}(0,k-n)}^{\mbox{\tiny{min}}(n-1,k)}}
	\tinychoice{n-1}{i} \tinychoice{n}{k-i} (b_{i,1} - b_{i+1,1})
\end{equation}
%
and control points $\{c_k\}_{k=0}^{2n-1}$ where:
\begin{displaymath}
\tiny{
c_k = \frac{\alpha}{w_k} 
\left(
\begin{array}{l}
	\displaystyle{\sum_{\begin{array}{c} \mbox{\tiny{$0 \leq i \leq n-1$}} \\ 
		       \mbox{\tiny{$0 \leq j \leq n$}} \\ 
		       \mbox{\tiny{$i+j=k$}} \end{array}}}
	\tinychoice{n-1}{i} \tinychoice{n}{j} \beta_{i,j} \\
%	
	\displaystyle{\sum_{i=\mbox{\tiny{max}}(0,k-n)}^{\mbox{\tiny{min}}(n-1,k)}}
	\tinychoice{n-1}{i} \tinychoice{n}{k-i} (b_{i+1,0} - b_{i,0})
\end{array}
\right)
}
\end{displaymath}
where $\alpha = \tiny{\frac{n}{(t_1 - t_0) \tinychoice{2n-1}{k}}}$
and $\beta_{i,j} = (b_{i+1,1} - b_{i,1}) b_{j,0} - (b_{i+1,0} - b_{i,0}) b_{j,1}$.
\end{theorem}
%
% -------------------------------------------------------------------
%
\prf
We will translate the \atang (in projective space) to a Bezier curve representation,
interpret this Bezier curve in projective space
as a rational Bezier curve in Cartesian space,
and finally extract control points and weights.
Applying Lemma~\ref{lem:rattangacurve} to a polynomial curve,
the tangential a-curve of $C(t) = (c_0(t), c_1(t))$ is, in projective space,
\begin{equation}
\label{eq:dual2}
	(c_1'c_0 - c_0' c_1, \ \ c_0', \ \ -c_1')
\end{equation}
We want to translate to a Bezier curve representation.
Since $C$ is a Bezier curve,
\[
(c_0(t), c_1(t)) = (\sum_{i=0}^n b_{i,0} B_i^n(t), \sum_{i=0}^n b_{i,1} B_i^n(t)).
\]
and
\[
(c_0'(t), c_1'(t)) = (\frac{1}{t_1-t_0} \sum_{i=0}^{n-1}  n \Delta b_{i,0} B_i^{n-1}(t),\ \ 
		      \frac{1}{t_1-t_0} \sum_{i=0}^{n-1}  n \Delta b_{i,1} B_i^{n-1}(t)
\]
where $\Delta b_{i,j} = b_{i+1,j} - b_{i,j}$ and
$B_i^n(t) = \tinychoice{n}{i} (1-t)^{n-i} t^i$ is the $i^{th}$ Bernstein
polynomial of degree $n$.
Consider the first coordinate of the \atang (\ref{eq:dual2}), in Bezier form:
\[
\scriptstyle{\frac{n}{t_1-t_0}}
\displaystyle{\sum_{i=0}^{n-1}} \scriptstyle{\Delta b_{i,1} B_i^{n-1}(t)}
\displaystyle{\sum_{j=0}^n}     \scriptstyle{       b_{j,0} B_j^n(t)}
\ \  - \ \ 
\scriptstyle{\frac{n}{t_1-t_0}}
\displaystyle{\sum_{i=0}^{n-1}} \scriptstyle{\Delta b_{i,0} B_i^{n-1}(t)}
\displaystyle{\sum_{j=0}^n}	\scriptstyle{       b_{j,1} B_j^n(t)}
\]
We can use the product formula\footnote{$B_i^m(t) B_j^n(t) = \frac{ \tinychoice{m}{i} \tinychoice{n}{j} }{ \tinychoice{m+n}{i+j} } B_{i+j}^{m+n}(t)$.}
for Bernstein polynomials \cite{farin97} to simplify this to
\[
\scriptstyle{\frac{n}{t_1-t_0}}
\displaystyle{\sum_{i=0}^{n-1} \sum_{j=0}^n }
\frac{\tinychoice{n-1}{i} \tinychoice{n}{j}}{ \tinychoice{2n-1}{i+j} } 
\scriptstyle{ B_{i+j}^{2n-1}(t) [ \Delta b_{i,1} b_{j,0} - \Delta b_{i,0} b_{j,1} ]}.
\]
Letting $k=i+j$, this becomes
\[
\sum_{k=0}^{2n-1} B_k^{2n-1}(t) 
\sum_{\begin{array}{c} \mbox{\footnotesize{$0 \leq i \leq n-1$}} \\ 
			     \mbox{\footnotesize{$0 \leq j \leq n$}} \\ 
			     \mbox{\footnotesize{$i+j=k$}}
			     \end{array}}
\frac{n}{t_1-t_0} \frac{ \tinychoice{n-1}{i} \tinychoice{n}{j} }{ \tinychoice{2n-1}{k} }
\beta_{i,j}
\]
The first coordinate of the \atang is now expressed as a 1-dimensional Bezier curve of 
degree $2n-1$ with control points $\{b^*_{k,0} \}_{k=0}^{2n-1}$ where:
\begin{equation}
\label{eq:dualw2}
b^*_{k,0} = \alpha
\sum_{\begin{array}{c} \mbox{\footnotesize{$0 \leq i \leq n-1$}} \\ 
			     \mbox{\footnotesize{$0 \leq j \leq n$}} \\ 
			     \mbox{\footnotesize{$i+j=k$}}
			     \end{array}}
\scriptchoice{n-1}{i} \scriptchoice{n}{j} \beta_{i,j}.
\end{equation}

The second and third coordinates $c_0'(t)$ and $-c_1'(t)$
of the \atang (\ref{eq:dual2}) are 1-dimensional Bezier curves of degree $n-1$,
which we need to degree-elevate to degree $2n-1$
for compatibility with the first coordinate.
A Bezier curve of degree $n$ with control points $\{ d_k \}_{k=0}^n$ 
is degree elevated \cite{farin97} to a Bezier curve of degree $n+r$ with control points
$\{ d_k^{(r)} \}_{k=0}^{n+r}$ where:
\[
d_k^{(r)} = \sum_{i=\mbox{\footnotesize{max}}(0,k-r)}^{\mbox{\footnotesize{min}}(n,k)} 
		d_i \frac{\tinychoice{n}{i} \tinychoice{r}{k-i} }{ \tinychoice{n+r}{k} }.
\]
%
Consequently, the 1-dimensional Bezier curve $c_0'(t)$ 
of degree $n-1$ with control points $\frac{n}{t_1-t_0} \Delta b_{i,0}$ 
can be degree elevated to a Bezier curve of degree $2n-1$ with control points 
$\{b^*_{k,1} \}_{k=0}^{2n-1}$ where:
\begin{equation}
\label{eq:2ndcoord}
b^*_{k,1} = 
\sum_{i=\mbox{\footnotesize{max}}(0,k-n)}^{\mbox{\footnotesize{min}}(n-1,k)} 
	\frac{n}{t_1-t_0} \Delta b_{i,0}
	\frac{ \tinychoice{n-1}{i} \tinychoice{n}{k-i} }{ \tinychoice{2n-1}{k} }
\end{equation}
%
Similarly, the Bezier curve $-c_1'(t)$
of degree $n-1$ with control points $-\frac{n}{t_1-t_0} \Delta b_{i,1}$
is degree elevated to a 1-dimensional Bezier curve of degree $2n-1$
with control points $\{b^*_{k,2} \}_{k=0}^{2n-1}$ where:
\begin{equation}
\label{eq:3rdcoord}
b^*_{k,2} = 
\sum_{i=\mbox{\footnotesize{max}}(0,k-n)}^{\mbox{\footnotesize{min}}(n-1,k)} 
	\frac{-n}{t_1-t_0} \Delta b_{i,1}
	\frac{ \tinychoice{n-1}{i} \tinychoice{n}{k-i} }{ \tinychoice{2n-1}{k} }
\end{equation}
%
\Comment{
By expanding $\frac{ \tinychoice{n-1}{j} \tinychoice{n}{k-j} }{ \tinychoice{2n-1}{k} }$
and regrouping, we can reexpress it as 
$\frac{ \tinychoice{2n-1-k}{n-1-j} \tinychoice{k}{j} }{ \tinychoice{2n-1}{n} }$
and (\ref{eq:1stcoord}) becomes
\begin{equation}
b^*_{k,1} = 
- \frac{n}{\Delta \tinychoice{2n-1}{n}} 
\sum_{j=\mbox{\footnotesize{max}}(0,k-n)}^{\mbox{\footnotesize{min}}(n-1,k)} 
	\tinychoice{2n-1-k}{n-1-j} \tinychoice{k}{j} (b_{j+1,2} - b_{j,2}).
\end{equation}
}
Combining (\ref{eq:dualw2}), (\ref{eq:2ndcoord}) and (\ref{eq:3rdcoord}),
we conclude that the \atang
is a Bezier curve of degree $2n-1$ in projective space with control points 
$\{b^*_k\}_{k=0}^{2n-1}$:
\[
\label{eq:dualcoord}
\tiny{
b^*_k = \alpha
\left(
\begin{array}{l}
	\displaystyle{\sum_{\begin{array}{c} \mbox{\tiny{$0 \leq i \leq n-1$}} \\ 
		       \mbox{\tiny{$0 \leq j \leq n$}} \\ 
		       \mbox{\tiny{$i+j=k$}} \end{array}}}
	\tinychoice{n-1}{i} \tinychoice{n}{j} \beta_{i,j} \\
%	
	\displaystyle{\sum_{i=\mbox{\tiny{max}}(0,k-n)}^{\mbox{\tiny{min}}(n-1,k)}}
	\tinychoice{n-1}{i} \tinychoice{n}{k-i} (b_{i+1,0} - b_{i,0})\\
%	
	\displaystyle{\sum_{i=\mbox{\tiny{max}}(0,k-n)}^{\mbox{\tiny{min}}(n-1,k)}}
	\tinychoice{n-1}{i} \tinychoice{n}{k-i} (b_{i,1} - b_{i+1,1})
\end{array}
\right)}
\]
%
A Bezier curve in projective 2-space is equivalent to a 
rational Bezier curve in 2-space.
Weights of the rational Bezier curve correspond to the
projective coordinate of the control points of the polynomial Bezier curve.
This leads to the rational Bezier curve of the theorem.
\QED

\noindent The Bezier form of the \btang is equivalent, 
simply reinterpreting the projective coordinate.
%
\begin{theorem}
\label{thm:rationaldualb}
Let $C(t)$ be a plane Bezier curve of degree $n$ 
with control points $\{ b_i \}_{i=0}^n$ over $[t_0,t_1]$.
The \btang $C_b^*(t)$ is a rational Bezier curve of degree $2n-1$ 
over $[t_0,t_1]$ with weights $\{w_k\}_{k=0}^{2n-1}$ where: 
\begin{displaymath}
\scriptsize{
w_k = \alpha
\displaystyle{\sum_{i=\mbox{\tiny{max}}(0,k-n)}^{\mbox{\tiny{min}}(n-1,k)}}
	\tinychoice{n-1}{i} \tinychoice{n}{k-i} (b_{i+1,0} - b_{i,0})
}
\end{displaymath}
%
and control points $\{c_k\}_{k=0}^{2n-1}$ where:
\begin{displaymath}
\tiny{
c_k = \frac{\alpha}{w_k} 
\left(
\begin{array}{l}
	\displaystyle{\sum_{i=\mbox{\tiny{max}}(0,k-n)}^{\mbox{\tiny{min}}(n-1,k)}}
	\tinychoice{n-1}{i} \tinychoice{n}{k-i} (b_{i,1} - b_{i+1,1})\\
%
	\displaystyle{\sum_{\begin{array}{c} \mbox{\tiny{$0 \leq i \leq n-1$}} \\ 
		       \mbox{\tiny{$0 \leq j \leq n$}} \\ 
		       \mbox{\tiny{$i+j=k$}} \end{array}}}
	\tinychoice{n-1}{i} \tinychoice{n}{j} \beta_{i,j}
\end{array}
\right)
}
\end{displaymath}
where $\alpha$ and $\beta_{i,j}$ are as in Theorem~\ref{thm:Bezier}.
\end{theorem}

% ---------------------------------------------------------------------------

The clipping of tangential curves in a tangential curve system
has an added benefit in the context of Bezier tangential curves: 
it removes the problem of negative weights.
Looking at (\ref{eq:wt}),
it is clear that some of the weights can be negative.
Negative weights are undesirable, 
since curve segments with negative weights
lack the convex hull property that is so useful for divide and conquer
techniques of intersection using subdivision.
Goldman and DeRose \cite{goldman86} have shown how to intersect curves without
the convex hull property, using an expanded convex hull,
but the tangential curve system does not need this special treatment,
thanks to clipping.

\begin{lemma}
Although the tangential curve of a Bezier curve may (and usually does) have
negative weights,
the tangential curve system of a Bezier curve does not.
\end{lemma}
\prf
Since horizontal tangents are associated with points at infinity on the
tangential a-curve, the clipping of shallow tangents from a \atang 
is equivalent to the clipping 
of a neighbourhood about all points at infinity (zero weights).
% Recall that a horizontal tangent is a point at infinity on the $a$-curve.
This is equivalent to the clipping of all sign changes of the weight,
so the weights of the remaining segments are purely positive or purely negative.  
Since negative weights on a segment with purely negative weights can be easily
removed by multiplying all of the weights by $-1$ (rational Bezier curves 
are invariant under transformations $w_i = kw_i$ of the weights),
the segments of a clipped \atang only have positive weights.
The same argument can be applied to clipped tangential b-curves.
\QED

The removal of negative weights is another indication that the tangential curve
system is the correct representation of tangent space.

% -----------------------------------------------------------------------------

\section{Bitangency as intersection}
\label{sec:bitang}

To illustrate the power of the new representation of tangent space,
we can apply it to the problem of bitangency.
Suppose that we want to find all of the bitangents of two curves.\footnote{A bitangent
	of the curves $C$ and $D$ is a line that is tangent to 
	both $C$ and $D$.}
The bitangents of $C$ and $D$ in primal space can be naturally computed as
the intersections of the tangential curve systems of $C$ and $D$ in dual space.
A steep bitangent will be found as an
intersection of the clipped tangential a-curves, while a shallow bitangent
will be found as an intersection of the clipped tangential b-curves.
Since the intersection of parametric curves is an efficient and well-studied problem
(e.g., \cite{sederberg86}),
this is an attractive reduction of the problem.
This approach is illustrated by Figures~\ref{fig:biTang}-\ref{fig:eg1}.
Duality's translation of bitangency to intersection
is analogous to the Fourier transform's translation of convolution to multiplication.
Both are simplifications in point of view.
% \cite{jj03} provides a full discussion of the importance of bitangency
% and its computation using tangential curve systems.

	% tangentialCurve -p data/circle.pts (with one tangent shown)
\begin{figure}[h]
\begin{center}
\includegraphics*[scale=.6]{img/jjdubi1.jpg}
\end{center}
% \centerline{\epsfig{figure=img/jjdubi1.ps,height=3.494in,width=6.509in}}
% 65% reduction
\caption{Bitangency is intersection in dual space}
\label{fig:biTang}
\end{figure}

	% bitang -p data/ob1.pts
\begin{figure*}[h]
\begin{center}
\includegraphics*[scale=.6]{img/jjdubi2.jpg}
\end{center}
% \centerline{\epsfig{figure=img/jjdubi2.ps,height=3.494in,width=6.509in}}
% 65% reduction
\caption{Computing bitangents in dual space}
\label{fig:eg1}
\end{figure*}

	% save for bitangency paper
	% bitang -p data/ob3.pts	
% \begin{figure*}[h]
% \centerline{\epsfig{figure=img/jjdubi3.ps,height=3.476in,width=6.509in}}
% 65% reduction
% \caption{Another complicated example}
% \label{fig:eg2}
% \end{figure*}

% -----------------------------------------------------------------------------

\section{Related work}
\label{sec:comparison}

There are several structures that relate to our tangential curve.
We now consider these relationships.

% -----------------------------------------------------------------------------

\subsection{The tangential variety}

The abstract idea of dualizing tangent spaces into curves
has been introduced in algebraic geometry \cite[p. 54]{hartshorne77}.
It is called the tangential variety.
The tangential curve system and this paper may be viewed as an elaboration
and implementation of this idea of the tangential variety.

% -----------------------------------------------------------------------------

\subsection{The hodograph}
\label{sec:hodo}

The tangential curve system is a representation of the tangent space of a curve.
What are its advantages over the hodograph, the standard representation of 
the tangent space of a curve?
A point $H(t)$ of the hodograph of a curve $C(t)$ represents the tangent 
vector at $C(t)$.
The problem with the hodograph is that it represents tangent vectors rather
than tangent lines.
The tangent line at $C(t)$ is only available indirectly, as $C(t) + sH(t)$,
and is dependent on both the hodograph and the original curve.
Moreover, many points of the hodograph can represent the same 
tangent line (viz., all points of the hodograph along the same line through the origin).
% represent the same direction and thus the same tangent line.

Consider how this affects the application of the hodograph to problems
involving curve tangent spaces.
We use bitangency as our example.
Recall that the intersection of tangential curve systems yields bitangents.
In contrast, the intersection of the hodographs of $A$ and $B$ does not yield the 
bitangents of $A$ and $B$.
% Unfortunately, the intersection of hodographs does not yield bitangents.
An intersection does not imply a bitangent, just two parallel tangents.
Conversely, a bitangent of the curves does not imply an intersection
of the hodographs, unless the tangent vectors at either end of the bitangent
are exactly the same length.
Thus, hodographs are not useful for finding bitangents.

More generally, this weakness of the hodograph (its representation of the tangent
vector only) hobbles it
for problems that work with scenes of many curves and their interactions.
This is precisely the type of problem 
that the tangential curve system is designed for:
the global analysis of a scene of curves.
This explains why the tangential curve is better suited to the solution of bitangency,
which is a global property of a scene of curves.

% -----------------------------------------------------------------------------

\subsection{The dual Bezier curve}
\label{sec:dualBezier}

The tangential curve uses duality.
How does it compare to the dual Bezier curve of Hoschek \cite{hoschek83}?
% a dual representation of a plane curve
The dual Bezier curve is the representation of a plane curve by control lines
rather than control points (Figure~\ref{fig:Hoschekdual}).
In particular, a plane curve is represented as the envelope of a line family
and the line family is represented in terms of control lines.
	% The curve is represented using a control 'polygon' whose control points are duals of lines.
	% as $\sum_{i=0}^n \mbox{dual}(L_i) B_i^n(t)$
Thus, the dual Bezier curve is an alternative representation for a plane curve,
using a dual control structure (control lines rather than control points).
The differences between the dual Bezier curve
and tangential curve may be captured most clearly as follows:
the dual Bezier curve is the representation of a curve in primal space
using a dual control structure,
while the tangential curve is the representation of a curve's {\em tangent space}
in dual space using a primal structure (such as the Bezier control structure
of Theorem~\ref{thm:Bezier}).
	% Hoschek's dual Bezier curve is almost a mirror image of our tangential curve.
Note that the tangential curve of $C$ is a completely different curve from $C$
(Figures~\ref{fig:tangcurve1}-\ref{fig:tangcurve3}).

% In particular, the dual Bezier curve is not appropriate for our intersection
% problem in dual space.
% For example, Hoschek never builds a curve in dual space.

\begin{figure}
\begin{center}
\includegraphics*[scale=.4]{img/jjdube.jpg}
\end{center}
% \centerline{\epsfig{figure=img/jjdube.ps,height=3.35in,width=3.2in}}
% ?% reduction
\caption{A dual Bezier curve: defined by control lines rather than control points}
\label{fig:Hoschekdual}
\end{figure}

\Comment{
	Not after Pottmann translates back and forth:
	Hoschek's dual Bezier curve is almost a mirror image of our tangential curve: 
	%
	% it is given a family of lines and builds a curve, while the tangential curve
	% is given a curve and builds a representation of its family of tangents.
	%
	it builds a plane curve from a family of lines 
	% (specified indirectly by a set of control lines), 
	while our \tang builds a family of {\bf tangent} lines from a curve.
}

% -----------------------------------------------------------------------------

\subsection{\plucker coordinates}
\label{sec:plucker}

We have used geometric duality to map tangent lines to points.
What would happen if we instead used \plucker coordinates,
the classical representation for lines
by points in projective 5-space \cite{bottemaRoth79}? 	% p. 26
	% also Sommerville, Analytic Geometry of 3 Dimensions, Chapter 16,
	% used in Teller, SIGGRAPH 92, Antipenumbra paper
	% also Pottmann/Peternell/Ravani, CAD 1999, Intro to Line Geometry
A representation of tangent space can indeed be developed using \plucker
coordinates, but it turns out that this \plucker representation is almost 
identical to our dual representation.

\begin{defn2}
The {\bf \plucker tangential curve} of $C(t)$ is the curve $\hat{C}(t) \subset P^5$,
where $\hat{C}(t)$ is the \plucker coordinates of the tangent at $C(t)$.
\end{defn2}

\begin{lemma}
The \plucker tangential curve is essentially equivalent to the tangential a-curve,
but in a higher-dimensional space.
\end{lemma}
\prf
The \plucker coordinates of the line $P+uV$ are $(V,P \times V)$.
Since the tangent line at $C(t)$ is $C(t) + uC'(t)$,
the \plucker tangential curve of $C(t)$
% the image of the tangent space of the curve $C(t)$ in \plucker space 
is the curve $(C'(t), C(t) \times C'(t))$.
If $C(t) = (c_0(t),c_1(t),0)$ is a plane polynomial curve,
this \plucker tangential curve becomes
\[
	(c'_0, c'_1, 0, 0, 0, c_0c'_1 - c_1c'_0)
\]
in projective 5-space.
Compare this to the tangential a-curve (\ref{eq:dual2}) in projective 2-space
$(c_0c'_1 - c_1c'_0, c'_0, -c'_1)$.
This establishes the essential equivalence of the \plucker representation.
\QED

We use the dual space representation since its development is more natural
for lines in 2-space and leads to a slightly simpler implementation.
\plucker coordinates are designed for the more general representation
of lines in 3-space,
and consequently \plucker space is a higher-dimensional version of our dual space:
a 5-dimensional, rather than 2-dimensional, projective space.
Our dual representation of a line (Definition~\ref{defn:dual})
is explicitly designed for lines in 2-space.
\Comment{
Indeed, it only works in 2-space, since the line is a hyperplane
(representable by a single implicit equation) in 2-space but not in 3-space.
%	A line in 3-space requires two implicit equations for its representation,
% 	which moreover are not unique even in a projective sense,
%	so it cannot be represented using this simple version of duality.
}
Thus, the dual representation of the tangent space of plane curves is simpler, more intuitive, and 
more direct than the \plucker representation.

\Comment{
Since a \plucker solution is identical,
the computations in the previous sections can be trivially lifted to
a solution in \plucker space, if desired.
For example, the tangential curve is still 3-dimensional,
and one coordinate must be interpreted as a projective coordinate,
leading to the same solution to points at infinity by trimming two curves.

Curve intersection in the plane is well understood, even if this plane
resides in projective 5-space (affine 6-space).
We can ignore the presence of the higher dimensions by rotating the
two-dimensional plane of the curves to the $\{z_3=0,z_4=0,z_5=0\}$
plane and ignoring the 3rd, 4th, and 5th coordinates.
After computing the intersection using classical 2-dimensional algorithms
(Sederberg), the intersection points should be rotated back before
translating their Plucker coordinates to lines.
}

% -----------------------------------------------------------------------------

\section{Mapping the tangential curve back to primal space}
\label{sec:back}

Since a tangential curve $C^*(t)$ is a plane curve,
it is natural to wonder what {\em its} tangential curve would look like.
One is tempted to hope that this would be the original curve $C(t)$.
And it is.
It is not immediately clear that this must be the case: 
although a point $C(t)$ must dualize to a line, 
% \footnote{Actually, the principle of duality dictates that $C(t)$ must dualize to a line through the point $C^*(t)$.}
it is not clear that it necessarily needs to dualize to the tangent of $C^*(t)$ (Figure~\ref{fig:pt2tang}).
That it does shows that duality respects tangency.
We know that $(P^*)^* = P$ when $P$ is a point or a line and $^*$ is geometric duality.
Now we know that this result lifts to tangent spaces: 
$(C^*)^* = C$ when $C$ is a plane curve 
and $^*$ is the dualization of a curve's tangent space from Definition~\ref{defn:tangentialcurve}.
% Just as with points and lines, two applications of duality restores the original point or line,
% this also occurs for tangent spaces of curves.

\begin{lemma}
Let $C$ be a rational plane curve, and
let $D$ be the tangential a-curve of $C$.
The tangential a-curve of $D$ is $C$.
\end{lemma}
\prf
Let $C$ be the rational curve $(\frac{c_0(t)}{c_2(t)},\frac{c_1(t)}{c_2(t)})$.
Let $d_0 := c_0c'_1 - c'_0c_1$, $d_1 := c'_0c_2 - c_0c'_2$
and $d_2 := c_1c'_2 - c'_1c_2$.
From Lemma~\ref{lem:rattangacurve}, 
the tangential a-curve $D$ is $(\frac{d_0}{d_2}, \frac{d_1}{d_2})$.
Similarly, the tangential a-curve of $D$ is 
\[
	(\frac{d_0 d'_1 - d'_0 d_1}{d_1 d'_2 - d'_1 d_2},
	 \frac{d'_0 d_2 - d_0 d'_2}{d_1 d'_2 - d'_1 d_2}).
\]
Let 
\begin{equation}
\label{eq:R}
	R = c_0(c''_1c'_2 - c'_1c''_2) + c'_0(c_1c''_2 - c''_1c_2) + c''_0(c'_1c_2 - c_1c'_2).
\end{equation}
Then, after some simplification, we can show that 
$d_0d'_1 - d'_0d_1 = c_0R$,
$d'_0 d_2 - d_0 d'_2 = c_1R$, and
$d_1 d'_2 - d'_1 d_2 = c_2R$.
We conclude that the tangential a-curve of $D$ is $(\frac{c_0R}{c_2R},\frac{c_1R}{c_2R})$
or $C$.
\QED

\begin{corollary}
% We observe that this means that 
The point $C(t)$ a-dualizes to the tangent at $C_a^*(t)$ (Figure~\ref{fig:pt2tang}).
\end{corollary}
% We have confirmed this experimentally.

% Of course, the same results hold for tangential b-curves too.

The redundant factor (\ref{eq:R}) resolves another dilemma:
how can $(C^*)^* = C$ when the degree of $C^*$ is higher than $C$
(Lemma~\ref{lem:rattangacurve} and Theorem~\ref{thm:Bezier})?
The introduction of the redundant factor (\ref{eq:R}) explains the 
near-quadrupling of the degree of $(C^*)^*$.

	% tangentialCurve -p data/circleOne.pts (with dual window showing 'tangent at active pt')
\begin{figure}[h]
\begin{center}
\includegraphics*[scale=.5]{img/jjdupt.jpg}
\end{center}
% \centerline{\epsfig{figure=img/jjdupt.ps,height=2.956in,width=5.508in}}
% \centerline{\epsfig{figure=img/jjdupt.ps,height=2.688in,width=5.007in}}
% 50% reduction
\caption{A point in primal space dualizes to a tangent of the tangential curve in dual space}
% \caption{$C(t)$ dualizes to the tangent of $C^*(t)$}
\label{fig:pt2tang}
\end{figure}

\Comment{
THIS IS THE FORMULA FOR C-CURVES
Note that if one wants to use duality
to map points of dual space to lines in primal space,
care must be taken to work in projective space.
If the intersection point is $C^*(t) = (c_1,c_2)$ and the weight function 
of the rational Bezier curve $C^*(t)$ at
this parameter value is $c_3$, the coordinates of the intersection
in projective dual space are $(c_3 c_1, c_3 c_2, c_3)$
and the associated line in primal space is $c_3 c_1 x + c_3 c_2 y + c_3 = 0$.
% Since this line is not enough for the filtering step (4)
% (its endpoints on the curves are also needed)
% and it is expensive to compute the endpoints of the tangent 
% from the implicit equation of the line, we use the parameter pair representation instead.
}

% -----------------------------------------------------------------------------

\section{Conclusions}
\label{sec:conclude}

We have studied the tangential curve, a dual image of the tangent space of
a plane curve, and the tangential curve system, a robust model of the entire tangent
space of a plane curve decomposed into steep and shallow tangents.
Some key choices in the implementation of the tangential curve system
are the dualities,
the use of a Cartesian representation of dual space to preserve the 
rationality and 2-dimensionality of the tangential curve,
and the dichotomy of line space into steep and shallow lines
that imposes a clipping on the 
tangential curves of a tangential curve system.
This clipping serendipitously leads to short curves 
and removes negative weights from the rational Bezier representation
of the tangential curve.

The modeling of a tangent space by a tangential curve system is
a useful tool in the analysis of the interactions in a scene of curves, 
as we illustrated in its application to bitangency 
and its favourable comparison to the hodograph.
We are studying the application of the tangential curve system 
to the solution of problems in visibility analysis,
lighting and motion planning amongst smooth obstacles.
We are also working on the extension of the ideas of this paper to 
a dual representation of the tangent space of a surface.

% Our understanding of a curve's tangent space is improved
% through an appreciation of a tangent's dual nature as a line in primal space 
% and a point in dual space.

% -----------------------------------------------------------------------------

\section{Acknowledgements}

I appreciate discussions with Gershon Elber, Helmut Pottmann, Alan Sprague
and Wenping Wang that improved this paper.
I also thank Xiao Hu for Figure~\ref{fig:Hoschekdual}.
% Pottmann: alerted me to Hoschek's dual Bezier curves
% Sprague: suggested change to a-duality: (c,b,a) rather than (b,c,a)
% Elber: suggested looking at mapping tangential curve back to primal space

% -----------------------------------------------------------------------------

\bibliographystyle{plain}
\begin{thebibliography}{99}	% {Lozano-Perez 83}

\bibitem{bottemaRoth79}
Bottema, O. and B. Roth (1979)
Theoretical Kinematics.
Dover (New York).

\bibitem{chazelle85}
Chazelle, B., L. Guibas and D.T. Lee (1985)
The power of geometric duality.
BIT, 76--90.

\bibitem{edels87}
Edelsbrunner, H. (1987)
Algorithms in Combinatorial Geometry.
Springer Verlag (Heidelberg), Chapters 1 and 12.

\bibitem{farin97}
Farin, G. (2002)
Curves and Surfaces for CAGD: A Practical Guide (5th edition).
Morgan Kaufmann (New York).

\bibitem{foley96}
Foley, J., A. van Dam, S. Feiner and J. Hughes (1996)
Computer Graphics: Principles and Practice.
2nd edition, Addison-Wesley (Reading, MA).

\bibitem{goldman86}
Goldman, R. and T. DeRose (1986)
Recursive subdivision without the convex hull property.
Computer Aided Geometric Design 3, 247--265.

\bibitem{hartshorne77}
Hartshorne, R. (1977)
Algebraic Geometry.
Springer-Verlag (New York).

\bibitem{zorin00}
Hertzmann, A. and D. Zorin (2000)
Illustrating smooth surfaces.
SIGGRAPH 2000, 517--526.

\bibitem{hoschek83}
Hoschek, J. (1983)
Dual Bezier curves and surfaces.
In {\em Surfaces in Computer Aided Geometric Design},
R. Barnhill and W. Boehm, eds.,
North Holland (Amsterdam), 147--156.

\bibitem{jj01a}
Johnstone, J. (2001)
A Parametric Solution to Common Tangents.
International Conference on Shape Modelling and Applications (SMI 2001), Genoa, Italy,
IEEE Computer Society, 240--249.

\bibitem{orourke94}
O'Rourke, J. (1994)
Computational Geometry in C.
Cambridge University Press (New York).

\bibitem{patterson01}
Patterson, R. (2001)
Duality and Bezier Curves.
Technical Report 01-06, Dept.\ of Mathematical Sciences,
Indiana University Purdue University Indianapolis.

\bibitem{pedoe70}
Pedoe, D. (1970)
Geometry: A Comprehensive Course.
Dover Publications (New York).

\bibitem{sederberg86}
Sederberg, T. and S. Parry (1986)
Comparison of three curve intersection algorithms.
Computer Aided Design 18, 58--63.

\bibitem{wenping01}
Wenping Wang, personal communication.

\bibitem{welzl85}
Welzl, E. (1985)
Constructing the visibility graph for $n$ line segments
in $O(n^2)$ time.
Information Processing Letters 20, 167--171.
% see Lemma 2 for use of duality

\end{thebibliography}

\end{document}
