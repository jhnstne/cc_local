\documentclass[11pt]{letter}
\signature{John K. Johnstone}

\setlength{\oddsidemargin}{0pt}
\setlength{\evensidemargin}{0pt}
%\setlength{\headsep}{0pt}
%\setlength{\topmargin}{0pt}
%\setlength{\textheight}{8.75in}
%\setlength{\topmargin}{-0.4in}
%\setlength{\textheight}{9.5in}
\setlength{\textwidth}{6.5in}
\setlength{\headsep}{.2in}

\pagestyle{empty}

\begin{document}
\begin{letter}
{}

I submit 'The Tangential Curve' for your consideration
for publication in {\em Computer Aided Geometric Design}.
This paper builds upon ideas introduced in my conference paper
'A Parametric Solution to Common Tangents' 
in Shape Modeling 2001.
I make the following advances on the results of the
conference paper:

\begin{itemize}
\item   We look at the tangential curve in a more general context,
	allowing it to stand alone as a theoretical development.
	The previous development was tightly intertwined with
	bitangency.
	It is now presented independently.
\item	The theory of geometric duality is discussed, giving a context for
	our particular use of duality.
\item	Our choice of dual map is explained and 
	compared to other line-point dualities.
\item 	The a-duality map is changed to better match duality theory.
\item   The behaviour of our duality at infinity is carefully analyzed.
	Our Cartesian representation of dual space is contrasted
	with other representations in dealing with points at infinity,
	particularly its preservation of rationality and 2-dimensionality.
\item 	We develop the computational framework of tangential curves
	for rational curves in Section~3.
\item 	We discuss the dualization of a tangential curve back to primal space
	in Section~7 and explain the apparent jump in degree.
\item 	New terms have been introduced that clarify concepts,
	such as steep/shallow lines and tangential curve system.
\end{itemize}

Please contact me at jj@cis.uab.edu or (205) 975-5633
if you have any questions.

\closing{Sincerely,}

\end{letter}
\end{document}
