%
% Latex instructions for ACM SE Conf. papers; ``two-point.eps'' is
% a separate attachment.
%
\documentclass[twocolumn]{article}
\usepackage{epsfig}
\pagestyle{empty}
\textheight=9.0in
\textwidth=7.0in
\columnsep=0.25in
% local printer adjustments:
\topmargin=0.25in
\oddsidemargin=-0.25in
\evensidemargin=-0.25in

\begin{document}
\bibliographystyle{plain}

\thispagestyle{empty}

\title{My Paper in Correct Format}
\author{
Robert Geist \\
\quad\\
\\
Department of Computer Science \\
Clemson University \\
Clemson, South Carolina USA 29634-0974\\
rmg@cs.clemson.edu
}
\date{\quad}
\maketitle
{\small
\parskip=12pt
\parindent=0pt

{\bf Abstract-} This is a description of the desired camera-ready
format for papers for the $38^{th}$ Annual ACM Southeast Conference,
to be held in Clemson, South Carolina, April 7-8, 2000.  It also
includes the all-important camera-ready deadline: March 1, 2000. The basic
idea for formatting is to use the default LaTeX two-column article style.  
Additional details follow below.
}

\parindent=2em
\section{Introduction.}

\thispagestyle{empty}

Use plain, white, 8.5 x 11 paper.  Papers should be typeset in 
two-column format, using 10pt Computer Modern or Times Roman font.  Titles
should be in either 17pt font (LaTeX $\backslash$LARGE) or 16pt 
font (MSWord), centered at the top of the first page across both columns.  
Authors names and affiliations, including email addresses, should be in 
12pt font (LaTeX $\backslash$large), also centered across both columns 
below the title.

An abstract of at most 200 words should begin the first column.
It should begin with the word ``Abstract-'' in bold font, and
should be 1 paragraph, without indentation.  Major section headings
should be in 14pt font (LaTeX $\backslash$Large), bold, numbered, and 
left-justified.  With the exception of the first paragraph in each section, 
paragraphs should be indented 2em.  Text should be single-spaced, with no 
additional spacing between paragraphs. 

\section{Margins.}

Side margins should be 0.75in, with a 0.25in column separation.
This leaves 6.75in for two columns of text, so each
column should be 3.375in wide.  Top and bottom 
\quad\\
\quad\\
\quad\\
\quad\\
margins should be 1.0in each.  At the bottom of the first column, 
leave 1.0in additional space for the ACM Copyright notice.

\section{Pagination.}

Do not number the pages within the text.  
Authors of refereed papers will be allowed 10 pages in the Proceedings.
Authors of work-in-progress reports will be allowed 2 pages.  These
are strict limits, and include EVERYTHING, e.g. diagrams, figures,
references, acknowledgements. These limits will be enforced by discarding 
any extra pages, starting with the title page.

\section{References.}

The bibliography should appear at the end of the article, organized
alphabetically by first author and numbered sequentially.  References
within the text should use the numbers.  An example is shown here
\cite{danforth}.

Displayed equations should also be numbered and referenced by number
within the text.  From \cite{cohen:progressive},
for a given patch $i$, its radiosity $B_i$ is:
\begin{equation}
\label{eqn:radiosity}
	B_i = E_i + \rho_i \sum_{j = 1}^{n} B_j \frac{A_j}{A_i} F_{ji}
\end{equation}
where \\
$E_i$ =  emission of patch $i$, the extent to which patch $i$ acts as
        a light source\\
$F_{ji}$ =  form factor from j to i, the fraction of 
energy leaving patch $j$ that arrives at patch $i$ \\
$A_i$ = area of patch $i$ \\
$\rho_i$ = reflectivity of patch $i$ \\
$n$ = number of discrete patches.  
Equation (\ref{eqn:radiosity}) describes patch irradiance.

\section{Figures and Tables.}

Figures and tables should appear within the text and within a single column, 
if possible.  If too large, they may be centered across both columns.  A
figure from \cite{danforth} can be seen in Figure 
\ref{fig:hemisphere_proj}.

% Suck in the postscript figure:
\begin{figure}
\centerline{\epsfig{figure=two-point.eps,height=2.5in,width=2.0in}}
\caption{\label{fig:hemisphere_proj}Projecting patches provides form factors.}
\end{figure}

\section{Deadlines.}

An electronic ACM Copyright form has been included as a separate attachment.  
You must print a hard copy of this form, sign it, and return it to the 
Conference Program Chair:\\
\quad\\
Robert Geist \\
Department of Computer Science \\
411 Edwards Hall \\
Clemson University \\
Clemson, SC 29634-0974\\
\quad\\
It must arrive in Clemson on or before March 1, 2000.  This is a
hard deadline.

An electronic copy of your camera-ready paper, correctly formatted in 
either LaTeX (.tex) or MSWord (.doc), must be sent to the Program Chair at:\\
\quad\\
rmg@cs.clemson.edu

It must also arrive on or before March 1, 2000.
We need document source in order to paginate and add copyright notices.
If you cannot supply either LaTeX (.tex) or MSWord (.doc) source, please
contact the Program Chair immediately.

\section{Presentations.}

We will insist that at least one author registrar for the
conference, attend the conference, and present the paper.  Final
acceptance is conditional upon these terms.  Conference attendees
pay registration fees in anticipation of an opportunity to discuss
results with authors.  Failure to provide that opportunity would be
unfair.  If you know that you will be unable to attend, please, in
fairness, withdraw your paper now.

Authors of refereed papers will be given 30 minute time slots to
present their papers.  Authors of works-in-progress will be given
15 minute time slots.  Students who are finalists in the student
paper competition should practice presentations before the conference.
A large component of the final evaluation is quality of presentation.

Any questions may be directed to either Robert Geist, rmg@cs.clemson.edu, 
864-656-2258.

\begin{thebibliography}{1}

\bibitem{cohen:progressive}
M.~Cohen, S.~Chen, J.~Wallace, and D.~Greenberg.
\newblock A Progressive Refinement Approach to Fast Radiosity Image Generation.
\newblock {\em SIGGRAPH}, 22(4), 1988, pp. 75-84.

\bibitem{danforth}
R. Danforth and R. Geist,``A Spring-loaded Vertex Model for Automatic Surface
Meshing'' {\it Proc. IASTED Int. Conf. on Computer Graphics and Imaging
(CGIM'99)}, Palm Springs, CA, October, 1999, pp. 10 - 16.
\end{thebibliography}
\end{document}

