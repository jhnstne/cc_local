\documentclass[twocolumn,10pt]{article}
\usepackage{epsfig}
\pagestyle{empty}
\textheight=9.0in
\textwidth=7.0in
\columnsep=0.25in
% local printer adjustments:
\topmargin=-0.5in
\oddsidemargin=-0.25in
\evensidemargin=-0.25in

\newcommand{\prf}{\noindent{{\bf Proof}:\ \ \ }}
\newcommand{\QED}{\vrule height 1.4ex width 1.0ex depth -.1ex\ \vspace{.2in}} % square box
\newcommand{\choice}[2]{\mbox{\footnotesize{$\left( \begin{array}{c} #1 \\ #2 \end{array} \right)$}}}      
\newcommand{\scriptchoice}[2]{\mbox{\scriptsize{$\left( \begin{array}{c} #1 \\ #2 \end{array} \right)$}}}
\newcommand{\tinychoice}[2]{\mbox{\tiny{$\left( \begin{array}{c} #1 \\ #2 \end{array} \right)$}}}
\newcommand{\plucker}{Pl\"{u}cker\ }
\newcommand{\tang}{tangential curve\ }
\newcommand{\tangs}{tangential curves\ }
\newcommand{\Tang}{Tangential curve\ }
\newcommand{\atang}{tangential $a$-curve\ }
\newcommand{\btang}{tangential $b$-curve\ }
\newcommand{\ctang}{tangential $c$-curve\ }
\newcommand{\atangs}{tangential $a$-curves\ }
\newcommand{\btangs}{tangential $b$-curves\ }
\newcommand{\ctangs}{tangential $c$-curves\ }
\newtheorem{theorem}{Theorem}[section]
\newtheorem{rmk}[theorem]{Remark}
\newtheorem{example}[theorem]{Example}
\newtheorem{conjecture}[theorem]{Conjecture}
\newtheorem{claim}[theorem]{Claim}
\newtheorem{notation}[theorem]{Notation}
\newtheorem{lemma}[theorem]{Lemma}
\newtheorem{corollary}[theorem]{Corollary}
\newtheorem{defn2}[theorem]{Definition}

% -----------------------------------------------------------------------------

\begin{document}

\title{Smooth Visibility from a Point}
\author{J.K. Johnstone \\
  	\quad\\
	\\
	Geometric Modeling Lab \\
	Computer and Information Sciences \\
	University of Alabama at Birmingham \\
	University Station, Birmingham, AL 35294 \\
	jj@cis.uab.edu
}
\date{\quad}
\maketitle
\thispagestyle{empty}
{\small
\parskip=12pt
\parindent=0pt

{\bf Abstract-} We propose a solution in dual space to the problem of
computing the tangents of a plane Bezier curve that pass through a point,
an important problem in the analysis of visibility.
Our solution reduces to the intersection of a line with a Bezier curve
in dual space.
It is equivalent in complexity to the standard solution,
but introduces an alternative, geometric interpretation of the problem.
This new characterization has promise for the solution of more general problems in visibility.
We also compare the dual solution to the solution using polar theory,
which reduces to the intersection of two algebraic curves after implicitization.

Keywords: visibility, tangents through a point, duality, polar.
}


\parindent=2em
\section{Introduction}

The tangents of a plane curve that meet a point $P$ (Figure~\ref{fig:problem})
define important boundaries of visibility from $P$
for robotics, graphics, and geometric analysis.
A shortest path from $P$ will begin along one of these tangents 
(Figure~\ref{fig:motion}).
A point light at $P$ will cast shadows bounded by these tangents,
when the curve is interpreted as an opaque obstacle (Figure~\ref{fig:shadow}).
The 2-dimensional silhouette from the viewpoint $P$
is defined by the visible points of tangency (Figure~\ref{fig:problem}).

\newbox\jjpoprbox
\newdimen\jjpoprwd
\font\jjpopra=jjpopra at 72.27truept
\setbox\jjpoprbox=\vbox{\hbox{%
\jjpopra\char0\char1\char2}}
\jjpoprwd=\wd\jjpoprbox
\setbox\jjpoprbox=\hbox{\vbox{\hsize=\jjpoprwd
\parskip=0pt\offinterlineskip\parindent0pt
\hbox{\jjpopra\char0\char1\char2}
\hbox{\jjpopra\char3\char4\char5}
\hbox{\jjpopra\char6\char7\char8}}}
\ifx\parbox\undefined
    \def\setjjpopr{\box\jjpoprbox}
\else
    \def\setjjpopr{\parbox{\wd\jjpoprbox}{\box\jjpoprbox}}
\fi

\begin{figure}
\hspace{.5in} \setjjpopr
\caption{Tangents through a point}
\label{fig:problem}
\end{figure}

\newbox\jjposhbox
\newdimen\jjposhwd
\font\jjposha=jjposha at 72.27truept
\setbox\jjposhbox=\vbox{\hbox{%
\jjposha\char0\char1}}
\jjposhwd=\wd\jjposhbox
\setbox\jjposhbox=\hbox{\vbox{\hsize=\jjposhwd
\parskip=0pt\offinterlineskip\parindent0pt
\hbox{\jjposha\char0\char1}
\hbox{\jjposha\char2\char3}}}
\ifx\parbox\undefined
    \def\setjjposh{\box\jjposhbox}
\else
    \def\setjjposh{\parbox{\wd\jjposhbox}{\box\jjposhbox}}
\fi

\begin{figure}
\hspace{1in} \setjjposh
\caption{A shortest path}
\label{fig:motion}
\end{figure}

The solution to the discrete polygonal version of this problem,
where the curve is replaced by a polygon, is straightforward
(Figure~\ref{fig:polygon}).
For each vertex of the polygon, consider the outward-pointing normals
of its two adjacent edges.
Vertices such that one outward-pointing
\clearpage
\noindent normal points towards the point $P$
and the other points away become the discrete analog of the points of 
tangency on the curve.
A smooth solution using curves is more involved, 
yet preferable for its precision and, 
if properly implemented, its improved efficiency over a densely sampled polygon.
We are especially interested in the solution of our problem for Bezier
curves, since they are the dominant modeling tool for curves.

\newbox\jjpocabox
\newdimen\jjpocawd
\font\jjpocaa=jjpocaa at 72.27truept
\setbox\jjpocabox=\vbox{\hbox{%
\jjpocaa\char0\char1\char2}}
\jjpocawd=\wd\jjpocabox
\setbox\jjpocabox=\hbox{\vbox{\hsize=\jjpocawd
\parskip=0pt\offinterlineskip\parindent0pt
\hbox{\jjpocaa\char0\char1\char2}
\hbox{\jjpocaa\char3\char4\char5}}}
\ifx\parbox\undefined
    \def\setjjpoca{\box\jjpocabox}
\else
    \def\setjjpoca{\parbox{\wd\jjpocabox}{\box\jjpocabox}}
\fi

\begin{figure}
\hspace{.4in} \setjjpoca
\caption{Casting a shadow}
\label{fig:shadow}
\end{figure}

\newbox\jjpolybox
\newdimen\jjpolywd
\font\jjpolya=jjpolya at 72.27truept
\setbox\jjpolybox=\vbox{\hbox{%
\jjpolya\char0\char1\char2}}
\jjpolywd=\wd\jjpolybox
\setbox\jjpolybox=\hbox{\vbox{\hsize=\jjpolywd
\parskip=0pt\offinterlineskip\parindent0pt
\hbox{\jjpolya\char0\char1\char2}
\hbox{\jjpolya\char3\char4\char5}}}
\ifx\parbox\undefined
    \def\setjjpoly{\box\jjpolybox}
\else
    \def\setjjpoly{\parbox{\wd\jjpolybox}{\box\jjpolybox}}
\fi

\begin{figure}
\hspace{.25in} \setjjpoly
\caption{The polygonal problem}
\label{fig:polygon}
\end{figure}

The classical solution to the tangents of a curve $C$
through a point $P$ uses polar theory \cite{semple85}.
The points of the curve whose tangents intersect $P$ are 
the intersections of two algebraic curves:
the curve $C$ and the polar of $P$ with respect to $C$
(Figure~\ref{fig:polar}).

The standard parametric solution (e.g., \cite{kim88})
solves for the points $C(t) = (x(t),y(t))$ of the curve whose normal $N(t)$
is orthogonal to the line to $P = (p_1,p_2)$ (Figure~\ref{fig:standard}):
\begin{equation}
\label{eq:standard}
 	(C(t) - P) \cdot N(t) = 0
\end{equation}
or
\[
	f(t) = (x(t) - p_1, y(t) - p_2) \cdot (-y'(t), x'(t)) = 0
\]
This is a univariate equation of degree $2n-1$ in $t$,
where $n$ is the degree of $C(t)$.

\newbox\jjpostbox
\newdimen\jjpostwd
\font\jjposta=jjposta at 72.27truept
\setbox\jjpostbox=\vbox{\hbox{%
\jjposta\char0\char1\char2}}
\jjpostwd=\wd\jjpostbox
\setbox\jjpostbox=\hbox{\vbox{\hsize=\jjpostwd
\parskip=0pt\offinterlineskip\parindent0pt
\hbox{\jjposta\char0\char1\char2}
\hbox{\jjposta\char3\char4\char5}}}
\ifx\parbox\undefined
    \def\setjjpost{\box\jjpostbox}
\else
    \def\setjjpost{\parbox{\wd\jjpostbox}{\box\jjpostbox}}
\fi

\begin{figure}
\hspace{.3in} \setjjpost
\caption{As normals orthogonal to the line to $P$}
\label{fig:standard}
\end{figure}

In this paper, we propose a solution 
that reduces to the intersection of a line and a Bezier curve 
of degree $2n-1$ in dual space.
This compares favourably with the polar solution.
A curve-curve intersection is replaced by a simpler line-curve intersection.
Also, the dual of a Bezier curve
can be easily represented as a rational Bezier curve (Theorem~\ref{thm:rationalduala}),
while the polar of a curve cannot.
Consequently, the intersection of implicit algebraic curves in the polar solution
is replaced by the more familiar intersection of parametric Bezier curves.

The dual solution is similar in efficiency to the standard parametric
solution (\ref{eq:standard}), since the intersection of a line and a parametric
curve of degree $2n-1$ is comparable to the solution of a univariate equation
of degree $2n-1$.
The major difference is in point of view and generalizability.
The dual method introduces a new paradigm with great potential for the
solution of visibility problems in dual space,
in which an algebraic solution is replaced by a geometric solution.
In other work, we will show that it can lead to a better solution
for the common tangents of two curves \cite{jj00}
(and we shall appeal to some of the results from this paper).
We are also presently working on a solution to the common tangent planes of two
surfaces that has no comparable solution under the algebraic interpretation.
By studying the dual solution for the simpler problem of 
the tangents through a point,
we establish a template for the dual space method
and clearly distinguish the geometric and algebraic approaches.

The basic argument of our solution in dual space is as follows.
In dual space, points become lines and lines become points.
In particular, the point $P$ dualizes to the line $P^*$ and 
the tangent lines of a curve dualize to the points of a so-called 
tangential curve (Figure~\ref{fig:duality}).
In dual space, the intersections of the line $P^*$ with the tangential curve
encode the tangent lines through $P$
(Figures~\ref{fig:ob2}-\ref{fig:ob3}).
In particular, an intersection point maps back to a line, which
must be a tangent (since the intersection lies on the tangential curve)
that contains $P$ (since the intersection also lies on $P^*$).

\newbox\jjpolfbox
\newdimen\jjpolfwd
\font\jjpolfa=jjpolfa at 72.27truept
\font\jjpolfb=jjpolfb at 72.27truept
\font\jjpolfc=jjpolfc at 72.27truept
\setbox\jjpolfbox=\vbox{\hbox{%
\jjpolfa\char0\char1\char2\char3\char4\char5}}
\jjpolfwd=\wd\jjpolfbox
\setbox\jjpolfbox=\hbox{\vbox{\hsize=\jjpolfwd
\parskip=0pt\offinterlineskip\parindent0pt
\hbox{\jjpolfa\char0\char1\char2\char3\char4\char5}
\hbox{\jjpolfa\char6\jjpolfb\char0\char1\char2\char3\char4}
\hbox{\jjpolfb\char5\char6\jjpolfc\char0\char1\char2\char3}}}
\ifx\parbox\undefined
    \def\setjjpolf{\box\jjpolfbox}
\else
    \def\setjjpolf{\parbox{\wd\jjpolfbox}{\box\jjpolfbox}}
\fi

\begin{figure*}
\hspace{.7in} \setjjpolf
\caption{The tangent space of a curve, and its dual (clipped to $x \in [-1,1]$)}
\label{fig:duality}
\end{figure*}

We review the polar solution in the next section,
and present our algorithm, including several examples, 
in Section~\ref{sec:point}.
The paper ends with some conclusions and suggestions for future work.
Throughout this paper, we shall work in projective 2-space.
We recall some of the important properties of projective 2-space 
in the following definition.

\begin{defn2}
\label{defn:proj}
{\bf Projective 2-space} $P^2$ is the space 
$\{(x_1,x_2,x_3) : x_i \in \Re, \mbox{ not all zero}\}$
under the equivalence relation $(x_1,x_2,x_3) = k(x_1,x_2,x_3),\ k \neq 0 \in \Re$.
The point $(x_1,x_2,x_3)$ in projective 2-space
is equivalent to the point $(\frac{x_1}{x_3},\frac{x_2}{x_3})$
in Cartesian 2-space.
Points of $P^2$ with $x_3=0$ are associated with points at infinity.
The 3 coordinates in projective 2-space are called {\bf homogeneous coordinates}.
We distinguish the third homogeneous coordinate by calling it the 
{\bf projective coordinate}.
\end{defn2}

The use of projective space is motivated by the fact that the equation 
of a line is invariant under multiplication by a constant:
the line $ax+by+c=0$ is equivalent to the line $kax+kby+kc=0$ 
for $k \neq 0$.

\section{Polar theory}
\label{sec:polar}

The classical mathematical method for finding the tangents of a curve $C$
through a point $P$ involves polar theory.

\begin{defn2}
Let $C$ be a plane algebraic curve defined by the polynomial $f(x_1,x_2,x_3)=0$
of degree $n$ and let $P = (p_1,p_2,p_3)$ be a point, where both $C$ and $P$ 
are expressed in projective 2-space.
The {\bf (first) polar of $P$ with respect to $C$} 
is an algebraic curve of degree $n-1$ defined by
\[
	p_1 \frac{\partial f}{\partial x_1} +
	p_2 \frac{\partial f}{\partial x_2} +
	p_3 \frac{\partial f}{\partial x_3} = 0
\]
or $P \cdot \nabla f = 0$ \cite{semple85}.	% p. 10-11
\end{defn2}

\begin{example}
The first polar of $P=(5,0,1)$ with respect to the circle 
$x_1^2 + x_2^2 - x_3^2 = 0$ is the line
$5(2x_1) - 2x_3 = 0$ or $x_1 = \frac{1}{5}$ in Cartesian space
(Figure~\ref{fig:polar}).
In this special case of a circle, the first polar can also be computed
using inversion: it is the line through the inverse $P'$ of $P$ 
and perpendicular to $PP'$.

The first polar of $(1,1,1)$ with respect to the quartic trisectrix curve
$(x^2 + y^2 - 2ax)^2 = a^2(x^2 + y^2)$ \cite{lawrence72} % p. 115
is the cubic curve $p(x,y) = 
(4-4a)(x^3 + xy^2) + 4(y^3 + x^2y) + (6a^2 - 12a)x^2 - (4a+2a^2)y^2 - 8axy + 6a^2x - 2a^2y = 0$.
\end{example}

\begin{theorem}
The intersections of $C$ and the first polar of $P$ with respect to $C$ 
define the locus of points of $C$ whose tangents intersect $P$ 
(Figure~\ref{fig:polar}).
\end{theorem}

\newbox\jjpopobox
\newdimen\jjpopowd
\font\jjpopoa=jjpopoa at 72.27truept
\setbox\jjpopobox=\vbox{\hbox{%
\jjpopoa\char0\char1\char2\char3}}
\jjpopowd=\wd\jjpopobox
\setbox\jjpopobox=\hbox{\vbox{\hsize=\jjpopowd
\parskip=0pt\offinterlineskip\parindent0pt
\hbox{\jjpopoa\char0\char1\char2\char3}
\hbox{\jjpopoa\char4\char5\char6\char7}
\hbox{\jjpopoa\char8\char9\char10\char11}}}
\ifx\parbox\undefined
    \def\setjjpopo{\box\jjpopobox}
\else
    \def\setjjpopo{\parbox{\wd\jjpopobox}{\box\jjpopobox}}
\fi

\begin{figure}[h]
\hspace{0in} \setjjpopo
\caption{The polar of a point with respect to a circle}
\label{fig:polar}
\end{figure}

Polar theory leads to the following algorithm for the computation of the 
tangents from a Bezier curve $C$ through a point $P$:
\begin{enumerate}
\item	Implicitize $C$ (i.e., translate from the parametric to the implicit
representation of $C$), yielding $f(x_1,x_2,x_3)=0$.
\item	Compute the first polar of $P$ with respect to $C$, say $g(x_1,x_2,x_3)=0$.
\item	Intersect the algebraic curves $f(x_1,x_2,x_3)=0$ and $g(x_1,x_2,x_3)=0$.
\end{enumerate}

The first step, implicitization,
involves resultants \cite{sederberg84,sederberg95}.
The implicitization of a Bezier spline is particularly challenging.
For example, the Bezier spline of Figure~\ref{fig:problem} has 69 segments,
yielding a set of algebraic curves.

For steps (2) and (3),
Sederberg \cite{sederberg89} has developed a robust and efficient algorithm
for the computation of polars of piecewise algebraic curves, algebraic curves
in the Bernstein basis, as well as the subsequent intersection of the 
algebraic curve and its polar.

In conclusion, the polar method is awkward for parametric Bezier curves, 
since it is fundamentally a method for implicit curves.

\section{A dual solution}
\label{sec:point}

In this section, we present our algorithm in dual space
for computing the tangents through a point.
This algorithm works directly with Bezier curves.
Our basic algorithm for computing the tangents from a curve $C$ through
a point $P$ is as follows:
%
\begin{description}
\item[(1)]	Dualize $P$ and the tangent space of $C$, to the line $P^*$
		and the tangential curve $C^*$.
\item[(2)]	Intersect the line $P^*$ and curve $C^*$ in dual space.
\item[(3)]	Map the intersections in dual space back to lines in primal space.
		These are the tangents of $C$ through $P$.
\end{description}
%
We shall now elaborate on the components of this algorithm
and add refinements to handle points at infinity.

We begin with a definition of our duality.
Although the most natural point-line duality is between the line $ax+by+c=0$
and the point $(a,b,c)$ in projective space \cite{fulton69}, % Fulton, p. 96
we choose two subtly different dualities
that map a different coefficient of the line equation to the projective coordinate.
This choice is motivated by the fact that lines dualized 
to points with zero projective coordinates
will be mapped to infinity (Definition~\ref{defn:proj}) and effectively lost by the duality.
%
\begin{defn2}
In the {\bf $a$-duality}, the line $ax+by+c=0$ is dual to the point $(b,c,a) \in P^2$.
In the {\bf $b$-duality}, the line $ax+by+c=0$ is dual to the point $(a,c,b) \in P^2$.
The {\bf \atang}  (resp., $b$-curve) of a plane curve $C(t)$
is the curve $C^*(t) \subset P^2$ where $C^*(t)$ is the $a$-dual
(resp., $b$-dual) of the tangent at $C(t)$.
Properties of the tangential curve are discussed in \cite{jj00}.
\end{defn2}

The $a$-dual of the horizontal line $by+c=0$ is $(b,c,0)$.
That is, horizontal lines are mapped to infinity
and the tangential $a$-curve is not a robust representation of horizontal tangents.
Consequently, we restrict the \atang to the representation of steep tangents, as follows.

\begin{defn2}
A curve's tangent space can be divided into shallow and steep tangents.
A tangent $ax+by+c=0$ is {\bf steep} if $|a| \geq |b|$
(its angle to the $x$-axis is greater than 45 degrees).
A tangent is {\bf shallow} if $|a| < |b|$.
\end{defn2}

\begin{lemma}
The \atang clipped by $x = \pm 1$,
$C^* \cap \{(x,y) : x \in [-1,1] \}$,
represents the steep tangents of the curve $C$.
\end{lemma}
\prf
Diagonal lines $ax \pm ay+c=0$ form the boundary between steep and shallow lines.
They dualize to $(\pm a,c,a) \in P^2$, or $x=\pm 1$ in Cartesian space,
for both $a$-duality and $b$-duality.
\QED

The top right of Figures~\ref{fig:ob2}-\ref{fig:ob3} illustrates
clipped \atangs (of the curves on the left of each figure).
It is simple to clip a curve by $x \in [-1,1]$.

The \btang is a perfect complement to the $a$-curve.
Although $b$-duality maps vertical lines to infinity,
the \btang clipped by $x = \pm 1$ is a robust representation of the 
shallow tangent space.
Thus, we compute the steep tangents through $P$ using the $a$-curves
and the shallow tangents through $P$ using the $b$-curve, as follows.
%
\begin{description}
\item[(1a)]	Dualize $P$ and the tangent space of $C$, using $a$-duality,
		to $P_a^*$ and $C_a^*$.  Clip $C_a^*$ to $x \in [-1,1]$.
\item[(1b)]	Dualize $P$ and the tangent space of $C$, using $b$-duality,
		to $P_b^*$ and $C_b^*$.  Clip $C_b^*$ to $x \in [-1,1]$.
\item[(2a)]	Intersect the line $P_a^*$ and the clipped curve $C_a^*$ 
		in $a$-dual space.
\item[(2b)]	Intersect the line $P_b^*$ and the clipped curve 
		$C_b^*$ in $b$-dual space.
\item[(3a)]	Dualize the intersections in $a$-dual space 
		back to steep tangents through $P$ in primal space.
\item[(3b)]	Dualize the intersections in $b$-dual space 
		back to shallow tangents through $P$ in primal space.
\end{description}

In summary, the tangents through $P$ are calculated by two line-curve
intersections, one in $a$-space and one in $b$-space.
The two intersections mesh perfectly to generate all of the 
tangents through $P$, in a mutually exclusive manner.
The clipping of the curves to $x \in [-1,1]$ makes the intersection very efficient,
since intersection with a Bezier curve involves subdivision,
which is related to the length of the curve.

Figures~\ref{fig:ob2}-\ref{fig:ob3} offer examples of the dual algorithm,
varying from simple to complicated.
In each case, primal space with the tangents of $C$ through $P$ is illustrated
on the left, $a$-dual space with the \atangs and dual line $P_a^*$ is illustrated
on the top right, and $b$-dual space with the \btangs and dual line $P_b^*$ 
is illustrated on the bottom right.

\newbox\jjpoegabox
\newdimen\jjpoegawd
\font\jjpoegaa=jjpoegaa at 72.27truept
\font\jjpoegab=jjpoegab at 72.27truept
\setbox\jjpoegabox=\vbox{\hbox{%
\jjpoegaa\char0\char1\char2\char3\char4\char5}}
\jjpoegawd=\wd\jjpoegabox
\setbox\jjpoegabox=\hbox{\vbox{\hsize=\jjpoegawd
\parskip=0pt\offinterlineskip\parindent0pt
\hbox{\jjpoegaa\char0\char1\char2\char3\char4\char5}
\hbox{\jjpoegaa\char6\char7\jjpoegab\char0\char1\char2\char3}
\hbox{\jjpoegab\char4\char5\char6\char7\char8\char9}}}
\ifx\parbox\undefined
    \def\setjjpoega{\box\jjpoegabox}
\else
    \def\setjjpoega{\parbox{\wd\jjpoegabox}{\box\jjpoegabox}}
\fi

\begin{figure*}
\hspace{.8in} \setjjpoega
\caption{A simple example of our algorithm}
\label{fig:ob2}
\end{figure*}

\newbox\jjpoegbbox
\newdimen\jjpoegbwd
\font\jjpoegba=jjpoegba at 72.27truept
\font\jjpoegbb=jjpoegbb at 72.27truept
\setbox\jjpoegbbox=\vbox{\hbox{%
\jjpoegba\char0\char1\char2\char3\char4\char5}}
\jjpoegbwd=\wd\jjpoegbbox
\setbox\jjpoegbbox=\hbox{\vbox{\hsize=\jjpoegbwd
\parskip=0pt\offinterlineskip\parindent0pt
\hbox{\jjpoegba\char0\char1\char2\char3\char4\char5}
\hbox{\jjpoegba\char6\char7\jjpoegbb\char0\char1\char2\char3}
\hbox{\jjpoegbb\char4\char5\char6\char7\char8\char9}}}
\ifx\parbox\undefined
    \def\setjjpoegb{\box\jjpoegbbox}
\else
    \def\setjjpoegb{\parbox{\wd\jjpoegbbox}{\box\jjpoegbbox}}
\fi

\begin{figure*}
\hspace{.8in} \setjjpoegb
\caption{A second example}
\label{fig:ob1}
\end{figure*}

\newbox\jjpoegcbox
\newdimen\jjpoegcwd
\font\jjpoegca=jjpoegca at 72.27truept
\font\jjpoegcb=jjpoegcb at 72.27truept
\setbox\jjpoegcbox=\vbox{\hbox{%
\jjpoegca\char0\char1\char2\char3\char4\char5}}
\jjpoegcwd=\wd\jjpoegcbox
\setbox\jjpoegcbox=\hbox{\vbox{\hsize=\jjpoegcwd
\parskip=0pt\offinterlineskip\parindent0pt
\hbox{\jjpoegca\char0\char1\char2\char3\char4\char5}
\hbox{\jjpoegca\char6\char7\jjpoegcb\char0\char1\char2\char3}
\hbox{\jjpoegcb\char4\char5\char6\char7\char8\char9}}}
\ifx\parbox\undefined
    \def\setjjpoegc{\box\jjpoegcbox}
\else
    \def\setjjpoegc{\parbox{\wd\jjpoegcbox}{\box\jjpoegcbox}}
\fi

\begin{figure*}
\hspace{.8in} \setjjpoegc
\caption{A more complicated example}
\label{fig:ob3}
\end{figure*}

The individual steps of the algorithm, dualization and intersection,
require some elaboration.
A crucial fact is that the tangential curve of a 
Bezier curve can be expressed as a rational Bezier curve.
This simplifies the intersection in dual space,
since intersection of Bezier curves is well understood.

\begin{theorem}
\label{thm:rationalduala}
Let $C(t)$ be a plane Bezier curve of degree $n$ 
with control points $\{ (b_{i,1}, b_{i,2}) \}_{i=0}^n$ over the parameter interval $[t_1,t_2]$.
The \atang $C_a^*(t)$ is a rational Bezier curve of degree $2n-1$ 
over $[t_1,t_2]$ with weights $\{w_k\}_{k=0}^{2n-1}$ where: 
\begin{displaymath}
\scriptsize{
w_k = \alpha
\sum_{j=\mbox{max}(0,k-n)}^{\mbox{min}(n-1,k)} 
\tinychoice{n-1}{j} \tinychoice{n}{k-j} (b_{j,2} - b_{j+1,2})
}
\end{displaymath}
%
and control points $\{c_k\}_{k=0}^{2n-1}$ where:
\begin{displaymath}
\tiny{
c_k = \frac{\alpha}{w_k} 
\left(
\begin{array}{l}
	\displaystyle{\sum_{j=\mbox{\tiny{max}}(0,k-n)}^{\mbox{\tiny{min}}(n-1,k)}}
	\tinychoice{n-1}{j} \tinychoice{n}{k-j} (b_{j+1,1} - b_{j,1})\\
	\displaystyle{\sum_{\begin{array}{c} \mbox{\tiny{$0 \leq i \leq n-1$}} \\ 
			     \mbox{\tiny{$0 \leq j \leq n$}} \\ 
			     \mbox{\tiny{$i+j=k$}}
			     \end{array}}}
\tinychoice{n-1}{i} \tinychoice{n}{j} \beta_{i,j}
\end{array}
\right)
}
\end{displaymath}
where $\alpha = \tiny{\frac{n}{(t_2 - t_1) \tinychoice{2n-1}{k}}}$,
and $\beta_{i,j} = (b_{i+1,2} - b_{i,2}) b_{j,1} - (b_{i+1,1} - b_{i,1}) b_{j,2}$.
\end{theorem}
\prf
We omit the proof due to lack of space and its technical nature.
The reader is referred to \cite{jj00} for a full proof.
\QED

The representation of the \btang as a rational Bezier curve is analogous.

The line-curve intersection of steps (2a) and (2b) is the intersection
of two Bezier curves.
It is simple to represent a line as a Bezier curve of degree 1,
and intersection with a line is a particularly simple case of Bezier
curve intersection.
The intersection of Bezier curves is well understood and efficient \cite{sederberg86}.
We use the classical subdivision approach, where the two curves are recursively
subdivided into shorter segments until their intersection can be adequately
approximated by the intersection of two lines.

We end this section by explaining why we did not use the duality 
$ax+by+c=0 \rightarrow (a,b,c)$.
It maps lines through the origin to infinity.
Since a horizontal line can pass through the origin,
as can a vertical line, this duality does not cooperate well with either of
the other dualities.

\section{Conclusions}
\label{sec:comparison}

In this paper, we have studied a new approach to computing the tangents
through a point, a basic visibility operation.
An algebraic method that reduces to the solution of a univariate equation of degree $2n-1$
was replaced by a geometric method that reduces to the intersection
of a line and Bezier curve of degree $2n-1$ in dual space.
Although, in the context of this problem, the two approaches are similar,
the dual method has the better capacity for generalization and
other important visibility problems, especially for motion and lighting in 
3-space, can be attacked anew using the dual approach.
By studying the dual solution for the tangents through a point,
we see the development of this new method as it mutates away from the
algebraic solution.

The simplicity of the dual method of this paper makes it feasible
for many visibility computations from a point to be lifted from an approximate
polygonal world to a more realistic curved world.
The next step is the application of this smooth tangent operation 
to the construction of smooth visibility graphs, shortest path motion
amongst smooth obstacles, lighting of smooth environments,
and other visibility analysis in a smooth curved world.

\bibliographystyle{plain}
\begin{thebibliography}{1}

\bibitem{farin97}
Farin, G. (1997)
Curves and Surfaces for CAGD: A Practical Guide (4th edition).
Academic Press (New York).

\bibitem{fulton69}
Fulton, W. (1969)
Algebraic Curves.
Benjamin/Cummings (Menlo Park).

\bibitem{jj00}
Johnstone, J.K. (2001)
A Parametric Solution to Common Tangents.
International Conference on Shape Modelling and Applications (SMI 2001),
to appear.

\bibitem{kim88}
Kim, M.-S. (1988)
Motion Planning with Geometric Models.
Ph.D. Thesis, Purdue University, Dept. of Computer Science.

\bibitem{lawrence72}
Lawrence, J.D. (1972)
A Catalog of Special Plane Curves.
Dover (New York).

\bibitem{sederberg84}
Sederberg, T. and D. Anderson and R. Goldman (1984)
Implicit Representation of Parametric Curves and Surfaces.
Computer Vision, Graphics, and Image Processing 28, 72--84.

\bibitem{sederberg86}
Sederberg, T. and S. Parry (1986)
Comparison of three curve intersection algorithms.
Computer Aided Design 18, 58--63.

\bibitem{sederberg89}
Sederberg, T. (1989)
Algorithm for algebraic curve intersection.
Computer Aided Design 21(9), 547--554.

\bibitem{sederberg95}
Sederberg, T. and F. Chen (1995)
Implicitization using Moving Curves and Surfaces.
SIGGRAPH '95, 301--308.

\bibitem{semple85}
Semple, J. and L. Roth (1985)
Introduction to Algebraic Geometry.
Oxford University Press (Oxford).

\end{thebibliography}

\end{document}

