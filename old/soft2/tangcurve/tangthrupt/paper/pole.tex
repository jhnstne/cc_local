\documentclass[11pt]{article}
\newif\ifVideo
\Videofalse
\newif\ifTalk
\Talkfalse
\newif\ifPole
\Polefalse
\input{header}
\newcommand{\plucker}{Pl\"{u}cker\ }
\newcommand{\tang}{tangential curve\ }
\newcommand{\tangs}{tangential curves\ }
\newcommand{\Tang}{Tangential curve\ }
\newcommand{\atang}{tangential $a$-curve\ }
\newcommand{\btang}{tangential $b$-curve\ }
\newcommand{\ctang}{tangential $c$-curve\ }
\newcommand{\atangs}{tangential $a$-curves\ }
\newcommand{\btangs}{tangential $b$-curves\ }
\newcommand{\ctangs}{tangential $c$-curves\ }
\DoubleSpace

\setlength{\oddsidemargin}{0pt}
\setlength{\topmargin}{-.2in}	% should be 0pt for 1in
% \setlength{\headsep}{.5in}
\setlength{\textheight}{8.5in}
\setlength{\textwidth}{6.5in}
\setlength{\columnsep}{5mm}	% width of gutter between columns
% \markright{Tangents through a point: \today \hfill}
% \pagestyle{myheadings}
% -----------------------------------------------------------------------------

\title{Smooth visibility from a point}
% Efficient visibility from a point in a smooth world}
% The tangents of a curve through a point
\author{J.K. Johnstone\thanks{Geometric Modeling Lab, 
	Computer and Information Sciences, University of Alabama at Birmingham, 
	University Station, Birmingham, AL 35294.}}

\begin{document}
\maketitle

\begin{abstract}
We propose a solution in dual space to the problem of
computing the tangents of a plane curve that pass through a point,
an important problem in the analysis of visibility.
Our solution reduces to the intersection of a line with a Bezier curve
in dual space.
It is equivalent in complexity to the standard parametric solution,
% which computes the tangents of $C(t)$ through $P$ as the roots of $(C(t) - P) . N(t) = 0$, 
but introduces an alternative, geometric interpretation of the problem.
This new characterization has promise for the solution of more general problems in visibility.
The dual solution is superior to a solution using polar theory,
which reduces to the intersection of algebraic curves after implicitization.
\end{abstract}

Keywords: visibility, tangents through a point, duality, polar.

\section{Introduction}

The tangents of a plane curve that meet a point $P$ (Figure~\ref{fig:problem})
define important boundaries of visibility from $P$
for robotics, graphics, and geometric analysis.
A shortest path from $P$ will begin along one of these tangents 
(Figure~\ref{fig:motion}).
% (unless the destination is visible from $P$).
A point light at $P$ will cast shadows bounded by these tangents,
when the curve is interpreted as an opaque obstacle (Figure~\ref{fig:shadow}).
The 2-dimensional silhouette from the viewpoint $P$
is defined by the visible points of tangency (Figure~\ref{fig:problem}).

\newbox\jjpoprbox
\newdimen\jjpoprwd
\font\jjpopra=jjpopra at 72.27truept
\setbox\jjpoprbox=\vbox{\hbox{%
\jjpopra\char0\char1\char2}}
\jjpoprwd=\wd\jjpoprbox
\setbox\jjpoprbox=\hbox{\vbox{\hsize=\jjpoprwd
\parskip=0pt\offinterlineskip\parindent0pt
\hbox{\jjpopra\char0\char1\char2}
\hbox{\jjpopra\char3\char4\char5}
\hbox{\jjpopra\char6\char7\char8}}}
\ifx\parbox\undefined
    \def\setjjpopr{\box\jjpoprbox}
\else
    \def\setjjpopr{\parbox{\wd\jjpoprbox}{\box\jjpoprbox}}
\fi


\begin{figure}
\hspace{2.1in} \setjjpopr
\caption{Tangents through a point}
\label{fig:problem}
% dual -p 0 -1 ob3.rawctr
\end{figure}

\newbox\jjposhbox
\newdimen\jjposhwd
\font\jjposha=jjposha at 72.27truept
\setbox\jjposhbox=\vbox{\hbox{%
\jjposha\char0\char1}}
\jjposhwd=\wd\jjposhbox
\setbox\jjposhbox=\hbox{\vbox{\hsize=\jjposhwd
\parskip=0pt\offinterlineskip\parindent0pt
\hbox{\jjposha\char0\char1}
\hbox{\jjposha\char2\char3}}}
\ifx\parbox\undefined
    \def\setjjposh{\box\jjposhbox}
\else
    \def\setjjposh{\parbox{\wd\jjposhbox}{\box\jjposhbox}}
\fi


\begin{figure}
\hspace{2.6in} \setjjposh
\caption{A shortest path}
\label{fig:motion}
% an entire shortest path from A to B
% smoothpath.rgb from vgraph/img (could try regenerating vgraph datasets)
% (../img/fig1.rgb is another possibility, although inferior since it
% also shows common tangents)
\end{figure}

\newbox\jjpocabox
\newdimen\jjpocawd
\font\jjpocaa=jjpocaa at 72.27truept
\setbox\jjpocabox=\vbox{\hbox{%
\jjpocaa\char0\char1\char2}}
\jjpocawd=\wd\jjpocabox
\setbox\jjpocabox=\hbox{\vbox{\hsize=\jjpocawd
\parskip=0pt\offinterlineskip\parindent0pt
\hbox{\jjpocaa\char0\char1\char2}
\hbox{\jjpocaa\char3\char4\char5}}}
\ifx\parbox\undefined
    \def\setjjpoca{\box\jjpocabox}
\else
    \def\setjjpoca{\parbox{\wd\jjpocabox}{\box\jjpocabox}}
\fi


\begin{figure}
\hspace{2in} \setjjpoca
\caption{Casting a shadow}
\label{fig:shadow}
% shaded umbra from a point light
\end{figure}

The solution to the discrete polygonal version of this problem,
where the curve is replaced by a polygon, is straightforward
(Figure~\ref{fig:polygon}).
For each vertex of the polygon, consider the outward-pointing normals
of its two adjacent edges.
Vertices such that one outward-pointing normal points towards the point $P$
and the other points away become the discrete analog of the points of tangency
on the curve.
A smooth solution using curves is more involved, 
yet it is preferable for its precision and, 
if properly implemented, its improved efficiency over a densely sampled polygon.
We are especially interested in the solution of our problem for Bezier
curves, since they are the dominant modeling tool for curves.

\newbox\jjpolybox
\newdimen\jjpolywd
\font\jjpolya=jjpolya at 72.27truept
\setbox\jjpolybox=\vbox{\hbox{%
\jjpolya\char0\char1\char2}}
\jjpolywd=\wd\jjpolybox
\setbox\jjpolybox=\hbox{\vbox{\hsize=\jjpolywd
\parskip=0pt\offinterlineskip\parindent0pt
\hbox{\jjpolya\char0\char1\char2}
\hbox{\jjpolya\char3\char4\char5}}}
\ifx\parbox\undefined
    \def\setjjpoly{\box\jjpolybox}
\else
    \def\setjjpoly{\parbox{\wd\jjpolybox}{\box\jjpolybox}}
\fi


\begin{figure}
\hspace{1.9in} \setjjpoly
\caption{The polygonal problem}
\label{fig:polygon}
\end{figure}

The classical solution to the tangents of a curve $C$
through a point $P$ uses polar theory \cite{semple85}.
The points of the curve whose tangents intersect $P$ are 
the intersections of two algebraic curves:
the curve $C$ and the polar of $P$ with respect to $C$
(Figure~\ref{fig:polar}).

The standard parametric solution (e.g., \cite{kim88})
solves for the points $C(t) = (x(t),y(t))$ of the curve whose normal $N(t)$
is orthogonal to the line to $P = (p_1,p_2)$ (Figure~\ref{fig:standard}):
\begin{equation}
\label{eq:standard}
 	(C(t) - P) \cdot N(t) = 0
\end{equation}
or
\[
	f(t) = (x(t) - p_1, y(t) - p_2) \cdot (-y'(t), x'(t)) = 0
\]
This is a univariate equation of degree $2n-1$ in $t$,
where $n$ is the degree of $C(t)$.

\newbox\jjpostbox
\newdimen\jjpostwd
\font\jjposta=jjposta at 72.27truept
\setbox\jjpostbox=\vbox{\hbox{%
\jjposta\char0\char1\char2}}
\jjpostwd=\wd\jjpostbox
\setbox\jjpostbox=\hbox{\vbox{\hsize=\jjpostwd
\parskip=0pt\offinterlineskip\parindent0pt
\hbox{\jjposta\char0\char1\char2}
\hbox{\jjposta\char3\char4\char5}}}
\ifx\parbox\undefined
    \def\setjjpost{\box\jjpostbox}
\else
    \def\setjjpost{\parbox{\wd\jjpostbox}{\box\jjpostbox}}
\fi


\begin{figure}
\hspace{1.9in} \setjjpost
\caption{Solving for normals orthogonal to the line to $P$}
\label{fig:standard}
\end{figure}

In this paper, we propose a solution to the tangency problem
that reduces to the intersection of a line and a Bezier curve in dual space.
This compares favourably with the polar solution.
A curve-curve intersection is replaced by a line-curve intersection.
Also, the dual of a Bezier curve
can be easily represented as a rational Bezier curve (Theorem~\ref{thm:rationalduala}),
while the polar of a curve cannot.
Consequently, the intersection of implicit algebraic curves in the polar solution
is replaced by the simpler intersection of parametric Bezier curves.

The dual solution is very similar in efficiency to the standard parametric
solution (\ref{eq:standard}), since the intersection of a line and a plane
curve of degree $2n-1$ is comparable to the solution of a univariate equation
of degree $2n-1$.
The major difference is in point of view and generalizability.
The dual method introduces a new paradigm with great potential for the
solution of visibility problems in dual space,
in which an algebraic solution is replaced by a geometric solution.
In other work, we will show that it can lead to a better solution
for the common tangents of two curves \cite{jj00}
(and we shall appeal to some of the results from this paper).
We are also presently working on a solution to the common tangent planes of two
surfaces that has no comparable solution under the algebraic interpretation.
By studying the dual solution for the simpler problem of 
the tangents through a point,
we shall establish a template for the dual space method
and clearly distinguish the geometric and algebraic approaches.

The basic argument of our solution in dual space is as follows.
In dual space, points become lines and lines become points.
In particular, the point $P$ dualizes to the line $P^*$ and 
the tangent lines of a curve dualize to the points of a so-called 
tangential curve (Figure~\ref{fig:duality}).
% the family of tangent lines of a curve's tangent space dualizes to a family
% of points, or a curve, in dual space.
% We call this image of a tangent space a tangential curve.
The intersections of the line $P^*$ with the tangential curve
encode, in dual space, the tangent lines through $P$
(Figure~\ref{fig:ob2}).
In particular, an intersection point maps back to a line, which
must be a tangent (since the intersection lies on the tangential curve)
that contains $P$ (since the intersection also lies on $P^*$).

\newbox\jjpolfbox
\newdimen\jjpolfwd
\font\jjpolfa=jjpolfa at 72.27truept
\font\jjpolfb=jjpolfb at 72.27truept
\font\jjpolfc=jjpolfc at 72.27truept
\setbox\jjpolfbox=\vbox{\hbox{%
\jjpolfa\char0\char1\char2\char3\char4\char5}}
\jjpolfwd=\wd\jjpolfbox
\setbox\jjpolfbox=\hbox{\vbox{\hsize=\jjpolfwd
\parskip=0pt\offinterlineskip\parindent0pt
\hbox{\jjpolfa\char0\char1\char2\char3\char4\char5}
\hbox{\jjpolfa\char6\jjpolfb\char0\char1\char2\char3\char4}
\hbox{\jjpolfb\char5\char6\jjpolfc\char0\char1\char2\char3}}}
\ifx\parbox\undefined
    \def\setjjpolf{\box\jjpolfbox}
\else
    \def\setjjpolf{\parbox{\wd\jjpolfbox}{\box\jjpolfbox}}
\fi


\begin{figure}
\hspace{.4in} \setjjpolf
\caption{The tangent space of a curve, and its dual (clipped to $x \in [-1,1]$)}
\label{fig:duality}
% dual ob2.rawctr (one obstacle, line field, just b-curves showing)
% simplelinefield.rgb
\end{figure}

\Comment{
Consequently, the tangent lines of $C$ through $P$ 
become the points of $C^*$ that intersect $P^*$.
That is, the tangency problem reduces to the intersection of a line and curve.
}

\Comment{
The problem of computing the tangents of a curve through a point is related 
to the problem of computing the common tangents of two curves,
which we have also solved in dual space \cite{jj00},
and we shall appeal to some of its results.
The tangents through a point introduce interesting issues from
the special nature of the dualization of a point and the
relationship with polar theory.
}
\Comment{
It is a special case where one curve has degenerated to a point.
The associated curve in dual space degenerates to a line.
We have developed a parametric solution in dual space for the common tangent
problem \cite{jj00}.
Many of the same problems are encountered in this paper, and we shall
appeal to its methods.
In particular, this paper shows how to dualize a tangent space to a rational
Bezier curve.
We shall also follow this paper's use of two versions of dual space to handle
intersections at infinity, and clip these curves to remove redundancy.
The main difference is that the intersection of two Bezier curves in dual space
is replaced by the intersection of a line and a Bezier curve in dual space.
}

We present the polar solution in the next section,
and our algorithm in Section~\ref{sec:point}, including several examples.
The paper ends with some conclusions and suggestions for future work.
Throughout this paper, we shall work in projective 2-space.
We recall some of the important properties of projective 2-space 
in the following definition.

\begin{defn2}
\label{defn:proj}
{\bf Projective 2-space} $P^2$ is the space 
$\{(x_1,x_2,x_3) : x_i \in \Re, \mbox{ not all zero}\}$
under the equivalence relation $(x_1,x_2,x_3) = k(x_1,x_2,x_3),\ k \neq 0 \in \Re$.
The point $(x_1,x_2,x_3)$ in projective 2-space
is equivalent to the point $(\frac{x_1}{x_3},\frac{x_2}{x_3})$
in Cartesian 2-space.
Points of $P^2$ with $x_3=0$ are associated with points at infinity.
The 3 coordinates in projective 2-space are called {\bf homogeneous coordinates}.
We distinguish the third homogeneous coordinate by calling it the 
{\bf projective coordinate}.
\end{defn2}

The use of projective space is motivated by the fact that the equation 
of a line is invariant under multiplication by a constant:
the line $ax+by+c=0$ is equivalent to the line $kax+kby+kc=0$ 
for $k \neq 0$.

\section{Polar theory}
\label{sec:polar}

The classical mathematical method for finding the tangents of a curve $C$
through a point $P$ involves polar theory.

\begin{defn2}
Let $C$ be a plane algebraic curve defined by the polynomial $f(x_1,x_2,x_3)=0$
of degree $n$ and let $P = (p_1,p_2,p_3)$ be a point, where both $C$ and $P$ 
are expressed in projective 2-space.
The {\bf (first) polar of $P$ with respect to $C$} 
is an algebraic curve of degree $n-1$ defined by
\[
	p_1 \frac{\partial f}{\partial x_1} +
	p_2 \frac{\partial f}{\partial x_2} +
	p_3 \frac{\partial f}{\partial x_3} = 0
\]
or $P \cdot \nabla f = 0$ \cite{semple85}.	% p. 10-11
\end{defn2}

\begin{example}
The first polar of $P=(5,0,1)$ with respect to the circle 
$x_1^2 + x_2^2 - x_3^2 = 0$ is the line
$5(2x_1) - 2x_3 = 0$ or $x_1 = \frac{1}{5}$ in Cartesian space
(Figure~\ref{fig:polar}).
In this special case of a circle, the first polar can also be computed
using inversion: it is the line through the inverse $P'$ of $P$ 
and perpendicular to $PP'$.

The first polar of $(1,1,1)$ with respect to the quartic trisectrix curve
$(x^2 + y^2 - 2ax)^2 = a^2(x^2 + y^2)$ \cite{lawrence72} % p. 115
is the cubic curve $p(x,y) = 
(4-4a)(x^3 + xy^2) + 4(y^3 + x^2y) + (6a^2 - 12a)x^2 - (4a+2a^2)y^2 - 8axy + 6a^2x - 2a^2y = 0$.
\end{example}

\begin{theorem} % \cite{semple85}
The intersections of $C$ and the first polar of $P$ with respect to $C$ 
define the locus of points of $C$ whose tangents intersect $P$ 
(Figure~\ref{fig:polar}).
\end{theorem}

\newbox\jjpopobox
\newdimen\jjpopowd
\font\jjpopoa=jjpopoa at 72.27truept
\setbox\jjpopobox=\vbox{\hbox{%
\jjpopoa\char0\char1\char2\char3}}
\jjpopowd=\wd\jjpopobox
\setbox\jjpopobox=\hbox{\vbox{\hsize=\jjpopowd
\parskip=0pt\offinterlineskip\parindent0pt
\hbox{\jjpopoa\char0\char1\char2\char3}
\hbox{\jjpopoa\char4\char5\char6\char7}
\hbox{\jjpopoa\char8\char9\char10\char11}}}
\ifx\parbox\undefined
    \def\setjjpopo{\box\jjpopobox}
\else
    \def\setjjpopo{\parbox{\wd\jjpopobox}{\box\jjpopobox}}
\fi


\begin{figure}[h]
\hspace{1.5in} \setjjpopo
\caption{A simple example: the first polar of a point with respect to a circle}
% file: polar.showcase
% unit circle and (5,0) with tangents to circle and polar line x=1/5
\label{fig:polar}
\end{figure}

Polar theory leads to the following algorithm for the computation of the 
tangents from a Bezier curve $C$ through a point $P$:
\begin{enumerate}
\item	Implicitize $C$ (i.e., translate from the parametric to the implicit
representation of $C$), yielding $f(x_1,x_2,x_3)=0$.
\item	Compute the first polar of $P$ with respect to $C$, say $g(x_1,x_2,x_3)=0$.
\item	Intersect the algebraic curves $f(x_1,x_2,x_3)=0$ and $g(x_1,x_2,x_3)=0$.
\end{enumerate}

Implicitization involves resultants \cite{sederberg84,sederberg95}.
The implicitization of a Bezier spline is particularly challenging,
since implicitization must be done for each segment.
(For example, the Bezier spline of Figure~\ref{fig:problem} has 69 segments.)
Thus, for a Bezier spline, the implicitization of step (1) yields 
a set of algebraic curves
and the intersection of step (3) is actually a series of intersections.
Moreover, each Bezier segment must be translated from the Bernstein basis
to the power basis in preparation for implicitization.
Finally, the intersection of two algebraic curves involves resultants
and many solutions of univariate equations \cite{kajiya82}. 
% \cite{bajajkim?} (or parameterization (for lower degree)?)

In short, the polar method is expensive, and awkward in its treatment
of parametric Bezier curves, since it is fundamentally a method
for implicit curves.

\Comment{
Both the polar and dual methods for computing the tangents of a curve $C$
through a point reduce to a transformation (polar or dual) of the curve 
followed by an intersection.
However, the polar method works with implicit curves
while the dual solution works directly with parametric Bezier curves,
which has many consequences.
These consequences are particularly visible when working with Bezier curves,
the most common form of the problem.
The polar method demands implicitization of the curve before the polar is computed,
while the dual method does not.
Curves with multiple segments are handled awkwardly by the polar solution
but elegantly by the dual solution.
And the simpler intersection of Bezier curves can be used rather than
intersection of algebraic curves.
Finally, a line-curve intersection is simpler than a curve-curve intersection.
}

\Comment{
FOLLOWING IS INCORRECT: only nonrational curves can have multiple components,
and we are inherently dealing with rational (Bezier) curves.

We must also deal with the fact that an implicit curve may represent
a superset of the associated parametric curve, if the implicit curve has
several components (such as $y^2 - x^3 + x = 0$ \cite{fulton69} % p. 8
or the devil's curve $y^4 - 4y^2 - x^4 + 9x^2 = 0$ \cite{lawrence72})
and the parametric curve represents one of these components.
Thus, the intersections must be filtered to verify their correctness.
}

\section{A dual solution}
\label{sec:point}

In this section, we present our algorithm in dual space
for computing the tangents through a point.
This algorithm works directly with Bezier curves.
Our basic algorithm for computing the tangents from a curve $C$ through
a point $P$ is as follows:
%
\begin{description}
\item[(1)]	Dualize $P$ and the tangent space of $C$, to the line $P^*$
		and the tangential curve $C^*$.
\item[(2)]	Intersect the line $P^*$ and curve $C^*$ in dual space.
\item[(3)]	Map the intersections in dual space back to lines in primal space.
		These are the tangents of $C$ through $P$.
\end{description}
%
We shall now elaborate on the components of this algorithm
and add refinements to handle points at infinity.

We begin with a definition of duality.
Although the most natural point-line duality is between the line $ax+by+c=0$
and the point $(a,b,c)$ in projective space \cite{fulton69}, % Fulton, p. 96
we choose two different dualities
that map a different coefficient of the line equation to the projective coordinate.
Notice that the lines dualized to points with zero projective coordinates
will be mapped to infinity (Definition~\ref{defn:proj}), and effectively lost by the duality.
\Comment{
Since it is inevitable that some lines map to infinity,
and these lines may correspond to the desired tangents,
we need to use two dualities (which map different lines to infinity)
and coordinate them carefully.
}
\begin{defn2}
In the {\bf $a$-duality}, the line $ax+by+c=0$ is dual to the point $(b,c,a) \in P^2$.
In the {\bf $b$-duality}, the line $ax+by+c=0$ is dual to the point $(a,c,b) \in P^2$.
The {\bf \atang}  (resp., $b$-curve) of a plane curve $C(t)$
is the curve $C^*(t) \subset P^2$ where $C^*(t)$ is the $a$-dual
(resp., $b$-dual) of the tangent at $C(t)$.
Properties of the tangential curve are discussed in \cite{jj00}.
\end{defn2}

\Comment{
\begin{defn2}
The $c$-dual of the line $ax+by+c=0$ is the point $(a,b,c) \in P^2$ in projective space.
The $a$-dual of the line $ax+by+c=0$ is the point $(b,c,a) \in P^2$.
The $b$-dual of the line $ax+by+c=0$ is the point $(a,c,b) \in P^2$.

TECHNICALLY DON'T NEED TO WORRY ABOUT $c$-DUAL,
BUT IT IMPROVES THE CLARITY AND ALSO ALLOWS US TO EXPLAIN 
WHY WE DON'T USE THE CLASSICAL CHOICE OF THE $c$-DUAL.

The $c$-dual of the point $(a,b,c) \in P^2$ is the line $ax+by+cw=0$
(where $w$ is the projective coordinate).
The $a$-dual of the point $(a,b,c) \in P^2$ is the line $bx+cy+aw=0$.
% since $a$ is associated with the projective coordinate.
The $b$-dual of the point $(a,b,c) \in P^2$ is the line $ax+cy+bw=0$.

Define $a$-space.

The tangential $\beta$-curve $C^* \subset P^2$ of a plane curve $C$ is the 
curve $C^*(t)$
where $C^*(t)$ is the $\beta$-dual of the tangent at $C(t)$.
Here, $\beta$ may be $a$, $b$, or $c$.
\end{defn2}
}

The $a$-dual of an horizontal line $by+c=0$ is $(b,c,0)$.
That is, horizontal lines are mapped to infinity.
This means that the tangential $a$-curve is not a robust representation of horizontal tangents.
Consequently, we restrict the \atang to the representation of 'steep' tangents.

\Comment{
while vertical lines map to infinity under $b$-duality.
We shall combine the two dualities so that all lines are mapped to a finite point
in at least one dual space.
To avoid overlap (i.e., to ensure that all tangents map to a finite point
in {\bf exactly} one dual space), we map half the tangent space to one dual space
and half to the other, using clipping.
}

\begin{defn2}
A curve's tangent space can be divided into shallow and steep tangents.
A tangent $ax+by+c=0$ is {\bf steep} if $|a| \geq |b|$
(its angle to the $x$-axis is greater than 45 degrees).
A tangent is {\bf shallow} if $|a| < |b|$.
\end{defn2}

\begin{lemma}
The \atang clipped by $x = \pm 1$,
$C^* \cap \{(x,y) : x \in [-1,1] \}$,
represents the steep tangents of the curve $C$.
\end{lemma}
\prf
Diagonal lines $ax \pm ay+c=0$ form the boundary between steep and shallow lines.
They dualize to $(\pm a,c,a) \in P^2$, or $x=\pm 1$ in Cartesian space,
for both $a$-duality and $b$-duality.
\QED

The top right of Figures~\ref{fig:ob2}-\ref{fig:ob3} illustrates
clipped \atangs (of the curves on the left of each figure).
It is simple to clip a curve by $x \in [-1,1]$.

The \btang is a perfect complement to the $a$-curve.
Although $b$-duality maps vertical lines to infinity,
the \btang clipped by $x = \pm 1$ is a robust representation of the 
shallow tangent space.
Thus, we compute the steep tangents through $P$ using the $a$-curves
and the shallow tangents through $P$ using the $b$-curve, as follows.
%
\begin{description}
\item[(1a)]	Dualize $P$ and the tangent space of $C$, using $a$-duality,
		to $P_a^*$ and $C_a^*$.  Clip $C_a^*$ to $x \in [-1,1]$.
\item[(1b)]	Dualize $P$ and the tangent space of $C$, using $b$-duality,
		to $P_b^*$ and $C_b^*$.  Clip $C_b^*$ to $x \in [-1,1]$.
\item[(2a)]	Intersect the line $P_a^*$ and the clipped curve $C_a^*$ 
		in $a$-dual space.
\item[(2b)]	Intersect the line $P_b^*$ and the clipped curve 
		$C_b^*$ in $b$-dual space.
\item[(3a)]	Map the intersections in dual space 
		back to steep tangents through $P$ in primal space.
\item[(3b)]	Map the intersections in dual space 
		back to shallow tangents through $P$ in primal space.
\end{description}

\Comment{
If we are not careful, some tangents will be missed as they are mapped
to infinity (Example~\ref{eg:infinity}).
We refine the algorithm to guarantee that all tangents are computed robustly:
more-vertical tangents (DEFINE) through $P$ are computed in $a$-space
while more-horizontal tangents through $P$ are computed in $b$-space.
Since we only need to represent more-vertical tangents in $a$-space,
we can clip the $a$-curves at $x = \pm 1$, the dual image of diagonal lines
$x = \pm y + c$ (the boundary between more horizontal and more vertical lines).
Similarly for $b$-space and more-horizontal lines.
}

In summary, the tangents through $P$ are calculated by two line-curve
intersections, one in $a$-space and one in $b$-space.
The two intersections mesh perfectly to generate all of the 
tangents through $P$, in a mutually exclusive manner.
The clipping of the curves to $x \in [-1,1]$ makes the intersection very efficient,
since intersection with a Bezier curve involves subdivision,
which is related to the length of the curve.

Figures~\ref{fig:ob2}-\ref{fig:ob3} offer examples of the dual algorithm,
varying from simple to complicated.
In each case, primal space with the tangents of $C$ through $P$ is illustrated
on the left, $a$-dual space with the \atangs and dual line $P_a^*$ is illustrated
on the top right, and $b$-dual space with the \btangs and dual line $P_b^*$ 
is illustrated on the bottom right.
	% ob2pole.rgb; dual -p -.2 .05 ob2.rawctr
	% ob1pole.rgb; dual ob1.rawctr
	% ob3pole.rgb; dual -p 0 -1 ob3.rawctr

\newbox\jjpoegabox
\newdimen\jjpoegawd
\font\jjpoegaa=jjpoegaa at 72.27truept
\font\jjpoegab=jjpoegab at 72.27truept
\setbox\jjpoegabox=\vbox{\hbox{%
\jjpoegaa\char0\char1\char2\char3\char4\char5}}
\jjpoegawd=\wd\jjpoegabox
\setbox\jjpoegabox=\hbox{\vbox{\hsize=\jjpoegawd
\parskip=0pt\offinterlineskip\parindent0pt
\hbox{\jjpoegaa\char0\char1\char2\char3\char4\char5}
\hbox{\jjpoegaa\char6\char7\jjpoegab\char0\char1\char2\char3}
\hbox{\jjpoegab\char4\char5\char6\char7\char8\char9}}}
\ifx\parbox\undefined
    \def\setjjpoega{\box\jjpoegabox}
\else
    \def\setjjpoega{\parbox{\wd\jjpoegabox}{\box\jjpoegabox}}
\fi


\begin{figure}
\hspace{.4in} \setjjpoega
\caption{A simple example of our algorithm}
\label{fig:ob2}
\end{figure}

\newbox\jjpoegbbox
\newdimen\jjpoegbwd
\font\jjpoegba=jjpoegba at 72.27truept
\font\jjpoegbb=jjpoegbb at 72.27truept
\setbox\jjpoegbbox=\vbox{\hbox{%
\jjpoegba\char0\char1\char2\char3\char4\char5}}
\jjpoegbwd=\wd\jjpoegbbox
\setbox\jjpoegbbox=\hbox{\vbox{\hsize=\jjpoegbwd
\parskip=0pt\offinterlineskip\parindent0pt
\hbox{\jjpoegba\char0\char1\char2\char3\char4\char5}
\hbox{\jjpoegba\char6\char7\jjpoegbb\char0\char1\char2\char3}
\hbox{\jjpoegbb\char4\char5\char6\char7\char8\char9}}}
\ifx\parbox\undefined
    \def\setjjpoegb{\box\jjpoegbbox}
\else
    \def\setjjpoegb{\parbox{\wd\jjpoegbbox}{\box\jjpoegbbox}}
\fi


\begin{figure}
\hspace{.4in} \setjjpoegb
\caption{A second example}
\label{fig:ob1}
\end{figure}

\newbox\jjpoegcbox
\newdimen\jjpoegcwd
\font\jjpoegca=jjpoegca at 72.27truept
\font\jjpoegcb=jjpoegcb at 72.27truept
\setbox\jjpoegcbox=\vbox{\hbox{%
\jjpoegca\char0\char1\char2\char3\char4\char5}}
\jjpoegcwd=\wd\jjpoegcbox
\setbox\jjpoegcbox=\hbox{\vbox{\hsize=\jjpoegcwd
\parskip=0pt\offinterlineskip\parindent0pt
\hbox{\jjpoegca\char0\char1\char2\char3\char4\char5}
\hbox{\jjpoegca\char6\char7\jjpoegcb\char0\char1\char2\char3}
\hbox{\jjpoegcb\char4\char5\char6\char7\char8\char9}}}
\ifx\parbox\undefined
    \def\setjjpoegc{\box\jjpoegcbox}
\else
    \def\setjjpoegc{\parbox{\wd\jjpoegcbox}{\box\jjpoegcbox}}
\fi


\begin{figure}
\hspace{.4in} \setjjpoegc
\caption{A more complicated example}
\label{fig:ob3}
\end{figure}

\Comment{
\begin{figure}[bh]
\vspace{3in}
\special{psfile=/usr/people/jj/modelTR/3-spline/img/--.ps}
\caption{Tangents through a fixed point}
% tops pointTang.rgb -m 6.5 2 > pointTang.ps
% a curve and a point, with tangents from curve that intersect point;
%	dual curve and dual line, with intersections
\label{fig:pointTang}
\end{figure}
}
%
\Comment{
video:  curve and point in left window, dual curve and dual line in right window;
	point moving on dual curve with associated tangent moving on curve,
	illustrating that as the point approaches an intersection with the line
	in dual space, the tangent meets the point in affine space
}

The individual steps of the algorithm, dualization and intersection,
require some elaboration.
A crucial fact is that the tangential curve of a 
Bezier curve can be expressed as a rational Bezier curve.
This simplifies the intersection in dual space,
since intersection of Bezier curves is well understood.

\begin{theorem}
\label{thm:rationalduala}
Let $C(t)$ be a plane Bezier curve of degree $n$ 
with control points $\{ (b_{i,1}, b_{i,2}) \}_{i=0}^n$ over the parameter interval $[t_1,t_2]$.
The \atang $C_a^*(t)$ is a rational Bezier curve of degree $2n-1$ 
over $[t_1,t_2]$ with weights $\{w_k\}_{k=0}^{2n-1}$ where: 
\begin{displaymath}
\scriptsize{
w_k = \alpha
\sum_{j=\mbox{max}(0,k-n)}^{\mbox{min}(n-1,k)} 
\tinychoice{n-1}{j} \tinychoice{n}{k-j} (b_{j,2} - b_{j+1,2})
}
\end{displaymath}
%
and control points $\{c_k\}_{k=0}^{2n-1}$ where:
\begin{displaymath}
\tiny{
c_k = \frac{\alpha}{w_k} 
\left(
\begin{array}{l}
	\displaystyle{\sum_{j=\mbox{\tiny{max}}(0,k-n)}^{\mbox{\tiny{min}}(n-1,k)}}
	\tinychoice{n-1}{j} \tinychoice{n}{k-j} (b_{j+1,1} - b_{j,1})\\
	\displaystyle{\sum_{\begin{array}{c} \mbox{\tiny{$0 \leq i \leq n-1$}} \\ 
			     \mbox{\tiny{$0 \leq j \leq n$}} \\ 
			     \mbox{\tiny{$i+j=k$}}
			     \end{array}}}
\tinychoice{n-1}{i} \tinychoice{n}{j} \beta_{i,j}
\end{array}
\right)
}
\end{displaymath}
where $\alpha = \tiny{\frac{n}{(t_2 - t_1) \tinychoice{2n-1}{k}}}$,
and $\beta_{i,j} = (b_{i+1,2} - b_{i,2}) b_{j,1} - (b_{i+1,1} - b_{i,1}) b_{j,2}$.
\end{theorem}
\prf
We omit the proof for lack of space and due to its technical nature.
The reader is referred to \cite{jj00} for a full proof.
\QED

\Comment{
\prf
We only give a sketch of the proof.
A full proof is available in \cite{jj00}.
Let $C(t) = (c_1(t), c_2(t))$.
The normal vector of $C(t)$ is $(-c'_2(t), c'_1(t))$.
The implicit equation of the tangent of $C(t)$ is:
\[	
	(x-c_1(t), y-c_2(t))\ \  \cdot \ \ (-c'_2(t), c'_1(t)) = 0
\]
\ifTalk
Notice that this is analogous to \ref{eqn:standard}
but not specific to $P$.
In some ways, the difference between the dual and standard solutions
is like the difference between the hodograph and the second column
of the de Casteljau triangle: it is a re-definition of the
order of the steps.
\fi
or
\[
	-c_2'(t) x + c_1'(t) y + [c_2'(t) c_1(t) - c_1'(t)c_2(t)] = 0 
\]
and the \atang of $C(t)$ is
\begin{equation}
\label{eq:dual1}
  (c_1'(t),\ \ c_2'(t) c_1(t) - c_1'(t)c_2(t),\ \ -c_2'(t))
\end{equation}
%
This is a polynomial curve in projective 2-space.
We want to express it as a Bezier curve.
We can do so as follows.
The product formula for Bernstein polynomials \cite{farin97} is used to translate
the second coordinate of (\ref{eq:dual1}) to a univariate Bezier curve
of degree $2n-1$.
We then degree elevate the first and third coordinates, expressed as 
univariate Bezier curves, from degree $n-1$ to degree $2n-1$.
This yields a Bezier curve of degree $2n-1$ in projective 2-space.
We then use the fact that a Bezier curve in projective 2-space is equivalent
to a rational curve in 2-space.
\QED
}

\Comment{
% shortened, though complete, version of \cite{jj00} proof:
Consider the second coordinate of the tangential curve,
which can be expressed as: 
\[
\begin{array}{l}
\scriptstyle{\frac{1}{(t_2 - t_1)}}
\displaystyle{\sum_{i=0}^{n-1}} \scriptstyle{n \Delta b_{i,2} B_i^{n-1}(t)}
 \displaystyle{\sum_{j=0}^n}     \scriptstyle{         b_{j,1} B_j^n(t)} \\
  - 
\scriptstyle{\frac{1}{(t_2 - t_1)}}
 \displaystyle{\sum_{i=0}^{n-1}} \scriptstyle{n \Delta b_{i,1} B_i^{n-1}(t)}
 \displaystyle{\sum_{j=0}^n}	 \scriptstyle{         b_{j,2} B_j^n(t)}
\end{array}
\]
where $\Delta b_{i,j} = b_{i+1,j} - b_{i,j}$.
We can use the product formula
% \footnote{$B_i^m(t) B_j^n(t) = \frac{ \tinychoice{m}{i} \tinychoice{n}{j} }{ \tinychoice{m+n}{i+j} } B_{i+j}^{m+n}(t)$.}
for Bernstein polynomials \cite{farin97} to express this as
a 1-dimensional Bezier curve of degree $2n-1$ 
with control points $\{b^*_{k,3} \}_{k=0}^{2n-1}$ where:
\begin{equation}
\label{eq:dualw2}
\scriptstyle{
b^*_{k,3} = \frac{n}{(t_2 - t_1) \tinychoice{2n-1}{k}}
\displaystyle{\sum_{\begin{array}{c} \mbox{\footnotesize{$0 \leq i \leq n-1$}} \\ 
			     \mbox{\footnotesize{$0 \leq j \leq n$}} \\ 
			     \mbox{\footnotesize{$i+j=k$}}
			     \end{array}}}
\tinychoice{n-1}{i} \tinychoice{n}{j} \beta_{i,j}.
}
\end{equation}

Now consider the first and third coordinates of the \tang (\ref{eq:dual1}), 
\begin{equation}
\label{eq:dualx}
-c_2'(t) = \frac{1}{t_2 - t_1} \sum_{k=0}^{n-1} -n \Delta b_{k,2} B_k^{n-1}(t)
\end{equation}
and
\begin{equation}
\label{eq:dualy}
c_1'(t)  = \frac{1}{t_2 - t_1} \sum_{k=0}^{n-1}  n \Delta b_{k,1} B_k^{n-1}(t)
\end{equation}
These are 1-dimensional Bezier curves of degree $n-1$,
which we need to degree-elevate to degree $2n-1$
for compatibility with the third coordinate.
After this degree elevation, $-c_2 ' (t)$ becomes a 1-dimensional Bezier curve of degree $2n-1$
with control points $\{b^*_{k,1} \}_{k=0}^{2n-1}$ where:
\begin{equation}
\label{eq:1stcoord}
b^*_{k,1} = 
\sum_{j=\mbox{\footnotesize{max}}(0,k-n)}^{\mbox{\footnotesize{min}}(n-1,k)} 
	\frac{-n}{t_2 - t_1} \Delta b_{j,2}
	\frac{ \tinychoice{n-1}{j} \tinychoice{n}{k-j} }{ \tinychoice{2n-1}{k} }
\end{equation}
%
and $c_1 ' (t)$ becomes a Bezier curve of degree $2n-1$ with control points 
$\{b^*_{k,2} \}_{k=0}^{2n-1}$ where:
\begin{equation}
\label{eq:2ndcoord}
b^*_{k,2} = 
\sum_{j=\mbox{\footnotesize{max}}(0,k-n)}^{\mbox{\footnotesize{min}}(n-1,k)} 
	\frac{n}{t_2 - t_1} \Delta b_{j,1}
	\frac{ \tinychoice{n-1}{j} \tinychoice{n}{k-j} }{ \tinychoice{2n-1}{k} }
\end{equation}
%
Combining (\ref{eq:dualw2}), (\ref{eq:1stcoord}) and (\ref{eq:2ndcoord}),
we conclude that the \tang $C^*(t)$
is a rational Bezier curve of degree $2n-1$ with control points 
and weights as in the statement of the lemma,
using the fact that a Bezier curve in projective 2-space is equivalent
to a rational curve in 2-space.
\QED
}

The representation of the \btang as a rational Bezier curve is analogous.

The line-curve intersection of steps (2a) and (2b) is the intersection
of two Bezier curves.
It is simple to represent a line as a Bezier curve of degree 1,
and intersection with a line is a particularly simple case of Bezier
curve intersection.
The intersection of Bezier curves is well understood and efficient \cite{sederberg86}.
We use the classical subdivision approach, where the two curves are recursively
subdivided into shorter segments until their intersection can be adequately
approximated by the intersection of two lines.

We end this section by explaining why we did not use the duality 
$ax+by+c=0 \rightarrow (a,b,c)$.
It maps lines through the origin to infinity.
Since a horizontal line can pass through the origin,
as can a vertical line, this duality does not cooperate well with either of
the other dualities.

\Comment{
In order to compute all intersections robustly, we must actually compute
two different versions in two different dual spaces,
so that the intersections at infinity in one dual space will be finite
intersections in the other dual space.
To avoid computing intersections redundantly in the two spaces,
we clip the curves in the dual spaces along $x=\pm 1$, which are associated
with diagonal tangents.

\begin{rmk}
In another example of the inferiority of the c-dual,
(just like in common tangent computation, see Section~\ref{sec:negative}),
we observe that it does not treat poles at the origin well.\footnote{Of course,
	we also don't want to use the c-dual now 
	since it will not be consistent with the \atangs and \btangs.}
After all, the point $(a,b)$ in primal space is dual to the line $ax+by+1=0$,
or in projective space, the line $ax+by+w=0$ where $w$ is the projective coordinate.
So the dual of the origin is the line w=0 at infinity!
% We can either avoid the origin or translate the obstacles and pole together
% away from the origin before computing.
The origin (0,0,1) does not map to infinity if the a-dual and b-dual is used
(and neither does any other point).
\end{rmk}
}

\section{Conclusions}
\label{sec:comparison}

In this paper, we have studied a new approach to computing the tangents
through a point, a basic visibility operation.
An algebraic method that reduces to the solution of a univariate equation of degree $2n-1$
was replaced by a geometric method that reduces to the intersection
of a line and Bezier curve of degree $2n-1$ in dual space.
Although, in the context of this problem, the two approaches are similar,
the dual method has the better capacity for generalization and
other important visibility problems, especially for motion and lighting in 
3-space, can be attacked anew using this new approach.
By studying the dual solution for the tangents through a point,
we see the development of this new method as it mutates away from the
algebraic solution.

The simplicity of the dual method of this paper makes it feasible
for many visibility computations from a point to be lifted from an approximate
polygonal world to a more realistic curved world.
The next step is the application of this smooth tangent operation 
to the construction of smooth visibility graphs, shortest path motion
amongst smooth obstacles, lighting of smooth environments,
and other visibility analysis in a smooth curved world.

\bibliographystyle{plain}
\begin{thebibliography}{Lozano-Perez 83}

\bibitem[Farin 97]{farin97}
Farin, G. (1997)
Curves and Surfaces for CAGD: A Practical Guide (4th edition).
Academic Press (New York).

\bibitem[Fulton 69]{fulton69}
Fulton, W. (1969)
Algebraic Curves.
Benjamin/Cummings (Menlo Park).

\bibitem[Johnstone 00]{jj00}
Johnstone, J.K. (2000)
A parametric solution to common tangents.
Technical report.

\bibitem[Kajiya 82]{kajiya82}
Kajiya, J. (1982)
Ray Tracing Parametric Patches.
SIGGRAPH '82, 245--254.

\bibitem[Kim 88]{kim88}
Kim, M.-S. (1988)
Motion Planning with Geometric Models.
Ph.D. Thesis, Purdue University, Dept. of Computer Science.

\bibitem[Lawrence 72]{lawrence72}
Lawrence, J.D. (1972)
A Catalog of Special Plane Curves.
Dover (New York).

\bibitem[Sederberg 84]{sederberg84}
Sederberg, T. and D. Anderson and R. Goldman (1984)
Implicit Representation of Parametric Curves and Surfaces.
Computer Vision, Graphics, and Image Processing 28, 72--84.

\bibitem[Sederberg 86]{sederberg86}
Sederberg, T. and S. Parry (1986)
Comparison of three curve intersection algorithms.
Computer Aided Design 18, 58--63.

\bibitem[Sederberg 95]{sederberg95}
Sederberg, T. and F. Chen (1995)
Implicitization using Moving Curves and Surfaces.
SIGGRAPH '95, 301--308.

\bibitem[Semple 85]{semple85}
Semple, J. and L. Roth (1985)
Introduction to Algebraic Geometry.
Oxford University Press (Oxford).

\end{thebibliography}

\end{document}

