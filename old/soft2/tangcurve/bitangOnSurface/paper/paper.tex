\section{Visibility graphs of curved environments and shortest paths}

\section{Shading with curved area lights}

\section{Curved Minkowski sums, for curved robots}

(The following is a reasonable topic in the context of my work on 
swept surfaces and envelopes, but unrelated to tangential computation
and curves on surfaces.)
Curved extension of Minkowski sum (see Bajaj and Kim, Generation of Configuration
Space Obstacles: The Case of Moving Algebraic Curves
and Compliant Motion Planning with Geometric Models).
That is, for a non-point robot, the robot must first be shrunk to a point
and the obstacles expanded by their Minkowski sum with the robot
in order to apply the visibility graph algorithm.
Typically the robot and obstacles have been polygonal.
(A circular robot has also been dealt with by Yap.)
We have shown how to deal with curved obstacles.
Now we want to deal with a curved robot.

\section{Other tangential relationships from \tangs}

are tangents of the dual curve = points of the original curve? (test empirically)
proof: common tangents of dual curves must be intersections in original space, 
	by symmetry, thus tangents of dual must be points of original curve
a way to compute the envelope of a family of lines:
	translate family of lines to dual space, where it becomes a curve;
	compute the tangents of this dual curve (hodograph), 
	whose duals become the envelope curve in the original space
	[is this the striction curve of the ruled surface?]
	not interesting in 2-space, but analogue in 3-space would be interesting:
	computing envelope of planes
	
Hypothesis: Consider a plane moving through 3-space.
	Its image in dual space is a curve (in 3-space).
	The tangents of this space curve are lines in dual space.
	I believe that as points dualize to planes and planes to points,
	lines dualize to lines.
	If so, the tangent lines of the space curve in dual space
	(which form a ruled surface, actually a tangent developable)
	dualize to a ruled surface in the primal space.
	Could this be the envelope of the moving planes?

\section{Common tangents of curves on surfaces}

We have considered common tangents of plane curves.
It is also useful to compute common tangents of curves on surfaces.
This is a more challenging computation 
because tangents must be geodesics [define geodesic] and 
geodesics on a surface are typically not straight lines, and are difficult
to characterize and compute.
To compute a tangent, we must be able to compute the geodesic
from a given point in a given direction (the direction of the derivative
of the curve at the point).
[Discussion of computation of a geodesic: solution of differential equations]

Although geodesics are difficult to compute on most surfaces,
there are certain important surfaces with simpler geodesics.
The design of curves on a sphere is a very important problem,
yet the geodesics of a sphere are all great circles of the sphere.
We now consider common tangents on \Sn{2}\ and \Sn{3}, the unit spheres 
at the origin in 3-space and 4-space.

Curves on \Sn{3}, called quaternion splines, are crucial for motion planning
in computer animation and robotics.
It is natural to introduce orientation obstacles (orientations disallowed for
kinematic reasons or for their undesirability, such as an upside-down 
orientation during the motion of a cup or camera), just as we have position
obstacles.
A classical tool for obstacle avoidance in the plane is the visibility graph,
which involves the computation of common tangents of the planar obstacles.
A visibility graph for avoidance of orientation obstacles on \Sn{3}\
requires the computation of common tangents of curves on \Sn{3}.

\section{Common tangents on \Sn{2}}

Check out Tony Woo's work on curves on Gaussian sphere for
visibility for cutting out an object or building an object with multiple
hands.

A geodesic, and thus a tangent, on \Sn{2}\ is a great circle.
In order to dualize the tangent space of a curve on \Sn{2},
we need the implicit equation of a tangent.
Although the implicit equation of a circle in 3-space involves 2 equations,
a sphere and a plane, our sphere will always be \Sn{2}\ and so 
a great circle is characterized by a plane, a single linear equation.
Another serendipity is that the plane will always pass through the origin,
so its constant term disappears and our plane is defined by 3 rather than
4 coefficients.
This allows the tangent space of a curve on \Sn{2}\ to be captured just 
like the tangent space of a plane curve, by a plane rational Bezier curve 
in dual space.

\begin{defn2}
On the surface \Sn{2}, 
the great circle $\{x^2+y^2+z^2-1=0,\ ax+by+cz=0\}$ and
the point $(a,b,c) \in P^2$ are {\bf duals}.
\end{defn2}

Notice that the dual $(a,b,c)$ is the normal of the plane $ax+by+cz=0$.

\begin{defn2}
The {\bf dual of a curve $C(t)$ on \Sn{2}} is the curve $C^*(t) \subset P^2$
where the point $C^*(t)$ is the dual of the tangent circle at $C(t)$.
\end{defn2}

Let $C(t)$ be a curve on \Sn{2}.
Since the plane of the tangent circle at $C(t)$ contains the vectors $C(t)$
and $C'(t)$, it has the normal $C(t) \times C'(t)$.
Thus, the dual of the tangent circle at $C(t)$ is $C(t) \times C'(t)$.
Notice that this dual is the binormal vector at $C(t)$.\footnote{Thus, in
	computing a dual curve, we are computing a binormal curve.
	This is related to Saito/Sederberg's normal hodograph and 
	Catmull's normal surface.
	A major difference is that we compute the dual curve in projective
	space and project to a rational curve in 2-space (to allow intersection), 
	so the binormal vector cannot be read off directly.}
	
We now develop a formula for the dual of a Bezier curve on \Sn{2}.
Let $C(t)$ be a rational Bezier curve on \Sn{2}:
\[
	C(t) = \frac{\sum_{k=0}^n w_k b_k B_k^n(t)}{\sum_{k=0}^n w_k B_k^n(t)}
\]
We must deal with rational Bezier curves, since curves on \Sn{2}\ (and \Sn{3})
are not polynomial, in general [Johnstone and Williams 99].
{\bf Try to generalize dual curves in 2-space using the same method.}
The dual curve is $C(t) \times C'(t)$.
Since the dual curve is originally expressed in projective space,
we can compute $\alpha(t) C(t) \times C'(t)$, since vector length
is immaterial in projective space.
In particular, in computing the dual curve, 
we are only interested in the direction of $C(t)$ and $C'(t)$, not
their lengths.
This allows a simplification: we can replace the rational curves
$C(t)$ and $C'(t)$ by polynomial curves of the same direction.
A polynomial expression for the direction of $C(t)$ is
\[
	\sum_{k=0}^n w_k b_k B_k^n(t)
\]
We use a result of Saito, Wang and Sederberg (1993, 
Hodographs and normals of rational curves and surfaces, manuscript)
for a polynomial expression for the direction of $C'(t)$.

\begin{lemma}
Let $C(t) = \frac{\sum_{k=0}^n w_k b_k B_k^n(t)}{\sum_{k=0}^n w_k B_k^n(t)}$ 
be a rational Bezier curve in 3-space.
A vector in the direction $C'(t)$ is 
\[
	\sum_{k=0}^{2n-2} H_k B_k^{2n-2}(t)
\]
where
\[
	H_k = \frac{1}{\tinychoice{2n-2}{k}}
		\sum_{i=\mbox{max}(0,k-n+1)}^{\lfloor k/2 \rfloor}
		(k-2i+1)\tinychoice{n}{i} \tinychoice{n}{k-i+1}
		(w_i b_{k-i+1,0} - w_{k-i+1} b_{i,0},
		 w_i b_{k-i+1,1} - w_{k-i+1} b_{i,1},
		 w_i b_{k-i+1,2} - w_{k-i+1} b_{i,2}),\ \ 
	b_i = (b_{i,0}, b_{i,1}, b_{i,2})
\]
\end{lemma}

algorithm for common tangent computation (is filtering necessary?):

\section{Common tangents on \Sn{3}}

Discuss applications to orientation obstacles.

\section{Common tangents on developable surfaces}

A geodesic on a developable surface is also tractable:
it is a rolled-up line.


\section{Dual curves of space curves}

We have shown how to compute the dual of a plane curve and a surface.
What about a space curve?
Unfortunately, duals only exist for objects of codimension 1.
(Define codimension.)
The tangent of a space curve is a line in 3-space, which can no longer
be expressed as a single linear equation since it is no longer a hyperplane.
Consequently, the dual of a space curve tangent, and thus of a space curve,
is undefined.

\section{Common tangents of two subdivision curves}

\end{document}

