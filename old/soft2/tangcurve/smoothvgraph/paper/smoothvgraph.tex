\documentclass[11pt,twocolumn]{article}
\newif\ifVideo
\Videofalse
\newif\ifTalk
\Talkfalse
\newif\ifJournal
\Journalfalse
\input{header}
\newcommand{\plucker}{Pl\"{u}cker\ }
\newcommand{\vgraph}{visibility graph\ }
\newcommand{\vgraphs}{visibility graphs\ }
\SingleSpace

\setlength{\oddsidemargin}{0pt}
\setlength{\topmargin}{-.2in}	% should be 0pt for 1in
% \setlength{\headsep}{.5in}
\setlength{\textheight}{8.5in}
\setlength{\textwidth}{6.5in}
\setlength{\columnsep}{5mm}	% width of gutter between columns
\markright{The smooth visibility graph: \today \hfill}
\pagestyle{myheadings}
% -----------------------------------------------------------------------------

\title{On the smooth visibility graph}
% The precision and simplicity of the smooth visibility graph
% The precision and terseness of smooth visibility graphs
% Smooth visibility graphs are smaller and more precise
% Smooth visibility graphs are terse
% Visibility graphs from smooth environments
\author{J.K. Johnstone\thanks{Geometric Modeling Lab, 125 Campbell, 
	Computer and Information Sciences, UAB, Birmingham, AL 35294.}}

\begin{document}
\maketitle

\begin{abstract}
% The visibility graph is an important structure for the analysis of visibility
% in many applications such as lighting and motion.
Visibility graphs are commonly built for a polygonal scene.
This paper explores the visibility graph for a smooth scene.
% with curved boundaries.
This smooth visibility graph is built from common tangents of the smooth curves
defining the scene.
We compare the polygonal and smooth visibility graphs of the same scene
(where the polygons are discrete samplings of the curves).
The smooth visibility enjoys two significant improvements over the polygonal visibility graph:
it has optimal accuracy and yet it is also very small (indeed, optimally small).
Since accuracy of the polygonal visibility graph 
improves by increasing the sampling rate,
its size balloons as its accuracy improves.
The smooth visibility graph captures the exact information required
for visibility in the most efficient way.
The visibility graph is important for much visibility analysis required
in lighting, motion planning, and many
other applications in graphics and robotics.

The challenge of performing visibility computation in a smooth world
is the refinement of visibility tools for smooth models.
For example, visibility analysis in the plane requires common tangents
of a pair of curves, a single curve, and a point and a curve.
We have recently developed a new method for the computation of common tangents,
which is fundamental to the construction of smooth visibility graphs.

The sharp drop in size and complexity of the smooth visibility graph
is more surprising.

We argue that performing visibility computations with smooth models
is more efficient than the comparable visibility computation with
polygonal models.
Even though the complexity of each individual operation is larger in a 
smooth world, less operations need to be done.
This is illustrated by the drop in complexity of a visibility graph
of a curved environment over a visibility graph of a polygonal 
environment (Figure).
Moreover, the smooth world allows more accurate shadow computation 
using smooth models and the introduction of smooth lights.
\end{abstract}

\section{Introduction}

The visibility graph of a scene encodes information about the visibility 
between the objects in the scene.
The classical visibility graph is defined for a polygonal scene.

\begin{defn2}
The visibility graph of a collection of polygons is the graph $G=(V,E)$ 
where $V$ is the set of vertices of the polygons
and two vertices $v$ and $w$ are connected by an edge in E if 
(a) the interior of the line segment $vw$ has no intersections with the polygons
($v$ is visible to $w$), or (b) $vw$ is an edge of a polygon.
The cost of an edge is the Euclidean distance between its vertices.
In many applications, two of the polygons are degenerate, representing
source and destination points.
See Figure~\ref{fig:poly}.
{\bf In fact, for visibility analysis, only edges that are tangent to two polygons
need to be considered (Figure~\ref{---}).}
This can greatly reduce the size of the visibility graph.
A line is tangent to a polygon at $P$ if it does not lie in the interior 
of the polygon in some neighbourhood of $P$.
\end{defn2}

One can also define a visibility graph for smooth curves,
which we call the smooth visibility graph.

\begin{defn2}
The visibility graph of a collection of plane curves is the graph $G=(V,E)$
where $V$ is the set of endpoints of the common tangents of the curves
and $E$ is the set of common tangents between the curves
(in analogy to edges between visible vertices in the discrete case)
and curve segments between vertices $v$ and $w$ that lie on the same curve
and are neighbouring (the arc between $v$ and $w$ does not contain any
common tangent endpoints in its interior)
(in analogy to the edges of the polygon in the discrete case).
Note that common tangents between the same curve are valid common tangents.
The cost of an edge is the Euclidean distance between its vertices
for common tangent edges and the arc length of the curve segment for
curve segment edges.
Again, in many applications, two of the curves are degenerate source
and destination 'curves' (points).
\end{defn2}

Figures~\ref{fig:poly} and \ref{fig:smooth} 
are an illustration of the tangible difference between
the polygonal and smooth visibility graphs of the same scene.
Here the polygon is a discrete sampling of the curve.
Consider a scene of curved objects.
If we use a polygonal \vgraph to analyze the visibility of this scene,
the curves must be sampled into polygons.
To improve the precision of the analysis,
the sampling must be made denser and denser.
Unfortunately, this leads to a very large visibility graph.
In other words, the goals of accuracy and complexity conflict in the
polygonal visibility graph of a smooth scene:
an accurate graph is very large,
while a small graph is very inaccurate.
Also notice that the error will never truly disappear except in the limit
of infinitely dense sampling (which would generate infinite visibility graphs).
In contrast, the smooth \vgraph captures visibility exactly in an optimally
small graph.
The error in a smooth \vgraph truly has disappeared (modulo machine precision),
yet the size of the graph has not increased.
In fact, the graph is as small as it can be: all edges are necessary,
encoding a crucial extreme of visibility.
In other words, the smooth \vgraph elegantly combines two virtues:
it is as small as possible yet also as accurate as possible.

% \input{jjvegap.tex}

\begin{figure}
% \setjjvegap
\caption{A visibility graph}
\label{fig:poly}
% svgraph vg3.rawctr, polygonal visibility graph on
\end{figure}

% \input{jjvegas.tex}

\begin{figure}
% \setjjvegas
\caption{A smooth visibility graph of the same scene}
\label{fig:smooth}
% svgraph vg3.rawctr, visible common tangents on
\end{figure}

In the rest of this paper, we will quantitatively examine the smooth \vgraph's
size (Section~\ref{sec:size}) and accuracy (Section~\ref{sec:accuracy}), 
in comparison to the polygonal \vgraph.
We begin with a review of the literature on visibility graphs in Section~\ref{sec:lit}
and a discussion of the construction of a smooth visibility graph
in Section~\ref{sec:alg},
and end in Section~\ref{sec:application} with a discussion of the application
of the visibility graph to problems in motion and lighting.

\section{Construction of the smooth visibility graph}
\label{sec:alg}

\begin{enumerate}
\item 
   The common tangents between two nondegenerate curves are 
   computed using a method of \cite{jj00a}.
   The common tangents between a curve and a point (that is, the tangents
   of a curve that meet the point) are computed using a method of \cite{jj00b}.
\item 
   The arc length of a curve segment (for edge cost) is computed
   using a method of \cite{png91}.
   In particular, the arc length is computed using subdivision and
   a sandwiching of the arc length of a segment
   between the length of the straight line segment between the points 
   and the perimeter of the control polygon.
\item 
   Removal of interior visible common tangents (ones inside an obstacle):
  test midpoint by casting ray to infinity; if it is inside the obstacle
  (odd number of intersections with obstacle), then discard the tangent.
  See 'dual ob3.rawctr'.
\end{enumerate}

\Comment{
Then apply the visibility graph:
compute shortest path using Dijkstra's algorithm \cite{ahu83}
on the visibility graph.
(Edelsbrunner says that Fredman and Tarjan 1984 has an $O(n \log n + m)$
time algorithm for Dijkstra (m = no. of edges).)
}

\section{Smooth \vgraphs are smaller}
\label{sec:size}

\subsection{Experiments}

\begin{enumerate}
\item Input polygon.
\item Interpolate curve.
\item Sample curve at $\epsilon$ rate ($\epsilon$ is input parameter).
\item Build visibility graph of smooth curve and sampled polygon.
\item Allow mouse to specify source/destination interactively (with new
	results posted with each change).
\item Compute shortest path in smooth and polygonal visibility graphs.
\item Output size of visibility graphs.
\item Compute length of several shortest paths, including length
	of each edge between obstacles and length of each self-edge.
	Randomly generate source/destination or use the mouse.
\item Here we are using the length of the shortest path to encode the
	accuracy of the visibility graph.
	The direct computation, say of distance of polygonal points of
	tangency from smooth points of tangency, is too expensive and verbose.
	The shortest path captures all of this information in one package.
\end{enumerate}

Consider a curve and various samplings and the resulting error
in the length of the shortest path, the maximal distance from the
true shortest path and the size of the visibility graph.

\subsection{}

In the polygonal world, increased accuracy leads to an exponentially
increasing search space.
In a smooth world, the search space is of constant size.
$n$ polygonal vertices lead to $O(n^2)$ edges in the visibility graph.
To analyze this practically, we show the sizes of visibility graphs 
for several examples.

Consider the sampling necessary for $\epsilon$ accuracy on a curve
(no point on edge more than $\epsilon$ distance from curve;
or no edge more than $\epsilon$ different in length from associated
curve segment).

These huge V-graphs are inefficient to compute, and inefficient later to search.
If one reacts to this inefficiency problem by using
less densely sampled polygonal approximations,
the error of the resulting motion is increased.
Consider the simple case of motion planning amongst two circular obstacles again.
(Compare the size of error in the path and the size of the V-graph.)

\section{Smooth \vgraphs are more accurate}
\label{sec:accuracy}

\section{Literature}
\label{sec:lit}

\begin{itemize}
\item
	Lozano-Perez on visibility graph for shortest path motion amongst polygons.
\item
	Computational geometry optimal results.
	Lee 1978 (visibility graph in $n^2 \log n$ time);
	Nilsson?
	Welzl 1985: $n^2$ visibility graph computation.
\item
	Visibility graph for 2D lighting, sound, and walkthroughs 
	(see Durand thesis).
\item
	Visibility complex (lighting) work.
\item
	O'Rourke web page.
	Other web pages (see Durand).
\end{itemize}

\section{Smooth motion and lighting}
\label{sec:application}

\section{Future work}

Improvement of smooth version of Minkowski sum (see Bajaj),
which is related to the convex hull of the smooth curves and may perhaps
be aided by the improved common tangent computation.

\section{}

A classical application of the visibility graph is 
to the shortest path motion problem,
either of a point moving amongst polygons in the plane, or of a polygon
amongst polygons (after some Minkowski sum reduction)
%  if the moving polygon is reduced to a point
% and the obstacle polygons are grown using Minkowski sums 
\cite{lozanoperez1and2}.
Shortest path motion in this geometric environment reduces to
shortest paths in the discrete visibility graph.
Visibility graphs are also heavily used in graphics, 
for lighting \cite{durand}, sound? \cite{?}, and ?.
All(?) of these applications use polygonal visibility graphs for polygonal scenes.
The smooth visibility graph is superior to the polygonal visibility
graph in many ways.

An obstacle to the development of smooth visibility graphs is an efficient
computation of common tangents of smooth curves.
We have recently developed a new method for this computation that allows
us to revisit the issue of smooth visibility graphs.

Visibility graphs are important in visibility analysis amongst geometric
models for motion, lighting, sound, etc.
Previous work has been with polygonal visibility graphs.
By moving to a curved world, the complexity of the visibility graph is markedly
reduced (Figure~\ref{}), yet its accuracy is actually improved.
At the heart of the computation of the visibility graph is the computation
of common tangents between obstacles.
We have developed a simple technique for the computation of common tangents
between curves \cite{jjCommonTang, jjPole},
so the computation of the reduced and more precise visibility graph 
in a smooth world becomes feasible.
We show how to build the visibility graph and analyze the improvement over
polygonal visibility graphs.
This includes an analysis of the tradeoff between precision and visibility
graph size in the polygonal visibility graph, compared with the optimality of both
precision and size with a smooth visibility graph.

Motion planning amongst curved obstacles is more elegant
than motion planning amongst the associated polygonal approximations
of the obstacles.
Densely sampled polygonal approximations are required for accurate motion
planning, which leads to very large visibility graphs.
Consider the V-graph of a pair of circular obstacles.
(Compare V-graph size for various samplings of the circles.)



Shortest paths amongst obstacles are built from common tangents
between the obstacles.
This is at the heart of the classical visibility graph algorithm for 
shortest path motion in the plane.
Typically, obstacles and robot have been polygonal, leading to simple
computation of tangents and common tangents (line segments between polygon
vertices).
The dual curve allows the obstacles and robot to become curved.

In the standard visibility 
graph algorithm for shortest paths amongst polygons in 2-space \cite{lozano79},
% ancient solution: see Wesley 1979 paper for reference of Nilsson c. 1969
edges of the visibility graph consist of (i) edges between visible polygon vertices,
(ii) edges between the source and visible polygon vertices,
and (iii) edges between the destination and visible polygon vertices 
(Figure~\ref{fig:shortest}a).
In shifting to a curved environment still in 2D,
the edges of the visibility graph become (1) common tangents between the curves,
(2) common tangents of a curve with itself, 
(3) visible tangents from the curves to the source, 
(4) visible tangents from the curves to the destination,
and (5) curve segments between the endpoints of (1)-(4), as shown in
Figure~\ref{fig:shortest}b.

\begin{defn2}
Visibility graph. Use Edelsbrunner 12.2 for review of classical visibility graph.
\end{defn2}

We illustrate the utility of the new method for common tangents of curves in the
computation of visibility graphs,
and quantify its superiority in both speed and accuracy 
over the associated use of visibility graphs of polygons,
thus establishing the advantage of using curves with the new tools to compute
visibility information for tasks such as lighting and motion.


\section{The appeal of motion amongst curved obstacles}

The smooth visibility graph is definitely a win for precision
and for graph size.
We should measure the added cost of computing the smooth visibility graph.

The number of resulting edges in the V-graph will be massively reduced.
For example, the $O(n^2)$ visible pairs of vertices of a circular polygon
will be replaced by 2 common tangents on the curved circle.
We must evaluate if this is an overall win.
(Perhaps V-graph search is negligibly fast for any size V-graph?)

\bibliographystyle{plain}
\begin{thebibliography}{Lozano-Perez 83}

\bibitem[AHU 83]{ahu83}
Aho, A. and J. Hopcroft and J. Ullman (1983)
Data Structures and Algorithms.
Addison-Wesley (Reading, MA).

\bibitem[Durand 00]{durand00}
Durand, F. (2000)
A Multidisciplinary Survey of Visibility.
SIGGRAPH Course Notes on Visibility.

\bibitem[Lee 78]{lee78}
Lee, D. (1978)
The all nearest-neighbor problem for convex polygons.
Information Processing Letters 7, 189--192.
% referenced in Edelsbrunner, Algs in Comb Geometry

\bibitem[Lozano Perez]{lp}
Lozano-Perez, T.

\bibitem[Png 91]{png91}
Png, T. (1991)
Adaptive Bounding Algorithms for the Arclength and Surface Area.
M.S. thesis, U. of Waterloo.

\bibitem[Welzl 85]{welzl85}
Welzl, E. (1985)
Constructing the visibility graph for $n$ line segments in $O(n^2)$ time.
Information Processing Letters 20, 167--171.

\end{thebibliography}


\end{document}
