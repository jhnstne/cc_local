% Planning shortest path motion amongst circles and spheres
\newif\ifFull
\Fullfalse
\documentstyle[12pt]{article} 
\input{macros}
% \input{ruledmacros}
\input{pageformat-double}
\newtheorem{defn2}{Definition}
\begin{document}
\bibliographystyle{plain}

\title{}
\author{John K. Johnstone\thanks{Department of Computer Science, The Johns 
	Hopkins University, Baltimore, Maryland 21218.
	This work was supported by National Science Foundation grant
	IRI-8910366.}}
\date{\today}

\maketitle

\SingleSpace

% *****************************************************************

\section{Common tangent of two circles}

Inversion is also the key to computing common tangents of two circles,
which are important in shortest path motion amongst circles.

Length of common tangents.
Position of centers of similitude.
Can use methods described elsewhere, but also methods using inversion?

\noindent Property (1.4) (on inversion and polars) shows 
that a knowledge of $\mbox{inv}_{S}(p)$ 
makes it easy to find the points of tangency from $p$: 
the polar is the hyperplane perpendicular to $\lyne{pc}$ through 
$\mbox{inv}_{S}(p)$.

(This property is stated on p. 46 of Johnson, Modern Geometry.)

% *****************************************************************

Define: common tangent (outside and inside)
	center of similitude (outside and inside)
	pole and polar (of a point with respect to a circle)

\begin{lemma}
\label{lem-similitude}
\cite[p. 105?]{SalmonConicSections}
Let $C$ be a circle with center $(\alpha,\beta)$ 
and radius $r$.
Let $C'$ be a circle with center $(\alpha',\beta')$ 
and radius $r'$.
The outside(?) 
% (NEED TO PROVE THAT THIS IS THE OUTSIDE ONE) 
% (WE THINK SO BECAUSE IT IS FOR NORMALIZED CIRCLES IN THEOREM BELOW)
center of similitude of $C$ and $C'$ is
$(\frac{\alpha'r - \alpha r'}{r - r'},
  \frac{\beta'r - \beta r'}{r - r'})$.
The inside center of similitude of $C$ and $C'$ is
$(\frac{\alpha'r + \alpha r'}{r + r'},
  \frac{\beta'r + \beta r'}{r + r'})$.
\end{lemma}
\Heading{Proof:}
\QED

\begin{lemma}
\label{lem-polar}
\cite[p. 82?, also see p. 103]{SalmonConic}
Polar of point $(x',y')$ with respect to circle 
$(x+\alpha)^{2} + (y+\beta)^{2} = r^{2}$
(tangent of point if point is on circle) 
is $(x-\alpha)(x'-\alpha) + (y-\beta)(y'-\beta) = r^{2}$.
(More suggestively, if the circle is centered at the origin,
then the polar is $(x,y) \dot (x',y') = r^{2}$.)
\end{lemma}

METHOD: use centers of similitude and polars from these points
to compute intersections of common tangents with both circles.

\begin{theorem}
Let $C_{1}$ and $C_{2}$ be two circles in the same plane,
with centers $c_{1}$ and $c_{2}$, respectively, and
radii $r_{1}$ and $r_{2}$, respectively.
The length of the outside common tangent between these two circles is 
$\sqrt{\|c_{1} - c_{2}\|^{2} - (r_{1} - r_{2})^{2}}$.
The length of the inside common tangent between these two circles is
$\sqrt{\|c_{1} - c_{2}\|^{2} + (r_{2} - r_{1})(r_{1} + 3r_{2})}$.
\end{theorem}
\Heading{Proof:}
Without loss of generality, assume that $C_{1}$ is centered at the origin
and $C_{2}$ is centered on the $x$-axis: $c_{2} = (\alpha,0)$.
We wish to compute the points of intersection of the outside common tangent
with the circles (Figure~\ref{fig-length}).
The outside center of similitude of $C_{1}$ and $C_{2}$ is 
$(\frac{\alpha r_{1}}{r_{1} - r_{2}}, 0)$ (Lemma~\ref{lem-similitude}).
The polar of the outside center of similitude with respect to $C_{1}$ is 
$x \frac{\alpha r_{1}}{r_{1} - r_{2}} = r_{1}^{2}$
or $x = \frac{r_{1}(r_{1} - r_{2})}{\alpha}$.
The intersections of this polar with $C_{1}$ are easily computed to be
$(\frac{r_{1}(r_{1} - r_{2})}{\alpha}, 
  \pm \frac{r_{1}}{\alpha} \sqrt{\alpha^{2} - (r_{1} - r_{2})^{2}})$,
which are the points of intersection of the two outside common tangents 
with $C_{1}$.
Similarly, the polar of the outside center of similitude with respect 
to $C_{2}$ is 
$(x-\alpha)(\frac{\alpha r_{1}}{r_{1} - r_{2}} - \alpha) = r_{2}^{2}$
or $x = \frac{r_{2}(r_{1} - r_{2})}{\alpha} + \alpha$.
The intersections of this polar with $C_{2}$ are
$(\frac{r_{2}(r_{1} - r_{2})}{\alpha} + \alpha, 
  \pm \frac{r_{2}}{\alpha} \sqrt{\alpha^{2} - (r_{1} - r_{2})^{2}})$.
It is clear that, of the four intersections of the outside common tangents
with the two circles, 
the two intersections with positive $y$-coordinates are on the same tangent.
The distance between these two points is $\sqrt{\alpha^{2} - (r_{1} - r_{2})^{2}}$.

% LENGTH OF INSIDE COMMON TANGENT
The length of the common inside tangent is found in exactly the same way.
Since the inside center of similitude of $C_{1}$ and $C_{2}$ is 
$(\frac{\alpha r_{1}}{r_{1} + r_{2}}, 0)$, which is very similar to the
outside center of similitude, many of the computations are very similar
and we omit them.
% (Therefore, the first computations are very similar to the outside 
% tangent with $r_{1} - r_{2}$ replaced by $r_{1} + r_{2}$.)
\Comment{
Its polar with respect to $C_{1}$ is $x = \frac{r_{1}(r_{1} + r_{2})}{\alpha}$.
The intersections of this polar with $C_{1}$ are 
$(\frac{r_{1}(r_{1} + r_{2})}{\alpha}, 
  \pm \frac{r_{1}}{\alpha} \sqrt{\alpha^{2} - (r_{1} + r_{2})^{2}})$.
The polar of the inner center of similitude with respect 
to $C_{2}$ is $x = \frac{r_{2}(r_{1} + r_{2})}{\alpha} + \alpha$.
The intersections of this polar with $C_{2}$ are
$(\frac{r_{2}(r_{1} + r_{2})}{\alpha} + \alpha, 
  \pm \frac{r_{2}}{\alpha} \sqrt{\alpha^{2} - (r_{1} + r_{2})^{2}})$.}
The length of the inside common tangent is 
$\sqrt{\alpha^{2} + (r_{2} - r_{1})(r_{1} + 3r_{2})}$.
\QED

\figg{fig-length}{The length of the outside common tangent}{1in}
% include both circles, outside center of similitude, outside common tangent
% note that centers of sim are collinear with centers

\section{}

Common tangents can be found as follows.
Let $P$ and $C$ be a point and a circle, respectively.
Let 

\end{document}