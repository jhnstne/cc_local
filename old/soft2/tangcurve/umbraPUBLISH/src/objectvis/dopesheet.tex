\begin{array}
frame      A     B#   P    entire scene   tangents    probes    concavities

Soundtrack (ST)
Consider a smooth object, call it A,
           1                                                                introduce star
nestled in a scene of many other smooth objects.
	   1               1                                                introduce its context
We want to study the visibility of A from a known viewpoint.
	   1          1    1 [SWEEP FROM 0 TO PI]                           introduce viewpoint
	              sweep a ray around the viewpoint (like a lighthouse beacon)

ST: We are interested in finding a line of sight that sees A 
           1                all but B4    piercing tangent of B2 (or any other line of sight)

ST: or, in the absence of such a line of sight,
    the chain of objects that blocks A from the viewpoint.

           1               all scene (minus some added redundant objects behind)

ST: Since visual events occur at tangents of the scene that pass through the viewpoint,

           1               B1/B2          piercing tangent of B2            introduce visual event

ST: these tangents are the focus of our analysis.
    All probes of the scene will be at one of these tangents through the viewpoint.

           1         1                    all tangents thru P

ST: The first probe will be at an extreme tangent of A.
    Each object has two extreme tangents, which define the extreme lines of sight to the object.

           1    [SWEEP FROM ONE EXTREME TANGENT TO THE OTHER]      extreme tangents of A

    We can start from either extreme tangent, say the right one.
           1                              right extreme tangent of A

    This probe is blocked by another object.
           1     B1 (in red; show intersection pt of B1 and probe; and pt of tangency on A)

    This object B does not completely block A, either behind B or within one of its concavities,
    so we must continue looking for a visible line of sight.

    Since the starting point of A does lie inside a concavity of B,
           1     B1 (with probe, and pt of tangency)           concavity of B1

    the next opportunity for A to become visible is at the opening of this concavity, which
    is defined by a tangent through the viewpoint.
    [i.e., next opportunity moving locally on A]
           1     B1 (with piercing tangent of concavity) [SWEEP FROM 1ST PROBE TO 2ND PROBE]
	   
    This is the second probe.
    This probe is blocked by another object B2:
	   1     B1/B2 (with piercing tangent probe, and its intersection with B2; 
	                first probe drawn as dotted line; range between 1st and 2nd probe in grey)

    A very important point is that A could squeeze between B2 and B1 to become visible.

           new-1;B1/B2
	   (where new-1 squeezes between B1 and B2)

    We must check for this possibility.
    Therefore, every probe is followed by a backtracking step that checks whether A becomes
    visible by sliding between two objects in the blocking chain in this fashion.
    In this case, we backtrack to the extreme tangent of B2.

           1 (old) B1/B2 (with extreme tangent of B2, the 3rd probe) [SWEEP FROM 2ND PROBE TO 3RD, BACKWARDS]

    Since this probe hits B1 before it hits A, we have established that A does not squeeze out here

	   1       B1 (blue) /B2 (red) /3rd probe (with first intersections with B1 and A)

    Returning to the earlier probe, which is blocked by B2,

           1      B2 (with 2nd probe)

    we continue looking for the next opportunity for A to become visible from the viewpoint.
    This again occurs at the tangent of B2 that seals its concavity,

           1      B2 (with 4th probe, the piercing tangent of B2)

    This probe hits another object before A (B4)

           1      B4, 4th probe
\end{array}


ST: Two important classes of tangent are extreme tangents and piercing tangents.  
    An extreme tangent of the object B is a tangent of B through the viewpoint
    such that all of B lies on one side of the tangent.

           1      extreme tangents of A   
	      then
	   B1     extreme tangents of B1

    A piercing tangent of the object B is associated with a concavity that faces the viewpoint.
    It is a tangent of B through the viewpoint that intersects B after its first point of tangency
    with B, but not before.
    The point of tangency should not be an inflection point {\bf [because ---].}
    
           B1     piercing tangent of B1
	      then
	   1      piercing tangents of 1

ST: The case when the viewpoint lies inside the convex hull of an object must be treated
    as a special case. Consider the extreme tangent.

           1      viewpoint inside 1

how do we cue all of these contexts?
we need a mechanism for turning objects on/off (with a certain colour), turning tangents on/off 
    (cued by object, type, and number), turning probes on/off (always including first 
    intersection point), turning ranges on/off, timing the length between context switches,
probes could be considered a special case of tangent (where intersection is also drawn),
    and so use the same fields, but with negative integers

array with column for every object (6), drawing style, every tangent!? (21), drawing style,
    sweep interval (for sweeping a line centered at the viewpt),
    range beg angle, range end angle, timing in seconds
drawing style of object is normal colour, bold colour, light colour, concavities-bold
drawing style of tangent is solid tangent, dotted tangent, and probe (with intersection pt)
e.g., 1 3 / R B / 13 15 / S P / 0 0 / 3
  which encodes object 1 and 3, colored red and blue, tangents 13 (solid tangent) and 15 (probe), 
  no range since (0,0) interval, 3 seconds

thus, input format is:
{\bf 
   objects / drawing style of each object / tangents / drawing style of each tangent /
   sweep interval / range interval [timing is replaced by spacebar prompt]
   { comment }
}
where object drawing style = R (red), G (green), B (blue), W (black), 
                             C (black with bold red concavities)
(convention is to draw star in blue, active object B in scene in red)
and tangent drawing style = S (solid), D (dashed), P (probe)
and sweep interval is defined by two tangent indices (-1 = 0 radians, -2 = PI)

if timing has run to 0, display loop reads a row to determine what to display
   (in debug mode, use space bar to cycle to next row)
timing: want something proportional to time, not PC speed: perhaps pause after glutPostRedisplay)

an input file for expository mode with eg19.pts

0 / B / / / /                      { the star }
0 1 2 3 4 5 / B W W W W W / / / /  { the scene }
0 1 2 3 4 5 / B W W W W W / / / -1 -2 / { the viewpoint and the act of viewing }

simplified dopesheet style:

0               star
0 1 2 3 4 5     scene
0 / 2 22        star's extreme tangents
0 / 2           starting probe (one of star's extreme tangents)
0 5 / 2 (as probe) the blocking object
skipping concavity, since this doesn't fit within this simple style
0 5 / 6         piercing tangent, the next probe
4 5 / 6         next blocking object
0 4 5 / 3       sweep backwards to blocking object's extreme tangent
0 4 / 6         return to the earlier probe (before backtracking)
0 4 / 10        next probe is at the blocking object's piercing tangent
                this is a visible line of sight (if we have removed object 2)
