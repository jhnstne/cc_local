\documentclass[12pt]{article}
\usepackage{times}
\usepackage[pdftex]{graphicx}
\makeatletter
\def\@maketitle{\newpage
 \null
 \vskip 2em                   % Vertical space above title.
 \begin{center}
       {\Large\bf \@title \par}  % Title set in \Large size. 
       \vskip .5em               % Vertical space after title.
       {\lineskip .5em           %  each author set in a tabular environment
        \begin{tabular}[t]{c}\@author 
        \end{tabular}\par}                   
  \end{center}
 \par
 \vskip .5em}                 % Vertical space after author
\makeatother

% default values are 
% \parskip=0pt plus1pt
% \parindent=20pt

\newcommand{\SingleSpace}{\edef\baselinestretch{0.9}\Large\normalsize}
\newcommand{\DoubleSpace}{\edef\baselinestretch{1.4}\Large\normalsize}
\newcommand{\Comment}[1]{\relax}  % makes a "comment" (not expanded)
\newcommand{\Heading}[1]{\par\noindent{\bf#1}\nobreak}
\newcommand{\Tail}[1]{\nobreak\par\noindent{\bf#1}}
\newcommand{\QED}{\vrule height 1.4ex width 1.0ex depth -.1ex\ \vspace{.3in}} % square box
\newcommand{\arc}[1]{\mbox{$\stackrel{\frown}{#1}$}}
\newcommand{\lyne}[1]{\mbox{$\stackrel{\leftrightarrow}{#1}$}}
\newcommand{\ray}[1]{\mbox{$\vec{#1}$}}          
\newcommand{\seg}[1]{\mbox{$\overline{#1}$}}
\newcommand{\tab}{\hspace*{.2in}}
\newcommand{\se}{\mbox{$_{\epsilon}$}}  % subscript epsilon
\newcommand{\ie}{\mbox{i.e.}}
\newcommand{\eg}{\mbox{e.\ g.\ }}
\newcommand{\figg}[3]{\begin{figure}[htbp]\vspace{#3}\caption{#2}\label{#1}\end{figure}}
\newcommand{\be}{\begin{equation}}
\newcommand{\ee}{\end{equation}}
\newcommand{\prf}{\noindent{{\bf Proof}:\ \ \ }}
\newcommand{\choice}[2]{\mbox{\footnotesize{$\left( \begin{array}{c} #1 \\ #2 \end{array} \right)$}}}      
\newcommand{\scriptchoice}[2]{\mbox{\scriptsize{$\left( \begin{array}{c} #1 \\ #2 \end{array} \right)$}}}
\newcommand{\tinychoice}[2]{\mbox{\tiny{$\left( \begin{array}{c} #1 \\ #2 \end{array} \right)$}}}
\newcommand{\ddt}{\frac{\partial}{\partial t}}
\newcommand{\Sn}[1]{\mbox{{\bf S}$^{#1}$}}
\newcommand{\calP}[1]{\mbox{{\bf {\cal P}}$^{#1}$}}

\newtheorem{theorem}{Theorem}	
\newtheorem{rmk}[theorem]{Remark}
\newtheorem{example}[theorem]{Example}
\newtheorem{conjecture}[theorem]{Conjecture}
\newtheorem{claim}[theorem]{Claim}
\newtheorem{notation}[theorem]{Notation}
\newtheorem{lemma}[theorem]{Lemma}
\newtheorem{corollary}[theorem]{Corollary}
\newtheorem{defn2}[theorem]{Definition}
\newtheorem{observation}[theorem]{Observation}

% \font\timesr10
% \newfont{\timesroman}{timesr10}
% \timesroman

\newif\ifCompanion   % material for companion paper
\Companionfalse
\newif\ifSurrounding    % if covering 'A surrounding L' case
\Surroundingtrue
\newif\ifJournal
\Journaltrue
\newif\ifTalk
\Talkfalse
\newif\ifComment
\Commentfalse
\newif\ifCommentary
\Commentaryfalse
\setlength{\headsep}{.5in}
\markright{\today \hfill}
\pagestyle{myheadings}

% outer to supporting? not as suggestive a term, so no
% inner to separating? 

% \DoubleSpace

\setlength{\oddsidemargin}{0pt}
\setlength{\topmargin}{0in}	% should be 0pt for 1in
\setlength{\textheight}{8.6in}
\setlength{\textwidth}{6.875in}
\setlength{\columnsep}{5mm}	% width of gutter between columns

% -----------------------------------------------------------------------------

\title{The umbra of spheres}
% \author{John K. Johnstone\thanks{This 
%    work was partially supported by the National Science Foundation under grant CCR-0203586.}\\
% Computer and Information Sciences\\
% University of Alabama at Birmingham\\
% University Station, Birmingham, AL, USA 35294}

\begin{document}
\maketitle

% -------------------------------------------------------------------------------------

Let A and B be closed, simple surfaces whose interiors have no intersection.
Suppose that we are interested in visual events that abruptly reveal A.

\begin{defn2}
A {\bf ruled surface} is a surface that is generated by sweeping a line through space.
Any line on a ruled surface is called a {\bf generator}.
A {\bf bitangent plane} of A and B is a plane that is tangent to both A and B.
A pair of surfaces may have several one-parameter families of bitangent planes.
A {\bf bitangent developable} of A and B is the envelope of a one-parameter family of 
bitangent planes of A and B.
We will typically define a bitangent developable of A and B by a lofting between 
its curves of tangency with A and B: $(1-s)C_1(t) + sC_2(t)$.
Given a subsegment $S = C_1[a,b]$ of one of the bitangent developable's curves of 
tangency,
the generators through S generate a subpatch $(1-s)C_1[a,b] + sC_2[a,b]$, 
which we call the {\bf S-patch}.
The {\bf A-early patch} of a bitangent developable of A and B is the infinite patch 
starting at the curve of tangency with A and moving away from B.
The {\bf middle patch} of a bitangent developable is the patch between its curves 
of tangency.
The {\bf B-early patch} of a bitangent developable of A and B is the infinite patch 
starting at the curve of tangency with B and moving away from A.
The {\bf middle segment} of a generator of a bitangent developable is the segment between 
its points of tangency with A and B.
A bitangent developable of A and B is {\bf inner} if it separates A and B.
A bitangent developable of A and B is {\bf outer} if A and B lie entirely on the same
side of this bitangent developable.
These developables will be called AB-inner and AB-outer if we want to emphasize
the underlying surfaces.
\end{defn2}

For a pair of spheres, the AB-inner developable is a cone and the AB-outer developable
is a cylinder.

Given a bitangent developable (computed using intersection of tangential surfaces), 
an algorithm for determining the type (inner, outer or neither) of this developable is ---

Consider two spheres A and B.
A visual event for A is defined by the B-early patch of the AB-outer developable.
Consider the addition of a third sphere C to the scene.
The earlier visual event will be moved if and only if C intersects the middle patch
of the AB-outer developable.
Let D be the curve of tangency on C of the BC-inner developable.
Let E be the curve of tangency on C of the AC-outer developable.
Let $D'$ be the subsegment of D that lies inside the AB-outer developable.
Let $E'$ be the subsegment of E that lies inside the AB-outer developable.
Let F be the B-early portion of the $D'$-patch of the BC-inner developable, and
let G be the $E'$-patch of the AC-outer developable.

\begin{claim}
1) When C interferes with the middle patch of the AB-outer developable,
the visual event behind B is expanded to F.
How does it combine with the old visual boundary, 
the B-early patch of the AB-outer developable?

2) This new visual boundary is clipped by G.  
   Specify exactly how.
\end{claim}

Open problem from SoCG04: 
Given two spheres from a set of arbitrary spheres (non-intersecting), 
do these two spheres have a free bitangent plane?
Want a very quick answer.

\end{document}
