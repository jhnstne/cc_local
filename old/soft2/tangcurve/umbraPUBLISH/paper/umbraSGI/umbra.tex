\documentclass[9pt]{article}
\usepackage{times}
\usepackage[pdftex]{graphicx}
\makeatletter
\def\@maketitle{\newpage
 \null
 \vskip 2em                   % Vertical space above title.
 \begin{center}
       {\Large\bf \@title \par}  % Title set in \Large size. 
       \vskip .5em               % Vertical space after title.
       {\lineskip .5em           %  each author set in a tabular environment
        \begin{tabular}[t]{c}\@author 
        \end{tabular}\par}                   
  \end{center}
 \par
 \vskip .5em}                 % Vertical space after author
\makeatother

% default values are 
% \parskip=0pt plus1pt
% \parindent=20pt

\newcommand{\SingleSpace}{\edef\baselinestretch{0.9}\Large\normalsize}
\newcommand{\DoubleSpace}{\edef\baselinestretch{1.4}\Large\normalsize}
\newcommand{\Comment}[1]{\relax}  % makes a "comment" (not expanded)
\newcommand{\Heading}[1]{\par\noindent{\bf#1}\nobreak}
\newcommand{\Tail}[1]{\nobreak\par\noindent{\bf#1}}
\newcommand{\QED}{\vrule height 1.4ex width 1.0ex depth -.1ex\ \vspace{.3in}} % square box
\newcommand{\arc}[1]{\mbox{$\stackrel{\frown}{#1}$}}
\newcommand{\lyne}[1]{\mbox{$\stackrel{\leftrightarrow}{#1}$}}
\newcommand{\ray}[1]{\mbox{$\vec{#1}$}}          
\newcommand{\seg}[1]{\mbox{$\overline{#1}$}}
\newcommand{\tab}{\hspace*{.2in}}
\newcommand{\se}{\mbox{$_{\epsilon}$}}  % subscript epsilon
\newcommand{\ie}{\mbox{i.e.}}
\newcommand{\eg}{\mbox{e.\ g.\ }}
\newcommand{\figg}[3]{\begin{figure}[htbp]\vspace{#3}\caption{#2}\label{#1}\end{figure}}
\newcommand{\be}{\begin{equation}}
\newcommand{\ee}{\end{equation}}
\newcommand{\prf}{\noindent{{\bf Proof}:\ \ \ }}
\newcommand{\choice}[2]{\mbox{\footnotesize{$\left( \begin{array}{c} #1 \\ #2 \end{array} \right)$}}}      
\newcommand{\scriptchoice}[2]{\mbox{\scriptsize{$\left( \begin{array}{c} #1 \\ #2 \end{array} \right)$}}}
\newcommand{\tinychoice}[2]{\mbox{\tiny{$\left( \begin{array}{c} #1 \\ #2 \end{array} \right)$}}}
\newcommand{\ddt}{\frac{\partial}{\partial t}}
\newcommand{\Sn}[1]{\mbox{{\bf S}$^{#1}$}}
\newcommand{\calP}[1]{\mbox{{\bf {\cal P}}$^{#1}$}}

\newtheorem{theorem}{Theorem}	
\newtheorem{rmk}[theorem]{Remark}
\newtheorem{example}[theorem]{Example}
\newtheorem{conjecture}[theorem]{Conjecture}
\newtheorem{claim}[theorem]{Claim}
\newtheorem{notation}[theorem]{Notation}
\newtheorem{lemma}[theorem]{Lemma}
\newtheorem{corollary}[theorem]{Corollary}
\newtheorem{defn2}[theorem]{Definition}
\newtheorem{observation}[theorem]{Observation}

% \font\timesr10
% \newfont{\timesroman}{timesr10}
% \timesroman

\newif\ifJournal
\Journalfalse
\newif\ifTalk
\Talkfalse
\setlength{\headsep}{.5in}
\markright{Umbra: \today \hfill}
\pagestyle{myheadings}

\setlength{\oddsidemargin}{0pt}
\setlength{\topmargin}{0in}	% should be 0pt for 1in
\setlength{\textheight}{8.6in}
\setlength{\textwidth}{6.875in}
\setlength{\columnsep}{5mm}	% width of gutter between columns

\title{Global illumination from a smooth area light:\\the discontinuity mesh of a smooth scene in 2-space}
% The geometry of shadow:\\the umbra cast by a curved light source\\in a scene of curved obstacles}
% Shadows in flatland
% Bitangency in a scene of smooth curves,\\with applications to visibility}	
% Tangential curves for bitangency, with applications to smooth versions of the convex hull and visibility graph, and anticipations of shading from smooth lights in 3d	
\author{J.K. Johnstone\thanks{This work was supported by the National Science Foundation under grant CCR-0203586.}\\
Geometric Modeling Lab\\
Computer and Information Sciences\\
University of Alabama at Birmingham\\
University Station, Birmingham, AL, USA 35294}

\begin{document}
\maketitle

% -----------------------------------------------------------------------------

\begin{abstract}

We build the complete discontinuity mesh for a smooth scene in 2-space,
which solves the lighting problem for area lights in a smooth environment.
This is similar (equivalent?) to the construction of the aspect graph
of a smooth scene in 2-space.

The umbra cast by a smooth area light source in a 2d scene of smooth
obstacles is studied.
A different approach is taken from previous work in lighting.
First, the umbra is decomposed into (possibly overlapping) subumbrae cast
by each obstacle, and these regions are fully defined geometrically.
Second, the area light source and scene of obstacles are both smooth,
bounded by freeform parametric curves.
Since the geometric analysis of smooth environments is challenging,
the analysis is restricted to 2d and 2$\frac{1}{2}$d environments.

The umbra cast by an obstacle A is computed in two stages.
First, the umbra is computed as if A was the sole obstacle in the scene.
Next, this umbra is refined to incorporate the context of neighbouring
obstacles using two sweeps.
The first sweep may expand the umbra cast by A due to obstacles that
lie between A and the light, and is computed by sweeping a tangent of A.
The second sweep may shrink the umbra cast by A and is computed by
sweeping a tangent of the light.
Both sweeps are discretized by observing that visual events always occur
at bitangents.
The umbra cast by the obstacle A is bounded by a collection of bitangents
between A, the light, and intervening obstacles.
Using the above analysis, the umbra of the scene is 
simply the union of the umbrae cast by each obstacle.
\end{abstract}


% full definition of the region, not point-by-point query;
% divide and conquer strategy

% -----------------------------------------------------------------------------

\section{Introduction}

\begin{quote}
\ldots but everything exposed by the light becomes visible \ldots (Ephesians 5, 13)
\end{quote}

We are interested in the following analytic lighting problem:
the lighting by a smooth area light source 
of a room full of smooth obstacles.
Previous work on lighting has primarily concentrated on the simpler problem of
polyhedral obstacles and point or polygonal light sources.
{\bf The umbra cast by a smooth area light in a 2d scene
of smooth obstacles, where smooth means bounded by $C^1$-continuous curves,
is already a very interesting problem, and is the focus of this paper.}
In this paper, we compute the umbra cast in 2-space 
% or 2$\frac{1}{2}$-space: not really since angle looking up changes visibility
when the light and the obstacles % $\{B_i\}_{i=1}^n$ 
in the scene are bounded by smooth closed curves,
already a challenging problem that paves the way for an attack 
on the general lighting problem in 3-space.
% This is a very challenging problem, too broad to consider immediately 
% in its fullest generality.
% Problem: computing the umbra cast by a smooth area light source in a smooth 
% scene in 2-space or 2$\frac{1}{2}$-space.
% This is already an interesting and challenging problem.
The umbra of a scene is the area that does not see any of the light.
% collection of regions in the scene
Umbral computation can be interpreted
more generally as the computation of regions that are entirely invisible
to a distinguished object, an important visibility question.
For example, if the light is reinterpreted as the camera,
the umbra defines parts of the scene that do not need to be rendered.
If the light is reinterpreted as a robot, the umbra defines parts of the
scene that do not see the robot.
Visibility in 2-space is particularly important in robotics, as the robot
moves on a 2-dimensional floor.
% The theory developed in this paper also paves the way for an attack on the 
% general lighting problem in 3-space.

\ifJournal
	\begin{defn2}
	The {\bf umbra of a scene} is the collection of regions 
	in the scene that do not see any of the light.
	\end{defn2}
\fi

The main characteristic of this paper that distinguishes it from much work on
lighting is its concentration on a smooth light and smooth obstacles.
Its other distinguishing characteristic is that we are interested in 
defining the global structure of the umbra.
(THIS ISN'T DISTINGUISHING: OTHER WORK ON DISCONTINUITY MESH DOES THIS.)
That is, we are interested in defining the umbra region by region,
rather than point by point.
It is popular to define lighting point by point,
answering the question 'is point P in the umbra?'.
We are interested in a deeper understanding of the structure of the umbra,
through a direct definition of the area that defines the umbra.
This is more intellectually satisfying but certainly more challenging as well.

\ifTalk
This distinction between point and regional approaches is analogous to the 
two ways for computing a tangent on a Bezier curve:
computing the tangent at a point
or computing the hodograph of the curve, representing all of the tangents.)
\fi

% ------------------------------------------------------------------------------

{\tiny
\section{Earlier work}

Drettakis and Fiume \cite{drettakis94}
build a discontinuity mesh for a polygonal light and polyhedral scene.

We are not aware of any work on a geometric (region-based) definition
of the shadow cast by a smooth area light source
in a scene of smooth obstacles.

Polygonal lighting articles (I believe all of these are point-based).

Tony Woo's work on reachability analysis (using Gaussian sphere).

Visibility complex/skeleton.

Durand review of visibility.

NSF review of literature.

Artistic literature on shadow and lighting.

How does antipenumbra paper relate?

Bitangents are required throughout the paper, since they define
umbral boundaries.
We have developed efficient algorithms for computing bitangents
using intersection in dual space \cite{jj01,jj02}.
}	% end of \small

% \clearpage

% ------------------------------------------------------------------------------

\section{Taking umbrage} % A divide and conquer strategy

We will compute the umbra of a scene using a divide and conquer strategy,
one obstacle at a time.
First, we compute the umbra cast by each obstacle in isolation
(the local umbra).
We then refine and expand the umbra of each obstacle
by taking other obstacles into account (the global umbra of an obstacle).
Finally, the umbra of the scene is defined in terms of these obstacle umbrae.

The computation of the umbra one obstacle at a time has the advantage
of simplicity, parallelizability, and incremental computation.
The concentration on one obstacle at a time is
an important simplification to the structure of the computation,
since it allows one obstacle to (temporarily) dominate all of the other obstacles.
That is, in the computation of the umbra cast by an obstacle A, 
all obstacles are of interest only through their relationship
to A or the light, rather than their mutual interrelationships.
This dominance of one obstacle yields important simplifications when
reasoning about lighting.
Another benefit is parallelizability:
the computation of the umbra of an obstacle A can be performed at the same
time as the computation of the umbra of an obstacle B.
This can be an important factor in a large scene.
% This keeps less balls up in the air at once.
% Thus, we only need to consider n-to-1 relationships, not n-to-n relationships.
A third benefit of computing the umbra obstacle by obstacle
is that a point of the umbra will know which obstacles are
responsible for its inclusion in the umbra.\footnote{The umbra cast by 
	A and the umbra cast by B may certainly overlap.}
This will allow efficient incremental updates to the umbra upon the addition
or removal of an obstacle.


SECTION REVIEW:

Section~\ref{sec:innersweep};
Section~\ref{sec:outersweep};
While the first sweep refines the local umbra bitangent,
the second sweep refines the global umbra bitangent
just computed in the first sweep.
% ------------------------------------------------------------------------------

% \clearpage

\section{A glossary of terms}

\begin{defn2}
\label{defn:surround}
A {\bf surrounds} B if B lies entirely inside the convex hull of A.
\end{defn2}

Since there is no umbra if the light surrounds the obstacle, 
we can assume without loss of generality 
that the light does not surround the obstacle.
If the obstacle surrounds the light (Figure~\ref{fig:G2}c),
the theory is more complicated.
% As we develop the theory, we shall point out how it simplifies when
% the obstacle does not surround the light,
% which is true in the vast majority of cases.
Since the obstacle does not surround the light in the vast majority
of cases, we postpone this unusual case until Section~\ref{sec:surround}
and assume that the obstacle does not surround the light.

\begin{defn2}
\label{defn:outer}
Let A and B be closed curves, where neither curve surrounds the other.
Let $t$ be a tangent of A,
and let $T$ be a bitangent of A and B with points of tangency $P_A \in A$ and 
$P_B \in B$.

\begin{itemize}
\item $t$ is {\bf extremal} if $t$ does not intersect A
	(i.e., A lies in one of the halfplanes defined by $t$).
\item If $t$ is extremal, the {\bf inside} of $t$ is
	the halfplane that contains A.
% \item Case 1: A does not surround B and B does not surround A.
% \begin{itemize}
\item $T$ is {\bf extremal} if $T$ does not intersect A or B.
\begin{itemize}
\item $T$ is {\bf outer} if it is extremal and 
A and B lie on the same side of $T$ ({\bf Figure G4}).
\item $T$ is {\bf inner} if it is extremal and
A and B lie on opposite sides of $T$.
\end{itemize}
% \end{itemize}
\item
The {\bf early segment} of T is the component of T between A and B.
% It may be abbreviated early(T).
\item 
If B is the light,
the {\bf late segment} of T is the infinite ray of T starting at
the point of bitangency with A and moving away from the light.
\item
If B is an obstacle that lies between A and the light L\footnote{This 
	restriction on B is necessary to distinguish A from B.}
and T intersects L,\footnote{This restriction on T is necessary
	to guarantee that the direction 'moving away from the light'
	is well defined.}
the {\bf late segment} of T is the infinite ray of T starting at the point
of bitangency with A and moving away from the light.
\item If B is the light,
T is a {\bf piercing bitangent} if \seg{P_A P_B} is free
(does not intersect A or B) but \lyne{P_A P_B} intersects A,
and $P_A$ is not an inflection point.
\item If B is the light,
	T is a {\bf self-shading bitangent} if T is a piercing bitangent
	and A and B lie on the same side of T in the neighbourhood of 
	the points of tangency.
\end{itemize}
\end{defn2}

\begin{lemma}
Let A and L be smooth closed curves, where neither A nor L surrounds the
other.\footnote{This lemma extends to the surrounding case with an
	appropriate definition of inner and outer bitangent, as shown in
	Section~\ref{sec:surround}.}
$A$ and $L$ have exactly 2 outer bitangents and 2 inner bitangents.
\end{lemma}
\ifJournal
\prf
This lemma relies on the fact that the inner and outer tangents are extremal.
There is an alternative definition of inner and outer bitangents
that has a more intuitive appeal: the inner and outer bitangents of A and B
are the only bitangents of the convex hull of A and the convex hull of B.
That is, compute the convex hull of A and the convex hull of B, then take
bitangents: there are only 4 bitangents (proof?) and these are the inner and
outer bitangents of A and B.
\QED
\else
\prf
Proofs are omitted for lack of space.
\QED
\fi

See Figure~\ref{fig:inout}.

	% umbra data/umbra/vg9.pts
\begin{figure}[h]
\begin{center}
\includegraphics*[scale=.25]{img/vg9bitang.jpg}
\includegraphics*[scale=.25]{img/vg9inner.jpg}
\includegraphics*[scale=.25]{img/vg9outer.jpg}
\end{center}
\caption{(a) The bitangents
	 (b) The inner bitangents 
	 (c) The outer bitangents
	 (d) The self-shading bitangents}
\label{fig:inout}
\end{figure}

In all figures, the light is drawn in black and the obstacles in grey.
In the following sections, A will be a distinguished obstacle for which
we are computing the umbra, B another obstacle, and L the light.

% ------------------------------------------------------------------------------

% \clearpage

\section{The local and global umbra of an obstacle}
\label{sec:localumbra}

\begin{defn2}
The {\bf local umbra of an obstacle} $A$ is the umbra that would be
cast if $A$ were the only obstacle in the scene.
A {\bf local umbral bitangent} of A is a bitangent 
that forms the boundary of the local umbra of A.
\end{defn2}

The local umbra cast by an obstacle is a region bounded by the obstacle
and two of its bitangents to the light.
Which bitangents, and which parts of these bitangents (Figure~\ref{fig:inout}a)?
\ifTalk
We are looking for bitangents that do not intersect the obstacle or light.
However, we must relax our constraints when the obstacle surrounds the light.
\fi
Figure~\ref{fig:G2} illustrates the answer, which is captured by the following lemma.

% ------------------------------------------------------------------------------

% It is the outer bitangents of A and L that bound the local umbra of A
% (Figure~\ref{fig:G2}).
% We capture this fact in a theorem.

\begin{theorem}
\label{thm:localumbra}
Let A be an obstacle, L the light, and
$T_1$ and $T_2$ the late segments of their outer bitangents.
The local umbra of A is the region bounded by $T_1$, $T_2$, 
and the relevant part of A.
% and inside both $T_1$ and $T_2$ in the
% neighbourhood of their points of tangency with $A$.
\end{theorem}
\ifJournal
\prf
Every other part of A sees some of L.
Argue that the light becomes visible as you cross these bitangents.
\QED
\fi

% Put another way, 
\begin{corollary}
All local umbra bitangents of A are outer bitangents of A and L.
\end{corollary}

	% umbra data/umbra/vg9.pts
\begin{figure}[h]
\begin{center}
\includegraphics*[scale=.25]{img/vg9localumbra.jpg}
% \includegraphics*[scale=.25]{img/ob1-bitang.jpg}
% \includegraphics*[scale=.25]{img/ob1.jpg}
\end{center}
% \centerline{\epsfig{figure=img/ob1-bitang.ps,height=1.542in,width=1.528in}}
% 20% reduction
% \centerline{\epsfig{figure=img/ob1.ps,height=1.567in,width=1.553in}}
% 20% reduction
\caption{The local umbra of an obstacle}
\label{fig:G2}
\end{figure}

\Comment{
	If A does not surround L,
	the local umbra is bounded if and only if the late segments intersect
	(Figures~\ref{fig:G2} and \ref{fig:globalumb3}).
	If A surrounds L,
	the local umbra is unbounded.
}

\ifJournal
\vspace{.1in}

{\bf Algorithm:}
\begin{enumerate}
\item Find the bitangents of A and L.
\item Find the two bitangents that are L-outer.
\item Find the late segments of these outer bitangents and their insides.
\end{enumerate}

% \begin{rmk}
% The case described in this section actually subsumes the case of 
% Section~\ref{sec:notsurround}, since an outer bitangent is always
% a pseudo-outer bitangent too.
% That is, you can look for pseudo-outer bitangents
% in all cases.
% We have presented them separately
% since the definition of pseudo-outer bitangents is less intuitive
% without an earlier understanding of outer bitangents,
% and the simpler outer bitangent is sufficient for the vast majority of cases.
% \end{rmk}

\begin{rmk}
More computationally, the local umbra is defined as follows.
Let the two L-outer bitangents be \lyne{P_L P_A} and \lyne{Q_L Q_A},
where $P_L,Q_L \in L$ and $P_A,Q_A \in A$.
If the rays \ray{P_L P_A} and \ray{Q_L Q_A} intersect,
let $P=Q$ be this intersection;
otherwise let $P$ and $Q$ be the points 
at infinity on the rays \ray{P_L P_A} and \ray{Q_L Q_A}, respectively.

The local umbra is the region bounded by the (potentially infinite)
segments \seg{P_A P}, \seg{Q_A Q},
and the curve segment \arc{P_A Q_A} of A.
\end{rmk}

\begin{implementation}
In our implementation, we build all of the scene inside a predefined cube.
% Equivalently, one can define the scene arbitrarily and then find a bounding box.
In the test for bitangent extremality (Definition~\ref{defn:outer}),
the intersection of an infinite line or ray
with a curve then reduces to the intersection of a finite segment with a curve
(the part of the ray/line inside the box).
This is simpler and more robust.
\end{implementation}
\fi

% ------------------------------------------------------------------------------------

% \clearpage

% \section{The global umbra of an obstacle}

\begin{defn2}
\label{defn:global}
The {\bf global umbra of an obstacle $A$} is the umbra cast by $A$
when all other obstacles are taken into account.
A {\bf global umbral bitangent} of A is a bitangent 
that forms the boundary of the global umbra of A.
\end{defn2}

The local umbra captures the umbra of an obstacle in isolation (Figure~\ref{fig:vg900}b).
However, the umbra cast by a single obstacle can be influenced by other obstacles.
The global umbra expands the local umbra to consider these other obstacles in the scene
(Figure~\ref{fig:vg900}c).
In many cases, the global umbra is the same as the local umbra.
We shall first determine when the global umbra differs from the local umbra.
Then, we shall show how the expanded umbra is defined in the cases when
the global umbra is larger than the local umbra.

In Section~\ref{sec:localumbra}, we saw that
the local umbra is defined by outer bitangents of A.
If no obstacle interferes with the early segments of these outer bitangents,
the global umbra is identical to the local umbra.
However, as an obstacle B begins to interfere with these early segments,
the umbra of A is stretched by B (Figure~\ref{fig:vg900}).

\begin{lemma}
\label{lem:differs}
The global umbra of A differs from the local umbra of A
if and only if an obstacle B intersects the early segment
of an outer bitangent of A and L.
\end{lemma}
\prf
To affect the global umbra of A,
an obstacle must conspire with A to block more of the light.
It cannot do so if it lies behind A or to its left or right.
Even an obstacle that lies strictly between the outer bitangents of A
does not directly affect A's umbra, although it may do so by combining
with other obstacles as we shall see in Section~\ref{sec:innersweep}, 
one of which must satisfy the condition of this lemma.
The discussion of the inner sweep in Section~\ref{sec:innersweep}
will clarify how the obstacle B expands the umbra.
\QED

	% umbra data/umbra/vg900.pts
\begin{figure}[h]
\begin{center}
\includegraphics*[scale=.25]{img/vg900outer.jpg}
\includegraphics*[scale=.25]{img/vg900localumbra.jpg}
\includegraphics*[scale=.25]{img/vg900globalumbra.jpg}
\end{center}
\caption{The crossing of an outer bitangent of A and L 
	     marks a difference between the local and global umbra
	 (a) the outer bitangents of A and L
% An obstacle with other obstacles interfering with its early outer bitangent
	 (b) the local umbra of A
 	 (c) the global umbra of A
}
\label{fig:vg900}
\end{figure}

The global umbra is bounded by bitangents, just like the local umbra.
However, the set of bitangents that can form the 
boundary of the global umbra is richer than the local umbra.
In addition to the outer bitangents of A and L used by the
local umbra, the global umbra may be bounded by inner bitangents of A and B
(where B is an obstacle intervening between A and L)
and outer bitangents of B and L.

% ------------------------------------------------------------------------------------

% \clearpage

\section{Refining the local umbra}

We define the global umbra of an obstacle A by dynamically expanding its local umbra,
using two sweeps of a tangent across interfering obstacles.
% There are two types of refinement of the umbra.
Both sweeps start with an outer bitangent of the obstacle A and the light L.
% which can be interpreted either as a tangent of A or as a tangent of L.
The first sweep moves this tangent smoothly within the tangent space of A.
The second sweep moves this tangent smoothly within the tangent space of L,
in the opposite direction.
Both sweeps move the tangent across all obstacles intervening between A and L.
The first sweep continues until the tangent sees the light (or has swept
completely past it).
The second sweep continues until the tangent sees the umbral boundary added
in the first sweep (or has swept completely past it).

These continuous tangent sweeps are transformed into discrete sweeps
by observing that the sweep must always stop at a bitangent.
This allows the sweep to jump directly to the next bitangent candidate
and test for completion.
The resulting sweep involves a finite set of jumps between bitangents,
analogous to the line sweep of computational geometry \cite{preparataShamos}.
In the first sweep,
the inner bitangents of A with a blocking obstacle B define the possible
stopping positions of the sweeping tangent.
In the second sweep, the outer bitangents of L with a blocking obstacle B
define the possible stopping positions.
The next two sections elaborate on these two sweeps that refine
the local umbra into the global umbra.

% duplicate of fig:vg900
	% ---
	% umbra data/globalumb2.pts
	% umbra data/globalumb2.pts
\begin{figure}[h]
\begin{center}
\includegraphics*[scale=.25]{img/globalumb1outer.jpg}
\includegraphics*[scale=.25]{img/globalumb1gu.jpg}
\includegraphics*[scale=.25]{img/globalumb2gu.jpg}
\includegraphics*[scale=.25]{img/globalumb2bgu.jpg}
\includegraphics*[scale=.25]{img/globalumb2cgu.jpg}
% \includegraphics*[scale=.25]{img/globalumb2localumbra.jpg}
% \includegraphics*[scale=.25]{img/globalumb2gumbUnfill.jpg}
% \includegraphics*[scale=.25]{img/globalumb2globalumbra.jpg}
% \includegraphics*[scale=.25]{img/globalumb2-local.jpg}
% \includegraphics*[scale=.25]{img/globalumb2.jpg}
% \includegraphics*[scale=.25]{img/globalumb2-PO.jpg}
\end{center}
\caption{An obstacle interfering with the outer bitangent expands the umbra}
% (a) The crossing of an outer bitangent of A and L 
% 	     marks a difference between the local and global umbra
%	 (b) Local umbra 
%	 (c-d) Global umbra}
\label{fig:globalumb2}
\end{figure}

% \clearpage

\subsection{The inner sweep: looking for the light}
\label{sec:innersweep}
% expanding the local umbra while looking for the light}

The first sweep moves in the tangent space of A,
looking for the light L by sweeping across obstacles that block A from L
(Figure~\ref{fig:innersweep}).
Its purpose is to refine a local umbra bitangent into a global umbra
bitangent.
Its effect is to expand the local umbra.
We call this sweep an {\bf inner sweep},
because the local umbra bitangent will be repeatedly replaced by 
inner bitangents, and because the sweep is towards the interior of the light.

	% umbra data/umbra/globalumb3.pts
\begin{figure}[h]
\begin{center}
\includegraphics*[scale=.25]{img/globalumb3innersweep1.jpg}
\includegraphics*[scale=.25]{img/globalumb3innersweep2.jpg}
\includegraphics*[scale=.25]{img/globalumb3innersweep3.jpg}
\end{center}
\caption{An inner sweep
	 (a) T = outer bitangent of A and L
	 (b) T = inner bitangent of A and B1
	 (c) T = inner bitangent of A and B2}
\label{fig:innersweep}
\end{figure}

Consider a local umbra bitangent T, 
which is necessarily an outer bitangent of A and L (Theorem~\ref{thm:localumbra}).
Suppose that an obstacle B intersects T's early segment,
% the early segment of an outer bitangent T of A and L,
implying that the global umbra is different from the local umbra
(Lemma~\ref{lem:differs}).
See Figure~\ref{fig:innersweep}a.
T will be swept in the tangent space of A, 
towards the inside of the light, until it sees the light.
This continuous sweep can be realized discretely by
repeatedly replacing T by an inner bitangent of A and B,
until there is no obstacle B that hits this inner bitangent
between A and the light, or the sweep passes the light.
Since the inner sweep is looking for the light,
it will stop if it sweeps past the light.
% (without ever seeing it).
This is marked by sweeping past an inner bitangent of A and L (Figure~\ref{fig:maximalumbra}).
The global umbra is never larger than 
the region bounded by the late inner bitangents of A and L (but often smaller).
% This implies that the maximal size of the global umbra is the region 
% bounded by the late inner bitangents of A and L 
% Recall that the minimal size of the global umbra is the region
% bounded by the late {\em outer} bitangents of A and L.

Here is the algorithm for an inner sweep from the local umbra bitangent T of obstacle A.

% Put another way, the global umbra of A is no larger than the region
% bounded by the late inner bitangents of A and L,
% so the expansion of the inner sweep must stop there.

\vspace{.2in}

\centerline{innersweep(T,A)}

\begin{enumerate}
\item T = outer bitangent of A and L
\item while an obstacle B interferes with T on the segment between A and L
\begin{enumerate}
\item T = inner bitangent of A and B (the closest from the present T 
	in the sweep direction)
\item if T has swept past the light (i.e., past an inner bitangent of A and L)
\begin{enumerate}
\item  T = this inner bitangent of A and L
\item  return
% done = true
\end{enumerate}
\end{enumerate}
\end{enumerate}

If more than one obstacle B interferes in step 2,
an arbitrary choice of B is made.

This inner sweep replaces a local umbra bitangent by a global umbra bitangent.
Once both local umbra bitangents of A have been refined
to global umbra bitangents, their late segments define new boundaries
of the global umbra.
% just as with the local umbra.
However, the second sweep may further refine this boundary.

\Comment{
\centerline{innerSweep (T, A, L)}
\begin{enumerate}
\item 
	G = T;\ \ \ done = false;\ \ \ first = true;
	% \item While the umbral bitangent T moves (has been freshly updated on previous step):
\item 	
	sweep direction = direction towards inside of light from T
\item 
	while (!done)
\begin{enumerate}
\item   if (first)
\begin{enumerate}
\item	SEG = early segment of G;\ \ \ first = false;
\end{enumerate}
\item	else
\begin{enumerate}
\item	$G_A = $ G's point of bitangency with A
\item	$G_I = G \cap L$
\item	SEG = segment between $G_A$ and $G_I$
\end{enumerate}
\item 
	$\mbox{HIT} := \{B_i \neq A : B_i \mbox{ is an obstacle that intersects
	SEG} \}$
\item 
	if $\mbox{HIT} = \emptyset$, done = true
\item   
	else
\begin{enumerate}
\item
	choose an arbitrary $B'$ from HIT.
\item 
	\ [search for the light, around the blocker] 
	G = first A-inner bitangent of A and $B'$ that is found
	in the sweep direction.
\end{enumerate}
\end{enumerate}
\item	return(G)
\end{enumerate}
}

% (That is, any umbra outside this range is generated by a different obstacle.)
% Outside of this range, A has no impact: it doesn't block any of the light.
% So A can be ignored outside this region.

% The sweep of A's tangent can be stopped at an inner tangent of A and L.

\ifTalk
Animation of sweep.
\fi

	% umbra data/umbra/maxumbra.pts (with 'Maximal umbra' option)
\begin{figure}[h]
\begin{center}
\includegraphics*[scale=.25]{img/maxumbra.jpg}
\end{center}
\caption{The maximal size of a global umbra}
%  (bounded by late inner bitangents of A and L)}
% using one obstacle B that blocks the entire light 
% and big enough that its outer bitangents with the light 
% also don't intersect the late inner of A and L
\label{fig:maximalumbra}
\end{figure}

	% umbra data/umbra/globalumb3.pts
\begin{figure}[h]
\begin{center}
\includegraphics*[scale=.25]{img/globalumb3luWithSweep.jpg}
\includegraphics*[scale=.25]{img/globalumb3lu.jpg}
\end{center}
\caption{The umbra after refinement by the inner sweep}
\label{fig:umbraafterinnersweep}
\end{figure}

%%%%%%%%%%%%%%%%%%

% \clearpage

\subsection{The outer sweep: looking for the global umbra from the light}
\label{sec:outersweep}

While the first sweep expands the umbra, the second sweep shrinks it.
Its purpose is to refine a global umbra bitangent just computed by an
inner sweep, by clipping it as it reappears to the light.
Rather than sweeping a tangent along the obstacle,
it sweeps a tangent along the light.

	% umbra data/umbra/globalumb7.pts
\begin{figure}[h]
\begin{center}
\includegraphics*[scale=.25]{img/globalumb7outersweep1.jpg}
\includegraphics*[scale=.25]{img/globalumb7outersweep2.jpg}
\includegraphics*[scale=.25]{img/globalumb7outersweep3.jpg}
\includegraphics*[scale=.25]{img/globalumb7gu.jpg}
\end{center}
\caption{The outer sweep of T (a) T = outer bitangent of L and A
	 (b) T = outer bitangent of L and B1
	 (c) T = outer bitangent of L and B2
	 (d) the resulting global umbra}
\label{fig:severalstages}
\end{figure}

Consider a local umbra bitangent T and the global umbra bitangent G
computed from T by the inner sweep.
Although G marks a boundary of the shadow,
the light will often become visible again at some point of G,
on the other side of the blocking obstacles.
The purpose of the second sweep is to find this point of reappearance
into light and clip G there.
To accomplish this, a tangent of the light L is swept across 
all obstacles that block it from G.
As with the inner sweep,
the tangent is initialized to the local umbra bitangent T, 
but we now sweep in the opposite direction (towards the outside of A).
At the point when L's tangent sees G, 
this tangent is added to the boundary of the global umbra, effectively clipping G.

% While the inner sweep expands the umbra by looking for the light,
% the second sweep shrinks the umbra by looking for the umbra from the light.
% The job of 
% We know that this bitangent defines a boundary of light
% visibility at $P_A$ (its point of bitangency with A) as it leaves A
% on the late segment.
% But the light will often become visible once again as the bitangent
% crosses out of darkness on the other side of the obstacles.
% Therefore, this bitangent needs to be clipped.

% The search for G's reentry into light must be done from the perspective of the 
% light, using a sweep of the light's tangent looking for the newly
% computed global umbra bitangent T.

As with the inner sweep, we leverage the fact that changes of visibility
must occur at discrete bitangent events
to convert the continuous sweep into a discrete sweep.
Recall that we can assume that an obstacle B hits the early segment of 
the local umbra bitangent T,
otherwise the global umbra is the same as the local umbra
(and neither inner nor outer sweep is necessary).
T is repeatedly replaced by the outer bitangent of B and L,
until the early segment of this outer bitangent is free of obstacles B (Figure~\ref{fig:severalstages}).
The sweep can be stopped as soon as the sweeping tangent passes G,
since it is then guaranteed to never see G.
Since the tangent is repeatedly replaced by outer
bitangents, and it sweeps towards the outside of A,
the sweep is called an {\bf outer sweep}.
Here is the algorithm for an outer sweep from the local umbra bitangent T of obstacle A.

% \item[iii.]	If angle(H) 'exceeds' (in the direction of current travel) angle(G), return (NULL).



\vspace{.2in}

\centerline{outersweep (T,A)}

\begin{enumerate}
\item T = outer bitangent of A and L
\item G = innersweep(T,A)
\item B = obstacle that interferes with T on the segment between A and L
\item do
\begin{enumerate}
\item T = outer bitangent of B and L (the closest from the present T in the
	sweep direction)
\item if T has swept past G, return since G does not need to be refined
\end{enumerate}
while a new obstacle B interferes with T on the segment between the old B and L
\item clip G by T 
\item return(T)
\end{enumerate}

The outer sweep further refines the global umbra.
The output of the outer sweep algorithm is a new boundary of the global
umbra, which should start after its intersection with 
innerSweep(T,A) (Figure~\ref{fig:severalstages}).
The {\bf inside} of this new global umbra bitangent 
is the side that contains the light L.
Its {\bf late segment} is the segment after its intersection with 
innerSweep(T,A).

% {\bf outerSweep(T,A,L)}, the output of this second sweep from T is a tangent of L.

% Notice that this may introduce a bend into the umbral boundary.
% (two linked lines)

\Comment{
\begin{figure}[h]
\begin{center}
\includegraphics*[scale=.25]{img/globalumb7gumbWithOuter.jpg}
\includegraphics*[scale=.25]{img/globalumb7gumb.jpg}
% \includegraphics*[scale=.25]{img/globalumb4umbra.jpg}
% \includegraphics*[scale=.25]{img/globalumb7umbra.jpg}
\end{center}
\caption{The outer sweep further refines the global umbra (add a figure before these two
	 showing the global umbra after the inner sweep but before the refinement
	 of the outer sweep}
\label{fig:outersweep}
\end{figure}
}

% \clearpage

\Comment{
\centerline{outerSweep (T, A, L)}
\begin{enumerate}
\item	H = T;\ \ G = innerSweep (T,A,L);\ \ done = false;\ \ first = true;
\item	sweep direction = direction towards outside of A from T
\item   while (!done)
\begin{enumerate}
\item   if (first) 
\begin{enumerate}
\item	SEG = early segment of H; \ \ \ first = false;
\end{enumerate}
\item	else
\begin{enumerate}
\item	$H_L = $ H's point of bitangency with the light
\item	$H_I = G \cap H$
\item	SEG = segment between $H_L$ and $H_I$
\end{enumerate}
\item	HIT := $\{ B_i: B_i \mbox{ is an obstacle that intersects SEG}\}$
\item   if $\mbox{HIT} = \emptyset$, done = true;
\item 	else
\begin{enumerate}
\item	choose an arbitrary $B'$ from HIT
\item   \ [search for G, around the blocker] H = first L-outer bitangent of L and $B'$ that is found
	in the sweep direction.
\end{enumerate}			
\end{enumerate}
\item 	return(H)
\end{enumerate}
}

% ----------------------------------------------------------------------------------

\section{The self-umbra: the shadow cast by an object upon itself}

Use piercing bitangents.
The region bounded by the obstacle and the part of the piercing
bitangent between the point of tangency and first intersection
is a portion of the umbra cast by the obstacle onto itself,
called the {\bf local self-umbra}.

% ----------------------------------------------------------------------------------

\section{The global umbra of an obstacle and a scene}

% The boundary of the global umbra is defined by refining each bitangent
% of the boundary of the local umbra, using an inner sweep,
% followed by refining this new boundary, using an outer sweep.

The global umbra of the obstacle A is now defined by applying
an inner and outer sweep to the two bitangents of its local umbra,
generating up to four global umbra bitangents.
Let $T_1$ and $T_2$ be the two outer bitangents of A and L.
The global umbra of A is defined by A and the four bitangents
innerSweep($T_1$,A), innerSweep($T_2$,A),
outerSweep($T_1$,A), and outerSweep($T_2$,A),
lying to the inside of their late segments (Figure~\ref{}).
% In other words, the global umbra of A is the polygon defined by the intersection
% of the inside halfspaces of innerSweep($T_1$,A), innerSweep($T_2$,A),
% outerSweep($T_1$,A), and outerSweep($T_2$,A).
% (See \cite{preparataShamos} for a discussion of the
% intersection of halfspaces, including algorithms.)
The global umbra of an obstacle is pseudo-convex: convex except possibly
for its curved boundary.
If the umbra is restricted to a polygonal room,
the room wall's halfspaces are also used to restrict the umbra.

Notice that an obstacle B may affect the global umbra of an obstacle A
without lying between A and the light (Figure~\ref{see drawn example}).
Nevertheless, a space-partitioning scheme (e.g., octrees) can still 
be used to constrain
the obstacles that need to be considered in the computation of the global
umbra of an obstacle A.
For example, obstacles on the opposite side of the light from A 
do not have to be considered in the computation of the global umbra of A.
Precisely, the obstacles inside the following region can affect the 
global umbra of A: the region spanned by the tangents of L parallel to 
the inner bitangents of A and L.
In particular, the local umbra after the refinement of the inner sweeps
is no larger than the region spanned by the inner bitangents I of A and L,
and once an outer bitangent of B and L does not intersect
(after it sweeps past parallel with I)
it cannot affect this umbra by clipping (Figure~\ref{active region drawing}).

\ifTalk
We are computing dawns and sunsets, where the light (sun) appears or disappears
over the horizon of an obstacle.
(just as, when flying around the Earth in an airplane, one may pass into night
but then, as you continue travelling, pass back into day).
\fi

The {\bf umbra of a scene} is the union of the global umbrae of the
obstacles in the scene (Figure~\ref{fig:scene1}).
% An {\bf inner umbral bitangent} of A is
% a global umbra bitangent of A generated by an inner sweep.
% An {\bf outer umbral bitangent} of A is
% a global umbra bitangent of A generated by an outer sweep.
	% % from a local umbra bitangent of A 

Figure~\ref{fig:scene2} illustrates the effect of adding an obstacle to a scene.

% \subsubsection{Penumbrae conspiring to create umbra}

% \vspace{.5in}

\ifJournal
\begin{rmk}
Care must be taken if obstacles touch.
The proposed solution inherently assumes nontouching obstacles.
Since the proposed solution is correct in the limit as the obstacles approach
but do not yet touch, the touching solution should be realized as this
limit rather than computed directly.

% NOT A PROBLEM: in the limit (when obstacles approach but never touch) our solution is correct.
MUST SIMPLY TREAT TOUCHING OBSTACLES DIFFERENTLY.
Consider a light surrounded by circular obstacles, which abut, allowing no
light to pass their ring.
This appears to be a vivid example of the insufficiency of local umbrae for computing
the global umbra.
Notice that the stretching of the umbra addressed in the previous section
is not the cause of the larger global umbra, since no obstacle intersects
another obstacle's PO bitangents.
Instead, this is a case of the penumbrae of obstacles conspiring together
to create umbra.
This is a more subtle umbra to compute, using .3 + .7 = 1 ideas,
rather than 1 + anything else = 1.
\end{rmk}
\fi

	% umbra data/umbra/vg9001.pts
\begin{figure}[h]
\begin{center}
\includegraphics*[scale=.25]{img/vg9001bitang.jpg}
\includegraphics*[scale=.25]{img/vg9001localumbra.jpg}
\includegraphics*[scale=.25]{img/vg9001globalumbra.jpg}
\end{center}
\caption{The global umbra of a scene: (a) its bitangents (b) its local umbrae (c) its global umbra}
\label{fig:scene1}
\end{figure}

% \begin{figure}[hb]
%\begin{center}
% \includegraphics*[scale=.25]{img/--}
%\end{center}
% \caption{Sometimes the global umbra is the union of the local umbrae (umbra data/globalumb1.pts)}
% \label{fig:globalumb1}
% \end{figure}

% \begin{figure}
% \begin{center}
% \includegraphics*[scale=.25]{img/globalumb3.jpg}
% \end{center}
% % \centerline{\epsfig{figure=img/globalumb3.ps,height=2.906in,width=2.911in}}
% % 40% reduction
% \caption{Pushing an umbral bitangent through several obstacles (game of tag)}
% \label{fig:globalumb3}
% \end{figure}

	% umbra data/umbra/ovoid2.pts
% \begin{figure}
% \begin{center}
% \includegraphics*[scale=.25]{img/ovoid2-local.jpg}
% \end{center}
% \caption{A simple local umbra: too simple?}
% \label{fig:G1}
% \end{figure}

% \clearpage

\section{If the obstacle surrounds the light (CAN POSTPONE TO JOURNAL VERSION)}
\label{sec:surround}

If A surrounds B, the definition of 'extremal', 'outer' and 'inner' must
be changed.
When the obstacle surrounds the light, 
intersections are only allowed on the 'other' side of the light.
	% (from the point of bitangency with A).
	% Between the light and obstacle, the umbral bitangent will act 
	% the same as our earlier outer bitangent

\begin{defn2}
Let A and B be closed curves, where A surrounds B.
Let $t$ be a tangent of A,
and let $T$ be a bitangent of A and B with points of tangency $P_A \in A$ and 
$P_B \in B$.

\begin{itemize}
\item $T$ is {\bf extremal} if $T$ does not intersect B
	and the ray \ray{P_B P_A} does not intersect A.\footnote{$T$ 
	is allowed to intersect A before $P_B$, as in Figure~\ref{fig:G2}c.}
\begin{itemize}
\item $T$ is {\bf outer} if it is extremal and, in the neighbourhood of $P_A$,  
A and B lie on the same side of $T$.
\item $T$ is {\bf inner} if it is extremal and, in the neighbourhood of $P_A$,
A and B lie on opposite sides of $T$.
\end{itemize}
\end{itemize}
\end{defn2}

	% umbra data/vg10.pts
\begin{figure}[h]
\begin{center}
\includegraphics*[scale=.25]{img/vg10.jpg}
\end{center}
% \centerline{\epsfig{figure=img/vg10.ps,height=1.567in,width=1.553in}}
% 20% reduction
\caption{Local umbra of an obstacle surrounding the light}
\end{figure}

\begin{rmk}
We need a generalization of the inner bitangent, the A-inner bitangent,
to cover the case of an obstacle B that surrounds another obstacle A (Figure H1).
\end{rmk}

% ------------------------------------------------------------------------------------

\section{Penumbra}

This is the difference between the local umbra (the maximal umbra)
and the global umbra.

% ------------------------------------------------------------------------------------

\section{$2 \frac{1}{2}$-D visibility}

Our umbral solution in 2-space also gives an approximate solution
to lighting and visibility in a $2 \frac{1}{2}$-D environment.
Consider flythroughs of a city defined by digital elevation maps.

% ------------------------------------------------------------------------------------

\section{Conclusions}

Remind them that this theory is also applicable to general visibility
analysis of smooth objects.

We show how the analysis of an entire scene can be broken down into
the analysis of a single obstacle at a time.
We also show how two quite far-removed obstacles can affect each other,
even if their lines of sight to the light do not overlap.

If the light is interpreted as the sun and the umbra as the night,
the inner sweep may be interpreted as a search for the sunset,
and the outer sweep a search for the dawn of the light.


The inner umbral bitangents are tangent to A,
while the outer umbral bitangents are tangent to L.

Both sweeps define a global umbra bitangent G of A
through refinement of a local umbra bitangent T of A.
% (Recall that T is an outer bitangent of A and L.)

The sweeping of a tangent (and the associated widening of the umbral boundary)
can continue through several obstacles, like a game of tag.
% The pushing will stop when no obstacle intersects the early segment associated
% with the umbral boundary (Figure~\ref{fig:globalumb3}).

We don't want to try to compute the entire global umbra all at once:
by breaking it up into the global umbra generated by each obstacle,
the problem becomes tractable.

The obvious next step with this theory is the computation of the penumbra
and the extension to 3-space.

\begin{figure}
\begin{center}
\includegraphics*[scale=.25]{img/vg90umbra.jpg}
\includegraphics*[scale=.25]{img/vg900umbra.jpg}
\end{center}
\caption{The effect of adding one obstacle: (a) global umbra before (b) global umbra after}
\label{fig:scene2}
\end{figure}

% ------------------------------------------------------------------------------------

\bibliographystyle{latex8}
% \bibliographystyle{plain}
\begin{thebibliography}{99}

\bibitem{farin97}
Farin, G. (1997)
Curves and Surfaces for CAGD: A Practical Guide (4th edition).
Academic Press (New York).

\bibitem{jj01}
Johnstone, J. (2001)
A Parametric Solution to Common Tangents.
International Conference on Shape Modelling and Applications (SMI2001),
Genoa, Italy, IEEE Computer Society, 240--249.

\bibitem{jj02}
Johnstone, J. (2002)
The tangential curve.
In review.

\end{thebibliography}

% \begin{figure}[h]
% \begin{center}
% \includegraphics*[scale=.1]{img/globalumb2-PI.jpg}
% \end{center}
% \caption{Inner bitangents refine the local umbra into the global umbra}
% \label{fig:innerbitang}
% \end{figure}

\ifJournal
\section{Dagstuhl Talk}

Sweeping a tangent to (i) compute the umbra and (ii) compute the convex hull.

\section{Appendix}
\label{sec:appendix}

\subsection{Software}

\begin{itemize}
\item draw B/W umbral regions for a collection of obstacles
\item draw B/W local penumbral regions for a collection of obstacles
\item draw B/W global penumbral regions for a collection of obstacles 
\item draw greyscale local penumbral region for one obstacle
\item draw greyscale global penumbral regions for a collection of obstacles
\item find the extreme bitangents
\end{itemize}

\subsection{Defining the umbra point by point}
\label{sec:local}

It is simple to determine if a point P is in the umbral shadow cast by a 
single obstacle A (Figure I):
P is in umbra of A iff all of the light's tangents through P (it is enough to test
the extremal two) hit A.
However, several obstacles can conspire to place P in umbra, although it is not
in any one obstacle's umbra (Figure III).
Finding this type of umbral point requires more subtlety.
For example, it is not enough to demand simply that all of the light's tangents through P
hit some obstacle (Figure II).
In effect, we need to test that there are no cracks between obstacles through which
the light becomes visible, by 'sweeping' a beam centered at P across the light.
The following algorithm for testing if P lies in umbra achieves this virtual sweep.

\begin{defn2}
A tangent T of the curve C through a point P is {\bf extremal}
if the entire curve C lies in the same halfplane defined by T
(i.e., T does not intersect C).
The {\bf inside halfplane} of an extremal tangent T of C through a point P is
the halfplane defined by T that contains C.
An extremal tangent is {\bf left-extremal} if its inside halfplane 
contains the point at infinity (1,0,0) along the positive x-axis,
otherwise it is {\bf right-extremal}.
\end{defn2}

Is the point P in the umbra?
\begin{description}
\item[] T $\leftarrow$ left-extremal tangent of the light through P
\item[] while (T has not swept past light \&\& 
       T hits some obstacle A before it hits the light)
\begin{description}
\item[]	T = right-extremal tangent of A through P
\end{description}
\item P is in umbra iff T has swept past light
\end{description}

We are sweeping a line about P searching for a passageway to the light, 
leaping past obstacles that block the light by jumping to their right-extremal tangents, 
until we either find the light or sweep past it.
Figure IV (Figure III with sweep annotations) illustrates this algorithm.

What is the complexity of this algorithm?

% ------------------------------------------------------------------------------------

\centerline{{\bf What is new}}
\begin{enumerate}
\item implementation of algebraic geometry's dual curve that is
\begin{enumerate}
\item robust using 2 dual spaces, and
\item details worked out for Bezier curves, and
\item a primal structure in dual space, unlike dual Bezier curves
\end{enumerate}
\item dual Chaikin
\begin{enumerate}
\item 	don't even evaluate the parts of the curve unaffected by bitangents
		(really be lazy!)  (change the demo to show this)
\item 	analyze the improvement in accuracy as you move through subdivision stages
\item 	when do bad mistakes disappear and refinement only sets in?
\end{enumerate}
\item lazy subdivision idea for subdivision curves
\item application of bitangents to fundamental applications that are only known
	for polygons, not smooth curves
\begin{enumerate}
\item to smooth convex hull, extending Graham scan
\item to smooth visibility graph, improving quality of shortest paths
			[might be able to extend Welzl's algorithm]
\item to smooth lights in 2d, with penumbra and umbra calculation,
	     	anticipating a lifting to smooth lights in 3d
\begin{enumerate}
\item   software to shade the umbra/penumbra
\item 	to determine efficiently whether a curve is in umbra
\item 	to determine efficiently which parts of a curve are in penumbra
\item 	to determine their relative lighting (how much of the light they see)
\end{enumerate}
\end{enumerate}
\end{enumerate}
	     
%%%%%%%%%%%%%%%%%%%%%%%%%%%%%%%%%%%%%%%%%%%%%%%%%%%%%%%%%%%%%%%%%%%%%%%%%%%
%%%%%%%%%%%%%%%%%%%%%%%%%%%%%%%%%%%%%%%%%%%%%%%%%%%%%%%%%%%%%%%%%%%%%%%%%%%
			
\section{Introduction}

3 categories of lighting: umbra (fully shaded), penumbra (partially lit), 
	and noon (fully lit)

\section{Flawed approaches}

Don't want to cast light from each point of the light: 
	then every point of 2-space
	has an infinite number of light points casting light on it,
	for which it must sum the contributions
	Notice that every light point casts across a halfspace minus the
	areas blocked by obstacles or the neighbouring parts of the light (if
	light is concave)
	
On the other hand, computing the light intensity at every pixel
	in free space is also undesirable (?).
	
%%%%%%%%%%%%%%%%%%%%%%%%%%%%%%%%%%%%%%%%%%%%%%%%%%%%%%%%%%%%%%%%%%%%%%%%%%%
			
\section{General algorithm}

\subsection{Part I: Definition of global penumbra}

Instead, first decompose 2-space into global penumbra and global umbra using bitangents:
	this allows us to ignore all but the global penumbral regions;
	this is done for each obstacle: a point is in the global penumbra
	if it is in at least one local penumbral region and it is not in any
	local umbral region, which would block it;
	this is a form of point location in cells from a line arrangement
			
local umbra 
	(from the bitangents of the light and an obstacle A)
	defn: region of space that is entirely blocked from the light by A
	- elegant definition of umbral region: cell in a line/curve arrangement
	- hacky but simple definition of umbra: intersection of halfspaces defined
		by bitangents and obstacles
	- the hacky definition is simple to implement and simple for point 
		classification, although not optimal efficiency
local penumbra 
	(from the bitangents of the light and an obstacle)
	defn: region of space that is partially blocked from the light by A
local noon
	= 2-space - (local umbra and local penumbra)
	defn: region of space that sees the entire light w.r.t. A
		(i.e., region that is not blocked from the light at all by A)
global umbra 
	= union of local umbras
	defn: region of space that is entirely blocked from the light 
	      (by one or a combination of the obstacles)
*global penumbra*	(the nontrivial one)
	= (union of local penumbras) - global umbra
	defn: region of space that is partially blocked from the light 
	      (by one or a combination of the obstacles)
global noon
	= intersection of local noons
	= 2-space - (global umbra and global penumbra)
	
%%%%%%%%%%%%%%%%%%%%%%%%%%%%%%%%%%%%%%%%%%%%%%%%%%%%%%%%%%%%%%%%%%%%%%%%%%%
			
\subsection{Part II: Definition of light intensity inside global penumbra}

for each point P in a global penumbral region,
  for each obstacle A such that P is inside A's penumbral region,
	intersect tangent of A through P with light, yielding $L_A$
  choose L = $L_A$ that yields the smallest visible light segment
	(the larger light segments are blocked by one of the other obstacles)
  compute (read off) proportion of light visible from L on the
	outside of tangent
	   - this involves computing tangent PP' from light 
	     through P; then measuring visible (from P) portion of 
	     light from L to P', which is basically arc length
	     of convex hull from L to P'
  assign (paint) this intensity to every pixel (point) on 
	     the tangent from L to P and past P to the next 
	     intersection with an obstacle

%%%%%%%%%%%%%%%%%%%%%%%%%%%%%%%%%%%%%%%%%%%%%%%%%%%%%%%%%%%%%%%%%%%%%%%%%%%
			
\subsection{Part III: Definition of light intensity inside global umbra and global noon}

similarly paint pure black shadow or pure white light
     from pixels (points) that are found in global umbra
     or global noon, by continuing line --- (how?)

%%%%%%%%%%%%%%%%%%%%%%%%%%%%%%%%%%%%%%%%%%%%%%%%%%%%%%%%%%%%%%%%%%%%%%%%%%%

\section{Examples}

Test data.
\begin{itemize}
\item bitangency: circle, ob1, ob3, vg7.pts
\item convex hull: vgraph: vg4.pts, vg1.pts (contour), vg3.pts (contour), vg4.pts (small natural)
	vg5.pts (large natural)
\item lighting: ob1, vg5.pts
\end{itemize}

\fi	% of \ifJournal at beginning of appendix
			
\end{document}
