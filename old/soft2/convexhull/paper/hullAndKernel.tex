\documentclass[12pt]{article}
\usepackage[pdftex]{graphicx}
\usepackage{times}
% \usepackage{epsfig}
\input{header}

\newif\ifJournal
\Journalfalse
\newif\ifTalk
\Talkfalse

\newcommand{\plucker}{Pl\"{u}cker\ }
\newcommand{\tang}{tangential surface\ }
\newcommand{\tangs}{tangential surfaces\ }
\newcommand{\Tang}{Tangential surface\ }
\newcommand{\atang}{tangential $a$-surface\ }
\newcommand{\btang}{tangential $b$-surface\ }
\newcommand{\ctang}{tangential $c$-surface\ }
\newcommand{\atangs}{tangential $a$-surfaces\ }
\newcommand{\btangs}{tangential $b$-surfaces\ }
\newcommand{\ctangs}{tangential $c$-surfaces\ }

\setlength{\oddsidemargin}{0pt}
\setlength{\topmargin}{-1in}	% should be 0pt for 1in
% \setlength{\headsep}{.5in}
% \setlength{\textheight}{8.875in}
\setlength{\textheight}{8.6in}
\setlength{\textwidth}{6.875in}
\setlength{\columnsep}{5mm}	% width of gutter between columns

% \DoubleSpace

\title{Smooth hulls and kernels in 2-space}
% Hulls, kernels, and giftwrapping of smooth curves
% A smooth Jarvis march
% The convex hull of a smooth curve
\author{J.K. Johnstone\thanks{Supported in part by the National Science Foundation
	under grant CCR-0203586.}\\
	Computer and Information Sciences\\
	University of Alabama at Birmingham}
%	Geometric Modeling Lab, Computer and Information Sciences, University
%	Station, Birmingham, AL, USA 35294; Phone: (205) 975-5633; 
%	Fax: (205) 934-5473; jj@cis.uab.edu

\begin{document}
\maketitle

\begin{abstract}
\noindent
\begin{itemize}
\item	convex hull of smooth curve
\item	convex hull of curve with cusps
\item	bitangents (for convex hull)
\item	tangents through a point (for cusp version)
\item 	kernel of curve (origin in kernel)
\item	kernel of curve (origin not in kernel)
\end{itemize}

We present two algorithms for the smooth kernel,
one that involves the convex hull and one that does not.

The kernel, which consists of points not struck by tangents,
is naturally computed with the help of tangential curves.

The challenging part of the computation is finding
a single seed point of the kernel, to bootstrap the computation
of the entire kernel using the convex hull.

We will try to find a free bitangent of the tangential c-curve.
This will generate our seed point of the kernel.

For nontrivial concave curves, 
a point of the kernel can be found from a free bitangent
of the tangential c-curve.

We develop an efficient algorithm to compute the convex hull of a smooth curve.
We effectively sweep a line, not radially or vertically but tangentially,
% as defined by the tangent space of the curve, 
and identify events in this sweep through the computation of bitangents.

We also compute the kernel of a smooth curve,
which can be viewed as the dual of the convex hull problem,
if the origin of our primal space is a kernel point of the curve.

Once the bitangents of a collection of curves are known, 
it becomes feasible to build many secondary structures at a reasonable cost,
such as the convex hull, the visibility graph, and the kernel.
These algorithms have been classically developed for polygons.
We consider the construction of all of these structures for smooth curves,
using smooth analogues of Jarvis march and line sweep (and Welzl's algorithm?).
The calculation of these structures in a smooth environment,
as opposed to a polygonal environment built from a discrete sampling of the
smooth environment, can significantly simplify the structures
and improve their accuracy.
\end{abstract}

Keywords: convex hull, kernel, dual space, tangential curve, bitangent,
	  giftwrapping, Jarvis march.

\tableofcontents

\clearpage

%%%%%%%%%%%%%%%%%%%%%%%%%%%%%%%%%%%%%%%%%%%%%%%%%%%%%%%%%%%%%%%%%%%%%%%%%%%%%

\section{Introduction}

We develop an efficient algorithm for computing the convex hull and the
kernel of a closed curve (Figure~\ref{fig:curveob1b}).
The convex hull and kernel of a polygon have been intensely studied
for decades, but the convex hull and kernel of a curve
are much less well understood \cite{kimBajajOnHull,elber01,elber02}.
The approach used in this paper is novel in its use of dual space
and tangential curves \cite{jj01,jj02} to build the hull and kernel.
Our construction of the kernel also highlights its dual relationship
with the convex hull.

Our hull algorithm works by sweeping around the curve,
in a smooth version of the classical 
line sweep, using the lines of tangent space rather than vertical lines.
The algorithm can also be interpreted as a smooth version of
giftwrapping,
using bitangents to define events where convexity tests must be performed.

Our kernel algorithm is motivated by the observation that the kernel
is dual to the convex hull.
That is, the convex hull in dual space can be used to compute the kernel
in primal space.
	
Algorithms for the convex hull and kernel belong together, since they are
dual problems.
We shall see that the computation of the kernel can be reduced
to the computation of a convex hull in dual space,
after some massaging.

Since both problems involve tangent spaces, they are naturally attacked
using tangential curves, dual images of tangent spaces.
Convex hulls are built from bitangents and kernels are the locus
avoided by the tangent space.

\begin{figure}[h]
\begin{center}
\includegraphics*[scale=.5]{img/jjhu1.jpg}
\includegraphics*[scale=.5]{img/jjhu2.jpg}
\includegraphics*[scale=.5]{img/jjke0.jpg}
\end{center}
\caption{A smooth curve, its convex hull, and its kernel}
\label{fig:curveob1b}
% convexhull -p -d 10 data/ob1b.pts
% kernel -p -d 10 data/ob1b.pts
\end{figure}

\clearpage

%%%%%%%%%%%%%%%%%%%%%%%%%%%%%%%%%%%%%%%%%%%%%%%%%%%%%%%%%%%%%%%%%%%%%%%%%%%%%

\section{Related work}

There are many famous algorithms for the convex hull of a finite set of points,
and by extension a polygon \cite{preparata85}.
We mention those most related to the smooth methods in this paper.
The Jarvis march \cite{jarvis73} uses giftwrapping to compute the hull.
Graham's scan \cite{graham72}, when applied to a polygon,
may also be interpreted as a punctuated sweep around the boundary of the
polygon.

Consider the computation of the convex hull of $N$ points in the plane
by Graham's scan.
In Graham's scan, we sort the $N$ points radially about an internal point
and then traverse this list, discarding points $p_i$ such that
$p_{i-1} p_i p_{i+1}$ makes a right turn (reflex angle).
If the $N$ points are the ordered vertices of a polygon,
notice that Graham's scan may be interpreted as a sweep around 
the boundary of the polygon, testing for reflex angles.

The line sweep \cite{edelsbrunner,mehlhorn}
is another algorithmic technique that is related to 
our smooth hull algorithm.
Sweeping a vertical line (such as in the computation
of the intersections of a line arrangement).
Events.
In a classical vertical sweep, say of a line arrangement, events are
defined by intersections and vertex endpoints and the order of the lines 
is adjusted at these events.

The kernel of a polygon has been studied in \cite{lee79}.

The convex hull and kernel of a curve have been studied by Elber et. al.
\cite{elber01,elber02},
and this paper is directly motivated by this work.
The convex hull is computed by ---.
{\bf Gershon, can you remind me how you compute the kernel?}

A comparison to Elber's hull and kernel solutions.

\begin{defn2}
bitangent
\end{defn2}

The bitangents of a curve will be important tools in this paper.
In \cite{jj01}, we showed how to compute the bitangents of a smooth curve
efficiently using a dual representation of the tangent space of the curve, 
called the tangential curve.


\clearpage

%%%%%%%%%%%%%%%%%%%%%%%%%%%%%%%%%%%%%%%%%%%%%%%%%%%%%%%%%%%%%%%%%%%%%%%%%%%%%

\section{Smooth hull}
\label{sec:hull}

Consider the convex hull of a closed plane curve $C$.
Free tangents define this hull.

\begin{defn2}
A line is {\bf free} if it does not intersect $C$.\footnote{We do not count
	a point of tangency as an intersection.}
\end{defn2}

\begin{lemma}
\label{lem:condition}
Let $C$ be a closed plane curve.
The point $P \in C$ lies on the convex hull boundary of $C$ if and only if
the tangent at $P$ is free.
\end{lemma}
\prf
The tangent of a point inside a concavity will clearly
intersect the curve.
But the tangent of a point on the convex hull will define
a halfplane that contains the curve 
(since the convex hull of a point set is the intersection of the halfplanes 
that contain the point set),
so it will be free.
%	More formally but no better,
%	A tangent defines a containing halfplane if and only if it is free.
%	The halfplane defined by the tangent at $P$ is clearly a limit halfplane
%	in this direction and therefore a boundary point of the hull if it is free.
%	Conversely, if $P$ lies on the convex hull boundary, 
%	then some halfplane through $P$ is containing.
%	But the only possible containing halfplane through $P$
%	is its tangent halfplane.
\QED

	% convexhull -p -d 10 data/ob1b.pts
\begin{figure}[h]
\begin{center}
\includegraphics*[scale=.4]{img/jjhu2.jpg}
\includegraphics*[scale=.4]{img/jjhu3a.jpg}
\includegraphics*[scale=.4]{img/jjhu3b.jpg}
\includegraphics*[scale=.4]{img/jjhu3c.jpg}
\end{center}
\caption{A point of the convex hull has a free tangent}
\label{fig:tangent}
\end{figure}

Since there are an infinite number of tangents to test for freedom,
we need to apply coherence.
Consider walking along the curve.
If the tangent is free at some point,
it will remain free for some time.
In order to lose freedom, the tangent must start to intersect
the curve, at which point it will be a bitangent.
By the same argument, a tangent will only regain freedom
as it passes through a bitangent.
We conclude that we do not need to test every tangent for freedom,
only the bitangents.
We can look at this issue in another way.
The convex hull must cap the concavities of $C$.
It is easy to see that the appropriate caps are bitangents
of $C$ (Figure~\ref{fig:curveob1b}b), so the only question is which bitangents?
Lemma~\ref{lem:condition} shows that the bitangents on the hull
are the free bitangents.

	% convexhull -p -d 10 data/ob3a.pts
\begin{figure}
\begin{center}
\includegraphics*[scale=.5]{img/jjhu4a.jpg}
\includegraphics*[scale=.5]{img/jjhu4b.jpg}
\includegraphics*[scale=.5]{img/jjhu4c.jpg}
\end{center}
\caption{A curve, its bitangents, and its convex hull}
\label{fig:convhullob3a}
\end{figure}

\begin{theorem}
\label{thm:chull}
Let $C$ be a closed curve.
The convex hull of $C$ is bounded by its free bitangents
and the curve segments of $C$ between these bitangents.
\end{theorem}

% algorithm is unnecessary (idea is already clear, and clearer w.o. elaboration)
% A better algorithm (a form of smooth giftwrapping):
% \begin{enumerate}
% \item compute bitangents
% \item filter down to bitangents that don't intersect the curve (active bitangents)
% \item sort the endpoints of these active bitangents
% \item the sorted endpoints of the active bitangents define the segments
% 	of the convex hull
% \end{enumerate}

Using Theorem~\ref{thm:chull}, we could compute the $b$ bitangents of $C$,
then perform $b$ line-curve intersections to filter these bitangents 
down to the free bitangents. 
However, we can do better.
We can compute the free bitangents with fewer intersection tests
using a walk around the hull.

Imagine walking around the convex hull, starting at a point of $C(t)$
that lies on the convex hull.
To stay on the hull, it is enough to ensure that we never enter a concavity.
Fortunately, a free bitangent guards every concavity,
so encountering an endpoint of a bitangent during our hull walk
indicates that we are entering a concavity.
Whenever we meet a bitangent endpoint, we shall mark the associated bitangent
as a free bitangent, leap over to the other side of the bitangent
(thus leaping over the concavity), and continue the hull walk from there.
In this way, the free bitangents can be gathered by walking around the curve,
staying on the convex hull:
they are the only bitangents that are met during this walk.
% leaping over any bitangents that are met.
This walk also gathers the hull curve segments.
Notice that the walk is virtual:
'walking' across a curve segment is equivalent to moving to the next
element in a sorted list of parameter values.  (See Algorithm 2 below.)

\vspace{.2in}

\centerline{{\bf Algorithm 1: Computing the convex hull of $C$}}

\begin{enumerate}
\item Compute the bitangents of $C$.
% \item Sort the parameter values of the endpoints of these bitangents.
\item Find a seed point $P \in C$ of the convex hull.
%	(This can be found using another walk.)
%	interior to a hull curve segment.
\item Walk around the curve starting at $P$, gathering free bitangents.
	This walk defines the entire convex hull.
\end{enumerate}

Steps 2 and 3 of Algorithm 1 require some elaboration.
Consider the walk of step 3.
Let $\{\seg{E_i E_{\mbox{\tiny{buddy}}(i)}}\}$ be the bitangents of $C$,
where $\{E_1 = C(e_1), E_2 = C(e_2), \ldots, E_{2n} = C(e_{2n})\}$
are the endpoints of the bitangents
and $e_1 < e_2 < \ldots < e_{2n}$.
Let $\oplus$ and $\ominus$ represent 
addition and subtraction in modulo arithmetic.
Suppose that we have already found the starting point $P = C(t_{\mbox{\tiny{start}}})$
for this walk.

\vspace{.2in}

\centerline{{\bf Algorithm 2: A walk around the convex hull}}

\begin{enumerate}
\item 	If $C$ has no bitangents, $\mbox{convhull}(C) = C$.
\item	Initialize $i$ such that $t_{\mbox{\tiny{start}}} \in (e_{i \ominus 1}, e_i)$
	and $\mbox{convhull}(C) = \emptyset$.
\clearpage
\item	do
\begin{enumerate}
\item	% [Walk to the next bitangent] 
	add the bitangent with endpoint $E_i$ to the convex hull
% 	(which will be a free bitangent)
% \item   $\mbox{convhull}(C) = \mbox{convhull}(C) \cup$ 
%	bitangent with endpoint $E_i$ 
\item	% [Jump across this bitangent and continue to the next bitangent] 
	add the curve segment $S$ starting at $E_{\mbox{\tiny{buddy}}(i)}$
	and ending at $E_{\mbox{\tiny{buddy}}(i) \oplus 1}$
	to the convex hull
% \item	$\mbox{convhull}(C) = \mbox{convhull}(C) \cup$ 
%	curve segment $S$ starting at $E_{\mbox{buddy}(i)}$.
\item	$i \leftarrow \mbox{buddy}(i) \oplus 1$
\end{enumerate}
 	while $S$ does not contain the starting point $P$
\end{enumerate}

Step (3a) is equivalent to walking across a free bitangent of the hull.
Step (3b) is equivalent to walking across the next curve segment of the hull.
This walk will require no line-curve intersections
and will take $O(f)$ time, where $f$ is the number of free bitangents.

To bootstrap the walk,
a seed point $P = C(t_{\mbox{\tiny{start}}})$
of the convex hull must be found.
This seed point can itself be found using a walk.
Starting at the beginning of the curve $C(a)$
(where $C(t)$ is defined over $t \in [a,b]$)
and until we reach the convex hull, we walk out of the concavity
by moving to the next bitangent endpoint.
This will soon reach a free bitangent.
We finish by moving the start point to the middle of a hull curve segment
in order to make the ensuing walk robust.
% the seed point should be in the interior of a hull curve segment
% (e.g., not a bitangent endpoint).
% For the same reason, if $t_{\mbox{\tiny{start}}} = a$ is a free bitangent endpoint, 
% we should move 
% $t_{\mbox{\tiny{start}}}$ forward or backward to the midpoint of the
% next segment, whichever is free.
\Comment{
\item if the tangent at $C(a)$ does not intersect $C$
\begin{itemize}
\item if $a=e_i$ (some bitangent endpoint)
\begin{itemize}
\item if the tangent at $C(\frac{e_i + e_{i \oplus 1}}{2})$ does not intersect the curve
\begin{itemize}
\item choose $\frac{e_i + e_{i \oplus 1}}{2}$ as the starting point.
\end{itemize}
\item else choose $\frac{e_{i \ominus 1} + e_i}{2}$ as the starting point.
\end{itemize}
\item else choose $a$ as the starting point.
\end{itemize}
\item else
\begin{itemize}
}

\vspace{.2in}

\centerline{{\bf Algorithm 3: A walk to find a starting point on the convex hull}}

\begin{enumerate}
\item Initialize $t_{\mbox{\tiny{start}}}$ to $a \in [e_{i \ominus 1}, e_i)$.
\item while the tangent at $C(t_{\mbox{\tiny{start}}})$ intersects the curve
\begin{enumerate}
\item	$t_{\mbox{\tiny{start}}} \leftarrow e_{i}$
\item	$i \leftarrow i \oplus 1$
\end{enumerate}
\item $t_{\mbox{\tiny{start}}} \leftarrow \frac{e_i + e_{i \oplus 1}}{2}$.
	(If $t_{\mbox{\tiny{start}}} = a$, we may have to move to 
	$\frac{e_{i \ominus 1} + e_i}{2}$ instead, whichever is free.)
\end{enumerate}

This walk to a starting point will require at least one line-curve intersection,
but on average far fewer than the $b$ line-curve intersections
of Theorem~\ref{thm:chull} (where $b$ is the number of bitangents).
% This is why we prefer finding the free bitangents through a walk.

\begin{example}
Let's trace the algorithm on the simple example of Figure~\ref{fig:algtrace}.
A starting point is found at $P$, by walking out of the concavity
at the beginning of the curve $b$.
During our walk of the convex hull, 
we first add the bitangent between endpoints 1 and 2,
then the curve segment between 2 and 3, the bitangent between 3 and 4,
and so on until we add the curve segment between 0 and 1,
which contains $P$ and signals the completion of the hull.
\end{example}

	% convexhull -p -d 10 data/complexKernel.pts
\begin{figure}[h]
\begin{center}
\includegraphics*[scale=.5]{img/jjhu5a.jpg}
\includegraphics*[scale=.5]{img/jjhu5c.jpg}
\end{center}
\caption{Computing the convex hull with a walk}
\label{fig:algtrace}
\end{figure}

Figure~\ref{fig:convhullob3a} is a good illustration of the power of the 
algorithm.
This curve has 23 bitangents, yet our walk can find the free bitangents
with only one line-curve intersection (to find the starting point)
rather than 23.

%%%%%%%%%%%%%%%%%%%%%%%%%%%%%%%%%%%%%%%%%%%%%%%%%%%%%%%%%%%%%%%%%%%%%%%%%%%%%

\subsection{A smooth version of Graham's scan}

Our algorithm for the convex hull of a curve
is similar to the Graham scan algorithm for the convex hull of a polygon.
The sweep around the polygon, testing for reflex angles, of Graham's scan is 
replaced by a sweep around the curve, testing for free tangents.
Like Graham's scan, 
we identify a finite number of events at which to test.
In Graham's scan, these are the polygon's vertices;
in our smooth Graham scan, these are the bitangents.
We have made one improvement over the polygonal Graham scan.
We observed that if we traverse the curve properly,
we don't actually have to perform the test at the bitangent events, 
since the answer will always be in the affirmative.

% Ours is a giftwrapping algorithm.

% The reflex-angle criterion of Graham's scan 
% is less useful when we work with a smooth curve
% rather than a polygon, since there are an infinite number of points
% and no "angles".

% At an event, the traversal inserts the bitangent into the convex hull
% and resumes the traversal at the other end of the bitangent.
% Notice that events associated with inactive bitangents are never encountered.

%%%%%%%%%%%%%%%%%%%%%%%%%%%%%%%%%%%%%%%%%%%%%%%%%%%%%%%%%%%%%%%%%%%%%%%%%%%%%

\ifTalk
We can interpret our algorithm as a tangent sweep (a sweep through tangent space),
a smooth version of line sweep.

In our sweep, we sweep a line controlled by the curve's tangent space,
where events are defined by bitangents (intersections in dual space).
In fact, we can avoid a physical sweep (just appealing to its structure)
and get its effects from a sorting of bitangent parameter values.
\fi

\clearpage

%%%%%%%%%%%%%%%%%%%%%%%%%%%%%%%%%%%%%%%%%%%%%%%%%%%%%%%%%%%%%%%%%%%%%%%%%%%%%

\section{Duality and tangential curves}
\label{sec:duality}

For convex hull (bitangents and tangents through a point):

\begin{itemize}
\item	a-duality and b-duality
\item	tangential a-curves and tangential c-curves
\item	clipped tangential curves and tangential curve system
\item   primal space
\end{itemize}

For kernel:

The kernel requires the use of the tangential c-curve.
Moreover, its use is made robust by the special nature of the kernel problem.
since we shall compute a convex hull in dual space, which is inherently
closed and therefore inherently straddles both a-dual and b-dual spaces,
which complicates the hull computation.
The tangential c-curve is robust in this application because
we guarantee that the tangent space of the curve does not intersect the origin.

\begin{itemize}
\item	$c$-duality
\item	tangential c-curve
\item	tangents through the origin map to infinity
\end{itemize}

\begin{lemma}
\label{lem:cinfty}
Tangents through the origin map to infinity under c-duality.
\end{lemma}

\begin{lemma}
\label{lem:join}
The join of two lines $a$ and $b$ (the line between $a$ and $b$) 
dualizes to the intersection of the lines $a^*$ and $b^*$.
\end{lemma}

\clearpage

%%%%%%%%%%%%%%%%%%%%%%%%%%%%%%%%%%%%%%%%%%%%%%%%%%%%%%%%%%%%%%%%%%%%%%%%%%%%%

\section{Computing bitangents and tangents through a point}
\label{sec:bitang}

\subsection{Bitangents}

\begin{defn2}
\label{defn:bitang}
A {\bf conventional bitangent} of a curve $C$ is a line
that is tangent to $C$ at two or more distinct points.
If a curve contains cusps, tangents of the curve through the cusps
and lines between the cusps must also be considered bitangents
(Figure~\ref{fig:cusps}).
These will be called {\bf cusp bitangents}.
\end{defn2}

\Comment{
\begin{figure}[h]
\caption{The bitangents of a curve with cusps}
\label{fig:cusps}
\end{figure}

\begin{figure}[h]
\caption{The convex hull of a curve with cusps}
\label{fig:convhullwithcusps}
\end{figure}
}

We use a dual space to compute bitangents.
Bitangents of $C$ are self-intersections of the tangential curve system of $C$
(i.e., self-intersections of the clipped tangential a-curve $C_a^*$
and self-intersections of the clipped tangential b-curve $C_b^*$).

Cusps are the solutions of $C_x = C_y = 0$?

\subsection{Computing tangents through a point}

This is needed for computing convex hulls of curves with cusps.

\noindent Special case: horizontal and vertical tangents

\noindent We serendipitously discovered (from a bug in computing
	the c-dual of a line) that vertical tangents are intersections
	of the $x$-axis ($y=0$) with the tangential a-curve.
Probably horizontal tangents are intersections of the $y$-axis ($x$-axis)
with the tangential b-curve.

\clearpage

%%%%%%%%%%%%%%%%%%%%%%%%%%%%%%%%%%%%%%%%%%%%%%%%%%%%%%%%%%%%%%%%%%%%%%%%%%%%%

\section{Smooth kernel}
\label{sec:smoothkernel}

\begin{defn2}
Let $C$ be a plane curve.\footnote{In this section, $C$ will always refer 
	to a plane curve and $C^*$ its tangential c-curve.}
The point $P$ can {\bf see the point} $Q$ on $C$
if the line segment $\seg{PQ}$ does not intersect $C$.
The {\bf kernel} of $C$ is the locus of points that can see every
point of $C$:
\[
\mbox{kernel}(C) = \{ P \in R^2: \seg{PQ} \ \cap \ C = \emptyset 
		\hspace{.25in} \forall \ Q \in C \}
\]
\end{defn2}

\begin{lemma}
The kernel is convex and connected.
\end{lemma}

\begin{figure}[h]
\begin{center}
\includegraphics*[scale=.5]{img/jjke1.jpg}
\includegraphics*[scale=.5]{img/jjke2.jpg}
\end{center}
\caption{(a) The kernel of a curve (b) The tangent space of a curve, revealing the kernel}
\label{fig:kernel}
% kernel -p -d 10 data/ob1b.pts, with left window grown
\end{figure}

\noindent \cite{elber02} characterized the kernel in a fresh way that is much more
useful algorithmically.

\begin{lemma}[Elber]
\label{lem:kernel}
$P \in \mbox{kernel}(C)$ if and only if no tangent of $C$ intersects $P$.
\end{lemma}

\noindent In other words, the tangent space 
of a curve carves out the kernel (Figure~\ref{fig:kernel}b).
We can reinterpret this statement in dual space.

\begin{corollary}
$P \in \mbox{kernel}(C)$ if and only if dual($P$) does not intersect the 
tangential c-curve $C^*$.
\end{corollary}

\begin{defn2}
A line in dual space is {\bf free} if it does not intersect the 
tangential c-curve $C^*$.
\end{defn2}

\begin{corollary}
\label{cor:free}
$P \in \mbox{kernel}(C)$ if and only if dual($P$) is free.
\end{corollary}

\noindent Consequently, points of the kernel are easily recognizable in dual space:
they are free lines (Figure~\ref{fig:freeline}).
This observation will be central to our computation of the kernel.

% picture of kernel, points in and out of kernel, and associated dual lines
% including some tangents
\begin{figure}
\begin{center}
\includegraphics*[scale=.5]{img/jjke3a.jpg}
\includegraphics*[scale=.5]{img/jjke3b.jpg}
\includegraphics*[scale=.5]{img/jjke3c.jpg}
\end{center}
\caption{Kernel points in primal space are associated with free lines in dual space}
\label{fig:freeline}
% kernel -p -d 10 -x .5 data/ob1b.pts (origin now in kernel)
\end{figure}

\subsection{Computing the kernel from the convex hull}

To get the kernel of $C$, we shrink $C$ until every point of the new $C$
can see every point of the original $C$.
This is similar to the convex hull of $C$, where 
we instead expand $C$ until every point of
the new $C$ can see every point of the {\em new} $C$.
Both are closures under the 'see' operation ($\seg{PQ}\ \cap \ C = \emptyset$):
one is an expansion and the other a contraction.
It will perhaps not be a surprise, therefore, that the convex hull
will figure in the computation of the kernel.
The following theorem makes the tightness of this relationship even clearer.

\begin{theorem}[Wang]
\label{thm:hullisdual}
Suppose that the kernel of $C$ contains the origin $(0,0)$.
Then the kernel of $C$ is dual to the convex hull of $C^*$.\footnote{Wenping Wang made this observation 
	at the 2002 Dagstuhl
	Seminar on Geometric Modeling, at which I gave 
	a talk on tangential curves and the convex hull
	and Gershon Elber gave a talk on the kernel of a curve.
	I am indebted to Wenping and Gershon for
	motivating me to explore this relationship.}
\end{theorem}
\prf
More precisely, we will show that the kernel boundary of $C$ is dual 
to the tangent space of the convex hull boundary of $C^*$.
This shows that the kernel is trivially computable from the convex hull
(see Algorithm 4).
Since the kernel contains the origin, no tangent will intersect the origin
(Lemma~\ref{lem:kernel}) and no tangent will dualize to infinity (Lemma~\ref{lem:cinfty}).
Thus, $C^*$ will be finite, bounded, and closed,
and its convex hull is well defined.
Since kernel points correspond to free lines (Corollary~\ref{cor:free}),
boundary points of the kernel correspond to 
free lines that are just about to touch $C^*$: free tangents of $C^*$.
But the free tangents of a curve bound its convex hull (Lemma~\ref{lem:condition}).
\QED

% does not reveal anything new
% more importantly, this figure misleads, since we are really interested in the
% tangent space of the hull
% the previous figure is a better indication of the result (pt goes to line)
% figure of translated ob1b, its kernel, and the convex hull of its tangential c-curve
% \begin{figure}[h]
% \begin{center}
% \includegraphics*[scale=.5]{img/--.jpg}
% \end{center}
% \caption{The kernel and the convex hull are dual, if the origin lies in the kernel}
% \label{fig:kernelob1a}
% \end{figure}

Unfortunately, this dual relationship between the kernel and convex hull 
only holds when the kernel contains the origin,
because we implicitly rely on the boundedness of $C^*$.
If the kernel of $C$ does not contain the origin, a reduction to the convex hull
of $C^*$ will not work since $C^*$ will be unbounded
and disconnected, and the convex hull of $C^*$
will be uninteresting (Figure~\ref{fig:originNotInKernel}).

% figure of ob1b, its kernel, and its unclipped tangential c-curve
\begin{figure}
\begin{center}
\includegraphics*[scale=.5]{img/jjke4.jpg}
\end{center}
\caption{A curve whose kernel does not contain the origin, and its infinite tangential c-curve}
\label{fig:originNotInKernel}
% kernel -p -d 10 data/ob1b.pts
\end{figure}

%%%%%%%%%%%%%%%%%%%%%%%%%%%%%%%%%%%%%%%%%%%%%%%%%%%%%%%%%%%%%%%%%%%%%%%%%%%%%

Theorem~\ref{thm:hullisdual} leads to the following algorithm for computing
the kernel.
	
\vspace{.2in}

\centerline{{\bf Algorithm 4: Computing the kernel of $C$}}

\begin{enumerate}
\item 	Find a point P of the kernel.  If no such point exists, 
	the kernel is empty.
\item   Translate P to the origin.
\item	Compute the tangential c-curve $C^*$.
\item	Compute the convex hull of $C^*$.
	This defines the kernel of $C$:
	the curved segments $C^*(t_1,t_2)$ of the hull boundary define
	the curved segments $C(t_1,t_2)$ of the kernel boundary,
	and the bitangents $\seg{C^*(t_1)C^*(t_2)}$ of the hull boundary 
	define the straight line segments $\seg{C(t_1)C(t_2)}$ of the
	kernel boundary.
\end{enumerate}

%%%%%%%%%%%%%%%%%%%%%%%%%%%%%%%%%%%%%%%%%%%%%%%%%%%%%%%%%%%%%%%%%%%%%%%%%%%%%

% We have solved the kernel problem if the origin lies in the kernel.
% But this will rarely be the case.
% Indeed, for all curves with empty kernels, this will not be the case.

This establishes that the problem of computing the kernel reduces to the apparently
easier problem of computing a single point of the kernel
(or discovering that there is none to find).
However, this subproblem is slippery.
	% Notice that if the kernel is empty, there isn't even one to find!
In fact, we shall discover that finding one point of the kernel
is just as hard as finding the entire kernel (Section~\ref{sec:nohull}).

%%%%%%%%%%%%%%%%%%%%%%%%%%%%%%%%%%%%%%%%%%%%%%%%%%%%%%%%%%%%%%%%%%%%%%%%%%%%%

\subsection{Computing a single point of the kernel}
\label{sec:candidate}

% We must find one point of the kernel.
We shall attack this problem by finding 
a finite collection of points that must necessarily contain
a kernel point, if one exists.
It will then be a simple process of testing each of these candidates against
the kernel, using Corollary~\ref{cor:free}.
If none are in the kernel, we shall conclude that the kernel is empty.
% Otherwise, we shall have one or more seed points for the kernel.

\begin{defn2}
A set of points $S \subset \Re^2$ is a 
{\bf necessary and sufficient set of kernel candidates} 
(or {\bf kernel set} for short)
if $S \cap \mbox{kernel}(C) \neq \emptyset$ 
whenever  $\mbox{kernel}(C) \neq \emptyset$.
\end{defn2}

\noindent 
We shall compute the kernel set in primal space as a set of points,
but often interpret it in dual space as a set of lines.

Convex curves are a special case in this development.
Since the kernel of a convex curve $C$ is simply the interior of $C$,
we can assume without loss of generality that $C$ is concave.
Moreover, convex curves are easily recognized as curves without inflection
points.
We can also assume that the kernel of $C$ is
not empty, otherwise the choice of kernel set is arbitrary.
Curves with self-intersections are another trivial case,
since these curves always have empty kernels.
(As the tangent sweeps from the self-intersection $I$ back to $I$ again,
it will sweep out the entire plane, leaving no room for the kernel.)
Therefore, we can assume that $C$ is a concave and
simple curve with nonempty kernel (what we shall call a {\bf nontrivial} curve).

% 	\begin{defn2}
% 	An {\bf inflection point} is a point at which the sign of the curvature changes.
% 	% Not just a zero curvature.
% 	\end{defn2}

Inflection points are fundamental to the kernel and will be central
to our construction of kernel candidates.
Thinking of the kernel as the region carved out by the tangent space (Figure~\ref{fig:kernel}b),
it becomes clear why inflection points will play an important role,
as they represent local extrema of the tangent's sweep in a particular
direction.\footnote{The inflection tangents 
	that you see in 
	Figure~\ref{fig:kernel}b are not explicitly drawn.
	They are artifacts of the tangent space, indicating where
	tangents naturally bunch together as they change direction
	at an inflection point.}
Nontrivial curves will have inflection points.
Indeed, every concavity must contain an inflection point.
% Consider walking along a (concave, simple) curve and entering a concavity.
% The curve must eventually change curvature in order to leave the concavity.
% At this change, there is an inflection point.

\begin{figure}[h]
\begin{center}
\includegraphics*[scale=.5]{img/jjke5.jpg}
\end{center}
\caption{Inflection points of $C$ and their tangents, and the associated cusps of the clipped $C^*$}
\label{fig:infl}
% kernel -p -d 10 data/ob1b.pts
\end{figure}

%	\begin{lemma}
%	If $C$ is convex, $\mbox{kernel}(C) = \mbox{interior}(C)$.
%	\end{lemma}
%	
% 	\begin{lemma}
%	\label{lem:hasinfl}
%	Suppose $\mbox{kernel}(C) \neq \emptyset$.
%	If $C$ is not convex, then $C$ has inflection points.
%	\end{lemma}
%	\prf
%	Consider walking along a curve and entering a concavity.
%	The curve must eventually change curvature in order to leave the concavity.
%	At this change, there is an inflection point.
%	\QED

Here is the punch line of our development.

\begin{theorem}
\label{thm:kernelset}
Let $T$ be the tangents of the inflection points of $C$.
Let $H_1$ be the intersections of $T$.
Let $H_2$ be the intersections of $T$ with $C$.
A valid kernel set for a concave $C$ is $H_1 \cup H_2$.
\end{theorem}

Figures~\ref{fig:comp1}-\ref{fig:comp2} 
illustrate $H_1$ and $H_2$, as points in primal space
and lines in dual space.
% \footnote{In this and 
%	subsequent figures, $C^*$ is trimmed as described in Section~\ref{sec:trim}.}
Now let us establish why $H_1 \cup H_2$ is a valid kernel set.
We need to shift our attention to dual space.
The tangent of an inflection point in primal space becomes a cusp in dual space
(Figure~\ref{fig:infl}).
After all, at an inflection point the tangent reverses direction,
so in dual space the point will reverse direction, yielding a cusp.

\begin{figure}[h]
\begin{center}
\includegraphics*[scale=.5]{img/jjke6.jpg}
\end{center}
\caption{The first component $H_1$ of the kernel set}
\label{fig:comp1}
% fig 11
\end{figure}

\begin{figure}
\begin{center}
\includegraphics*[scale=.5]{img/jjke7.jpg}
\end{center}
\caption{The second component $H_2$ of the kernel set}
\label{fig:comp2}
\end{figure}

\begin{figure}
\begin{center}
\includegraphics*[scale=.5]{img/jjke8.jpg}
\end{center}
\caption{The kernel points in the kernel set of Figures~\ref{fig:comp1}-\ref{fig:comp2}}
\label{fig:free}
\end{figure}

\begin{lemma}
The tangent of an inflection point of $C$ dualizes to a cusp of $C^*$.
\end{lemma}

% \begin{defn2}
% A {\bf cusp} is a point at which the incoming and outgoing tangents are
% in opposite directions.
% At a cusp, the first derivative is zero.
% \end{defn2}

%	\begin{lemma}
%	\label{lem:inflcusp}
%	An inflection point of $C$ corresponds to a cusp of $C^*$.
%	\end{lemma}
%	\prf
%	As you pass through an inflection point, the tangent reverses direction
%	\QED

In primal space, we are looking for a kernel point.
In dual space, we are looking for a free line (Corollary~\ref{cor:free}).
The next lemma shows that we can restrict our search to bitangents.

\begin{lemma}
\label{lem:existence}
If $C$ is a nontrivial curve,
one of the bitangents of $C^*$ is free.
\end{lemma}
\prf
If $C$ is nontrivial, its kernel is not empty and 
there exists a free line in dual space.
We can move this free line to a bitangent while preserving its freedom.
In particular, we can move the free line until it touches the curve once,
then pivot until it touches the curve again, generating a bitangent.
This pivoting is always possible because the bitangent is a cusp
bitangent (see below).
%
%	This will not necessarily work for conventional bitangents.
%	But it does when we incorporate cusps, as follows.
%	Push the free line until it touches a cusp (what if it can't reach a cusp?).
%	Then rotation is free about this cusp, until we become tangent to the curve.
%	(The problem with rotation about a non-cusp is that you are moving along
%	the curve in order to maintain tangency and you may go to infinity
%	before a second point of tangency is found.)
%
%	Notice that $C^*$ will have a (cusp) bitangent, since 
%	$C$ has an inflection point and $C^*$ has a cusp.
\QED

\vspace{-.3in}

\begin{corollary}
If $C$ is concave,
the bitangents of $C^*$ define a kernel set.
% need concave C, since a convex C will have C* with no bitangents
\end{corollary}

% A kernel set is a set of points that necessarily includes a kernel point.
% In dual space, this becomes a 
% collection of lines that necessarily includes a free line (Corollary~\ref{cor:free}).
% Equivalently, there exists a free line if and only if there exists
% a free line that is a bitangent of $C^*$.

But what are the bitangents of $C^*$?
We saw in Section~\ref{sec:bitang} that
bitangents come in two flavours: conventional bitangents
and cusp bitangents.
It turns out that $C^*$ only has cusp bitangents.

\begin{lemma}
If $C$ is a nontrivial curve,
all of the bitangents of $C^*$ are cusp bitangents.
\end{lemma}
\prf
Conventional bitangents of $C^*$ are associated
with self-intersections of $(C^*)^*$ (Section~\ref{sec:bitang}).
But $(C^*)^* = C$ \cite{jj02}.
Since $C$ is nontrivial, it has no self-intersections
and we conclude that $C^*$ has no conventional bitangents.
The only remaining source of bitangents is cusps.
\QED

\vspace{-.2in}

What are the points in primal space associated with cusp bitangents in
dual space?
A cusp bitangent is either a line between two cusps
% (right side of Figure~\ref{fig:comp1}) 
or a tangent through a cusp.
% (right side of Figure~\ref{fig:comp2})
Since cusps dualize to inflection tangents,
a line between two cusps dualizes to the intersection of the associated inflection
tangents (Lemma~\ref{lem:join}).
And a tangent of $C^*$ through a cusp dualizes to the intersection of 
the associated inflection tangent with the curve $C$.\footnote{Here, we again 
	appeal to the fact that $(C^*)^* = C$.}
We conclude that the bitangents of $C^*$ dualize to
the intersections of the inflection tangents of $C$ ($H_1$) 
and the intersections of these inflection tangents with $C$
($H_2$).
This is the kernel set of Theorem~\ref{thm:kernelset}
and proves the theorem.

% By definition, our kernel set $S$ will contain a kernel point (if the kernel is not empty).
Our kernel set $S$ will contain many kernel points, not just one.
However, since the kernel set corresponds to extremal free lines (free tangents)
in dual space, 
all of these points will inherently lie on the boundary of the kernel
(Figure~\ref{fig:free}).
Recall that our purpose in finding a point $P$ of the kernel was to create
a finite tangential curve $C^*$ for robust convex hull computation (Algorithm 4).
A boundary kernel point will not 
lead to the most robust convex hull computation,
since the tangential curve will almost be infinite.
To get a robust, interior point of the kernel,
we compute the centroid of the kernel points in $S$ 
(Figure~\ref{fig:mean}).
Since the kernel is convex, this centroid will also lie in the kernel.

\begin{figure}[h]
\begin{center}
\includegraphics*[scale=.5]{img/jjke9.jpg}
\end{center}
\caption{The centroid of $K$ is a good seed point of the kernel}
\label{fig:mean}
\end{figure}

%%%%%%%%%%%%%%%%%%%%%%%%%%%%%%%%%%%%%%%%%%%%%%%%%%%%%%%%%%%%%%%%%%%%%%%%%%%%%

We gather this material into an algorithm
for the computation of a kernel set, and an algorithm for
the trimming of this kernel set down to a seed point of the kernel.

\vspace{.2in}

\centerline{{\bf Algorithm 5: Computing the kernel set}}

\begin{enumerate}
\item Compute the inflection points of $C$.
\item If $C$ has none, it is convex and $\mbox{kernel}(C) = \mbox{interior}(C)$.
\item Compute the intersections $H_1$ of the inflection tangents.
\item Compute the intersections $H_2$ of the inflection tangents with $C$.
\item $S = H_1 \cup H_2$ is our kernel set.
\end{enumerate}

The tangential curve is used to isolate the kernel points in $S$.

\vspace{.2in}

\centerline{{\bf Algorithm 6: Computing a seed point from the kernel set}}

\begin{enumerate}
\item	Compute the tangential c-curve $C^*$.
\item	Compute $K = \{s \in S : \mbox{dual}(s) \cap C^* = \emptyset\}$.\footnote{This
		step will be altered in Section~\ref{sec:trim}.}
\item	$K$ is the set of kernel points in the kernel set.
	If $K = \emptyset$, the kernel is empty.
\item	Let $P$ be the centroid of $K$,
	an interior point of the kernel.
\end{enumerate}

$P$ is the desired seed point of the kernel,
bootstrapping the kernel computation.
Algorithms 5 and 6 implement step (1) of Algorithm 4
and complete this first algorithm for the kernel.

%%%%%%%%%%%%%%%%%%%%%%%%%%%%%%%%%%%%%%%%%%%%%%%%%%%%%%%%%%%%%%%%%%%%%%%%%%%%%

\subsubsection{Trimming the infinite tangential c-curve}
\label{sec:trim}

There is a robustness issue in the use of the tangential c-curve $C^*$
to compute the seed point $P$ of the kernel.
The purpose of computing a seed point is to allow $C$ to be moved
so that $C^*$ becomes finite and the convex hull of this finite $C^*$
is dual to the kernel.
Consequently, $C^*$ will inherently be infinite during the computation
of the seed point in Algorithm 6.
The infinite parts of $C^*$ will disrupt the analysis 
in dual space (testing free lines).
To make the computation of $K$ robust,
it is crucial that the infinite segments be trimmed away 
(Figure~\ref{fig:trim}).
% 	The removal of infinite regions
%	from tangential a-curves and b-curves is called 'clipping'
%	because it involves horizontal and vertical line intersections
%	with the curve that resembles clipping by a window.
%	In the context of tangential c-curves, it involves removing curve
%	segments, more suggestive of traditional trimming.
Fortunately, this is a simple task.

The tangential curve goes to infinity as the tangent of $C$
approaches the origin.
We shall trim $C^*$ surrounding tangents through the origin.

\begin{defn2}
The {\bf trimmed tangential c-curve} $C^*_{\mbox{\tiny{trim}}}$
is $C^*$ with the segments
\[
\{C^*(t_i - \epsilon, t_i + \epsilon),\ \ i=1,\ldots,n\}
\]
trimmed away,
where the tangents at $C(t_1),\ldots,C(t_n)$ pass through the origin.
See Figures~\ref{fig:tangthruorig} and \ref{fig:trim}.
$C^*_{\mbox{\tiny{trim}}}$ is a finite curve.
\end{defn2}

Section~\ref{sec:bitang} discusses how the 
tangents of $C$ through the origin can be computed,
using the tangential a- and b-curves of the tangential
curve system rather than the tangential c-curve
(which would be inherently unstable for this analysis).
% Step (2) of Algorithm 6 is changed to use the trimmed tangential c-curve.

% \begin{description}
% \item[(2)] Compute $K =
%	\{s \in S : \mbox{dual}(s) \cap C^*_{\mbox{\tiny{trim}}} = \emptyset\}$.
% \end{description}
% \vspace{.2in}

% tangents thru origin in primal space, untrimmed tangential curve in dual space
	% ADD DUAL SPACE TO FIGURE
\begin{figure}[h]
\begin{center}
\includegraphics*[scale=.5]{img/jjke10.jpg}
\end{center}
\caption{Tangents through the origin guide trimming}
\label{fig:tangthruorig}
\end{figure}

\begin{figure}[h]
\begin{center}
\includegraphics*[scale=.5]{img/jjke11.jpg}
\end{center}
\caption{The trimmed tangential curve, with associated trim regions in primal space}
\label{fig:trim}
\end{figure}

%%%%%%%%%%%%%%%%%%%%%%%%%%%%%%%%%%%%%%%%%%%%%%%%%%%%%%%%%%%%%%%%%%%%%%%%%%%%%

% \subsubsection{Accounting for trimming}
% \label{sec:trimarea}

If we are not careful,
the trimming of $C^*$ may lead to another type of mistake in the computation
of $K$.
A kernel point is a point avoided by the tangent space.
Since trimming $C^*$ is equivalent to removing some of the tangent
space of $C$,
a nonkernel point may now be misdiagnosed as a kernel point because the only
part of tangent space that touched it was trimmed away.
We must compensate for the trimming of $C^*$
by adding a test of $K$ against the tangent space that was trimmed away.
This additional test is performed in primal space using trim regions.

% The trimming of $C^*$ may affect the diagnosis of free lines in dual space.
% In particular, the set $K$ of kernel points (in Algorithm 6)
% may include some impostors if we test against $C^*_{\mbox{trim}}$ rather 
% than $C^*$.

\begin{defn2}
Suppose that the tangent at $C(t_i)$ passes through the origin,
so that the tangent space of $C(t_i - \epsilon, t_i + \epsilon)$ was removed
from $C^*$.
The {\bf trim region associated with $C(t_i)$} 
is the region of primal space covered by the tangent space
of $C(t_i - \epsilon, t_i + \epsilon)$ (Figure~\ref{fig:trimregion}).
The {\bf trim region} is the union of all of the
trim regions associated with points whose tangents pass through the origin.
\end{defn2}

\begin{figure}[h]
% \begin{center}
% \includegraphics*[scale=.5]{img/jjke12.jpg}
% \end{center}
\caption{A trim region}
\label{fig:trimregion}
\end{figure}

Step (2) of Algorithm 6 now becomes:

\begin{description}
\item[(2a)] Compute $\hat{K} = \{s \in S : \mbox{dual}(s) \cap C^*_{\mbox{\tiny{trim}}} = \emptyset\}$.
\item[(2b)] Remove the points from $\hat{K}$ that lie in one of the trim regions,
	yielding $K$.
\end{description}

This is a robust method of finding $K$, the kernel points in the kernel set.
The trim region is quite simple to compute.
Let $L_1$ and $L_2$ be the tangents at $C(t_i - \epsilon)$ and $C(t_i + \epsilon)$
respectively
and define the inside of $L_1$ (resp., $L_2$) 
to be the side that contains $L_2$ (resp., $L_1$).
The trim region associated with $C(t_i)$ 
is the region inside $L_1$ and outside $L_2$,
or inside $L_2$ and outside $L_1$.

%%%%%%%%%%%%%%%%%%%%%%%%%%%%%%%%%%%%%%%%%%%%%%%%%%%%%%%%%%%%%%%%%%%%%%%%%%%%

\subsection{Computing the kernel without the convex hull}
\label{sec:nohull}

In this section, we present our preferred algorithm for the smooth kernel,
which does not require any convex hull computation.
It does rely, however, on the tools that we have developed along the way
to the kernel-from-hull algorithm.
In preparation for a computation of the kernel
from the convex hull, we computed a single point of the
kernel and, in so doing, determined a collection of kernel points $K$.
We can compute the kernel directly from $K$,
without any subsequent convex hull computation.

We have observed that the points of $K$ all lie on the boundary of the kernel.
Actually, they lie at the {\em corners} of the boundary because
$K$ corresponds to free bitangents in dual space.
Free lines correspond to kernel points, free tangents to kernel
points on the boundary, and free bitangents to kernel points at the
corner of the boundary (since they are extremal in two ways).
This leads to the following algorithm for computing the kernel from $K$.

\begin{enumerate}
\item Compute $K$ using Algorithms 5 and 6 (including the trimming refinements of step 2).
\item Sort $K$ about the sample mean of $K$.
\item The kernel of $C$ is defined from $K$ as follows.
      Two consecutive points of the sorted $K$ should be joined by:
\begin{enumerate}
\item a curve segment of $C$ if both points are from $H_2$ 
(intersections of inflection tangents with $C$)
\item a straight line if either point is from $H_1$ 
(intersections of inflection tangents).
\end{enumerate}
\end{enumerate}

Figure~\ref{fig:nohull} gives an example.

\begin{figure}
\begin{center}
\includegraphics*[scale=.5]{img/jjke13.jpg}
\end{center}
\caption{The points in $K$ fully define the kernel}
\label{fig:nohull}
\end{figure}

We conclude that computing a single point of the kernel is as difficult
as computing the entire kernel,
since the computation of $K$ needed to find a single point can be used
instead to compute all of the kernel.

% \begin{lemma}
% These seed kernel points fully define the kernel (without any subsequent
% convex hull computation), as follows.
% Let $\{C^*(t_1), \ldots, C^*(t_n)\}$ be the cusps of $C^*$, sorted along $C^*$.
% Let $\{C^*(t_i) C^*(u_i): i = 1,\ldots,n\}$ be the tangents through the cusps.
% Kernel is curve segments $C(u_i) C(u_{i+1})$ and line segments
% $C(t_i) C(u_i)$(?).
% \end{lemma}

%%%%%%%%%%%%%%%%%%%%%%%%%%%%%%%%%%%%%%%%%%%%%%%%%%%%%%%%%%%%%%%%%%%%%%%%%%%%%

\subsection{Examples}
\label{sec:eg}

The algorithm has been fully implemented and we offer 
several examples in this section.
Figure~\ref{fig:banana} shows that 
the kernel of a banana will vary depending on the intersection
point of its inflection tangents.
Figure~\ref{fig:kernelob1a} shows a kernel
that contains the origin, showing the clear relationship with
the convex hull in dual space.
Figure~\ref{fig:complex} illustrates a more complicated multi-sided kernel.

	%	kernel data/banana1.pts; 
	%	kernel data/banana4.pts.
\begin{figure}
\begin{center}
\includegraphics*[scale=.5]{img/jjke15.jpg}
\includegraphics*[scale=.5]{img/jjke14.jpg}
\end{center}
\caption{The kernel of two bananas, one empty and one not}
\label{fig:banana}
\end{figure}

\begin{figure}
\begin{center}
\includegraphics*[scale=.5]{img/jjke16.jpg}
\end{center}
\caption{A kernel clearly dual to the convex hull}
\label{fig:kernelob1a}
\end{figure}

\begin{figure}
\begin{center}
\includegraphics*[scale=.5]{img/jjke17.jpg}
\end{center}
\caption{The kernel of a cloverleaf}
\label{fig:complex}
\end{figure}

%%%%%%%%%%%%%%%%%%%%%%%%%%%%%%%%%%%%%%%%%%%%%%%%%%%%%%%%%%%%%%%%%%%%%%%%%%%%%

\section{Conclusions}

We have developed algorithms for the convex hull of a closed curve
and the kernel of a closed curve.
Both involve dual space (to find bitangents or to find free lines).
Both use bitangents, the hull in primal space and the kernel in dual space.
Both require a seed point before the algorithm can begin
(the seed point of the kernel is considerably harder to find).
Both use the convex hull (or at least can)!
Both use free lines.

\begin{rmk}
Once we find a point of the kernel and translate it to the origin,
we can exclusively use tangential c-curves for any problem (e.g., bitangent),
since they have no problems with infinity, and ignore the subtlety of
clipping and tangential curve systems (Section~\ref{sec:duality}).
Notice however that many interesting curves have empty kernels,
for which this approach breaks down.
\end{rmk}

The convex hull of a curve is built out of bitangents.
And the kernel set in dual space is built out of bitangents.
Thus, no matter how we compute the kernel, as a convex hull
or through the kernel set, its computation is based upon bitangents.

Notice that our algorithm for finding the seed point of the kernel
is similar to finding the hull bitangents that define the
convex hull in dual space.

We can reveal which inflection tangents combine to define boundary points
of the kernel by mapping to dual space.
Lines between the duals establish
pairings and the correct pairings are recognized by being free.

\section{Acknowledgements}

I appreciate talks with Wenping and Gershon Elber on the kernel,
and the opportunity for research and a stimulating
environment offered by the Dagstuhl seminars.
Gershon Elber suggested the bananas of Section~\ref{sec:eg}.

%%%%%%%%%%%%%%%%%%%%%%%%%%%%%%%%%%%%%%%%%%%%%%%%%%%%%%%%%%%%%%%%%%%%%%%%%%%%%

\bibliographystyle{plain}
\begin{thebibliography}{Preparata 85}

\bibitem[Elber 01]{elber01}
Elber, G., M.-S. Kim and H.-S. Heo (2001)
The Convex Hull of Rational Plane Curves.
Graphical Models 63, 151--162.

\bibitem[Elber 02]{elber02}
Elber, G. (2002)
The kernel of a curve(?).
Presented at Dagstuhl 2002.

\bibitem[Graham 72]{graham72}
Graham, R. (1972)
An efficient algorithm for determining the convex hull of a finite planar set.
Information Processing Letters 1, 132--133.

\bibitem[Jarvis 73]{jarvis73}
Jarvis, R. (1973)
On the identification of the convex hull of a finite set of points
in the plane.
Information Processing Letters 2, 18--21.

\bibitem[Johnstone 01]{jj01}
Johnstone, J. (2001)
A Parametric Solution to Common Tangents.
International Conferenced on Shape Modelling and Applications (SMI2001),
Genoa, Italy, IEEE Computer Society, 240--249.

\bibitem[Johnstone 02]{jj02}
Johnstone, J. (2002)
The tangential curve.
Submitted for publication.

\bibitem[Lee 79]{lee79}
D.T. Lee and F. Preparata (1979)
An Optimal Algorithm for Finding the Kernel of a Polygon.
Journal of the ACM 26(3), 415--421.

\bibitem[Preparata 85]{preparata85}
Preparata, F. and M. Shamos (1985)
Computational Geometry: An Introduction.
Springer-Verlag (New York).

% \bibitem[Catmull 74]{catmull74}
% Catmull, E. (1974)
% A Subdivision Algorithm for Computer Display of Curved Surfaces.
% Ph.D. thesis, University of Utah.

\end{thebibliography}

%%%%%%%%%%%%%%%%%%%%%%%%%%%%%%%%%%%%%%%%%%%%%%%%%%%%%%%%%%%%%%%%%%%%%%%%%%%%%

{\bf Can postpone in first draft:}
Our kernet set is sharp, in the sense that an upperbound is sharp:
there is no smaller kernel set that would always define the kernel.
There are examples where the intersections of inflection tangents
define the kernel (if the convex hull is built out of lines between cusps,
such as a 3-cusped concave triangle) and other examples where the
intersections of inflection tangents with the curve define the kernel
(if the tangential curve has one cusp only),
so both components of the kernel set are necessary in some cases.

Open curves:

Our method won't work as is on an open concave curve,
because it assumes that K defines a closed set
and open curves will have open kernels that go off to infinity.
However, it could probably be easily changed to work:
at least to define the kernel inside the convex hull of
the open curve.
For example, add the line segment between the endpoints as
a pseudo-inflection tangent.

\section{Smooth visibility graph}

\end{document}
