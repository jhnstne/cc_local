% hull.tex
% Last updated: 10 February 2004
% Alternate ACM SIG Proceedings document using LaTeX2e
\documentclass{sig-alternate}

\newcommand{\Comment}[1]{\relax}  % makes a "comment" (not expanded)
\newcommand{\QED}{\vrule height 1.4ex width 1.0ex depth -.1ex\ \vspace{.3in}} % square box
\newcommand{\lyne}[1]{\mbox{$\stackrel{\leftrightarrow}{#1}$}}
\newcommand{\seg}[1]{\mbox{$\overline{#1}$}}
\newcommand{\prf}{\noindent{{\bf Proof}:\ \ \ }}
\newcommand{\choice}[2]{\mbox{\footnotesize{$\left( \begin{array}{c} #1 \\ #2 \end{array} \right)$}}}      
\newcommand{\scriptchoice}[2]{\mbox{\scriptsize{$\left( \begin{array}{c} #1 \\ #2 \end{array} \right)$}}}
\newcommand{\tinychoice}[2]{\mbox{\tiny{$\left( \begin{array}{c} #1 \\ #2 \end{array} \right)$}}}

\newtheorem{theorem}{Theorem}	
\newtheorem{rmk}[theorem]{Remark}
\newtheorem{example}[theorem]{Example}
\newtheorem{conjecture}[theorem]{Conjecture}
\newtheorem{claim}[theorem]{Claim}
\newtheorem{notation}[theorem]{Notation}
\newtheorem{lemma}[theorem]{Lemma}
\newtheorem{corollary}[theorem]{Corollary}
\newtheorem{defn2}[theorem]{Definition}
\newtheorem{observation}[theorem]{Observation}

\title{Giftwrapping a curve with the convex hull}
% A smooth Jarvis march
% Smooth convex hull of a curve
% A smooth Graham's scan for the convex hull of a curve
% Hulls, kernels, and giftwrapping of smooth curves
% The convex hull of a smooth curve
\numberofauthors{1}
\author{
\alignauthor J.K. Johnstone\thanks{This work was supported in part by the National Science Foundation under grant CCR-0203586.}\\
\affaddr{Geometric Modeling Lab}\\
\affaddr{Computer and Information Sciences}\\
\affaddr{University of Alabama at Birmingham}\\
\affaddr{University Station, Birmingham, AL, USA 35294}\\
\email{jj@cis.uab.edu}
}

\begin{document}
\conferenceinfo{ACMSE}{'04, April 2-3, 2004, Huntsville, Alabama USA.}
\CopyrightYear{2004}
\maketitle

\begin{abstract}
We develop an algorithm to compute the convex hull of a closed planar parametric curve.
The algorithm sweeps a tangent around the curve, using bitangents
to define events and effectively wrapping a tangent around the hull.
% giftwrapping the curve with the hull.
First, a point on the hull is found using a walk out of a concavity and next,
the entire hull is found using a walk punctuated by bitangents.
The algorithm is a smooth analogue to two classical computational geometry algorithms:
plane sweep and giftwrapping.
% We giftwrap the curve with the hull,
% This algorithm is a smooth analogue to the Jarvis march giftwrapping algorithm
% for computing the convex hull of a set of points.
% using a smooth analogue of the plane sweep algorithm, in which the vertical line
% is replaced by a tangent and bitangents are used as events.
It leverages our recent work on the computation of bitangents.
% The algorithm is practical since it only requires the computation of bitangents 
% and a few line-curve intersections.
% What about curves with cusps?} No problem, you simply have degenerate curve segments between
% bitangent endpoints.
% {\bf What about tritangents?} Not considered by Elber either. Simply ignore the middle point of
% tangency.
\end{abstract}

\category{I.3.5}{Computer Graphics}{Computational Geometry and Object Modeling}[Geometric algorithms, languages, and systems, splines.]
\keywords{Convex hull, bitangent, plane sweep, giftwrapping.}

%%%%%%%%%%%%%%%%%%%%%%%%%%%%%%%%%%%%%%%%%%%%%%%%%%%%%%%%%%%%%%%%%%%%%%%%%%%%%

\section{Introduction}

The bitangents of a curve are the lines that are tangent to the curve in two distinct places
(Figure~\ref{fig:convhullob3a}).
Once the bitangents of a curve are known, 
it becomes feasible to build many secondary structures at a reasonable cost,
such as the convex hull and the kernel.
In this paper, we consider the construction of the convex hull of a closed planar parametric curve
(Figure~\ref{fig:curveob1b}).
The convex hull is useful to simplify algorithms such as interference detection,
to diagnose concavities, and for general scene analysis.
Structures like the hull and kernel have been classically computed for polygons.
The calculation of these structures in a smooth environment,
as opposed to a polygonal environment built from a discrete sampling of the
smooth environment, can significantly simplify the structures and improve their accuracy.

In \cite{jj01}, we showed how to compute the bitangents of a parametric curve
efficiently using a dual representation of the tangent space of the curve,
and intersection in dual space.
Last year, we studied the kernel of a smooth curve 
and reduced it to a convex hull problem \cite{jj03}.
In this paper, we address the hull problem itself.

Our hull algorithm sweeps a tangent around the curve, 
using bitangents to define events % where convexity tests must be performed
and effectively wrapping a tangent around the hull.
This is a smooth analogue to two classical computational geometry algorithms:
plane sweep and giftwrapping.
We define these related approaches and previous work on the convex hull in Section~\ref{sec:work},
and explore the relationships to our algorithm in Section~\ref{sec:analysis}.
The algorithm for the convex hull is presented in Section~\ref{sec:hull}.
We conclude in Section~\ref{sec:conclusions}.

\begin{figure}[h]
\begin{center}
\begin{picture}(10,10)(0,0)
\put(70,-10){\makebox(0,0){(a)}}
\end{picture}
\includegraphics*[scale=.4]{../img/jjhu1Gray.jpg}
% \vspace{.2in}
\begin{picture}(10,10)(0,0)
\put(70,-10){\makebox(0,0){(b)}}
\end{picture}
\includegraphics*[scale=.4]{../img/jjhu2Gray.jpg}
\end{center}
\caption{(a) A smooth curve and (b) its convex hull}
\label{fig:curveob1b}
% convexhull -p -d 10 data/ob1b.pts
% kernel -p -d 10 data/ob1b.pts
\end{figure}

%%%%%%%%%%%%%%%%%%%%%%%%%%%%%%%%%%%%%%%%%%%%%%%%%%%%%%%%%%%%%%%%%%%%%%%%%%%%%

% \clearpage

\section{Related work}
\label{sec:work}

The convex hull of S is the smallest convex set that contains S.
The convex hull of a finite set of points, and by extension a polygon,
is one of the classical computational geometry problems,
and has been intensely studied for decades \cite{preparata85}.
We mention those algorithms most related to the smooth algorithm in this paper.

Graham's O($n \log n$) algorithm for the convex hull of a set of points in 2-space
 \cite{graham72}
is arguably the first paper in computational geometry. % \cite{orourke94}. % p. 8
% Consider the computation of the convex hull of $N$ points in the plane by Graham's scan.
In Graham's scan, we sort the $n$ points radially about an internal point
and then traverse this list, discarding points $p_i$ such that
$p_{i-1} p_i p_{i+1}$ makes a right turn (reflex angle).
If the points are the ordered vertices of a polygon,
Graham's scan may be interpreted as a sweep around 
the boundary of the polygon, testing for reflex angles.

The Jarvis march \cite{jarvis73} uses giftwrapping to compute the hull.
Starting at the lowest point, which must be a hull point, the next point
on the hull is the one with the smallest turning angle from horizontal.
This defines the first hull edge.
Subsequent hull points are again found as the point with the smallest turning angle,
this time from the present hull edge.
Notice that this can be interpreted as wrapping a cord around the hull, picking
up a hull vertex and hull edge whenever the cord touches a point.
Thus, it is a simple example of giftwrapping.
The Jarvis march is an $O(hn)$ algorithm, where $h$ is the number of vertices on the hull,
so it behaves best when many points lie in the interior of the hull.

The plane (or line) sweep \cite{preparata85} % also edelsbrunner,mehlhorn
is another algorithmic technique that is related to our smooth hull algorithm.
A vertical line is swept across the plane, punctuated by events.
% such as in the computation of the intersections of a line arrangement).
In a classical vertical sweep, say of a line arrangement, events are
defined by intersections and vertex endpoints and the order of the lines 
is adjusted at these events.

The convex hull of a smooth curve has received far less attention
than the convex hull of a set of points or a polygon.
Two other approaches are notable.
Bajaj and Kim \cite{bajaj91} compute the convex hull of an algebraic curve $f(x,y)=0$.
Elber, Kim and Heo \cite{elber01} compute the convex hull of a rational curve
using the zero-set of a system of equations to characterize the segments of the hull.
An emphasis is placed on avoiding the computation of bitangents that do not participate
in the hull.
Our emphasis is instead on finding the correct bitangents that 
lie on the hull when all bitangents are available, since
we are able to find all bitangents efficiently using our dual method \cite{jj01}.
We are also particularly interested in an easily implementable algorithm.
\cite{seong04} recently studied the convex hull of a parametric surface.

A note on the computation of bitangents, which are fundamental to this algorithm.
The computation of bitangents reduces to curve intersection in dual space, as follows
\cite{jj01}.
Using the dualization of a line in 2-space to a point in projective 2-space,
the tangent space of a curve (the space of its tangent lines) can be dualized
to a curve, called the tangential curve.
The bitangents of two curves A and B 
correspond to intersections of the tangential curves of A and B.

%%%%%%%%%%%%%%%%%%%%%%%%%%%%%%%%%%%%%%%%%%%%%%%%%%%%%%%%%%%%%%%%%%%%%%%%%%%%%

\section{Smooth hull}
\label{sec:hull}

Consider the convex hull of a closed plane curve $C$.

\begin{defn2}
A line is {\bf free} if it does not intersect $C$.
A point of tangency is not counted as an intersection.
\end{defn2}

The following lemma is also proved in \cite{elber01}. 

\begin{lemma}
\label{lem:condition}
Let $C$ be a closed plane curve.
The point $P \in C$ lies on the boundary of the convex hull of $C$ if and only if
the tangent at $P$ is free.
\end{lemma}
\prf
The tangent of a point on the boundary of the convex hull 
defines a halfplane that contains the curve 
(since the convex hull of a point set is the intersection of the halfplanes 
that contain the point set).
The tangent of a point inside a concavity lies inside the convex hull, so it
clearly intersects the curve.
%	More formally but no better,
%	A tangent defines a containing halfplane if and only if it is free.
%	The halfplane defined by the tangent at $P$ is clearly a limit halfplane
%	in this direction and therefore a boundary point of the hull if it is free.
%	Conversely, if $P$ lies on the convex hull boundary, 
%	then some halfplane through $P$ is containing.
%	But the only possible containing halfplane through $P$
%	is its tangent halfplane.
\QED

	% convexhull -p -d 10 data/ob1b.pts
\begin{figure}[h]
\begin{center}
\includegraphics*[scale=.4]{../img/jjhu3aGray.jpg} 
\includegraphics*[scale=.4]{../img/jjhu3bGray.jpg}
\includegraphics*[scale=.4]{../img/jjhu3cGray.jpg}
\end{center}
\caption{A point of the convex hull has a free tangent}
\label{fig:tangent}
\end{figure}

Since there are an infinite number of tangents to test for freedom on a curve,
Lemma~\ref{lem:condition} does not yield a useful algorithm for constructing the convex hull.
However, adding coherence does \cite{foleyvandam}.
Consider walking along the curve.
If the tangent is free at some point, it will remain free for some time.
In order to lose freedom, the tangent must begin to intersect the curve,
an event that will occur at a free bitangent. % technically, immediately after
By the same argument, a tangent will only regain freedom as it passes through a free bitangent.
We conclude that we do not need to test every tangent for freedom, only the bitangents.

\begin{theorem}
\label{thm:chull}
Let $C$ be a closed plane curve.
The convex hull of $C$ is bounded by its free bitangents
and the curve segments of $C$ between these bitangents.
\end{theorem}

This issue can be interpreted in another way.
The convex hull must cap the concavities of $C$,
and the appropriate caps are bitangents of $C$ (Figure~\ref{fig:curveob1b}b).
The only question is which bitangents?
Lemma~\ref{lem:condition} shows that the bitangents on the hull are the free bitangents.

% A better algorithm (a form of smooth giftwrapping)

The obvious algorithm suggested by Theorem~\ref{thm:chull} is to compute
the $b$ bitangents of $C$ \cite{jj01} and perform $b$ line-curve intersections 
to filter these bitangents down to free bitangents. 
The rest of this section shows that we can do better, on average.
The free bitangents can be computed with fewer intersection tests using a walk around the hull.

Imagine walking around the convex hull, starting at a point of $C(t)$
that lies on the convex hull.
To stay on the hull, it is enough to ensure that we never enter a concavity.
Fortunately, a free bitangent guards every concavity,
so encountering an endpoint of a bitangent during our hull walk
indicates that we are entering a concavity.
Whenever we meet a bitangent endpoint, we shall mark the associated bitangent
as a free bitangent, leap over to the other side of the bitangent
(thus leaping over the concavity), and continue the hull walk from there.
In this way, the free bitangents can be gathered by walking around the curve:
they are the only bitangents that are met during this walk.
The walk also gathers the hull curve segments.
Notice that the walk is virtual:
'walking' across a curve segment is equivalent to moving to the next
element in a sorted list of parameter values, as described below.
Here is the high level algorithm for computing the convex hull.

\vspace{.2in}

\centerline{{\bf Algorithm 1: Computing the convex hull of $C$}}

\begin{enumerate}
\item Compute the bitangents of $C$.
% \item Sort the parameter values of the endpoints of these bitangents.
\item Find a point $P \in C$ on the convex hull.
\item Walk around the curve starting at $P$, leaping across every bitangent that is met.
  The hull consists of the bitangents and curve segments encountered in this walk.
  Note that only free bitangents are found during this walk.
\end{enumerate}

This algorithm requires some elaboration.
Algorithms~2 and 3 below implement steps~2 and 3.
Since the boundary of the convex hull of a curve with no bitangents (a convex curve)
is simply $C$, we may assume without loss of generality that $C$ has at least one bitangent.
Let $\{E_0 = C(e_0), E_1 = C(e_1), \ldots, E_{2n-1} = C(e_{2n-1})\}$
be the endpoints of the bitangents of $C(t)$, where $e_0 < e_1 < \ldots < e_{2n-1}$, and
let $\{\seg{E_i E_{\mbox{\tiny{buddy}}(i)}}\}$ be the bitangents.
In Algorithms~2 and 3, $\oplus$ and $\ominus$ represent addition and subtraction in modulo arithmetic.

The walk must begin on the convex hull.
Motivated by the Jarvis march, we could find a point of the hull by finding the lowest
point of the curve.
The lowest point has an horizontal tangent or it is a cusp, and can be found by
solving $y'(t) = 0$.
A second approach is to use a walk to find a point of the hull, which unifies step 2 with
step 3.
% To bootstrap the walk,
% a seed point $P = C(t_{\mbox{\tiny{start}}})$
% of the convex hull must be found (Step 2).
% This seed point can be found using a walk.
Starting at an arbitrary point of the curve
% the beginning of the curve $C(a)$
% (where $C(t)$ is defined over $t \in [a,b]$)
and until we reach the convex hull, we can walk out of a concavity
by moving to the next bitangent endpoint.
This will soon reach a free bitangent.
Notice that the following algorithm places the starting point in the middle of a segment 
for robustness.
% We finish by moving the start point to the middle of a hull curve segment
% in order to make the ensuing walk robust.
% the seed point should be in the interior of a hull curve segment
% (e.g., not a bitangent endpoint).
% For the same reason, if $t_{\mbox{\tiny{start}}} = a$ is a free bitangent endpoint, 
% we should move 
% $t_{\mbox{\tiny{start}}}$ forward or backward to the midpoint of the
% next segment, whichever is free.
\Comment{
\item if the tangent at $C(a)$ does not intersect $C$
\begin{itemize}
\item if $a=e_i$ (some bitangent endpoint)
\begin{itemize}
\item if the tangent at $C(\frac{e_i + e_{i \oplus 1}}{2})$ does not intersect the curve
\begin{itemize}
\item choose $\frac{e_i + e_{i \oplus 1}}{2}$ as the starting point.
\end{itemize}
\item else choose $\frac{e_{i \ominus 1} + e_i}{2}$ as the starting point.
\end{itemize}
\item else choose $a$ as the starting point.
\end{itemize}
\item else
\begin{itemize}
}

\vspace{.2in}

\centerline{{\bf Algorithm 2:}}
\centerline{{\bf A walk to find a starting point on the convex hull}}

\begin{enumerate}

\item $i \leftarrow 0$; $t_{\mbox{\tiny{start}}} \leftarrow \frac{e_0 + e_1}{2}$
\item while the tangent at $C(t_{\mbox{\tiny{start}}})$ intersects $C$
\begin{enumerate}
\item	$i \leftarrow i + 1$
\item	$t_{\mbox{\tiny{start}}} \leftarrow \frac{e_{i} + e_{i \oplus 1}}{2}$
\end{enumerate}
\item $C(t_{\mbox{\tiny{start}}})$ is a point on the convex hull.
% \item $t_{\mbox{\tiny{start}}} \leftarrow \frac{e_i + e_{i \oplus 1}}{2}$.
%	(If $t_{\mbox{\tiny{start}}} = a$, we may have to move to 
%	$\frac{e_{i \ominus 1} + e_i}{2}$ instead, whichever is free.)
\end{enumerate}

Algorithm 2 walks out of a local concavity, probing the bitangents inside this concavity.
Each probe requires a line-curve intersection.
Since the bitangents inside a concavity are usually a small minority of the 
bitangents of the entire curve, 
the starting point can be found, on average, in far fewer than $b$ line-curve intersections
(where $b$ is the number of bitangents).
% since it only has to walk out of the local concavity.
% That is, the bitangents in a local concavity are probed, rather than the bitangents across the entire
% curve.
The exact number of line-curve intersections depends on the nesting of concavities and on
the position of the first probe, and varies from 1 in the best case to $b$ in the worst case
(consider a cup with only one bitangent covering its one concavity).
% not around the entire curve.
% This is why we prefer finding the free bitangents through a walk.
In theory, the lowest point solution is more efficient (no line curve intersections!), 
but in practice Algorithm~2 is also a good choice.

The remainder of the algorithm, the walk around the hull in Step~3 of Algorithm~1, 
requires no more line-curve intersections.
Consider an implementation of this walk.
This algorithm collects the boundary of the convex hull in $CH$.


\vspace{.2in}

\centerline{{\bf Algorithm 3: A walk around the convex hull}}

\begin{enumerate}
\item 	If $C$ has no bitangents, return($C$).
\item   $CH = \emptyset$ and let 
        $(e_{i \ominus 1},e_i)$ be the segment that contains $t_{\mbox{\tiny{start}}}$.
\item	do
\begin{enumerate}
\item	$CH = CH \cup \{\mbox{bitangent with endpoint $E_i$}\} $
        % [Walk to the next bitangent] 
	% add the bitangent with endpoint $E_i$ to the convex hull
% 	(which will be a free bitangent)
% \item   $\mbox{convhull}(C) = \mbox{convhull}(C) \cup$ 
%	bitangent with endpoint $E_i$ 
\item	$CH = CH \cup \{ S = \mbox{ segment from } E_{\mbox{\tiny{buddy}}(i)}$ 
  $ \mbox{to }$ $E_{\mbox{\tiny{buddy}}(i) \oplus 1} \}$
  % [Jump across this bitangent and continue to the next bitangent] 
	% add the curve segment $S$ starting at $E_{\mbox{\tiny{buddy}}(i)}$
	% and ending at $E_{\mbox{\tiny{buddy}}(i) \oplus 1}$
	% to the convex hull
% \item	$\mbox{convhull}(C) = \mbox{convhull}(C) \cup$ 
%	curve segment $S$ starting at $E_{\mbox{buddy}(i)}$.
\item	$i \leftarrow \mbox{buddy}(i) \oplus 1$
\end{enumerate}
 	while $S$ does not contain the starting point $P$
\item return($CH$)
\end{enumerate}

Step (3a) is equivalent to walking across a free bitangent of the hull.
Step (3b) is equivalent to walking across the next curve segment of the hull.
This walk requires no line-curve intersections and takes $O(f)$ time, 
where $f$ is the number of free bitangents.

\begin{example}
Let's trace the algorithm on the simple example of Figure~\ref{fig:algtrace}.
A starting point is found at $P$, by walking out of the concavity
at the beginning of the curve $b$.
During the walk of the convex hull, 
the bitangent between endpoints 1 and 2 is added first,
then the curve segment between 2 and 3, the bitangent between 3 and 4,
and so on until we add the curve segment between 0 and 1,
which contains $P$ and signals the completion of the hull.
\end{example}

	% convexhull -p -d 10 data/complexKernel.pts
\begin{figure}[h]
\begin{center}
\includegraphics*[scale=.5]{../img/jjhu5aGray.jpg} 
\includegraphics*[scale=.5]{../img/jjhu5cGray.jpg}
\end{center}
\caption{Computing the convex hull with a walk}
\label{fig:algtrace}
\end{figure}

Note that a curve segment in step~3b of Algorithm~3 may be degenerate of zero length,
if two bitangents meet at a point.
This can be true at a cusp of the curve.

Figure~\ref{fig:convhullob3a} is a good illustration of the power of the 
algorithm.
This curve has 23 bitangents, yet our walk can find the free bitangents
with only one line-curve intersection (to find the starting point)
rather than 23.

	% convexhull -p -d 10 data/ob3a.pts
\begin{figure}
\begin{center}
\includegraphics*[scale=.5]{../img/jjhu4aGray.jpg}  
\includegraphics*[scale=.5]{../img/jjhu4bGray.jpg}
\includegraphics*[scale=.5]{../img/jjhu4cGray.jpg}
\end{center}
\caption{A curve, its bitangents, and its convex hull}
\label{fig:convhullob3a}
\end{figure}

We conclude that the convex hull can be computed from the bitangents of a curve
simply by finding the lowest point and sorting the bitangent endpoints.

%%%%%%%%%%%%%%%%%%%%%%%%%%%%%%%%%%%%%%%%%%%%%%%%%%%%%%%%%%%%%%%%%%%%%%%%%%%%%

\section{The kinship to polygonal algorithms}
\label{sec:analysis}

The proposed algorithm shares traits with three classical algorithms from computational geometry.
First, it is a smooth version of the classical plane sweep algorithm, 
in which the vertical line is replaced by a tangent line
and events are defined by bitangents rather than line intersections.

% \ifTalk
% We can interpret our algorithm as a tangent sweep (a sweep through tangent space),
% a smooth version of plane sweep.

% In our sweep, we sweep a line controlled by the curve's tangent space,
% where events are defined by bitangents (intersections in dual space).
% In fact, we can avoid a physical sweep (just appealing to its structure)
% and get its effects from a sorting of bitangent parameter values.
% \fi

The algorithm is also related to the Graham scan.
Observe that a reflex angle on a polygon indicates a concavity.
The sweep around the polygon of Graham's scan, testing for reflex angles, is 
replaced by a sweep around the curve, 'testing' for free bitangents.
However, there are two significant differences.
% In the Graham scan, we must walk around the inside of the concavity until a nonreflex angle is
% found, at which point the hull continues.
Since the bitangents have been precomputed, we can immediately 
bridge the concavity without traversing it,
whereas the Graham scan must descend into the concavity.
Moreover, we don't actually have to perform a freedom test at the bitangent events, 
since the answer will always be in the affirmative.

Consequently, the proposed algorithm looks most like the Jarvis march.
We use a smooth version of giftwrapping, in which 
the boundary of the hull is found by wrapping around until the next bitangent endpoint is found,
which is necessarily the bitangent with the smallest turning angle.
Just as the Jarvis march never visits the vertices and edges inside the hull,
in the final walk of our algorithm, we do not visit the parts of the curve inside the hull.
Consequently, our algorithm shares the output-sensitivity (efficiency 
related to the size of the hull) of the Jarvis march.
Once again, no computation (in this case of the smallest turning angle) is necessary
once the bitangents have been sorted.

% Notice that events associated with inactive bitangents are never encountered.

%%%%%%%%%%%%%%%%%%%%%%%%%%%%%%%%%%%%%%%%%%%%%%%%%%%%%%%%%%%%%%%%%%%%%%%%%%%%%

\section{Conclusions}
\label{sec:conclusions}

We have presented an algorithm for computing the convex hull of a parametric curve.
It defines the hull from the free bitangents that span its concavities
and finds the free bitangents without a direct test for freedom, using a walk that
stays on the hull simply by sorting the parameter values of bitangent endpoints.
We are presently studying other problems in computer graphics that rely on the
isolation of a subset of bitangents from a scene.

\section{Acknowledgements}

This work was supported in part by the National Science Foundation under grant
CCR-0203586.

\bibliographystyle{plain}
\begin{thebibliography}{99}

\bibitem{bajaj91}
Bajaj, C. and M.-S. Kim (1991)
Convex hulls of objects bounded by algebraic curves.
Algorithmica 6, 533--553.

\bibitem{elber01}
Elber, G., M.-S. Kim and H.-S. Heo (2001)
The Convex Hull of Rational Plane Curves.
Graphical Models 63, 151--162.

\bibitem{foleyvandam}
J. Foley, A. van Dam, S. Feiner and J. Hughes (1996)
Computer Graphics: Principles and Practice.
Second Edition in C.
Addison-Wesley (Reading, MA).

\bibitem{graham72}
Graham, R. (1972)
An efficient algorithm for determining the convex hull of a finite planar set.
Information Processing Letters 1, 132--133.

\bibitem{jarvis73}
Jarvis, R. (1973)
On the identification of the convex hull of a finite set of points
in the plane.
Information Processing Letters 2, 18--21.

\bibitem{jj01}
Johnstone, J. (2001)
A Parametric Solution to Common Tangents.
International Conference on Shape Modelling and Applications (SMI2001),
Genoa, Italy, IEEE Computer Society, 240--249.

\bibitem{jj03}
Johnstone, J. (2003)
Kernels from hulls.
41st Annual ACM Southeast Conference, Savannah, 354--358.

\bibitem{preparata85}
Preparata, F. and M. Shamos (1985)
Computational Geometry: An Introduction.
Springer-Verlag (New York).

\bibitem{seong04}
Seong, J., G. Elber, J. Johnstone and M.-S. Kim (2004)
The Convex Hull of Freeform Surfaces.
Computing, to appear.

\end{thebibliography}

\balancecolumns

\end{document}
