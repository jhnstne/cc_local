\documentclass[12pt]{article}
\usepackage[pdftex]{graphicx}
% \usepackage{latex8}
\usepackage{times}
\input{header}

\newif\ifVideo
\Videofalse
\newif\ifTalk
\Talkfalse
\newif\ifJournal
\Journalfalse

\newcommand{\plucker}{Pl\"{u}cker\ }
\newcommand{\tang}{tangential curve\ }
\newcommand{\tangs}{tangential curves\ }
\newcommand{\Tang}{Tangential curve\ }
\newcommand{\atang}{tangential $a$-curve\ }
\newcommand{\btang}{tangential $b$-curve\ }
\newcommand{\ctang}{tangential $c$-curve\ }
\newcommand{\acurve}{$a$-curve\ }
\newcommand{\bcurve}{$b$-curve\ }
\newcommand{\ccurve}{$c$-curve\ }
\newcommand{\atangs}{tangential $a$-curves\ }
\newcommand{\btangs}{tangential $b$-curves\ }
\newcommand{\ctangs}{tangential $c$-curves\ }

\setlength{\oddsidemargin}{0pt}
\setlength{\topmargin}{-1in}	% should be 0pt for 1in
% \setlength{\headsep}{.5in}
\setlength{\textheight}{8.6in}
\setlength{\textwidth}{6.875in}
\setlength{\columnsep}{5mm}	% width of gutter between columns

% \markright{Chaikin: \today \hfill}
% \pagestyle{myheadings}

% \SingleSpace

% -----------------------------------------------------------------------------

\title{Stability allows laziness: the stability of Chaikin curves in dual space}
% Lazy subdivision and Chaikin bitangents
% The many uses of dual subdivision curves
% Primal Evaluation of Dual Chaikin Curves
% Dual Chaikin curves, primal evaluation, and bitangents
\author{J.K. Johnstone\\
	Geometric Modeling Lab\\
	Computer and Information Sciences\\
	University of Alabama at Birmingham\\
	University Station, Birmingham, AL 35294}

\begin{document}
\maketitle

\begin{abstract}
We develop an algorithm for the construction of bitangents of Chaikin curves.
The computation is performed in dual space, using dual Chaikin curves.
An adaptive subdivision strategy is used, controlled by the search
for bitangents, which we call lazy subdivision.

This paper is also an advertisement for dual Chaikin curves,
whose stationary vertex property leads to efficient intersection
and interpolation of the initial data points defining the Chaikin polygon.
% and predictable interpolation from early stages of the subdivision.
Upon developing a direct subdivision step for dual Chaikin curves,
the direct use of dual Chaikin curves is shown to be expensive, 
suggesting the use of the associated Chaikin curve to support operations
on the dual Chaikin curve.
\end{abstract} 

\clearpage

% -----------------------------------------------------------------------------

\section{Introduction}

Bitangents are important.
Subdivision curves are swiftly becoming a standard smooth curve representation.
Bitangents can be computed as intersections in dual space.
A lazy subdivision technique guarantees efficient convergence to the bitangent.

% -----------------------------------------------------------------------------

\section{Subdivision curves}

A subdivision curve is defined as the limit of a series of subdivision steps.
Each subdivision step can itself be expressed as a splitting step followed
by an averaging step \cite{stollnitz96}.
Consider a subdivision curve initially defined by vertices 
$(c_0^0, c_1^0, \ldots, c_{n-1}^0)$
(Figure~\ref{fig:splitavg}).
The splitting step, which effectively adds the midpoint to each edge, is 
\[
%	\stackrel{\circ}{c}
\begin{array}{ll}
	\bar{c}_{2i}^j   & = c_i^{j-1} \\
	\bar{c}_{2i+1}^j & = \frac{c_i^{j-1} + c_{i+1}^{j-1}}{2} \\
\end{array}
\]
yielding $(\bar{c}_0^1, \ldots, \bar{c}_{2n-1}^1)$ after the first split.
The averaging step
\[
	c_i^j 	= \sum_k r_k \bar{c}_{i+k}^j
\]
where $\{r_k\}_k$ is called the averaging mask,
yields $(c_0^1,\ldots,c_{2n-1}^1)$ after the first subdivision step.
% This splitting and averaging must be performed for each coordinate.
This representation in terms of splitting and averaging has the advantage of defining any subdivision curve.
The averaging mask is the only element that distinguishes two subdivision curves.

\begin{figure}[h]
\begin{center}
\includegraphics*[scale=.3]{img/splitavg.jpg}
\end{center}
\caption{Chaikin subdivision. Top left: original Chaikin polygon.
	Top right: After splitting; Bottom left and right: After averaging}
\label{fig:splitavg}
% splitavg.rgb
\end{figure}

% -----------------------------------------------------------------------------

\section{Chaikin curves}

The Chaikin curve is the simplest subdivision curve,
independently invented by \cite{chaikin74} and \cite{deRham47,deRham56}.
The averaging mask for the Chaikin curve is $(r_0,r_1) = (\frac{1}{2},\frac{1}{2})$.
That is,
\[
	c_i^j 	= \frac{\bar{c}_i^j + \bar{c}_{i+1}^j}{2}
\]
The entire subdivision step is:
\[
\]
This yields the local subdivision matrix for the Chaikin curve:
\[
\frac{1}{4} \left[
\begin{array}{cc}
3 & 1 \\
1 & 3
\end{array}
\right]
\]
Through eigenanalysis of this matrix, an evaluation mask can be found \cite{stollnitz96}. % p. 72
The vertex $c^0$, with neighbours $c_{-}^0$ and $c_{+}^0$,
converges to the point:
\[
	c^{\infty} = \frac{c_{-}^0 + c^0}{2}
\]
{\bf Is this correct: shouldn't it be $\frac{c^0 + c_{+}^0}{2}$?}

Through further eigenanalysis of the local subdivision matrix (using the second
left eigenvector rather than the dominant left eigenvector \cite{halstead93}),
a tangent mask can also be developed:
\[
\]

% put fig:splitavg here

Riesenfeld showed that Chaikin curves generate quadratic B-spline curves 
\cite{riesenfeld75}.

\clearpage

% -----------------------------------------------------------------------------

\section{Dualization}

We deal with closed Chaikin polygons, so every vertex has 2 adjacent edges.

The line $ax+by+c=0$ is $a$-dual to the point $(c,b,a)$ in projective 2-space
and $b$-dual to the point $(a,c,b)$.
Horizontal lines are mapped to infinity under $a$-duality,
while vertical lines are mapped to infinity under $b$-duality.
The edge PQ of a Chaikin polygon is mapped to the vertex dual(PQ) of a dual Chaikin polygon,
in both a-dual space and b-dual space.
Horizontal lines are not mapped under a-duality and
vertical lines are not mapped under b-duality.
The vertex V common to edges E and F of a Chaikin polygon is mapped to the edge between
the vertices dual(E) and dual(F).
This edge of the dual Chaikin polygon is not defined (and should not be drawn)
if either vertex dual(E) or dual(F) is at infinity
(or if a sweep from the edge E to F passes through a line that will map to infinity,
since as the curve is subdivided further, an 'infinite' line will eventually appear in primal space,
separating these two vertices in dual space).
Consider a test if a sweep from E to F passes through a horizontal line.
--- [see straddleHoriz].

\section{Dual Chaikin curves}
\label{sec:dualchai}

\cite{sabin87,dyn92,levin00}.
What is the closed form for the added vertex in dual Chaikin?

% -----------------------------------------------------------------------------

\subsection{Dual Chaikin subdivision is corner adding}

Insights: 
\begin{enumerate}
\item Corner adding adds a new edge between two existing edges.
	In dual space, a new vertex is added between two existing vertices
	(and the edge connecting the two old vertices becomes
	a pair of edges).
\item {\bf Chaikin curves are built as an envelope of lines in the limit.}
	The envelope is sampled more and more heavily as the subdivision
	continues.  The final curve is the envelope of all of the lines
	defined by the edges.  (Notice that edges are stationary throughout
	a subdivision process.)
	{\bf Dual Chaikin curves are the dual: they are built as a locus of a point
	in the limit, sampling this locus more and more densely as the subdivision
	continues.}
\item While subdivision of Chaikin curves involves corner cutting, 
	subdivision of dual Chaikin curves involves corner {\em adding}
	(see \cite[p. 212, Fig. 1]{dyn92}).
	
      The effect of a primal subdivision step is the introduction of a new edge
	joining two interpolated vertices (cutting a corner).
	The effect of a dual subdivision step is the introduction of a new vertex
	at the intersection of two 'interpolated lines' (adding a corner).
\end{enumerate}

The joining of 2 new vertices by a new edge in primal Chaikin
is replaced in dual Chaikin by the intersection of 2 new lines, 
yielding a new vertex.
In primal space, splitting and averaging are the construction of new vertices
through linear interpolation of old vertices.
In dual space, splitting and averaging are the construction of new lines
through the intersection point of old lines.  
{\bf What is it called when you create a new line in a pencil?}


% -----------------------------------------------------------------------------
\subsection{The stationary vertices of dual Chaikin curves}

{\bf In primal space}, all vertices change from one step to the next,
but {\bf the edges remain constant} (i.e., the lines defining the edges remain
constant, as well as an interior part of the edge).
{\bf In dual space}, all edges change from one step to the next, but 
{\bf the vertices remain constant and stationary}.
Vertices are added by subdivision, but once introduced, they never move.
Similarly, in primal space, each edge remains stationary (although it shortens,
eventually to a line of zero length, or a point) and defines a tangent in the
final subdivision curve.
In other words, in dual space, position data is always correct, 
and is refined in each subsequent subdivision,
while, in primal space, tangent data is always correct but continually expanded.
The position-preservation property in dual space is useful,
since position is often more important than tangent information.
The description of a Chaikin curve in primal space is fundamentally
as an envelope of lines, with continually increasing information on the defining lines,
but complete information about points only in the limit.
Whereas {\bf the description of a Chaikin curve in dual space is fundamentally
as a collection of points, with a continually increasing sampling of these
points} (and thus complete information about tangents only in the limit).
Since the latter description is more standard, the dual Chaikin curve is
in many ways a more natural description.

Another way of expressing the stationary vertex property of dual Chaikin curves 
is that dual Chaikin curves are an interpolatory subdivision scheme.
This makes them comparable to other interpolatory subdivision schemes
like DLG curves \cite{dyn87}.

\begin{claim}
Any (non-interpolatory, corner-cutting) subdivision scheme can be made into an interpolatory scheme using
its dual scheme.
For example, all of Lane-Riesenfeld's subdivision schemes.
\end{claim}

Dealing with infinity.
(Notice that we cannot clip at diagonal tangents as with parametric curves,
since we only have discrete information about a subdivision curve.)

\begin{itemize}
\item	Must turn off infinite dual points (e.g., a = 0 for a-duality)
	and dual edges incident to them.
\item	In initial (any other?) case, primal vertices whose dihedral angle
	straddles the x-axis (a horizontal tangent)
	must have their associated a-dual edge turned off (since
	it represents an edge through infinity).
	Same for b-duality.
	See code for how to test if a dihedral angle straddles the x-axis.
\end{itemize}

\clearpage

% -----------------------------------------------------------------------------

\section{Designing with dual Chaikin curves}

Given some vertex input, we would like to design a curve or surface interpolating this input.  
The design of interpolating curves is a difficult task 
for subdivision curves and surfaces: 
they are better suited to building approximating curves.
For example, consider the natural shapes built by Catmull-Clark surfaces,
and the loss of continuity and relative reduction in shape quality 
when a Catmull-Clark surface is replaced by an interpolating butterfly surface.
The discussion of Section~\ref{sec:dualchai} helps us to understand this phenomenon
[lift the above arguments to Catmull-Clark surfaces]:
primal subdivision is naturally an approximating method, or equivalently
an interpolating method for tangents, not points.
This suggests the following approach: design approximating curves and surface in primal space
(classical subdivision curves and surfaces) but design interpolating curves and surfaces
in dual space.

\subsection{Evaluating dual Chaikin curves in primal space}

We would like to be able to design with dual Chaikin curves
because of their natural interpolatory properties.
However, evaluation of subdivision curves is more natural in primal space
than dual space, which leads to some difficulties, which we now elaborate.

Dyn et. al. and Levin et. al. present dual Chaikin as a line-based version of 
the point-based Chaikin algorithm:
\[
  l_{2i} = 3/4 l_i + 1/4 l_{i+1} \\
  ---
\]
However, this is expensive to evaluate.

Direct evaluation of dual Chaikin.

It is more efficient to evaluate them in the original primal space
and dualize the result.
This requires some elaboration.

----
Pseudo-horizontal edges must be refined in b-primal space.
Pseudo-vertical edges in a-primal space.
Intermediate edges in either.

Only have to dualize even edges (2j-1,2j) since 
edges (2j,2j+1) remain the same as previous stage.

And what about concave polygons? (Not covered by Dyn/Levin/Liu.)

\clearpage

% -----------------------------------------------------------------------------

\section{Intersecting dual Chaikin curves}

Some computations with Chaikin curves are more easily handled by translating
to their dual Chaikin brothers.
An example is the computation of bitangency:
\begin{quote}
The bitangents of a pair of Chaikin curves can be found using the intersections
of the corresponding pair of dual Chaikin curves.
\end{quote}
This simplification of a problem by translating to the dual world is analogous
to the Fourier analysis of signals.
Many operations on a signal are simplified by mapping to the frequency domain,
the tool for which is the Fourier transform.
For example, convolution reduces from integration in the time domain
to multiplication in the frequency domain.
In this sense, the translation of subdivision curves/surfaces from the primal to
dual representation is analogous to Fourier analysis.

The advantage of mapping bitangency to the dual world is that intersection is a simpler
operation than bitangency.
Moreover, we shall see that {\bf intersection in dual space is even simpler than intersection
in primal space}.

\subsection{Bitangency is intersection}

\subsection{Intersection in dual space using lazy subdivision}
% Chaikin and dual Chaikin intersection

You never compute intersection points in dual space:
if you know two dual edges intersect (without computing their intersection),
then the associated primal vertices define a bitangent.

This leads to a different definition of bitangent:
a bitangent is an edge AB between vertices A and B of two Chaikin curves
such that edges $A^*$ and $B^*$ of the associated dual Chaikin curves intersect.
There is no need to evaluate tangents of the Chaikin curves using tangent masks.
This is a multiresolution definition of bitangent.
The definition is correct at the present level of granularity.
In the limit, this is a true bitangent.

\begin{defn2}
Lazy subdivision during intersection.
\end{defn2}

See Sabin tutorial notes.
{\bf An intersection of Chaikin curves at stage $i$ 
may disappear in future stages as the curve contracts inward.}
FIGURE.
Thus, lazy subdivision is not fruitful in Chaikin intersection.

{\bf However, an intersection of dual Chaikin curves at stage $i$ indicates
a corresponding intersection in all future stages, including the limit.}
The position of the intersection will change, but not its presence.
This can be seen using the stationary property of dual Chaikin vertices.
An intersection is an edge crossing.
Suppose that edge ab of dual Chaikin C1 crosses edge AB of dual Chaikin C2.
In future stages, C1 will still travel from a to b.
FORMALIZE THE PROOF.

\begin{defn2}
An edge in dual space is active if and only if it contains an intersection point
with the other tangential curve in dual space
(or it is an infinite edge, which we cannot diagnose for intersection yet).
Don't want to discard an infinite edge yet, since 
one of its finite children may intersect.
\end{defn2}

Active edges in dual space correspond to active corners in primal space
(Figure).

\begin{theorem}
If a segment is inactive in stage $i$, it remains inactive in stage $j>i$.
(With the exception of anomalies at infinity, which are probably concentrated
in the first stages of subdivision.  For example, see bichai1.pts.)
\end{theorem}

Try to only subdivide the active segments.
Will the subdivision process allow it?
Must we store the entire array of vertices, even though only some are active.
Think of a way to condense the array, storing only the active vertices,
but allowing simple access to the correct index item.

Statistics on efficiency improvement.
Compare the number of active segments to the total number of segments
in several examples.
Analyze the convergence of the 2 bases of the bitangent through subdivision
levels.

-----

Tangents of a Chaikin curve are fixed (once introduced) in primal space,
while vertices of the dual Chaikin curve are fixed in dual space.
Edges are not fixed in dual space, so if two edges intersect in stage i,
their children edges may not intersect in stage i+1.
{\bf We want a condition that guarantees that the intersection will be preserved in
the next stage (and then eventually in the limit).}
In terms of one curve, we want to know how far a new point can move from its parent edge
in dual space.
{\bf Instead of a condition that guarantees that 2 edges continue to intersect in the next stage,
it may be better to develop a condition that guarantees that 2 curve segments continue to
intersect in the next stage.}

It is also important to understand how a subdivision curve (in dual space) can be split
into 2 halves.
Actually, this is trivial since the curve is defined simply by a polygon,
so splitting the polygon in two is sufficient.

% -----------------------------------------------------------------------------

\section{A bitangent mask}

Can we build a bitangent mask?  Involves 2 curves, so very difficult.

Other subdivision curves: usually split/average generalization of Chaikin
in Stollnitz et.al. (Restrict to Chaikin in conference paper.)
See \cite{lane80} for extensions of Chaikin as well.

% -----------------------------------------------------------------------------

\clearpage

\section{Future work}

Dual Loop surfaces and bitangent developables.

%%%%%%%%%%%%%%%%%%%%%%%%%%%%%%%%%%%%%%%%%%%%%%%%%%%%%%%%%%%%%%%%%%%%%%%%%%

\bibliographystyle{unsrt}
\begin{thebibliography}{Lozano-Perez 83}

\bibitem[Chaikin 74]{chaikin74}
Chaikin, G. (1974)
An Algorithm for High-Speed Curve Generation.
Computer Graphics and Image Processing 3, 346--349.

\bibitem[de Rham 47]{deRham47}
de Rham, G. (1947)
Un peu de mathematique a propos d'une courbe plane.
Elemente der mathematik 2, 73--76, 89--97.

\bibitem[de Rham 56]{deRham56}
de Rham, G. (1956)
Sur une courbe plane.
J. Math. Pures Appl. 39, 25--42.

\bibitem[Dyn 87]{dyn87}
Dyn, N. and D. Levin and J. Gregory (1987)
A 4-point interpolatory subdivision scheme for curve design.
Computer-Aided Geometric Design 4(4), 257--268.

\bibitem[Dyn 92]{dyn92}
Dyn, N. and D. Levin and D. Liu (1992)
Interpolatory convexity-preserving subdivision schemes for curves and surfaces.
Computer-Aided Design 24(4), 211--216.

\bibitem[Halstead 93]{halstead93}
Halstead, M. and M. Kass and T. DeRose (1993)
Efficient, Fair Interpolation using Catmull-Clark Surfaces.
SIGGRAPH '93, 35--44.

\bibitem[Lane 80]{lane80}
Lane, J. and R. Riesenfeld (1980)
A Theoretical Development for the Computer Generation and Display of
Piecewise Polynomial Surfaces.
IEEE Transactions on Pattern Analysis and Machine Intelligence 2(1), 35--46.

\bibitem[Levin 00]{levin00}
Levin, D. and I. Wartenberg (2000)
Convexity-preserving Interpolation by Dual Subdivisions Schemes.
Proceedings of Saint-Malo Curves and Surfaces Conference, 1--10.

\bibitem[Riesenfeld 75]{riesenfeld75}
Riesenfeld, R. (1975)
On Chaikin's Algorithm.
Computer Graphics and Image Processing 4, 304--310.

\bibitem[Sabin 87]{sabin87}
Sabin, M. (1987)
Envelope Curves and Surfaces.
In 'The Mathematics of Surfaces', R. Martin, editor, 413--418.

\bibitem[Stollnitz 96]{stollnitz96}
Stollnitz, E. and T. DeRose and D. Salesin (1996)
Wavelets for Computer Graphics: Theory and Applications.
Morgan Kaufmann, San Francisco.

\end{thebibliography}

\section{Visibility graphs of subdivision curves}

\end{document}
