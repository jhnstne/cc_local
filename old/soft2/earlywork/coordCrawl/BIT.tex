\documentstyle[11pt]{article} 
\input{macros}
\DoubleSpace
\setlength{\oddsidemargin}{0pt}
\setlength{\evensidemargin}{0pt}
\setlength{\headsep}{0pt}
\setlength{\topmargin}{0pt}
\setlength{\textheight}{8.75in}
\setlength{\textwidth}{6.5in}
%
\title{A Localized Method for Intersecting Plane Algebraic Curve Segments}
\author{John K.\ Johnstone and Michael T.\ Goodrich\thanks{This 
	author's research was supported by
	the National Science Foundation under Grant CCR-8810568.}
	\\[5pt]
	Dept.\ of Computer Science, The Johns Hopkins Univ., Baltimore,
	MD 21218}
% Technical Report 88-20
%
\begin{document}
%
\maketitle
%
\begin{summary}{Summary of Results}
In this paper, we present a local method for the 
computation of the intersections of plane algebraic curve segments.
The conventional method of intersection is global, because it must first find all 
of the intersections between two curves before it can restrict to the segments in question;
hence, it cannot take advantage of situations in which one is dealing with the intersection of
short curve segments on complex curves.
Our local method, on the other hand, will directly find only those intersections that lie 
on the segments, as it is based upon an extension of methods for tracing along a curve.
\end{summary}

{\bf Keywords: intersection, algebraic curves, curve tracing, plane sweep, resultants,
	theory of elimination, geometric modeling, solid modeling}

{\bf CR classification: I.3.5}

\section{Introduction} 

Intersection is one of the most universal and basic problems in geometric modeling.
Although all of the boolean operations are axiomatic to geometric modeling, 
intersection is particularly important and particularly challenging.
It is usually required for the definition of a geometric model (e.g., CSG (Requicha 1980))  
and it is fundamental to applications of the model such as interference detection or 
hidden-surface elimination (Mortenson 1985; McKenna 1986; Sechrest and Greenberg 1982).
The classical view of the intersection of algebraic curves and surfaces is 
that it is equivalent to the solution of a 
simultaneous system of equations, such as \{$f(x,y)=0$, $g(x,y)=0$\} for two
plane algebraic curves.
(Since the representation of the curves is
	the implicit representation, methods of intersection that rely on 
	the parametric representation (Mortenson 1985) are not applicable.)
Canonically, the system of equations is reduced to a single univariate equation, 
the univariate equation is solved, and full solutions are built from these partial 
solutions.
% (Section~\ref{sec-soe} provides more detail on this procedure.)
An artifact of this approach is that all of the intersections are found.
But in geometric modeling one is usually interested only in the intersections
between two short segments of the curves.
Thus, if one follows this approach, then one must first compute 
all of the intersections between the entire curves
and then decide which of these intersections actually lie on the segments 
(a decision that is decidedly nontrivial, involving the sorting of 
points along a curve (Johnstone 1987)).
Therefore, segment intersection is clearly more complex than curve intersection
when one uses the traditional approach.

One would hope, then, for a method that makes segment intersection simpler than curve 
intersection, especially if the segments are short.
This is especially urgent because, as geometric models become more complex with 
the natural maturation of the field, intersection of higher-degree curves 
with the global system of equations method is becoming prohibitively expensive, 
while the curve segments involved remain short.
In this paper, we shall present a new method for the intersection of plane
algebraic curve segments.
The method is input-sensitive: the simpler and shorter the segment, 
the more efficient the intersection computation.

Our method is based upon crawling, a method for moving along a curve that has 
received much attention of late (Bajaj et. al. 1987; Dobkin et. al. 1986; Hoffmann 1987;
Hoffmann and Lynch 1987; Owen and Rockwood 1987; Timmer 1977).
We find intersections by crawling along the two segments in a coordinated fashion.
Since crawling is a method for moving along a single curve, we must adapt it to two curves
(Section~\ref{sec-coor}).
This crawling is easiest if both segments are {\em xy-monotone} (monotone with 
respect to both coordinate axes), so the curve segments are first decomposed into 
xy-monotone segments (Section~\ref{sec-decomp}).
Two methods of coordinated crawling along xy-monotone segments are presented: 
the simultaneous and the staircase crawl (Sections~\ref{sec-simul} and \ref{sec-stair}).
Several optimizations are also suggested,
including a way of recognizing when two xy-monotone segments cannot intersect 
% (so that in the intersection of each pair of xy-monotone segments, some pairs can be 
% ignored and the coordinated crawl along other pairs can be aborted early)
and two ways of eliminating long crawls within a coordinated crawl (Section~\ref{sec.improve}).
% by replacing with more direct (and hopefully faster) line-curve intersections.
A variant of the plane sweep method (familiar from computational geometry:
Bentley and Ottmann 1979; Edelsbrunner 1987; Preparata and Shamos 1985) is used to find
the intersections of the collection of xy-monotone segments that comprise the two segments
that we are intersecting, using several calls to coordinated crawling (Section~\ref{sec-sweep}).
We end with some conclusions.

\section{The system-of-equations (SOE) method of intersection}
\label{sec-soe}

In this section we review the global method of intersection that we are trying to replace, 
which we refer to as the {\em system of equations (SOE) method of intersection}.
An example will clarify the details of the SOE method.

\Comment{
IS IT WORTHWHILE TO GO INTO THE DETAILS OF THE SOLUTION OF THE SYSTEM OF EQUATIONS?
NO: JUST COMPLICATES DESCRIPTION: YOU DON'T HAVE TO KNOW ABOUT RESULTANTS TO 
	APPRECIATE THAT THE GLOBAL METHOD IS WORSE.
YES: BUT IT IS GOOD TO FULLY UNDERSTAND THE SOE METHOD WHEN DISCUSSING LINE-CURVE 
	INTERSECTIONS IN SECTION~\ref{IMPROVE}
}

\begin{example}
Suppose that we wish to find the intersection of the two plane algebraic curves 
$f(x,y)=0$ and $g(x,y)=0$.
The two equations are reduced to a single equation and a variable is eliminated
by taking the (Sylvester or Bezout/Cayley) resultant $h(x)$ of $f$ and $g$ 
% Sylvester is typical method presented in Walker; see SAG84 for another description
% Bezout/Cayley is fh-gj=0 method: see SAG84
% Kronecker is multivariate resultant: see VW, p. 1 and Canny; uses indeterminates
with respect to $y$ (Sederberg et. al. 1984; van der Waerden 1953; Walker 1950).
Even if there are more than two equations, techniques from the theory of elimination
can be used to reduce the system to a single univariate equation, perhaps by using several
rounds of Sylvester resultants or a multivariate resultant.
Next, the univariate equation (which encodes the common roots) is solved, 
perhaps by Newton's method, yielding one coordinate of each solution of the original 
system.
For example, if $x_{0}$ is a root of h(x),
then there exists $y_{0}$ such that $f(x_{0},y_{0})=0=g(x_{0},y_{0})$.
Finally, the full solutions are built up from these partial solutions
by solving more univariate equations.
For example, f($x_{0}$,y) is solved, yielding $y_{1},\ldots,y_{k}$,
and $y_{i}$'s such that $g(x_{0},y_{i}) \neq 0$ are discarded.
The remaining $y_{j}$'s form full solutions or intersections $(x_{0},y_{j})$.
\end{example}

Notice that all of the intersections between the curves are found.
In particular, it is impossible to find only the intersections on a given segment of each curve 
with the SOE method.
(There is no way of predicting where the intersection will be until it is fully computed, and
the method deals with the equation of the entire curve.)
Segment intersection would require expensive postprocessing. 
(The intersections would be sorted along each curve and those that are not between the endpoints
of the appropriate segment would be discarded.)
Thus, with the SOE method, segment intersection is noticeably more complex than curve
intersection.

Another problem with the SOE method is that it requires the solution of a univariate equation
of high degree.
In particular, the degree of the resultant polynomial $h(x)$ is potentially
the product of the degree of $f(x,y)$ and the degree of $g(x,y)$.

\section{Coordinated crawling}
\label{sec-coor}

In this section we give a short introduction to crawling and a general overview
of coordinated crawling.
The reader is referred to Bajaj et. al. (1987), Hoffmann (1987), and Hoffmann and Lynch (1987) 
for the details of crawling.
% and to Sections~\ref{sec-simul} and \ref{sec-stair} for the details of the coordinated 
% crawling (CC) method.
Crawling is a method of traversing a curve.  
Progress is made by repeatedly making short steps away from the curve and
relaxing back onto the curve (Fig.~\ref{pic.crawl}).
There are various ways of stepping away from the curve, such as stepping
along the tangent or in a direction parallel to the axes.
The relaxation back onto the curve can be achieved with Newton's method.
One of the useful properties of crawling is its locality: it relies only on the
behavior of the curve in a restricted neighborhood of the current position.

\figg{pic.crawl}{Crawling along a curve}{2.5in}

The size of each step of a crawl can be adjusted.
It cannot be too large, since it might lose the curve, and it must be particularly
small near singularities and other places where confusion is likely.
However, within the bounds of these restrictions, it is possible to talk of 
coarse crawls with large steps and fine crawls with small steps.
We shall be intent upon keeping the crawl as coarse as possible,
since the larger the steps, the faster the crawl.

One of the contributions of this paper is to show that crawling can also be used to 
discover the intersections between two xy-monotone curve segments.
Let \arc{AB} and \arc{CD} be two xy-monotone curve segments.
Starting at the beginning of each segment, we shall crawl along the two segments in 
tandem, alternating the crawl along \arc{AB} with the crawl along \arc{CD} so that,
at any given time, progress is being made along only one of the segments 
(Fig.~\ref{fig.stair}).
The crawl along a segment continues until a {\em switching condition} becomes true.
The alternation between segments continues until an {\em end condition} becomes true, 
signalling that an intersection has been found or that the two segments do not intersect.
The segment along which one is presently crawling (resp., not crawling) is called the 
{\em active} (resp., {\em dormant}) segment.
A crawl along an active segment between switching conditions
will often be referred to simply as a crawl (of the coordinated crawl).

\figg{fig.stair}{(a-b) Two staircase crawls}{4.625in}

\begin{example}
\label{eg1}
Consider the coordinated crawl of Fig.~\ref{fig.stair}(a).
Horizontal and vertical lines have been added to the picture 
to reveal the structure of the crawls.
The first crawl is along \arc{Aa_{1}} of \arc{AB}.
\arc{AB} then becomes dormant and the second crawl is made along \arc{Cc_{1}} of \arc{CD},
and so on.
Eventually, the crawls get progressively smaller and converge to an intersection $x$.
If no intersection exists, then the coordinated crawl reaches the end of one 
of the segments (Fig.~\ref{fig.stair}(b)).
\end{example}

The coordinated crawl as we have presented it will only find the first intersection.
The second intersection is found by starting another coordinated crawl from the
first intersection.
A new coordinated crawl should be begun from each intersection 
until it is determined by the end condition that the segments do not intersect any further.

We shall present two variants of coordinated crawling, because there are two types
of xy-monotone segment.
A {\em rising} xy-monotone segment increases in $y$ as it increases in $x$, while
a {\em falling} xy-monotone segment decreases in $y$ as it increases in $x$.
The first method of coordinated crawling, which we call the {\em simultaneous crawl}, 
can be used to 
crawl along any pair of xy-monotone segments;
however it is best suited to crawling along one rising and one falling segment.
The second method, which we call the {\em staircase crawl}, will 
only apply to two rising or two falling 
segments.

We need to define some notation and assumptions.
Our notation for a curve segment will not only specify the endpoints,
but also the order of the endpoints, in the sense that it is assumed that
$x(A) \leq x(B)$ is always true of the segment \arc{AB}, where
$x(A)$ denotes the $x$-coordinate of the point $A$.
$P_{\mbox{\footnotesize{active}}}$ (resp., $P_{\mbox{\footnotesize{dormant}}}$)
is our notation for the present point on the active (resp., dormant) segment during 
a coordinated crawl. 
Finally, in the remainder of this paper, we assume that all curves are nonlinear, 
irreducible, plane algebraic curves.

\subsection{No Intersection Method is Exact}
\label{sub-exact}

It is usually impossible to be exact in geometric computations.
The reason for this is twofold: (i)
because of the use of numerical methods that converge rather
than compute exactly, and (ii) because of finite machine precision.
This is certainly the case with the system-of-equations method of intersection, 
which uses numerical methods such as Newton's method.
Coordinated crawling is no different, and so we make the following natural assumptions:
\begin{enumerate}
\item
	Two intersections that are less than $\epsilon$ distance apart are considered 
	to be the same, where $\epsilon > 0$ is very small and must be part of the input 
	of an intersection problem.
\item
	If the distance between the two segments decreases below $\epsilon$, 
	then we are free to decide that there is an intersection near this point.
        In other words, if there is an intersection, then we will always recognize it;
\label{ass2}
	but if there is no intersection, then we may sometimes make a mistake and 
	diagnose an intersection.
\end{enumerate}
$\epsilon$ may be chosen as small as desired without affecting the efficiency of the 
coordinated crawl (see Lemma~\ref{lem-contextswitch} below).

These two assumptions simplify the presentation of coordinated crawling considerably.
Moreover, for the purposes of applications such as graphics,
the two assumptions are valid not only because of inherent error in computations 
but because of the inherent crudeness of algorithms, e.g., two segments may as well 
intersect if they get closer than a pixel.
However, if necessary, it is possible to do without them.
For a coordinated crawl along a rising and a falling segment,
neither of the assumptions is necessary.
(That is, a near-intersection will never be mistaken for an intersection, 
and intersections will always be found within the accuracy 
	of the crawling method that is being used.)
%
%	The entire concept of \epsilon is unnecessary for simultaneous crawl:
%	there is only one intersection; and if the segments get within epsilon, then
% 	they always do intersect because they are going in opposing directions.
%
% 	No need for $\epsilon$-zone with simultaneous crawl: there is
% 	no problem with diagnosing another intersection immediately 
% 	afterward, since you abort the simultaneous crawl after the first intersection.
%

The first assumption can always be removed by the choice of a proper $\epsilon$.
For example, one can use Canny's `gap theorem' (Theorem~\ref{thm-gap}), which 
reveals that the intersections of two algebraic curves are never too close together.

\begin{lemma}[Canny's gap theorem (Canny 1987)]
% p.54
\label{thm-gap}
Let $\wp$(d,c) be the class of polynomials of degree d and coefficient magnitude c.
Let $f_{1}(x_{1},\ldots,x_{n}),\ldots,f_{n}(x_{1},\ldots,x_{n}) \in \wp(d,c)$
be a collection of n polynomials in n variables which has only 
finitely-many solutions when projectivized.
Then if $(\alpha_{1},\ldots,\alpha_{n})$ is a solution of the system,
then for any j either
$\alpha_{j} = 0 \mbox{\ \ \ \ or \ \ \ } \mid \alpha_{j} \mid > (3dc)^{-nd^{n}}$. \QED
\end{lemma}

\begin{corollary}
\label{cor-gap}
Let $f_{1}(x,y) = 0,\ f_{2}(x,y) = 0$ be two irreducible plane algebraic curves of degree 
$d$ and coefficient magnitude $c$.
If $\epsilon < (3dc)^{-2d^{2}}$, then no two intersections will 
be within an $\epsilon$-distance of each other.
%
% slight improvement: could use $\epsilon < \sqrt{2 (3dc)^{-2nd^{n}}}$, which 
% is $\sqrt{2\epsilon^{2}}$ (the hypotenuse of the triangle of sides $\epsilon$).
%
%  NOT WORTH IT: OBFUSCATES THE ARGUMENT AND DOESN'T DO MUCH TO IMPROVE EPSILON
\end{corollary}
\Heading{Proof:}
In order to apply the lemma, one of the two intersections must be translated to the origin.
\QED

The second assumption may be controversial, because the presence of an 
intersection, or lack thereof, might be important.
Therefore, in Section~\ref{sec.improve}, we show how to remove it by checking whether the segments 
do actually intersect in the neighbourhood in which the distance between the segments is very 
small.  % i.e., in the $\epsilon$-zone

\subsection{Coarse vs. Fine Crawls}

In coordinated crawling, we shall distinguish between coarse and fine crawls, depending on
the size of each crawl step.
A fine crawl will be used to find something accurately and to avoid skipping over an intersection.
Thus, fine crawl steps are of length less than $\epsilon$.
Coarse crawl steps are as long as possible without losing the curve.
Obviously, for reasons of efficiency, it is important that coarse crawls be used as much as 
possible.
The desired paradigm is to use coarse crawls to get close to the intersection and 
fine crawls only at the end to accurately find the intersection.
% In practice, fine crawls will also be needed for fine tuning of the coarse crawls.

For any pair of xy-monotone segments, we shall present a coordinated crawling method such 
that fine crawls are always short-lived and coarse crawls dominate.
Later sections (Section~\ref{sub-linecurve} and \ref{sub-linecurve2}) will investigate the use 
of even coarser traversals of the curve, where one skips over a large subsegment of the curve 
with a single bound (using a line-curve intersection).

\section{The simultaneous crawl}
\label{sec-simul}

In order to fully define a coordinated crawl, the switching and end conditions must be defined.
In this section, we present the first of our two coordinated crawling methods, the 
{\em simultaneous crawl}.
With the simultaneous crawl, one simulates crawling along both 
segments at the same time while maintaining the same velocity with respect to the $x$-axis,
hence its name (Fig.~\ref{fig.simul}(a)).
The associated switching condition is 
$ x(P_{\mbox{\footnotesize{active}}})\ >\ x(P_{\mbox{\footnotesize{dormant}}}) $.

\figg{fig.simul}{(a) a simultaneous crawl (b) at the end, it must back up finely}{3.7in}

The end condition must signal an intersection or the end of a segment.
In the neighbourhood of an intersection between a rising segment and a falling segment, 
the relative vertical order of the segments is reversed.
Therefore, a simultaneous crawl along a rising and falling segment can proceed with coarse steps 
until the relative vertical order of the segments is reversed, and then crawl backwards
with finer steps to accurately find the intersection 
(Fig.~\ref{fig.simul}(b)).
(The backwards
	crawl should continue until the relative vertical order switches once more, which 
	is where the intersection is placed.)
There is no danger of skipping over two intersections with the coarse crawl, because
a rising and a falling segment can have only one intersection.

Because of x-monotonicity, it is simple to recognize the end of a segment \arc{AB}: 
the condition is $x(P_{\mbox{\footnotesize{active}}}) \geq \mbox{min} \{x(B),x(D)\}$.
Thus, for a simultaneous crawl along a rising segment \arc{AB} and a falling segment \arc{CD},
the entire end condition is 
\[ ((y(A') < y(C')) \neq (y(A) < y(C))) \ \vee \ 
x(P_{\mbox{\footnotesize{active}}}) \geq \mbox{min} \{x(B),x(D)\} \]
where $A'$ (resp., $C'$) is the present point on \arc{AB} (resp., \arc{CD}) during the crawl.
The end condition that signals an intersection actually signals only the passing of an 
intersection: i.e., one must retrace steps back to the intersection before outputting it.

Theoretically, the simultaneous crawl can also be used for two rising (or two falling) segments.
A simultaneous crawl along two rising segments is dangerous, 
because it is possible to skip over a pair of intersections without noticing 
(Fig.~\ref{fig.counter2}).
By Lemma~\ref{cor-gap}, this danger could be avoided if the crawl is fine enough.
That is, the crawl steps must be finer than $2\epsilon$ where 
$\epsilon < \sqrt{2 (3dc)^{-2nd^{n}}}$.
(The switch and end conditions would be the same as above, except that there is no need to 
back up to the intersection after the end condition is signalled since steps are already fine.)
Because a simultaneous crawl along two rising segments would require fine steps at all times, 
the staircase crawl of the next section is preferred for these segments.

\figg{fig.counter2}{The danger of a simultaneous crawl along two rising segments}{2.8in}

\section{The staircase crawl}
\label{sec-stair}

In this section we present our second method of coordinated crawling, namely for crawling
along two rising segments.
(Two falling segments are dealt with analogously.)
Rather than switching segments as soon as the x-coordinate of the active segment exceeds
the x-coordinate of the dormant segment,
in the staircase crawl one waits until both the x-coordinate and the y-coordinate of the active
segment exceed those of the dormant segment before switching.
We call this a {\em staircase crawl} because if the 
endpoints of the crawls are joined by straight lines, a staircase leading towards an intersection
will result (Fig.~\ref{fig.stair} and Lemma~\ref{lem-converge}).
The associated switching condition is
\begin{eqnarray*}
	x(P_{\mbox{\footnotesize{active}}}) \  > \ 
	x(P_{\mbox{\footnotesize{dormant}}}) \ \ \wedge \ \ 
	y(P_{\mbox{\footnotesize{active}}}) \ > \ 
	y(P_{\mbox{\footnotesize{dormant}}})
\end{eqnarray*}
Because of the granularity of the crawl, an intersection might be overlooked with this switch
condition (Fig.~\ref{fig.skip}).
To correct this, before switching segments one should back up one step so that the active 
segment remains behind the staircase.
(This is similar to backing up in the simultaneous crawl to find the intersection accurately.)

\figg{fig.skip}{The danger of a staircase crawl that does not back up one step
		after each crawl}{3.2in}

An intersection is signalled when a stair of height less than $\epsilon$ is 
encountered.
(This is the only place that the second assumption about $\epsilon$ from Section~\ref{sub-exact}
is used.)
\Comment{
NOT WORTH INCLUDING
This is slightly stronger than signalling an intersection as soon as the distance between
the segments becomes less than $\epsilon$ (Fig.~\ref{fig.epsilon.stair}).
% MAYBE SHOW PICTURE OF SEGMENTS APPROACHING WITHIN EPSILON WITHOUT AN INTERSECTION 
% (ERRONEOUSLY) BEING SIGNALLED
}
Every intersection will be found with this end condition, because the staircase converges
to an intersection (see Lemma~\ref{lem-converge}) and we do not jump past the staircase.
Thus, the end condition for a staircase crawl is simply:
\[ \mid y(P_{\mbox{\footnotesize{active}}}) - y(P_{\mbox{\footnotesize{dormant}}}) \mid 
	< \epsilon   \ \ \vee \ \ 
   x(P_{\mbox{\footnotesize{active}}}) > \mbox{min} \{x(B),x(D)\}
\]

Coarse crawls are used until the coordinated crawl approaches near an intersection.
At this point, each crawl then becomes very short and fine crawls should be used.
If a coarse crawl is used near an intersection, it is possible to enter an infinite loop:
continually going forward one step (at which point both x and y coordinates of the active segment
exceed those of the dormant) and then back one step (to stay behind the staircase).  
With a fine crawl, however, this will not happen, because if only one fine step (of length 
less than 
$\epsilon$) separates the x and y-coordinates, the height of the present stair must be less 
than $\epsilon$ and an intersection will be signalled.
Since an infinite loop is only possible when the length of a stair of the staircase becomes 
as short as a step of a coarse crawl, fine crawls are only necessary near the 
intersection.
(In an intermediate phase when strictly coarse crawls are too crude 
but strictly fine crawls are too slow, one might use a coarse crawl to crawl forward 
and a fine crawl to back up.)

We must show that the staircase crawl converges to the first intersection of the segments,
if one exists.

\begin{lemma}
\label{lem-converge}
Let \arc{AB} and \arc{CD} be two rising segments.
The staircase crawl along \arc{AB} and \arc{CD} will converge
to the first intersection of \arc{AB} and \arc{CD}, if such an 
intersection exists.
Otherwise, it will reach the endpoint (B or D) of one of the segments.
\end{lemma}
\Heading{Proof:}
Firstly, one does not jump over an intersection.
This is best seen by considering the perfect staircase: the staircase consisting of true
horizontal and vertical line segments. % starting at the beginning of one curve.
(The stairs of the staircase connecting the endpoints of the actual staircase crawl would not be
perfectly horizontal or vertical.)
It is easy to see that the perfect staircase converges to an intersection.
The staircase crawl is guided by the perfect staircase.
Because the staircase crawl backs up before switching segments, it is indeed constrained by 
the perfect staircase.

Secondly, progress is made with each crawl. 
If there is no progress, then the stair must
be of height less than $\epsilon$ and we say that an intersection has been found.
In particular, progress of at least $\epsilon$ (usually much more) is made with each crawl.
Thus, the staircase crawl must eventually find the first intersection, if one exists.
\QED

A staircase crawl diagnoses an intersection when the stairs become shorter than $\epsilon$.
Two questions arise: where should the intersection be placed  and
where should the crawl start over to look for the next intersection?
The crawl cannot place the intersection where it stopped and continue
from there, because it will immediately stop and diagnose another intersection.
We introduce the concept of an $\epsilon$-zone to provide the answer to these questions.
The {\em $\epsilon$-zone} is a pair of subsegments of the curves that stay within a 
vertical distance of $\epsilon$.
We enter an $\epsilon$-zone when the vertical distance between the segments becomes
less than $\epsilon$ (in practice, the $\epsilon$-zone actually begins when a 
	stair of height $< \epsilon$ is found)
and exit it when the vertical distance becomes greater than $\epsilon$
(Fig.~\ref{eps.zone}).

\figg{eps.zone}{$\epsilon$-zones}{3in}

Now the two questions can be answered.
When a stair of a staircase crawl becomes shorter than $\epsilon$, we 
skip over the associated $\epsilon$-zone, and restart the staircase crawl there.
The intersection is placed in the middle of the $\epsilon$-zone (or it can be found exactly
with the method outlined in Section~\ref{sub-exact2}).\Comment{\footnote{This requires an
	addition to our assumptions about $\epsilon$: if there are many intersections
	that lie in the same $\epsilon$-zone (not necessarily $\epsilon$ apart but such that 
	the segments do not separate by more than $\epsilon$ between the intersections),
	we assume that they can be replaced by a single intersection in the middle of the zone.
	Once again, this is a valid assumption for graphical purposes.}}
A simultaneous crawl is used to cross the $\epsilon$-zone.
(As mentioned at the end of Section~\ref{sec-simul}, this will use fine crawls so 
	it is perfect for crossing the $\epsilon$-zone.)

The first crawl after an intersection must be treated as a special case, because neither 
segment dominates.
In order to get things started, one should make an $\epsilon$-crawl along one of the segments.
It might seem that this crawl is dangerous because it may skip over an intersection,
being blind and unconstrained by any staircase.
However, recall that any two intersections that are within $\epsilon$ of each other
are considered equivalent;  thus, it is impossible to skip over a relevant intersection.

We end this section and our discussion of methods for coordinated crawling by noting
that the staircase crawl cannot be used to find the intersection of a rising and a 
falling segment.
This is shown by Fig.~\ref{countereg}.
Since we have already noted that the simultaneous crawl is not well suited for the intersection
of two rising segments (because fine crawls are always necessary),
one can see that both types of coordinated crawling are necessary.
With the ability to do a simultaneous crawl and a staircase crawl, one can find the intersection(s)
of any pair of xy-monotone segments by crawling, and moreover most of the crawling uses coarse 
steps.

\figg{countereg}{A staircase crawl cannot be used for a rising and a falling segment}{3in}

\section{Efficiency}

A simultaneous crawl is continually switching from one segment to the other.
Indeed, of the two segments between any two adjacent vertical lines
in Fig.~\ref{fig.simul}(a), one will be a single crawl step long.
It might appear that this large number of switches will make the crawl expensive.
The following lemma shows that this is not the case, because switches are essentially free.

\begin{lemma}
\label{lem-contextswitch}
The number of switches in a coordinated crawl is irrelevant.
\end{lemma}
\Heading{Proof:}
A coordinated crawl along two segments is slightly less efficient than
crawling independently along the entire first segment and then crawling 
independently along the entire second segment, because one must test
the switching and end conditions at each step.
However, the number of switches does not matter because 
stopping and starting a crawl takes no time.
This can be seen as follows.
Let \arc{AB} and \arc{CD} be xy-monotone segments.
We can keep two separate regions in memory, one set up for crawling
along \arc{AB}, the other for crawling along \arc{CD}.
Switching crawls merely involves jumping to the other part of memory. 
(In Section~\ref{sec.improve} we shall show that a coordinated crawl need not crawl along 
all of the two segments,
so that the complexity of a coordinated crawl is actually less than the complexity of making two 
`independent crawls with condition testing'.)
\QED

It might seem that if the segments remain very close for a long time, then a 
staircase crawl will be slow because the staircase is very fine with very short stairs. 
Similarly, it might appear that a staircase crawl will slow down as its stairs 
get very short during convergence to an intersection.
Lemma~\ref{lem-contextswitch} shows that these intuitions are also wrong.

\section{xy-monotone decomposition}
\label{sec-decomp}

Since our coordinated crawling methods work upon xy-monotone segments,
the first step in intersecting two segments with the coordinated crawling method is to
partition each curve segment into xy-monotone curve segments.
Observe that a curve segment is xy-monotone if and only if it contains 
no local extrema (no changes in direction with respect to the x-axis or y-axis).
An xy-monotone decomposition of a segment can be computed by crawling along the segment.
One simply marks points at which x(P) or y(P) changes direction,
where P = (x(P),y(P)) is the present point on the crawl.
The nature of crawling guarantees that one will not miss any direction changes during a crawl.
Therefore, the complexity of the xy-monotone decomposition of a segment is the
complexity of crawling along the segment.

An alternative method is to compute the local extrema of the segment algebraically,
using the fact that the local extrema of a curve $f(x,y)=0$ are the solutions of
\mbox{$\{f_{x}=0,f=0\}$} and \mbox{$\{f_{y}=0,f=0\}$}.
Indeed, since it is trivial to compute the local extrema of a curve as part of computing
the singularities of a curve (since the singularities are the solutions of 
\mbox{$\{f_{x}=0,f_{y}=0,f=0\}$}) and the singularities of an algebraic curve are fundamental to
many geometric modeling algorithms (e.g., Abhyankar and Bajaj 1986; Johnstone 1987),
the local extrema may already be known.
If not, one can use the SOE method to compute the local extrema via 
$\{f_{x}=0,f=0\}$ and $\{f_{y}=0,f=0\}$.
This may seem to lead to a circularity in our method.
However, the key is to consider the xy-monotone decomposition as a one-time overhead.
That is, the amortization of a total of two curve-curve intersections over many intersections 
(when coordinated crawling is used) is
an improvement over the need for a curve-curve intersection for each intersection.
The expense of preprocessing is well warranted for curves in a solid model,
because they are relatively permanent and intersection is a common operation.
After computing the extrema, they must be sorted along the curve in order to pair the extrema
into xy-monotone segment endpoints.
%
% NOT IMPORTANT BECAUSE WE ARE TALKING ABOUT A SEGMENT OF THE CURVE, NOT THE ENTIRE CURVE,
% SO N^{2} IS MISLEADING.
% A curve of degree $n$ will have at most $O(n^{2})$ xy-monotone segments, since there are
% $O(n^{2})$ local extrema.

Although we mention the algebraic method for xy-monotone decomposition for completeness,
xy-monotone decomposition by crawling is probably the best choice, since we are dealing
with segments.

\Comment{
IGNORE CONVEX DECOMPOSITION: ONLY A MINOR IMPROVEMENT NOT WORTH THE CONFUSION
\section{Convex xy-monotone decomposition}

The decomposition of a curve into convex 
segments (for Section~\ref{sec-convabort}) is described in \cite{johnstone87}.
A convex decomposition would precede an xy-monotone decomposition.

TECH REPORT ONLY
(a) if you need convex decomposition to be able to sort, 
then xy-monotone-and-convex decomposition is no harder than xy-monotone decomposition.

(b) if sorting can be done without convex decomposition,
then xy-monotone decomposition should be quite a bit easier
than xy-monotone+convex decomposition, since
finding extrema is roughly equivalent to finding singularities rather 
than (i) finding flexes and singularities (ii) finding their tangents
(iii) intersecting these tangents with the curve, and
(iv) pairing up the endpoints of the convex segments.
END OF ``TECH REPT ONLY''

In Section~\ref{sec.improve}, it shall be made clear that there are many advantages to
decomposing the segments into convex xy-monotone subsegments rather than xy-monotone 
subsegments.
E.G.: TWO CONVEX SEGMENTS CAN ONLY INTERSECT TWICE, SO CAN ABORT EARLY.

A convex xy-monotone decomposition of a segment can be done by crawling along the segment.
One simply marks points at which x(P), y(P), or tangent(P) changes direction,
where P = (x(P),y(P)) is the present point on the crawl and tangent(P) is the tangent
to the segment at P (Fig.~\ref{fig.decompose}).
The tangent of the plane algebraic curve $f(x,y,w)=0$ (the homogenization of $f(x,y)=0$)
at a nonsingular point P is $f_{x}(P)x + f_{y}(P)y + f_{w}(P)w = 0$ \cite{W50}.
We can simply ignore the value of tangent(P) at singularities.
Therefore, the complexity of convex xy-monotone decomposition of a segment is the
complexity of crawling along the segment.

It should be noted that convex decomposition of a segment is noticeably easier
when we combine it with an xy-monotone decomposition.
In particular, the marking of points where the direction of tangent(P) changes is not
sufficient for convex decomposition, because of spiral curves and singularities
(Fig.~\ref{fig.exceptions}); but it works fine when combined with the marking of points
where x(P) or y(P) change direction.\footnote{The reason that it is enough to look for 
	tangent(P) direction changes when also marking x(P) and y(P) changes is that 
	the tangent to a curve can change by at most $\pi/2$ on an xy-monotone segment.}

Note that crawling along an xy-monotone segment should be
simpler than crawling along a generic segment, since the crawl,
knowing that it should travel up and right (say), is not so easily confused
(e.g., as it passes through a singularity).?
}

\section{Improvements}
\label{sec.improve}

\subsection{Aborting Early}

In this section we outline some methods for improving the efficiency of a 
coordinated crawl.
A coordinated crawl can be aborted as soon as it becomes apparent that the two 
segments cannot intersect.
We begin with a set of conditions that guarantee the distinctness of two 
xy-monotone segments.

\begin{lemma}
\label{lem-badconds}
Let \arc{AB}\ and \arc{CD}\ be xy-monotone segments.
If any of the following conditions is true, then \arc{AB}\ and \arc{CD}\ do not
intersect.  
\begin{enumerate}
	\item $x(B) < x(C)$
	\item $x(D) < x(A)$
	\item $\{y(A),y(B)\} < \{y(C),y(D)\}$
	\item $\{y(C),y(D)\} < \{y(A),y(B)\}$
	\item \arc{AB} and \arc{CD} are convex (a segment is {\em convex} if
			no line has more than two distinct intersections with it)
	    	and $\Delta AB \cap \Delta CD = \emptyset$ (Fig.~\ref{fig.delta}).
	$\Delta AB$ is the triangle whose sides are the tangent at 
	A, the tangent at B, and \seg{AB}.
\Comment{
LEAVE THIS MINOR DETAIL TO THE READER
 	(Since the tangent is not well defined at 
 	a singularity, care must be taken if A or B is singular.
 	In this case, it suffices to crawl an $\epsilon$ distance from A or B 
	before creating $\Delta AB$.)
}
\end{enumerate}
\end{lemma}
%
\Heading{Proof:} 
The sufficiency of conditions (1)-(4) is a straightforward consequence of the 
xy-monotonicity of the segments.
(Recall that $x(A) \leq x(B)$ is implicit 
	from the notation \arc{AB}.)
The sufficiency of the fifth condition follows from noting that
if \arc{AB} is convex, then $\Delta AB$ contains \arc{AB}.
\QED

\figg{fig.delta}{\arc{AB} and \arc{CD} cannot intersect}{3.25in}

These conditions should be tested throughout the coordinated crawl.
(Of course, the condition involving convex segments would only be tested if we had
prior knowledge that the segments were convex as well as xy-monotone.)
In order to lighten the computational burden, they might only be tested intermittently, 
rather than after every step.

\Comment{
NOT WORTH IT
DOESN'T GIVE NEW INSIGHT (SHOULD BE OBVIOUS ONCE YOU WORK WITH AN EXAMPLE FOR A WHILE)
MAY NOT BE WORTHWHILE (SEE PARAGRAPH AFTER LEMMA)

HOW TO ENTIRELY AVOID A SEGMENT-SEGMENT INTERSECTION
\begin{lemma}
Let \arc{AB}\ and \arc{CD}\ be xy-monotone segments such that \arc{AB} is 
rising and \arc{CD} is falling,
and \arc{AB} and \arc{CD}  do not satisfy any of the
conditions of Lemma~\ref{lem-badconds}.
Let $\arc{A_{0}B_{0}} \subseteq \arc{AB}$ and
$\arc{C_{0}D_{0}} \subseteq \arc{CD}$  be subsegments such that 
  $x(A_{0}) = x(C_{0})$ and 
  $x(B_{0}) = x(D_{0})$.
(These subsegments always exist.)
Then \arc{AB} and \arc{CD} intersect if and only if
$\arc{A_{0}B_{0}}$ and $\arc{C_{0}D_{0}}$ do not satisfy any of the conditions
of Lemma~\ref{lem-badconds}.
NEED A PICTURE
\end{lemma}
\Heading{Proof:}
Since \arc{AB} and \arc{CD} do not satisfy any of the conditions of 
Lemma~\ref{lem-badconds}, $x(A) \leq x(D)$ and $x(C) \leq x(B)$.
Hence, by our naming assumption (*), $\{x(A),x(C)\} \leq \{x(B),x(D)\}$.
In particular, 
\begin{eqnarray*}
x(A) \leq \mbox{max} \{x(A),x(C)\} \leq \mbox{min} \{x(B),x(D)\} \leq x(B)
\end{eqnarray*}
Let $A_{0}$ (resp., $B_{0}$) be the point of \arc{AB} with abscissa
\mbox{max \{x(A),x(C)\}} (resp., \mbox{min \{x(B),x(D)\}}).
Thus, $\arc{A_{0}B_{0}} \subseteq \arc{AB}$.
(See Fig.~\ref{fig1onsheet}.)
Similarly,
\begin{eqnarray*}
x(C) \leq \mbox{max} \{x(A),x(C)\} \leq \mbox{min} \{x(B),x(D)\} \leq x(D)
\end{eqnarray*}
Let $C_{0}$ (resp., $D_{0}$) be the point of \arc{CD} with abscissa
\mbox{max \{x(A),x(C)\}} (resp., \mbox{min \{x(B),x(D)\}}).
$\arc{C_{0}D_{0}} \subseteq \arc{CD}$.
Note that $x(A_{0}) = x(C_{0})$ and $x(B_{0}) = x(D_{0})$.

Suppose that $\arc{A_{0}B_{0}}$ and $\arc{C_{0}D_{0}}$ do not satisfy any of the 
conditions of Lemma~\ref{lem-badconds}.
Then it is clear that $\arc{A_{0}B_{0}}$ must intersect
$\arc{C_{0}D_{0}}$ (Fig.).
%
%We shall show that $\arc{A_{0}B_{0}}$ intersects $\arc{C_{0}D_{0}}$.
%Assume w.l.o.g. that $A_{0} \neq C_{0}$ and $B_{0} \neq D_{0}$.
%Consider the curve
%\begin{eqnarray*}
%\alpha := \arc{A_{0}B_{0}} \cup \{(t,y(A_{0})) : -\infty < t \leq x(a_{0})\}
%			   \cup \{(t,y(B_{0})) :  x(B_{0}) \leq t < \infty \}
%\end{eqnarray*}
%\figg{2}{The curve $\alpha$}{1.5in}
%$\alpha$ partitions the plane into two regions.\footnote{After all,
%	it is continuous, non-self-intersecting, and it goes off to infinity
%	at both ends.}
%
%I claim that $C_{0}$ and $D_{0}$ lie on opposite sides of $\alpha$.
%Consider $C_{0}$.
%If $y(C_{0}) < y(A_{0})$, then 
%$y(D_{0}) \leq y(C_{0}) < y(A_{0}) \leq y(B_{0})$,
%satisfying condition (3) of Lemma~\ref{lem-badconds}, which is a contradiction.
%Moreover, we have assumed $A_{0} \neq C_{0}$.
%Therefore, $y(C_{0}) > y(A_{0})$.
%Since $x(C_{0}) = x(A_{0})$, this establishes that $C_{0}$ lies in the
%upper region, above $\alpha$.
%A similar argument establishes that $y(D_{0}) < y(B_{0})$ and that $D_{0}$
%lies in the lower region, below $\alpha$.
%Therefore, $C_{0}$ and $D_{0}$ lie on opposite sides of the extended curve
%$\alpha$, and the continuous curve segment $\arc{C_{0}D_{0}}$ must cross
%$\alpha$.
%Since $x(\arc{C_{0}D_{0}}) = x(\arc{A_{0}B_{0}})$, 
%$\arc{C_{0}D_{0}}$ must cross $\alpha$ through $\arc{A_{0}B_{0}}$.
%This establishes that \arc{AB} has at least one intersection with \arc{CD}.\\

Suppose that $\arc{A_{0}B_{0}}$ and $\arc{C_{0}D_{0}}$ satisfy one or more of the 
conditions of Lemma~\ref{lem-badconds}.
Then $\arc{A_{0}B_{0}} \cap \arc{C_{0}D_{0}} = \emptyset$.
I claim that the other parts of \arc{AB} and \arc{CD} also do not intersect.
SHOW!
\QED

$\arc{A_{0}B_{0}}$ and $\arc{C_{0}D_{0}}$ are found by crawling from the
beginning and then backwards from the end of the segments (Fig.~\ref{fig.A0B0}).
They might allow you to recognize non-intersection more quickly, 
but it is not clear that it would be wise to invest the effort.
The crawl from A to A0 (or B to B0) is  done anyway to satisfy 
the initial condition of the coordinated crawl.
Perhaps  the effort to crawl backwards from D to d0 (or B to B0)
should be invested iff x(B) - x(D) is small (at least relative to
x(B) - x(A0)).
}

\subsection{Eliminating Long Crawls in the Staircase Crawl}
\label{sub-linecurve}

We have noted that a coordinated crawl should use coarse crawls whenever possible.
In this section, we show how to make even larger jumps in a staircase crawl.
In particular, it is possible to replace a crawl (from one stair endpoint to the next)
by a single line-curve intersection.

In a staircase crawl, one climbs a staircase towards an intersection.
This process can be fully characterized by the series of endpoints of the stairs.
For example, the staircase crawl of Fig.~\ref{fig.stair} can be represented by 
$C,\ a_{1},\ c_{1},\ a_{2},\ c_{2}, \ldots,\ x$.
The act of climbing a stair (i.e., finding the next endpoint in the series)
is equivalent to finding the intersection of a line with one of the curve segments.
In particular, the endpoint that follows endpoint E on curve segment 1 is the intersection 
with curve segment 2 of a horizontal or vertical line through E.
This suggests another method for climbing the stair: use the system-of-equations (SOE)
method to find the intersection of the line and curve segment.

Since a staircase crawl is being used, supposedly the SOE method is not practical 
for the intersection of the two curve segments.
However, it may be feasible for the simpler intersection of a line and one of the curve segments.
The time to climb a stair $E_{i}E_{i+1}$ of the staircase $E_{1},E_{2},\ldots,E_{n}$
by crawling depends upon the length of the segment \arc{E_{i-1}E_{i+1}}, 
whereas the time to climb a stair with a line-curve intersection depends on the degree 
of the curve that we are climbing to.
Therefore, the taller the stair and the lower the degree of the curve,
the more attractive it is to climb the stair with a line-curve intersection.

It may be difficult to decide when the next stair should be climbed with a line-curve intersection
rather than a crawl.
\Comment{
TOO AD-HOCish TO INCLUDE; NOT VERY GOOD IDEAS EITHER
One method for deciding would be to climb the first stair (which is probably far from an 
intersection and tall) with a line-curve intersection and record the time T spent in doing 
so.\footnote{Alternatively, T could be approximated from line-curve-intersection climbs on other 
	curves of the same degree.}
Future stairs could be climbed with a crawl, but the climb would be aborted and replaced by 
a line-curve intersection if it took time longer than T.
Alternatively, one could try to predict the cost of climbing the present stair with a crawl (say C)
and use a crawl only if C is less than T.
C could be approximated from the cost of climbing the previous stair with a crawl, under 
the assumption that adjacent stairs tend to be of the same length.
If the previous stair was not climbed by a crawl, then
the following lemma could be used to approximate the cost of climbing it with a crawl.
}
The following lemma could be used to approximate the cost of climbing it with a crawl.

\begin{lemma}
\label{lem-bound}
The length of an xy-monotone segment \arc{AB} is bounded by $\sqrt{2}$dist(A,B).
\end{lemma}
\Heading{Proof:}
Consider the right triangle with hypotenuse \seg{AB}, whose other sides are horizontal
and vertical (Fig.~\ref{fig.triangle}).
	% notate $\theta$ as the angle between the hypotenuse and horizontal side,
	% notate horiz side = a, vertical side = b, hypotenuse = c
By xy-monotonicity, it is easy to see that the length of \arc{AB} is bounded by
two of the sides of the triangle, $\mid x(B) - x(A) \mid + \mid y(B) - y(A) \mid = a+b$.
The result follows by noticing that $a+b = c(\mbox{sin}\theta + \mbox{cos}\theta)$
and $\mbox{max}_{\theta} (\mbox{sin}\theta + \mbox{cos}\theta) = \sqrt{2}$ 
(at $\theta = \frac{\pi}{4}$).
\QED

\figg{fig.triangle}{The length of \arc{AB} is bounded by $a+b$}{2.4in}

\subsection{Speeding Up the Simultaneous Crawl}
\label{sub-linecurve2}

The simultaneous crawl can also benefit from the use of line-curve intersections to replace 
crawls.
In this case, the analogy is to root-finding of univariate polynomials, where binary search 
is used to isolate a region for the root before Newton's method is applied.
In finding the unique intersection of a rising and a falling segment, it may be useful to 
use a binary search for the intersection with line-curve intersections
before beginning the actual simultaneous crawl (Fig.~\ref{fig.binary}).
(A binary 
	search is not possible for the intersection of two rising segments, since
	there may be more than one intersection and a binary search only makes sense when
	searching for a single element.)
A probe of this `binary search' is the intersection of a vertical line with both curve 
segments to determine their relative vertical order.  
If the relative vertical order is the same as the beginning of the segments, 
then the intersection must lie to the right of the probe.

\figg{fig.binary}{A binary search for the intersection}{3.4in}

The binary search may allow the simultaneous crawl to begin closer to the intersection.
It is difficult to determine the number of probes that should be made.
There is a tradeoff between the number of line-curve intersections
that are computed and the amount of crawling that is saved.
As a general rule, the binary search would continue longer if the degree
of the curves is low (since line-curve intersections will be cheap) or
if the curve segments are long (since one can anticipate a lot of crawling).

\subsection{Finding an Intersection Exactly}
\label{sub-exact2}

In the staircase crawl, we only approximated the intersection by placing it in the middle
of an $\epsilon$-zone.
Moreover, we assumed that there actually was an intersection in the $\epsilon$-zone, although
there might not have been one.
This was justified by the second assumption of Section~\ref{sub-exact}.
However, it is possible to discover whether or not there is an intersection and, if there is,
to find it more exactly.
The idea is to use two Newton's method searches for the intersection, restricting the search
to the $\epsilon$-zone of the potential intersection.

Suppose that we are intersecting the xy-monotone segments \arc{AB} and \arc{CD} on curves
$f(x,y)$ and $g(x,y)$, respectively.
Suppose that an $\epsilon$-zone begins at $p \in \arc{AB}$ and ends at $q \in \arc{AB}$.
We shall restrict our search to the rectangle R bounded by $x = x(p),\ x = x(q),\ y = y(p)$,
and $y = y(q)$.
Let $h(x)$ be the resultant of $f$ and $g$ with respect to $x$ (see Section~\ref{sec-soe}),
so $h(x_{0}) = 0$ if and only if $f$ and $g$ have an intersection $(x_{0},y_{0})$.
First, perform a search for a root $x_{0}$ of $h(x)$, using Newton's method starting at
$\frac{x(p)+x(q)}{2}$ and abandoning the search if it leaves [x(p),x(q)].
Second, perform a search for a root $y_{0}$ of $f(x_{0},y)$, using Newton's method starting at
$\frac{y(p)+y(q)}{2}$ and abandoning the search if it leaves [y(p),y(q)].
$(x_{0},y_{0})$ is an intersection in the $\epsilon$-zone.
If no $(x_{0},y_{0})$ is found, then report that the segments do not intersect.
Note that both searches take very little time, since we are restricting them to a very
small region.

\section{Intersecting arbitrary curve segments}
\label{sec-sweep}

We have discussed how to intersect two xy-monotone segments.
However, the original goal was to intersect two arbitrary segments.
Since the two original segments were decomposed into xy-monotone segments, we must show
how to find the intersections of a collection of xy-monotone segments.
Rather than using the naive $O(n^{2})$ algorithm of intersecting every pair, 
we shall use a variant of plane sweep to reduce this to looking at only $O(n + k)$ pairs,
where $k$ is the number of intersections.
The advantage of this plane sweep method (based on the familiar plane sweep of Bentley 
and Ottmann 1979) is that it avoids testing pairs that are never vertically adjacent.

We begin by inserting all of the xy-monotone segment endpoints into a priority queue E
(sorted by $x$-coordinate).
We will be sweeping a vertical line L through the plane from left to right.
As we sweep we will maintain a data base D, which consists of all curve
segments that intersect L, stored in sorted order by their intersections with L.
We represent D as a (2,3)-tree (Aho et. al. 1974) (or some equivalent efficient dynamic
search structure).
Note that since the segments are xy-monotone, each segment will intersect L at most
once.  As we sweep L to the right we need to stop at various {\em event} points to 
maintain the consistency of the data base D.
The priority queue E determines the events.
An event is either an endpoint or an intersection point.
With each curve C we also keep a priority queue E(C), which stores the names of 
all the curves that we have compared with C already.
These lists will prevent us from performing any redundant intersection tests.

A generic step in the plane-sweep algorithm is as follows.
Remove the point in E with minimum $x$-coordinate.
Let $p$ be this point.
Intuitively, this corresponds to moving L to the right until it hits $p$.
We must then update D depending on the identity of $p$.
We identify each of the possible cases below.

{\bf Case 1}. The point $p$ is the left endpoint of a curve segment $C$ 
(Fig.~\ref{fig.pleft}).
In order to maintain the consistency of our data base, we must insert $C$ in $D$.
To do this we must find the curve segment $C_{1}$ in $D$ such that $C_{1}$
intersects $L$ in the highest point below $p$, i.e., $C_{1}$ is directly below $p$.
We can do this by making $O(\log n)$ curve comparisons to find
a path in the tree $D$ from the root to the place where $C$ belongs.
Each such curve comparison determines if a curve $C'$ intersects $L$ above or below $p$,
and is implemented by crawling along $C'$ from the previous event point on $C'$ until you 
reach L, as in Fig.~\ref{fig.pleft} (or by performing a line-curve intersection).
After we have located where $C$ belongs in $D$, suppose that $C_{1}$ (resp., $C_{2}$)
is C's predecessor (resp., successor) curve in $D$.
We check if $C_{1}$ is already in $E(C)$ and, if not, intersect $C$ with $C_{1}$
(using coordinated crawling).
Similarly, we check if $C_{2}$ is already in $E(C)$ and, if not, intersect $C$ with $C_{2}$.
We add all discovered intersection points $p$ to the priority queue $E$ as long 
as the two curves cross at $p$ (as opposed to simply `touching').
We also add $C$ to $E(C_{1})$ and $E(C_{2})$ and add $C_{1}$ and $C_{2}$ to $E(C)$.
At first glance, one might worry that the crawling involved in the curve 
	comparisons might become prohibitive.  However, even in the pathological worst case,
	the entire time required for inserting C into D is bounded by the time to
	crawl along $\log n$ segments, which is not a significant expense when compared to
	the alternative of finding the intersections between all $O(n^{2})$ segment-pairs.

\figg{fig.pleft}{A curve comparison between $C'$ and p: crawl from q to r}{3.4in}

\Comment{ already dealt with in improvements section
Before intersecting any two xy-monotone segments (such as C and $C_{1}$ above),
one should check whether it is impossible for C and $C_{1}$ to intersect: i.e., if the 
bounding rectangle of C (the rectangle with sides x=x(P), x=x(Q), y=y(P), y=y(Q), where 
C = \arc{PQ}) does not intersect 
the bounding rectangle of $C_{1}$, then there is no need to perform the intersection.
Moreover, 
}
The coordinated crawl for intersecting two xy-monotone segments should be started from L, not 
from the beginning of the segments.

{\bf Case 2}. The point $p$ is an intersection point.
If $C_{1}$ and $C_{2}$ are the two curves that intersect at $p$, then we swap them
in $D$.  Without loss of generality, assume $C_{2}$ now occurs before $C_{1}$ in the
list $D$.  Let $C_{0}$ be the new predecessor of $C_{2}$ and let $C_{3}$ be the 
new successor of $C_{1}$.
Provided $C_{0}$ is not in $E(C_{2})$, we find the intersections of $C_{0}$ and $C_{2}$
(and insert them into the event queue E).
Similarly, we intersect $C_{1}$ and $C_{3}$, provided $C_{3}$ is not in $E(C_{1})$. 
We then update $E(C_{0})$, $E(C_{1})$, $E(C_{2})$, and $E(C_{3})$ as necessary.

{\bf Case 3}.  The point $p$ is a right endpoint of a curve $C$.
In this case, we delete $C$ from $D$.
We then need to intersect the two neighbours $C_{1}$ and $C_{2}$ of $C$ at $p$
(which are now adjacent), after checking if $C_{1}$ is already in $E(C_{2})$.
Of course, we then update $E(C_{1})$ and  $E(C_{2})$ as necessary.

Since these are all the possible cases, this completes the algorithm.
We summarize with the following theorem:
\clearpage
\begin{theorem}
Given n xy-monotone curve segments in the plane, one can compute all of their 
intersection points with $O(n+k)$ segment-segment intersections, 
where $k$ is the number of intersection points.
\end{theorem}
\Heading{Proof:}
The only time that segment-segment intersections are made is when inserting or deleting an
event from the event queue E.
Each (of $2n$) segment endpoints and each (of $k$ intersections) are inserted and deleted from E,
and at most two segment-segment intersections are performed with each insertion or deletion.
\QED

The benefits of this algorithm will be most strongly felt when the segments S are of 
a different order of complexity from the curves $C \supset S$.
Note that the same plane-sweep algorithm can be applied to the intersection of any 
collection of algebraic curve segments, just as well as two algebraic curve segments.

%\section{Implementation Results}
%
% (1) get crawling software from Purdue/Cornell by asking Bajaj or Hopcroft,
%     to use both in the present coordinated crawling work and 
%     in comparing crawling more rigorously with sorting methods
%
%			!!!!!!!!!!!!!!!!!!!!!!!!!!!!
%
% (2) get an account on VMS for MACSYMA access so that we can time the
%     elimination method of curve intersection
%

\Comment{
DETRIMENTAL TO OUR ARGUMENT (I.E., LIKE SAYING `DID WE NEED TO DO WHAT WE DID?')

\section{Extensions}

A natural question to ask about the coordinated crawl is whether the preprocessing 
step of xy-monotone decomposition is necessary.
That is, is coordinated crawling possible along ordinary segments?
It is certainly possible to extend the coordinated crawl technique to
segments that are monotone in any two linearly independent directions.
However, it is not clear whether it is possible to extend to non-monotone segments.
A coordinated crawl needs a simple switching condition, since this condition
must be tested repeatedly during a subcrawl (viz., after each crawl step).
With monotone segments, switching only depends upon the relative size of the
x and y coordinates, which is particularly easy to check.
}

\section{Conclusions}

By extending the technique of crawling along one segment to a technique of 
coordinated crawling along two segments,
we have introduced a new method for intersecting plane algebraic curve segments.
It takes advantage of the locality and simplicity of the segments, unlike the
system-of-equations method of intersection.
Rather than first finding all of the intersections between the curves (which 
seems unnatural), our method directly finds only the intersections between the segments.

The coordinated crawling method can be especially useful when combinatorial explosion
of degree sets in with the SOE method.
That is, among other things, the conventional SOE method entails the solution of a 
univariate equation (the resultant of the two curves) whose degree is the product 
of the degrees of the curves, and the solution of an equation of high degree  
soon becomes prohibitive.
The coordinated crawling method only involves the evaluation of equations whose
degree is the degree of the curves (for crawling), which is simpler for two reasons:
because of the lower degree of the equation and because evaluation
is easier than solution.
Therefore, the coordinated crawling method may be the only feasible method when the
degree of the curves becomes large.

\section*{Acknowledgements}
We would like to thank C. Bajaj and M. S. Kim for helpful conversations.

\section*{References}

% return to defaults

\parskip=0pt
\parindent=20pt
\hangindent=2em  % amount to indent
\hangafter=1	 % indent after 1st line of paragraph
Aho A, Hopcroft J, Ullman J (1974)
The design and analysis of computer algorithms.
Addison-Wesley, Reading, MA.

\hangindent=2em  % amount to indent
\hangafter=1	 % indent after 1st line of paragraph
\noindent 
Abhyankar S, Bajaj C (1986)
Automatic parameterization of rational curves and surfaces III:
algebraic plane curves.
Tech. Rep. CSD-TR-619, Dept. of Computer Science, Purdue University.

\hangindent=2em  % amount to indent
\hangafter=1	 % indent after 1st line of paragraph
\noindent 
Bajaj C, Hoffmann C, Hopcroft J (1987)
Tracing planar algebraic curves.
Tech. Rep. \mbox{CSD-TR-637}, Dept. of Computer Science, Purdue University.

\hangindent=2em  % amount to indent
\hangafter=1	 % indent after 1st line of paragraph
\noindent 
Bentley JL, Ottmann TA (1979)
Algorithms for reporting and counting geometric intersections.
IEEE Trans. on Computers C-28(9): 643--647.

\hangindent=2em  % amount to indent
\hangafter=1	 % indent after 1st line of paragraph
\noindent 
Canny JF (1987) The complexity of robot motion planning.
Ph.D. Thesis, Dept. of Computer Science, MIT.

\hangindent=2em  % amount to indent
\hangafter=1	 % indent after 1st line of paragraph
\noindent 
Dobkin DP, Thurston WP, Wilks AR (1986)
Robust contour tracing.
Tech. Rep. CS-TR-054-86, Dept. of Computer Science, Princeton University.

\hangindent=2em  % amount to indent
\hangafter=1	 % indent after 1st line of paragraph
\noindent 
Edelsbrunner H (1987) Algorithms in combinatorial geometry.
Springer-Verlag, New York.

\hangindent=2em  % amount to indent
\hangafter=1	 % indent after 1st line of paragraph
\noindent 
Hoffmann CM (1987) 
Algebraic curves.
Tech. Rep. CSD-TR-675, Dept. of Computer Science, Purdue University.

\hangindent=2em  % amount to indent
\hangafter=1	 % indent after 1st line of paragraph
\noindent 
Hoffmann CM, Lynch RE (1987) 
Following space curves numerically.
Tech. Rep. CSD-TR-684, Dept. of Computer Science, Purdue University.

\hangindent=2em  % amount to indent
\hangafter=1	 % indent after 1st line of paragraph
\noindent 
Johnstone JK (1987) The sorting of points along an algebraic curve.
Tech. Rep. 87-841, Ph.D. Thesis, Dept.\ of Computer Science, Cornell University.

\hangindent=2em  % amount to indent
\hangafter=1	 % indent after 1st line of paragraph
\noindent 
McKenna M (1987)
Worst-case optimal hidden-surface removal.
ACM Trans. on Graphics 6(1): 19--28.


\hangindent=2em  % amount to indent
\hangafter=1	 % indent after 1st line of paragraph
\noindent 
Mortenson ME (1985) Geometric modeling.
John Wiley and Sons, New York.

\hangindent=2em  % amount to indent
\hangafter=1	 % indent after 1st line of paragraph
\noindent 
Owen JC, Rockwood AP (1987)
Intersection of general implicit surfaces.
In: Farin G (ed) \mbox{Geometric} modeling: algorithms and new trends.
SIAM, Philadelphia, pp. 335--345.

\hangindent=2em  % amount to indent
\hangafter=1	 % indent after 1st line of paragraph
\noindent 
Preparata FP, Shamos MI (1985) Computational geometry: an introduction.
Springer-Verlag, New York.

\hangindent=2em  % amount to indent
\hangafter=1	 % indent after 1st line of paragraph
\noindent 
Requicha AAG (1980) 
Representations for rigid solids: theory, methods, and systems.
Computing Surveys 12(4): 437--464.

\hangindent=2em  % amount to indent
\hangafter=1	 % indent after 1st line of paragraph
\noindent 
Sechrest S, Greenberg DP (1982)
A visibility polygon reconstruction algorithm.
ACM Trans. on Graphics 1(1): 25--42.

\hangindent=2em  % amount to indent
\hangafter=1	 % indent after 1st line of paragraph
\noindent 
Sederberg TW, Anderson DC, Goldman RN (1984)
Implicit representation of parametric curves and surfaces.
Computer Vision, Graphics, and Image Processing 28: 72--84.

\hangindent=2em  % amount to indent
\hangafter=1	 % indent after 1st line of paragraph
\noindent 
Timmer HG (1977) 
Analytical background for computation of surface intersections.
Douglas Aircraft Company Technical Memorandum C1-250-CAT-77-036, 
cited in (Mortenson 1985). 

\hangindent=2em  % amount to indent
\hangafter=1	 % indent after 1st line of paragraph
\noindent 
van der Waerden BL (1953) Modern Algebra.
Frederick Ungar, New York.

\hangindent=2em  % amount to indent
\hangafter=1	 % indent after 1st line of paragraph
\noindent 
Walker RJ (1950) Algebraic curves.
Springer-Verlag, New York.

\end{document}
