%
% coor.tex		M. Goodrich and J. Johnstone
%
\newif\ifFull
\Fullfalse		% This disables some proofs, etc.
% \Fulltrue		
%
% Define document style
%
\ifFull
  \documentstyle[12pt,titlepage]{article}
\else
  \documentstyle[11pt]{article} 
\fi
%
\input{macros}
%
% Page format
%
\DoubleSpace
\setlength{\oddsidemargin}{0pt}
\setlength{\evensidemargin}{0pt}
\setlength{\headsep}{0pt}
\setlength{\topmargin}{0pt}
\setlength{\textheight}{8.75in}
\setlength{\textwidth}{6.5in}
%
\title{Coordinated Crawling: A Method for Intersecting Algebraic Curve Segments}
\author{Michael T.\ Goodrich and John K.\ Johnstone \\[5pt]
	Dept.\ of Computer Science, The Johns Hopkins Univ., Baltimore,
	MD 21218}
%
\begin{document}
%
\maketitle
%
\begin{summary}{Summary of Results}
In this paper, we present a novel alternative to the 
conventional method of computing the intersections of planar algebraic curve segments.
Our method differs from the conventional method
in that it relies upon geometric locality, and hence can take advantage of
situations in which one is dealing with curve segments rather than entire curves.
The method is based upon an extension of methods for tracing along a curve.
We develop the tools to intersect simple, monotone segments, and then
address the problem of computing all intersection points determined by
a given collection of algebraic curve segments in the plane.
\end{summary}

\section{Introduction} 
The boolean operations of union, intersection, and 
difference are axiomatic to geometric modeling.
A constructive solid geometry representation of an object is no more than
a (regularized) Boolean combination of basic elements.
Intersection is particularly important, being necessary in applications from
interference detection to hidden-surface elimination.
In this paper we present a novel alternative to the 
conventional method of computing the intersections of planar algebraic curve segments.

A planar algebraic curve is the solution set of a (bivariate) polynomial
equation $f(x,y) = 0$.
Taking this view, the intersection of two curves 
(and indeed of any collection of algebraic varieties) is simply the
solution set of a simultaneous system of polynomial equations.
Therefore, classical techniques from the theory of elimination can be applied,
as follows.
The system of equations is reduced to a single univariate equation by
eliminating variables, using resultants \cite{walker,canny}.
The univariate equation is then solved, say by Newton's method, yielding
one coordinate of each solution of the original system.
Finally, the full solutions are built up from these partial solutions,
by solving more univariate equations.
\ifFull\footnote{For example, the 
	resultant \cite{walker} of f(x,y) and g(x,y) (w.r.t. y) 
	is a univariate equation h(x) that encodes the common roots of f and g:
	$h(x_{0})=0$ iff $(x_{0},y_{0})$ is a common root of f and g, for 
	some $y_{0}$.  The potential $y_{0}$ are found by solving the
	univariate equation $f(x_{0},y)=0$, and then $y_{0}$ such that
	$g(x_{0},y_{0}) \neq 0$ are eliminated.}\fi
\ This mathematical approach is the conventional way of 
intersecting algebraic varieties.

An artifact of the system-of-equations approach to intersection 
is that all of the intersections between the two curves are found.
If one only wants the intersections between two segments of the curves,
one must first compute all of the intersections and then decide which
of these intersections actually lie on the segments in question.
The restriction of such a set $S$ of intersection
points to a segment \arc{AB} of a curve $C$ 
involves sorting $S \cup \{A,B\}$ along $C$,
an operation which is challenging in its own right~\cite{johnstone87}.
In other words, with the elimination method, the intersection of two
curve segments is actually more expensive than the intersection of the entire 
curves.
This is even true for the intersection of very simple segments of very
complex curves.

We present a new method that allows the localization of intersection: 
that is, a method for the intersection of planar algebraic curve {\em segments}.
The method is sensitive to the input: the simpler and shorter the segment, 
the more efficient the intersection computation.
The motivation for this work comes from geometric modeling, where
the intersection of (often simple) curve segments, as opposed to curves, is
important.

The main result of this paper is a method for solving the following problem:
compute all intersection points of 
a given collection of planar algebraic curve segments.
Our method consists of two phases: a decomposition phase and a 
sweeping phase.
In the decomposition phase, we partition each curve segment into
{\it xy-monotone} curve segments, that is, segments which are monotone with
respect to both coordinate axes.
(Note that the decomposition phase is a preprocessing step that only needs
to be done once for each curve. Future intersections do not bear the
expense of decomposition.)
Then, in the sweeping phase, we apply a variant of the familiar plane
sweeping method of Bentley and Ottmann~\cite{BeO79} to compute all the
intersection points.
This sweeping phase is non-trivial, however, in that it relies on the
primitive operation of intersecting two xy-monotone curve segments
(which may intersect many times).
We accomplish the intersection of two xy-monotone segments by an extension
of crawling along a curve.
Crawling is a method for moving along a curve that has received
much attention of late~\cite{dtw,hl,or,bhh,h}.
We introduce two new methods of crawling in a coordinated fashion
along two curves at the same time:
a {\it staircase} crawl 
and a {\it simultaneous} crawl.
We show that
these coordinated crawls always converge to an intersection, if there is
one, and output ``no intersection,'' otherwise.
After presenting the basic method, we offer two optimizations:
recognizing when two segments cannot intersect (i.e., when early exit is
possible)
\ifFull first for general xy-monotone segments (Section~\ref{sec-endconds}) 
and then an even better method for convex xy-monotone 
segments (Section~\ref{sec-convabort}).\fi
% more end conditions
and using line-curve intersections to isolate a small region for a
segment-segment intersection before coordinated crawling is 
applied.
\ifFull(Section~\ref{sec-linecurve}),\footnote{This tactic 
	is used in root-finding of univariate polynomials:
	binary search is used to isolate a region for the root
	before Newton's method is applied.}\fi

The presentation of this paper is as follows.
In the section which follows we describe our methods for coordinated
crawling, including the staircase crawl and the simultaneous crawl.
In section~\ref{sec-intersect} we show how to apply these crawling
techniques to the problem of computing all intersections of a collection
of planar algebraic curve segments.

\section{Coordinated Crawling}

\Comment{		% Comment out this subsection:
\subsection{Definitions}
\begin{definition}
A segment \arc{AB}\ of a curve is {\em x-monotone (y-monotone)} if the 
value of x (y) is monotonic on \arc{AB}.  
\arc{AB}\ is {\em xy-monotone} if it is both x-monotone and y-monotone.
An xy-monotone segment \arc{AB} is {\em y-increasing} if 
$x(A) < x(B)$ and $y(A) < y(B)$.  
Otherwise, it is {\em y-decreasing}.
\end{definition}

Note that there are only four flavours of convex, xy-monotone 
segment (Figure~\ref{fourflavours}).

\figg{fourflavours}{(i) a segment falling from left to right 
			(concavely and convexly)
(ii) a segment rising from left to right (concavely and convexly)}{1.5in}

\begin{definition}
\[ x(A) = \left\{ \begin{array}{ll}
		a_{x} & \mbox{if $A = (a_{x},a_{y})$} \\
	\{a_{1,x},\ldots,a_{n,x}\} & 
		\mbox{if $A = \{(a_{1,x},a_{1,y}),\ldots,(a_{n,x},a_{n,y})\}$}
		\end{array}
	  \right.  \]
x() extracts the abscissa of a point or the abscissae of a set of points. 
\end{definition}
}	% comment

Crawling is a method of traversing a curve.  Progress is made by
repeatedly making short steps away from the curve and
then relaxing back onto the curve.
There are various ways of stepping away from the curve (such as
along the tangent or in a direction parallel to the axes), while the relaxation
back onto the curve can be achieved with Newton's method.
The size of each step can be refined near singularities, where confusion
is more likely.
One of the useful properties of crawling 
is its locality: it only relies on the behavior of the curve in a
restricted neighborhood of the current position.
The reader is referred to \cite{bhh,h,hl,or} for the details of this method.

In this section, we show that crawling can be used to discover
the intersections between two curve segments.
For this purpose, we introduce a technique
which we call {\it coordinated crawling}.
Let \arc{AB} and \arc{CD} be two xy-monotone curve segments.
The idea of the coordinated crawling technique, which will
find the first intersection of \arc{AB} and \arc{CD} (if it exists),
is to crawl
along the two segments in tandem, starting at the beginning of each segment.
The crawl along \arc{AB} alternates with the crawl along \arc{CD} so that,
at any given time, progress is being made along only one of the 
segments (Figure~\ref{fig.stair}).
A particular crawl on one segment
continues until a switching condition becomes true, at which
time it is interrupted and the crawl along the other segment is resumed.
The alternation between segments continues until an end condition becomes
true, signalling that an intersection has been found or that the two segments
do not intersect.

For any point $A$, we let $x(A)$ and $y(A)$ denote the $x$- and
$y$-coordinates of $A$ respectively.
In the remainder of this paper
we assume that all curves are non-linear, planar, algebraic curves;
and the name 
\arc{AB} of a curve segment is always chosen so that the abscissa of
$A$ is less than or equal to the abscissa of $B$, i.e., $x(A) \leq x(B)$.
Coordinated crawling has the following general algorithm.

\begin{enumerate}
\item 
	If the projections of \arc{AB} and \arc{CD} onto the x-axis (resp.,
	y-axis) are disjoint, then output `no intersection', otherwise
	continue.
\item
	Flag one of the segments as active and the other as dormant.
\item
	Repeat the following until the end condition is satisfied
	(e.g., an intersection is discovered or we ``run off the end''
	of one of the segments):
\begin{itemize}
\item
	Crawl along the active segment until the switching condition 
	is satisfied.
\item
	Exchange the active and dormant segments.
\end{itemize}
\end{enumerate}
%
%
\ifFull
\begin{example}
\label{eg1}
Consider the coordinated crawl of Figure~\ref{fig.stair}(a).
Horizontal and vertical lines have been added to the picture 
to reveal the structure of the crawls.
Each individual (dashed) subcrawl continues until both of the coordinates of the
point on the active segment exceed the coordinates of the current point on the 
dormant segment.
This series of subcrawls follows a staircase pattern (Figure~\ref{fig.stair}(b))
which converges to the first intersection of the segments.
If no intersection exists, then the coordinated crawl reaches the end of one 
of the segments (Figure~\ref{fig.stair}(c)).

A different switching condition is needed for the coordinated crawl of 
Figure~\ref{fig.simul}(a).... FINISH
\end{example}
\fi
%

The coordinated crawl as we have presented it will only find the first 
intersection (say, $x \in \arc{AB} \cap \arc{CD}$).
The second intersection can be found by starting another coordinated crawl
along \arc{xB} and \arc{xD}.
In general, a new coordinated crawl should be begun from each intersection,
until it is
determined by the end condition that the segments do not intersect any further.

\Comment{
Note that crawling along an xy-monotone segment should be
simpler than crawling along a generic segment, since the crawl,
knowing that it should travel up and right (say), is not so easily confused
(e.g., as it passes through a singularity).?
}

In the following subsections, we describe the two variants of coordinated
crawling that are needed for determining the intersection points of
xy-monotone segments: the staircase crawl and the simultaneous crawl.
We first introduce some notation.
Let \arc{AB} and \arc{CD} be the two curve segments.
Let $A'$ ($C'$) be the point on \arc{AB} (\arc{CD}) to which we have crawled, 
and let
$P_{\mbox{\footnotesize{active}}}$ ($P_{\mbox{\footnotesize{dormant}}}$)
be the point to which we have crawled on the active (dormant) segment.

\subsection{The Staircase Crawl}
\label{sec-stair}
We distinguish two types of xy-monotone segments:
an xy-monotone segment is {\it rising} if it increases in $y$ as 
it increases in $x$, otherwise it is {\it falling}.
This section describes a coordinated crawl that is appropriate when
both segments are rising or both segments are falling: 
the staircase crawl (Figure~\ref{fig.stair}).
(The next subsection describes a markedly different coordinated 
crawl for the case when one segment is rising and the other is falling.)

\Comment{
This section gives a full description and justification of the coordinated 
crawl method for the case when both segments are rising 
(an xy-monotone segment is rising if it increases in y as it increases in x),
which we call the staircase crawl (see Figure~\ref{fig.stair}).
The next section describes a markedly different method for the case when one
segment is rising and the other is falling
(an xy-monotone segment is falling if it decreases in y as it increases in x).
In light of the general algorithm that we have given for a coordinated crawl,
the staircase crawl can be described by specifying the
end conditions and switching conditions that define it,
and then showing that the coordinated crawl so defined converges to an 
intersection if one exists.
} % comment

In light of the general algorithm that we have given for a coordinated crawl,
the staircase crawl can be described by specifying the
switching condition and end condition that define it.
We begin with 
the switching condition of a coordinated crawl along 
two rising segments.
A crawl along the active segment will continue until both coordinates of 
$P_{\mbox{\footnotesize{active}}}$ exceed the coordinates of 
$P_{\mbox{\footnotesize{dormant}}}$.
Then, before beginning to crawl along the other segment, 
the crawl will back up one step, which ensures that the granularity of the crawl
does not cause an intersection to be overlooked (Figure~\ref{fig-skip}).
We also add a universal end condition to the switching condition:
to ensure that a crawl stops if it reaches the end of a segment.
\Comment{
Therefore, after adding a condition to ensure that a crawl stops if it 
reaches the end of a segment, 
the switching condition of the coordinated crawl for rising
segments becomes the following:
\begin{eqnarray*}
	[x(P_{\mbox{\footnotesize{active}}} + \mbox{1 crawl step}) \  > \ 
	x(P_{\mbox{\footnotesize{dormant}}}) \ \ \wedge \ \ 
	y(P_{\mbox{\footnotesize{active}}} + \mbox{1 crawl step}) \ > \ 
	y(P_{\mbox{\footnotesize{dormant}}})] \ \ \vee
\end{eqnarray*}
\vspace*{-.125in}
\begin{eqnarray*}
	x(P_{active}) \ \ > \ \ \{x(A), x(B), x(C), x(D)\}
\end{eqnarray*}
where $r > S$ means that $r > s, \forall s \in S$.}
Notice that at the end of each crawl along \arc{AB},
$x(A') \approx x(C')$ and $y(A') > y(C')$;
while at the end of each crawl along \arc{CD},
$x(A')  <  x(C')$ and $y(A') \approx y(C')$ (Figure~\ref{fig.stair}).
That is, the coordinated crawl between rising segments defined by this
switching condition follows a staircase between the segments,
for which reason it is dubbed a {\it staircase crawl}.

Next, we define the end condition of the staircase crawl.
The staircase crawl will stop when an intersection is found or 
when it is determined that the two segments cannot intersect
(e.g., the end of one of the segments is reached).
We assume without loss of realism that an intersection has been found if
the distance between the two segments becomes very small 
(i.e., less than a predetermined $\epsilon$).
This is legitimate because, for the purposes of applications such as graphics
and geometric modeling, two segments may as well intersect 
if they get very close together (e.g., closer than a pixel).
%Moreover, the system-of-equations approach does not find intersections exactly either.
(We shall see in Lemma~\ref{lem-9} that the size 
of $\epsilon$ does not affect the efficiency of the crawl, 
so $\epsilon$ can be chosen as small as desired.)
Therefore, an end condition is $\mbox{dist}(A',C') < \epsilon$.
\ifFull
((Another possibility: if $\epsilon$ is very small or the crawl is very coarse,
the coordinated crawl may not be able to make any progress on either segment
near an intersection
(because the length of the stair becomes shorter than a crawl step),
even though the segments are not yet within $\epsilon$ of each other.
Therefore, it is probably necessary to add an end condition reflecting this
no-progress situation: 
$x(A') \leq x(\mbox{previous }A') \ \wedge\  x(C') \leq x(\mbox{previous } C')$
CLEAN UP \fi
The end condition associated with reaching the end of one 
of the segments is simple, since the segments are x-monotone:
$x(A') > x(B) \ \vee \  x(C') > x(D)$.
%
%The following 'gap theorem' reveals that two segments must intersect if they 
%get too close together, which allows the coordinated crawl to 
%confidently decide that an intersection has been found as soon as the 
%crawl points on the two segments become closer than some fixed 
%positive $\epsilon$.
%
%\begin{lemma}[Canny{ \cite[p.54]{canny} }]
%Let $\cal{p}$(d,c) be the class of polynomials of degree d and 
%coefficient magnitude c.
%Let $f_{1}(x_{1},\ldots,x_{n}),\ldots,f_{n}(x_{1},\ldots,x_{n}) \in 
%\cal{p}(d,c)$
%be a collection of n polynomials in n variables which has only 
%finitely-many solutions when projectivized.
%Then if $(\alpha_{1},\ldots,\alpha_{n})$ is a solution of the system,
%then for any j either
%\begin{eqnarray*}
%\alpha_{j} = 0 \mbox{or} \mid \alpha_{j} \mid > (3dc)^{-nd^{n}}
%\end{eqnarray*}
%\end{lemma}
%

%We must show that the staircase crawl converges to an intersection 
%if one exists.
%
%
\begin{lemma}
Let \arc{AB} and \arc{CD} be two rising segments.
The staircase crawl along \arc{AB} and \arc{CD} will converge
to the first intersection of \arc{AB} and \arc{CD}, if such an 
intersection exists.
Otherwise, it will reach the endpoint (B or D) of one of the segments.
\ifFull 
\end{lemma}
ONLY IN TECH REPT.
\Heading{Proof:}
Don't jump over an intersection: since the crawl follows a staircase (and does not
jump past staircase even though crawl makes discrete jumps), this is clear.
[A complex proof would not help anyone.)

Progress is made with each subcrawl (i.e., you get closer to the end of the active
segment with each subcrawl) unless we have found an intersection:
	(a) can assume that the switching condition for one of the segments is false,
since otherwise the crawl has stopped making progress and we should 
recognize an intersection
	(b) show that if the switching condition for one of the segments is false,
then progress is made along that segment in the next subcrawl [easy]

Write an explanation of why the discreteness of the crawl, and the resulting approx
nature of equality and solutions, does not compromise our results.
END OF 'ONLY IN TECH REPT'
\QED
\else
\end{lemma}
\Heading{Proof:} Omitted. \QED
\fi

The staircase crawl along two falling segments is entirely analogous to the
staircase crawl along two rising segments.
	\Comment{
	\footnote{Indeed, 
	the only change is in the second switching condition, which becomes
	$y(P_{\mbox{\footnotesize{active}}} + \mbox{1 crawl step}) \  < \ 
	y(P_{\mbox{\footnotesize{dormant}}})$.}
	} % comment
We next describe the method for performing coordinated crawling when one
segment is rising and the other is falling.

\subsection{The Simultaneous Crawl}
\label{sec-simul}

Consider a coordinated crawl along a rising segment \arc{AB} and a falling
segment \arc{CD} (Figure~\ref{fig.simul}).
The staircase crawl cannot be used to find the intersection of these
segments.
\ifFull(Figure~\ref{countereg}).\fi
Thus, we derive an entirely different coordinated crawl for this case.
As before, we begin with the switching condition.
A crawl along the active segment will continue until the abscissa of
$P_{\mbox{\footnotesize{active}}}$ exceeds the abscissa of 
$P_{\mbox{\footnotesize{dormant}}}$, yielding the following switching 
condition:
$x(P_{\mbox{\footnotesize{active}}})\ >\ x(P_{\mbox{\footnotesize{dormant}}})$.
(Since each crawl--except possibly the first one--is short,
there is no need to include an end condition in the switching condition,
as we did with the staircase crawl.)
Thus, as with the staircase crawl, each crawl continues until it catches
up with the dormant segment.
However, for this coordinated crawl, only the abscissa has to catch up,
whereas, with the staircase crawl, both coordinates had to catch up.
This coordinated crawl is dubbed a {\it simultaneous crawl},
since its behaviour is very similar to
crawling along both segments at the same time (while maintaining 
the same velocity with respect to x).

The end condition for a simultaneous crawl consists of three parts:
\begin{itemize}
\item 
	Since \arc{AB} is rising and \arc{CD} is falling,
	the two segments cannot intersect if $y(A) > y(C)$.
	This condition need only be tested at the beginning of the crawl.
\item
 A switch in the relative order of $y(A')$ and $y(C')$ 
 alerts the simultaneous crawl that an intersection has been encountered during
 the last crawl.
 Moreover, since each crawl is stopped as close to $x(A') = x(C')$ as the
 granularity of the crawls allows, the present $A'$ (when the relative order 
 switches) is the best approximation to the intersection.
\item
 Finally, the crawl should stop when the end of one of the segments is reached, 
 which is recognized by the condition 
 $x(A') \geq x(B)\ \vee \  x(C') \geq x(D)$.
\end{itemize}
Written as a boolean formula, the end condition is the following:
\begin{eqnarray*}
	y(A) > y(C)\ \ \ \vee \ \ \ y(A') > y(C')\ \ \ \vee \ \ \ 
	x(A') \geq x(B)\ \ \ \vee \ \ \ x(C') \geq x(D).
\end{eqnarray*}

The switch condition, as described above
(``change to the other segment {\em as soon as} 
the relative order of $x(A')$ and $x(C')$ changes''),
can make for very short crawls and very many switches.
In order to avoid this, we may also ask that each crawl 
be at least $i$ steps long.
In this case, finding the intersection is a bit more difficult.
The intersection is not necessarily $A'$ 
when $y(A') > y(C')$ first becomes true:
the intersection is somewhere during the last crawl.
That is, since this
crawl did not necessarily stop near $x(A') = x(C')$, it did not necessarily
stop near the intersection.
Therefore, backtracking is necessary: the last crawl must be repeated, but 
with no lowerbound on the number of steps per crawl, continuing until 
$y(A') > y(C')$ as before.
Note that the 
last crawl may become several crawls when the lowerbound is removed.
Also note that a rising and falling segment can intersect at most once so,
unlike the staircase crawl,
there is no need to continue a simultaneous crawl from the first intersection.

\ifFull
TECH REPT ONLY
Two asides.
First, a monotonically rising segment (of a non-linear algebraic curve) must
be strictly monotonically rising, by Bezout's Theorem \cite{walker}.
Second, it is easy to see that a strictly monotonically rising segment can
intersect a strictly monotonically falling segment in at most one point.
(Therefore, the simultaneous crawl need only look for one intersection.)
END OF 'TECH REPT ONLY'
\fi

\Comment{
\begin{lemma}
\label{lem-incdec}
Let \arc{AB} be a rising segment and \arc{CD} a falling segment. 
Then \arc{AB} intersects \arc{CD} at most one time.
\end{lemma}

\Heading{Proof:}
OBVIOUS WITHOUT PROOF
Suppose that \arc{AB} has more than one intersection with \arc{CD}.
Let $p \neq q \in \arc{AB} \cap \arc{CD}$.
Assume w.l.o.g. that $x(p) < x(q)$ (using Lemma~\ref{strict}).
By x-monotonicity, $q \in \arc{pB} \cap \arc{pD}$.
Since $y(A) < y(B)$ and \arc{AB} is strictly y-monotone,
$q \in \arc{pB}$ implies $y(q) > y(p)$.
But, since $y(C) > y(D)$ and \arc{CD} is strictly y-monotone,
$q \in \arc{pD}$ implies that $y(q) < y(p)$.
This contradiction establishes that \arc{AB} can have at most one
intersection with \arc{CD}.
\QED
}  % comment


\begin{lemma}
The simultaneous crawl is an effective method of finding the intersection
between a rising and falling segment, or of diagnosing that no such
intersection exists.
\end{lemma}
\Heading{Proof}: Omitted. \QED

\Comment{
\begin{lemma}
Let $\alpha$ be the unique intersection of \arc{AB}\ and \arc{CD}.
Then, $\arc{A\alpha}$ lies strictly below $\arc{C\alpha}$ (i.e., 
$y(\arc{A\alpha} \setminus \{\alpha\}) < y(\arc{C\alpha} 
\setminus \{\alpha\}))$.
Thus, $\alpha$ can be found by simultaneously crawling along \arc{AB} 
and \arc{CD} until the ordinate on \arc{AB} becomes equal
to the ordinate on \arc{CD}.
\end{lemma}

\Heading{Proof:}
Obvious.
\arc{AB} is strictly y-increasing  and \arc{CD} is strictly y-decreasing
(Lemma~\ref{strict}).
Therefore, $y(\arc{A\alpha} \setminus \{\alpha\}) < y(\alpha)$, and
$y(\alpha) < y(\arc{C\alpha} \setminus \{\alpha\}))$.
\QED
} % comment

We end our description of the simultaneous crawl by noting that it 
cannot be used to find the intersection of two rising segments,
because it is possible to skip over intersections.
Thus, we must indeed have both the staircase crawl and the simultaneous crawl.

\subsection{Improvements}
\label{sec.improve}
\ifFull In this subsection, we outline some methods for improving the efficiency of a 
coordinated crawl.
%
% THIS SUBSECTION DONE
\subsection{Recognition of non-intersection}
\label{sec-endconds}
%
A coordinated crawl can be aborted as soon as
it becomes apparent that the two segments cannot intersect.
The following lemma outlines some conditions that guarantee non-intersection.
These conditions should be added to the end conditions, and perhaps
the switch condition as well.
(However, since the purpose of these conditions is merely to 
avoid unnecessary crawling, they need not and probably should not be 
tested after each step: only intermittently, so as not to add a computational
burden to each step.)
%
\begin{lemma}
\label{lem-badconds}
Let \arc{A'B}\ and \arc{C'D}\ be xy-monotone segments (during a coordinated 
crawl).
If any of the following conditions is true, then \arc{AB}\ and \arc{CD}\ do not
intersect:
\begin{enumerate}
	\item $x(B) < x(C)$
	\item $x(D) < x(A)$
	\item $\{y(A),y(B)\} < \{y(C),y(D)\}$
	\item $\{y(C),y(D)\} < \{y(A),y(B)\}$
\end{enumerate}
(Recall that we have assumed that $x(A') \leq x(B)$ and $x(C') \leq x(D)$.)
\end{lemma}
%
ONLY IN TECH REPORT
{\bf Proof} 
\begin{enumerate}
	\item $x(B) < x(C)$\\
      	      $\Rightarrow x(A) \leq x(B) < x(C) \leq x(D)$, by our naming
convention (*)\\
	      $\Rightarrow x(\arc{AB}) < x(\arc{CD})$, since \arc{AB}\ and \arc{CD}\ 
are x-monotone
	      $\Rightarrow$ \arc{AB} and \arc{CD}\ cannot intersect, since they do not
share any x-values
	\item Symmetric to 1.
	\item Since both segments are y-monotone,
		$\{y(A),y(B)\} < \{y(C),y(D)\} \Rightarrow y(\arc{AB}) < y(\arc{CD})
\Rightarrow$ \arc{AB} and \arc{CD} cannot intersect
	\item Symmetric to 3.
\end{enumerate}
\hspace{3in}QED\\
END ``only in tech report''\\
%
% THIS SUBSECTION DONE
\subsection{More recognition of non-intersection, for convex segments}
\label{sec-convabort}
%
If the two segments are convex as well as xy-monotone, 
then there are still more ways to recognize when they do not intersect.
(A segment is {\it convex} if no line hits it in more than two points.)
The necessary tool is the {\it tangent triangle} 
$\Delta AB$ of a convex segment
\arc{AB}, which is the triangle whose sides are the tangent at A, the tangent
at B, and \seg{AB}.\footnote{If A or B is singular, 
	then we must crawl a short distance from it before considering 
	the tangent triangle, since the tangent is not well defined at 
	a singularity.}
%
\begin{lemma}
If \arc{AB}\ and \arc{CD} are convex segments and
$\Delta AB \cap \Delta CD = \emptyset$,
then $\seg{AB} \cap \seg{CD} = \emptyset$.
\end{lemma}
\Heading{Proof:}
Since \arc{AB} is convex, $\Delta AB$ contains \arc{AB}.
Similarly, $\Delta CD$ contains \arc{CD}.
The result follows immediately.
\QED
%
Therefore, a coordinated crawl along convex segments can be aborted as soon as 
$\Delta A'B \cap \Delta C'D = \emptyset$.
%
%
\subsection{Using line-curve intersections}
\label{sec-linecurve}
%
Further optimization is possible using line-curve intersections (viz., 
intersections of horizontal and vertical lines with the curve segments).
These can be used to avoid long crawls (in staircase crawls)
and to get closer to the intersection before beginning the coordinated crawl
(for simultaneous crawls).
%
\subsubsection{Staircase crawl}
%
% there is another way of finding the endpoints of the stairs:
% by intersecting horizontal and vertical lines with the curves
%
The essential purpose of a staircase crawl is to find the series of 
endpoints of the stairs.
For example, the staircase crawl of Figure --- is characterized by the list
of stair endpoints:   .
The stair $C_{0}A_{1}$ that follows $A_{0}C_{0}$ in the series can be 
found by crawling from $A_{0}$ until the coordinates of $A'$ exceed those
of $C_{0}$.
However, it can also be found by finding the intersection
of the horizontal line through $C_{0}$ with \arc{AB}.
Similarly, the next stair $A_{1}C_{1}$ can be found by finding the 
intersection of the vertical line through $A_{1}$ with \arc{CD}.

%
% create some of the stairs by line-curve intersection rather than crawling,
% until the length of the stair gets small (i.e., until crawling becomes 
% faster)
%
% also, have an upperbound on the time that you will allow a crawl to the next
% stair to continue before you abort the crawl and use a line-curve intersection
% to find the next point
%
Therefore, any of the crawls of a staircase crawl can be replaced by a 
line-curve intersection.
The time to crawl between stair endpoints E,F depends upon the length of the
curve \arc{EF}; but the time for a line-curve intersection is essentially
constant (for a given curve).
Therefore, only long crawls should be replaced by line-curve intersections.
An upperbound U\footnote{Of the same order of magnitude as the time for a 
	line-curve intersection.}
can be introduced, such that a single crawl will be aborted if it takes
more time than U and the next stair endpoint will instead be found by a 
line-curve intersection.
%
\subsubsection{Simultaneous crawl}
%
Line-curve intersections can be used in a simultaneous crawl 
to perform a preliminary binary search for the intersection
(Figure~\ref{fig.binary}).
%
% After a binary search, the intersection will be isolated between
%  two horizontal lines. Begin the crawl from the left intersections
%  of these lines.
%
The binary search will isolate the intersection between two lines and,
in general, will allow the simultaneous crawl to begin closer to the
intersection.
%
Both horizontal and vertical binary searches are possible.
The binary search does not continue to completion.
As above, there is a tradeoff between the number of line-curve intersections
that are computed and the amount of crawling that is saved.
As a general rule, the binary search continues longer when the degree
of the curves is low (so that line-curve intersection is cheap)
and when the endpoints of the curve segments are far apart  
(so that one can anticipate a lot of crawling, making it more likely
that a binary search will indeed reduce the crawling substantially).
%
Note that it is not possible to binary search for an intersection when
both segments are rising, because it is possible to skip over intersections 
without noticing it.
In particular, any even number of intersections between two 
stabs of the binary search are invisible (Figure~\ref{fig.counter2}).
This could not happen with the simultaneous crawl, because there is 
only one intersection.
%
%
%

ONLY IN TECH REPT.
\subsubsection{Computation of line-curve intersections}
%
-reduces to solution of a univariate equation of the same degree as the curve 
	(parameterize the line and substitute this parameterization 
	into the curve's equation)\\
 plus restriction of these roots to the segment in question 
	desired root must have abscissa in the correct range (eg. (x(A),x(B)))
	if more than one root lies in this range, then sorting must be used
		to disambiguate (certainly at most one root lies on \arc{AB}
		since the lines are horizontal or vertical)\\
-fact(?): one can solve an exponential number of line-curve intersections
	in the same time as one curve-curve intersection\\
-especially easy if a low-degree parameterization is known for the curve
	(reduces to solving a univariate equation of the curve param's degree)
END OF 'TECH REPT ONLY'

\else % (end of a long \ifFull, starting with previous \subsection)
In the full version of this paper, we discuss conditions for
establishing that two segments cannot intersect, which allows
early termination of the coordinated crawl.
If the segments are also convex, then additional conditions for early termination
are available.
Optimization is also possible using line-curve intersections (viz., 
intersections of horizontal and vertical lines with the curve segments).
These can be used to avoid long crawls (in staircase crawls)
and to get closer to the intersection before beginning the coordinated crawl
(for simultaneous crawls).
The details are omitted here.
\fi
%
\subsection{Efficiency}
%
% SECTION DONE (FOR NOW)
%
%WE STOP WHEN SEGMENTS GET WITHIN $\epsilon$ OF EACH OTHER.
%IS THIS NECESSARILY CLOSE TO THE ACTUAL INTERSECTION?
%
%We have seen that two segments cannot get too close together (i.e., closer
%to $\epsilon$) without intersecting.
%
The efficiency of a coordinated crawl depends almost entirely 
upon the distance that must be crawled along the two segments before 
an intersection is found (or it is determined that none exists).
In particular, the efficiency of a coordinated crawl is affected little by
the number of subcrawls that it is broken into 
(e.g., the number of stairs in the staircase).
It might seem that if the segments remain very close
for a long time, then a staircase crawl will be slow because
the staircase is very fine with very short stairs. 
The following lemma shows that this is not the case.
%
%\begin{lemma}
%Each subcrawl is of length $l \geq \epsilon$, where $\epsilon$
%is the closest that two nonintersecting segments can get.
%\end{lemma}
%
\begin{lemma}
\label{lem-9}
Let \arc{AB} and \arc{CD} be xy-monotone segments of algebraic 
curves of degree $m$ and $n$, respectively.
The complexity of finding all of the intersections of \arc{AB} and \arc{CD} 
with a coordinated crawl is the complexity of crawling a distance
$O(\mbox{min}(dist(A,B), dist(C,D)))$ along a curve of degree max\{m,n\},
where $dist(\cdot,\cdot)$ is Euclidean distance.
\end{lemma}

\Heading{Proof:}
By a straightforward application of the triangle inequality,
the length of an xy-monotone segment \arc{AB} is bounded by $\sqrt{2}\,dist(A,B)$.
Thus, the length of a crawl along \arc{AB} is $O(dist(A,B))$.
In order to show that the number of stairs doesn't matter,
we must show that stopping and starting a crawl takes essentially no time.
This can be seen as follows.
We can keep two separate regions in memory, one set up for crawling
along \arc{AB}, the other for crawling along \arc{CD}.
Switching crawls merely involves jumping to the other part of memory,
i.e., a context switch.
\QED
\Comment{
Let \arc{AB} be an xy-monotone segment.
It is easy to see that the length of \arc{AB} is bounded by
$\mid x(B) - x(A) \mid + \mid y(B) - y(A) \mid$.
The triangle of Figure~\ref{fig.triangle} shows that 
$\mid x(B) - x(A) \mid + \mid y(B) - y(A) \mid = (\mbox{cos}\theta 
+ \mbox{sin}\theta) \mbox{ dist(A,B)}$.
I claim that $\mbox{max}_{\theta} (\mbox{cos}\theta + \mbox{sin}\theta) = \sqrt{2}$ (at $\theta = \frac{\pi}{4}$).
Therefore, $\mid x(B) - x(A) \mid + \mid y(B) - y(A) \mid \leq
(\sqrt{2}) \mbox{ dist(A,B)}$.
} % comment

\section{Intersecting Many Planar Algebraic Curve Segments}
\label{sec-intersect}

In this section we show how to use the coordinated crawling technique as
a basic operation in a plane-sweep algorithm to enumerate all
intersection points of a collection of planar algebraic curve segments.
As outlined above, we begin our method by first partitioning each curve
segment into xy-monotone curve segments.
We describe this decomposition step next.

\subsection{xy-Monotone Decomposition}
\label{sec-decomp}

We begin by observing that
a curve segment is xy-monotone if and only if it contains no local extrema
(no changes in direction with respect to the x-axis or y-axis).
Therefore, the local extrema of a curve form a natural partition
of the curve into xy-monotone segments.
An xy-monotone decomposition will involve finding
and sorting the curve's local extrema.
The local extrema of a curve $f(x,y)=0$ are the solutions of
$\{f_{x}=0,f=0\}$ and $\{f_{y}=0,f=0\}$,
sorting the extrema along the curve (in order to pair them 
into segment endpoints) can be accomplished in various 
ways (see \cite{johnstone87} for details).

It might seem that the computation of extrema, which itself involves a curve-curve
intersection, leads to a circularity in our method.
This is not the case, however.
First of all, computing the singularities of an algebraic curve is 
so fundamental that
it is 
needed in many geometric modeling algorithms (e.g., \cite{abba,johnstone87}).
It is trivial to determine the extrema of a curve as part of the singularity 
computation (since the singularities are the solutions of $\{f_{x}=0,f_{y}=0,f=0\}$).
Therefore, it is reasonable to assume that when it comes time to intersect two
curves from a geometric model, their extrema will already be available.
But let us suppose that the extrema are not known.
Second, we are trying to find the intersections of $n$ segments, and 
if we do this in the straightforward manner, then we shall perform $O(n^{2})$
curve-curve intersections.
But finding the extrema of the $n$ curves only requires
$2n$ curve-curve intersections.
A final reason that it is not unreasonable to ask for a curve-curve
intersection at this juncture is that curves in a geometric model are relatively
permanent, and intersection is a common operation.
Therefore, we are willing to incur an expense for preprocessing a curve if this
will allow us to speed up all future intersections with this curve.

\Comment{
The decomposition of a curve into convex 
segments (for Section~\ref{sec-convabort}) is described in \cite{johnstone87}.
(It would precede a decomposition into xy-monotone segments.)
A curve of degree $n$ has $O(n)$ xy-monotone segments, since there are
$O(n)$ local extrema.
}  % comment

\ifFull
TECH REPORT ONLY
(a) if you need convex decomposition to be able to sort, 
then xy-monotone-and-convex decomposition is no harder than xy-monotone
decomposition

(b) if sorting can be done without convex decomposition,
then xy-monotone decomposition should be quite a bit easier
than xy-monotone+convex decomposition, since
finding extrema is roughly equivalent to finding singularities rather 
than (i) finding flexes and singularities (ii) finding their tangents
(iii) intersecting these tangents with the curve, and
(iv) pairing up the endpoints of the convex segments.
END OF ``TECH REPT ONLY''
\fi

For the remainder of this section we assume that we are given a
collection of xy-monotone curve segments and wish to find all of the
intersection points that they determine.

\subsection{A Plane Sweep Algorithm for xy-Monotone Curve Segments}

Now that we have developed the necessary primitives, we can describe the
plane-sweep procedure to identify all intersection points.
We begin by inserting all of the xy-monotone segment endpoints into a priority
queue E.
We will be sweeping a vertical line L through the plane from left to right.
As we sweep we will maintain a data base D, which consists of all curve
segments that intersect L, stored in sorted order by their intersections with L.
We represent D as a (2,3)-tree \cite{ahu74} (or some equivalent efficient dynamic
search structure).
Note that since the segments are xy-monotone, each segment will intersect L at most
once.  As we sweep L to the right we need to stop at various {\em event} points to 
maintain the consistency of the data base D.

The priority queue E determines the events.
An event is either an endpoint or an intersection point.
With each curve C we also keep a priority queue E(C) which stores the names of 
all the curves which we have compared with C already.
These lists will prevent us from performing any redundant intersection tests.
A generic step in the plane-sweep algorithm is as follows.
Remove the point in E with minimum $x$-coordinate.
Let $p$ be this point.
Intuitively, this corresponds to moving L to the right until it ``hits'' $p$.
We must then update D depending on the identity of $p$.
We identify each of the possible cases below.

{\bf Case 1}. The point $p$ is the left endpoint of a curve segment $C$.
In order to maintain the consistency of our data base, we must insert $C$ in $D$.
To do this we must find the curve segment $C_{1}$ in $D$ such that $C_{1}$
intersects $L$ in the lowest point above $p$, i.e., $C_{1}$ is directly above $p$.
We can do this by making $O(\mbox{log n})$ curve-segment comparisons to find
a path in the tree $D$ to the place where $C$ belongs.
Each such curve comparison is to determine if a curve $C'$ intersects $L$ above
or below $p$, and can be implemented by a procedure similar to the coordinated
crawl (actually simpler than that, since one of the curves is the vertical line $L$).
After we have located where $C$ belongs in $D$, we must then determine all of the
intersections of $C$ with its predecessor curve $C_{1}$ and successor curve $C_{2}$
in $D$. These can all be determined by the coordinated crawl procedure described
in the previous section.
We add all discovered intersection points $p$ to the priority queue $E$ as long 
as the two curves cross at $p$ (as opposed to simply ``touching'').
We also add $C$ to $E(C_{1})$ and $E(C_{2})$ and add $C_{1}$ and $C_{2}$ to $E(C)$.

{\bf Case 2}. The point $p$ is an intersection point.
If $C_{1}$ and $C_{2}$ are the two curves which intersect at $p$, then we swap them
in $D$.  Without loss of generality, assume $C_{2}$ now occurs before $C_{1}$ in the
list  $D$.  Let $C_{0}$ be the new predecessor of $C_{2}$ and let $C_{3}$ be the 
new successor of $C_{1}$.
Provided $C_{0}$ is not in $E(C_{2})$, we then test $C_{0}$ and $C_{2}$ for
intersections.  Similarly, we test $C_{1}$ and $C_{3}$ for intersections, provided
$C_{3}$ is not in $E(C_{1})$. 
We then update $E(C_{0})$, $E(C_{1})$, $E(C_{2})$, and $E(C_{3})$ as necessary.

{\bf Case 3}.  The point $p$ is a right endpoint of a curve $C$.
In this case we delete $C$ from $D$.
We then need to test the two neighbors $C_{1}$ and $C_{2}$ of $C$ at $p$
(which are now adjacent) for their respective intersection points, if any.
We first check if $C_{1}$ is in $E(C_{2})$, and if not, then perform the intersection
test between $C_{1}$ and $C_{2}$ as outlined above.
Of course, we then update $E(C_{1})$ and  $E(C_{2})$ as necessary.

Since these are all the possible cases, this completes the algorithm.
We summarize with the following theorem:
\begin{theorem}
Given n xy-monotone curve segments in the plane, one can compute all of 
the intersection points determined by these curves in 
$O((n+k)\log n + (n \log n) \alpha + (n+k) \beta)$ time and
$O(n+k)$ space, where $k$ is the number of intersection points, 
$\alpha$ is the time needed to perform an ``above curve'' test, 
and $\beta$ is the time needed to perform an intersection test
(that is, a coordinated crawl). \QED
\end{theorem}

The benefits of this algorithm will be most strongly felt when the segments S are of 
a different order of complexity than the curves $C \supset S$, because
of the associated benefits of coordinated crawling in this case.

\Comment{
\section{The intersection of entire algebraic curves}
TECH REPT ONLY

The method 
may be impractical for entire algebraic curves, since there are O($n^3$) 
convex segments and $O(n)$ xy-monotone segments per curve.
Certainly each segment-pair intersection had better be very fast
(this is where an implementation would be lovely).

END OF 'TECH REPT ONLY'
} % comment

%\section{Implementation Results}
%
% (1) get crawling software from Purdue/Cornell by asking Bajaj or Hopcroft,
%     to use both in the present coordinated crawling work and 
%     in comparing crawling more rigorously with sorting methods
%
%			!!!!!!!!!!!!!!!!!!!!!!!!!!!!
%
% (2) get an account on VMS for MACSYMA access so that we can time the
%     elimination method of curve intersection
%

\section{Extensions}

It is possible to extend the coordinated crawl technique to
segments that are monotone in any two linearly independent directions.
However, it is not clear whether it is possible to extend to non-monotone
segments.
A coordinated crawl needs a simple switching condition, since this condition
must be tested repeatedly during a subcrawl (viz., after each crawl step).
With monotone segments, switching only depends upon the relative size of the
x and y coordinates, which is particularly easy to check.

\section{Conclusions}

We have proposed a method for intersecting curve segments that takes advantage
of the locality and simplicity of the segments.
The complexity of the intersection depends upon the complexity of the segments
rather than the complexity of the curves that contain the segments.
With the system-of-equations approach to curve intersection,
the intersection of simple segments of complex curves is still complex,
since the method must essentially find all of the intersections.
This seems unnatural, or at least undesirable.
The intersection of simple segments becomes simple with 
coordinated crawling.
%
\ifFull Another advantage of coordinated crawling is that
the system-of-equations approach requires the *solution* of univariate 
equations of *high* degree (the product of the degrees of the original curves);
whereas coordinated crawling involves only the *evaluation* of equations 
of *relatively low* degree (the degree of the original curves). \fi
%

We have also presented two extensions of the crawling technique,
the staircase and the simultaneous crawls.

Finally, using the coordinated crawl as a primitive step,
we have extended the well-known plane-sweep algorithm for the intersection
of line segments to an algorithm for the intersection of algebraic curve segments,
which not only reduces (when $k \ll \frac{n^2}{\mbox{log }n}$) 
the number of intersections from $O(n^2)$ to $O((n+k) \mbox{log }n)$, but also
makes the primitive step in the plane sweep a simple intersection of relatively
short
segments rather than a (potentially) complex intersection of entire curves.
%
\ifFull
\section{Appendix}
%
TECH REPT ONLY

\begin{lemma}
Let \arc{AB}\ and \arc{CD}\ be xy-monotone segments such that \arc{AB} is 
rising and \arc{CD} is falling,
and \arc{AB} and \arc{CD}  do not satisfy any of the
conditions of Lemma~\ref{lem-badconds}.
Let $\arc{A_{0}B_{0}} \subseteq \arc{AB}$ and
$\arc{C_{0}D_{0}} \subseteq \arc{CD}$  be subsegments such that 
  $x(A_{0}) = x(C_{0})$ and 
  $x(B_{0}) = x(D_{0})$.
(These subsegments always exist.)
Then \arc{AB} and \arc{CD} intersect if and only if
$\arc{A_{0}B_{0}}$ and $\arc{C_{0}D_{0}}$ do not satisfy any of the conditions
of Lemma~\ref{lem-badconds}.
\end{lemma}
\Heading{Proof:}
Since \arc{AB} and \arc{CD} do not satisfy any of the conditions of 
Lemma~\ref{lem-badconds}, $x(A) \leq x(D)$ and $x(C) \leq x(B)$.
Hence, by our naming assumption (*), $\{x(A),x(C)\} \leq \{x(B),x(D)\}$.
In particular, 
\begin{eqnarray*}
x(A) \leq \mbox{max} \{x(A),x(C)\} \leq \mbox{min} \{x(B),x(D)\} \leq x(B)
\end{eqnarray*}
Let $A_{0}$ (resp., $B_{0}$) be the point of \arc{AB} with abscissa
\mbox{max \{x(A),x(C)\}} (resp., \mbox{min \{x(B),x(D)\}}).
Thus, $\arc{A_{0}B_{0}} \subseteq \arc{AB}$.
(See Figure~\ref{fig1onsheet}.)
Similarly,
\begin{eqnarray*}
x(C) \leq \mbox{max} \{x(A),x(C)\} \leq \mbox{min} \{x(B),x(D)\} \leq x(D)
\end{eqnarray*}
Let $C_{0}$ (resp., $D_{0}$) be the point of \arc{CD} with abscissa
\mbox{max \{x(A),x(C)\}} (resp., \mbox{min \{x(B),x(D)\}}).
$\arc{C_{0}D_{0}} \subseteq \arc{CD}$.
Note that $x(A_{0}) = x(C_{0})$ and $x(B_{0}) = x(D_{0})$.

Suppose that $\arc{A_{0}B_{0}}$ and $\arc{C_{0}D_{0}}$ do not satisfy any of the 
conditions of Lemma~\ref{lem-badconds}.
Then it is clear that $\arc{A_{0}B_{0}}$ must intersect
$\arc{C_{0}D_{0}}$ (Figure).
%
%We shall show that $\arc{A_{0}B_{0}}$ intersects $\arc{C_{0}D_{0}}$.
%Assume w.l.o.g. that $A_{0} \neq C_{0}$ and $B_{0} \neq D_{0}$.
%Consider the curve
%\begin{eqnarray*}
%\alpha := \arc{A_{0}B_{0}} \cup \{(t,y(A_{0})) : -\infty < t \leq x(a_{0})\}
%			   \cup \{(t,y(B_{0})) :  x(B_{0}) \leq t < \infty \}
%\end{eqnarray*}
%\figg{2}{The curve $\alpha$}{1.5in}
%$\alpha$ partitions the plane into two regions.\footnote{After all,
%	it is continuous, non-self-intersecting, and it goes off to infinity
%	at both ends.}
%
%I claim that $C_{0}$ and $D_{0}$ lie on opposite sides of $\alpha$.
%Consider $C_{0}$.
%If $y(C_{0}) < y(A_{0})$, then 
%$y(D_{0}) \leq y(C_{0}) < y(A_{0}) \leq y(B_{0})$,
%satisfying condition (3) of Lemma~\ref{lem-badconds}, which is a contradiction.
%Moreover, we have assumed $A_{0} \neq C_{0}$.
%Therefore, $y(C_{0}) > y(A_{0})$.
%Since $x(C_{0}) = x(A_{0})$, this establishes that $C_{0}$ lies in the
%upper region, above $\alpha$.
%A similar argument establishes that $y(D_{0}) < y(B_{0})$ and that $D_{0}$
%lies in the lower region, below $\alpha$.
%Therefore, $C_{0}$ and $D_{0}$ lie on opposite sides of the extended curve
%$\alpha$, and the continuous curve segment $\arc{C_{0}D_{0}}$ must cross
%$\alpha$.
%Since $x(\arc{C_{0}D_{0}}) = x(\arc{A_{0}B_{0}})$, 
%$\arc{C_{0}D_{0}}$ must cross $\alpha$ through $\arc{A_{0}B_{0}}$.
%This establishes that \arc{AB} has at least one intersection with \arc{CD}.\\

Suppose that $\arc{A_{0}B_{0}}$ and $\arc{C_{0}D_{0}}$ satisfy one or more of the 
conditions of Lemma~\ref{lem-badconds}.
Then $\arc{A_{0}B_{0}} \cap \arc{C_{0}D_{0}} = \emptyset$.
I claim that the other parts of \arc{AB} and \arc{CD} also do not intersect.
SHOW!
\QED
%
$\arc{A_{0}B_{0}}$ and $\arc{C_{0}D_{0}}$ are found by crawling from the
beginning and then backwards from the end of the segments (Figure~\ref{fig.A0B0}).
They might allow you to recognize non-intersection more quickly, 
but it is not clear that it would be wise to invest the effort.
The crawl from A to A0 (or B to B0) is  done anyway to satisfy 
the initial condition of the coordinated crawl.
Perhaps  the effort to crawl backwards from D to d0 (or B to B0)
should be invested iff x(B) - x(D) is small (at least relative to
x(B) - x(A0)).

END OF 'TECH REPT ONLY'
%
%\begin{lemma}
%\label{strict}
%An x-monotone segment of a non-linear, algebraic curve is strictly x-monotone.
%Likewise for y-monotone segments.
%\end{lemma}
%\Heading{Proof:}
%Use Bezout.
%\QED
%
\fi

\SingleSpace
\subsection*{Acknowledgements}
We would like to thank C. Bajaj and M. S. Kim for helpful conversations.
%
\begin{thebibliography}{88}

\bibitem{ahu74} Aho, A., Hopcroft, J., and Ullman, J.,
{\it The Design and Analysis of Computer Algorithms},
Addison-Wesley (Reading, MA: 1974).

\bibitem{abba} Abhyankar, S., and Bajaj, C., 
``Automatic Parameterization of Rational Curves and Surfaces III:
Algebraic Plane Curves,''
Tech. Rep. CSD-TR-619, Dept. of Computer Science, Purdue Univ.,
August 1986.

\bibitem{bhh} Bajaj, C., Hoffmann, C., and Hopcroft, J.,
``Tracing Planar Algebraic Curves,''
Tech. Rep. CSD-TR-637, Dept. of Computer Science, Purdue Univ.,
September 1987.

\bibitem{BeO79} Bentley, J. L., and Ottmann, T. A., 
``Algorithms for Reporting and Counting Geometric Intersections,''
{\it IEEE Trans.\ on Computers}, Vol.~C-28, No.~9, September 1979,
643--647.

\bibitem{canny} Canny, J. F.,
``The Complexity of Robot Motion Planning,''
Ph.D. Thesis, Dept. of Computer Science, MIT, 1987.

\bibitem{dtw} Dobkin, D. P., Thurston, W. P., and Wilks, A. R.,
``Robust Contour Tracing,''
Tech. Rep. CS-TR-054-86, Dept. of Computer Science, Princeton Univ.,
(September 1986).

\bibitem{h} Hoffmann, C. M., 
``Algebraic Curves,''
Tech. Rep. CSD-TR-675, Dept. of Computer Science, Purdue Univ.,
May 1987.

\bibitem{hl} Hoffmann, C. M., and Lynch, R. E., 
``Following Space Curves Numerically,''
Tech. Rep. CSD-TR-684, Dept. of Computer Science, Purdue Univ.,
May 1987.

\bibitem{johnstone87} Johnstone, J. K.,
``The Sorting of Points Along an Algebraic Curve,''
Tech. Rep. 87-841, Ph.D. Thesis, Dept.\ of Computer Science, 
Cornell Univ., June 1987.

%\bibitem{johnstone87b} J. K. Johnstone.
%{\em Journal Article on Sorting}.
%\bibitem{johnstone87c} J. K. Johnstone.
%{\em Journal article on Convex Decomp}.
%for convex decomposition and sorting

\bibitem{or} Owen, J. C., and Rockwood, A. P.,
``Intersection of General Implicit Surfaces,''
In {\it Geometric Modeling: Algorithms and New Trends}, G. Farin (ed.),
SIAM, 335-345 (1987).

\bibitem{walker} Walker, R. J.,
{\it Algebraic Curves,}
Springer-Verlag (New York: 1950).

\end{thebibliography}
\clearpage
\section*{Figures}
									% 1st
\figg{fig.stair}{A staircase crawl.}{3.5in}
%(a) a series of coordinated subcrawls
%		 (b) a compact representation of a staircase crawl}{.1in}
%		 (c????) a coordinated crawl to the end of a segment}{.1in}
%
									% 2nd
\figg{fig-skip}{An intersection may be overlooked unless each subcrawl
		ends by backing up a step.}{1.5in}
									% 3rd
\figg{fig.simul}{A simultaneous crawl.}{2in}
%		 (b) a grid of vertical lines}{.1in}
									% 4th
							% picture is in countereg.pic
%\figg{countereg}{The staircase crawl cannot be used for rising/falling segments: it will
%skip over intersections}{.1in}
									% 5th
						        % picture is in skipover.pic
%\figg{skip}{The simultaneous crawl cannot replace the staircase crawl either}{.1in}
%\figg{fig1onsheet}{old Figure 1 on sheet}{.1in}
%
\end{document}
