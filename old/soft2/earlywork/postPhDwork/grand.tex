\documentstyle [titlepage]{article}
%
\newtheorem{lemma}{Lemma}
\newtheorem{theorem}{Theorem}
\newtheorem{corollary}{Corollary}
\newtheorem{conjecture}{Conjecture}
%
\newcommand{\arc}[1]{\mbox{$\stackrel{\frown}{#1}$}}
\newcommand{\lyne}[1]{\mbox{$\stackrel{\leftrightarrow}{#1}$}}
\newcommand{\ray}[1]{\mbox{$\vec{#1}$}}
\newcommand{\seg}[1]{\mbox{$\overline{#1}$}}
\newcommand{\hence}{\\ \mbox{$.\raisebox{1.5ex}{.}.$}\ } % three-dot therefore
                                             % symbol is desired
\newcommand{\qed}{\nopagebreak\begin{flushright}\mbox{$\Box$}\end{flushright}}
                                       % put a box on the
                                       % righthand side of the page
\newcommand{\se}{\mbox{$_{\epsilon}$}}  % subscript epsilon
\newcommand{\proof}{{\bf Proof}:\nopagebreak\\}
\newcommand{\wlogg}{\mbox{w.\ l.\ o.\ g.\ }}
\newcommand{\wrt}{\mbox{w.\ r.\ t.\ }}
\newcommand{\ww}{\mbox{$W_{1}W_{2}$}}
\newcommand{\wwa}{\arc{\ww}}
\newcommand{\wwh}{\mbox{$\widehat{\ww}$}}
\newcommand{\wone}{\mbox{$W_{1}$}}
\newcommand{\wtwo}{\mbox{$W_{2}$}}
\newcommand{\xone}{\mbox{$x_{1}$}}
\newcommand{\xtwo}{\mbox{$x_{2}$}}
\newcommand{\Rxy}{\mbox{$R_{xy}$}}
\newcommand{\defn}{{\bf Definition}:\ }
\newcommand{\fig}[2]{\begin{center}Figure #1 of #2\end{center}}
\newcommand{\ie}{i.\ e.\ }
% text of the two Grand Theorems
\begin{document}
%
When you consider plane curves of degree $> 3$, it 
is no longer true that a nonconvex segment must contain either a 
singularity or flex.
Consider \arc{PQ}\ in the following curve:
\begin{center}
Figure 1 of grand.tex
\end{center}
It will still be true that by using the tangents at flexes and singularities
to create a cell partition of the curve, we can break the curve up into
convex segments.
However, now a cell 
may contain several distinct convex segments:
\begin{center}
Figure 2 of grand.tex
\end{center}
This did not occur in the cubic case because the curve could cross a wall
only once, at the flex/singularity defining it.
Now that we have multi-segment cells, when given a sort-point in such a cell,
we shall require a nontrivial computation to determine which segment
the sort-point lies on.
We shall also have to determine which wall-points pair up in a multi-segment
cell.
That is, given a wall-point where a segment enters the cell, find the 
associated wall-point where it leaves the cell (if it does indeed leave).
\section{Methods}
The following methods for solving the above problems concerning
multi-segment cells are elegant because they are applicable to curves of
arbitrary degree.
We first attack the problem of deciding which segment a curve-point 
(probably a sort-point) in the middle of a cell lies on.
Before the statement of the theorem, we need a definition and a lemma.
\\ \defn If x is not a singularity or a flex, then the {\em inside of x's
tangent} is the side that all of the curve in the neighbourhood of x lies on.
The inside includes the tangent, but the "strict inside" does not.
If x is a singularity, then x's tangent is the line that the branch of the 
curve corresponding to x (as a wall-point) is tangent to.
If x is a singularity or a flex, then the inside of x's tangent (which is
 a wall) \wrt a cell C is the inside of that wall \wrt C.
{\bf Notation}: $\seg{xy}_{in}$ is equivalent to $\seg{xy} \setminus \{x,y\}$.
Similarly, $\arc{xy}_{in}$ is equivalent to $\arc{xy} \setminus \{x,y\}$.
\begin{lemma}\nopagebreak
\label{alpha}
Let SEG be one of the convex segments of a cell of the cell partition of a
curve F.
Let x,y be two points of SEG, such that x and y are not wall-points on the
same wall.
Then
\begin{enumerate}
\item  \seg{xy}\ hits the curve F in an even number of points.
	(Points on \seg{xy}\ which are tangent to the curve are either
	counted as 0 or 2 points of intersection.  It is easier to
	ignore such points by counting them as 0 points.)
\item  As you travel from x to y along $\seg{xy}_{in}$,
	the count of curve points P that face x (i.e., x lies
	inside P's tangent) never exceeds the count of curve points
	that face y.
\item  The total number of curve points on $\seg{xy}_{in}$
        that face x equals the total
	number that face y.
\end{enumerate}
\begin{center}
Figure 4 of grand.tex
\end{center}
\end{lemma}
\proof
Consider the closed region \Rxy\ bounded by \seg{xy}\ and the convex segment
\arc{xy}.
\begin{center}
Figure 5 of grand.tex
\end{center}
$\Rxy \setminus \{x,y\}$ lies entirely in the interior of 
\arc{xy}'s cell (where the interior of a cell 
is the cell without its boundary):
\begin{quote}
	$\arc{xy} \setminus \{x,y\}$ lies in the interior of the cell
	$\Rightarrow \seg{xy} \setminus \{x,y\}$ lies in the interior 
	of the cell, since the cell is convex and x and y do not lie on
	the same wall $\Rightarrow$ the region bounded by \arc{xy}\ and 
	\seg{xy}\ lies in the interior of the cell, except possibly for x
	and y, again by cell convexity.
\end{quote}
Thus, $\Rxy \setminus \{x,y\}$ cannot contain 
\marginpar{(*)}
a flox or a singularity---floxes and singularities only lie on cell
boundaries of a cell partition.
Moreover, if the curve crosses into \Rxy\ at x (or y), then x must be a
singularity and the tangent of the branch that crosses into \Rxy\
at x will form a wall of the cell partition which will split \Rxy\ in
two, contradicting the fact that all of \Rxy\ lies in the same cell.
\hence The curve can only cross into (or out of) \Rxy\ through 
$\seg{xy} \setminus \{x,y\}$.
Moreover, if the curve enters \Rxy, then, since 
it cannot cycle back into itself 
within \Rxy\ (by Lemma~\ref{nochoice} [OF PAIR.TEX's PRELIMINARY LEMMATA])
and it cannot travel around this closed region forever
(by Lemma~\ref{nonbounded}), it must also leave \Rxy.
And (*) tells us that this point of departure must be distinct from the point
of entry.
\hence The intersections of the curve with $\seg{xy}_{in}$
pair up, into what we shall couples, so there must be an even
number of them.
However, we can say more than that to every point P where
the curve enters the region
there corresponds a point Q where the curve leaves \Rxy,
such that \arc{PQ} lies entirely inside \Rxy.
Since the region \Rxy\ lies within one cell, all of the segments within
\Rxy\ are convex, so P and Q lie on the same convex segment.
\hence The points of a couple face each other.
(That is, if P,Q are a couple, then P lies inside Q's tangent, and
vice versa.)
Let P,Q be a couple.
Assume \wlogg that P is closer to x than Q.
\begin{center}
Figure 6 of grand.tex
\end{center}
Then, P faces Q $\Rightarrow$ P faces y, and Q faces P $\Rightarrow$ Q
faces x.\footnote{Note that P is not a flex (by (*)), so
all of the curve in a neighbourhood of P is on one 
side of P's tangent.
Thus, since the curve crosses over \seg{xy} at P, P's tangent
is not \lyne{xy}, so P cannot face both x and y.  Similarly with Q.}
\hence The number of curve points on $\seg{xy}_{in}$ that face
x is equal to the number that face y.
Also, as you travel from x to y along \seg{xy}, 
you always meet the point of a couple that faces y before you meet the point
of the couple that faces x.
\hence As you travel from x to y along $\seg{xy}_{in}$
the count of curve
points that face y is always at least equal to the count of curve
points that face x.
\qed
\begin{theorem}[How to find the wall-points associated with a
point of the curve inside of a cell]\nopagebreak
\label{grandnonwall}
Let C be a cell of the cell partition of the curve F.\\
Let \wone, \wtwo\ be paired wall-points of C.
(That is, \arc{\ww}\ is a convex segment in C.)\\
Let $x \in \arc{\ww}_{in}$.\\
Let \mbox{S = \{ wall-points W of C $\mid$ }
\begin{enumerate}
	\item W lies on the strict inside of x's tangent;
	\item \seg{xW}\ hits the curve F in an even number of points (ignoring
points of tangency); and
	\item as you travel from x to W along $\seg{xW}_{in}$,
the count of curve points P that `face' x (i.e., x lies inside P's
tangent) never exceeds the count of curve points that `face' W.\}
\end{enumerate}
Let \xone, \xtwo\ be the two points at which x's tangent strikes the cell 
walls.
(If cell C is open, then place a temporary wall segment across its opening in
order to close it up: this ensures that both \xone\ and \xtwo\ exist.)
Finally, order the points of S from \xone\ to \xtwo.
That is, $S = \{S_{1},\ldots,S_{p}\}$ where $S_{i}$ is encountered before
$S_{i+1}$ as one travels from \xone\ to \xtwo\ along the cell walls.
(The direction of travel from \xone\ to \xtwo\ is clear, since
all of S lies on
one side of x's tangent.)
Then $\{S_{1},S_{p}\} = \{\wone,\wtwo\} $.
(Computationally speaking, if we create the convex hull of
$S \cup \{\xone,\xtwo\}$, then \xone\ will be connected to \xtwo\ and either
\wone\ or \wtwo\ (say \wone), and \xtwo\ will be connected to \xone\ and
\wtwo.)
\end{theorem}
As the statement of this theorem is rather daunting, 
an illustrative example is appropriate before the proof.
\begin{center}
Figure 3 of grand.tex
\end{center}
\proof
We start by showing that $\wone,\wtwo \in S$.
Since $x \in \wwa$ and \wwa\ is a convex segment, $\wwa \setminus \{x\}$
and, in particular, \wone\ and \wtwo\ lie on the inside of
x's tangent.
By Lemma~\ref{alpha} (letting SEG = \wwa, $x=x$, $y=W_{i}$), 
conditions 2) and 3) of S are satisfied by $W=\wone,\wtwo$.
Thus, $\wone,\wtwo \in S$.\\
Let \wwh\ be the perimeter of the cell from \wone\ to \wtwo, 
such that $\xone,\xtwo \not\in\wwh$ (i.e., \wwh\ does
not cross x's tangent).\footnote{Since \wwa\ is a convex segment, both \wone\ 
and \wtwo\ lie on the same 
side---the inside---of x's tangent, so \wwh\ is well-defined.}
\begin{center}
Figure 7 of grand.tex
\end{center}
We wish to show that $S \subset \wwh$.
Let $s \in S$.
Suppose that $\seg{xs} \setminus \{x,s\}$ strikes \wwa, say at y.
\begin{center}
Figure 8 of grand.tex
\end{center}
An application of Lemma~\ref{alpha} (with SEG = \wwa\ and x,y)
tells you that the number of curve points 
on $\seg{xy} \setminus \{x,y\}$ that face x is equal to the number that 
face y.
But y faces x, since x and y are on the same convex segment.
Thus, as you travel from x to s along $\seg{xs} \setminus \{x,s\}$,
the count of curve points that face x {\em does} exceed the count
of those that face s at some point, namely y.
This contradicts $s\in S$.
\hence $\seg{xs} \setminus \{x,s\}$ cannot strike \wwa.
This implies that s either lies on the outside of x's tangent or
inside the region bounded by \wwh\ and \wwa:\footnote{We are also
making use here of the fact that s cannot lie outside of the cell.}
\begin{center}
Figure 9 of grand.tex
\end{center}
But condition 1) of S ensures that s lies on the inside of
x's tangent, so either s lies on  x's tangent or s lies inside
the region bounded by \wwh\ and \wwa.
Hence, either $s = \xone,\xtwo$ or s lies on \wwh, since s is a wall-point.
\mbox{$.\raisebox{1.5ex}{.}.$}\ Indeed, once we have removed \xone and/or \xtwo,$S \subset \wwh$.\\
We know that $\xone,\xtwo \not\in \wwh$ (by definition), 
so clearly, in order to travel
along the cell perimeter from \xone\ or \xtwo\ to a point of $s \in \wwh$,
one must pass through \wone\ or \wtwo.
\hence $\{S_{1},S_{p}\} = \{\wone,\wtwo\}$.
\qed
\begin{theorem}[An extension of theorem~\ref{grandnonwall}]\nopagebreak
\label{extension}
Let C be a cell of the cell partition of the curve F.\\
Let \wone be the source of an open segment of C.\\
Let x lie on the interior of this open segment.\\
Let \mbox{S = \{ wall-points W of C $\mid$ }
\begin{enumerate}
	\item W lies on the strict inside of x's tangent;
	\item \seg{xW}\ hits the curve F in an even number of points (ignoring
points of tangency); and
	\item as you travel from x to W along $\seg{xW}_{in}$,
the count of curve points P that `face' x (i.e., x lies inside P's
tangent) never exceeds the count of curve points that `face' W.\}
\end{enumerate}
Let \xone, \xtwo\ be the two points at which x's tangent strikes the cell 
walls.
Finally, order the points of S from \xone\ to \xtwo.
Then $\wone \in \{S_{1},S_{p}\} $.
\end{theorem}
\proof
See the proof of Theorem~\ref{grandnonwall}.
Let \wtwo\ be the appropriate tangent intersection \xone\ or \xtwo.
We must only show that there are no points of S between \xone\ and \wone.
The proof is exactly like that of Theorem~\ref{grandnonwall}.
\qed
We can use a similar technique to solve the equally important problem of
pairing wallpoints in a multi-segment cell.
\begin{theorem}[How to find the partner of a flox/incidental wallpoint]
\label{newgrand}
Let F be a curve with no cusps of the second kind (Hilbert, p.173).\\
(This isn't quite enough: we also need to restrict tacnodes and the like.)
Let C be a cell of the cell partition of F.\\
Let SEG be a convex segment in C.\\
Let x be a wallpoint of SEG such that x is a flox, incidental point (or 
singularity with one distinct tangent such that
the curve crosses the tangent at the singularity?).
Let \mbox{S = \{ wall-points W of C $\mid$ }
\begin{enumerate}
	\item W lies on the strict inside of x's tangent;
        \item x lies on the strict inside of W's tangent;
	\item $\seg{xW}_{in}$ hits the curve F in an even number of points (ignoring
points of tangency); 
	\item as you travel from x to W along $\seg{xW}_{in}$,
the count of curve points P that `face' x (i.e., x lies inside P's
tangent) never exceeds the count of curve points that `face' W; and
        \item the total number of curve points on $\seg{xW}_{in}$ that face x\\
= the total number of curve points on $\seg{xW}_{in}$ that face W.\}
\end{enumerate}
Then
\begin{enumerate}
        \item[(a)] if x has a partner wallpoint on a 
different wall, then S contains this partner;
        \item[(b)] if x either has a partner on the 
same wall or no partner at all, then $S=\emptyset$.
\end{enumerate}
\end{theorem}
As the statement of this theorem is rather daunting, 
an illustrative example is appropriate before the proof.
\begin{center}
Figure 14 of grand.tex
\end{center}
\proof
Let s be a point that satisfies conditions (1),(3),(4), and (5) of set S.
Suppose that s is not a wallpoint of SEG.
We shall show that s does not satisfy condition (2).
\begin{enumerate}
      \item[Case 1.] Suppose that s has a partner t.\\
Since s is not a wallpoint of SEG, neither is t.
Suppose, for purposes of contradiction, that x lies strictly inside
of s's tangent.
Thus, $\seg{xs}_{in}$ strikes \arc{st} (if it didn't, then SEG would have
to cross \arc{st}\ in order to reach x), say at $\lambda$.
\begin{center}
Figure 15 of grand.tex
\end{center}
An application of Lemma~\ref{alpha} (with SEG = \arc{st}, x = s, y = $\lambda$)
tells us that the number of curve points on $\seg{\lambda s}_{in}$ that face
s is equal to the number that face $\lambda$ (and thus x).
\hence the number on $\seg{\lambda s}_{in} \cup \{\lambda\}$ that face s exceeds
the number that face x, since $\lambda$ faces s.
\hence By condition (5), the number of curve points on $\seg{x\lambda}_{in}$
that face s is less than the number that face x, which contradicts 
condition (4).
\hence x does not lie strictly inside of s's tangent.
\hence $s \not\in S$.
     \item[Case 2.] Suppose that s has no partner (\ie, it is the source of an
open segment).
Again suppose that x lies strictly inside s's tangent.
The open side of the cell must lie inside s's tangent, since s's curve segment
is convex and it must go off to infinity through that open side.
Thus, s's open segment must cross \lyne{xs}\ in order to get to the open side.
Thus, since the open segment cannot cross SEG,
$\seg{xs}_{in}$ must strike s's open curve segment.
Proceed as in case 1 to yield the conclusion that $s \not\in S$.
\end{enumerate}
$.\raisebox {1.5ex}{.}. S \subset$ {the wallpoints of SEG}
If x is not a wallpoint, then the convexity of SEG ensures that
conditions (1) and (2), and Lemma~\ref{alpha} ensures that 
conditions (3), (4), and (5), are satisfied by the wallpoints of SEG.
If x is a wallpoint with a partner on a different wall, then similarly
all five conditions are satisfied by the partner, but of course x does not
satisfy conditions (1) or (2).
If x is a wallpoint with a partner on the same wall, then condition (1) and
(2) are violated by the partner, so S is empty.
If x is a wallpoint with no partner, then S is again empty because x violates
conditions (1) and (2).
\qed
\begin{theorem}[How to find the partner of a singular wallpoint]\nopagebreak
\label{grandwall}
Let \wone,\wtwo\ be paired wall-points on {\em different} walls of a cell C,
such that \wone\ is a singularity with distinct tangents.
Let \mbox{S = \{ wall-points W of C $\mid$ }
pt(W)$ = $pt(\wone)  OR
\begin{enumerate}
	\item W lies on the strict inside of \wone's tangent;
	\item \seg{\wone W} hits the curve F in an
even number of points (ignoring
points of tangency); and
	\item as you travel from \wone\ to W 
along $\seg{\wone W} \setminus \{\wone,W\}$, the count of curve points
that face \wone\ never exceeds the count of curve points that face W.\}
\end{enumerate}
pt(\wone) will be an endpoint of two segments
of the cell: let \xone be the other endpoint of the segment that \wone's
branch is tangent to.
\fig{12}{grand.tex}
If $S = \emptyset$, then \wone's partner is the other wallpoint of the
singularity (\ie, $pt(\wtwo) = pt(\wone)$).
Otherwise, if we create the convex hull of $S \cup \{\xone,\wone\}$,
then \xone\ will be connected to \wone\ and \wtwo.
That is, \wtwo\ is the first member of S encountered as one travels
around the cell from \wone\ in the direction that the curve leaves \wone.
\fig{10A}{grand.tex}
\end{theorem}
\proof
Since \wone\ and \wtwo\ lie on different walls of a convex cell,
\wtwo\ lies on the strict inside of \wone's tangent.
By Lemma~\ref{alpha} (letting SEG = \wwa, $x = \wone$, $y=\wtwo$),
conditions 2) and 3) of S are satisfied by $W = \wtwo$.
Thus, $\wtwo \in S $.\\
Let \wwh\ be the perimeter of the cell from \wone\ to \wtwo
such that $\xone \not\in \wwh$.
(This is well defined since $\xone \neq \wone,\wtwo$: \wtwo is not
on \wone's wall, by assumption, and is not on \wone's tangent, by convexity,
so $\wtwo \neq \xone$.)
We wish to show that $S \subset \wwh$.
Let $s\in S$.
Suppose that $\seg{\wone s} \setminus \{\wone,s\}$ strikes \wwa, say at y.
An application of Lemma~\ref{alpha} (with SEG = \wwa, $x = \wone$, $y = y$)
tells you
that the number of curve points
on  $\seg{\wone y} \setminus \{\wone,y\}$ 
that face \wone\ is equal to the number that face y.
But y faces \wone, since $y \in \wwa$ implies that y and \wone\ lie on the
same convex segment.
Thus, as you travel from \wone\ to s
along  $\seg{\wone s} \setminus \{\wone,s\}$, the count of curve
points that face \wone\ {\em does} exceed the count of those that face s
at some point, y.
This contradicts $s\in S$.
\hence $\seg{\wone s} \setminus \{\wone,s\}$ cannot strike \wwa.
\hence Using a picture as motivation rather than a formal proof,
\marginpar{Add rigour.}
$s \in \wwh$. (Any line segment from \wone\ to a point 
of the picture's hashed region 
will strike \wwa.)
\begin{center}
Figure 11 of grand.tex
\end{center}
Thus, $S \subset \wwh$.\\
$\xone \not\in \wwh$ (by definition of \wwh), so in order to travel 
along the cell perimeter from \xone\ to a point of S on \wwh,
one must pass through \wone\ or \wtwo.
\hence \xone\ is connected to \wone\ and \wtwo\ in the convex hull of 
$S \cup \{\xone,\wone\} $.
\qed
\begin{theorem}
Let \wone,\wtwo\ be paired wallpoints on the same wall of a cell C.
Let \mbox{S = \{ wall-points W of C $\mid$ }
\begin{enumerate}
	\item W lies on the strict inside of \wone's tangent;
	\item \seg{\wone W} hits the curve F in an
even number of points (ignoring
points of tangency); and
	\item as you travel from \wone\ to W 
along $\seg{\wone W} \setminus \{\wone,W\}$, the count of curve points
that face \wone\ never exceeds the count of curve points that face W.\}
\end{enumerate}
Then $S=\emptyset$.
\end{theorem}
\proof
By the same reasoning as used in the proof of Theorem~\ref{grandwall},
$S \subset \wwh$.
However, if \wone\ and \wtwo\ lie on the same wall, then \wwh\ is a subsegment
of this wall.
But condition (1) of S ensures that no point of S lies on \wone's wall.
\hence $S \cap \wwh = \emptyset$
\hence $S = \emptyset$
\qed
\begin{theorem}[Pairing wallpoints on the same wall]\nopagebreak
Let $\wone \neq \wtwo$ be paired wallpoints on the same wall of cell C.\\
(Note: \wone cannot be a singularity, since the curve cannot loop back to pair
with \wtwo on the same wall, by convexity:
\fig{13}{grand.tex}
Let \mbox{S = \{ wall-points W of C on \wone's wall $\mid$ }
\begin{enumerate}
	\item W lies inside of \wone's tangent (if \wone\ is incidental) or
{\em not} in the curve direction from \wone \wrt C (if \wone\ is a flox)
        \item \wone lies inside of W's tangent (if \wone\ is incidental) or
{\em not} in the curve direction from W \wrt C (if W is a flox), and
        \item there are an even number of wallpoints between W and \wone \}
\end{enumerate}
Then $\wtwo \in S$ and \wtwo is the closest element of S to \wone.
\end{theorem}
\proof
Write this up!
\qed
\begin{itemize}
\item Implicit in all of the above is the basic assumption that we are
dealing with a single connected component X of the curve.
Thus, for example, `the curve points on $\seg{xy} \setminus \{x,y\}$'
is to be read `the points on $\seg{xy} \setminus \{x,y\}$ which
intersect the connected component X of the curve'.
Certainly, many curves will only have a single component.
When dealing with multi-component curves, hopefully either the other
components will not interfere or points of undesired components will be
able to be readily distinguished.
We have done some preliminary work on segregation of one component from
others, and have even solved the problem for cubics.
The multi-component problem is not peculiar to the convex-segment method.
If one opts for parameterizing such a curve, one will require a different
parameterization for each component.
\end{itemize}
\subsection{A Warning}
Unfortunately, elegance does not always translate into efficiency.
Theorems~\ref{grandnonwall} and \ref{grandwall} are elegant but they can
probably be beaten when efficiency becomes the overriding concern.
In being general, they may not take advantage of local idiosyncracies of
particular degrees.
For example, in the case of quartic curves, Theorem~\ref{grandwall}'s 
method is probably (much) slower than another method 
\marginpar{I hope to clinch and write up this method soon.}
that decides \wone's partner simply on the basis of where 
the second intersection
of \wone's wall (with the curve) lies.
And we can probably improve Theorem~\ref{grandwall}'s method 
for quartics by making 
use of a result that tells us that a wall-point of a quartic curve 
always pairs up with a neighbouring wall-point ; unless finding those
neighbours takes more time than considering all wall-points.
However, since it is infeasible  to examine each degree for idiosyncracies,
and since the degree-specific methods may only beat the general methods by
an insignificant margin (especially if multi-segment cells are rare),
the general methods are still very useful.
\end{document}
