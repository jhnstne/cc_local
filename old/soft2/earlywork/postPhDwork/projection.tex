%   LOOKS LIKE GARBAGE
\newif\ifFull
\Fullfalse
% \Fulltrue
\documentstyle[11pt]{article} 
%
\input{macros}
%\include{setup}
%
\newcommand{\px}{\mbox{$p_{x}$}}
\newcommand{\py}{\mbox{$p_{y}$}}
\newcommand{\pz}{\mbox{$p_{z}$}}
\newcommand{\pxyz}{(\px,\py,\pz)}
\newcommand{\projinvF}{\mbox{proj$^{-1}$(F)}}
% Page format
%
\SingleSpace
\setlength{\oddsidemargin}{0pt}
\setlength{\evensidemargin}{0pt}
\setlength{\headsep}{0pt}
\setlength{\topmargin}{0pt}
\setlength{\textheight}{8.75in}
\setlength{\textwidth}{6.5in}
%
\begin{document}
\label{tangplane}
Let C be a space curve, ($x_{0},y_{0},z_{0}) \in C$,
P a source of projection with \pz $\neq$ 0, $\pz \neq z_{0}$.
Consider the plane containing P and the tangent line $T_{1}$ to the
plane curve proj(C)
at proj($x_{0},y_{0},z_{0}$).
This plane is tangent to C at ($x_{0},y_{0},z_{0}$).
\end{corollary}
\begin{proof}\nopagebreak
We must show that the plane contains the tangent line $T_{2}$ to 
($x_{0},y_{0},z_{0}$).
\begin{center}
FIGURE 1
\end{center}
By the very nature of projection, any point of $T_{2}$ lies on the line 
connecting its projection to P.
But, by the previous lemma, its projection lies on $T_{1}$, and any line
between a point of $T_{1}$ and P lies in our plane.
Therefore, every point of $T_{2}$ lies on the plane.
\end{proof}
%
{\bf Definition:}
A plane curve [space curve] $C_{1}$ 
{\em cuts through} another plane curve [a plane] $C_{2}$ if, when we
consider $C_{2}$ as cutting the plane [3-space] into two regions
(inside vs.\ outside, or left vs.\ right), $C_{1}$ lies in both
regions in any $\epsilon$-nhood of the point of intersection.
%
\begin{lemma}
\label{planarcutsthru}
["Lemma 7" of cubic.tex: replace there]
A line L cuts through a plane curve C at a nonsingular point P
iff the multiplicity of intersection of L with C at P is odd.
\end{lemma}
\begin{proof}\nopagebreak
Assume WLOG that L is the y-axis and P is the origin. (Rigid
transformations---rotations and translations---do not affect the 
multiplicity or singularity of the curve at an intersection.)
Let $N_{\epsilon}(0)$ = {$c\in C |$dist$_{C}(c,(0,0)) < \epsilon$}
where dist$_{C}(c_{1},c_{2})$ is the length of the curve C between
two points $c_{1},\ c_{2}$ on it (already defined in cubic.tex).
That is, $N_{\epsilon}(0)$ is a small part of the curve near the
origin.  We claim that $\exists \epsilon > 0$ such that $N_{\epsilon}(0)$
can be represented by a function y = g(x) (i.e., $\not\exists
c_{1},\ c_{2} \in N_{\epsilon}(0)$ with the same abscissa).
The directed tangent at the origin is ($\pm$1,0), depending upon
the direction that you are travelling.  
The tangent to an algebraic curve
\marginpar{Prove! This proof is also needed in Lemma 10 of cubic.tex}
changes continuously except at singularities (Lemma~\ref{tangcont} of
lemma9.tex).
Thus, it takes some positive distance from the nonsingular origin for
the directed tangent vector of a point to reverse the sign of its
abscissa from 1 to $-\delta$ (or -1 to $+\delta$).
In other words, $\exists \epsilon > 0$ such that the curve does not
change directions (or become constant) with respect to x for an
$\epsilon$ distance along the curve.
\hence $\exists \epsilon > 0$ such that $\not\exists c_{1},\ c_{2} \in
N_{\epsilon}(0)$ with the same abscissa.
Within this neighbourhood, 
\marginpar{We should show that g(x) is a polynomial, since I think
that you need this in order to take the Taylor series as we have.}
the curve can be represented by a function,
say y = g(x).\\ \\
Expand g(x) into a Taylor series:
\begin{equation}
\label{eq:Taylor}
g(x) = g(0) + g'(0)x + g''(0)x^2 + \ldots  
\end{equation}
Since $y-g(x) = 0$ represents the curve C near to the origin
and $x=t,\ y=0$ is a parameterization of L, the intersections of L
with C near to the origin are associated with the roots of $0-g(t)=0$
(substituting the parameterization into the curve equation).
In particular, using (\ref{eq:Taylor}) above, we can see that the
multiplicity of the intersection of L with C at the origin is
min ${i|g^{(i)}(0) \neq 0}$.
Now, we know that g(x) changes sign as x changes sign $\Leftrightarrow$
C cuts through L, the y-axis, at P, the origin.
Also, as x $\rightarrow$ 0 the lowest order term in (\ref{eq:Taylor})
dominates all other terms.
Thus, g(x) changes sign as x changes sign $\Leftrightarrow$ the lowest
order term in (\ref{eq:Taylor}) is odd.
\hence C cuts through L at P $\Leftrightarrow$ the lowest order term
in (\ref{eq:Taylor}) is odd.
\hence C cuts through L at P $\Leftrightarrow$ the multiplicity
of intersection of L with C at P is odd.
\end{proof}
\begin{conjecture}
[A 3D version of Lemma~\ref{planarcutsthru}]
\label{spacecutsthru}
A plane T cuts through a space curve C at a nonsingular point P iff
the multiplicity of intersection of T with C at P is odd.
\end{conjecture}
\begin{conjecture}
[A simple inclusion lemma]
\label{inclusion}
Let L be a line in the plane P.
Suppose that a curve C is tangent to L at q.
Then the multiplicity of intersection of L with C at q is equal
to the multiplicity of intersection of P with C at q.
\end{conjecture}
{\bf Proof}:\nopagebreak \\
\begin{conjecture}
The projection of the intersection of two quadric surfaces does not
contain a finite flex (i.e., a flex at a point of infinity is still allowed).
\end{conjecture}
\begin{proof}\nopagebreak
(Note:  By projection we mean central projection.)
[We must prove that the point left out because the source of
projection does not map
anywhere is not a flex. This may well be a flex at infinity!]
[Note that proj$^{-1}$(F) cannot be the source of projection since
the projection of this point is not defined (or is a point at infinity).]
Let the quadrics be Q and $Q'$, $C = Q \cap Q'$.
Suppose, for the purposes of contradiction, that the projection of
C does have a finite flex F. \\
Using Corollary~\ref{tangplane}, let T be a plane tangent to the space
curve C at proj$^{-1}$(F) such that T contains the tangent line to the 
plane curve proj(C) at F.
(proj$^{-1}$(F) is unique since F, as a flex, is nonsingular.)
Note that, by definition, the tangent line to proj$^{-1}$(F) hits C
at least twice at proj${^-1}$(F); and so, by Lemma~\ref{inclusion},
C hits T at least twice at proj$^{-1}$(F). \\
By Lemma~\ref{planarcutsthru}, proj(C) cuts through the tangent to F.
Hence, C cuts through T at proj$^{-1}$(F).
[We have almost proved this (see following).  We have a problem with
showing that, for some nhood of proj$^{-1}$(F), a point either always
flips to the opposite side or always stays on the same side of the plane
T as you project.  I think that it hinges on looking at nhoods of
proj$^{-1}$(F) that do not contain the source.  I think that
these will satisfy this property.]
(Proof:Case 1: Suppose that the source of projection lies in between
F and proj$^{-1}$(F), recalling that these points are collinear on T.
Then a point flips over to the other side of the plane T when it is projected???
Also, an $\epsilon$-neighbourhood of proj$^{-1}$(F) is projected to a
$\delta$-neighbourhood of F, since projection is a continuous map.
Since points of any $\epsilon$-neighbourhood of F are on both sides of T,
points of any $\epsilon$-neighbourhood of proj$^{-1}$(F) are on both sides.)
\hence By Lemma~\ref{spacecutsthru}, C hits T an odd number of times at
\projinvF.
\hence C hits T at least three times at \projinvF.
\hence By Lemma~\ref{inclusion}, the tangent line to C at \projinvF
hits the curve at least three times there.
\hence This tangent line hits each quadric at least three times, so,
by an extension of Bezout's Theorem, it is a component of both quadrics.
\hence The tangent line at \projinvF is part of the intersection curve. \\
But we saw above that the intersection curve C cuts through the tangent plane,
and thus the tangent line, at \projinvF.
\hence \projinvF is a singular point of C.
Moreover, the projection of this singular point is a singular point.
(We know that the projection of the tangent line is in the tangent plane.
The projection of the space curve is not.
Hence, the two components that meet at \projinvF do not become the same
component on the plane.
Hence, F = proj(\projinvF) is a singular point too.)
This is a contradiction, since a flex is a nonsingular point.
\hence The projected curve does not contain a flex.
\end{proof}
\begin{corollary}
The projection of the intersection of two quadrics can never be a cubic
of the nonsingular ($y^{2}=x^{3}+ax^{2}+bx$) equivalence class, or a
nodal or cuspidal cubic with a finite flex (like $y=x^3$).
\end{corollary}
\begin{proof}
These cubics have finite flexes.
\end{proof}
We have ruled out flexes.  How about singularities?  The next lemma
provides a necessary condition for their presence.
\begin{lemma}
[A characterization of intersection curves that yield singularities
under projection]
\label{lem-sings}
Let C be the intersection of two quadrics, C nonplanar.
If the projection of C contains a singularity, then C must be reducible
into two components: a line and a cubic space curve.
\end{lemma}
\begin{proof}
Let $P\in C$ be the source of projection.
Suppose that proj(C) has a singularity S.
Consider the line L from the source to the singularity.
Since S is a singularity, at least two points map to it.
Therefore, counting P, L must contain at least three points of C,
and this makes it a component of C (by two applications of Bezout).
By \cite{CS655}, since C is nonplanar and reducible, it must consist
of a degree 1 and a degree 3 component (i.e., a line and a cubic curve).
\end{proof}
In light of the previous two lemmas, I claim that it is indeed
possible to sort the projection of a quadric intersection curve by the
convex segment method, even though we do not have the equation of this
plane curve but only the equation of its parent, the intersection curve.
First, consider projections that are cubics.
Since there are no flexes to find, we can use the convex segment method
as long as we can find the projection's singularities and the
tangents at these singularities.  
(The projection must be a singular cubic, since nonsingular 
cubics contain flexes.)
Consider finding a singularity.
The projection can have at most one point of singularity, since 
singularities correspond to line components of the space curve (see 
the proof of Lemma~\ref{lem-sings})
Therefore, you can find the singularity by finding the (unique) line 
that the two quadrics have in common.  
In turn, this can be done by solving the system:
\[ V \cdot (grad\ f)(P+tV) = 0 \\
   V \cdot (grad\ g)(P+tV) = 0 \]
for P and V (6 eqns, 6 unknowns), these being the equations that
define a ruled line of a surface.
There will remain a degree of freedom for P (anywhere along $P+tV$).
This will yield the line.  You must then find the point on this line
where the cubic component hits it.
This is the inverse image of the projection's singularity.

The tangents of the singularity are even more easily found,
since projection preserves tangents.
Simply find the tangent vectors at the space curve's singularity
and project these vectors to the projection plane.

Next, consider projections that are quadratics.

\begin{conjecture}
For fixed m and n, the set of degree mn curves created by intersecting
a degree m surface with a degree n surface is a limited subclass of the
set of all degree mn curves.
\end{conjecture}
\begin{proof}
"For example: the intersection of two quadrics does not include planar
quartics"
\end{proof}
%
\begin{conjecture}
[I have no feelings either way on the truth of this conjecture. It is
interesting if it is true and equally interesting if it is not.]
The union over all ${m,n|m+n=k}$ of the intersection curves of a degree m
and a degree n surface is equivalent to the set of all degree k curves.
(So you can get any curve by intersecting some pair of surfaces.)
\end{conjecture}
\end{document}

