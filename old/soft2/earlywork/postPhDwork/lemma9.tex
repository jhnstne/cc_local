\documentstyle [titlepage]{article}
%
\newtheorem{lemma}{Lemma}
\newtheorem{theorem}{Theorem}
\newtheorem{corollary}{Corollary}
\newtheorem{conjecture}{Conjecture}
%
\newcommand{\arc}[1]{\mbox{$\stackrel{\frown}{#1}$}}
\newcommand{\lyne}[1]{\mbox{$\stackrel{\leftrightarrow}{#1}$}}
\newcommand{\ray}[1]{\mbox{$\vec{#1}$}}
\newcommand{\seg}[1]{\mbox{$\overline{#1}$}}
\newcommand{\hence}{\\ \mbox{$.\raisebox{1.5ex}{.}.$}\ } % three-dot therefore
                                             % symbol is desired
\newcommand{\qed}{\nopagebreak\begin{flushright}\mbox{$\Box$}\end{flushright}}
                                       % put a box on the
                                       % righthand side of the page
\newcommand{\se}{\mbox{$_{\epsilon}$}}  % subscript epsilon
\newcommand{\proof}{{\bf Proof}:\nopagebreak\\}
\newcommand{\wlogg}{\mbox{w.\ l.\ o.\ g.\ }}
\newcommand{\wrt}{\mbox{w.\ r.\ t.\ }}
\newcommand{\st}{\mbox{s.\ t.\ }}
\newcommand{\ie}{\mbox{i.\ e.\ }}
\newcommand{\pq}{\arc{PQ}}
\newcommand{\pqi}{\mbox{$\pq_{in}$}}
\newcommand{\tx}{\mbox{$T_{x}$}}
\newcommand{\e}{\mbox{$\epsilon$}}
\newcommand{\eo}{\mbox{$\epsilon_{1}$}}   % note:\e1 gets confused with \e
\newcommand{\et}{\mbox{$\epsilon_{2}$}}   % \e2 gets confused with \e
\newcommand{\xeo}{\mbox{$x_{\epsilon,1}$}}
\newcommand{\xet}{\mbox{$x_{\epsilon,2}$}}
\newcommand{\xya}{\arc{xy}}
\newcommand{\lep}{\mbox{$L_{\epsilon}$}}  % note: \le is already taken
\newcommand{\xpe}{\mbox{$x_{+\epsilon}$}}
\newcommand{\xme}{\mbox{$x_{-\epsilon}$}}
\newcommand{\ye}{\mbox{$y_{\epsilon}$}}
\newcommand{\xo}{\mbox{$x_{1}$}}
\newcommand{\xt}{\mbox{$x_{2}$}}
\newcommand{\xth}{\mbox{$x_{3}$}}
\newcommand{\yo}{\mbox{$y_{1}$}}
\newcommand{\xotha}{\arc{\xo\xth}}
\newcommand{\fxo}{\mbox{$F_{x1}$}}
\newcommand{\fig}[2]{\begin{center}Figure #1\ of #2 \end{center}}
\newcommand{\defn}{{\bf Definition}:\ }
%
\begin{document}
% 
\begin{lemma}\nopagebreak
\label{tangcont}
The tangent of an algebraic curve changes continuously, except possibly at 
singularities (even at singularities
if we consider the correct tangent (?)).
\end{lemma}
\proof\marginpar{Prove this.}
? (Note: put this lemma early on in the presentation, since, for instance,
Lemma~\ref{planarcutsthru} of Cosmosort.tex uses it.)
\qed
We need a technical lemma about nonconvex segments for the next theorem.\\
{\bf Notation:}\ \pqi\ is shorthand for $\pq \setminus \{P,Q\}$.
\begin{lemma}\nopagebreak
\label{threecross}
Let \pq\ be a nonconvex segment of a curve F such that
\pqi\ contains no singularities or floxes.
Then one can find a line L such that \pqi\ {\em crosses} L at three
distinct points.
\end{lemma}
\proof
(Warning: this is a grundgy proof.)
\pq\ is not convex and \pqi\ is nonsingular
$\Rightarrow \exists\ x \in \pqi$
\st x's tangent, \tx, strikes \pq\ at another point.
Note that x is not a flox or singularity.\\
Let $y \neq x$ be an intersection of \tx\ with \pq\ \st \xya\ lies strictly
inside \tx\ (\ie, y is the `closest' intersection to x).
$\forall\ \e > 0$, let \lep\ be the line that (i) is parallel to \tx,
(ii) lies inside \tx, and (iii) is at a distance of \e\ from \tx.
We will show that, for some $\e > 0$, \lep\ crosses the curve at least three
times.\\
$\forall\ \e > 0$, if $\lep \cap \xya \neq \emptyset$, then let \xpe\ be the
intersection of \lep\ with \xya\ \st $\arc{x\xpe} \cap \lep = \{\xpe\}$
(\ie, \xpe\ is the `first' intersection of \xya\ with \lep\ after x); and, if 
$\lep \cap (F \setminus \xya) \neq \emptyset$, then 
let \xme\ be the intersection
of \lep\ with $F \setminus \xya$ \st $\arc{x\xme} \cap \lep = \{\xme\}$.
\begin{center}
Figure 4 of lemma9.tex
\end{center}
The points of the curve
\marginpar{    (*)}
F which have tangents parallel to \tx\ are isolated, otherwise some line
parallel to \tx\ would strike the curve in an infinite number of points and
thus be a reducible component of the irreducible curve F.
Let 
\[ \mbox{dist} = \left\{ \begin{array}{ll}
	\mbox{$+\infty$}  &\mbox{if there are no points P of F} \\
                          &\mbox{with tangents parallel to \tx} \\
			  &\mbox{\st P lies strictly inside \tx}\\
	\mbox{min\{$\e >0\ |$\lep\ is tangent to the curve\}}&\mbox{otherwise}
	\end{array}
	\right.  \]
By (*), dist is well-defined.\\
Since x is not a flox or singularity, the curve does not cross \tx\ at x
(by Lemma~\ref{planarcutsthru}), so there is a neighbourhood of x that lies
inside \tx.
By (*), the curve must actually lie {\em strictly} inside \tx\ in some
neighbourhood of x, and $\exists\ \eo > 0$ \st $\forall\ \e < \eo$,
\xpe\ and \xme\ exist.
Thus, \mbox{$\forall\ 0 < \e < $ min\{\eo,dist\} }, \xpe\ and \xme\ exist
{\em and} \lep\ crosses the curve at \xpe\ and \xme\ (\ie, \lep\ is not
tangent to \xpe\ or \xme).
We have two of our desired three crossings.\\
Now consider the intersections of \lep\ with \xya\ close to y.\\
$\forall\ \e > 0$, if $\lep \cap \xya \neq \emptyset$, then let \ye\ be the
intersection of \lep\ with \xya\ \st $\arc{y\ye} \cap \lep = \{\ye\}$.
\xya\ lies strictly inside \tx, by the choice of y, so
$\exists\ \et > 0$ \st \ye\ exists $\forall\ \e < \et$.
Thus, \mbox{$\forall\ 0 < \e <$ min \{\et,dist\} }, \ye\ exists {\em and}
\lep\ crosses the curve at \ye.
\hence \mbox{$\forall\ 0 < \e <$ min \{\eo,\et,dist\} }, \xpe, \xme\ and
\ye\ exist and \lep\ crosses the curve at these points.\\
Note that if \xya\ crosses \lep\ once, then it must do so again, in order
to return to \tx.
Thus, if \lep\ crosses the curve at \xpe\ and \ye, then $\xpe \neq \ye$,
because \xpe\ is the closest crossing to x, while \ye\ is the closest
crossing to y.
Clearly, $\xme \neq \xpe$ and $\xme \neq \ye$.
Thus, \mbox{$\forall\ 0 < \e <$ min \{\eo,\et,dist\} }, \lep\ crosses F at 
three {\em distinct} points.\\
$\xpe, \ye \in \xya_{in} \subset \pqi$, but it is possible 
that $\xme \not\in \pq$.
However, since $x \neq P$ or Q, if we choose \e\ small enough, \xme\ will
be close enough to x that it will remain on \pqi.
\hence $\exists\ \e > 0$ \st line \lep\ crosses \pqi\ at 
three distinct points.
\qed
We cannot prove a result as strong as Lemma~\ref{flexorsi} for curves of
degree $\geq 4$. 
The following curve provides a counterexample:
\begin{center}
Figure 5 of lemma9.tex
\marginpar{Find the equation of this curve.}
\end{center}
The following result will show that the boundaries of convex segments must
now include {\em all} intersections of the curve with the tangents at floxes
and singularities.
Lemma~\ref{flexorsi} is a special case of this result, since the tangents
at floxes and singularities of cubic curves only intersect the curve at those
floxes and singularities (by Bezout's Theorem).\\
\defn A singularity is {\em ordinary} if its tangents are distinct.
A singularity is {\em extraordinary} if it is not ordinary.
For example, a node is ordinary while a cusp is not.
Notice that as the curve passes through an ordinary singularity,
it will cross one of the singularity's tangents;
however, the curve need not cross the tangent of an extraordinary singularity,
as witnessed by the tacnode and the rhampoid cusp:
\fig{5A}{lemma9.tex}
\defn A {\em cell partition} of a curve F is the partitioning of the plane
into convex polygons by the tangents of F's floxes and singularities.
\begin{theorem}\nopagebreak
Let S be a nonconvex segment of the curve F.
If S contains no extraordinary singularities, then some wall
of the cell partition of F will cross $S_{in}$.
If S contains an extraordinary singularity, then some wall of the 
cell partition of F will either touch or cross $S_{in}$.
\end{theorem}
\proof
Let $S=\pq$ be a nonconvex segment of F.
Assume \wlogg that P and Q lie in the same cell (otherwise, clearly some
wall crosses \pq).
Call the cell C.
We shall show that \pqi\ contains a flox,
singularity, or incidental point.
Then the wall through this point will cross \pqi\ if the point
is not an extraordinary singularity; and the wall
will touch \pqi\ otherwise.\\
Assume \wlogg that \pqi\  does not contain a flox or 
singularity.
By Lemma~\ref{threecross}, $\exists$ line L \st \pqi\ crosses L at at least
three distinct points.
Choose three points \xo,\xt,\xth\ at which \pqi\ crosses L and 
\st $\xt \in \xotha$ and
the only points of $\xotha \cap L$ are \xo,\xt,\xth\ (\ie, \xo,\xt,\xth\ 
are as close together as possible on \pqi).
Since $\xotha \subset \pq$ does not contain a flox, it 
does not change its direction
of curvature, and so, in order to strike L three times,
\xotha\ must spiral around:
\begin{center}
Figure 7 of lemma9.tex
\end{center}
or \xotha\ must be a straight line segment.
If \xotha\ is a line segment, then this line is a component of the curve
since it intersects the curve at an infinite number of points
(Lemma~\ref{Bezout}).
But then the curve (which is nonconvex and thus not simply 
a line) is reducible, a 
contradiction.\footnote{Remember that at the very beginning, we made the
implicit assumption that `curve' means `irreducible curve'.}
\hence \xotha\ spirals around exactly as in one of the above pictures.
We can assume \wlogg that $\xo \in \seg{\xt\xth}$
(by exchanging the names of \xo\ and \xth, if necessary).
Let R be the closed region bounded by \xotha\ and \seg{\xo\xth}:
\fig{8}{lemma9.tex}
We wish to show that R contains a flox or a singularity.
If we can do this, then we are finished because a line through any point of R
must cross \xotha\ at least once.
Thus, in particular, the tangent (which will be a wall) of a flox (or
 a singularity) inside R would cross $\xotha \subset \pqi$, creating an
incidental point.\\
Let 
\[ \fxo = \left\{ \begin{array}{ll}
	\mbox{$F \setminus \xotha$}  &  \mbox{if the curve F is closed}\\ \\
	\mbox{the segment of $F\setminus\xotha$}   &\mbox{if F is open}\\
        \mbox{\ \ \ with endpoint \xo}
	\end{array}
	\right.   \]
Suppose that F is open, so that \fxo\ is an infinite segment.
As \fxo\ leaves \xo\ it lies in the closed region R, but 
it cannot remain there (by Lemma~\ref{nonbounded}).
Since \xotha\ does not contain a singularity (by assumption),
\fxo\ must leave R through \seg{\xo\xth}.\\
If F is closed, then \fxo\ is inside of R as it leaves \xo\ and outside 
of R as it leaves \xth.
Thus, again because \xotha\ is nonsingular,
\fxo\ must pass through \seg{\xo\xth} as it travels from \xo\ to \xth.
\hence \fxo\ crosses \seg{\xo\xth}.\\
Note that \xth\ lies outside of \xo's tangent (since $(i)$ \xo\ 
is in between \xt\
and \xth\ on \seg{\xt\xth}, $(ii)$ \xo's tangent 
crosses the line $\lyne{\xo\xth} = L$, and
$(iii) \xt \in \xotha$ ---\ie, \xt\ is met before \xth\ as you spiral from \xo).
Let \yo\ be the point where \xo's tangent crosses $\xotha \setminus \{\xo\}$:
\fig{9}{lemma9.tex}
Consider the region $R_{x1y1}$ bounded by \arc{\xo\yo}\ and \seg{\xo\yo}.
If there is a singularity within $R_{x1y1} \subset R$, then we have shown
what we wanted.
Thus, assume that $R_{x1y1}$ does not contain a singularity.
If \fxo\ does not change its direction of curvature within $R_{x1y1}$,
it will be forced to spiral around in $R_{x1y1}$ forever 
(in ever smaller circles),
since it cannot cross \arc{\xo\yo}\ ($\arc{\xo\yo} \subset \xotha$ is
nonsingular) or itself within $R_{x1y1}$
($R_{x1y1}$ is nonsingular);
in particular, \fxo\ will not cross \seg{\xo\xth}.
\hence \fxo\ must change its direction of curvature within $R_{x1y1}$.
\hence $R_{x1y1} \subset R$ contains a flox.
\marginpar{We had better make a formal statement that the direction of
curvature can only change at a flex.}
\qed
\begin{corollary}
The cell partition of a curve without extraordinary singularities splits
the curve into convex segments.
\end{corollary}
\end{document}
