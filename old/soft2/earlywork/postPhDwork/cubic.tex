\documentstyle [titlepage]{article}
%
\newtheorem{lemma}{Lemma}
\newtheorem{theorem}{Theorem}
\newtheorem{corollary}{Corollary}
\newtheorem{conjecture}{Conjecture}
%
\newcommand{\arc}[1]{\mbox{$\stackrel{\frown}{#1}$}}
\newcommand{\lyne}[1]{\mbox{$\stackrel{\leftrightarrow}{#1}$}}
\newcommand{\ray}[1]{\mbox{$\vec{#1}$}}
\newcommand{\seg}[1]{\mbox{$\overline{#1}$}}
\newcommand{\hence}{\\ \mbox{$.\raisebox{1.5ex}{.}.$}\ } % three-dot therefore
                                             % symbol is desired
\newcommand{\qed}{\nopagebreak\begin{flushright}\mbox{$\Box$}\end{flushright}}
                                       % put a box on the
                                       % righthand side of the page
\newcommand{\se}{\mbox{$_{\epsilon}$}}  % subscript epsilon
\newcommand{\proof}{{\bf Proof}:\nopagebreak\\}
\newcommand{\defn}{{\bf Definition}:\ }
\newcommand{\wlogg}{\mbox{w.\ l.\ o.\ g.\ }}
\newcommand{\wrt}{\mbox{w.\ r.\ t.\ }}
\newcommand{\pq}{\arc{PQ}}
%
\begin{document}
\title{Sorting Points on a Plane Curve} % From Conics to Octics
\date{\today}
\author{}
\maketitle
%
{\bf Terminology and Definitions}:
\begin{itemize}
\item
    We shall say 'curve' when we mean 'algebraic curve'.
I deal only with irreducible plane curves in this paper.
\item
    A {\em limb} of a curve is a maximal set of points
such that the curve joins any pair of them.  For example, an hyperbola has two limbs.
********Limb is shorthand for the more formal term
********'a connected component of the real locus of the curve'.
\item
    A point is said to be {\em strictly inside} a region if it is inside
the region and not on its boundary.
\end{itemize}
{\bf Notation}:
\begin{itemize}
   \item If P and Q are distinct points of the same limb of a curve,
then \arc{PQ} is the segment of the curve joining P to Q.  (If the curve
is closed so that there is a choice of segments, then the context should 
make clear which is meant.)
    \item dist(P,Q) $\equiv \ $ Cartesian distance between P and Q
    \item $dist_{c}(P,Q) \equiv\ $ distance from P to Q along C
\end{itemize}
\begin{theorem}[Bezout]\nopagebreak
Two plane curves of orders m and n with no common components have
exactly mn intersections.
\end{theorem}
{\bf Proof}: [Wa, p. 111].
****\begin{lemma}[Sorting on a Conic]\nopagebreak
*Let C be an irreducible conic.  Let P and Q be distinct points of the 
*same limb of C.  Let \arc{PQ} be a segment of C joining
*P and Q.  Let $\vec{T}_{p}$ (resp. $\vec{T}_{q}$) be the tangent
*ray at P (resp. Q) pointing in the direction of \arc{PQ}.
*Each of \lyne{PQ}, $\vec{T}_{p}$, and $\vec{T}_{q}$ splits the plane
*into two halfplanes.  Let R be a point of C, $R \neq P$, $R \neq Q$.
*Then, R is on \arc{PQ} \ $\Leftrightarrow$ \  R is strictly inside
*(1)~the \lyne{PQ} halfplane pointed to by the tangents and
*(2)~the $T_{p}$ halfplane containing Q.
*This region looks like
*\begin{center}
*FIGURE 10
*\end{center}
*\end{lemma}
*\proof
*$\Rightarrow$: (1) If there are points of \arc{PQ} on opposite sides of
*\lyne{PQ}, then, since the conic is continuous, there is a third point of
*\arc{PQ} (and thus the conic) which intersects the line \lyne{PQ},
*contradicting Bezout's Theorem.
*\hence All of \arc{PQ} lies on one side of \lyne{PQ}.

*\ (2) Since a tangent represents two intersections with a curve, the same
*argument as used in (1) forces all of \arc{PQ} to lie on one side of 
*$\stackrel{\leftrightarrow}{T}_{p}$,
*namely the side containing Q.
*(By Bezout again, Q cannot lie on $\stackrel{\leftrightarrow}{T}_{p}$, so the
*side containing Q is well defined.)\vspace{.25in}\\
*$\Leftarrow$ : Let R be a point of the conic C in the region
* defined by (1) and (2). Suppose
*that R is not on \arc{PQ}.
*%
*% put figure here
*%
*Consider the line \lyne{PR}.
*I claim that it intersects \arc{PQ} at a point other
*than P.  Since the curve is continuous, by definition
*of tangent [TF, p. 28],
*slope~$T_{p} = \lim_{s \rightarrow 0}$~(slope~of~line~through~P~and~P+s),
*where P+s is a point on \arc{PQ} s.t. $dist_{\arc{PQ}}$(P,P+s) = s\ .
*Since slope(\lyne{PR}) $\neq$ slope($T_{p}$) and all points of \arc{PQ}
lie on the same side of $T_{p}$, $\exists\ V \in \arc{PQ}$ such that
\ray{PV} lies in between $\vec{T}_{p}$ and \ray{PR}; i.e., V
is in the region bounded by $\vec{T}_{p}$ and \ray{PR}.
\begin{center}
FIGURE 1
\end{center}
Q does not lie in this region, since R is in the region defined by
$\vec{T}_{p}$ and \ray{PQ}.
\hence The points Q and V of the conic C lie on opposite sides
of \lyne{PR}.
\hence Since the conic is continuous, $\exists\ W \in C$ such that W
lies on \lyne{PR} and the segment \arc{VQ} of the conic.
\hence \lyne{PR} intersects the conic in at least three points: P,
R, and W.  Contradiction.
\hence R must lie on \arc{PQ}.
\qed
\begin{lemma}[A preview of Lemma~\ref{flexorsi}]\nopagebreak
Let C be a ''functional'' cubic (i.e., its equation is
$y = f(x) = x^{3} + Dx^{2} + Ex + F)$ and let L be a line that intersects C
in three distinct, real, finite points P, Q, and R, where Q is on
\arc{PR}.  Then the segment \arc{PR} of C contains an inflection point.
\end{lemma}
{\bf Proof}:\nopagebreak\\
Since a line can intersect an irreducible cubic in only three points,
 where a point of tangency or singularity counts as two or more points,
 the cubic cannot stay entirely on one side of L.  L cannot
be vertical since a vertical line strikes a function in at most one
 finite real point.  Let the slope of L
 be s.  Assume w.l.o.g. that the curve rises
above L between P and Q and remains below between Q and R.
By two applications of the mean value theorem, $\exists\ p_{1} \in C$
strictly between P and Q and $p_{2} \in C$ strictly between Q and R
such that $f'(p_{1}) = s = f'(p_{2})$.
\begin{center}
FIGURE 2
\end{center}
Since f is a polynomial, all derivatives of f exist and are continuous
everywhere.  Since the line L has slope s, if $f'$ (the slope of the cubic)
remains constant at s between $p_{1}$ and $p_{2}$, then the curve
 will remain on the same side of L between these two points.
But $p_{1}$ and $p_{2}$ are on opposite sides of L, so $f'$ is not constant
but rather must increase and decrease between $p_{1}$ and $p_{2}$.
Hence, $\exists\ q_{1} \in C$ and $q_{2} \in C$ between $p_{1}$ and
$p_{2}$ such 
that $f''(q_{1}) > 0$ and $f''(q_{2}) < 0$.
\hence Since $f''$ is continuous, $\exists\ r \in C$ between $q_{1}$ and
$q_{2}$ such that $f''(r) = 0$ and $f''$ changes sign at r.
By definition, r is an inflection point [TF, p.\ 133].
Moreover, r is on \arc{PR}.
\qed
\section{The convexity of Polygons and Curves}
\subsection{Three equivalent definitions of convexity for a polygon}
The two common definitions for convexity of a polygon P are:
\begin{itemize}
  \item $v,w \in P \Rightarrow tv + (1-t)w \in P,$ if $0<t<1$
  \item $P = \{ \sum_{i=1}^{n} \lambda _{i}v_{i} \mid 
     \sum_{i=1}^{n} \lambda _{i} = 1, 0 \leq \lambda _{i} \leq 1,
     i = 1, \ldots ,n \},$ where $v_{1}, \ldots ,v_{n}$ are the 
     vertices of P.
\end{itemize}
We present an alternative characterization of convexity which works
with the boundary rather than the interior of the polygon, and therefore
serves better to motivate the definition of convexity for curves.
(This lemma is also useful in a later proof.)
\begin{lemma}[A third definition of convexity]\nopagebreak 
\label{conv3}
Let P be a connected polygon with vertices $v_{1}, \ldots ,v_{n}$ such that
no edge touches any other edge except at endpoints.
Then P is convex $\Leftrightarrow$
\\
for any edge E = 
\marginpar{(*)}
\seg{v_{i}v_{j}}, the continuation \lyne{v_{i}v_{j}}
of E does not strike any other edge of the polygon outside of E.
\end{lemma}
{\bf Proof}:\nopagebreak\\
$\Rightarrow$:\ \ Assume that P is convex.  Let \seg{v_{i}v_{j}}
be an edge of P.  Suppose that the continuation \lyne{v_{i}v_{j}}
strikes P outside of \seg{v_{i}v_{j}}.  Let w be the first such intersection
and assume w.l.o.g. that w lies on the ray \ray{v_{i}v_{j}}.
Then, since dist($v_{j},w) > 0$ and w is the first intersection,
$\exists\ x \in$ \seg{v_{j}w} s.t. $x \not\in P.$
This contradicts the first definition of convexity above.
\hence \lyne{v_{i}v_{j}} cannot strike another edge of P.\vspace{.25in} \\
$\Leftarrow$:\ \ Assume (*).  Let $v,w \in P$ and suppose 
$\exists\ t, 0<t<1$ s.t. $tv + (1-t)w \not\in P.$
Then there must be an edge E that strikes \seg{vw}.
The line defined by E separates v from w,
 so some edge F must cross over it; i.e., the line intersects F
in between its endpoints, say at x.
x cannot be a point of E.
Thus, the continuation of E strikes an edge of P outside of E, a 
contradiction of (*).
\hence $\forall\ t, 0<t<1 , tv + (1-t)w \in P$, so P is convex.
\qed
\subsection{A definition for curve convexity}
We translate the last characterization of convexity into an analogue
for curves.\vspace{.25in} \\
{\bf Definition}: A segment \pq\ of an irreducible curve is {\em convex} if 
$\pq \setminus \{P,Q\}$ is nonsingular and the 
tangent at any point of
$\pq \setminus \{P,Q\}$ does not further intersect \pq.
Note: under this definition, the segment 
\center{FIGURE 2A of cubic.tex}
is not convex.
(Note: this definition was changed on 9/19/86 to exclude the endpoints of 
the segment.  I hope that this doesn't affect other proofs.)
\begin{lemma}[Monotonicity of the tangent on convex curves]\nopagebreak
\label{mono}
Fix a ray in the plane from which to measure angles from.  Let
the measurement of angles be continuous: for example, as an angle
approaches 0 from the positive side and passes through 0, the angle becomes 
negative rather than jumping to a value just under $2\pi$.
Let C be a convex, singularity-free segment of an irreducible curve.
Then, as one moves along C in a given direction, the angle of the 
directed tangent (pointing in the direction one is moving) will change 
strictly monotonically.  (Equivalently, modulo jumps at $\pm\infty$, the
slope of the tangent changes strictly monotonically.)
\end{lemma}
{\bf Proof}:\nopagebreak\\
Since C contains no singularities, the tangent changes smoothly on C
(Lemma~\ref{tangcont} of lemma9.tex).
Suppose that the angle of the tangent is not monotonic.
Then we can assume w.l.o.g. that $\exists$ a point P of C and 
$\delta > 0$ s.t. 
$\angle$ tangent at Q $\leq \angle$ tangent at P, $\forall\ Q \in C$
   s.t. $dist_{c}(P,Q) < \delta$ .
Pick any $\epsilon$ neighbourhood of P, $0< \epsilon \leq \delta$.
This neighbourhood will contain two points $R_{\epsilon},S_{\epsilon}
\in C$ on opposite sides of P which have parallel tangent slopes:
\begin{quote}
	Let $P_{-\epsilon}$ and $P_{+\epsilon}$ be the points of C s.t. 
	\mbox{$dist_{C}(P,P_{-\epsilon})=\epsilon=dist_{C}(P,P_{+\epsilon})$}
	Let $A_{1}$ and $A_{2}$ be the angles of the tangent 
        at $P_{-\epsilon}$
	and $P_{+\epsilon}$, respectively. Assume w.l.o.g. $A_{1} \leq A_{2}$.
	Since $A_{1} \leq A_{2} \leq \angle$ tangent at P and the tangent
	is continuous, $\exists$ a point of \arc{P_{-\epsilon} P} 
        whose tangent's
	angle is $A_{2}$.
\end{quote}
Because C is convex, all of it must lie on one side of the tangent
at $R_{\epsilon}$.  Similarly on one side of $S_{\epsilon}$'s tangent.
But the tangents at $R_{\epsilon}$ and $S_{\epsilon}$ are parallel, and 
yet not the same (otherwise this tangent intersects the cubic four times).
\hence All of C must lie in between $R_{\epsilon}$ and $S_{\epsilon}$'s
tangents.\\
$dist(R\se,S\se) \leq dist_{C}(R\se,S\se) = dist_{C}(R\se,P) + dist_{C}
(P,S\se) \leq \epsilon + \epsilon = 2\epsilon$\\
But $\epsilon$ was chosen arbitrarily s.t. $0 < \epsilon \leq \delta$.
\hence C is sandwiched in between two parallel lines that lie arbitrarily
close together.
\hence C is a line, contradicting the convexity of C and the irreducibility
of the curve.
\hence The angle of the tangent must change monotonically.\\
Moreover, the angle of the tangent cannot ever 
remain constant as one moves along
the cubic curve, otherwise two points would share the same tangent, 
contradicting Bezout.  Thus, the angle of the tangent must change strictly
monotonically.
\qed
\subsection{Some useful lemmas: an interlude}
We digress to present three results that become necessary later:
\begin{lemma}[A bound to the number of singularities on an
irreducible curve]\nopagebreak
\label{singbound}
If an irreducible curve of order n has
multiplicities $r_{i}$, then
 \[ (n-1)(n-2) \geq \sum r_{i}(r_{i}-1) \]
\end{lemma}
\proof
\cite[p.\ 65]{wa}
\qed
{\bf Corollary}\ \ 
{\em An irreducible cubic can have at most one singularity, and this
singularity is a double point.}
\begin{lemma}\nopagebreak 
\label{nonbounded}
An infinitely long segment of an algebraic curve is not contained within any
closed region.
\end{lemma}
\proof
Lichtenbaum verifies that this is indeed true, but does not know, offhand,
a theorem to quote.  It is probably a basic enough fact to state as a fact
without proof.\\
Intuition:  Curve cannot wiggle too many times, o.w. you can throw a line
through it that contradicts Bezout.  But it must wiggle infinitely often
if it is to avoid crossing the boundaries.
\qed
\begin{lemma}\nopagebreak 
\label{planarcutsthru}
A line L cuts through a plane curve C at a nonsingular point P
iff the multiplicity of intersection of L with C at P is odd.
\end{lemma}
\proof
Assume WLOG that L is the y-axis and P is the origin. (Rigid
transformations---rotations and translations---do not affect the 
multiplicity or singularity of the curve at an intersection.)
Let $N_{\epsilon}(0)$ = {$c\in C |$dist$_{C}(c,(0,0)) < \epsilon$}
where dist$_{C}(c_{1},c_{2})$ is the length of the curve C between
two points $c_{1},\ c_{2}$ on it (already defined in cubic.tex).
That is, $N_{\epsilon}(0)$ is a small part of the curve near the
origin.  We claim that $\exists \epsilon > 0$ such that $N_{\epsilon}(0)$
can be represented by a function y = g(x) (i.e., $\not\exists
c_{1},\ c_{2} \in N_{\epsilon}(0)$ with the same abscissa).
The directed tangent at the origin is ($\pm$1,0), depending upon
the direction that you are travelling.  
The tangent to an algebraic curve
\marginpar{Prove! This proof is also needed in Lemma 10 of cubic.tex}
changes continuously except at singularities.
Thus, it takes some positive distance from the nonsingular origin for
the directed tangent vector of a point to reverse the sign of its
abscissa from 1 to $-\delta$ (or -1 to $+\delta$).
In other words, $\exists \epsilon > 0$ such that the curve does not
change directions (or become constant) with respect to x for an
$\epsilon$ distance along the curve.
\hence $\exists \epsilon > 0$ such that $\not\exists c_{1},\ c_{2} \in
N_{\epsilon}(0)$ with the same abscissa.
Within this neighbourhood, 
\marginpar{We should show that g(x) is a polynomial, since I think
that you need this in order to take the Taylor series as we have.}
the curve can be represented by a function,
say y = g(x).\vspace{.25in} \\
Expand g(x) into a Taylor series:
\begin{equation}
\label{eq:Taylor}
g(x) = g(0) + g'(0)x + g''(0)x^2 + \ldots  
\end{equation}
Since $y-g(x) = 0$ represents the curve C near to the origin
and $x=t,\ y=0$ is a parameterization of L, the intersections of L
with C near to the origin are associated with the roots of $0-g(t)=0$
(substituting the parameterization into the curve equation).
In particular, using (\ref{eq:Taylor}) above, we can see that the
multiplicity of the intersection of L with C at the origin is
min ${i|g^{(i)}(0) \neq 0}$.
Now, we know that g(x) changes sign as x changes sign $\Leftrightarrow$
C cuts through L, the y-axis, at P, the origin.
Also, as x $\rightarrow$ 0 the lowest order term in (\ref{eq:Taylor})
dominates all other terms.
Thus, g(x) changes sign as x changes sign $\Leftrightarrow$ the lowest
order term in (\ref{eq:Taylor}) is odd.
\hence C cuts through L at P $\Leftrightarrow$ the lowest order term
in (\ref{eq:Taylor}) is odd.
\hence C cuts through L at P $\Leftrightarrow$ the multiplicity
of intersection of L with C at P is odd.
\qed
\subsection{More on convexity}
Back to the main development.\vspace{.25in} \\
{\bf Definition}: A {\em flex} of a curve F is
a nonsingular point $P \in F$ such that P's tangent has
three or more intersections with F at P.  
The number of intersections of the flex's tangent with the curve at the
flex will be called the {\em order} of the flex.
A flex of odd order will be called a {\em flox}.
For example, the origin of $y=x^{3}$ is a flex and a flox.
The property of flexes that interests us is that they are the only
places where the curve can change from concave to convex (or vice versa) or,
in other words, where the curve about a point P can lie on both sides of P's
tangent.
For this reason, we will only be interested in floxes, since flexes of even
order do not exhibit this property (by Lemma~\ref{planarcutsthru}).
For example, the origin of $y=x^{4}$ is a flex of even order:
\begin{center}
Figure 6 of lemma9.tex
\end{center}
We shall try to use the term `flox' to underline the fact that we are
concerned with flexes of odd order only.
However, this is rather a clumsy term so we shall often refer to floxes
simply as flexes.
Unless we explicitly say 'flex of even order', we shall always be 
referring to floxes (just as `curve' implicitly means 
`irreducible algebraic curve').
\begin{lemma}[A characterization of convexity] \nopagebreak
\label{convmeans2}
Let C be a segment of an irreducible cubic curve, or a 
singularity-free segment of an arbitrary irreducible curve.
C is convex and has no flexes \ $\Leftrightarrow$ \  no line has more
than two real intersections with C.
\end{lemma}
{\bf Proof}:\nopagebreak\\
$\Leftarrow$:\ \ Assume that no line intersects C more than twice in the real
plane.  C must be nonsingular (otherwise, consider 
the tangent at the singularity)
and contain no flexes.
Since a tangent intersects C twice at its point of tangency, it cannot 
further intersect the curve.  Hence, C is convex.\vspace{.25in} \\
$\Rightarrow$:\ \ Assume that C is convex and has no flexes.
Suppose that line L has at least three real intersections with C.  If 
two of these intersections are at the same point, then L is tangent 
to a point of the curve and intersects another point, contradicting convexity.
If all three are at the same point, then this point is a flex.
\hence L's intersections with C are at distinct points: say R, S, and T
are three of them.
Assume w.l.o.g. that S lies on \arc{RT}.  Since the tangent changes
continuously except at singularities of the curve, and an irreducible
cubic has at most
one singularity, we can further assume w.l.o.g. that the tangent changes
continuously on \arc{RS}. (None of R, S, T can be singularities,
otherwise the cubic intersects \lyne{RT} more than three times.)
By linear transformations of the coordinate system, move the curve so that 
\ray{RT} is the positive x-axis.  Assume w.l.o.g. that \arc{RS} is above the
x-axis.
\begin{center}
FIGURE 3
\end{center}
We wish to show that there is a point of 
\arc{RS} whose tangent strikes \arc{ST},
thus contradicting convexity.

We shall apply lemma~\ref{mono}.  Measure angles from the positive
x-axis (\ray{RT}) counterclockwise.\\
$\angle$(tangent~at~R~directed~toward~\arc{RS})~$\in (0,\pi)$, since \arc{RS}
lies above the x-axis.
Similarly, $\angle$(tangent~at~S~directed~toward~\arc{ST})~$\in (\pi,2\pi)$
[or $(-\pi,0)$].
By the convexity of C, all of \arc{RT} lies on one side of the tangent
at R.
\hence The angle of the tangents decreases as you leave R along \arc{RS}, 
otherwise the curve would move to the opposite side of R's tangent from T.
Thus, by lemma~\ref{mono}, $\exists~p \in \arc{RS}$ whose tangent has
angle 0.  As the tangent sweeps from angle 0 (parallel to \lyne{RT}) at p
to an angle less than 0 at S, the intersection of the tangent with the 
x-axis \lyne{RT} sweeps from a point at $\infty$ to S, through T.
\hence Some point of \arc{RS} (actually infinitely many) strikes \seg{ST}.
Since the curve is continuous, any line separating the two points S and T
of the curve must intersect the curve somewhere on \arc{ST}.  Hence, the 
tangent of some point of \arc{RS} strikes \arc{ST}, 
contradicting convexity.
We conclude that no line can have more than two real intersections
with C.
\qed
\begin{corollary}
An irreducible quadratic curve is convex.
\end{corollary}
\proof
A quadratic curve is always nonsingular (Lemma~\ref{singbound}.
Any line hits an irreducible quadratic curve in at most two real
points (Bezout).  Apply the previous lemma.
\qed
\begin{lemma}[Sorting on a convex segment]\nopagebreak
Let \arc{AB} be a convex segment of a curve.  Let $p_{1},\ldots,p_{n}$
be distinct points of \arc{AB}, and let H be the convex hull of A, B, 
$p_{1},\ldots,p_{n}$.  Then the order of the points on H is the same
 as the order of the points on \arc{AB}.  (Thus, one can use a convex hull
algorithm to sort the points on a convex segment of a curve.)
\end{lemma}
{\bf Proof}:\nopagebreak\\
Since the convex hull of a set of points is unique, it is sufficient to
 show that the polygon created by joining the points $p_{i}$ in sorted
order is their convex hull.  Let this polygon 
be $P = r_{0}r_{1}\ldots r_{n+1}$\ , where $r_{0} = A$, $r_{n+1} = B$, and
$r_{1},\ldots ,r_{n}$ is the sorted order of the $p_{i}$ on the curve.
Let $E = \seg{r_{i}r_{i\oplus 1}}$ be an edge of P, \lyne{E}~=~line~through~E,
and \arc{E}~=~segment of \arc{AB} from $r_{i}$ to $r_{i\oplus 1}$.
All of \arc{E} stays on the same side of \lyne{E}, otherwise \lyne{E} strikes
\arc{E} in more than two points, contradicting the convexity of \arc{AB}
(Lemma~\ref{convmeans2}).
Let s $\in$ \arc{AB}, s $\not\in$ \arc{E}.  If s lies on the same side of
\lyne{E} as \arc{E}, then \lyne{E} intersects \arc{AB} in at least three
points (\arc{AB} must either be tangent at $r_{i}$ or $r_{i\oplus 1}$
to stay on the \arc{E} side of \lyne{E} or cut \lyne{E} again to get 
back to this side after leaving it), again contradicting convexity.
\hence All of \arc{E} lies on one side of \lyne{E} and all the rest of
\arc{AB} on the other.  Thus, since E joins consecutive points on the
curve, all of the points $r_{j}$ lie on one side of a halfplane defined
by the polygon edge E, and only the endpoints of E lie on \lyne{E}.
If the continuation \lyne{E} of E intersects another edge
\seg{r_{j}r_{j\oplus 1}}, then either $r_{j}$ or $r_{j\oplus 1}$ lies
on \lyne{E} or $r_{j}$ and $r_{j\oplus 1}$ lie on opposite sides of \lyne{E},
a contradiction.
\hence By lemma~\ref{conv3}, the polygon P is convex.
\hence P is the convex hull of the $r_{i}$'s, since a convex polygon is
the convex hull of its vertices.  
\marginpar{Clinch.}
\qed
\begin{lemma}\nopagebreak
\label{flexorsi}
Let C be a nonconvex segment of an irreducible, cubic curve.
Then C contains a flex or a singularity.
\end{lemma}
{\bf Proof}:\nopagebreak\\
Let $P \in C$ be a point whose tangent strikes C again at a point
$Q \in C, Q \neq P$.
Suppose that \arc{PQ} contains no singularity.
We wish to show that \arc{PQ} contains a flex.
\arc{PQ} is confined to one side of \lyne{PQ} since \lyne{PQ} already
has three intersections with the cubic. 
Shift the coordinate system so that P is the origin and \lyne{PQ} is the
positive x-axis (only to make the argument simpler).
Assume w.l.o.g. that \arc{PQ} lies below \lyne{PQ}.
Since all of \arc{PQ} lies below \lyne{PQ} (the x-axis) and the tangent
\marginpar{Prove that the tangent changes continuously except at sings.}
changes continuously except at singularities (Lemma~\ref{tangcont} of 
lemma9.tex), in some 
$\epsilon$-neighbourhood of P, the directed tangent sweeps smoothly
down below the x-axis, clockwise in one direction and counterclockwise
 in the other.
Let S be the infinite segment of C starting at P and proceeding in the
direction away from Q.  
Suppose that the directed tangent sweeps clockwise as one leaves
P along S.  Then, since S can neither hit \lyne{PQ} again (Bezout) nor
hit \arc{PQ} (\arc{PQ} is singularity-free), S is bounded within the region 
defined by \lyne{PQ} and \arc{PQ}.
\begin{center}
FIGURE 4
\end{center}
This contradicts Lemma~\ref{nonbounded}.
\hence The directed tangent sweeps clockwise as one 
\marginpar{(*)}
leaves P along \arc{PQ}.\\
$\forall\ \epsilon > 0$, let $Q_{\epsilon} \in \arc{PQ}$ s.t.
$dist_{C}(Q,Q_{\epsilon}) = \epsilon$.
$Q_{\epsilon}$ is below, but within $\epsilon$ of, the x-axis.
By (*), $\exists\ \epsilon > 0$ and $P' \in \arc{PQ}$ (near P) s.t. the
tangent from $P'$ hits $Q_{\epsilon}$.
Notice that $dist_{C}(P',Q_{\epsilon}) < dist_{C}(P,Q)$.
Also, notice that as long as $P' \neq Q_{\epsilon}$, 
one can repeat the entire argument used above, replacing P by
$P'$ and Q by $Q_{\epsilon}$, to find two points on \arc{PQ} that
are even closer together such that the tangent at one strikes the other.
And so on.
That is, as you go along \arc{PQ} towards Q, the third point of intersection
of the tangent moves along \arc{PQ} towards P, until they meet.
When the point of tangency and the third point of intersection meet, 
\marginpar{Make this rigorous.}
the tangent strikes the curve three times at that point.
Moreover, this point is nonsingular since \arc{PQ} does not
contain a singularity, by assumption.  Hence, it is a flex.
\qed
This proof yielded an interesting result, which we should record formally:
\begin{lemma}\nopagebreak
Let C be an irreducible cubic.  Let $P \in C$ and let Q be the third 
intersection of P's tangent with C.  Then, if C connects P to Q
and \arc{PQ} is nonsingular,
\arc{PQ} contains a flex.
\end{lemma}
We also have the following corollary:
\begin{corollary}\nopagebreak
If a segment of an irreducible, cubic curve contains no flexes or 
singularities, then it must be convex.
\end{corollary}
This suggests splitting a cubic curve into limbs and then partitioning 
each limb at flexes and singularities.  This yields a group of convex
segments, each of which can be sorted by creating convex hulls.
\section{The limbs of a cubic}
{\bf Definition}:\ \ A {\em closed} limb is a limb that is a 'cycle'.
That is, starting from any point P of the limb and proceeding along the
limb in either direction, one eventually returns to P.  
A limb is {\em open} if it is not closed.
\begin{lemma}\nopagebreak
A singular, irreducible cubic has one limb, and this limb is open.
\end{lemma}
{\bf Proof}:\nopagebreak \\
An irreducible curve of order n with points of multiplicity $r_{i}$
has a rational 
\marginpar{is rational = has a rational parameterization (see Wa)?}
parameterization(?) if\footnote{\cite[p.\ 67]{wa}}
\[ (n-1)(n-2) = \sum r_{i}(r_{i} - 1) \]
By the corollary to Lemma~\ref{singbound},
a singular, irreducible cubic has one
singularity, a double point.  Hence, substituting above, we discover
that it must have a rational parameterization.
\hence It has only 
\marginpar{Clinch, using Ganguli (unicursal $\Rightarrow$ unipartite),
pp. 206-7.}
one limb.\\
Prove that limb is open, using an argument that allows generalization.
(See scribbles on old page 8.)
\qed
{\bf Notation}:  We write homogeneous coordinates as (w,x,y), following
Walker. [Note: Fulton's (x,y,z) notation is much more intuitive
 (because of its
correspondence with the line through (x,y,z) and O) so it should be used
in future.]
\begin{lemma}\nopagebreak
A nonsingular, irreducible cubic C has two limbs, one of which is closed.
\end{lemma}
{\bf Proof}: [Relies on normal forms]\nopagebreak \\
We first show that C can have at most one 'open' limb.
By linear transformations (which do not affect the number or structure
\marginpar{Convince myself that linear transformations do not change the
number of pts. at infinity.}
of limbs: they only translate and/or rotate), the equation of C
can be put into the
normal form:
\[ y^{2} = g(x) = x^{3} + ax^{2} + bx \]
where g(x) has three distinct roots [Wa, p.\ 72].
In homogeneous coordinates, this equation is
\[ y^{2}w = x^{3} + ax^{2}w + bxw^{2} \]
Let the curve represented by this normal-form equation be called $C'$.
How many points at infinity does $C'$ have?
The w term of a point at $\infty$ is 0.  Substituting $w=0$
into the above equation yields $x^{3} = 0$, i.e. $x=0$.
Thus, points at $\infty$ of $C'$ are of the form (0,0,y).
There is only one point of this form: \mbox{(0,0,1)}.
\hence $C'$ has one point at $\infty$.  
But each open limb contains at least one point at $\infty$ since, going in
either direction along the limb, the limb does not loop back on itself 
\marginpar{Review this argument.}
(it is nonsingular and not closed) so it is of infinite length and 
thus unbounded (Lemma~\ref{nonbounded}).
(It will have {\em two} points at $\infty$ unless it is asymptotic to the 
same line in both directions.)
Moreover, limbs are disjoint, even at points at $\infty$: if two limbs
\marginpar{Verify this.}
share a point at $\infty$, then there is a singularity 
there(??): contradiction.
\hence $C'$ has at most one open limb, otherwise it would contain more than
two points at infinity.

When $C'$ has been put in the above normal form, it is symmetrical w.r.t.
the x-axis and strikes the x-axis in three distinct points.
If two of these three points belong to the same limb, then the
limb joins these points so, by symmetry, the limb joins the two points
in mirror image, and the limb is closed.
%
% put figure here?
%
\hence An open limb of $C'$ hits the
\marginpar{(1)}
x-axis at most once.\\
Conversely, a closed limb of $C'$ must intersect the x-axis twice:
\begin{quote}
	A line intersects a closed limb an even number of times, since after 
	it goes inside the limb it must eventually leave.  Thus, the x-axis
	hits a closed limb of a cubic either twice or never.  
	If a closed limb lies entirely on one side of the x-axis, then there
	is a mirror-image closed limb on the opposite side of the x-axis due
	to symmetry.  But a cubic cannot have two closed limbs because
	a line joining two points of the limbs, one from each, would
	intersect the cubic at least four times.  Thus, the x-axis 
\marginpar{(2)}
        hits a
	closed limb twice.
\end{quote}
Thus, since $C'$ intersects the x-axis three times, by $(1)$ and $(2)$ it
must either have
\begin{itemize}
\item
at least three open limbs, or
\item
one closed limb and at least one open limb
\end{itemize}
But we have shown that $C'$ has at most one open limb.
\hence $C'$ has one closed limb and one open limb.\\
Moreover, C and $C'$ have the same number and type of limbs.\\
(Note that, from comments made above, the open limb has only one point
at infinity and so, as one proceeds along it in either direction, one
asymptotically approaches the same line.)
\qed
Let C be an arbitrary curve.
Let $F(x) = 0$ be the homogeneous equation of C, 
$f(x) = 0$ the affine equation.\vspace{.25in} \\
\underline{Identifying singularities}
\begin{lemma} \nopagebreak
P is a singularity of F $\Leftrightarrow\ f_{x}(P) = 0 = f_{y}(P).$
\end{lemma}
{\bf Proof}:
[Fu, p. 64] and [Wa, pp. 53-4].\vspace{.25in} \\
Thus, finding singularities involves solving a system of two
\marginpar{Is this at all feasible?}
degree $n-1$ equations (where n is the degree of the curve).
One can distinguish nodes from cusps for double points (which
is all we are concerned with in the case of cubics) using the 
following fact:
\begin{lemma} \nopagebreak
If P is a double point of F, then \\
( P is a node $\Leftrightarrow\ F_{xy}(P)^{2} \neq F_{xx}(P)F_{yy}(P)$ ).
\end{lemma}
{\bf Proof}:
[Fu, p. 68].\vspace{.25in} \\
\underline{Identifying real, finite flexes}\vspace{.25in} \\
Compute the Hessian of the curve: $H(\vec{x}) = |F_{ij}(\vec{x})| = 0.$
Let $h(\vec{x}) = H(1,x,y)$ be the affine version of H.
Find the finite,real points of intersection of h with the curve.
Throw out all singular points.
You are left with the desired flexes.
(See [Wa, p. 71].)

The bottleneck is the intersection of the curve with its Hessian.  
\marginpar{Is this entirely prohibitive?}
The Hessian of a degree n equation is of degree $3(n-2)$.
Thus, in the case of a cubic, one must solve a system of two cubics,
which is already a daunting prospect.\vspace{.25in} \\
\underline{Flexes on cubics}\vspace{.25in} \\
\marginpar{Characterize the three classes of cubics here.}
A nonsingular cubic has  three real flexes, looking like so on
$y^{2} = x^{3} - x$:
\begin{center}
FIGURE 5
\end{center}
Nodal and cuspidal cubics both have one flex.  The normal forms
$y^{2} = x^{2}(x+1)$ [nodal] and $y^{2} = x^{3}$ [cuspidal] both
have their flex at a point of infinity.\vspace{.25in} \\
\underline{Partitioning at a flex}\vspace{.25in} \\
Since a flex is a nonsingular point, only two convex segments meet there.
By Lemma~\ref{planarcutsthru}, the two convex segments that meet at a flex lie
on opposite sides of the tangent to the flex.
Moreover, the curve only crosses this tangent at the flex (by Bezout).
\hence The flex's tangent effectively segregates the two convex
segments that it bounds.\vspace{.25in} \\
\underline{Partitioning at a singularity (for cubics only; uses
normal forms)}\vspace{.25in} \\
We wish to show that the tangents at a cubic's singularity do the job.
Consider the nodal cubic first.
Since every nodal cubic is projectively equivalent to
$y^{2} = x^{2}(x+1)$,\footnote{\cite[p.\ 128]{namba}}
and 'P is a singularity',
'T is tangent to P' are projective properties (i.e., properties 
preserved under projective equivalence)
\marginpar{give ref. or prove}
we need only show the result for $y^{2} = x^{2}(x+1)$.
The node is at the origin.  Let A, B, C be the three convex segments
that meet at the node: B the loop, A the segment totally above the 
y-axis,  and C the segment totally below.
The two tangents to the node are $y+x=0$ and $y-x=0$ (the factors
of the lowest degree form of the equation). Thus,\\
\begin{center}
FIGURE 6
\end{center}
Since the tangent at a double point strikes the curve three times
at that point (two for the singularity, one more for being tangent)
a cubic
\marginpar{(*)}
cannot intersect either of the tangents at a double point again.
But the segments A, B, and C are separated from one another by the
node's tangents:
\begin{itemize}
\item A and B are on opposite sides of $y=x$
\item B and C are on opposite sides of $y= -x$
\item A and C are on opposite sides of $y=x$ (and $y= -x$)
\end{itemize}
Thus, the tangents at a node of a cubic segregate the convex segments
that meet there.

For the cusp, we need only consider the normal form $y^{2} = x^{3}$,
since every cubic with cusp is projectively equivalent to this 
equation.\footnote{\cite[p.\ 128]{namba}}
The cusp is at the origin and the two coincident tangents at the cusp
are $y=\pm 0$.
$y^{2} = x^{3}$ looks like:
\begin{center}
FIGURE 7
\end{center}
Clearly the two convex segments that meet at the cusp are on opposite
sides of the y-axis near the origin.
\hence By (*), they are always on opposite sides of the tangent(s) to
 the cusp.\vspace{.25in} \\
\underline{Fulton algorithm to find tangents to a point}
\marginpar{$\leftarrow$}
\\ 
%
% A GENERAL TECHNIQUE FOR SHOWING THAT THE TANGENTS SEPARATE
%
% For a cubic, there are at most three convex segments bordering on a
% singularity:[there are at most two crossing segments at the singularity,
% since the sing. is double, and there is an incoming and outgoing part to
% each of these segments, i.e., before and after the sing.  This limits the 
% # of bordering convex segments to 4.  Exactly two of the parts are joined,
% otherwise the nonsing. cubic is not ONE OPEN limb: > 2 => closed, < 2 =>
% two limbs.]
% You don't have to worry about separating the two joined parts: if they
% are two different convex segments, then they will contain a flex at which
% they will be split.
\underline{Sorting the convex segments}\vspace{.25in} \\
There is no problem sorting the convex segments that meet
at a flex or a cubic's cusp, since there are only two of them.
The three convex segments that meet at a cubic's node can be easily sorted:
the looped segment must be the middle one (see picture of $y^{2} = x^{2}(x+1)$
).
\marginpar{how do you recognize the looped segment?}
\begin{conjecture} \nopagebreak
A convex segment contains no flex, except possibly at a 
boundary.
\end{conjecture}
N.B. This would allow Lemma~\ref{convmeans2} to be simplified.
\begin{corollary} \nopagebreak
\label{noflexonclosed}
The closed limb of a nonsingular cubic contains no flexes.
\end{corollary}
{\bf Proof}: \ A line hits a closed limb an even number of times.
Thus, a tangent to the closed limb of a nonsingular cubic does not hit
the closed limb again.
\hence The closed limb is convex.
\qed
\begin{conjecture} \nopagebreak
\label{2tantoclosed}
Let C be a closed, nonsingular segment or curve. \\
Let $P \in \Re^{2} \setminus (C \cup Interior(C)).$
Then
\begin{enumerate}
\item $\exists$ two points Q,R of C whose tangents strike P.
\item The region defined by \ray{PQ} and \ray{PR} contains all C.
\end{enumerate}
\end{conjecture}
\underline{Quick sketch of how to separate the two limbs of a
nonsingular cubic}\vspace{.25in} \\
Let N be a nonsingular cubic.
Let $L_{1}$ and $L_{2}$ be the open limb and closed limb of N, respectively.
Let $\alpha$ be one of the finite flexes of N.
By Corollary~\ref{noflexonclosed} and Conjecture~\ref{2tantoclosed},
$\exists$ P, Q on the closed limb $L_{2}$ s.t. the tangents at P
and Q both hit $\alpha$.
\begin{lemma} \nopagebreak
\label{isolate}
The tangents at P and Q properly isolate the two limbs $L_{1}$ and $L_{2}$.
\end{lemma}
{\bf Proof}: \ To come.
\marginpar{!}
\vspace{.25in} \\
We wish to develop an algorithm to find these two tangents.
It is a fact that there are three 
\marginpar{Develop some 'class' theory to motivate this.}
points whose tangents hit a flex of a cubic.
Let A, B, C be the points whose tangents $T_{A}, T_{B}, T_{C}$ hit
the flex $\alpha$.
We can find $T_{A}, T_{B},$ and $T_{C}$, since there is an algorithm
\marginpar{write this AlgX.}
to find all tangents that hit a given point.
Here is a picture of the situation:
\begin{center}
FIGURE 8
\end{center}
We need to eliminate one of the tangents.
The three tangents define three regions.  
Let $R_{AB}$ be the region defined by $T_{A}$ and $T_{B}$, and so on.
Let $\delta \neq \alpha$ be another finite flex 
\marginpar{Show that {\em any} nonsingular cubic has two finite flexes.}
of the cubic N.
$\delta$ cannot lie on the boundary of a region:
\begin{quote}
	By Bezout, the only points of N that lie on a boundary are
	A, B, C, and $\alpha$.  By Bezout again, $\delta$ is not
	A, B, or C.
\end{quote}
Assume w.l.o.g. that $\delta$ lies inside $R_{AB}$.
Then, by Corollary~\ref{noflexonclosed} and Lemma~\ref{isolate},
$R_{AB}$ is not the region that contains the closed limb.
\hence Either $T_{A}$ or $T_{B}$ is not tangent to the closed limb.
\hence $T_{C}$ is tangent to the closed limb.\vspace{.25in} \\
{\bf Claim}: \ $T_{B}$ is the other closed limb tangent $\Leftrightarrow$
the third intersection of \lyne{BA} or the third intersection of \lyne{BC}
with the cubic N lies inside the region $R_{BC}$.\vspace{.25in} \\
{\bf Proof of claim}: \ The pivotal fact is that a line either hits the
closed limb twice or never.
Let R~=~3rd intersection of \lyne{BA} with N,
S~=~3rd intersection of \lyne{BC} with N.
\begin{center}
FIGURE 9
\end{center}
C lies on the closed limb and one of A, B also does.
Thus, both \lyne{BA} and \lyne{BC} must hit the closed limb twice,
R must lie on the closed limb, and S lies on the closed limb iff B
does not.\vspace{.25in} \\
$\Rightarrow$: \ Assume that B is on the closed limb.
R lies on the closed limb $\Rightarrow$ R lies in $R_{BC}$.\vspace{.25in} \\
$\Leftarrow$: (contrapositive of $\Rightarrow$) \ Assume that
B does not lie on
the closed limb.  Then R and S lie on the closed limb, and
the region containing the closed limb must be $R_{AC}$.
\hence R and S lie in $R_{AC}$, not $R_{BC}$.
\qed
This gives us a quick and easy method of finding two tangents that
segregate the two limbs of a nonsingular cubic.
\begin{thebibliography}{The longest}
\bibitem[Fu]{fu} Fulton, William.
{\em Algebraic Curves}.
Menlo Park, California: Benjamin/Cummings, 1974.
\bibitem[Namba]{namba} Namba, Makoto.
{\em Geometry of Projective Algebraic Curves}.
New York: Marcel Dekker, 1984.
\bibitem[TF]{calc} Thomas, George B., and Finney, Ross L.
{\em Calculus and Analytic Geometry}.
Reading, Mass.\ :  Addison-Wesley, 1979.
\bibitem[Wa]{wa} Walker, Robert J. 
{\em Algebraic Curves}.
New York: Springer Verlag, 1978.
\end{thebibliography}
\end{document}
