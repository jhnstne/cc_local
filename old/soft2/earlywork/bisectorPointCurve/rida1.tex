\documentstyle[12pt]{article}

\begin{document}

\newcommand{\single}{\def\baselinestretch{1.0}\large\normalsize}
\newcommand{\double}{\def\baselinestretch{1.5}\large\normalsize}
\newcommand{\triple}{\def\baselinestretch{2.2}\large\normalsize}

\newcommand{\rd}{{\rm d}}
\newcommand{\re}{{\rm e}}
\newcommand{\ri}{{\rm i}}
\newcommand{\sgn}{{\rm sign}}
\newcommand{\half}{\textstyle{1 \over 2}\displaystyle}
\newcommand{\quarter}{\textstyle{1 \over 4}\displaystyle}
\newcommand{\dist}{{\rm distance}}
\newcommand{\cross}{\!\times\!}
\newcommand{\dotpr}{\!\cdot\!}
\newcommand{\vhat}{\hat {\bf v}}
\newcommand{\what}{\hat {\bf w}}
\newcommand{\zhat}{\hat {\bf z}}
\newcommand{\ldash}{\vrule height 3pt width 0.35in depth -2.5pt}
\newcommand{\be}{\begin{equation}}
\newcommand{\ee}{\end{equation}}
\newcommand{\ba}{\begin{eqnarray}}
\newcommand{\ea}{\end{eqnarray}}
\newcommand{\seg}[1]{\mbox{$\overline{#1}$}}

\newtheorem{dfn}{Definition}[section]
\newtheorem{rmk}{Remark}[section]
\newtheorem{lma}{Lemma}[section]
\newtheorem{propn}{Proposition}[section]
\newtheorem{exmpl}{Example}[section]
\newtheorem{conjec}{Conjecture}[section]
\newtheorem{claim}{Claim}[section]
\newtheorem{notn}{Notation}[section]
\newtheorem{thm}{Theorem}[section]
\newtheorem{crlry}{Corollary}[section]

\newcommand{\prf}{\noindent{{\bf Proof} :\ }}
\newcommand{\QED}{\vrule height 1.4ex width 1.0ex depth -.1ex\ \medskip}

\newcommand{\ACMTOG}{{\sl ACM Trans.\ Graph.\ }}
\newcommand{\AMM}{{\sl Amer.\ Math.\ Monthly\ }}
\newcommand{\BIT}{{\sl BIT\ }}
\newcommand{\CACM}{{\sl Commun.\ ACM\ }}
\newcommand{\CAD}{{\sl Comput.\ Aided Design\ }}
\newcommand{\CAGD}{{\sl Comput.\ Aided Geom.\ Design\ }}
\newcommand{\CJ}{{\sl Computer\ J.\ }}
\newcommand{\DCG}{{\sl Discrete\ Comput.\ Geom.\ }}
\newcommand{\IBMJRD}{{\sl IBM\ J.\ Res.\ Develop.\ }}
\newcommand{\IEEECGA}{{\sl IEEE Comput.\ Graph.\ Applic.\ }}
\newcommand{\IEEETPAMI}{{\sl IEEE Trans.\ Pattern Anal.\ Machine Intell.\ }}
\newcommand{\JACM}{{\sl J.\ Assoc.\ Comput.\ Mach.\ }}
\newcommand{\JAT}{{\sl J.\ Approx.\ Theory\ }}
\newcommand{\MC}{{\sl Math.\ Comp.\ }}
\newcommand{\MI}{{\sl Math.\ Intelligencer\ }}
\newcommand{\NM}{{\sl Numer.\ Math.\ }}
\newcommand{\SIAMJNA}{{\sl SIAM J.\ Numer.\ Anal.\ }}
\newcommand{\SIAMR}{{\sl SIAM Review\ }}

\pagenumbering{roman}

\title{
The bisector of a point \\
and a plane parametric curve
}

\author{
Rida~T.~Farouki \\
IBM Thomas~J.~Watson Research Center, \\
P.~O.~Box 218, Yorktown Heights, NY 10598. \\ \\
John~K.~Johnstone \\
Department of Computer Science, \\
The Johns Hopkins University, Baltimore, MD 21218.
}

\maketitle

\begin{abstract}
The {\it bisector\/} of a point ${\bf p}$ and a smooth plane curve
$C$ --- i.e., the locus traced by a point that remains equidistant
with respect to ${\bf p}$ and $C$ --- is investigated in the case
that $C$ admits a regular polynomial or rational parameterization.
In such a case it is shown that the bisector may be regarded as (a
subset of) a ``variable--distance offset'' curve to $C$ which has
the remarkable property, unlike fixed--distance offsets, of being
{\it generically\/} a rational curve. The parametric subsegments
that constitute the true bisector are identified by a ``trimming''
procedure which requires only the ability to isolate and approximate
the real roots of certain polynomials to a prescribed accuracy.
Finally, the irregular points of the bisector curve are analyzed
and are shown to fall into certain well--defined categories.
\end{abstract}

\newpage\thispagestyle{empty}\mbox{}\vfill\eject

\pagenumbering{arabic}
\setcounter{page}{1}
\thispagestyle{plain}

\section{Introduction}
\label{intro}

In descriptive geometry \cite{coxeter69} the parabola is
characterized as the locus traced by a point that remains
equidistant with respect to a fixed point ${\bf p}$ (the
{\it focus\/}) and a given straight line $L$ (the {\it
directrix\/}). Thus, points of the plane that lie to one
side of the parabola are closer to ${\bf p}$ than to $L$,
while those that lie on the other side are closer to $L$
than to ${\bf p}$. In this sense, the parabola may be
regarded as the ``bisector'' of the point ${\bf p}$ and
the line $L$.

If we substitute a smooth plane curve $C$ in place of
the straight line $L$, the bisector locus is of a more
subtle nature. This paper is concerned with investigating
the geometric properties of such loci, and formulating
tractable representations for them. Point/curve bisectors
arise in a variety of geometric ``reasoning'' and geometric
decomposition problems (e.g., planning paths of maximum
clearance in robotics or computing Voronoi diagrams for areas
with curvilinear boundaries). Although simpler than other
loci --- line/curve and curve/curve bisectors --- that arise
in these contexts, point/curve bisectors have not attracted
much attention in the literature (see, however, \cite
{yap87,yap89}).

\subsection{Regular parametric curves}

We shall focus here on the important case where the curve $C$
is described parametrically, ${\bf r}(u)=\{x(u),y(u)\}$, having
derivatives
\be \label{derivs}
{\bf r}'(u) = \{x'(u),y'(u)\} \,, \quad
{\bf r}''(u) = \{x''(u),y''(u)\} \,, \quad
\ldots {\rm etc.}
\ee
continuous to at least third order for all $u \in I$, where $I$
denotes some finite, semi--infinite, or infinite parameter domain
of interest. We assume that the parameterization of ${\bf r}(u)$
is {\it proper}, i.e., there is a one--to--one correspondence
between parameter values $u$ and points $(x,y)$ of the curve
locus --- except, possibly, for finitely many instances where
${\bf r}(u)$ crosses itself.

Since improper parameterizations arise rather infreqeuntly in
practice, and identifying them is not in general a straightforward
matter (see \cite{sederberg84,sederberg86}), we shall not dwell on
this issue. However, we do need to impose an additional constraint
on the parameterization of ${\bf r}(u)$ --- namely that it be {\it
regular\/} over the parameter domain of interest:

\begin{dfn}
The parametric speed of ${\bf r}(u)=\{x(u),y(u)\}$ is the
function
\be \label{sigma}
\sigma(u) = \sqrt{x'^2(u)+y'^2(u)}
\ee
of the parameter $u$, and the curve is said to have a {\it
regular\/} parameterization on the interval $I$ if and only if
$\sigma(u)\not=0$ for all $u \in I$.
\end{dfn}

We denote by $|{\bf v}|$ the Euclidean norm $\sqrt{v_x^2+
v_y^2}$ of a vector ${\bf v}=(v_x,v_y)$. Thus, we shall also
write $|{\bf r}'(u)|$ for the parametric speed of ${\bf r}(u)$,
to suit the context. Since we are concerned solely with {\it
real\/} functions $x(u)$ and $y(u)$ of a {\it real\/} parameter
$u$, we note that $\sigma(u)=0\;\Longleftrightarrow\;x'(u)=
y'(u)=0$.

Although much of the ensuing discussion holds for any regular
parametric curve, we shall deal primarily with the two functional
forms encountered most often in practice: the {\it polynomial\/}
curve ${\bf r}(u)=\{X(u),Y(u)\}$ of degree $n$ defined by
\be \label{polycurve}
X(u) = \sum_{k=0}^n a_k u^k \,, \quad
Y(u) = \sum_{k=0}^n b_k u^k \,,
\ee
the coefficients $\{a_k,b_k\}$ being real numbers that satisfy
$a_n^2+b_n^2\not=0$, and the {\it rational\/} curve ${\bf r}(u)=
\{X(u)/W(u),Y(u)/W(u)\}$ of degree $n$, where
\be \label{ratcurve}
X(u) = \sum_{k=0}^n a_k u^k \,, \quad
Y(u) = \sum_{k=0}^n b_k u^k \,, \quad
W(u) = \sum_{k=0}^n c_k u^k \,,
\ee
the coefficients $\{a_k,b_k,c_k\}$ being again real numbers
with either $a_n^2+b_n^2\not=0$ or $c_n\not=0$. In (\ref
{ratcurve}) we also assume that there are no factors common
to {\it all three\/} of the polynomials $X,Y,W$, i.e., that
${\rm GCD}(X,Y,W) ={\rm constant}$. (${\rm GCD}(\ldots)$
denotes the ``greatest common divisor'' of the indicated set
of polynomials, as determined by one or more applications of
Euclid's algorithm \cite{uspensky48}.)

Since polynomial curves are a special class of rational
curves with $W(u)={\rm constant}$, when we speak specifically
of rational curves it will be understood implicitly that $W(u)
\not={\rm constant}$. Roots of the polynomial $W(u)$ correspond
to ``points at infinity'' on the rational curve (\ref{ratcurve}).
In most applications, we are concerned only with the {\it affine\/}
part of a rational curve, i.e., its locus for all parameter values
$u$ such that $W(u)\not=0$.

\begin{rmk}
{\rm
A sufficient and necessary condition for the polynomial curve
$(\ref{polycurve})$ to have a regular parameterization is that
\be \label{regpoly}
{\rm GCD}(X',Y')={\rm constant} \,,
\ee
while in the case of the rational curve $(\ref{ratcurve})$ we
require
\be \label{regrat}
{{\rm GCD}(WX'-W'X,WY'-W'Y) \over {\rm GCD}(W,W')}={\rm constant}
\ee
for the affine locus to have a regular parameterization.
}
\end{rmk}

It should be noted that any polynomial or rational curve that
has an irregular parameterization will not, in general, exhibit
a smooth locus: the points where $\sigma(u)=0$ correspond to
{\it cusps\/} (sudden tangent reversals or, exceptionally,
discontinuities in higher--order differential characteristics
\cite{farouki91b}).

\begin{exmpl}
{\rm
We have already noted the simple nature of the bisector of a point
and a straight line. Another common case that yields an ``elementary''
bisector arises if we take the curve $C$ to be a {\it circle}. Assume,
without loss of generality, that $C$ is of unit radius and centered
on the origin, and let the given point be ${\bf p}=(\alpha,\beta)$.
Then the bisector is evidently the locus of points $(x,y)$ that
satisfy
\be
|\,\sqrt{x^2+y^2}-1\,| \,=\,
\sqrt{(\alpha-x)^2+(\beta-y)^2} \,,
\ee
where the left-- and right--hand sides represent the distance of
the variable point $(x,y)$ from the circle $C$ and the fixed point
${\bf p}$, respectively. Squaring twice to clear radicals, we see
that the bisector has the implicit equation
\ba \label{ellipse}
& & (1-\alpha^2)\,x^2 \,+\, (1-\beta^2)\,y^2
 \,-\, 2\alpha\beta\,xy \nonumber \\
& & \quad +\ (\alpha^2+\beta^2-1)\,(\alpha x+\beta y)
 \,-\, \quarter(\alpha^2+\beta^2-1)^2 \,=\, 0
\ea
which clearly represents a conic section. The nature of this conic
may be determined \cite{eisenhart60} by inspecting the signs of the
invariants
\be
k \,=\, (1-\alpha^2)(1-\beta^2) - \alpha^2\beta^2 \,,
\ee
which simplifies to $k=1-(\alpha^2+\beta^2)$, and
\be
D \,=\,
\left| \begin{array}{ccc}
1-\alpha^2 &
-\,\alpha\beta &
\half\alpha(\alpha^2+\beta^2-1) \\ \\
-\,\alpha\beta &
1-\beta^2 &
\half\beta(\alpha^2+\beta^2-1) \\ \\
\half\alpha(\alpha^2+\beta^2-1) &
\half\beta(\alpha^2+\beta^2-1) &
-\quarter(\alpha^2+\beta^2-1)^2
\end{array} \right| \,,
\ee
which gives the value $D=-\,\quarter[\,1-(\alpha^2+\beta^2)\,]^2
=-\,\quarter k^2$ on expansion.

When $k=0$ (i.e., the point ${\bf p}$ lies {\it on\/} the circle
$C$), equation (\ref{ellipse}) becomes $(\beta x-\alpha y)^2=0$,
and the bisector degenerates into a straight line --- the normal
to $C$ at ${\bf p}$ --- counted twice. Otherwise, the bisector is
a non--degenerate conic: an ellipse or a hyperbola according to
whether $k>0$ or $k<0$ (i.e., whether ${\bf p}$ lies {\it inside\/}
or {\it outside\/} $C$). Some examples are illustrated in Figure 1.
} \QED
\end{exmpl}

\subsection{The point/curve distance function}

In characterizing the parabola as the bisector of a point
${\bf p}$ and a straight line $L$, the meaning of the ``distance''
of any point from $L$ is clear: it is simply the length of the
{\it unique\/} perpendicular from the point in question to the
straight line $L$. In substituting a smooth parametric curve $C$
in place of $L$, we need to generalize this notion of distance
(see \cite{kelly79}):

\begin{dfn}
The distance of a point ${\bf p}$ from a regular parametric curve
${\bf r}(u)$ $=\{x(u),y(u)\}$ defined on the parameter interval
$I$ is given by
\be \label{distance}
\dist({\bf p},{\bf r}(u)) \,=\,
\inf_{u \,\in\, I} \, |\,{\bf p}-{\bf r}(u)\,| \,,
\ee
i.e., it is greatest lower bound, for all $u \in I$, on the distance
measured between the fixed point ${\bf p}$ and each point ${\bf r}(u)$
along the curve.
\end{dfn}

If ${\bf r}(u)$ is a polynomial curve, of course, the bound
(\ref{distance}) is always attained at a {\it finite\/} parameter
value $u$ regardless of whether $I$ has finite or infinite extent.
When ${\bf r}(u)$ is a rational curve and $I$ is not finite,
however, it is possible that (\ref{distance}) may be attained in
the limit $|u|\to\infty$ if the degree of $W(u)$ is not less than
the greater of the degrees of $X(u)$ and $Y(u)$. Note that as $|u|
\to\infty$ the rational curve (\ref{ratcurve}) converges to an
affine point $(x_\infty,y_\infty)$, where $|x_\infty|=|a_n/c_n|$
and $|y_\infty| =|b_n/c_n|$, if and only if ${\rm deg}(W)\ge\max
({\rm deg}(X),{\rm deg}(Y))$; otherwise it has a point at infinity
at infinite values of $u$. In the former case it is always possible
to re--parameterize (\ref{ratcurve}) by a bilinear transformation
of the parameter so as to make $(x_\infty,y_\infty)$ correspond to
a {\it finite\/} parameter value.

\begin{propn} \label{polydist}
For the point ${\bf p}=(\alpha,\beta)$ and the polynomial curve
${\bf r}(u)$ given by $(\ref{polycurve})$, let $\{u_1,\ldots,u_N\}$
be the set of distinct odd--multiplicity roots of the polynomial
\be \label{Pperp}
P_\perp(u) \,=\,
[\,\alpha-X(u)\,]\,X'(u) \,+\, [\,\beta-Y(u)\,]\,Y'(u)
\ee
of degree $2n-1$ on the interior of the interval $I$, augmented by
the finite end points, if any, of $I$. Then the distance function
$(\ref{distance})$ may be expressed as
\be \label{distance2}
\dist({\bf p},{\bf r}(u)) \,=\,
\min_{k \,\in\, \{1,\ldots,N\}} \, |\,{\bf p}-{\bf r}(u_k)\,| \,.
\ee
\end{propn}

\prf On differentiating the expression
\be \label{distsq}
|\,{\bf p}-{\bf r}(u)\,|^2 \,=\,
[\,\alpha-X(u)\,]^2 \,+\, [\,\beta-Y(u)\,]^2 \,,
\ee
we see that the distance $|\,{\bf p}-{\bf r}(u)\,|$ will attain
a ``stationary'' value whenever $[\,\alpha-X(u)\,]\,X'(u)\,+\,
[\,\beta-Y(u)\,]\,Y'(u)=0$, i.e., at the roots of the polynomial
$P_\perp(u)$. (Note that, by virtue of the constraint (\ref{regpoly}),
this equation will never be satisfied in the degenerate case $X'(u)=
Y'(u)=0$.) Only those roots of $P_\perp(u)$ that are of {\it odd\/}
multiplicity identify local extrema of $|\,{\bf p}-{\bf r}(u)\,|$,
however. To evaluate (\ref{distance}) we must compare the values of
$|\,{\bf p}-{\bf r}(u)\,|$ at each odd root of $P_\perp(u)$ on the
parameter interval $I$ and at its end points if they are finite
(since $|\,{\bf p}-{\bf r}(u)\,|\to\infty$ for any polynomial curve
as $|u|\to\infty$). Then $\dist({\bf p},{\bf r}(u))$ is given by
the smallest of these values. \QED

A result analogous to Proposition \ref{polydist} holds for regular
rational curves, provided we replace the odd roots of the polynomial
(\ref{Pperp}) by those of
\ba \label{Rperp}
P_\perp(u)
&=& [\,\alpha W(u)-X(u)\,]\,[\,W(u)X'(u)-W'(u)X(u)\,] \nonumber \\
&+& [\,\beta  W(u)-Y(u)\,]\,[\,W(u)Y'(u)-W'(u)Y(u)\,]
\ea
satisfying $W(u)\not=0$ on the interval $I$. (These roots never
correspond to the degenerate case $W(u)X'(u)-W'(u)X(u)=W(u)Y'(u)-
W'(u)Y(u)=0$ with $W(u)\not=0$ when the constraint (\ref{regrat})
is imposed.)

Thus, in computing $\dist({\bf p},{\bf r}(u))$ for a rational curve,
we compare the values of the distance $|\,{\bf p}-{\bf r}(u)\,|$
at each odd root of (\ref{Rperp}) with the values at finite
end points of the parameter interval $I$ satisfying $W(u)\not=0$
and/or at infinite end points in the case that ${\rm deg}(W)\ge
\max({\rm deg}(X),{\rm deg}(Y))$.

\begin{rmk}
{\rm
Note that equations (\ref{Pperp}) and (\ref{Rperp}) have an
obvious geometric interpretation: in each case, the roots of the
polynomial $P_\perp(u)$ identify points of the curve where lines
drawn from ${\bf p}$ meet ${\bf r}(u)$ orthogonally. The distance
(\ref{distance}) is then simply the smallest of the lengths of
these perpendiculars (and the chords drawn from ${\bf p}$ to
the affine end points of ${\bf r}(u)$, if any). Insisting that
${\bf r}(u)$ have a regular parameterization guarantees that
$P_\perp(u)$ will not vanish in degenerate cases where $x'(u)=
y'(u)=0$ --- which do not, in general, identify perpendiculars
to ${\bf r}(u)$ from ${\bf p}$.
}
\end{rmk}

The preceding characterizations of the distance function for
polynomial and rational curves extends to arbitrary analytic
curves. In general, we write
\be \label{distance3}
\dist({\bf p},{\bf r}(u)) \,=\,
\min_{k \,\in\, \{1,\ldots,N\}} \, |\,{\bf p}-{\bf r}(u_k)\,| \,,
\ee
where $u_1,\ldots,u_N$ identify all the points on the analytic
curve ${\bf r}(u)$ where a line drawn from ${\bf p}$ meets the
curve orthogonally (and the curve does not cross its tangent line),
as well as the affine end points of ${\bf r}(u)$ (if any). In the
case of an arbitrary analytic curve, of course, the determination
of these parameter values will generally be more difficult than
isolating and approximating the odd roots of the polynomials
(\ref{Pperp}) and (\ref{Rperp}) in the case of polynomial
and rational curves.

We now note some important properties of the distance function:

\begin{propn}
The function $\dist({\bf p},{\bf r}(u))$ varies continuously with
the location of the point ${\bf p}$, but is not always differentiable
with respect to ${\bf p}$, when ${\bf r}(u)$ is a regular polynomial
or rational curve.
\end{propn}

\prf The continuity of $\dist({\bf p},{\bf r}(u))$ follows immediately
from a more general result concerning the distance between a point
${\bf p}$ and a non--empty set ${\cal S}$ in a metric space (see \cite
{kelly79}, Theorem 3, p.~53). However, it is instructive to examine
this property in greater detail within the present context. For the
sake of brevity we discuss only the case of a polynomial curve
${\bf r}(u)$ below; the extension to a rational curve is relatively
straightforward.

Consider expression (\ref{Pperp}) as a polynomial in {\it three\/}
variables, namely, the parameter value $u$ and the coordinates
$(\alpha,\beta)$ of
the point ${\bf p}$:
\be \label{Pperp2}
P_\perp(u,\alpha,\beta) \,=\,
[\,\alpha-X(u)\,]\,X'(u) \,+\, [\,\beta-Y(u)\,]\,Y'(u) \,.
\ee
At the reference point $(\alpha_0,\beta_0)$, let $u_{k,0} \in I$ be a
simple root of (\ref{Pperp2}), so that $\partial P_\perp/\partial u
\not=0$ when $u=u_{k,0}$. Then by the {\it implicit function theorem\/}
\cite{buck78}, we deduce the existence of a function $\phi_k(\alpha,
\beta)$, analytic in some two--dimensional neighborhood ${\cal N}_k$
of $(\alpha_0,\beta_0)$, such that $\phi_k(\alpha_0,\beta_0)=u_{k,0}$
and
\be
P_\perp(\phi_k(\alpha,\beta),\alpha,\beta) \equiv 0
\quad {\rm for\ all\ } (\alpha,\beta) \in {\cal N}_k \,.
\ee
Intuitively, the function $\phi_k(\alpha,\beta)$ describes how the
root $u_k$ of $P_\perp(u)$ moves in the vicinity of its nominal
value $u_{k,0}$ as the point ${\bf p}=(\alpha,\beta)$ executes any
path within the neighborhood ${\cal N}_k$ of its nominal location
${\bf p}_0=(\alpha_0,\beta_0)$.

If $u_k$ represents a finite end point of the parameter interval
$I$ on which ${\bf r}(u)$ is defined, rather than a root of (\ref
{Pperp2}), it can be incorporated into the above framework by
simply taking $\phi_k(\alpha,\beta) \equiv u_k$.

Thus, about any nominal location ${\bf p}_0=(\alpha_0,\beta_0)$, we
may invoke (\ref{distance2}) to formulate the distance function in
a neighborhood of that location as
\be \label{distance4}
\dist({\bf p},{\bf r}(u)) \,=\,
\min_{k \,\in\, \{1,\ldots,N\}} \,
|\,{\bf p}-{\bf r}(\phi_k({\bf p}))\,|
\quad {\rm for\ all\ } {\bf p} \in {\cal N} \,,
\ee
where ${\cal N}=\bigcap\,{\cal N}_k$ represents the area common
to each of the neighborhoods of ${\bf p}_0=(\alpha_0,\beta_0)$ in
which the root functions $\phi_k({\bf p})=\phi_k(\alpha,\beta)$ are
analytic.

In the formulation (\ref{distance4}), the continuity of $\dist
({\bf p},{\bf r}(u))$ with respect to ${\bf p}=(\alpha,\beta)$ at
the (arbitrary) reference point $(\alpha_0,\beta_0)$ is now apparent:
each of the terms
\be \label{distancek}
|\,{\bf p}-{\bf r}(\phi_k({\bf p}))\,| \,=\,
\sqrt{ \, [\,\alpha-X(\phi_k(\alpha,\beta))\,]^2
         + [\,\beta-Y(\phi_k(\alpha,\beta))\,]^2 }
\ee
is continuous with respect to ${\bf p}$ at $(\alpha_0,\beta_0)$, since
the functions $\phi_k(\alpha,\beta)$ are analytic there and the curve
${\bf r}(u)=\{X(u),Y(u)\}$ is continuous everywhere, and although the
index $k$ that achieves the minimum in (\ref{distance4}) may suddenly
jump --- from $i$ to $j$, say --- as we move through $(\alpha_0,\beta_0)$,
we nevertheless have $|\,{\bf p}-{\bf r}(\phi_i({\bf p}))\,|=|\,{\bf p}
-{\bf r}(\phi_j({\bf p}))\,|$ at any such jump.

If such a jump occurs in traversing $(\alpha_0,\beta_0)$, however,
$\dist({\bf p},{\bf r}(u))$ will not, in general, be differentiable
with respect to ${\bf p}$ there. To see why, we consider the {\it
directional derivative}
\be \label{vnabla}
\nabla_{\bf v} \, \dist({\bf p},{\bf r}(u))
\,=\, {1 \over |{\bf v}|} \left[
v_x {\partial \over \partial\alpha} +
v_y {\partial \over \partial\beta} \, \right]
\dist({\bf p},{\bf r}(u)) \,,
\ee
which measures the rate of change of the distance function in the
direction of the vector ${\bf v}=(v_x,v_y)$ at the point $(\alpha_0,
\beta_0)$ at which the partial derivatives in (\ref{vnabla}) are
evaluated. Now by formal differentiation using the chain rule, and
noting that $P_\perp(\phi_k(\alpha,\beta),\alpha,\beta)=0$, the
partial derivatives of the functions $\Delta_k(\alpha,\beta)=|\,
{\bf p}-{\bf r}(\phi_k({\bf p}))\,|$ given by (\ref{distancek})
may be expressed as
\be \label{pderivs}
{\partial\Delta_k \over \partial\alpha}
= {\alpha-X(\phi_k(\alpha,\beta)) \over \Delta_k(\alpha,\beta)}
\quad {\rm and} \quad
{\partial\Delta_k \over \partial\beta}
= {\beta-Y(\phi_k(\alpha,\beta)) \over \Delta_k(\alpha,\beta)} \,.
\ee
In (\ref{pderivs}) it is understood that $k$ represents the index
minimizing $\Delta_k(\alpha,\beta)$, and if $k$ jumps from $i$ to $j$
on passing through $(\alpha_0,\beta_0)$ in the direction ${\bf v}$, it
is in general true that
\be
X(\phi_i(\alpha_0,\beta_0)) \not= X(\phi_j(\alpha_0,\beta_0))
\quad {\rm and} \quad
Y(\phi_i(\alpha_0,\beta_0)) \not= Y(\phi_j(\alpha_0,\beta_0)) \,,
\ee
although $\Delta_i(\alpha_0,\beta_0)=\Delta_j(\alpha_0,\beta_0)$.
Therefore, the magnitude of the derivative (\ref{vnabla}) jumps
for any direction ${\bf v}$ in which we traverse a point $(\alpha_0,
\beta_0)$ with which we associate a jump in the index $k$ that realizes
the mimimum value on the right--hand side of (\ref{distance2}). \QED

\section{Offset curves and bisectors}
\label{offsets}

In formulating a tractable representation for the bisector of a
point ${\bf p}$ and a regular parametric curve ${\bf r}(u)$, it will
be useful to recall the definition and some basic properties of the
{\it offset curves\/} to ${\bf r}(u)$ (see \cite{farouki90a} and
\cite{farouki90b} for a thorough discussion).

\subsection{Fixed--distance offsets}

We begin by noting that if the curve ${\bf r}(u)$ is regular on the
interval $u \in I$, its unit normal vector
\be \label{normal}
{\bf n}(u) \,=\, {(y'(u),-x'(u)) \over \sqrt{x'^2(u)+y'^2(u)}}
\ee
is defined and continuous for all $u \in I$.

\begin{dfn}
The untrimmed offset at (signed) distance $d$ to a regular parametric
curve ${\bf r}(u)$ is the locus defined by
\be \label{offset}
{\bf r}_o(u) \,=\, {\bf r}(u) + d\,{\bf n} (u) \,.
\ee
\end{dfn}

Note that when ${\bf r}(u)$ is a polynomial or rational curve, the
offset ${\bf r}_o(u)$ is not, in general, a polynomial or rational
curve, because of the radical expression in the denominator of (\ref
{normal}). Consequently, the offset curve is often approximated by
piecewise--polynomial forms \cite{hoschek88,klass83,pham88,tiller84}
in practical computer--aided design (CAD) applications, to render it
compatible with their representational and algorithmic infrastructure.
(The ``interior'' and ``exterior'' offsets, at distances $-d$ and
$+d$, together form an {\it algebraic curve\/} with an implicit
polynomial equation $f_o(x,y)=0$ \cite{farouki90b}. See also \cite
{farouki90c,farouki91a} for discussion of a special class of
polynomial curves whose offsets are rational.)

We call the locus (\ref{offset}) the ``untrimmed'' offset for the
following reason: {\it Corresponding points\/} ${\bf r}(u)$ and
${\bf r}_o(u)$ on the given curve and its untrimmed offset are
evidently distance $d$ apart, measured along their mutual normal
direction. However, the point ${\bf r}_o(u)$ of the untrimmed offset
is not necessarily distance $d$, in the sense of the distance function
(\ref{distance}), from the {\it entire curve\/} ${\bf r}(u)$. We call
the locus having this latter property the ``trimmed'' offset to
${\bf r}(u)$, since it is obtained by deleting certain continuous
segments of (\ref{offset}).

The trimming procedure may be characterized by the following
property of the untrimmed offset curve (\ref{offset}):

\begin{propn} \label{offtrim}
For a regular polynomial or rational curve ${\bf r}(u)$ defined on
the interval $u \in I$, let $\{i_1,\ldots,i_M\}$ be the ordered set
of parameter values on $I$ that correspond to self--intersections
of its untrimmed offset ${\bf r}_o(u)$ at distance $d$, i.e.,
${\bf r}_o(i_j)={\bf r}_o(i_k)$ for some $1 \le j\not=k \le N$.
Then, denoting the end points of $I$ by $u_0$ and $u_{M+1}$, we
have either
\be
\dist({\bf r}_o(t),{\bf r}(u)) \,\equiv\, d \quad {\rm for\ all\ }
t \in (i_k,i_{k+1})
\ee
or
\be
\dist({\bf r}_o(t),{\bf r}(u)) \,<\, d \quad {\rm for\ all\ }
t \in (i_k,i_{k+1})
\ee
on each span $(i_k,i_{k+1})$ for $k=0,\ldots,M$ between successive
self--intersections of the untrimmed offset.
\end{propn}

\prf See Theorem 4.4 in \cite{farouki90a}. \QED

Proposition \ref{offtrim} indicates that if we dissect the
untrimmed offset ${\bf r}_o(u)$ into the subsegments delineated by
its self--intersections, then each subsegment should be retained
or discarded in its entirety in forming the trimmed offset. It is
sufficient to test the distance of a single point interior to each
span $(i_k,i_{k+1})$ of ${\bf r}_o(u)$ from the given curve ${\bf r}
(u)$ (the mid point $\half(i_k+i_{k+1})$, say) to determine whether
or not that span should be eliminated.

Note that trimming an offset curve is a problem in the {\it global
topology\/} of the locus defined by (\ref{offset}); we know of no
simpler algorithm for the trimming process than the methodical
dissect--and--test procedure described above.

The fact that the untrimmed offset (\ref{offset}) is {\it not\/} a
rational curve complicates somewhat the problem of determining the
parameter values ${i_1,\ldots,i_M}$ of its self--intersections; see
\cite{farouki90b}. We shall encounter a similar ``trimming'' problem
in computing point/curve bisectors which, suprisingly, proves to be
much simpler because the untrimmed curve happens to be rational (see
below).

\begin{rmk}
{\rm
The trimmed offsets at distance $\pm d$ to a given curve ${\bf r}(u)$
are the ``level curves'' for the point/curve distance function $(\ref
{distance})$, i.e., they are the loci of points ${\bf p}$ that satisfy
$\dist({\bf p},{\bf r}(u))=|d|$.
}
\end{rmk}

\subsection{The untrimmed point/curve bisector}

We can generalize the notion of an (untrimmed) offset curve at fixed
distance $d$ to a given regular curve ${\bf r}(u)$ by substituting
any continuous function $d(u)$ of the parameter $u$ in place of the
constant $d$. The differentiability of the {\it variable--distance
offset curve}
\be \label{varoffset}
{\bf r}_o(u) \,=\, {\bf r}(u) + d(u) {\bf n}(u)
\ee
is then constrained by that of the ``displacement function'' $d(u)$.
We shall find the form (\ref{varoffset}) to be valuable in analyzing
point/curve bisectors.

Consider the {\it family of normal lines\/} to a given regular curve
${\bf r}(u)$. These lines may be parameterized in the form
\be \label{nline}
{\bf r}(u) + \lambda\,{\bf n}(u) \,,
\ee
where $u$ selects a point on the curve, and $\lambda$ measures the
signed distance along the normal line from that point.

Given any point ${\bf p}$ not on ${\bf r}(u)$, the location on the
normal line (\ref{nline}) that is equidistant from ${\bf p}$ and the
curve point ${\bf r}(u)$ is uniquely identified by the condition
\be
\lambda \,=\, |\,{\bf r}(u)+\lambda\,{\bf n}(u)-{\bf p}\,| \,.
\ee
Now for each $u$, let $d(u)$ denote the unique value $\lambda$
satisfying this condition. Using (\ref{normal}), it may be verified
that
\be \label{du}
d(u) \,=\, {|\,{\bf p}-{\bf r}(u)\,|^2 \over
2\,({\bf p}-{\bf r}(u))\dotpr{\bf n}(u)} \,,
\ee
which is evidently {\it not\/} (in general) a rational function of
$u$, because of the radical incurred in computing the unit normal
vector ${\bf n}(u)$.

\begin{dfn}
The {\it untrimmed bisector\/} of a point ${\bf p}$ and a regular
curve ${\bf r}(u)$ is the variable--distance offset $(\ref{varoffset})$
to ${\bf r}(u)$ defined by the function $(\ref{du})$.
\end{dfn}

The curve defined by (\ref{varoffset}) and (\ref{du}) is simply
the locus of points on the normal lines (\ref{nline}) which are
equidistant from each curve point ${\bf r}(u)$ and the given point
${\bf p}$. The motivation for the name we have attached to this
curve will become apparent below.

\begin{rmk}
{\rm
When ${\bf r}(u)$ is a polynomial or rational curve, the untrimmed
bisector defined by $(\ref{varoffset})$ and $(\ref{du})$ has a
{\it rational\/} parameterization, since the radicals in $d(u)$
and ${\bf n}(u)$ cancel precisely.
}
\end{rmk}

Let $\psi(u)$ denote the angle between the vector from ${\bf r}(u)$
to ${\bf p}$ and the curve normal ${\bf n}(u)$, measured in the
right--handed sense defined by a unit vector ${\bf z}$ orthogonal
to the plane of the curve. Then the function $d(u)$ may also be
written as
\be
d(u) \,=\, \half\,|\,{\bf p}-{\bf r}(u)\,|\sec\psi(u) \,,
\ee
where we take $-\pi/2<\psi(u)\le\pi/2$.

Note that the displacement function satisfies $d(u)\not=0$ for
all $u$ if the given point ${\bf p}$ does not lie on the curve
${\bf r}(u)$. However, at each parameter value $\tau$ such that
$({\bf p}-{\bf r}(\tau))\dotpr{\bf n}(\tau)=0$ (i.e., the curve
normal at ${\bf r}(\tau)$ is orthogonal to the vector from
${\bf r}(\tau)$ to ${\bf p}$) the untrimmed bisector exhibits a
``point at infinity.'' For the polynomial curve (\ref{polycurve}),
the parameter values corresponding to these points at infinity
are the roots of the polynomial
\be \label{Pinf}
P_\infty(u) \,=\,
[\,\alpha-X(u)\,]\,Y'(u) \,-\, [\,\beta-Y(u)\,]\,X'(u) \,,
\ee
which is of degree $2n-1$ (at most) when ${\bf r}(u)$ is of
degree $n$. Similarly, for the rational curve (\ref{ratcurve}),
the polynomial whose roots identify points at infinity on the
untrimmed bisector is
\ba \label{Rinf}
P_\infty(u)
&=& [\,\alpha W(u)-X(u)\,]\,[\,W(u)Y'(u)-W'(u)Y(u)\,] \nonumber \\
&-& [\,\beta  W(u)-Y(u)\,]\,[\,W(u)X'(u)-W'(u)X(u)\,] \,.
\ea

We will denote the untrimmed bisector defined by (\ref{varoffset})
and (\ref{du}) by ${\bf b}(u)$, with homogeneous coordinates given
by the polynomials $X_b(u)$, $Y_b(u)$, $W_b(u)$. For the polynomial
curve (\ref{polycurve}), it may be verified that
\ba \label{pbsctr}
X_b &=& [\,\alpha^2-X^2+(\beta-Y)^2\,]\,Y'
 \,-\, 2(\beta-Y)XX' \,, \nonumber \\
Y_b &=& 2(\alpha-X)YY'
 \,-\, [\,(\alpha-X)^2+\beta^2-Y^2\,]\,X' \,, \nonumber \\
W_b &=& 2\,[\,(\alpha-X)Y'-(\beta-Y)X'\,] \,,
\ea
while in the case of the rational curve (\ref{ratcurve}) we have
\ba \label{rbsctr}
X_b &=& [\,\alpha^2W^2-X^2+(\beta W-Y)^2\,]\,V
 \,-\, 2(\beta W-Y)XU \,, \nonumber \\
Y_b &=& 2(\alpha W-X)YV
 \,-\, [\,(\alpha W-X)^2+\beta^2W^2-Y^2\,]\,U \,, \nonumber \\
W_b &=& 2W\,[\,(\alpha W-X)V-(\beta W-Y)U\,] \,,
\ea
where $U=WX'-W'X$ and $V=WY'-W'Y$.

\begin{rmk}
{\rm
It may be verified from (\ref{pbsctr}) and (\ref{rbsctr}) that when
${\bf r}(u)$ is a {\it polynomial\/} curve of degree $n$, the untrimmed
bisector ${\bf b}(u)$ is a rational curve of degree $3n-1$ at most,
whereas if ${\bf r}(u)$ is a {\it rational\/} curve of degree $n$, the
untrimmed bisector is of degree $4n-1$ at most.
}
\end{rmk}

\begin{exmpl}
{\rm
The simplest polynomial curve, other than a straight line, is the
parabola. Consider the case ${\bf p}=(\alpha,\beta)$ and ${\bf r}(u)=
(u,u^2)$. From equations (\ref{pbsctr}), we have the representation
\ba \label{Bprbla}
X_b(u) &=& 2u\,[\,u^4-2\beta u^2+\alpha^2+\beta^2-\beta\,] \,,
\nonumber \\
Y_b(u) &=& -\ 3u^4+4\alpha u^3-u^2+2\alpha u-\alpha^2-\beta^2 \,,
\nonumber \\
W_b(u) &=& -\ 2u^2+2\alpha u-2\beta \,,
\ea
for the untrimmed bisector, which is evidently a rational curve of
degree {\it five}. Figure 2 illustrates representative examples of
the curves defined by (\ref{Bprbla}).
} \QED
\end{exmpl}

\begin{exmpl}
{\rm
As a simple example of the bisector of a point and a rational curve,
we consider the case where $C$ is an ellipse centered on the origin,
with semi--axes $a$ and $b$. Then $C$ has the rational parameterization
\be
X(u) \,=\, a(1-u^2) \,, \quad
Y(u) \,=\, 2bu \,, \quad
W(u) \,=\, 1+u^2 \,.
\ee
Substituting the above into (\ref{rbsctr}), we find that the untrimmed
bisector is a rational curve of degree {\it six}, defined by
\ba \label{Bellps}
X_b(u)
&=& (1-u^2)\,
[\,(\alpha^2+\beta^2-a^2)bu^4 \,+\,
4\beta(a^2-b^2)u^3 \nonumber \\
&+& \!\! 2(\alpha^2+\beta^2-3a^2+2b^2)bu^2
\,+\, 4\beta(a^2-b^2)u
\,+\, (\alpha^2+\beta^2-a^2)b\,] \,, \nonumber \\
Y_b(u)
&=& 2\,u\,
[\,((\alpha^2+\beta^2)a+2\alpha(a^2-b^2)+(a^2-2b^2)a)u^4 \nonumber \\
&+& \!\! 2(\alpha^2+\beta^2-a^2)au^2
\,+\, (\alpha^2+\beta^2)a-2\alpha(a^2-b^2)+(a^2-2b^2)a\,] \,, \nonumber \\
W_b(u)
&=& 2(1+u^2)^2\,
[-\,(\alpha+a)bu^2
\,+\, 2\beta au
\,+\,(\alpha-a)b\,] \,.
\ea
Some examples are illustrated in Figure 3.
} \QED
\end{exmpl}

\subsection{Irregular points of the untrimmed bisector}

It is evident from Figures 2 and 3 that in general the untrimmed
bisector of a point ${\bf p}$ and a regular curve ${\bf r}(u)$ is
{\it not\/} a smooth locus, even though ${\bf r}(u)$ is necessarily
smooth if it has a regular parameterization.

Recall \cite{kreyszig59} that for any regular parametric curve
${\bf r}(u)$, the elementary differential characteristics at each
point may be expressed in terms of the parametric derivatives
${\bf r}'(u),{\bf r}''(u),\ldots$ there as
\be \label{diffchar}
{\bf t} \,=\, {{\bf r}' \over |{\bf r}'|} \,, \quad
{\bf n} \,=\, {\bf t} \cross {\bf z} \,, \quad
\kappa \,=\, {({\bf r}' \cross {\bf r}'') \dotpr {\bf z}
  \over |{\bf r}'|^3} \,,
\ee
${\bf z}$ being a unit vector orthogonal to the plane of the curve.
The {\it normal\/} ${\bf n}(u)$ and {\it tangent\/} ${\bf t}(u)$ form
an orthonormal basis $({\bf n},{\bf t},{\bf z})$ with ${\bf z}$ at
each point $u$, while $\kappa(u)$ is the (signed) {\it curvature}.
The variation of the tangent and normal along the curve is described
by the {\it Frenet equations} \cite{kreyszig59},
\be \label{frenet}
{\bf t}' \,=\, -\,\sigma\kappa\,{\bf n}
\quad {\rm and} \quad
{\bf n}' \,=\, \sigma\kappa\,{\bf t} \,.
\ee
Higher-order derivatives of ${\bf t}(u)$ and ${\bf n}(u)$ are readily
expressed in terms of the vectors ${\bf t}(u)$, ${\bf n}(u)$ and the
scalar functions $\sigma(u)$, $\kappa(u)$ and their derivatives; for
example:
\ba \label{frenet2}
{\bf t}'' &=& -\,\sigma^2\kappa^2\,{\bf t} \,-\,
(\sigma'\kappa+\sigma\kappa')\,{\bf n} \,, \nonumber \\
{\bf n}'' &=& (\sigma'\kappa+\sigma\kappa')\,{\bf t}
 \,-\, \sigma^2\kappa^2\,{\bf n} \,.
\ea

If we denote by ${\bf b}(u)$ the parametric representation of the
untrimmed bisector obtained by substituting from (\ref{du}) into
(\ref{varoffset}), then the parametric derivatives of ${\bf b}(u)$
may be written as
\ba \label{bderivs}
{\bf b}' &=& {\bf r}' \,+\,
  d'\,{\bf n} \,+\, d\,{\bf n}' \,, \nonumber \\
{\bf b}'' &=& {\bf r}'' \,+\,
  d''\,{\bf n} \,+\, 2d'\,{\bf n}' \,+\, d\,{\bf n}'' \,,
\ea
$\ldots$ etc. By substituting from (\ref{frenet}) and (\ref{frenet2}),
we can re--write (\ref{bderivs}) as
\ba \label{bderivs2}
{\bf b}' &=& \sigma(1+\kappa d\,)\,{\bf t} \,+\, d'\,{\bf n} \,,
 \nonumber \\
{\bf b}'' &=&
[\,\sigma'(1+\kappa d\,)+\sigma(\kappa'd+2\kappa d')\,]\,{\bf t}
 \nonumber \\
 & & \quad +\ [\,d''-\sigma^2\kappa(1+\kappa d\,)\,]\,{\bf n} \,,
\ea
where the derivatives $d'$ and $d''$ of the displacement function
(\ref{du}) appropriate to the untrimmed bisector are most conveniently
expressed in the form
\ba \label{dderivs}
d'  &=& -\ \sigma(1+\kappa d\,)\tan\psi \,,
\nonumber \\
d'' &=& -\ [\,\sigma'(1+\kappa d)+\sigma\kappa'd\,]\tan\psi
\nonumber \\
    & & \quad +\ \half\sigma^2(1+\kappa d\,)d^{-1}
    [\,1+2\kappa d+(1+4\kappa d)\tan^2\psi\,] \,,
\ea
$\ldots$ etc. In deriving (\ref{dderivs}), we make use of the fact
that the angle $\psi$ between the vector from ${\bf r}(u)$ to ${\bf p}$
and the curve normal ${\bf n}(u)$ changes at the rate
\be \label{psideriv}
\psi' \,=\, -\,\half\sigma(1+2\kappa d\,)d^{-1}
\ee
with respect to $u$, as may be deduced by differentiating expression
(\ref{du}).

Using the above expression for $d'$, we now see that the first
parametric derivative of the untrimmed bisector has the form:
\be \label{bprime}
{\bf b}' \,=\, \sigma(1+\kappa d\,)\,({\bf t}-{\bf n}\tan\psi) \,.
\ee
At each value of $u$ such that ${\bf b}'(u)\not={\bf 0}$, the
tangent ${\bf t}_b(u)$ to the untrimmed bisector is a unit vector
in the direction of (\ref{bprime}). Note that the magnitude of (\ref
{bprime}) is simply
\be \label{magbprime}
|{\bf b}'| \,=\, \sigma\,|1+\kappa d\,|\,|\!\sec\psi| \,.
\ee

Since the given curve ${\bf r}(u)$ is regular, its unit tangent
and normal ${\bf t}(u)$ and ${\bf n}(u)$ are defined and linearly
independent at each $u$, so that ${\bf t}(u)-{\bf n}(u)\tan\psi(u)$
is never the zero vector. Moreover, this vector varies continuously
with $u$ except at those parameter values where $\psi(u)=\pm\pi/2$,
which correspond to the roots of the polynomial (\ref{Pinf}),
i.e., the points at infinity on ${\bf b}(t)$. Therefore, at each
$u$ such that $P_\infty(u)\not=0$, the unit vector:
\be \label{evector}
{\bf e}(u) \,=\, |\!\cos\psi(u)| \,
[\,{\bf t}(u)-{\bf n}(u)\tan\psi(u)\,]
\ee
in the direction of ${\bf t}(u)-{\bf n}(u)\tan\psi(u)$ is well
defined and varies continuously with $u$. Since for a regular curve
$\sigma(u)\not=0$ for all $u$, we may deduce from (\ref{diffchar})
that the tangent to the untrimmed bisector is given in terms of
${\bf e}(u)$ by:
\be \label{btangent}
{\bf t}_b(u) \,=\,
{1+\kappa(u)d(u) \over |\,1+\kappa(u)d(u)\,|} \; {\bf e}(u) \,.
\ee

\begin{lma}
The untrimmed bisector ${\bf b}(u)$ exhibits a cusp, or sudden
tangent inversion, at those parameter values where $P_\infty(u)
\not=0$ and the curvature $\kappa(u)$ of the given curve ${\bf r}
(u)$ attains the local critical value
\be \label{kappacrit}
\kappa_c(u) \,=\, -\ {1 \over d(u)} \,=\,
-\ {2\cos\psi(u) \over |\,{\bf p}-{\bf r}(u)\,|} \,,
\ee
without being an extremum, i.e., $\kappa'(u)\not=0$.
\end{lma}

\prf Let $\tau$ be a parameter value such that $P_\infty(\tau)
\not=0$ (i.e., ${\bf b}(\tau)$ is an affine point of the untrimmed
bisector), and the curvature of the given curve ${\bf r}(u)$
satisfies both $\kappa(\tau)=-1/d(\tau)$ and $\kappa'(\tau)\not=0$.
Then $\tan\psi(\tau)$ is finite, and the unit vector ${\bf e}(\tau)$
given by (\ref{evector}) is defined and continuous at $u=\tau$.
On the other hand, the scalar factor multiplying ${\bf e}(u)$ in
(\ref{btangent}) is a ``step function,'' which changes abruptly
from $-1$ to $+1$, or vice--versa, at $u=\tau$ whenever $d/du(1+
\kappa(u)d(u))\not=0$ at $u=\tau$, i.e., $\kappa'(\tau)d(\tau)+
\kappa(\tau)d'(\tau)\not=0$. But from (\ref{dderivs}) we observe
that $d'(\tau)=0$ whenever $\kappa(\tau)=-1/d(\tau)$, and since we
certainly have $d(\tau)\not=0$ because ${\bf p}$ does not lie on
${\bf r}(u)$, the condition $d/du(1+\kappa(u)d(u))\not=0$ at $u=
\tau$ is exactly equivalent to $\kappa'(\tau)\not=0$. \QED

\begin{rmk}
{\rm
It is interesting to note that the criteria $\kappa(u)=-1/d(u)$
and $\kappa'(u)\not=0$ identifying the cusps of the ``variable
offset'' $(\ref{varoffset})$ are identical to those for a simple
offset ($\kappa(u)=-1/d$, $\kappa'(u)\not=0$) with $d(u)={\rm constant}$
$\cite{farouki90a}$. This is not a {\it generic} feature of variable
offsets, but arises rather from the specific form $(\ref{du})$ of
$d(u)$ for the untrimmed bisector.
}
\end{rmk}

The local critical curvature (\ref{kappacrit}) is attained at those
roots of the equation
\be
P_c(u) \,=\,
|\,{\bf p}-{\bf r}(u)\,|^2\,{\bf r}'(u)\cross{\bf r}''(u) \,+\,
2\,|{\bf r}'(u)|^2\,[\,{\bf p}-{\bf r}(u)\,]\cross{\bf r}'(u)
\,=\, 0
\ee
that satisfy $\kappa'(u)\not=0$. These parameter values correspond
to cusps on the untrimmed bisector ${\bf b}(u)$. When ${\bf r}(u)$
is the polynomial curve (\ref{polycurve}), $P_c(u)$ is a polynomial
of degree $??$ in $u$. When ${\bf r}(u)$ is the rational curve
(\ref{ratcurve}), $P_c(u)$ is a rational function in $u$, whose
numerator is of degree $??$.

Consider now the behavior of the curvature $\kappa_b(u)$ along the
untrimmed bisector. By substituting from (\ref{bderivs2}) and (\ref
{dderivs}), a straightforward but laborious calculation gives
\be
({\bf b}'\cross{\bf b}'')\dotpr{\bf z}
\,=\, \half\sigma^3(1+\kappa d\,)^2 d^{-1}
\,[\,(1+4\kappa d\,)\tan^2\psi-1\,] \,,
\ee
and together with (\ref{magbprime}) the expression $\kappa_b(u)=
|{\bf b}'(u)|^{-3}[\,{\bf b}'(u)\cross{\bf b}''(u)\,]\dotpr{\bf z}$
for the curvature of the untrimmed bisector, as a function of the
parameter $u$, reduces to
\be \label{bkappa}
\kappa_b(u) \,=\, {(1+4\kappa d\,)\tan^2\psi \,-\, 1
 \over 2d\,|1+\kappa d\,|\,|\!\sec\psi|^3} \,.
\ee
It is worthwhile emphasizing the siginificance of equation (\ref
{bkappa}) in words: at any point $u$ on the untrimmed bisector, we
can express the curvature $\kappa_b(u)$ of ${\bf b}(u)$ simply in
terms of the curvature $\kappa(u)$ of the given curve ${\bf r}(u)$,
the displacement function $d(u)$ defined by (\ref{du}), and the
angle $\psi(u)$ between the vector ${\bf p}-{\bf r}(u)$ and the
curve normal ${\bf n}(u)$.

\begin{lma}
In general, the untrimmed bisector ${\bf b}(u)$ has an extraordinary
point --- i.e., a tangent--continuous point of infinite curvature
--- at those parameter values $u$ where $P_\infty(u)\not=0$ and
the curvature $\kappa(u)$ of the given curve ${\bf r}(u)$ attains
a local extremum $(\kappa'(u)=0\ but\ \kappa''(u)\not=0)$, equal
in value to the critical curvature $-\,1/d(u)$ defined by $(\ref
{kappacrit})$.
\end{lma}

\prf Let $\tau$ be such that $P_\infty(\tau)\not=0$ and the
curvature of ${\bf r}(u)$ satisfies $\kappa(\tau)=-1/d(\tau)$ with
$\kappa'(\tau)=0$ and $\kappa''(\tau)\not=0$. Then the unit vector
${\bf e}(\tau)$ given by (\ref{evector}) is continuous at $u=\tau$
and, by the same arguments as in the preceding Lemma, the scalar
factor $(1+\kappa d\,)/|1+\kappa d\,|$ multiplying it in (\ref
{btangent}) is of the {\it same\/} sign on either side of $\tau$.
Therefore, ${\bf t}_b(u)$ is continuous at $u=\tau$. However, since
$1+\kappa(\tau)d(\tau)=0$, it is evident from (\ref{bkappa}) that
the curvature $\kappa_b(u)$ of the untrimmed bisector increases
without bound as we approach $\tau$ (since the numerator of (\ref
{bkappa}) is, in general, non--zero at such a point). \QED

\section{The trimming procedure}
\label{trimming}

\subsection{Self--intersections}

We have seen that the untrimmed bisector of a point ${\bf p}$ and
a polynomial or rational curve ${\bf r}(u)$ of degree $n$ is a
rational curve ${\bf u}$ of degree $m=3n-1$ (if ${\bf r}(u)$ is
polynomial) or $m=4n-1$ (if ${\bf r}(u)$ is rational).

A self--intersection of the untrimmed bisector ${\bf b}(u)$ arises
when two or more distinct parameter values correspond to the same
geometric point on its locus (with this definition we include the
exceptional cases where ${\bf b}(u)$ ``touches'' itself among its
self--intersections).

In order to compute the self--intersections, we are interested in
identifying all parameter values $u$ that satisfy the vector equation
\be \label{selfint}
{\bf b}(u+\xi) \,=\, {\bf b}(u) \quad {\rm for\ some\ } \xi\not=0 \,.
\ee
Let $X_b(u)$, $Y_b(u)$, and $W_b(u)$ be the polynomials that give the
homogeneous coordinates of ${\bf b}(u)$, according to (\ref{pbsctr})
or (\ref{rbsctr}), as appropriate. Then we have:

\begin{propn}
Let polynomials $P_k(u)$ and $Q_k(u)$ be defined in terms of $X_b(u)$,
$Y_b(u)$, $W_b(u)$ and their $r$--th derivatives $X_b^{(r)}(u)$,
$Y_b^{(r)}(u)$, $W_b^{(r)}(u)$ by
\ba \label{pkandqk}
P_k(u) &=& {W_b(u)X_b^{(k+1)}(u)-W_b^{(k+1)}(u)X_b(u) \over (k+1)!} \,,
\nonumber \\
Q_k(u) &=& {W_b(u)Y_b^{(k+1)}(u)-W_b^{(k+1)}(u)Y_b(u) \over (k+1)!}
\ea
for $k=0,\ldots,m-1$. Then the parameter values that satisfy $(\ref
{selfint})$ coincide with the roots of the polynomial $I(u)$ defined
by the deteminant
\be \label{Iresltnt}
I \,=\, \left|\, \matrix{
 P_0 &P_1 &\cdot &\cdot &P_{m-1} &{} &{} &{} \cr
 {} &P_0 &P_1 &\cdot &\cdot &P_{m-1} &{} &{} \cr
 {} &{} &\cdot &\cdot &\cdot &\cdot &\cdot &{} \cr
 {} &{} &{} &P_0 &P_1 &\cdot &\cdot &P_{m-1} \cr
 Q_0 &Q_1 &\cdot &\cdot &Q_{m-1} &{} &{} &{} \cr
 {} &Q_0 &Q_1 &\cdot &\cdot &Q_{m-1} &{} &{} \cr
 {} &{} &\cdot &\cdot &\cdot &\cdot &\cdot &{} \cr
 {} &{} &{} &Q_0 &Q_1 &\cdot &\cdot &Q_{m-1} \cr } \,\right| \,.
\ee
\end{propn}

\prf Equation (\ref{selfint}) corresponds to the simultaneous scalar
equations
\ba \label{pandq}
P(u,\xi) &=&
{W_b(u)X_b(u+\xi)-W_b(u+\xi)X_b(u) \over \xi} \,=\, 0 \nonumber \\
Q(u,\xi) &=&
{W_b(u)Y_b(u+\xi)-W_b(u+\xi)Y_b(u) \over \xi} \,=\, 0
\ea
where the division by $\xi$ eliminates the trivial solution $\xi=0$ to
the numerators of expressions (\ref{pandq}), which does not correspond
to a self--intersection. Now by expanding $X_b(u+\xi)$, $Y_b(u+\xi)$,
$W_b(u+\xi)$ as (terminating) Taylor series in $\xi$, we can re--write
$P(u,\xi)$ and $Q(u,\xi)$ as polynomials of degree $m-1$ in $\xi$ whose
coefficients are the polynomials (\ref{pkandqk}) in $u$:
\be \label{pandq2}
P(u,\xi) \,=\, \sum_{k=0}^{m-1} P_k(u)\,\xi^k
\quad {\rm and} \quad
Q(u,\xi) \,=\, \sum_{k=0}^{m-1} Q_k(u)\,\xi^k \,.
\ee
Thus, we are interested in the values of $u$ at which the polynomials
(\ref{pandq2}) have a common root $\xi$. It is well known (see, for
example, \cite{uspensky48}) that the vanishing of the ``resultant''
of these polynomials with respect to $\xi$ expresses a sufficient and
necessary condition for them to have a common root. This resultant,
which we write symbolically as
\be \label{Pslfint}
I(u) \,=\, {\rm Resultant}_\xi (\,P(u,\xi),Q(u,\xi)\,) \,,
\ee
is a polynomial expression in the coefficients $P_k(u)$ and $Q_k(u)$
in (\ref{pandq2}), i.e., it is a polynomial $I(u)$ in $u$. Equation
(\ref{Iresltnt}) expresses this resultant as a Sylvester determinant
in the polynomials $P_k(u)$ and $Q_k(u)$, one of several equivalent
determinantal forms for the resultant. \QED

\begin{exmpl}
{\rm
Consider the ``self--intersection'' polynomial (\ref{Pslfint}) for
the case of Example \ref{offsets}.1 --- i.e., the bisector of the
point ${\bf p}=(\alpha,\beta)$ and the parabola ${\bf r}(u)=(u,u^2)$.
} \QED
\end{exmpl}

\section{Concluding remarks}
\label{conclusion}

\begin{thebibliography}{99}

\bibitem{bruce84}
J.~W.~Bruce and P.~J.~Giblin (1984), {\it Curves and Singularities},
Cambridge University Press.

\bibitem{buck78}
R.~C.~Buck (1978) {\it Advanced Calculus}, Third edition, McGraw--Hill,
New York, 362--367.

\bibitem{coolidge59}
J.~L.~Coolidge (1959), {\it A Treatise on Algebraic Plane Curves},
Dover Publications, New York.

\bibitem{coxeter69}
H.~S.~M.~Coxeter (1969), {\it Introduction to Geometry}, Wiley,
New York.

\bibitem{docarmo76}
M.~P.~do~Carmo (1976), {\it Differential Geometry of Curves and
Surfaces}, Prentice--Hall, Englewood Cliffs, N.J.

\bibitem{eisenhart60}
L.~P.~Eisenhart (1960), {\it Coordinate Geometry}, Dover Publications,
New York (reprint).

\bibitem{farouki89}
R.~T.~Farouki (1989), Hierarchical segmentations of algebraic curves
and some applications, in {\it Mathematical Methods in CAGD}, T.~Lyche
and L.~L.~Schumaker (eds.), Academic Press.

\bibitem{farouki90a}
R.~T.~Farouki and C.~A.~Neff (1990), Analytic properties of plane
offset curves \CAGD{\bf 7}, 83--99.

\bibitem{farouki90b}
R.~T.~Farouki and C.~A.~Neff (1990), Algebraic properties of plane
offset curves, \CAGD{\bf 7}, 100--127.

\bibitem{farouki90c}
R.~T.~Farouki and T.~Sakkalis (1990), Pythagorean hodographs,
\IBMJRD{\bf 34}, 736--752.

\bibitem{farouki91a}
R.~T.~Farouki (1991), Pythagorean--hodograph curves in practical
use, in {\it Geometry Processing\/} (R.~E.~Barnhill, ed.), SIAM,
Philadelphia, to appear.

\bibitem{farouki91b}
R.~T.~Farouki (1991), Watch your (parametric) speed! {\it The
Mathematics of Surfaces IV\/} (A.~Bowyer, ed.), Oxford University
Press, to appear.

\bibitem{hoschek85}
J.~Hoschek (1985), Offset curves in the plane, \CAD{\bf 17}, 77--82.

\bibitem{hoschek88}
J.~Hoschek (1988), Spline approximation of offset curves, \CAGD{\bf 5},
33--40.

\bibitem{hoschek88b}
J.~Hoschek and N.~Wissel (1988), Optimal approximate conversion of
spline curves and spline approximation of offset curves, \CAD{\bf 20},
475--483.

\bibitem{kelly79}
P.~J.~Kelly and M.~L.~Weiss, {\it Geometry and Convexity}, Wiley,
New York.

\bibitem{klass83}
R.~Klass (1983), An offset spline approximation for plane cubic
splines, \CAD{\bf 15}, 297--299.

\bibitem{kreyszig59}
E.~Kreyszig (1959), {\it Differential Geometry}, University of
Toronto Press.

\bibitem{leibniz92}
G.~W.~Leibniz (1692), Generalia de natura linearum, anguloque
contactus et osculi provocationibus aliisque cognatis et eorum
usibus nonnullis, Acta eruditorum: cited in [Kreyszig 1959].

\bibitem{nackman91}
L.~R.~Nackman and V.~Srinivasan (1991), Bisectors of linearly
separable sets, \DCG{\bf ?}, to appear.

\bibitem{pham88}
B.~Pham (1988), Offset approximation of uniform B-splines,
\CAD{\bf 20}, 471--474.

\bibitem{preparata85}
F.~P.~Preparata and M.~I.~Shamos (1985), {\it Computational
Geometry}, Springer, New York.

\bibitem{salmon79}
G.~Salmon (1879), {\it A Treatise on the Higher Plane Curves},
Chelsea, New York (reprint).

\bibitem{sederberg84}
T.~W.~Sederberg (1984), Degenerate parametric curves, \CAGD{\bf 1},
301--307.

\bibitem{sederberg86}
T.~W.~Sederberg (1986), Improperly parameterized rational curves,
\CAGD{\bf 3}, 67--75.

\bibitem{stoker69}
J.~J.~Stoker (1969), {\it Differential Geometry}, Wiley, New York.

\bibitem{tiller84}
W.~Tiller and E.~G.~Hanson (1984), Offsets of two-dimensional
profiles, \IEEECGA{\bf 4} (Sept.), 36--46.

\bibitem{uspensky48}
J.~V.~Uspensky (1948), {\it Theory of Equations}, McGraw--Hill,
New York.

\bibitem{winger62}
R.~M.~Winger (1962), {\it An Introduction to Projective Geometry},
Dover Publications, New York.

\bibitem{yap87}
C.--K.~Yap (1987), An ${\rm O}(n\log n)$ algorithm for the Voronoi
diagram of a set of simple curve segments, \DCG{\bf 2}, 365--393.

\bibitem{yap89}
C.--K.~Yap and H.~Alt (1989), Motion planning in the {\it
CL\/}--Environment, in {\it Lecture notes in computer science 382},
F.~Dehne, J.--R.~Sack, and N.~Santoro (eds.), Springer, New York,
373--380 (Proceedings of the Workshop on Algorithms and Data
Structures WADS~'89, Ottawa, Canada, August 17--19, 1989).

\end{thebibliography}

\section*{Figure captions}

\bigskip
\noindent{\bf Figure 1}:
Representative bisectors of a point and a circle: the bisector is
an ellipse, a straight line, or a hyperbola according to whether the
point lies inside, on, or outside the circle.

\medskip
\noindent{\bf Figure 2}:
Examples of the untrimmed bisector of a point and a parabola.

\medskip
\noindent{\bf Figure 3}:
Uuntrimmed bisectors of a point and an ellipse.

\end{document}

