\documentstyle[12pt,twocolumn]{article}

\newif\ifFull
\Fullfalse

\begin{document}

\newcommand{\single}{\def\baselinestretch{1.0}\large\normalsize}
\newcommand{\double}{\def\baselinestretch{1.5}\large\normalsize}
\newcommand{\triple}{\def\baselinestretch{2.2}\large\normalsize}

\newcommand{\rd}{{\rm d}}
\newcommand{\re}{{\rm e}}
\newcommand{\ri}{{\rm i}}
\newcommand{\sgn}{{\rm sign}}
\newcommand{\half}{\textstyle{1 \over 2}\displaystyle}
\newcommand{\quarter}{\textstyle{1 \over 4}\displaystyle}
\newcommand{\dist}{{\rm dist}}
\newcommand{\cross}{\!\times\!}
\newcommand{\dotpr}{\!\cdot\!}
\newcommand{\vhat}{\hat {\bf v}}
\newcommand{\what}{\hat {\bf w}}
\newcommand{\zhat}{\hat {\bf z}}
\newcommand{\ldash}{\vrule height 3pt width 0.35in depth -2.5pt}
\newcommand{\be}{\begin{equation}}
\newcommand{\ee}{\end{equation}}
\newcommand{\ba}{\begin{eqnarray}}
\newcommand{\ea}{\end{eqnarray}}
\newcommand{\seg}[1]{\mbox{$\overline{#1}$}}

\newtheorem{dfn}{Definition}[section]
\newtheorem{rmk}{Remark}[section]
\newtheorem{lma}{Lemma}[section]
\newtheorem{propn}{Proposition}[section]
\newtheorem{exmpl}{Example}[section]
\newtheorem{conjec}{Conjecture}[section]
\newtheorem{claim}{Claim}[section]
\newtheorem{notn}{Notation}[section]
\newtheorem{thm}{Theorem}[section]
\newtheorem{crlry}{Corollary}[section]

\newcommand{\prf}{\noindent{{\bf Proof} :\ }}
\newcommand{\QED}{\vrule height 1.4ex width 1.0ex depth -.1ex\ \medskip}

\newcommand{\ACMTOG}{{\sl ACM Trans.\ Graph.\ }}
\newcommand{\AMM}{{\sl Amer.\ Math.\ Monthly\ }}
\newcommand{\BIT}{{\sl BIT\ }}
\newcommand{\CACM}{{\sl Commun.\ ACM\ }}
\newcommand{\CAD}{{\sl Comput.\ Aided Design\ }}
\newcommand{\CAGD}{{\sl Comput.\ Aided Geom.\ Design\ }}
\newcommand{\CGIP}{{\sl Comput.\ Graph.\ Image Proc.\ }}
\newcommand{\CJ}{{\sl Computer\ J.\ }}
\newcommand{\DCG}{{\sl Discrete\ Comput.\ Geom.\ }}
\newcommand{\IBMJRD}{{\sl IBM\ J.\ Res.\ Develop.\ }}
\newcommand{\IJCGA}{{\sl Int.\ J.\ Comput.\ Geom.\ Applic.\ }}
\newcommand{\IEEECGA}{{\sl IEEE Comput.\ Graph.\ Applic.\ }}
\newcommand{\IEEETPAMI}{{\sl IEEE Trans.\ Pattern Anal.\ Machine Intell.\ }}
\newcommand{\JACM}{{\sl J.\ Assoc.\ Comput.\ Mach.\ }}
\newcommand{\JAT}{{\sl J.\ Approx.\ Theory\ }}
\newcommand{\MC}{{\sl Math.\ Comp.\ }}
\newcommand{\MI}{{\sl Math.\ Intelligencer\ }}
\newcommand{\NM}{{\sl Numer.\ Math.\ }}
\newcommand{\SIAMJNA}{{\sl SIAM J.\ Numer.\ Anal.\ }}
\newcommand{\SIAMR}{{\sl SIAM Review\ }}

\newcommand{\figg}[3]{\begin{figure}[htbp]\vspace{#3}\caption{#2}\label{#1}\end{figure}}

\setlength{\oddsidemargin}{0in}
\setlength{\evensidemargin}{0in}
\setlength{\headsep}{0pt}
\setlength{\topmargin}{0in}
\setlength{\textheight}{8.75in}
\setlength{\textwidth}{6.5in}

\title{
Computing the bisector \\
of a point and a plane curve \\
(Extended abstract)
}

\author{
Rida~T.~Farouki \\
IBM Thomas~J.~Watson Research Center, \\
P.~O.~Box 218, Yorktown Heights, NY 10598. \\
farouki@watson.ibm.com \\ \\
John~K.~Johnstone \\
Department of Computer Science, \\
The Johns Hopkins University, Baltimore, MD 21218. \\
jj@cs.jhu.edu
}

\date{}

\maketitle

\begin{abstract}
The {\it bisector\/} of a fixed point ${\bf p}$ and a smooth
plane curve $C$ --- {\it i.e.}, the locus traced by a point
that remains equidistant with respect to ${\bf p}$ and $C$ ---
is investigated in the case that $C$ admits a regular polynomial
or rational parameterization. It is shown that the bisector may be
regarded as (a subset of) a ``variable--distance'' offset curve to
$C$ which has the attractive property, unlike fixed--distance offsets,
of being {\it generically\/} a rational curve. 
A {\it trimming procedure}, which identifies the parametric subsegments
of this curve that constitute the true bisector, is described.
The time to compute the bisector of a point and a regular polynomial
parametric curve of degree $n$ is 
$O(9n^{2}) \varphi(2n-1) + \varphi(O(9n^2)) + O(9n^3)$, where 
$\varphi(n)$ is the time to solve an equation of degree $n$.
\end{abstract}

% \thispagestyle{empty}\mbox{}\vfill\eject
% \pagenumbering{arabic}
% \setcounter{page}{1}
% \thispagestyle{plain}

\section{Introduction}
\label{intro}

In descriptive geometry \cite{coxeter69} the parabola is
characterized as the locus traced by a point that remains
equidistant with respect to a fixed point ${\bf p}$ (the
{\it focus\/}) and a given straight line $L$ (the {\it
directrix\/}). In this sense, the parabola may be
regarded as the ``bisector'' of the point ${\bf p}$ and
the line $L$.
If we substitute a smooth plane curve $C$ in place of
the straight line $L$, the bisector locus is of a more
subtle nature. This paper is concerned with investigating
the geometric properties of such loci, and formulating
tractable representations for them. 

Point/curve bisectors
arise in a variety of geometric ``reasoning'' and geometric
decomposition problems ({\it e.g.}, planning paths of maximum
clearance in robotics, or computing Voronoi diagrams for areas
with curvilinear boundaries). Although they are much simpler
than other loci --- line/curve and curve/curve bisectors ---
that arise in these contexts, no systematic analysis of the
properties of point/curve bisectors is currently available
in the literature.

However, there has been considerable interest recently in
the general topic of bisectors. They play a key r\^ole in
computing the {\it medial axis transform\/} or ``skeleton''
of planar shapes (see, for example, Bookstein \cite{bookstein79}
and Lee \cite{lee82}). Yap \cite{yap87} discusses the bisectors
of points, lines, and circles in the context of Voronoi diagrams.
Yap and Alt \cite{yap89} analyze
the complexity of the bisector computation for two algebraic
curves, quoting an upper bound of $16m^6$ on the degree of
the bisector for curves of degree $m$. Nackman and Srinivasan
\cite{nackman91} discuss generic properties of the bisector of
two linearly--separable sets of arbitrary dimension, from the
perspective of point--set topology.
Hoffmann and Vermeer \cite{HV91} develop systems of equations
that define ``equal--distance'' curves and surfaces (another
term for bisectors, viewed as offsets from given curves/surfaces).
Voronoi surfaces --- {\it i.e.}, the bisectors of two given
surfaces --- are also discussed by Hoffmann \cite{H90} and Dutta
and Hoffmann \cite{DH90}.

The paper is structured as follows.
In Section~\ref{sec:untrim}, we show how to compute 
a superset of the bisector, called the untrimmed bisector 
(Figure~\ref{fig:ellipse}).
This superset is defined in terms of a variable-distance offset.
Surprisingly, it is a rational curve.
We then need to trim down to the true bisector (Figure~\ref{fig:trimellps}),
and this trimming process is presented in Section~\ref{trimming}.
The algorithm for point-curve bisector computation is summarized
in Section~\ref{sec:algorithm}, along with its complexity.
The paper ends with some conclusions.
An appendix gives a formal definition of distance,
and discusses the computation of 
self-intersections, which is an important step in the trimming.

We shall focus in this paper on the important case where the curve $C$
is described parametrically, ${\bf r}(u)=\{x(u),y(u)\}$, having
derivatives
continuous to at least third order for all $u \in I$, where $I$
denotes some finite, semi--infinite, or infinite parameter domain
of interest. We assume that the parameterization of ${\bf r}(u)$
is proper\footnote{A curve ${\bf r}(u)$ is proper if there 
is a one--to--one correspondence
between parameter values $u$ and points $(x,y)$ of the curve
locus --- except, possibly, for finitely many instances where
${\bf r}(u)$ crosses itself.}
and regular\footnote{A curve ${\bf r}(u)$ is regular over $I$ if
	$\sigma(u)\not=0$ for all $u \in I$, where 
	$\sigma(u) = \sqrt{x'^2(u)+y'^2(u)}$ is 
	the parametric speed of ${\bf r}(u)$.}
over $I$.
Note that if the curve ${\bf r}(u)$ is regular on the
interval $u \in I$, its unit normal vector
\be \label{normal}
{\bf n}(u) \,=\, {(y'(u),-x'(u)) \over \sqrt{x'^2(u)+y'^2(u)}}
\ee
is defined and continuous for all $u \in I$.

Although much of the ensuing discussion holds for any regular
parametric curve, we shall deal in this abstract solely with the functional
form encountered most often in practice: the {\it polynomial\/}
curve ${\bf r}(u)=\{X(u),Y(u)\}$ of degree $n$ defined by
\be \label{polycurve}
X(u) = \sum_{k=0}^n a_k u^k \,, \quad
Y(u) = \sum_{k=0}^n b_k u^k \,,
\ee
the coefficients $\{a_k,b_k\}$ being real numbers that satisfy
$a_n^2+b_n^2\not=0$.

The reader is referred to the full version of the paper \cite{faroukijj91} 
for a complete treatment of rational curves and finite polynomial
or rational curve segments, 
which we omit in this abstract for lack of space.
We also omit most proofs in this abstract.

\section{The {\mbox ``untrimmed'' point/curve} bisector}
\label{sec:untrim}

\begin{dfn} \label{defbsctr}
The bisector $B({\bf p},C)$ of a fixed point ${\bf p}$ and a plane curve $C$
is the locus traced by a point that remains equidistant with respect
to ${\bf p}$ and $C$ (in the sense of the distance function $(\ref{distance})$
of the appendix).
\end{dfn}

The bisector is closely related to the offset \cite{farouki90a,farouki90b}, 
a relationship that will be directly exploited in this paper.

\begin{dfn}
The ``untrimmed'' offset at (signed) distance $d$ to a regular parametric
curve ${\bf r}(u)$ is the locus defined by
\be \label{offset}
{\bf r}_o(u) \,=\, {\bf r}(u) + d\,{\bf n} (u) \,.
\ee
\end{dfn}

We call the locus (\ref{offset}) the ``untrimmed'' offset for the
following reason: {\it Corresponding points\/} ${\bf r}(u)$ and
${\bf r}_o(u)$ on the given curve and its untrimmed offset are
evidently distance $d$ apart, measured along their mutual normal
direction. However, the point ${\bf r}_o(u)$ of the untrimmed offset
is not necessarily distance $d$, in the sense of the distance function
(\ref{distance}), from the {\it entire curve\/} ${\bf r}(u)$. We shall
call the locus having this latter property the ``trimmed'' offset to
${\bf r}(u)$, since it is obtained by deleting certain continuous
segments of (\ref{offset}).
Similarly, we will distinguish an ``untrimmed'' bisector from 
a ``trimmed'' bisector (Figures~\ref{fig:ellipse} and \ref{fig:trimellps}).

The offset of a polynomial or rational curve 
is {\it not}, in general, a polynomial or rational curve, 
because of the radical in the denominator of (\ref{normal}).
Consequently, offset curves are often approximated
by piecewise--polynomial forms in computer--aided design (CAD)
applications \cite{hoschek88,klass83,pham88,tiller84}, to render
them compatible with existing representational and algorithmic
infrastructures. 
For the bisector, we will not have this problem, because the bisector
is actually a rational curve.

We \mbox{can generalize the} notion of an (untrimmed) offset curve at fixed
distance $d$ to a given regular curve ${\bf r}(u)$ by substituting
any continuous function $d(u)$ of the parameter $u$ in place of the
constant $d$, yielding the {\it variable--distance
offset curve}
\be \label{varoffset}
{\bf r}_o(u) \,=\, {\bf r}(u) + d(u) {\bf n}(u).
\ee
We shall define the untrimmed bisector as a variable-distance offset
(Figure~\ref{fig:varoffset}) as follows.

\figg{fig:varoffset}{Untrimmed bisector of parabola as variable--distance 
offset.}
{4in} % 2.75in

Consider the {\it family of normal lines\/} to a given regular curve
${\bf r}(u)$. These lines may be parameterized in the form
\be \label{nline}
{\bf r}(u) + \lambda\,{\bf n}(u) \,,
\ee
where $u$ selects a point on the curve, and $\lambda$ measures the
signed distance along the normal line from that point. Given any point
${\bf p} = (\alpha, \beta)$ not on ${\bf r}(u)$, the location ${\bf q}$
along (\ref{nline}) that is equidistant from ${\bf p}$ and
the curve point ${\bf r}(u)$ is uniquely identified by the condition
$|\,{\bf q}-{\bf r}(u)\,|=|\,{\bf q}-{\bf p}\,|$, which reduces to
\be \label{lambda}
|\lambda| \,=\, |\,{\bf r}(u)+\lambda\,{\bf n}(u)-{\bf p}\,| \,.
\ee

\figg{fig:du}{Definition of the displacement function $d(u)$.}{2in}

Now for each $u$, let $d(u)$ denote the unique value $\lambda$
satisfying condition (\ref{lambda}), and let $\psi(u)$ be the angle
between the vector from ${\bf r}(u)$ to ${\bf p}$ and the normal
${\bf n}(u)$, measured in the right--handed sense defined by a unit
vector ${\bf z}$ orthogonal to the plane of the curve. Referring to
Figure~\ref{fig:du} and noting that the points ${\bf p}$, ${\bf q}$,
and ${\bf r}(u)$ define an isosceles triangle, we have
\be \label{du0}
d(u) \,=\, \half\,|\,{\bf p}-{\bf r}(u)\,|\sec\psi(u) \,
\ee
by using the law of cosines.
Since $({\bf p}-{\bf r}(u))\cdot{\bf n}(u)=|\,{\bf p}-{\bf r}(u)\,|
\cos\psi(u)$, we can also express $d(u)$ as
\be \label{du}
d(u) \,=\, {|\,{\bf p}-{\bf r}(u)\,|^2 \over
2\,({\bf p}-{\bf r}(u))\cdot{\bf n}(u)} \,.
\ee

\begin{dfn}
The {\it untrimmed bisector\/} of a fixed point ${\bf p}$ and a
regular parametric curve ${\bf r}(u)$ is the variable--distance offset
$(\ref{varoffset})$ to ${\bf r}(u)$ with the displacement function
$(\ref{du})$.
(See Figure~\ref{fig:varoffset}.)
\end{dfn}

\begin{rmk}
{\rm
When ${\bf r}(u)$ is a polynomial or rational curve, the untrimmed
bisector defined by $(\ref{varoffset})$ and $(\ref{du})$ has a {\it
rational\/} parameterization, since the radicals in $d(u)$ and
${\bf n}(u)$ cancel each other.
}
\end{rmk}

We will denote the untrimmed bisector defined by (\ref{varoffset})
and (\ref{du}) by ${\bf b}(u)$, with homogeneous coordinates given by
polynomials $X_b(u)$, $Y_b(u)$, $W_b(u)$. 

\begin{rmk}
For the polynomial curve (\ref{polycurve}), 
the untrimmed bisector is
a rational curve of degree at most $3n-1$, described by Table~1.
\end{rmk}

\begin{table*}
\ba \label{pbsctr}
X_b \! &=& \! [\,\alpha^2-X^2+(\beta-Y)^2\,]\,Y'
 \,-\, 2(\beta-Y)XX' \,, \nonumber \\
Y_b \! &=& \! 2(\alpha-X)YY'
 \,-\, [\,(\alpha-X)^2+\beta^2-Y^2\,]\,X' \,, \nonumber \\
W_b \! &=& \! 2\,[\,(\alpha-X)Y'-(\beta-Y)X'\,] \,,
\ea
\label{tabp}
\caption{The untrimmed bisector ${\bf b}(u)$ for the polynomial curve (1)}
\end{table*}

\begin{propn}
\label{p:superset}
The untrimmed bisector of ${\bf p}$ and ${\bf r}(u)$ is a superset
of the true bisector of ${\bf p}$ and ${\bf r}(u)$.
\end{propn}
\ifFull
\prf
Let ${\bf q}$ be a point of the bisector of ${\bf p}$ and ${\bf r}(u)$.
By Remark~\ref{rmk:bis}, there exists a point ${\bf r}(u_{0})$ of the
curve that lies on the circle $C_{\bf q}$, such that the curve is tangent
to the circle at ${\bf r}(u_{0})$ or, equivalently, such that the normal
at ${\bf r}(u_{0})$ passes through the center ${\bf q}$ of the circle
$C_{\bf q}$. Thus, ${\bf r}(u_{0}) + d(u_{0}){\bf n}(u_{0}) = {\bf q}$
(recall that $d(u_0)=|\,{\bf q}-{\bf p}\,|={\rm radius\ of\ } C_{\bf q}$).
\QED
\fi

\figg{fig:ellipse}{Untrimmed bisectors of a point and an ellipse.}
{3.5in}

\subsection{Special points of the untrimmed bisector}
\label{sec:irregpts}

The points at infinity and cusps of the untrimmed 
bisector are important for trimming.
The untrimmed bisector will exhibit a point at infinity
for each parameter value $\tau$ that satisfies $({\bf p}-{\bf r}(\tau))
\cdot{\bf n}(\tau)=0$ ({\it i.e.}, the curve normal ${\bf n}(\tau)$
is orthogonal to the vector from ${\bf r}(\tau)$ to ${\bf p}$ or,
equivalently, the tangent line at ${\bf r}(\tau)$ passes through
${\bf p}$). For the polynomial curve (\ref{polycurve}),
the parameter values corresponding to these points
at infinity are the roots of the polynomial
\be \label{Pinf}
P_\infty(u) \,=\,
[\,\alpha-X(u)\,]\,Y'(u) \,-\, [\,\beta-Y(u)\,]\,X'(u) \,,
\ee
which is of degree $2n-2$ (at most) when ${\bf r}(u)$ is of degree $n$.

It is evident from Figure~\ref{fig:ellipse}
that in general the untrimmed bisector of a point ${\bf p}$ and a
regular curve ${\bf r}(u)$ is {\it not\/} a smooth locus.

\begin{lma}
The untrimmed bisector ${\bf b}(u)$ exhibits a cusp, or sudden
tangent reversal, at those parameter values where $P_\infty(u)
\not=0$ and the curvature $\kappa(u)$ of the given curve ${\bf r}
(u)$ attains the local critical value
\be \label{kappacrit}
\kappa_c(u) \,=\, -\ {1 \over d(u)} \,=\,
-\ {2\cos\psi(u) \over |\,{\bf p}-{\bf r}(u)\,|} \,,
\ee
without being an extremum, {\it i.e.}, $\kappa'(u)\not=0$.
\end{lma}
\ifFull
\prf Proof omitted in this abstract. \QED
\fi

When ${\bf r}(u)$
is the polynomial curve (\ref{polycurve}), 
the parameter values corresponding
to cusps on the untrimmed bisector ${\bf b}(u)$
are the roots of a polynomial of degree (at most) $4n-4$ in $u$, described
by Table~2.

\begin{table*}
\ba \label{Pcusp}
P_c \! &=& \! [\,(\alpha-X)^2+(\beta-Y)^2\,]\,(X'Y''-X''Y') \nonumber \\
&& +\ 2\,({X'}^2+{Y'}^2)\,[\,(\alpha-X)Y'-(\beta-Y)X'\,] \,.
\ea
\label{tabc}
\caption{The polynomial representing cusps on the untrimmed bisector}
\end{table*}

\subsection{The true bisector}

Note that the ``true'' bisector of ${\bf p}$ and ${\bf r}
(u)$ can be visualized as follows. Consider a circle through ${\bf p}$,
increasing in size until it just touches the curve ${\bf r}(u)$. Now
consider all such circles, inflating in all directions from ${\bf p}$
until they touch ${\bf r}(u)$. The bisector is the locus of the centers
of these maximal circles. In the next section, it is useful to
keep this representation of the bisector in mind.

\begin{dfn}
\label{d:Cq}
Let $C_{\bf q}$ denote the circle with center ${\bf q}$ and radius
$|\,{\bf q}-{\bf p}\,|$.
\end{dfn}

\figg{fig:bis}{Condition for a point ${\bf q}$ to lie on the bisector
$B({\bf p},{\bf r}(u))$.}{2in}

\begin{rmk}
\label{rmk:bis}
{\rm
A point ${\bf q}$ lies on the bisector of ${\bf p}$ and the regular
curve ${\bf r}(u)$ if and only if (see Figure~\ref{fig:bis}):
\begin{itemize}
\item
$C_{\bf q}$ is ``empty'' --- no point of ${\bf r}(u)$ lies in its
interior; and
\item
$C_{\bf q}$ is tangent to ${\bf r}(u)$ in at least one point.
\end{itemize}
%       If ${\bf r}(u)$ is not regular,
%       then the second condition should instead be
%       that the boundary of the circle $C_{\bf q}$
%       contains at least one point of ${\bf r}(u)$.
}
\end{rmk}

Since the second condition of Remark~\ref{rmk:bis} is satisfied for
{\it all\/} points ${\bf q}$ of the untrimmed bisector (by definition),
a point ${\bf q}$ of the untrimmed bisector is a point of the true
bisector if and only if the interior of $C_{\bf q}$ is empty.

\section{The trimming procedure}
\label{trimming}

The untrimmed bisector ${\bf b}(u)$ may be ``trimmed'' down to the true
bisector by deleting a finite number of segments. 
This is done by finding a finite number of special points that
identify possible deviations of ${\bf b}(u)$ from the true bisector.
There are four classes of these special points on the untrimmed bisector
(see Definitions~\ref{d:trim} and \ref{d:cri} below), and we split the trimming
process into two stages. 
In the next subsection, we describe the first stage, which removes
``inactive'' segments.

\subsection{Active and inactive segments}

Many points ${\bf b}(u_0)$ of the untrimmed bisector are not on the 
true bisector, because their ``corresponding'' point ${\bf r}(u_0)$ is not 
the closest point of the curve to ${\bf b}(u_0)$.

\begin{dfn}
The~points~${\bf r}(u_0)$~and ${\bf b}(u_0)={\bf r}(u_0)+d(u_0){\bf n}
(u_0)$ are called {\rm corresponding} points of the curve and the
untrimmed bisector.
\end{dfn}

\begin{dfn}
The point ${\bf q} = {\bf b}(u_{0})$ of the untrimmed bisector is
{\rm active} if either of the following conditions holds 
(see Figure~\ref{fig:active}):
\begin{description}
\item[{\rm (1)}]
        ${\bf q}$ has only one corresponding point ${\bf q}' =
        {\bf r}(u_0)$ on the curve, and either
\begin{description}
\item[{\rm (a)}]
        the circle of curvature at ${\bf q}'$ contains the point ${\bf p}$
        --- i.e., ${\bf p}$ lies on or inside the circle of curvature, or
\item[{\rm (b)}]
        the point ${\bf p}$ and the circle of curvature at ${\bf q}'$
        lie on opposite sides of the tangent at ${\bf q'}$
\end{description}
\item[{\rm (2)}]
        ${\bf q}$ has more than one corresponding point on the
        curve\footnote{There are a finite number of such points, and
        we will have more to say about them in Section~\ref{sec:critical} 
	below.}
\end{description}
A segment $S$ of ${\bf b}(u)$ is active
if every point of $S$ is active.
\end{dfn}

\figg{fig:active}{Active and inactive points on the untrimmed bisector.}{2in}

\begin{rmk}
\label{r:active}
{\rm
We observe that an active point appears to lie on the bisector, at least
locally. 
In particular, we can show that 
the curve in some neighborhood of an active point 
(with one corresponding
point) lies completely outside of $C_{\bf q}$ (Remark~\ref{rmk:bis}).
To see this, let ${\bf q}$ be an active point of ${\bf b}(u)$
with only one corresponding point ${\bf q}'$. If ${\bf p}$ lies inside or
on the circle of curvature at ${\bf q}'$, then 
it can be shown that 
all of $C_{\bf q}$ also lies in the circle of curvature at ${\bf q}'$ and,
in particular, there exists a neighborhood of ${\bf q}'$
% ({\it i.e.}, the image of a neighborhood of $t'$ where ${\bf q}' = {\bf r}(t')$)
that lies completely outside of $C_{\bf q}$ \cite[p.~176]{H52}.
On the other hand, if ${\bf p}$ (and thus $C_{\bf q}$) lies on the opposite side
of the tangent at ${\bf q}'$ from the circle of curvature at ${\bf q}'$,
then again the curve in a neighborhood of ${\bf q}'$ lies completely outside of
$C_{\bf q}$.
}
\end{rmk}

\begin{propn}
An inactive point of the untrimmed bisector of ${\bf p}$ and ${\bf r}(u)$
does not lie on the bisector of ${\bf p}$ and ${\bf r}(u)$.
\end{propn}
% \ifFull
\prf
(Sketch) If ${\bf q}$ is an inactive point of ${\bf b}(u)$,
then $C_{\bf q}$ must contain some points of the curve in its interior,
since $C_{\bf q}$ must strictly contain the circle of curvature at its
corresponding point ${\bf q}'$ (Figure~\ref{fig:active}(iii)).
\QED
% \fi

Since the definition of an active point depends on the side of the
tangent (at ${\bf q}'$) that ${\bf p}$ and the circle of curvature lie
on, we are interested in points where this can change.
We are also interested in points where the position of ${\bf p}$
relative to the circle of curvature can change.

\begin{dfn}
\label{d:trim}
A point of the curve ${\bf r}(u)$ is an {\rm inflection} if the curvature
is zero at that point;
% regular curve => no cusps, no advantage in avoiding self-intersections
% concavity of the curve changes at this point
a point of ${\bf r}(u)$ is a {\rm class point}\footnote
       {This term is chosen in allusion to the class of a curve,
        which is the number of tangents that pass through a typical point
        not on the curve \cite[p.~115]{W50}.}
if the tangent at that point passes through ${\bf p}$; and a point of
${\bf r}(u)$ is a {\rm circular point} if the circle of curvature at that
point passes through ${\bf p}$.
\end{dfn}

\begin{thm}
\label{thm:active}
For a regular polynomial or rational curve ${\bf r}(u)$ defined on
the interval $u \in I$, and a point ${\bf p}$ not on ${\bf r}(u)$,
let ${\bf b}(u)$ be the untrimmed bisector of ${\bf p}$ and ${\bf r}
(u)$, and let $\{i_{1},\ldots,i_{M}\}$ be the ordered set of parameter
values on $I$ that correspond to inflections, class points, and circular
points of ${\bf r}(u)$. Then, denoting the end points of $I$ by $i_{0}$
and $i_{M+1}$, we have either
\be
{\bf b}(u) {\rm \ is\ active\ }
\quad {\rm for\ all\ } u \in (i_k,i_{k+1})
\ee
or
\be
{\bf b}(u) {\rm \ is\ inactive\ }
\quad {\rm for\ all\ } u \in (i_k,i_{k+1})
\ee
on each span $(i_k,i_{k+1})$ for $k=0,\ldots,M$.
\end{thm}
\ifFull
\prf  Omitted in this abstract.
\QED
\fi

We have already encountered ``class'' points and ``circular'' points,
in a somewhat different guise (Section~\ref{sec:irregpts}): 
they are, respectively, points of ${\bf r}(u)$ that
induce points at infinity and cusps on the untrimmed bisector. 
(We omit the proof here.)
Thus, equations (\ref{Pinf}) and (\ref{Pcusp}) can be used to find them.

\subsection{Critical~points (self--intersections)}
\label{sec:critical}

In order to make the second --- and final --- refinement from active
segments to segments on the bisector, recall that a point ${\bf q}$ of
the untrimmed bisector is in the bisector if and only if its associated
circle $C_{\bf q}$ is empty. A boundary between the bisector and the
rest of the untrimmed bisector is marked by a transition from points
with empty circles to points with non--empty circles. These transitions
are associated with critical points.

\begin{dfn}
\label{d:cri}
A point ${\bf q}$ of the untrimmed bisector is a {\rm critical point}
if the circle $C_{\bf q}$ is tangent to ${\bf r}(u)$ at two or more
points.
\end{dfn}

\begin{propn}
\label{prop:cri}
A point ${\bf q}$ of the untrimmed bisector is a critical point if and
only if it is a self--intersection of ${\bf b}(u)$.
\end{propn}
\ifFull
\prf 
\QED
\fi

Proposition~\ref{prop:cri} allows us to trim using self-intersections
while arguing the validity of this trim using critical points.

\begin{thm}
\label{thm:trim2}
For a regular polynomial or rational curve ${\bf r}(u)$ defined on
the interval $u \in I$, and a point ${\bf p}$ not on ${\bf r}(u)$, let
${\bf b}(u)$ be the untrimmed bisector of ${\bf p}$ and ${\bf r}(u)$,
and let $\{i_{1},\ldots,i_{M}\}$ be the ordered set of parameter values
on $I$ that correspond to endpoints of active segments on ${\bf b}(u)$
or self-intersections of ${\bf b}(u)$, {\it i.e.}, ${\bf b}(i_j) =
{\bf b}(i_k)$ for some $1 \leq j \neq k \leq M$. Then for every active
segment $u \in [\,i_j,i_k\,]$ of ${\bf b}(u)$ $({\rm where\ } i_j <
i_k)$, we have either
%       Let $i_{j} < i_{k}$ be the parameter values of the endpoints of an
%       active segment. Then, either
\be
{\bf b}(u) {\rm \ is\ on\ the\ bisector\ }
\quad \forall\ u \in (i_l,i_{l+1})
\ee
or
\be
{\bf b}(u) {\rm \ is\ not\ on\ the\ bisector\ }
\quad \forall\ u \in (i_l,i_{l+1})
\ee
on each span $(i_l,i_{l+1})$ for $l=j,\ldots,k-1$ between successive
self--intersections on that active segment.
\end{thm}

\prf (Sketch)
Consider a point ${\bf q}$ moving along the untrimmed bisector.
In order for ${\bf q}$ to leave the
bisector, the curve must enter $C_{\bf q}$ (Remark~\ref{rmk:bis}). 
However, at all times
the curve in the neighborhood of the corresponding point ${\bf q}'$
lies completely outside $C_{\bf q}$ (since ${\bf q}$ is active; see
Remark~\ref{r:active}).
Therefore, the curve must enter 
$C_{\bf q}$ at some other point ${\bf q}'' \neq {\bf q}'$,
at which time ${\bf q}$ will be a critical point, because 
$C_{\bf q}$ is tangent at both ${\bf q}'$ and ${\bf q}''$.
\QED

\ifFull
\prf
(Sketch)
As a point ${\bf q}$ moves along the untrimmed bisector, 
the radius of the circle $C_{\bf q}$ changes
smoothly while it continues to pass through ${\bf p}$, and at all times
the curve in the neighborhood of the corresponding point ${\bf q}'$
lies completely outside $C_{\bf q}$ (since ${\bf q}$ is active; see
Remark~\ref{r:active}). 
In order for ${\bf q}$ to leave the
bisector, the circle $C_{\bf q}$ must become occupied, {\it i.e.},
the curve must enter $C_{\bf q}$ (Remark~\ref{rmk:bis}). 
Since in the neighborhood of ${\bf q}'$ the curve does
not enter $C_{\bf q}$,
% this is important because otherwise in the limit the curve might touch
% {\it and} enter at ${\bf q}'$, {\it e.g.}, inflection.
%
%and the curve is regular, the curve cannot enter $C_{\bf q}$ at ${\bf q}'$ and
%
it must first enter $C_{\bf q}$ at some other point ${\bf q}'' \neq
{\bf q}'$. In order for the curve to enter $C_{\bf q}$, it must first
become tangent to $C_{\bf q}$, and the location of ${\bf q}$ when
this occurs is a critical point of the untrimmed bisector, since its
circle $C_{\bf q}$ has {\it two\/} points of tangency with the curve
--- one at ${\bf q}'$ and one at the other point ${\bf q}''$ where the
curve is about to enter the circle.
\QED
\fi

The computation of parameter values that correspond to self-intersections is
rather involved.
We outline a method for their computation in the appendix.

% \ifFull
\begin{rmk}
Two interesting observations can be made about
the self-intersections of the untrimmed bisector of the ellipse (and other
conics), as illustrated in Figure~\ref{fig:symm}.
First, the self--intersections of the untrimmed bisector lie on
the major and minor axes of the ellipse
(according to whether ${\bf p}$ lies inside or outside of it). 
Second, when the point ${\bf p}$ is located {\it inside\/} the ellipse, we will
have zero, one, or two self--intersections of the untrimmed bisector,
according to whether ${\bf p}$ lies inside both, just one, or
neither of the circles of curvature to the ellipse at the vertices on
its major axis, respectively (the first case being possible only when
the circles of curvature actually overlap). 
If ${\bf p}$ is {\it
outside\/} the ellipse, we have zero, one, or two self--intersections
when ${\bf p}$ lies inside neither, just one, or both of the
circles of curvature to the ellipse at the vertices on its minor axis,
respectively.
\end{rmk}

\figg{fig:symm}{Self-intersections of the point/ellipse bisector}{2.5in}

\subsection{The algorithm}
\label{sec:algorithm}

We now have an algorithm for computing the bisector of a fixed point
${\bf p}$ and a regular parametric curve ${\bf r}(u)$:

\begin{enumerate}
\item
        Compute the untrimmed bisector of ${\bf p}$ and ${\bf r}(u)$,
        using (\ref{varoffset}) and (\ref{du}). For polynomial 
        curves, this untrimmed bisector is given by equation (\ref{pbsctr}).
\item
        Find the inflections of the curve ${\bf r}(u)$, and its class
        points (using (\ref{Pinf}))
	and circular points (using (\ref{Pcusp})) 
	with respect to ${\bf p}$.
\item
        For each segment on the untrimmed bisector delineated by these
        special points, compare the distances of the segment midpoint
        to ${\bf p}$ and ${\bf r}(u)$, using (\ref{distance2}), and
        discard the segment if these distances are unequal.
\item
        Find~the~critical~points (self--intersections) of the untrimmed
        bisector, using the methods described in Section~\ref{sec:slfint}
        below.
\item
        Split each remaining segment of the untrimmed bisector at these
        self--intersections. For each of the resulting segments, compare
        the distances of the midpoint to ${\bf p}$ and ${\bf r}(u)$,
        using (\ref{distance2}), and discard the segment if they are
        unequal. The remaining segments constitute the true bisector.
\end{enumerate}

\figg{fig:trimellps}{True bisectors of a point and an ellipse.}
{3.5in}

We now analyze the complexity of this calculation.
It is useful to compare this complexity with Yap and Alt's analysis of 
curve-curve bisector computation, where they roughly estimate
the complexity as that of solving an equation of degree $16n^6$, 
where $n$ is the degree of each curve \cite{yap89}.

\begin{thm}
Let $\varphi(n)$ be the time to solve an equation of degree $n$.
The time to compute the bisector of a point and a regular polynomial
parametric curve of degree $n$ is 
$O(9n^{2}) \varphi(2n-1) + \varphi(O(9n^2)) + O(9n^3)$.
\end{thm}
\prf (Sketch)
The computation of inflection, class, and circular points in step (2)
of the algorithm
involves the solution of equations of degree $2n-4$, $2n-2$, and $4n-4$,
respectively.
There are at most $8n-10$ solutions to these equations, and thus in step (3)
there are $O(8n-10)$ point-point and $O(8n-10)$ point-curve 
distance computations.
Each of the latter requires the solution of an equation of degree $2n-1$
and $O(2n-1)$ point-point distance computations (Proposition~\ref{polydist}).
Thus, step (3) requires $O(\mbox{$(8n-10)\varphi(2n-1)$} + \mbox{$(8n-10)n$})$ 
time.
(In all distance computations, one can compute squared distance,
thus avoiding a square root.)
Step (4) involves the solution of an equation of degree at most $(3n-2)(3n-3)$.
(The polynomial representing the self-intersections of the untrimmed
bisector ${\bf b}(u)$ of degree $3n-1$ is of degree at most $(3n-2)(3n-3)$,
since a rational curve of degree $m$ has exactly $\frac{(m-1)(m-2)}{2}$ 
double points, properly counted, which is an upperbound on the number of
self-intersections \cite{W50}, and 
each double point has at most two parameter values associated with it.)
By the argument used for step (3), step (5) requires
$O(\mbox{$(3n-2)(3n-3)$}\varphi(2n-1) + \mbox{$(3n-2)(3n-3)n$})$ time.
\QED

\begin{rmk}
The $O(9n^2)$ and $O(9n^3)$ terms are actually lower in practice, because the
self-intersection polynomial is of degree lower than $(3n-2)(3n-3)$.
This can be seen by observing that, in (\ref{pbsctr}),
the degree of $W_b$ is lower than $X_b$ and $Y_b$ by $n+1$,
meaning there is an $n+1$-fold point at infinity (corresponding to $t=\infty$).
For instance, the self-intersection polynomial of the parabola
is of degree 4 rather than 12.
\end{rmk}

\section{Concluding remarks}
\label{conclusion}

The locus of a variable point ${\bf q}$, which maintains equal distances with
respect to a fixed point ${\bf p}$ and a plane polynomial or rational
curve ${\bf r}(u)$, is amenable to an exact and relatively simple ({\it
i.e.}, rational) parametric representation. Such point/curve bisectors
are thus more compatible with existing computer--aided design systems
than other elementary ``procedurally--defined'' curves --- notably the
fixed--distance {\it offsets\/} to polynomial or rational curves \cite
{farouki90a,farouki90b}, which have no rational parameterizations in
general.

The principal difficulty in computing point/curve bisectors undoubtedly
lies in the  ``trimming'' procedure, {\it i.e.}, identifying the parameter
values which delimit those subsegments of the untrimmed bisector 
that constitute the
``true'' bisector. Nevertheless, as shown in Section~\ref{trimming}, it is
possible to attack this problem in an algorithmic manner, and for simple
curves ({\it e.g.}, conics) closed--form analytic expressions for the
trim points may even be written down {\it a priori\/} \cite{faroukijj91}.
Once the ordered set of trim points $u_1,u_2,\ldots$ is known, the
bisector can be represented by a single rational expression ${\bf b}(u)$,
restricted to a sequence of disjoint domains $u \in [\,u_1,u_2\,]$, $u
\in [\,u_3,u_4\,]$, $\ldots\,$.

The problem of curve/curve bisectors is more formidable than that
of the point/curve bisectors discussed here, even if the given curves
have simple parameterizations. We hope to address this problem, as well
as the related problem of Voronoi diagrams of curves, in future studies.

\begin{thebibliography}{99}

\bibitem{bookstein79}
F.~L.~Bookstein (1979), The line--skeleton, \CGIP{\bf 11}, 123--137.

\bibitem{coxeter69}
H.~S.~M.~Coxeter (1969), {\it Introduction to Geometry}, Wiley,
New York.

\bibitem{DH90}
D.~Dutta and C.~M.~Hoffmann (1990), A geometric investigation of the
skeleton of CSG objects, {\it Technical Report UM--MEAM--90--02},
Dept.\ of Mechanical Engineering, The University of Michigan.

\bibitem{faroukijj91}
R.~T.~Farouki and J.~K.~Johnstone (1991), The bisector of a point and a plane
parametric curve, Technical Report 91-21, Dept. of Computer Science,
Johns Hopkins University, November 1991.

\bibitem{farouki90a}
R.~T.~Farouki and C.~A.~Neff (1990), Analytic properties of plane
offset curves, \CAGD{\bf 7}, 83--99.

\bibitem{farouki90b}
R.~T.~Farouki and C.~A.~Neff (1990), Algebraic properties of plane
offset curves, \CAGD{\bf 7}, 100--127.

\bibitem{H52}
D.~Hilbert and S.~Cohn--Vossen (1952), {\it Geometry and the Imagination},
Chelsea, New York.

\bibitem{H90}
C.~M.~Hoffmann (1990), A dimensionality paradigm for surface interrogations,
\CAGD{\bf 7}, 517--532.

\bibitem{HV91}
C.~M.~Hoffmann and P.~Vermeer (1991), Eliminating extraneous solutions in
curve and surface operations, \IJCGA{\bf 1}, 47--66.

\bibitem{hoschek88}
J.~Hoschek (1988), Spline approximation of offset curves, \CAGD{\bf 5},
33--40.

\bibitem{klass83}
R.~Klass (1983), An offset spline approximation for plane cubic splines,
\CAD{\bf 15}, 297--299.

\bibitem{lee82}
D.~T.~Lee (1982), Medial axis transformation of a planar shape,
\IEEETPAMI{PAMI--{\bf 4}}, 363--369.

\bibitem{nackman91}
L.~R.~Nackman and V.~Srinivasan (1991), Bisectors of linearly separable
sets, \DCG{\bf 6}, 263--275.

\bibitem{pham88}
B.~Pham (1988), Offset approximation of uniform B-splines, \CAD{\bf 20},
471--474.

\bibitem{tiller84}
W.~Tiller and E.~G.~Hanson (1984), Offsets of two-dimensional profiles,
\IEEECGA{\bf 4} (Sept.), 36--46.

% \bibitem{uspensky48}
% J.~V.~Uspensky (1948), {\it Theory of Equations}, McGraw--Hill,
% New York.

\bibitem{W50}
R. J. Walker (1950), {\it Algebraic Curves}, Springer--Verlag, New York.

\bibitem{yap87}
C.--K.~Yap (1987), An ${\rm O}(n\log n)$ algorithm for the Voronoi
diagram of a set of simple curve segments, \DCG{\bf 2}, 365--393.

\bibitem{yap89}
C.--K.~Yap and H.~Alt (1989), Motion planning in the {\it
CL\/}--Environment, in {\it Lecture notes in computer science 382\/}
(F.~Dehne, J.--R.~Sack, and N.~Santoro, eds.), Springer, New York,
373--380 (Proceedings of the Workshop on Algorithms and Data
Structures WADS~'89, Ottawa, Canada, August 17--19, 1989).

\end{thebibliography}

\section{Appendix}

\subsection{Definition of distance}

\begin{dfn}
The distance of a point ${\bf q}$ from a regular parametric curve
${\bf r}(u)$ $=\{x(u),y(u)\}$ defined on the parameter interval
$I$ is given by
\be \label{distance}
\dist({\bf q},{\bf r}(u)) \,=\,
\inf_{u \,\in\, I} \, |\,{\bf q}-{\bf r}(u)\,| \,.
\ee
\end{dfn}

\begin{propn} \label{polydist}
Let ${\bf q}=(a,b)$ and ${\bf r}(u)$ be the polynomial curve 
given by $(\ref{polycurve})$.
Then 
\be \label{distance2}
\dist({\bf q},{\bf r}(u)) \,=\,
\min_{k \,\in\, \{1,\ldots,N\}} \, |\,{\bf q}-{\bf r}(u_k)\,| \,,
\ee
where $\{u_1,\ldots,u_N\}$ is the set of
distinct odd--multiplicity roots of the polynomial
\be \label{Pperp}
P_\perp(u) \,=\,
[\,a-X(u)\,]\,X'(u) \,+\, [\,b-Y(u)\,]\,Y'(u)
\ee
of degree $2n-1$ on the interior of the interval $I$, augmented by
the finite end points, if any, of $I$. 
\end{propn}
\ifFull
\prf $P_\perp(u)$ is the derivative of $|\,{\bf q}-{\bf r}(u)\,|^2$,
the square of the distance.
\QED
\fi

\begin{rmk}
{\rm
Equation (\ref{Pperp}) has an
obvious geometric interpretation: the roots of the
polynomial $P_\perp(u)$ identify points of the curve where lines
drawn from ${\bf q}$ meet ${\bf r}(u)$ orthogonally. 
Even--multiplicity roots of
$P_\perp(u)$ can be ignored, since they do not identify local extrema
of distance.
}
\end{rmk}

\subsection{Computing the self--intersections}
\label{sec:slfint}

A self--intersection of the untrimmed bisector ${\bf b}(u)$ arises
if two (or more) {\it distinct\/} parameter values correspond to the
same geometric point on its locus. Thus, we are interested in identifying all
parameter values $u$ that satisfy the vector equation
\be \label{selfint}
{\bf b}(u+\xi) \,=\, {\bf b}(u) \quad {\rm for\ some\ } \xi\not=0 \,,
\ee
where ${\bf b}(u)=\{\,X_b(u)/W_b(u),Y_b(u)/W_b(u)\,\}$ is the rational
curve defined by equations (\ref{pbsctr}).

If ${\bf b}(u)$ is the untrimmed bisector of the point ${\bf p}=(\alpha,
\beta)$ and the regular polynomial curve ${\bf r}(u)=\{X(u),Y(u)\}$, we
know that corresponding points of ${\bf b}(u)$ and ${\bf r}(u)$ lie on
the normal lines to the latter. Given distinct parameter values $u$ and
$v$, we may express the point of intersection $(x_i,y_i)$ of the normal
lines at ${\bf r}(u)$ and ${\bf r}(v)$ in homogeneous coordinates as
\be \label{pi}
x_i(u,v) \,=\, {X_i(u,v) \over W_i(u,v)} \quad , \quad
y_i(u,v) \,=\, {Y_i(u,v) \over W_i(u,v)} \,
\ee
where the polynomials $X_i$, $Y_i$, and $W_i$ are defined by
\ba \label{Pi}
X_i(u,v) \!\! &=& \!\! Z(u)Y'(v)-Z(v)Y'(u) \,, \nonumber \\
Y_i(u,v) \!\! &=& \!\! Z(v)X'(u)-Z(u)X'(v) \,, \nonumber \\
W_i(u,v) \!\! &=& \!\! X'(u)Y'(v)-X'(v)Y'(u) \,.
\ea
For brevity, we have introduced the notation $Z(u)=X(u)X'(u)+Y(u)Y'(u)$
in (\ref{Pi}).

In order for the point (\ref{pi}) to identify a self--intersection
of the untrimmed bisector --- corresponding to parameter values $u$
and $v$ --- $(x_i,y_i)$ must be equidistant from the three points
${\bf r}(u)$, ${\bf r}(v)$, and ${\bf p}$. This condition requires
the simultaneous satisfaction of the equations in Table~3.
Cancelling $x_i^2+y_i^2$ from both sides of these equations and
substituting from (\ref{pi}--\ref{Pi}), we see that $u$ and $v$ must
be simultaneous roots of the polynomials in Table~4.

\begin{table*}
\begin{eqnarray*}
{[\,x_i-X(u)\,]^2 \,+\, [\,y_i-Y(u)\,]^2} \! &=& \!
{[\,x_i-\alpha\,]^2 \,+\, [\,y_i-\beta\,]^2} \,, \nonumber \\
{[\,x_i-X(v)\,]^2 \,+\, [\,y_i-Y(v)\,]^2} \! &=& \!
{[\,x_i-\alpha\,]^2 \,+\, [\,y_i-\beta\,]^2} \,.
\end{eqnarray*}
\label{tabS}
\caption{}
\end{table*}

\begin{table*}
\ba \label{RStilde}
{\tilde R}(u,v) \!\!
&=& \!\! 2\,[\,\alpha-X(u)\,]\,[\,Z(u)Y'(v)-Z(v)Y'(u)\,] \nonumber \\
&+& \!\! 2\,[\,\beta-Y(u)\,]\,[\,Z(v)X'(u)-Z(u)X'(v)\,] \nonumber \\
&+& \!\! [\,X^2(u)+Y^2(u)-\alpha^2-\beta^2\,]
 \, [\,X'(u)Y'(v)-X'(v)Y'(u)\,] \,, \nonumber \\
{\tilde S}(u,v) \!\!
&=& \!\! 2\,[\,\alpha-X(v)\,]\,[\,Z(u)Y'(v)-Z(v)Y'(u)\,] \nonumber \\
&+& \!\! 2\,[\,\beta-Y(v)\,]\,[\,Z(v)X'(u)-Z(u)X'(v)\,] \nonumber \\
&+& \!\! [\,X^2(v)+Y^2(v)-\alpha^2-\beta^2\,]
 \, [\,X'(u)Y'(v)-X'(v)Y'(u)\,] \,.
\ea
\label{tabR}
\caption{The polynomials ${\tilde R}(u,v)$ and ${\tilde S}(u,v)$}
\end{table*}

Extracting the factor $v-u$ (since we want $v \neq u$), we get
\be \label{RS}
R(u,v) \,=\, {{\tilde R}(u,v) \over v-u}
\quad {\rm and} \quad
S(u,v) \,=\, {{\tilde S}(u,v) \over v-u}
\ee
as our working polynomials. We now set $v=u+\xi$ in (\ref{RS}) and
re--write these polynomials in the form
\be \label{RS2}
R(u,\xi) \,=\, \sum_{k=0}^r a_k(u)\,\xi^k
\ee
and
\be \label{RS3}
S(u,\xi) \,=\, \sum_{k=0}^s b_k(u)\,\xi^k \,,
\ee
{\it i.e.}, as polynomials in $\xi$ whose coefficients are polynomials
in $u$. Then if any particular value of $u$ is to be a parameter value
at which a self--intersection occurs, the polynomials (\ref{RS2}-\ref{RS3}) 
must have a common root $\xi$ at that value of $u$.

\begin{propn}
Let the polynomials $a_k(u)$ and $b_k(u)$ be given 
by {\rm (\ref{RS2}-\ref{RS3})},\footnote{The expressions for the coefficients 
	$\{a_k(u)\}$ and $\{b_k(u)\}$ in
	(\ref{RS2}-\ref{RS3}) in terms of $X(u)$, $Y(u)$, 
	and their derivatives and the
	coordinates $\alpha,\beta$ of ${\bf p}$ are rather complicated, so we
	shall not present them explicitly here --- they are readily derived
	using a computer algebra system.}
and let $\Gamma(u)$ be the polynomial defined in terms of them by the
determinant
\be \label{G}
\Gamma \;=\; \left|\, \matrix{
 a_r &\cdot &\cdot &a_2   &a_1   &a_0   &{}    &{} &{} &{} \cr
 {}  &\cdot &\cdot &\cdot &\cdot &\cdot &\cdot &{} &{} &{} \cr
 {}  &{}    &a_r   &\cdot &\cdot &a_2   &a_1   &1  &{} &{} \cr
 {}  &{}    &{}    &a_r   &\cdot &\cdot &a_2   &0  &1  &{} \cr
 {}  &{}    &{}    &{}    &a_r   &\cdot &\cdot &0  &0  &1  \cr
 b_s &\cdot &\cdot &b_2   &b_1   &b_0   &{}    &{} &{} &{} \cr
 {}  &\cdot &\cdot &\cdot &\cdot &\cdot &\cdot &{} &{} &{} \cr
 {}  &{}    &b_s   &\cdot &\cdot &b_2   &b_1   &1  &{} &{} \cr
 {}  &{}    &{}    &b_s   &\cdot &\cdot &b_2   &0  &1  &{} \cr
 {}  &{}    &{}    &{}    &b_s   &\cdot &\cdot &0  &0  &1
 } \,\right| \,
\ee
where there are $s$ rows of $a_k\!$'s followed by $r$ rows of $b_k\!$'s
and it is understood that blank areas are filled with zeros. 
Let $\Lambda(u)$ be the GCD of 
Resultant$_{\,\xi}\,(\,X_i(u,\xi),W_i(u,\xi))$
and Resultant$_{\,\xi}\,(\,Y_i(u,\xi),W_i(u,\xi))$.
Then the roots of the polynomial
\be \label{Pslfint}
P_i(u) \,=\, {\Gamma(u) \over \Lambda(u)}
\ee
identify those parameter values
at which the untrimmed bisector suffers a self--intersection.
\end{propn}
\ifFull
\prf
(Sketch)
The resultant of the two polynomials (\ref{RS2}) with respect to $\xi$
can be expressed as a Sylvester determinant \cite{uspensky48}, identical
to (\ref{G}), except that in the last three columns we have replaced
each occurrence of $a_0$ by 1, and each occurrence of $a_1$ or $a_2$ by
0. 
We do this in order to remove the factor $[\,a_0(u)\,]^3$, which can be
shown to be associated with the cusps of the untrimmed bisector, which
have already been discarded.
We also divide by  $\Lambda(u)$ to remove the 
pairs of points on ${\bf r}(u)$ that have {\it
identical normal lines\/}.

Paired roots of the remaining polynomial (\ref{Pslfint}) then identify
distinct points ${\bf r}(u)$ and ${\bf r}(v)$ whose normal lines are
distinct and intersect in a point $(x_i,y_i)$ equidistant from ${\bf r}
(u)$, ${\bf r}(v)$, and ${\bf p}$ --- {\it i.e.}, a self--intersection
of ${\bf b}(u)$. \QED
\fi

\begin{exmpl}
\label{exmpl:trimellps}
{\rm
For the parabola,
the self--intersection polynomial
$P_i(u)$ is of degree 4, while for the ellipse it is 
the product of two quartic factors.
Beyond the realm of conic sections, the degree of the self--intersection
polynomial grows very rapidly. For the (polynomial) cubics, for example,
a number of examples suggest that $P_i(u)$ is of degree 26 in general,
while the extraneous factor $\Lambda(u)$ is of degree 6. 
} \QED
\end{exmpl}

Although some of the roots of
$P_i(u)$ correspond to self--intersections of conjugate branches of the
complex locus of ${\bf b}(u)$, and others to self--intersections ``at
infinity,'' $P_i(u)$ is irreducible in general: that is, there
is no simpler representation of just the real, affine self--intersection
parameter values.

\end{document}

