\documentstyle[12pt]{article}

\begin{document}

\newcommand{\single}{\def\baselinestretch{1.0}\large\normalsize}
\newcommand{\double}{\def\baselinestretch{1.5}\large\normalsize}
\newcommand{\triple}{\def\baselinestretch{2.2}\large\normalsize}

\newcommand{\rd}{{\rm d}}
\newcommand{\re}{{\rm e}}
\newcommand{\ri}{{\rm i}}
\newcommand{\sgn}{{\rm sign}}
\newcommand{\half}{\textstyle{1 \over 2}\displaystyle}
\newcommand{\quarter}{\textstyle{1 \over 4}\displaystyle}
\newcommand{\dist}{{\rm dist}}
\newcommand{\cross}{\!\times\!}
\newcommand{\dotpr}{\!\cdot\!}
\newcommand{\vhat}{\hat {\bf v}}
\newcommand{\what}{\hat {\bf w}}
\newcommand{\zhat}{\hat {\bf z}}
\newcommand{\ldash}{\vrule height 3pt width 0.35in depth -2.5pt}
\newcommand{\be}{\begin{equation}}
\newcommand{\ee}{\end{equation}}
\newcommand{\ba}{\begin{eqnarray}}
\newcommand{\ea}{\end{eqnarray}}
\newcommand{\seg}[1]{\mbox{$\overline{#1}$}}

\newtheorem{dfn}{Definition}[section]
\newtheorem{rmk}{Remark}[section]
\newtheorem{lma}{Lemma}[section]
\newtheorem{propn}{Proposition}[section]
\newtheorem{exmpl}{Example}[section]
\newtheorem{conjec}{Conjecture}[section]
\newtheorem{claim}{Claim}[section]
\newtheorem{notn}{Notation}[section]
\newtheorem{thm}{Theorem}[section]
\newtheorem{crlry}{Corollary}[section]

\newcommand{\prf}{\noindent{{\bf Proof} :\ }}
\newcommand{\QED}{\vrule height 1.4ex width 1.0ex depth -.1ex\ \medskip}

\newcommand{\ACMTOG}{{\sl ACM Trans.\ Graph.\ }}
\newcommand{\AMM}{{\sl Amer.\ Math.\ Monthly\ }}
\newcommand{\BIT}{{\sl BIT\ }}
\newcommand{\CACM}{{\sl Commun.\ ACM\ }}
\newcommand{\CAD}{{\sl Comput.\ Aided Design\ }}
\newcommand{\CAGD}{{\sl Comput.\ Aided Geom.\ Design\ }}
\newcommand{\CGIP}{{\sl Comput.\ Graph.\ Image Proc.\ }}
\newcommand{\CJ}{{\sl Computer\ J.\ }}
\newcommand{\DCG}{{\sl Discrete\ Comput.\ Geom.\ }}
\newcommand{\IBMJRD}{{\sl IBM\ J.\ Res.\ Develop.\ }}
\newcommand{\IJCGA}{{\sl Int.\ J.\ Comput.\ Geom.\ Applic.\ }}
\newcommand{\IEEECGA}{{\sl IEEE Comput.\ Graph.\ Applic.\ }}
\newcommand{\IEEETPAMI}{{\sl IEEE Trans.\ Pattern Anal.\ Machine Intell.\ }}
\newcommand{\JACM}{{\sl J.\ Assoc.\ Comput.\ Mach.\ }}
\newcommand{\JAT}{{\sl J.\ Approx.\ Theory\ }}
\newcommand{\MC}{{\sl Math.\ Comp.\ }}
\newcommand{\MI}{{\sl Math.\ Intelligencer\ }}
\newcommand{\NM}{{\sl Numer.\ Math.\ }}
\newcommand{\SIAMJNA}{{\sl SIAM J.\ Numer.\ Anal.\ }}
\newcommand{\SIAMR}{{\sl SIAM Review\ }}

\newcommand{\figg}[3]{\begin{figure}[htbp]\vspace{#3}\caption{#2}\label{#1}\end{figure}}

\title{
The bisector of a point \\
and a plane parametric curve
}

\author{
Rida~T.~Farouki \\
IBM Thomas~J.~Watson Research Center, \\
P.~O.~Box 218, Yorktown Heights, NY 10598. \\ \\
John~K.~Johnstone \\
Department of Computer Science, \\
The Johns Hopkins University, Baltimore, MD 21218.
}

\date{}

\maketitle
\thispagestyle{empty}

\begin{abstract}
The {\it bisector\/} of a fixed point ${\bf p}$ and a smooth
plane curve $C$ --- {\it i.e.}, the locus traced by a point
that remains equidistant with respect to ${\bf p}$ and $C$ ---
is investigated in the case that $C$ admits a regular polynomial
or rational parameterization. It is shown that the bisector may be
regarded as (a subset of) a ``variable--distance'' offset curve to
$C$ which has the attractive property, unlike fixed--distance offsets,
of being {\it generically\/} a rational curve. This ``untrimmed
bisector'' usually exhibits irregular points and self--intersections
similar in nature to those seen on fixed--distance offsets. A {\it
trimming procedure}, which identifies the parametric subsegments
of this curve that constitute the true bisector, is described in
detail. The bisector of the point ${\bf p}$ and any finite segment
of the curve $C$ is also discussed.
\end{abstract}

\newpage\thispagestyle{empty}\mbox{}\vfill\eject
\pagenumbering{arabic}
\setcounter{page}{1}
\thispagestyle{plain}

\section{Introduction}
\label{intro}

In descriptive geometry \cite{coxeter69} the parabola is
characterized as the locus traced by a point that remains
equidistant with respect to a fixed point ${\bf p}$ (the
{\it focus\/}) and a given straight line $L$ (the {\it
directrix\/}). Thus, points of the plane that lie to one
side of the parabola are closer to ${\bf p}$ than to $L$,
while those that lie on the other side are closer to $L$
than to ${\bf p}$. In this sense, the parabola may be
regarded as the ``bisector'' of the point ${\bf p}$ and
the line $L$.

If we substitute a smooth plane curve $C$ in place of
the straight line $L$, the bisector locus is of a more
subtle nature. This paper is concerned with investigating
the geometric properties of such loci, and formulating
tractable representations for them. Point/curve bisectors
arise in a variety of geometric ``reasoning'' and geometric
decomposition problems ({\it e.g.}, planning paths of maximum
clearance in robotics, or computing Voronoi diagrams for areas
with curvilinear boundaries). Although they are much simpler
than other loci --- line/curve and curve/curve bisectors ---
that arise in these contexts, no systematic analysis of the
properties of point/curve bisectors is currently available
in the literature.

However, there has been considerable interest recently in
the general topic of bisectors. They play a key r\^ole in
computing the {\it medial axis transform\/} or ``skeleton''
of planar shapes (see, for example, Bookstein \cite{bookstein79}
and Lee \cite{lee82}). Yap \cite{yap87} discusses the bisectors
of points, lines, and circles in the context of Voronoi diagrams.
Also in the context of Voronoi diagrams, Held \cite{Held91}
treats the construction of bisectors in numerical--control
machining applications. Yap and Alt \cite{yap89} analyze
the complexity of the bisector computation for two algebraic
curves, quoting an upper bound of $16m^6$ on the degree of
the bisector for curves of degree $m$. Nackman and Srinivasan
\cite{nackman91} discuss generic properties of the bisector of
two linearly--separable sets of arbitrary dimension, from the
perspective of point--set topology.

Hoffmann and Vermeer \cite{HV91} develop systems of equations
that define ``equal--distance'' curves and surfaces (another
term for bisectors, viewed as offsets from given curves/surfaces).
Voronoi surfaces --- {\it i.e.}, the bisectors of two given
surfaces --- are also discussed by Hoffmann \cite{H90} and Dutta
and Hoffmann \cite{DH90}. Finally, we note that the notion of
the ``offset'' to a given curve (or surface) is closely related
to that of bisectors; this relationship will be directly exploited
in the development of this paper. A detailed discussion of offset
curves is given by Farouki and Neff \cite{farouki90a,farouki90b}.

\subsection{Regular parametric curves}

We shall focus here on the important case where the curve $C$
is described parametrically, ${\bf r}(u)=\{x(u),y(u)\}$, having
derivatives
\be \label{derivs}
{\bf r}'(u) = \{x'(u),y'(u)\} \,, \quad
{\bf r}''(u) = \{x''(u),y''(u)\} \,, \quad
\ldots {\rm etc.}
\ee
continuous to at least third order for all $u \in I$, where $I$
denotes some finite, semi--infinite, or infinite parameter domain
of interest. We assume that the parameterization of ${\bf r}(u)$
is {\it proper}, {\it i.e.}, there is a one--to--one correspondence
between parameter values $u$ and points $(x,y)$ of the curve
locus --- except, possibly, for finitely many instances where
${\bf r}(u)$ crosses itself.

Since improper parameterizations arise rather infrequently in
practice, and identifying them is not in general a straightforward
matter (see \cite{sederberg84,sederberg86}), we shall not dwell on
this issue. However, we do need to impose an additional constraint
on the parameterization of ${\bf r}(u)$ --- namely, that it be
{\it regular\/} over the parameter domain of interest:

\begin{dfn}
The parametric speed of ${\bf r}(u)=\{x(u),y(u)\}$ is the
function
\be \label{sigma}
\sigma(u) = \sqrt{x'^2(u)+y'^2(u)}
\ee
of the parameter $u$, and the curve is said to have a {\it
regular\/} parameterization on the interval $I$ if and only if
$\sigma(u)\not=0$ for all $u \in I$.
\end{dfn}

It should be noted that any polynomial or rational curve that
has an irregular parameterization will not, in general, exhibit
a smooth locus: the points where $\sigma(u)=0$ correspond to
{\it cusps\/} (sudden tangent reversals) or, exceptionally,
discontinuities in higher--order differential characteristics
\cite{farouki91b}.

We denote by $|{\bf v}|$ the Euclidean norm $\sqrt{v_x^2
+v_y^2}$ of a vector ${\bf v}=(v_x,v_y)$. Thus, we shall also
write $|{\bf r}'(u)|$ for the parametric speed of ${\bf r}(u)$,
to suit the context. Since we are concerned solely with {\it
real\/} functions $x(u)$ and $y(u)$ of a {\it real\/} parameter
$u$, we note that $\sigma(u)=0\;\Longleftrightarrow\;x'(u)=
y'(u)=0$.

Although much of the ensuing discussion holds for any regular
parametric curve, we shall deal primarily with the two functional
forms encountered most often in practice: the {\it polynomial\/}
curve ${\bf r}(u)=\{X(u),Y(u)\}$ of degree $n$ defined by
\be \label{polycurve}
X(u) = \sum_{k=0}^n a_k u^k \,, \quad
Y(u) = \sum_{k=0}^n b_k u^k \,,
\ee
the coefficients $\{a_k,b_k\}$ being real numbers that satisfy
$a_n^2+b_n^2\not=0$, and the {\it rational\/} curve ${\bf r}(u)=
\{X(u)/W(u),Y(u)/W(u)\}$ of degree $n$, where
\be \label{ratcurve}
X(u) = \sum_{k=0}^n a_k u^k \,, \quad
Y(u) = \sum_{k=0}^n b_k u^k \,, \quad
W(u) = \sum_{k=0}^n c_k u^k \,,
\ee
the coefficients $\{a_k,b_k,c_k\}$ being again real numbers
with either $a_n^2+b_n^2\not=0$ or $c_n\not=0$. In (\ref
{ratcurve}) we also assume that there are no factors common
to {\it all three\/} of the polynomials $X,Y,W$, {\it i.e.},
that ${\rm GCD}(X,Y,W) ={\rm constant}$. (${\rm GCD}(\cdots)$
denotes the ``greatest common divisor'' of the indicated set
of polynomials, as determined by one or more applications of
Euclid's algorithm \cite{uspensky48}.)

Since polynomial curves are a special class of rational
curves with $W(u)={\rm constant}$, when we speak specifically
of rational curves it will be understood implicitly that $W(u)
\not={\rm constant}$. Roots of the polynomial $W(u)$ correspond
to ``points at infinity'' on the rational curve (\ref{ratcurve}).
In most applications, we are concerned only with the {\it affine\/}
part of a rational curve, {\it i.e.}, its locus for all parameter
values $u$ such that $W(u)\not=0$.

\begin{rmk}
{\rm
A sufficient and necessary condition for the polynomial curve
$(\ref{polycurve})$ to have a regular parameterization is that
\be \label{regpoly}
{\rm GCD}(X',Y')={\rm constant} \,,
\ee
while in the case of the rational curve $(\ref{ratcurve})$ we
require
\be \label{regrat}
{{\rm GCD}(WX'-W'X,WY'-W'Y) \over {\rm GCD}(W,W')}={\rm constant}
\ee
for the affine locus to have a regular parameterization.
}
\end{rmk}

\begin{exmpl}
{\rm
We have already noted the simple nature of the bisector of a point
and a straight line. Another common case that yields an ``elementary''
bisector arises if we take the curve $C$ to be a {\it circle}. Assume,
without loss of generality, that $C$ is of unit radius and centered
on the origin, and let the given point be ${\bf p}=(\alpha,\beta)$.
Then the bisector is evidently the locus of points $(x,y)$ that
satisfy
\be
\left|\,\textstyle{\sqrt{x^2+y^2}}-1\,\right|
\,=\, \sqrt{(x-\alpha)^2+(y-\beta)^2} \,,
\ee
where the left-- and right--hand sides represent the distance of
the variable point $(x,y)$ from the circle $C$ and the fixed point
${\bf p}$, respectively. Squaring twice to clear radicals, we see
that the bisector has the implicit equation
\ba \label{ellipse}
& & (1-\alpha^2)\,x^2 \,+\, (1-\beta^2)\,y^2
 \,-\, 2\alpha\beta\,xy \nonumber \\
& & \quad +\ (\alpha^2+\beta^2-1)\,(\alpha x+\beta y)
 \,-\, \quarter(\alpha^2+\beta^2-1)^2 \,=\, 0 \,,
\ea
which clearly represents a conic section. The nature of this conic
may be determined \cite{eisenhart60} by inspecting the signs of the
invariants
\be
k \,=\, (1-\alpha^2)(1-\beta^2) - \alpha^2\beta^2 \,,
\ee
which simplifies to $k=1-(\alpha^2+\beta^2)$, and
\be
D \,=\,
\left| \begin{array}{ccc}
1-\alpha^2 &
-\,\alpha\beta &
%\half\alpha(\alpha^2+\beta^2-1) \\ \\
\half(\alpha^2+\beta^2-1)\alpha \\ \\
-\,\alpha\beta &
1-\beta^2 &
%\half\beta(\alpha^2+\beta^2-1) \\ \\
\half(\alpha^2+\beta^2-1)\beta \\ \\
%\half\alpha(\alpha^2+\beta^2-1) &
\half(\alpha^2+\beta^2-1)\alpha &
%\half\beta(\alpha^2+\beta^2-1) &
\half(\alpha^2+\beta^2-1)\beta &
-\quarter(\alpha^2+\beta^2-1)^2
\end{array} \right| \,,
\ee
which yields $D=-\,\quarter[\,1-(\alpha^2+\beta^2)\,]^2=-\,\quarter k^2$
on expansion.

When $k=0$, {\it i.e.}, the point ${\bf p}$ lies {\it on\/} the circle
$C$, equation (\ref{ellipse}) becomes $(\beta x-\alpha y)^2=0$, and
the bisector degenerates into a straight line --- the normal to $C$ at
${\bf p}$ --- counted twice. Otherwise, the bisector is a non--degenerate
conic: an ellipse or (one branch of a) hyperbola according to whether
$k>0$ or $k<0$, {\it i.e.}, whether ${\bf p}$ lies {\it inside\/} or {\it
outside\/} the circle $C$. Some examples are illustrated in Figure~\ref
{fig:circle}.
} \QED
\end{exmpl}

\figg{fig:circle}{Representative bisectors of a point and a circle.}
{2.75in}

\subsection{The point/curve distance function}

In characterizing the parabola as the bisector of a point
${\bf p}$ and a straight line $L$, the meaning of the ``distance''
of any point from $L$ is clear: it is simply the length of the
{\it unique\/} perpendicular from the point in question to the
straight line $L$. In substituting a smooth parametric curve $C$
in place of $L$, we need to generalize this notion of distance
(see \cite{kelly79}):

\begin{dfn}
The distance of a point ${\bf q}$ from a regular parametric curve
${\bf r}(u)$ $=\{x(u),y(u)\}$ defined on the parameter interval
$I$ is given by
\be \label{distance}
\dist({\bf q},{\bf r}(u)) \,=\,
\inf_{u \,\in\, I} \, |\,{\bf q}-{\bf r}(u)\,| \,,
\ee
{\it i.e.}, it is the greatest lower bound, for all $u \in I$, on the
distance measured between the specified point ${\bf q}$ and each point
${\bf r}(u)$ along the curve.
\end{dfn}

If ${\bf r}(u)$ is a polynomial curve, of course, the bound (\ref
{distance}) is always attained at a {\it finite\/} parameter value
$u$ regardless of whether $I$ has finite or infinite extent. When
${\bf r}(u)$ is a rational curve and $I$ is not finite, however, it
is possible that (\ref{distance}) may be attained in the limit $|u|
\to\infty$ if the degree of $W(u)$ is not less than the greater
of the degrees of $X(u)$ and $Y(u)$.

Consider, for example, the point ${\bf q}=(-2,0)$ and the unit circle
centered on the origin, ${\bf r}(u)=\{\,(1-u^2)/(1+u^2),2u/(1+u^2)\,\}$
for $u \in (-\infty,+\infty)$. The closest point on ${\bf r}(u)$ to
${\bf q}$ is clearly ${\bf r}(\pm\infty)=(-1,0)$.

Note that as $|u|\to\infty$ the rational curve (\ref{ratcurve})
converges to an affine point $(x_\infty,y_\infty)$, where
$|x_\infty|=|a_n/c_n|$ and $|y_\infty| =|b_n/c_n|$, if and only
if ${\rm deg}(W)\ge\max({\rm deg}(X),{\rm deg}(Y))$; otherwise it
has a point at infinity at infinite values of $u$. In the former
case it is always possible to re--parameterize (\ref{ratcurve})
by a bilinear transformation of the parameter so as to make
$(x_\infty,y_\infty)$ correspond to a {\it finite\/} parameter
value.

\begin{propn} \label{polydist}
For the point ${\bf q}=(a,b)$ and the polynomial curve ${\bf r}(u)$
given by $(\ref{polycurve})$, let $\{u_1,\ldots,u_N\}$ be the set of
distinct odd--multiplicity roots of the polynomial
\be \label{Pperp}
P_\perp(u) \,=\,
[\,a-X(u)\,]\,X'(u) \,+\, [\,b-Y(u)\,]\,Y'(u)
\ee
of degree $2n-1$ on the interior of the interval $I$, augmented by
the finite end points, if any, of $I$. Then the distance function
$(\ref{distance})$ may be expressed as
\be \label{distance2}
\dist({\bf q},{\bf r}(u)) \,=\,
\min_{k \,\in\, \{1,\ldots,N\}} \, |\,{\bf q}-{\bf r}(u_k)\,| \,.
\ee
\end{propn}

\prf On differentiating the expression
\be \label{distsq}
|\,{\bf q}-{\bf r}(u)\,|^2 \,=\,
[\,a-X(u)\,]^2 \,+\, [\,b-Y(u)\,]^2 \,,
\ee
we see that the distance $|\,{\bf q}-{\bf r}(u)\,|$ will attain a
``stationary'' value whenever $[\,a-X(u)\,]\,X'(u)\,+\,[\,b-Y(u)\,]
\,Y'(u)=0$, {\it i.e.}, at the roots of the polynomial $P_\perp(u)$.
(Note that, by virtue of the constraint (\ref {regpoly}), this
equation will never be satisfied in the degenerate case $X'(u)=
Y'(u)=0$.) Only those roots of $P_\perp(u)$ that are of {\it odd\/}
multiplicity identify local {\it extrema\/} of $|\,{\bf q}-{\bf r}
(u)\,|$, however. To evaluate (\ref{distance}) we must compare
the values of $|\,{\bf q}-{\bf r}(u)\,|$ at each odd root of
$P_\perp(u)$ on the parameter interval $I$ and at its end points
{\it if\/} they are finite (since $|\,{\bf q}-{\bf r}(u)\,| \to
\infty$ for any polynomial curve as $|u|\to\infty$). Then $\dist
({\bf q},{\bf r}(u))$ is given by the smallest of these values.
\QED

A result analogous to Proposition \ref{polydist} holds for regular
rational curves, provided we replace the odd roots of the polynomial
(\ref{Pperp}) by those of
\ba \label{Rperp}
P_\perp(u) \!
&=& \! [\,aW(u)-X(u)\,]\,[\,W(u)X'(u)-W'(u)X(u)\,] \nonumber \\
&+& \! [\,bW(u)-Y(u)\,]\,[\,W(u)Y'(u)-W'(u)Y(u)\,]
\ea
satisfying $W(u)\not=0$ on the interval $I$. (These roots never
correspond to the degenerate case $W(u)X'(u)-W'(u)X(u)=W(u)Y'(u)-
W'(u)Y(u)=0$ with $W(u)\not=0$ when the constraint (\ref{regrat})
is imposed.)

Thus, in computing $\dist({\bf q},{\bf r}(u))$ for a rational curve,
we compare the values of the distance $|\,{\bf q}-{\bf r}(u)\,|$
at each odd root of (\ref{Rperp}) and at finite end points of the
parameter interval $I$ satisfying $W(u)\not=0$ and/or at infinite
end points in the case that ${\rm deg}(W)\ge\max({\rm deg}(X),
{\rm deg}(Y))$.

\begin{rmk}
{\rm
Note that equations (\ref{Pperp}) and (\ref{Rperp}) have an
obvious geometric interpretation: in each case, the roots of the
polynomial $P_\perp(u)$ identify points of the curve where lines
drawn from ${\bf q}$ meet ${\bf r}(u)$ orthogonally. The distance
(\ref{distance}) is then simply the smallest of the lengths of these
perpendiculars (and the chords drawn from ${\bf q}$ to the affine
end points of ${\bf r}(u)$, if any). Even--multiplicity roots of
$P_\perp(u)$ can be ignored, since they identify points of ${\bf r}
(u)$ where the distance $|\,{\bf q}-{\bf r}(u)\,|$ ``levels off''
but then continues to increase or decrease --- {\it i.e.}, it
does not attain a local extremum. Insisting that ${\bf r}(u)$
have a regular parameterization guarantees that $P_\perp(u)$ will
not vanish in degenerate cases where $x'(u)=y'(u)=0$ (which do
not, in general, identify perpendiculars to ${\bf r}(u)$ from
${\bf q}$).
}
\end{rmk}

The preceding characterization of the point/curve distance function
for polynomial and rational curves extends to any smooth analytic
curve. In general, we write
\be \label{distance3}
\dist({\bf q},{\bf r}(u)) \,=\,
\min_{k \,\in\, \{1,\ldots,N\}} \, |\,{\bf q}-{\bf r}(u_k)\,| \,,
\ee
where $u_1,\ldots,u_N$ identify all the points on the analytic
curve ${\bf r}(u)$ where a line drawn from ${\bf q}$ meets the curve
orthogonally, as well as the affine end points (if any) of ${\bf r}(u)$.
In the case of general analytic curves, of course, the determination
of these parameter values will usually be more difficult than computing
the odd roots of the polynomials (\ref{Pperp}) and (\ref{Rperp}) in
the case of polynomial and rational curves.

\begin{rmk}
{\rm
For an algebraic curve $C$ defined by the implicit polynomial equation
$f(x,y)=0$, we have
\be
\dist({\bf q},C) \,=\,
\min_{k \,\in\, \{1,\ldots,N\}} \, |\,{\bf q}-{\bf r}_k\,| \,,
\ee
where ${\bf r}_1=(x_1,y_1),\ldots,{\bf r}_N=(x_N,y_N)$ identify all
points on the curve ({\it i.e.}, $f(x_k,y_k)=0$) that satisfy
\be
(a-x)\,f_y(x,y) \,-\, (b-y)\,f_x(x,y) \,=\, 0 \,,
\ee
$f_x$ and $f_y$ being the partial derivatives of $f$ with respect to
$x$ and $y$. The points ${\bf r}_k$ are either singular points of
$f(x,y)=0$ (in the sense of an algebraic curve) or identify locations
on $f(x,y)=0$ where the line drawn from ${\bf q}$ meets the curve
orthogonally (since the tangent at a point of $f(x,y)=0$ is parallel
to the vector $(f_y,-f_x)$ evaluated there \cite[p.~55]{W50}).
}
\end{rmk}

\begin{exmpl}
{\rm
It is interesting to observe that, in operation, the procedure of
Proposition~\ref{polydist} for computing $\dist({\bf p},{\bf r}(u))$
can run counter to geometric intuition. For example, one usually
expects that at the discrete parameter values $u_1,\dots,u_N$
entering on the right--hand side of (\ref{distance2}), the curve
${\bf r}(u)$ {\it lies locally to one side of its tangent line}.
However, a simple counter--example to this notion is provided by
the case ${\bf p}=(0,h)$ and ${\bf r}(u)=\{u,u^3\}$, for which
$u=0$ is always an odd--multiplicity root of $P_\perp(u)$ that
should enter in (\ref{distance2}) although the tangent line cuts
the curve there.
Moreover, the closest point of 
${\bf r}(u)$ to ${\bf p}$ can be the point ${\bf r}(0)$.
} \QED
\end{exmpl}

We now note some important properties of the distance function:

\begin{propn}
When ${\bf r}(u)$ is a regular polynomial or rational curve, the
function $\dist({\bf q},{\bf r}(u))$ is continuous --- but not
always differentiable --- with respect to the location of the point
${\bf q}$.
\end{propn}

\prf The continuity of $\dist({\bf q},{\bf r}(u))$ follows immediately
from a general result concerning the distance between a point ${\bf q}$
and a non--empty set ${\cal S}$ in any metric space (see \cite{kelly79},
Theorem 3, p.~53). However, it is instructive to examine this property
in greater detail within the present context. For the sake of brevity
we discuss only the case of a polynomial curve ${\bf r}(u)$ below; the
extension to rational curves is relatively straightforward.

Consider expression (\ref{Pperp}) as a polynomial in {\it three\/}
variables, namely, the parameter value $u$ and the coordinates
$(a,b)$ of the point ${\bf q}$:
\be \label{Pperp2}
P_\perp(u,a,b) \,=\,
[\,a-X(u)\,]\,X'(u) \,+\, [\,b-Y(u)\,]\,Y'(u) \,.
\ee
At the reference point $(a_0,b_0)$, let $u_{k,0} \in I$ be a simple
root of (\ref{Pperp2}), so that $\partial P_\perp/\partial u\not=0$
when $u=u_{k,0}$. Then by the {\it implicit function theorem\/} \cite
[p.~362]{buck78} we infer the existence of a function $\phi_k(a,b)$,
analytic in some two--dimensional neighborhood ${\cal N}_k$ of
$(a_0,b_0)$, such that $\phi_k(a_0,b_0)=u_{k,0}$ and
\be
P_\perp(\phi_k(a,b),a,b) \equiv 0
\quad {\rm for\ all\ } (a,b) \in {\cal N}_k \,.
\ee
Intuitively, the function $\phi_k(a,b)$ describes how the root $u_k$
of $P_\perp(u)$ moves in the vicinity of its nominal value $u_{k,0}$
as the point ${\bf q}=(a,b)$ executes any path within the neighborhood
${\cal N}_k$ of its nominal location ${\bf q}_0=(a_0,b_0)$.

(If $u_k$ represents a finite end point of the parameter interval $I$
on which ${\bf r}(u)$ is defined, rather than a simple root of (\ref
{Pperp2}), it can be incorporated into the above framework by simply
taking $\phi_k(a,b) \equiv u_k$.)

Thus, about any nominal location ${\bf q}_0=(a_0,b_0)$, we may
invoke (\ref{distance2}) to formulate the distance function in a
neighborhood of that location as
\be \label{distance4}
\dist({\bf q},{\bf r}(u)) \,=\,
\min_{k \,\in\, \{1,\ldots,N\}} \,
|\,{\bf q}-{\bf r}(\phi_k({\bf q}))\,|
\quad {\rm for\ all\ } {\bf q} \in {\cal N} \,,
\ee
where ${\cal N}=\bigcap\,{\cal N}_k$ represents the area common
to each of the neighborhoods of ${\bf q}_0=(a_0,b_0)$ in which the
root functions $\phi_k({\bf q})=\phi_k(a,b)$ are analytic.

In the formulation (\ref{distance4}), the continuity of $\dist
({\bf q},{\bf r}(u))$ with respect to ${\bf q}=(a,b)$ at the
(arbitrary) reference point ${\bf q}_0=(a_0,b_0)$ is now apparent
--- each of the terms
\be \label{distancek}
|\,{\bf q}-{\bf r}(\phi_k({\bf q}))\,| \,=\,
\sqrt{ \, [\,a-X(\phi_k(a,b))\,]^2 + [\,b-Y(\phi_k(a,b))\,]^2 }
\ee
is continuous with respect to ${\bf q}$ at $(a_0,b_0)$, since the
functions $\phi_k(a,b)$ are analytic there and the curve ${\bf r}(u)=
\{X(u),Y(u)\}$ is continuous everywhere, and although the index $k$
that achieves the minimum in (\ref{distance4}) may suddenly jump
--- from $i$ to $j$, say --- as we move through $(a_0,b_0)$, we
nevertheless have $|\,{\bf q}-{\bf r}(\phi_i({\bf q}))\,|=|\,{\bf q}
-{\bf r}(\phi_j({\bf q}))\,|$ at any such jump.

If such a jump occurs in traversing $(a_0,b_0)$, however, $\dist
({\bf q},{\bf r}(u))$ will not, in general, be differentiable with
respect to ${\bf q}$ there. To see why, we consider the {\it
directional derivative}
\be \label{vnabla}
{\bf v}\cdot\!\nabla_{\bf q} \; \dist({\bf q},{\bf r}(u))
\,=\, \left[\, \lambda\,{\partial \over \partial a}
+ \mu\,{\partial \over \partial b} \;\right]
\dist({\bf q},{\bf r}(u)) \,,
\ee
which measures the rate of change of the distance function in the
direction of the unit vector ${\bf v}=(\lambda,\mu)$ at the point
$(a_0,b_0)$ at which the partial derivatives in (\ref{vnabla}) are
evaluated. Now by formal differentiation using the chain rule, and
noting that $P_\perp(\phi_k(a,b),a,b)=0$, the partial derivatives of
the functions $\Delta_k(a,b)=|\,{\bf q}-{\bf r}(\phi_k({\bf q}))\,|$
given by (\ref{distancek}) may be expressed as
\be \label{pderivs}
{\partial\Delta_k \over \partial a}
= {a-X(\phi_k(a,b)) \over \Delta_k(a,b)}
\quad {\rm and} \quad
{\partial\Delta_k \over \partial b}
= {b-Y(\phi_k(a,b)) \over \Delta_k(a,b)} \,.
\ee
In (\ref{pderivs}) it is understood that $k$ represents the index
minimizing $\Delta_k(a,b)$, and if $k$ jumps from $i$ to $j$ on
passing through $(a_0,b_0)$ in the direction ${\bf v}$, it is in
general true that
\be
X(\phi_i(a_0,b_0)) \not= X(\phi_j(a_0,b_0))
\quad {\rm and} \quad
Y(\phi_i(a_0,b_0)) \not= Y(\phi_j(a_0,b_0)) \,,
\ee
although $\Delta_i(a_0,b_0)=\Delta_j(a_0,b_0)$. Therefore, the
magnitude of the derivative (\ref{vnabla}) is discontinuous in
general whenever we traverse a point $(a_0,b_0)$ for which there
is a jump in the index $k$ that realizes the mimimum value on the
right--hand side of expression (\ref{distance2}). \QED

\subsection{Point/curve bisectors}

We are now ready to give a formal definition of point/curve bisectors:

\begin{dfn} \label{defbsctr}
The bisector $B({\bf p},C)$ of a fixed point ${\bf p}$ and a plane curve $C$
is the locus traced by a point that remains equidistant with respect
to ${\bf p}$ and $C$, in the sense of the distance function $(\ref
{distance})$.
\end{dfn}

That the bisector of ${\bf p}$ and ${\bf r}(u)$ does indeed form
a continuous locus can be seen from the preceding discussion. For
if ${\bf q}=(a,b)$ is a point of the bisector for which $\dist
({\bf q},{\bf r}(u))$ is realized by a {\it unique\/} index $k$
on the right--hand side of (\ref{distance2}), then the direction
${\bf v}=(\lambda,\mu)$ of an infinitesimal displacement along
the bisector --- {\it i.e.}, the {\it tangent\/} to the bisector
at ${\bf q}$ --- is uniquely determined by the requirement of
maintaining equal distance with respect to ${\bf p}$ and
${\bf r}(u)$, namely
\be \label{equidist}
{\bf v}\cdot\!\nabla_{\bf q} \; |\,{\bf q}-{\bf p}\,| \,=\,
{\bf v}\cdot\!\nabla_{\bf q} \; \dist({\bf q},{\bf r}(u)) \,,
\ee
where ${\bf v}\cdot\nabla_{\bf q}=\lambda\partial/\partial a +
\mu\partial/\partial b$ denotes the directional derivative with
respect to ${\bf q}=(a,b)$ moving along ${\bf v}=(\lambda,\mu)$.

\figg{fig:Btangent}{Local tangent direction for the bisector
of ${\bf p}$ and ${\bf r}(u)$.}{2.75in}

Now $\nabla_{\bf q}\,|\,{\bf q}-{\bf p}\,|$ is simply a unit vector
in the direction of ${\bf q}-{\bf p}$, while $\nabla_{\bf q}\,\dist
({\bf q},{\bf r}(u))$ is the unit normal ${\bf n}(u_k)$ to ${\bf r}
(u)$ at the (unique) point $u_k$ thereof for which $\dist({\bf q},
{\bf r}(u))$ is realized. Thus, to satisfy (\ref{equidist}), the
direction of the tangent ${\bf v}$ to the bisector at ${\bf q}$ must
be such as to {\it bisect the angle between the vectors ${\bf q}-
{\bf p}$ and ${\bf n}(u_k)$} (see Figure~\ref{fig:Btangent}).

This provides a basis for numerically tracing the bisector, if the
parameter value $u_k$ at which $\dist({\bf q},{\bf r}(u))=|\,{\bf q}
-{\bf r}(u_k)\,|$ can be determined for any given ${\bf q}$. If
this value is {\it not\/} unique, a more sophisticated analysis is
required to choose among the possibilities (in general, the tangent
is discontinuous at such points --- the bisector is not smooth,
although it is point--continuous). We shall not pursue this approach
further, however.

\section{Offset curves and bisectors}
\label{offsets}

In formulating a tractable representation for the bisector of a
point ${\bf p}$ and a curve ${\bf r}(u)$, it will be useful to recall
the definition and some basic properties of the {\it offset curves\/}
to ${\bf r}(u)$ (see \cite{farouki90a,farouki90b} for a more thorough
discussion).

\subsection{Constant--distance offsets}

We begin by noting that if the curve ${\bf r}(u)$ is regular on the
interval $u \in I$, its unit normal vector
\be \label{normal}
{\bf n}(u) \,=\, {(y'(u),-x'(u)) \over \sqrt{x'^2(u)+y'^2(u)}}
\ee
is defined and continuous for all $u \in I$.

\begin{dfn}
The ``untrimmed'' offset at (signed) distance $d$ to a regular parametric
curve ${\bf r}(u)$ is the locus defined by
\be \label{offset}
{\bf r}_o(u) \,=\, {\bf r}(u) + d\,{\bf n} (u) \,.
\ee
\end{dfn}

Note that when ${\bf r}(u)$ is a polynomial or rational curve,
the offset ${\bf r}_o(u)$ is {\it not}, in general, a polynomial
or rational curve, because of the radical in the denominator of
(\ref{normal}). Consequently, offset curves are often approximated
by piecewise--polynomial forms in computer--aided design (CAD)
applications \cite{hoschek88,klass83,pham88,tiller84}, to render
them compatible with existing representational and algorithmic
infrastructures. (The ``interior'' and ``exterior'' offsets, at
distances $-d$ and $+d$, together constitute an {\it algebraic
curve\/} described by an implicit polynomial equation $f_o(x,y)=0$
\cite{farouki90b}. See also \cite {farouki90c,farouki91a} for
discussion of a special class of polynomial curves whose offsets
{\it are\/} rational.)

We call the locus (\ref{offset}) the ``untrimmed'' offset for the
following reason: {\it Corresponding points\/} ${\bf r}(u)$ and
${\bf r}_o(u)$ on the given curve and its untrimmed offset are
evidently distance $d$ apart, measured along their mutual normal
direction. However, the point ${\bf r}_o(u)$ of the untrimmed offset
is not necessarily distance $d$, in the sense of the distance function
(\ref{distance}), from the {\it entire curve\/} ${\bf r}(u)$. We shall
call the locus having this latter property the ``trimmed'' offset to
${\bf r}(u)$, since it is obtained by deleting certain continuous
segments of (\ref{offset}).

The trimming procedure may be characterized by the following
property of the untrimmed offset curve (\ref{offset}):

\begin{propn} \label{offtrim}
For a regular polynomial or rational curve ${\bf r}(u)$ defined on
the interval $u \in I$, let $\{i_1,\ldots,i_M\}$ be the ordered set
of parameter values on $I$ that correspond to self--intersections
of its untrimmed offset ${\bf r}_o(u)$ at distance $d$, {\it i.e.},
${\bf r}_o(i_j)={\bf r}_o(i_k)$ for some $1 \le j \not= k \le M$.
Then, denoting the end points of $I$ by $i_0$ and $i_{M+1}$, we
have either
\be
\dist({\bf r}_o(t),{\bf r}(u)) \,\equiv\, d \quad {\rm for\ all\ }
t \in (i_k,i_{k+1})
\ee
or
\be
\dist({\bf r}_o(t),{\bf r}(u)) \,<\, d \quad {\rm for\ all\ }
t \in (i_k,i_{k+1})
\ee
on each span $(i_k,i_{k+1})$ for $k=0,\ldots,M$ between successive
self--intersections of the untrimmed offset.
\end{propn}

\prf See Theorem 4.4 in \cite{farouki90a}. \QED

Proposition \ref{offtrim} indicates that if we dissect the
untrimmed offset ${\bf r}_o(u)$ into the subsegments delineated by
its self--intersections, then each subsegment should be retained
or discarded in its entirety in forming the trimmed offset. It is
sufficient to test the distance of a single point interior to each
span $(i_k,i_{k+1})$ of ${\bf r}_o(u)$ from the given curve ${\bf r}
(u)$ (the mid point $\half(i_k+i_{k+1})$, say) to determine whether
or not that span should be eliminated (see Figure~\ref{fig:offsets}).

\figg{fig:offsets}{Untrimmed and trimmed offsets to a parabola.}
{2.75in}

Note that trimming an offset curve is a problem in the {\it global
topology\/} of the locus defined by (\ref{offset}); we know of
no simpler algorithm for the trimming process than the methodical
dissect--and--test procedure described above. The parameter values
${i_1,\ldots,i_M}$ of the self--intersections can be specified as
the roots of certain polynomials of rather high degree; see \cite
{farouki90b}. We shall encounter a similar ``trimming'' problem in
computing point/curve bisectors.

\begin{rmk}
{\rm
The trimmed offsets at distance $\pm d$ to a given curve ${\bf r}(u)$
are the ``level curves'' for the point/curve distance function $(\ref
{distance})$, {\it i.e.}, they are the loci of points ${\bf q}$ that
satisfy $\dist({\bf q},{\bf r}(u))=|d\;\!|$.
}
\end{rmk}

\subsection{The ``untrimmed'' point/curve bisector}
\label{sec:untrim}

We can generalize the notion of an (untrimmed) offset curve at fixed
distance $d$ to a given regular curve ${\bf r}(u)$ by substituting
any continuous function $d(u)$ of the parameter $u$ in place of the
constant $d$. The differentiability of the {\it variable--distance
offset curve}
\be \label{varoffset}
{\bf r}_o(u) \,=\, {\bf r}(u) + d(u) {\bf n}(u)
\ee
is then constrained by that of the ``displacement function'' $d(u)$.
We shall find the form (\ref{varoffset}) to be valuable in analyzing
point/curve bisectors.

Consider the {\it family of normal lines\/} to a given regular curve
${\bf r}(u)$. These lines may be parameterized in the form
\be \label{nline}
{\bf r}(u) + \lambda\,{\bf n}(u) \,,
\ee
where $u$ selects a point on the curve, and $\lambda$ measures the
signed distance along the normal line from that point. Given any point
${\bf p}$ not on ${\bf r}(u)$, the location ${\bf q}$
along (\ref{nline}) that is equidistant from ${\bf p}$ and
the curve point ${\bf r}(u)$ is uniquely identified by the condition
$|\,{\bf q}-{\bf r}(u)\,|=|\,{\bf q}-{\bf p}\,|$, which reduces to
\be \label{lambda}
|\lambda| \,=\, |\,{\bf r}(u)+\lambda\,{\bf n}(u)-{\bf p}\,| \,.
\ee

\figg{fig:du}{Definition of the displacement function $d(u)$.}{2.75in}

Now for each $u$, let $d(u)$ denote the unique value $\lambda$
satisfying condition (\ref{lambda}), and let $\psi(u)$ be the angle
between the vector from ${\bf r}(u)$ to ${\bf p}$ and the normal
${\bf n}(u)$, measured in the right--handed sense defined by a unit
vector ${\bf z}$ orthogonal to the plane of the curve. Referring to
Figure~\ref{fig:du} and noting that the points ${\bf p}$, ${\bf q}$,
and ${\bf r}(u)$ define an isosceles triangle, we have
\be \label{du0}
d(u) \,=\, \half\,|\,{\bf p}-{\bf r}(u)\,|\sec\psi(u) \,.
\ee
by using the law of cosines.
Since $({\bf p}-{\bf r}(u))\cdot{\bf n}(u)=|\,{\bf p}-{\bf r}(u)\,|
\cos\psi(u)$, we can also express $d(u)$ as
\be \label{du}
d(u) \,=\, {|\,{\bf p}-{\bf r}(u)\,|^2 \over
2\,({\bf p}-{\bf r}(u))\cdot{\bf n}(u)} \,.
\ee
If we regard the tangent line to the curve at ${\bf r}(u)$ as dividing
the plane into two halves, it is evident from (\ref{du}) that $d(u)$
will be positive or negative according to whether or not ${\bf p}$
lies in the half--plane that ${\bf n}(u)$ points in to.

Note that (\ref{du}) is {\it not\/} (in general) a rational function
of $u$, because of the radical incurred in computing the unit normal
vector ${\bf n}(u)$.

\begin{dfn}
The {\it untrimmed bisector\/} of a fixed point ${\bf p}$ and a
regular parametric curve ${\bf r}(u)$ is the variable--distance offset
$(\ref{varoffset})$ to ${\bf r}(u)$ with the displacement function
$(\ref{du})$.
\end{dfn}

Thus, the untrimmed bisector is simply the locus of points on the
normal lines (\ref{nline}) that are equidistant from each curve point
${\bf r}(u)$ and the given point ${\bf p}$ (see Figure~\ref{fig:du}).
We will show below that it is appropriately named, {\it i.e.}, it
is a {\it superset\/} of the true bisector $B({\bf p},{\bf r}(u))$.
Figure~\ref{fig:varoffset} illustrates the formulation of untrimmed
point/curve bisectors as variable--distance offsets, in the simple
case of a parabola.

\figg{fig:varoffset}{Untrimmed bisectors as variable--distance offsets.}
{2.75in}

\begin{rmk}
{\rm
When ${\bf r}(u)$ is a polynomial or rational curve, the untrimmed
bisector defined by $(\ref{varoffset})$ and $(\ref{du})$ has a {\it
rational\/} parameterization, since the radicals in $d(u)$ and
${\bf n}(u)$ cancel each other.
}
\end{rmk}

Note that the displacement function satisfies $d(u)\not=0$ for all
$u$ if the given point ${\bf p}$ does not lie on the curve ${\bf r}(u)$.
However, the untrimmed bisector will exhibit a ``point at infinity''
for each parameter value $\tau$ that satisfies $({\bf p}-{\bf r}(\tau))
\cdot{\bf n}(\tau)=0$ ({\it i.e.}, the curve normal ${\bf n}(\tau)$
is orthogonal to the vector from ${\bf r}(\tau)$ to ${\bf p}$ or,
equivalently, the tangent line at ${\bf r}(\tau)$ passes through
${\bf p}$). For the polynomial curve (\ref{polycurve}) satisfying
(\ref{regpoly}), the parameter values corresponding to these points
at infinity are the roots of the polynomial
\be \label{Pinf}
P_\infty(u) \,=\,
[\,\alpha-X(u)\,]\,Y'(u) \,-\, [\,\beta-Y(u)\,]\,X'(u) \,,
\ee
which is of degree $2n-2$ (at most) when ${\bf r}(u)$ is of degree $n$.
Similarly, for the rational curve (\ref{ratcurve}), the polynomial whose
roots identify points at infinity on the untrimmed bisector is
\ba \label{Rinf}
P_\infty(u) \!
&=& \! [\,\alpha W(u)-X(u)\,]\,[\,W(u)Y'(u)-W'(u)Y(u)\,] \nonumber \\
&-& \! [\,\beta  W(u)-Y(u)\,]\,[\,W(u)X'(u)-W'(u)X(u)\,] \,.
\ea
The roots of (\ref{Rinf}) identify only the {\it affine\/} points of
${\bf r}(u)$ that induce points at infinity on the untrimmed bisector
--- assuming that (\ref{regrat}) holds. In addition, points at infinity
of ${\bf r}(u)$, corresponding to the roots of $W(u)$, will give rise
to points at infinity on the untrimmed bisector.

We will denote the untrimmed bisector defined by (\ref{varoffset})
and (\ref{du}) by ${\bf b}(u)$, with homogeneous coordinates given by
polynomials $X_b(u)$, $Y_b(u)$, $W_b(u)$. For the polynomial curve
(\ref{polycurve}), it may be verified that
\ba \label{pbsctr}
X_b \! &=& \! [\,\alpha^2-X^2+(\beta-Y)^2\,]\,Y'
 \,-\, 2(\beta-Y)XX' \,, \nonumber \\
Y_b \! &=& \! 2(\alpha-X)YY'
 \,-\, [\,(\alpha-X)^2+\beta^2-Y^2\,]\,X' \,, \nonumber \\
W_b \! &=& \! 2\,[\,(\alpha-X)Y'-(\beta-Y)X'\,] \,,
\ea
while in the case of the rational curve (\ref{ratcurve}) we have
\ba \label{rbsctr}
X_b \! &=& \! [\,\alpha^2W^2-X^2+(\beta W-Y)^2\,]\,V
 \,-\, 2(\beta W-Y)XU \,, \nonumber \\
Y_b \! &=& \! 2(\alpha W-X)YV
 \,-\, [\,(\alpha W-X)^2+\beta^2W^2-Y^2\,]\,U \,, \nonumber \\
W_b \! &=& \! 2W\,[\,(\alpha W-X)V-(\beta W-Y)U\,] \,,
\ea
where $U=WX'-W'X$ and $V=WY'-W'Y$. Note that $W_b \propto P_\infty$
in the case of a polynomial curve, while $W_b \propto WP_\infty$ in
the rational case.

\begin{rmk}
{\rm
It may be verified from (\ref{pbsctr}) and (\ref{rbsctr}) that when
${\bf r}(u)$ is a {\it polynomial\/} curve of degree $n$, the untrimmed
bisector ${\bf b}(u)$ is a rational curve of degree $3n-1$ at most,
whereas if ${\bf r}(u)$ is a {\it rational\/} curve of degree $n$, the
untrimmed bisector is of degree $4n-2$ at most.
}
\end{rmk}

\begin{exmpl}
\label{exmpl:prbla}
{\rm
The simplest polynomial curve, other than a straight line, is the
parabola. Consider the case ${\bf p}=(\alpha,\beta)$ and ${\bf r}(u)=
\{u,u^2\}$. From equations (\ref{pbsctr}), we have the representation
\ba \label{Bprbla}
X_b(u) \! &=& \! 2u\,(u^4-2\beta u^2+\alpha^2+\beta^2-\beta) \,,
\nonumber \\
Y_b(u) \! &=& \! -\ 3u^4+4\alpha u^3-u^2+2\alpha u-\alpha^2-\beta^2 \,,
\nonumber \\
W_b(u) \! &=& \! -\ 2\,(u^2-2\alpha u+\beta) \,,
\ea
for the untrimmed bisector, which is evidently a rational curve of
degree {\it five}. Note that the roots of the denominator polynomial
$W_b(u)$ are simply
\be \label{uinfprbla}
u \,=\, \alpha \pm \sqrt{\alpha^2-\beta}
\ee
which identify real points at infinity of (\ref{Bprbla}) at {\it finite\/}
parameter values when $\beta<\alpha^2$, {\it i.e.}, ${\bf p}$ lies
``outside'' the parabola. Since $\max({\rm deg}(X_b),{\rm deg}(Y_b)) >
{\rm deg}(W_b)$, the parameter values $u=\pm\infty$ also identify points
at infinity on the untrimmed bisector (\ref{Bprbla}), regardless of
the location of $(\alpha,\beta)$. Figure~\ref{fig:parabola} illustrates
representative examples of the curves defined by (\ref{Bprbla}).
} \QED
\end{exmpl}

\figg{fig:parabola}{Untrimmed bisectors of a point and a parabola.}
{2.75in}

\begin{exmpl}
\label{exmpl:ellps}
{\rm
As a simple example of the untrimmed bisector of a point and a rational
curve, we consider the case of an ellipse centered on the origin, with
semi--axes 1 and $k$. This has the rational parameterization
\be \label{pellipse}
X(u) \,=\, 1-u^2 \,, \quad
Y(u) \,=\, 2k\,u \,, \quad
W(u) \,=\, 1+u^2 \,.
\ee
Substituting the above into (\ref{rbsctr}), we find that the untrimmed
bisector is a rational curve of degree {\it six}, defined by
\ba \label{Bellps}
X_b(u) \! &=& \! (1-u^2)\,
[\, k(\alpha^2+\beta^2-1)u^4 \,+\, 4(1-k^2)\beta u^3 \nonumber \\
&& \!\!\!+\ 2k(\alpha^2+\beta^2-3+2k^2)u^2
\,+\, 4(1-k^2)\beta u \,+\, k(\alpha^2+\beta^2-1) \,] \,, \nonumber \\
Y_b(u) \! &=& \! 2u\,
[\, (\alpha^2+\beta^2+2(1-k^2)\alpha+1-2k^2)u^4 \nonumber \\
&& \!\!\!+\ 2(\alpha^2+\beta^2-1)u^2 \,+\, \alpha^2+\beta^2
-2(1-k^2)\alpha+1-2k^2 \,] \,, \nonumber \\
W_b(u) \! &=& \! 2(1+u^2)^2\,
[ -\,k(\alpha+1)u^2 \,+\, 2\beta u \,+\, k(\alpha-1) \,] \,.
\ea
In this case, the untrimmed bisector will have real points at infinity
only when the condition $\alpha^2+(\beta/k)^2>1$ is satisfied, {\it i.e.},
${\bf p}$ lies {\it outside\/} the ellipse. They occur at the parameter
values
\be \label{uinfellps}
u \,=\, {\beta \pm \displaystyle{\sqrt
{k^2\alpha^2+\beta^2-k^2}} \over k(\alpha+1)} \,.
\ee
Some examples are illustrated in Figure~\ref{fig:ellipse}.
} \QED
\end{exmpl}

\figg{fig:ellipse}{Untrimmed bisectors of a point and an ellipse.}
{5.5in}

\subsection{Irregular points of the untrimmed bisector}
\label{sec:irregpts}

It is evident from Figures~\ref{fig:parabola} and \ref{fig:ellipse}
that in general the untrimmed bisector of a point ${\bf p}$ and a
regular curve ${\bf r}(u)$ is {\it not\/} a smooth locus, even though
${\bf r}(u)$ is necessarily smooth if it has a regular parameterization.
The cusps of the untrimmed bisector become important for trimming 
in later sections.

Recall \cite{kreyszig59} that for any regular parametric curve
${\bf r}(u)$, the elementary differential characteristics at each
point may be expressed in terms of the parametric derivatives
${\bf r}'(u),{\bf r}''(u),\ldots$ there as
\be \label{diffchar}
{\bf t} \,=\, {{\bf r}' \over |{\bf r}'|} \,, \quad
{\bf n} \,=\, {\bf t} \cross {\bf z} \,, \quad
\kappa \,=\, {({\bf r}' \cross {\bf r}'') \cdot {\bf z}
  \over |{\bf r}'|^3} \,,
\ee
${\bf z}$ being a unit vector orthogonal to the plane of the curve.
The {\it normal\/} ${\bf n}(u)$ and {\it tangent\/} ${\bf t}(u)$ form
an orthonormal basis $({\bf n},{\bf t},{\bf z})$ with ${\bf z}$ at
each point $u$, while $\kappa(u)$ is the (signed) {\it curvature}.
The variation of the tangent and normal along the curve is described
by the {\it Frenet equations\/}:
\be \label{frenet}
{\bf t}' \,=\, -\,\sigma\kappa\,{\bf n}
\quad {\rm and} \quad
{\bf n}' \,=\, \sigma\kappa\,{\bf t} \,,
\ee
where $\sigma$ is the parametric speed (\ref{sigma}) of ${\bf r}(u)$.
Higher--order derivatives of ${\bf t}(u)$ and ${\bf n}(u)$ are readily
expressed in terms of ${\bf t}(u)$ and ${\bf n}(u)$ and the scalar
functions $\sigma(u)$, $\kappa(u)$, and their derivatives --- for
example,
\ba \label{frenet2}
{\bf t}'' \! &=& \! -\,\sigma^2\kappa^2\,{\bf t} \,-\,
(\sigma'\kappa+\sigma\kappa')\,{\bf n} \,, \nonumber \\
{\bf n}'' \! &=& \! (\sigma'\kappa+\sigma\kappa')\,{\bf t}
 \,-\, \sigma^2\kappa^2\,{\bf n} \,.
\ea

If we denote by ${\bf b}(u)$ the parametric representation of the
untrimmed bisector obtained by substituting from (\ref{du}) into
(\ref{varoffset}), then the parametric derivatives of ${\bf b}(u)$
may be written as
\ba \label{bderivs}
{\bf b}' \! &=& \! {\bf r}' \,+\,
  d'\,{\bf n} \,+\, d\,{\bf n}' \,, \nonumber \\
{\bf b}'' \! &=& \! {\bf r}'' \,+\,
  d''\,{\bf n} \,+\, 2d'\,{\bf n}' \,+\, d\,{\bf n}'' \,,
\ea
$\ldots$ etc. Setting ${\bf r}'=\sigma{\bf t}$ and ${\bf r}''=\sigma'
{\bf t}-\sigma^2\kappa{\bf n}$ and substituting from (\ref{frenet}) and
(\ref{frenet2}), we can re--write (\ref{bderivs}) as
\ba \label{bderivs2}
{\bf b}' \! &=& \! \sigma(1+\kappa d\,)\,{\bf t} \,+\, d'\,{\bf n} \,,
 \nonumber \\
{\bf b}'' \! &=& \!
[\,\sigma'(1+\kappa d\,)+\sigma(\kappa'd+2\kappa d')\,]\,{\bf t}
 \nonumber \\
 & & \quad +\ [\,d''-\sigma^2\kappa(1+\kappa d\,)\,]\,{\bf n} \,,
\ea
where, using the form (\ref{du0}), the derivatives of the displacement
function $d(u)$ appropriate to the untrimmed bisector are most conveniently
expressed as
\ba \label{dderivs}
d'  \! &=& \! \sigma(1+\kappa d\,)\tan\psi \,,
\nonumber \\
d'' \! &=& \! [\,\sigma'(1+\kappa d\,)+\sigma\kappa'd\,]\tan\psi
\nonumber \\
    & & \quad +\ {\sigma^2(1+\kappa d\,) \over 2d}
    \,[\,1+2\kappa d+(1+4\kappa d)\tan^2\psi\,] \,,
\ea
$\ldots$ etc. In deriving (\ref{dderivs}), we make use of the fact that
the angle $\psi$ between the vector from ${\bf r}(u)$ to ${\bf p}$ and
the curve normal ${\bf n}(u)$ changes at the rate
\be \label{psideriv}
\psi' \,=\, {\sigma(1+2\kappa d\,) \over 2d}
\ee
with respect to $u$, as may be deduced by differentiating the relation
$\cos\psi=({\bf p}-{\bf r}(u))\cdot{\bf n}(u)\,/\,|\,{\bf p}-{\bf r}(u)\,|$.

Using the above expression for $d'$, we now see that the first parametric
derivative of the untrimmed bisector has the form
\be \label{bprime}
{\bf b}' \,=\, \sigma(1+\kappa d\,)\,({\bf t}+{\bf n}\tan\psi) \,.
\ee
At each value of $u$ such that ${\bf b}'(u)\not={\bf 0}$, the tangent
${\bf t}_b(u)$ to the untrimmed bisector is a unit vector in the direction
of (\ref{bprime}). Note that the magnitude of (\ref{bprime}), {\it i.e.},
the {\it parametric speed\/} $\sigma_b(u)$ of the untrimmed bisector, is
simply
\be \label{magbprime}
|{\bf b}'| \,=\, \sigma\,|\;\!1+\kappa d\,|\,|\sec\psi\;\!| \,.
\ee

Since the given curve ${\bf r}(u)$ is regular, its tangent ${\bf t}(u)$
and normal ${\bf n}(u)$ are defined and linearly independent at each $u$,
so ${\bf t}(u)+{\bf n}(u)\tan\psi(u)$ is never the zero vector. Moreover,
this vector varies continuously with $u$ except at those parameter values
where $\psi(u)=\pm\pi/2$, which correspond to the roots of the polynomial
(\ref{Pinf}), {\it i.e.}, the points at infinity on ${\bf b}(u)$. Hence,
at each $u$ such that $P_\infty(u)\not=0$, the unit vector
\be \label{vvector}
{\bf v}(u) \,=\, |\cos\psi(u)\;\!| \;
[\,{\bf t}(u)+{\bf n}(u)\tan\psi(u)\,]
\ee
in the direction of ${\bf t}(u)+{\bf n}(u)\tan\psi(u)$ is defined and
varies continuously with $u$. Since $\sigma(u)\not=0$ for all $u$ on
a regular curve, we see that the tangent ${\bf t}_b(u)={\bf b}'(u) /
|{\bf b}'(u)|$ to the untrimmed bisector is given in terms of ${\bf v}
(u)$ by
\be \label{btangent}
{\bf t}_b(u) \,=\,
{1+\kappa(u)d(u) \over |\,1+\kappa(u)d(u)\,|} \; {\bf v}(u) \,.
\ee

\begin{lma}
The untrimmed bisector ${\bf b}(u)$ exhibits a cusp, or sudden
tangent reversal, at those parameter values where $P_\infty(u)
\not=0$ and the curvature $\kappa(u)$ of the given curve ${\bf r}
(u)$ attains the local critical value
\be \label{kappacrit}
\kappa_c(u) \,=\, -\ {1 \over d(u)} \,=\,
-\ {2\cos\psi(u) \over |\,{\bf p}-{\bf r}(u)\,|} \,,
\ee
without being an extremum, {\it i.e.}, $\kappa'(u)\not=0$.
\end{lma}

\prf Let $\tau$ be a parameter value such that $P_\infty(\tau)\not
=0$ ({\it i.e.}, ${\bf b}(\tau)$ is an affine point of the untrimmed
bisector), and the curvature of the given curve ${\bf r}(u)$ satisfies
both $\kappa(\tau)=-1/d(\tau)$ and $\kappa'(\tau)\not=0$. Then $\tan
\psi(\tau)$ is finite, and the unit vector ${\bf v}(\tau)$ given by
(\ref{vvector}) is defined and continuous at $u=\tau$. On the other
hand, the scalar factor multiplying ${\bf v}(u)$ in (\ref{btangent})
is a ``step function,'' which changes abruptly from $-1$ to $+1$,
or vice--versa, at $u=\tau$ whenever $d/du\,[\,1+\kappa(u)d(u)\,]
\not=0$ at $u=\tau$ ({\it i.e.}, $\kappa'(\tau)d(\tau)+\kappa(\tau)
d'(\tau)\not=0)$. But from (\ref{dderivs}) we observe that $d'(\tau)
=0$ whenever $\kappa(\tau)=-1/d(\tau)$, and since we certainly have
$d(\tau)\not=0$ because ${\bf p}$ does not lie on ${\bf r}(u)$, the
condition $d/du\,[\,1+\kappa(u)d(u)\,]\not=0$ at $u=\tau$ is exactly
equivalent to $\kappa'(\tau)\not=0$. \QED

\begin{rmk}
{\rm
It is interesting to note that the criteria $\kappa(u)=-1/d(u)$ and
$\kappa'(u)\not=0$ identifying the cusps of the ``variable--distance''
offset $(\ref{varoffset})$ are identical to those for fixed--distance
offsets ($\kappa(u)=-1/d$, $\kappa'(u)\not=0$) with $d(u)={\rm constant}$
$\cite{farouki90a}$. This is not a {\it generic} feature of variable
offsets, but arises rather from the specific form $(\ref{du})$ of
$d(u)$ for the untrimmed bisector.
}
\end{rmk}

Using $(\ref{du})$ and $(\ref{diffchar})$,
the local critical curvature (\ref{kappacrit}) is attained at those
roots of the equation
\be \label{cuspeqn}
\left[\;
|\,{\bf p}-{\bf r}(u)\,|^2\,{\bf r}'(u)\cross{\bf r}''(u) \,+\,
2\,|{\bf r}'(u)|^2\,[\,{\bf p}-{\bf r}(u)\,]\cross{\bf r}'(u)
\;\right] \cdot {\bf z}
\,=\, 0
\ee
that satisfy $\kappa'(u)\not=0$. These parameter values correspond
to cusps on the untrimmed bisector ${\bf b}(u)$. When ${\bf r}(u)$
is the polynomial curve (\ref{polycurve}), equation (\ref{cuspeqn})
corresponds to a polynomial of degree $4n-4$ (at most) in $u$:
\ba \label{Pcusp}
P_c \! &=& \! [\,(\alpha-X)^2+(\beta-Y)^2\,]\,(X'Y''-X''Y') \nonumber \\
&& +\ 2\,({X'}^2+{Y'}^2)\,[\,(\alpha-X)Y'-(\beta-Y)X'\,] \,.
\ea
If ${\bf r}(u)$ is the rational curve (\ref{ratcurve}), the left--hand
side of (\ref{cuspeqn}) is a rational function in $u$, whose numerator
is a polynomial of degree $7n-6$ (at most) in $u$:
\ba \label{Rcusp}
P_c \! &=& \! W\,[\,(\alpha W-X)^2+(\beta W-Y)^2\,]\,(U_1V_2-U_2V_1)
\nonumber \\
&& +\ 2\,(U_1^2+V_1^2)\,[\,(\alpha W-X)V_1-(\beta W-Y)U_1\,] \,,
\ea
where we denote $(WX'-W'X,WY'-W'Y)$ and $(WX''-W''X,WY''-W''Y)$ by
$(U_1,V_1)$ and $(U_2,V_2)$, respectively, for brevity.

Consider now the behavior of the curvature $\kappa_b(u)$ along the
untrimmed bisector. By substituting from (\ref{bderivs2}) and (\ref
{dderivs}), a straightforward but laborious calculation gives
\be
({\bf b}'\cross{\bf b}'')\cdot{\bf z}
\,=\, -\ {\sigma^3(1+\kappa d\,)^2 \over 2d}\,\sec^2\psi \,,
\ee
and together with (\ref{magbprime}) the expression $\kappa_b(u)=
|{\bf b}'(u)|^{-3}[\,{\bf b}'(u)\cross{\bf b}''(u)\,]\cdot{\bf z}$
for the curvature of the untrimmed bisector reduces to
\be \label{bkappa}
\kappa_b \,=\, -\ {|\cos\psi\;\!| \over 2d\,|\;\!1+\kappa d\,|} \,.
\ee
It is worthwhile emphasizing the significance of equation (\ref
{bkappa}) in words: at any point $u$ on the untrimmed bisector, we
can express the curvature $\kappa_b(u)$ of ${\bf b}(u)$ simply in
terms of the curvature $\kappa(u)$ of the given curve ${\bf r}(u)$,
the displacement function $d(u)$ defined by (\ref{du}), and the
angle $\psi(u)$ between the vector ${\bf p}-{\bf r}(u)$ and the
curve normal ${\bf n}(u)$.

\begin{lma}
The untrimmed bisector ${\bf b}(u)$ has an extraordinary point ---
{\it i.e.}, a tangent--continuous point of infinite curvature ---
at those parameter values $u$ where $P_\infty(u)\not=0$ and the
curvature $\kappa(u)$ of the given curve ${\bf r}(u)$ attains an
extremum $(\kappa'(u)=0\ but\ \kappa''(u)\not=0)$ equal in value
to the local critical curvature $\kappa_c(u)$ defined by $(\ref
{kappacrit})$.
\end{lma}

\prf Let $\tau$ be such that $P_\infty(\tau)\not=0$ and the curvature
of ${\bf r}(u)$ satisfies $\kappa(\tau)=-1/d(\tau)$ with $\kappa'(\tau)
=0$ and $\kappa''(\tau)\not=0$. Then the unit vector ${\bf v}(\tau)$
given by (\ref{vvector}) is continuous at $u=\tau$ and, by the same
arguments as in the preceding Lemma, the scalar factor $(1+\kappa d\,)
/|\;\!1+\kappa d\,|$ multiplying it in (\ref{btangent}) is of the {\it
same\/} sign on either side of $\tau$. Thus, the untrimmed bisector
tangent ${\bf t}_b(u)$ is {\it continuous\/} at $u=\tau$. However,
since $1+\kappa(\tau)d(\tau)=0$, it is evident from (\ref{bkappa})
that the curvature $\kappa_b(u)$ of the untrimmed bisector will
increase without bound as we approach $\tau$. (Note in (\ref{bkappa})
that $\cos\psi\not=0$, since $\tau$ is not a root of $P_\infty(u)$,
and $d$ is never zero if the point ${\bf p}$ does not lie on
${\bf r}(u)$.) \QED

We may regard extraordinary points, and all other irregular points
of the untrimmed bisector generated by points of ${\bf r}(u)$ which
satisfy
\be
\kappa(u) \,=\, -1/d(u)
\quad {\rm and} \quad
\kappa'(u) = \kappa''(u) = \cdots = \kappa^{(r)}(u) = 0
\ee
with $r \ge 1$, as ``higher--order'' cusps. Unlike the simple or
``ordinary'' cusps, the occurrence of such points is exceptional.

\begin{exmpl}
{\rm
In the case of the parabola ${\bf r}(u)=\{u,u^2\}$ of Example~\ref
{exmpl:prbla}, the polynomial (\ref{Pcusp}) is the quartic
\be
P_c(u) \,=\, 3u^4 - 8\alpha u^3
 + 6\beta u^2 - (\alpha^2+\beta^2-\beta) \,.
\ee
Although $P_c(u)$ is of degree four, {\it Descartes Law of Signs\/}
\cite{uspensky48} and a careful analysis of the behavior of its
coefficients for real values of $\alpha$ and $\beta$ reveals that
it has just two distinct real roots when $\alpha^2+\beta^2-\beta>0$,
and none when $\alpha^2+\beta^2-\beta<0$. Exceptionally, if $\alpha^2
+\beta^2-\beta=0$, $P_c(u)$ has a double root at $u=0$, corresponding
to an extraordinary point of the untrimmed bisector; this occurs when
$(\alpha,\beta)$ {\it lies on the circle of curvature to the vertex\/}
$(0,0)$ of the parabola.

For the ellipse (\ref{pellipse}) in Example~\ref{exmpl:ellps}, the
``cusp polynomial'' becomes
\ba
P_c(u) \!\! &=& \!\! k\,[\,\alpha^2+\beta^2+2(1-k^2)\alpha+1-2k^2\,]\,u^6
 \nonumber \\
&+& \!\! 3k\,[\,\alpha^2+\beta^2-2(1-k^2)\alpha-3+2k^2\,]\,u^4
 \nonumber \\
&+& \!\! 16\,(1-k^2)\beta\,u^3
 \nonumber \\
&+& \!\! 3k\,[\,\alpha^2+\beta^2+2(1-k^2)\alpha-3+2k^2\,]\,u^2
 \nonumber \\
&+& \!\! k\,[\,\alpha^2+\beta^2-2(1-k^2)\alpha+1-2k^2\,] \,.
\ea
%\ba
%P_c(u) \!\! &=& \!\! k\,[\,(\alpha+1-k^2)^2+\beta^2-k^4\,]\,u^6
% \nonumber \\
%&+& \!\! 3k\,[\,(\alpha-1+k^2)^2+\beta^2-(2-k^2)^2\,]\,u^4
% \nonumber \\
%&+& 16\,(1-k^2)\beta\,u^3
% \nonumber \\
%&+& \!\! 3k\,[\,(\alpha+1-k^2)^2+\beta^2-(2-k^2)^2\,]\,u^2
% \nonumber \\
%&+& k\,[\,(\alpha-1+k^2)^2+\beta^2-k^4\,] \,.
%\ea
In this case the maximum number of distinct real roots is four, again
less than is suggested by the degree of $P_c(u)$. Note that we have
double roots at zero or infinity, giving rise to extraordinary points,
when $(\alpha,\beta)$ is such that the coefficient of $u^0$ or $u^6$
vanishes, respectively. (The coefficients of these terms define the
circles of curvature to the ellipse at the vertices $u=0$ and $u=\pm
\infty$; see the discussion in Example~\ref{exmpl:trimellps} below.)
} \QED
\end{exmpl}

\subsection{The true bisector}

We now show that the untrimmed bisector given by (\ref{varoffset})
and (\ref{du}) is a superset of the ``true'' bisector (Definition~\ref
{defbsctr}). Note that the ``true'' bisector of ${\bf p}$ and ${\bf r}
(u)$ can be visualized as follows. Consider a circle through ${\bf p}$,
increasing in size until it just touches the curve ${\bf r}(u)$. Now
consider all such circles, inflating in all directions from ${\bf p}$
until they touch ${\bf r}(u)$. The bisector is the locus of the centers
of these maximal circles. In the following discussion, it is useful to
keep this representation of the bisector in mind.

\begin{dfn}
\label{d:Cq}
Let $C_{\bf q}$ denote the circle with center ${\bf q}$ and radius
$|\,{\bf q}-{\bf p}\,|$.
\end{dfn}

\begin{rmk}
\label{rmk:bis}
{\rm
A point ${\bf q}$ lies on the bisector of ${\bf p}$ and the regular
curve ${\bf r}(u)$ if and only if (see Figure~\ref{fig:bis}):
\begin{itemize}
\item
$C_{\bf q}$ is ``empty'' --- no point of ${\bf r}(u)$ lies in its
interior; and
\item
$C_{\bf q}$ is tangent to ${\bf r}(u)$ in at least one point.
\end{itemize}
%       If ${\bf r}(u)$ is not regular,
%       then the second condition should instead be
%       that the boundary of the circle $C_{\bf q}$
%       contains at least one point of ${\bf r}(u)$.
}
\end{rmk}
\prf
${\bf q}$ lies on the bisector if and only if the closest point
of ${\bf r}(u)$ to ${\bf q}$ is at distance $|\,{\bf q}-{\bf p}\,|$.
If $C_{\bf q}$ is tangent to ${\bf r}(u)$ at $u=u_0$ but otherwise
empty, then ${\bf r}(u_0)$ is closest to ${\bf q}$ and is at distance
$|\,{\bf q}-{\bf r}(u_0)\,|=|\,{\bf q}-{\bf p}\,|$.
\QED

\figg{fig:bis}{Condition for a point ${\bf q}$ to lie on the bisector
$B({\bf p},{\bf r}(u))$.}{2.75in}

(Note that if we are computing the bisector of a point and a finite
curve segment, the circle $C_{\bf q}$ may contain curve points that
correspond to parameter values outside the domain $I$ that defines
the segment of interest.)

\begin{propn}
\label{p:superset}
The untrimmed bisector of ${\bf p}$ and ${\bf r}(u)$ is a superset
of the true bisector of ${\bf p}$ and ${\bf r}(u)$.
\end{propn}
\prf
Let ${\bf q}$ be a point of the bisector of ${\bf p}$ and ${\bf r}(u)$.
By Remark~\ref{rmk:bis}, there exists a point ${\bf r}(u_{0})$ of the
curve that lies on the circle $C_{\bf q}$, such that the curve is tangent
to the circle at ${\bf r}(u_{0})$ or, equivalently, such that the normal
at ${\bf r}(u_{0})$ passes through the center ${\bf q}$ of the circle
$C_{\bf q}$. Thus, ${\bf r}(u_{0}) + d(u_{0}){\bf n}(u_{0}) = {\bf q}$
(recall that $d(u_0)=|\,{\bf q}-{\bf p}\,|={\rm radius\ of\ } C_{\bf q}$).
\QED

Since the second condition of Remark~\ref{rmk:bis} is satisfied for
{\it all\/} points ${\bf q}$ of the untrimmed bisector (by definition),
a point ${\bf q}$ of the untrimmed bisector is a point of the true
bisector if and only if the interior of $C_{\bf q}$ is empty.

\section{The trimming procedure}
\label{trimming}

The untrimmed bisector ${\bf b}(u)$ may be ``trimmed'' down to the true
bisector by deleting a finite number of segments. As with the untrimmed
offset, this is done by finding a finite number of special points that
identify possible deviations of ${\bf b}(u)$ from the true bisector.
There are four classes of these special points on the untrimmed bisector
(see Definitions~\ref{d:trim} and \ref{d:cri}), and we split the trimming
process into two stages. We now describe the first stage, which removes
``inactive'' segments.

\subsection{Active and inactive segments}

Along the normal line to every curve point ${\bf r}(u_0)$, there is a
corresponding point ${\bf b}(u_0)$ of the untrimmed bisector. Many of
these points ${\bf b}(u_0)$ do not belong to the true bisector, however,
because ${\bf r}(u_0)$ is clearly not the closest point of the curve to
${\bf b}(u_0)$ (see Figure~\ref{fig:notclosest}).

\figg{fig:notclosest}{${\bf r}(u_0)$ is not closest on ${\bf r}(u)$ to
its corresponding point ${\bf b}(u_0)$.}{2.75in}

\begin{dfn}
The points ${\bf r}(u_0)$ and ${\bf b}(u_0)={\bf r}(u_0)+d(u_0){\bf n}
(u_0)$ are called {\rm corresponding} points of the curve and the
untrimmed bisector.
\end{dfn}

\begin{dfn}
The point ${\bf q} = {\bf b}(u_{0})$ of the untrimmed bisector is
{\rm active} if either of the following conditions holds:
\begin{description}
\item[{\rm (1)}]
        ${\bf q}$ has more than one corresponding point on the
        curve\footnote{There are a finite number of such points;
        we will have more to say about them below.}
\item[{\rm (2)}]
        ${\bf q}$ has only one corresponding point ${\bf q}' =
        {\bf r}(u_0)$ on the curve, and either
\begin{description}
\item[{\rm (a)}]
        the circle of curvature at ${\bf q}'$ contains the point ${\bf p}$
        --- i.e., ${\bf p}$ lies on or inside the circle of curvature, or
\item[{\rm (b)}]
        the point ${\bf p}$ and the circle of curvature at ${\bf q}'$
        lie on opposite sides of the tangent at ${\bf q'}$
\end{description}
\end{description}
(see Figure~\ref{fig:active}). A segment $S$ of ${\bf b}(u)$ is active
if every point of $S$ is active.
\end{dfn}

\figg{fig:active}{Active points on the untrimmed bisector.}{2.75in}

To understand the significance of such ``active'' points on the untrimmed
bisector, we need the following result:

\begin{lma} \label{lma:circles}
Let ${\bf q}$ and ${\bf q}'$ be corresponding points on the untrimmed
bisector and the given curve. Then the circle $C_{\bf q}$ lies entirely
inside/outside the circle of curvature at ${\bf q}'$ according to whether
the point ${\bf p}$ lies inside/outside this circle of curvature.
\end{lma}
\prf
Let ${\cal C}$ denote the circle of curvature to ${\bf r}(u)$ at the
point ${\bf q}'$. Then the normal line to ${\bf r}(u)$ at ${\bf q}'$
is a diameter of ${\cal C}$, and the point ${\bf q}$ corresponding
to ${\bf q}'$ lies on this normal line. Noting that ${\bf q}$ is
the center of the circle $C_{\bf q}$ and that (by construction) both
${\bf q}'$ and ${\bf p}$ lie on this circle, we consider a variable
circle constrained to pass through ${\bf q}'$ whose center moves away
from ${\bf q}'$ along the normal line there. Such a circle evidently
lies entirely inside or outside of (or, exceptionally, coincides with)
the circle of curvature ${\cal C}$ at ${\bf q}'$. $C_{\bf q}$ is an
instance of such a circle --- therefore, the inclusion/exclusion of
any point of $C_{\bf q}$ in ${\cal C}$ is sufficient to guarantee that
all of $C_{\bf q}$ lies inside/outside ${\cal C}$. As already noted,
${\bf p}$ is a point of $C_{\bf q}$.
\QED

\begin{rmk}
\label{r:active}
{\rm
We observe that an active point appears to lie on the bisector, at least
locally. To see this, let ${\bf q}$ be an active point of ${\bf b}(u)$
with only one corresponding point ${\bf q}'$. If ${\bf p}$ lies inside or
on the circle of curvature at ${\bf q}'$, then by Lemma~\ref{lma:circles}
all of $C_{\bf q}$ also lies in the circle of curvature at ${\bf q}'$ and,
in particular, there exists a neighborhood of ${\bf q}'$
% ({\it i.e.}, the image of a neighborhood of $t'$ where ${\bf q}' = {\bf r}(t')$)
that lies completely outside of $C_{\bf q}$ \cite[p.~176]{H52}.
On the other hand, if ${\bf p}$ (and thus $C_{\bf q}$) lies on the opposite side
of the tangent at ${\bf q}'$ from the circle of curvature at ${\bf q}'$,
then again the curve in a neighborhood of ${\bf q}'$ lies completely outside of
$C_{\bf q}$.
Thus, the curve in some neighborhood of an active point (with one corresponding
point) lies completely outside of $C_{\bf q}$.
In other words, an active point acts at least locally
like a point of the bisector (Remark~\ref{rmk:bis}).
}
\end{rmk}

\begin{propn}
An inactive point of the untrimmed bisector of ${\bf p}$ and ${\bf r}(u)$
does not lie on the bisector of ${\bf p}$ and ${\bf r}(u)$.
\end{propn}
\prf
Let ${\bf q}$ be an inactive point of ${\bf b}(u)$ and let ${\bf q}'$
be the point of ${\bf r}(u)$ corresponding to ${\bf q}$. By definition,
${\bf p}$ and the circle of curvature at ${\bf q}'$ lie on the same side
of the tangent at ${\bf q}'$, but ${\bf p}$ lies outside this circle
(Figure~\ref{fig:active}(ii)). $C_{\bf q}$ and the circle of curvature
at ${\bf q}'$ are both circles that are tangent to ${\bf q}'$ (or,
equivalently, circles through ${\bf q}'$ with centers on the normal
line there), and they both lie on the same side of the tangent at
${\bf q}'$. Moreover, by Lemma~\ref{lma:circles}, $C_{\bf q}$ must
be strictly larger than (and strictly contain) the circle of curvature
at ${\bf q}'$. But all circles tangent at ${\bf q}'$ that are larger
than the circle of curvature there must lie on one side of the curve
in some neighborhood of ${\bf q}'$ \cite[p.~176]{H52}, and consequently
$C_{\bf q}$ must contain some points of the curve in its interior. We
conclude that ${\bf q}$ is not on the bisector (Remark~\ref{rmk:bis}).
\QED

Since the definition of an active point depends on the side of the
tangent (at ${\bf q}'$) that ${\bf p}$ and the circle of curvature lie
on, we are interested in points where this can change.
We are also interested in points where the position of ${\bf p}$
relative to the circle of curvature can change.

\begin{dfn}
\label{d:trim}
A point of the curve ${\bf r}(u)$ is an {\rm inflection} if the curvature
is zero at that point;
% regular curve => no cusps, no advantage in avoiding self-intersections
% concavity of the curve changes at this point
a point of ${\bf r}(u)$ is a {\rm class point}\footnote
       {This term is chosen in allusion to the class of a curve,
        which is the number of tangents that pass through a typical point
        not on the curve \cite[p.~115]{W50}.}
if the tangent at that point passes through ${\bf p}$; and a point of
${\bf r}(u)$ is a {\rm circular point} if the circle of curvature at that
point passes through ${\bf p}$. Equivalently, a point ${\bf r}(u_0)$ is
circular if the center of curvature there coincides with the corresponding
point ${\bf b}(u_{0})$ of the untrimmed bisector.
\end{dfn}

(To be more precise, an inflection of ${\bf r}(u)$ is a point where the
curvature {\it changes sign\/} --- so that $\kappa(u)=0$ and
$\kappa'(u)\not=0$ or, more generally, the lowest--order non--vanishing
derivative of $\kappa(u)$ is {\it odd}. For example, whereas $\kappa$
vanishes at $u=0$ on both ${\bf r}(u)=\{u,u^3\}$ and ${\bf r}=\{u,u^4\}$,
it changes sign in the former case but not in the latter. For simplicity
we adhere to the definition given above, thereby occasionally
including some points where the circle of curvature does not move from
one side of the curve to the other.)

We have already encountered ``class'' points and ``circular'' points,
in a somewhat different guise (Sections~\ref{sec:untrim} and \ref
{sec:irregpts}): they are, respectively, points of ${\bf r}(u)$ that
induce points at infinity and cusps on the untrimmed bisector. Thus,
for fixed $u$, equations (\ref{Pinf}) and (\ref{Rinf}) may be regarded
as expressing the condition for the point ${\bf p}=(\alpha,\beta)$ to
lie on the tangent line to the point ${\bf r}(u)$ of a polynomial or
rational curve. Likewise, for fixed $u$, equations (\ref{Pcusp}) and
(\ref{Rcusp}) are satisfied when ${\bf p}=(\alpha,\beta)$ lies on
the circle of curvature at the point ${\bf r}(u)$ of a polynomial or
rational curve (see Remark~\ref{active-comp} below).

\begin{thm}
\label{thm:active}
For a regular polynomial or rational curve ${\bf r}(u)$ defined on
the interval $u \in I$, and a point ${\bf p}$ not on ${\bf r}(u)$,
let ${\bf b}(u)$ be the untrimmed bisector of ${\bf p}$ and ${\bf r}
(u)$, and let $\{i_{1},\ldots,i_{M}\}$ be the ordered set of parameter
values on $I$ that correspond to inflections, class points, and circular
points of ${\bf r}(u)$. Then, denoting the end points of $I$ by $i_{0}$
and $i_{M+1}$, we have either
\be
{\bf b}(u) {\rm \ is\ active\ }
\quad {\rm for\ all\ } u \in (i_k,i_{k+1})
\ee
or
\be
{\bf b}(u) {\rm \ is\ inactive\ }
\quad {\rm for\ all\ } u \in (i_k,i_{k+1})
\ee
on each span $(i_k,i_{k+1})$ for $k=0,\ldots,M$.
\end{thm}
\prf
Consider a point ${\bf q}$ moving smoothly along the untrimmed
bisector ({\it i.e.}, such that its parameter value changes smoothly),
and let ${\bf q}'$ be the corresponding point on the given curve. By
definition, in order for ${\bf q}$ to change from active to inactive,
or vice versa, one of the following must occur: ${\bf p}$ must move to
a different side of the tangent at ${\bf q}'$; the circle of curvature
at ${\bf q}'$ must move to a different side of the tangent at ${\bf q}'$;
or ${\bf p}$ must move to a different side of the circle of curvature at
${\bf q}'$ ({\it i.e.}, from inside to outside or vice versa). The points
of the given curve associated with these occurrences are, respectively,
class points, inflections, and circular points. Thus, if ${\bf q}$ does
not traverse a point whose parameter value corresponds to one of these
special points, its status (active/inactive) will remain unchanged.
\QED

\begin{rmk}
\label{active-comp}
{\rm
Consider the computation of the endpoints of active segments. Using the
formula (\ref{diffchar}) for the curvature $\kappa(u)$ of ${\bf r}(u)$,
we see that inflections of the polynomial curve (\ref{polycurve}) occur
at the roots of the polynomial $X'Y''-X''Y'$, while for the rational
curve (\ref{ratcurve}) they are roots of $U_1V_2-U_2V_1$ where $(U_1,V_1)
=(WX'-W'X,WY'-W'Y)$ and $(U_2,V_2)=(WX''-W''X,WY''-W''Y)$. The class points
are points of ${\bf r}(u)$ where the condition $({\bf p}-{\bf r}(u))\cdot
{\bf n}(u)=0$ is satisfied, {\it i.e.}, ${\bf p}$ lies on the tangent line
at ${\bf r}(u)$ --- they are roots of (\ref{Pinf}) and (\ref{Rinf}) for
polynomial and rational curves, respectively. Finally, recall that a point
of ${\bf r}(u)$ is circular if its center of curvature is the corresponding
point ${\bf b}(u)$ of the untrimmed bisector. 
Recall that ${\bf b}(u)={\bf r}(u)+d(u){\bf n}(u)$; while
the circle of curvature has radius $\kappa
^{-1}(u)$ and center ${\bf e}(u)={\bf r}(u)-\kappa^{-1}(u){\bf n}(u)$
since, according to (\ref{diffchar}), $\kappa(u)$ is positive when
${\bf n}(u)$ points {\it away\/} from the center of curvature. (The
locus ${\bf e}(u)$ is called the {\it evolute\/} of the given curve
${\bf r}(u)$.)
Thus, the circular points satisfy $\kappa(u)=-1/d(u)\,$: for polynomial
and rational curves, they are roots of (\ref{Pcusp}) and (\ref{Rcusp}),
respectively.
}
\end{rmk}

\subsection{Critical points (self--intersections)}

In order to make the second --- and final --- refinement from active
segments to segments on the bisector, recall that a point ${\bf q}$ of
the untrimmed bisector is in the bisector if and only if its associated
circle $C_{\bf q}$ is empty. A boundary between the bisector and the
rest of the untrimmed bisector is marked by a transition from points
with empty circles to points with non--empty circles. These transitions
are associated with critical points.

\begin{dfn}
\label{d:cri}
A point ${\bf q}$ of the untrimmed bisector is a {\rm critical point}
if the circle $C_{\bf q}$ is tangent to ${\bf r}(u)$ at two or more
points (Figure~\ref{fig:critical}).
\end{dfn}

\figg{fig:critical}{Critical point (self--intersection) of the untrimmed
bisector.}{2.75in}

Although defined somewhat differently, critical points are actually
self--intersections of the untrimmed bisector.

\begin{propn}
\label{prop:cri}
A point ${\bf q}$ of the untrimmed bisector is a critical point if and
only if it is a self--intersection of ${\bf b}(u)$.
\end{propn}

\prf Let ${\bf q} \in {\bf b}(u)$ be a critical point, and let the
circle $C_{\bf q}$ be tangent to ${\bf r}(u)$ at the two points
${\bf r}(u_1)$ and ${\bf r}(u_2)$. Then the normal at ${\bf r}(u_1)$
passes through the center of $C_{\bf q}$ and, since both ${\bf r}
(u_1)$ and ${\bf p}$ lie on the boundary of this circle, a point
of the normal is equidistant from these two points at the center
(Figure~\ref{fig:critical}). Thus, ${\bf b}(u_{1}) = {\bf q}$. By
the same argument, we also have ${\bf b}(u_{2}) = {\bf q}$. Hence
${\bf q}$ must be a self--intersection of ${\bf b}(u)$.

Conversely, let ${\bf q}={\bf b}(u_{1})={\bf b}(u_{2})$ be a
self--intersection. Then ${\bf r}(u_{1})$, ${\bf r}(u_{2})$, and
${\bf p}$ are all equidistant from ${\bf q}$. Thus, ${\bf r}(u_{1})$
and ${\bf r}(u_{2})$ lie on the circle $C_{\bf q}$. Moreover, since
the curve normals at ${\bf r}(u_{1})$ and ${\bf r}(u_{2})$ pass
through the center ${\bf q}$ of this circle, the circle is also
tangent to the curve at the points ${\bf r}(u_{1})$ and ${\bf r}
(u_{2})$. Hence ${\bf q}$ is a critical point. \QED

Proposition~\ref{prop:cri} allows us to trim using self-intersections
while arguing the validity of this trim using critical points.

\begin{thm}
\label{thm:trim2}
For a regular polynomial or rational curve ${\bf r}(u)$ defined on
the interval $u \in I$, and a point ${\bf p}$ not on ${\bf r}(u)$, let
${\bf b}(u)$ be the untrimmed bisector of ${\bf p}$ and ${\bf r}(u)$,
and let $\{i_{1},\ldots,i_{M}\}$ be the ordered set of parameter values
on $I$ that correspond to endpoints of active segments on ${\bf b}(u)$
or self-intersections of ${\bf b}(u)$, {\it i.e.}, ${\bf b}(i_j) =
{\bf b}(i_k)$ for some $1 \leq j \neq k \leq M$. Then for every active
segment $u \in [\,i_j,i_k\,]$ of ${\bf b}(u)$ $({\rm where\ } i_j <
i_k)$, we have either
%       Let $i_{j} < i_{k}$ be the parameter values of the endpoints of an
%       active segment. Then, either
\be
{\bf b}(u) {\rm \ is\ on\ the\ bisector\ }
\quad {\rm for\ all\ } u \in (i_l,i_{l+1})
\ee
or
\be
{\bf b}(u) {\rm \ is\ not\ on\ the\ bisector\ }
\quad {\rm for\ all\ } u \in (i_l,i_{l+1})
\ee
on each span $(i_l,i_{l+1})$ for $l=j,\ldots,k-1$ between successive
self--intersections on that active segment.
\end{thm}

\prf
Let ${\bf q}_1={\bf b}(u_{1})$ and ${\bf q}_2={\bf b}(u_{2})$ (where
$u_1<u_2$) be two points on an active segment of the untrimmed bisector,
neither of them self--intersections, such that ${\bf q}_1$ belongs to
the bisector but ${\bf q}_2$ does not. We will show that the segment
$u \in (u_1,u_2)$ of ${\bf b}(u)$ must contain a self--intersection
(critical point). In other words, if an active segment of ${\bf b}(u)$
contains no self--intersections, that entire segment is or is not part
of the bisector.

Since ${\bf q}_1$ lies on the bisector, $C_{{\bf q}_1}$ touches
the curve at the point ${\bf q}'_1={\bf r}(u_{1})$ corresponding to
${\bf q}_1$ {\it and} $C_{{\bf q}_1}$ is empty (Remark~\ref{rmk:bis}).
As a point ${\bf q}$ moves along the untrimmed bisector from ${\bf q}
_1$ towards ${\bf q}_2$, the radius of the circle $C_{\bf q}$ changes
smoothly while it continues to pass through ${\bf p}$, and at all times
the curve in the neighborhood of the corresponding point ${\bf q}'$
lies completely outside $C_{\bf q}$ (since ${\bf q}$ is active; see
Remark~\ref{r:active}). Now in order for ${\bf q}$ to leave the
bisector, the circle $C_{\bf q}$ must become occupied, {\it i.e.},
the curve must enter $C_{\bf q}$ (Remark~\ref{rmk:bis}). We also know
that this happens eventually, because ${\bf q}_2$ is {\it not\/} on
the bisector. Since in the neighborhood of ${\bf q}'$ the curve does
not enter $C_{\bf q}$,
% this is important because otherwise in the limit the curve might touch
% {\it and} enter at ${\bf q}'$, {\it e.g.}, inflection.
%
%and the curve is regular, the curve cannot enter $C_{\bf q}$ at ${\bf q}'$ and
%
it must first enter $C_{\bf q}$ at some other point ${\bf q}'' \neq
{\bf q}'$. In order for the curve to enter $C_{\bf q}$, it must first
become tangent to $C_{\bf q}$, and the location of ${\bf q}$ when
this occurs is a critical point of the untrimmed bisector, since its
circle $C_{\bf q}$ has {\it two\/} points of tangency with the curve
--- one at ${\bf q}'$ and one at the other point ${\bf q}''$ where the
curve is about to enter the circle (Figure~\ref{fig:critical}). Thus,
the segment $u \in (u_1,u_2)$ of ${\bf b}(u)$ contains a critical point.
\QED

The computation of parameter values that correspond to critical points is
rather involved; we defer a full discussion to Section~\ref{sec:slfint}
below.

\begin{rmk}
{\rm
If the desired interpretation for the bisector of a point ${\bf p}$ and
a curve segment $S={\bf r}(I)$, defined on a {\it finite\/} parameter
domain $I$, is
\ba
        \{\,{\bf q}\!\! &|& \!\!{\rm dist}({\bf q},{\bf p})
                \,=\, {\rm dist}({\bf q},{\bf r}(u)) \nonumber \\
                && \!\!\! {\rm \ and\ the\ closest\ point\ of\ }
                {\bf r}(u) {\rm \ to\ } {\bf q} {\rm \ lies\ on\ } S \,\} \,,
\ea
rather than its usual meaning
\be
        \{\,{\bf q}\ |\ {\rm dist}({\bf q},{\bf p}) =
            {\rm dist}({\bf q},S) \,\} \,,
\ee
then the statement of Theorem~\ref{thm:trim2} is unchanged except that,
rather than using just the self--intersections of the segment ${\bf b}
(I)$, {\it i.e.}, self--intersections ${\bf b}(u_1) = {\bf b}(u_2)$
where $u_1, u_2 \in I$, one should use {\it all\/} self--intersections
of ${\bf b}(u)$ that lie on the segment ${\bf b}(I)$, {\it i.e.},
self--intersections ${\bf b}(u_1) = {\bf b}(u_2)$ where $u_1 \in I$
but $u_2$ need not lie on $I$.
}
\end{rmk}

We now have an algorithm for computing the bisector of a fixed point
${\bf p}$ and a regular parametric curve ${\bf r}(u)$:

\begin{enumerate}
\item
        Compute the untrimmed bisector of ${\bf p}$ and ${\bf r}(u)$,
        using (\ref{varoffset}) and (\ref{du}). For polynomial and
        rational curves, this untrimmed bisector is given by equations
        (\ref{pbsctr}) and (\ref{rbsctr}), respectively.
\item
        Find the inflections of the curve ${\bf r}(u)$, and its class
        points (using (\ref{Pinf}) and (\ref{Rinf}))
	and circular points (using (\ref{Pcusp}) and (\ref{Rcusp}))
	with respect to ${\bf p}$.
\item
        For each segment on the untrimmed bisector delineated by these
        special points, compare the distances of the segment midpoint
        to ${\bf p}$ and ${\bf r}(u)$, using (\ref{distance2}), and
        discard the segment if these distances are unequal.
\item
        Find the critical points (self--intersections) of the untrimmed
        bisector, using the methods described in Section~\ref{sec:slfint}
        below.
\item
        Split each remaining segment of the untrimmed bisector at these
        self--intersections. For each of the resulting segments, compare
        the distances of the midpoint to ${\bf p}$ and ${\bf r}(u)$,
        using (\ref{distance2}), and discard the segment if they are
        unequal. The remaining segments constitute the true bisector.
\end{enumerate}

\subsection{Computing the self--intersections}
\label{sec:slfint}

A self--intersection of the untrimmed bisector ${\bf b}(u)$ arises
if two (or more) {\it distinct\/} parameter values correspond to the
same geometric point on its locus (with this definition, we include
the exceptional cases where ${\bf b}(u)$ just ``touches'' itself among
the self--intersections). Thus, we are interested in identifying all
parameter values $u$ that satisfy the vector equation
\be \label{selfint}
{\bf b}(u+\xi) \,=\, {\bf b}(u) \quad {\rm for\ some\ } \xi\not=0 \,,
\ee
where ${\bf b}(u)=\{\,X_b(u)/W_b(u),Y_b(u)/W_b(u)\,\}$ is the rational
curve defined by equations (\ref{pbsctr}) or (\ref{rbsctr}), as
appropriate.

A number of approaches to constructing the minimal polynomial whose
roots correspond to self--intersection parameter values are possible.
That described below was found empirically to be the most ``economical,''
in the sense of generating the least extraneous intermediate data,
and at the same time offers a clear geometric picture. For brevity we
discuss only the case of polynomial curves, the rational case being a
straightforward but rather tedious generalization of this.

If ${\bf b}(u)$ is the untrimmed bisector of the point ${\bf p}=(\alpha,
\beta)$ and the regular polynomial curve ${\bf r}(u)=\{X(u),Y(u)\}$, we
know that corresponding points of ${\bf b}(u)$ and ${\bf r}(u)$ lie on
the normal lines to the latter. Given distinct parameter values $u$ and
$v$, we may express the point of intersection $(x_i,y_i)$ of the normal
lines at ${\bf r}(u)$ and ${\bf r}(v)$ in homogeneous coordinates as
\be \label{pi}
x_i(u,v) \,=\, {X_i(u,v) \over W_i(u,v)} \quad {\rm and} \quad
y_i(u,v) \,=\, {Y_i(u,v) \over W_i(u,v)} \,,
\ee
where the polynomials $X_i$, $Y_i$, and $W_i$ are defined by
\ba \label{Pi}
X_i(u,v) \!\! &=& \!\! Z(u)Y'(v)-Z(v)Y'(u) \,, \nonumber \\
Y_i(u,v) \!\! &=& \!\! Z(v)X'(u)-Z(u)X'(v) \,, \nonumber \\
W_i(u,v) \!\! &=& \!\! X'(u)Y'(v)-X'(v)Y'(u) \,.
\ea
For brevity, we have introduced the notation $Z(u)=X(u)X'(u)+Y(u)Y'(u)$
in (\ref{Pi}). Note that if $W_i(u,v)$ vanishes, the normal lines at
${\bf r}(u)$ and ${\bf r}(v)$ are parallel --- they intersect in a
``point at infinity.''

\begin{rmk}
\label{coincident}
{\rm
Expressions (\ref{pi}--\ref{Pi}) fail if $X_i(u,v)=Y_i(u,v)=W_i(u,v)
=0$, a situation that arises when the normal lines at ${\bf r}(u)$ and
${\bf r}(v)$ are {\it coincident}. If we set $v=u+\xi$ in the polynomials
(\ref{Pi}) and eliminate the variable $\xi$ between the denominator and
each component of the numerator, {\it i.e.}, if we compute the resultants
\ba \label{LxLy}
\Lambda_x(u) \! &=& \! {\rm Resultant}_{\,\xi}\,(\,X_i(u,\xi),W_i(u,\xi)\,) \,,
\nonumber \\
\Lambda_y(u) \! &=& \! {\rm Resultant}_{\,\xi}\,(\,Y_i(u,\xi),W_i(u,\xi)\,) \,,
\ea
then the parameter values that identify pairs of points on ${\bf r}(u)$
which have coincident normal lines are roots of the polynomial
\be \label{L}
\Lambda(u) \,=\, {\rm GCD}\,(\,\Lambda_x(u),\Lambda_y(u)\,) \,.
\ee
(The vanishing of the ``resultant'' of two polynomials with respect to
a given variable expresses a sufficient and necessary condition for them
to be satisfied at some value of that variable --- see, for example, \cite
{uspensky48}.)
}
\end{rmk}

In order for the point (\ref{pi}) to identify a self--intersection
of the untrimmed bisector --- corresponding to parameter values $u$
and $v$ --- $(x_i,y_i)$ must be equidistant from the three points
${\bf r}(u)$, ${\bf r}(v)$, and ${\bf p}$. This condition requires
the simultaneous satisfaction of the equations
\ba
{[\,x_i-X(u)\,]^2 \,+\, [\,y_i-Y(u)\,]^2} \! &=& \!
{[\,x_i-\alpha\,]^2 \,+\, [\,y_i-\beta\,]^2} \,, \nonumber \\
{[\,x_i-X(v)\,]^2 \,+\, [\,y_i-Y(v)\,]^2} \! &=& \!
{[\,x_i-\alpha\,]^2 \,+\, [\,y_i-\beta\,]^2} \,.
\ea
Cancelling $x_i^2+y_i^2$ from both sides of these equations and
substituting from (\ref{pi}--\ref{Pi}), we see that $u$ and $v$ must
be simultaneous roots of the polynomials
\ba \label{RStilde}
{\tilde R}(u,v) \!\!
&=& \!\! 2\,[\,\alpha-X(u)\,]\,[\,Z(u)Y'(v)-Z(v)Y'(u)\,] \nonumber \\
&+& \!\! 2\,[\,\beta-Y(u)\,]\,[\,Z(v)X'(u)-Z(u)X'(v)\,] \nonumber \\
&+& \!\! [\,X^2(u)+Y^2(u)-\alpha^2-\beta^2\,]
 \, [\,X'(u)Y'(v)-X'(v)Y'(u)\,] \,, \nonumber \\
{\tilde S}(u,v) \!\!
&=& \!\! 2\,[\,\alpha-X(v)\,]\,[\,Z(u)Y'(v)-Z(v)Y'(u)\,] \nonumber \\
&+& \!\! 2\,[\,\beta-Y(v)\,]\,[\,Z(v)X'(u)-Z(u)X'(v)\,] \nonumber \\
&+& \!\! [\,X^2(v)+Y^2(v)-\alpha^2-\beta^2\,]
 \, [\,X'(u)Y'(v)-X'(v)Y'(u)\,] \,.
\ea

A further reduction of expressions (\ref{RStilde}) is necessary before
proceeding. Each of the three terms comprising ${\tilde R}(u,v)$ and
${\tilde S}(u,v)$ exhibits a factor that vanishes on setting $v=u$. Thus,
$v-u$ is an overall factor of both ${\tilde R}(u,v)$ and ${\tilde S}(u,v)$.
This factor must be extracted since we are interested only in non--trivial
solutions that satisfy $v\not=u$. Hence we take
\be \label{RS}
R(u,v) \,=\, {{\tilde R}(u,v) \over v-u}
\quad {\rm and} \quad
S(u,v) \,=\, {{\tilde S}(u,v) \over v-u}
\ee
as our working polynomials. We now set $v=u+\xi$ in (\ref{RS}) and
re--write these polynomials in the form
\be \label{RS2}
R(u,\xi) \,=\, \sum_{k=0}^r a_k(u)\,\xi^k
\quad {\rm and} \quad
S(u,\xi) \,=\, \sum_{k=0}^s b_k(u)\,\xi^k \,,
\ee
{\it i.e.}, as polynomials in $\xi$ whose coefficients are polynomials
in $u$. Then if any particular value of $u$ is to be a parameter value
at which a self--intersection occurs, the polynomials (\ref{RS2}) must
have a common root $\xi$ at that value of $u$.

The expressions for the coefficients $\{a_k(u)\}$ and $\{b_k(u)\}$ in
(\ref{RS2}) in terms of $X(u)$, $Y(u)$, and their derivatives and the
coordinates $\alpha,\beta$ of ${\bf p}$ are rather complicated, so we
shall not present them explicitly here --- they are readily derived
using a computer algebra system.

\begin{propn}
Let the polynomials $a_k(u)$ and $b_k(u)$ be given by {\rm (\ref{RS2})},
and let $\Gamma(u)$ be the polynomial defined in terms of them by the
determinant
\be \label{G}
\Gamma \;=\; \left|\, \matrix{
 a_r &\cdot &\cdot &a_2   &a_1   &a_0   &{}    &{} &{} &{} \cr
 {}  &\cdot &\cdot &\cdot &\cdot &\cdot &\cdot &{} &{} &{} \cr
 {}  &{}    &a_r   &\cdot &\cdot &a_2   &a_1   &1  &{} &{} \cr
 {}  &{}    &{}    &a_r   &\cdot &\cdot &a_2   &0  &1  &{} \cr
 {}  &{}    &{}    &{}    &a_r   &\cdot &\cdot &0  &0  &1  \cr
 b_s &\cdot &\cdot &b_2   &b_1   &b_0   &{}    &{} &{} &{} \cr
 {}  &\cdot &\cdot &\cdot &\cdot &\cdot &\cdot &{} &{} &{} \cr
 {}  &{}    &b_s   &\cdot &\cdot &b_2   &b_1   &1  &{} &{} \cr
 {}  &{}    &{}    &b_s   &\cdot &\cdot &b_2   &0  &1  &{} \cr
 {}  &{}    &{}    &{}    &b_s   &\cdot &\cdot &0  &0  &1
 } \,\right| \,,
\ee
where there are $s$ rows of $a_k\!$'s followed by $r$ rows of $b_k\!$'s
and it is understood that blank areas are filled with zeros. Then
$\Lambda(u)$ defined by {\rm (\ref{LxLy})} and {\rm (\ref{L})} is a
factor of $\Gamma(u)$, and the roots of the polynomial
\be \label{Pslfint}
P_i(u) \,=\, {\Gamma(u) \over \Lambda(u)}
\ee
that remains on extracting this factor identify those parameter values
at which the untrimmed bisector suffers a self--intersection.
\end{propn}

\prf
The resultant of the two polynomials (\ref{RS2}) with respect to $\xi$
can be expressed as a Sylvester determinant \cite{uspensky48}, identical
to (\ref{G}), except that in the last three columns we have replaced
each occurrence of $a_0$ by 1, and each occurrence of $a_1$ or $a_2$ by
0. The motivation for this substitution is that the three lowest--order
coefficients of $\xi$ in (\ref{RS2}) are actually identical
\be \label{c012}
a_k(u) \,\equiv\, b_k(u) \quad ( = c_k(u), {\rm\ say} )
 \quad {\rm for\ } k=0,1,2 \,,
\ee
as can be seen by straightforward but laborious calculation from (\ref
{RStilde}), (\ref{RS}), and (\ref{RS2}). Because of this, $[\,c_0(u)\,]^3$
is a factor of the resultant of $R(u,\xi)$ and $S(u,\xi)$ with respect to
$\xi$, and expression (\ref{G}) is what remains on extracting this factor
(by performing elementary column operations on the last three columns of
the Sylvester determinant).

It may be verified that, apart from a multiplicative constant, $c_0(u)$
is actually identical to the polynomial $P_c(u)$, given by (\ref{Pcusp}),
which identifies {\it cusps\/} on the untrimmed bisector. This factor
must be discarded since we have already dealt with cusps, and are
concerned here only with {\it proper\/} self--intersections of the
untrimmed bisector (a cusp can be regarded as the limiting case $u_2
\to u_1$ of a self--intersection ${\bf b}(u_2)={\bf b}(u_1)$ with $u_2
\not=u_1$).

Now by construction --- see equations (\ref{LxLy}--\ref{L}) --- roots
of the polynomial $\Lambda(u)$ identify pairs of parameter values at
which {\it all three\/} of the quantities (\ref{Pi}) vanish. This
implies in turn that expressions (\ref{RStilde}) and (\ref{RS}) vanish
at these values, so $\Lambda(u)$ must be a factor of the resultant
of the two polynomials (\ref{RS2}) with respect to the variable $\xi$.

However, we know (by Remark~\ref{coincident}) that the roots of
$\Lambda(u)$ identify pairs of points on ${\bf r}(u)$ that have {\it
identical normal lines\/} --- ordinarily, such pairs do {\it not\/}
yield self--intersections of ${\bf b}(u)$. Thus, the factor $\Lambda(u)$
must be eliminated from $\Gamma(u)$. Unfortunately, there does not appear
to be any simple {\it a priori\/} means of achieving this by modification
of the Sylvester determinant.

Paired roots of the remaining polynomial (\ref{Pslfint}) then identify
distinct points ${\bf r}(u)$ and ${\bf r}(v)$ whose normal lines are
distinct and intersect in a point $(x_i,y_i)$ equidistant from ${\bf r}
(u)$, ${\bf r}(v)$, and ${\bf p}$ --- {\it i.e.}, a self--intersection
of ${\bf b}(u)$. \QED

There are interesting parallels between the above formulation and the
construction of the ``self--intersection'' polynomial for fixed--distance
offsets to a polynomial curve; see Section~4 of \cite{farouki90b}.

\begin{rmk}
{\rm
A word of caution regarding the polynomial $\Lambda(u)$ is in order.
We stated above that points ${\bf r}(u)$ and ${\bf r}(v)$ of the
given curve with {\it identical normal lines\/} do not ``ordinarily''
yield self--intersections of the untrimmed bisector. This is evident
if ${\bf r}(u)$ and ${\bf r}(v)$ are distinct, since for any ${\bf q}$
it is impossible that $|\,{\bf q}-{\bf p}\,|=|\,{\bf q}-{\bf r}(u)\,|=
|\,{\bf q}-{\bf r}(v)\,|$ with collinear points ${\bf r}(u)$, ${\bf r}
(v)$, and ${\bf q}$ when ${\bf r}(u)\not={\bf r}(v)$. Exceptionally,
however, we may have {\it coincident\/} points ${\bf r}(u)={\bf r}
(v)$ (where $u\not=v$) with {\it identical normal lines\/} --- these
correspond to points where the given curve ``touches'' itself, and
they yield tangential self--intersections of ${\bf b}(u)$ identified
by roots of $\Lambda(u)$.
However, these exceptional self-intersections should be ignored,
because a switch from a point on the true bisector to a point not 
on the true bisector 
must occur at a critical point whose circle $C_{\bf q}$ is tangent
to the curve ${\bf r}(u)$ at two or more distinct points,
not simply tangent to the curve at a double point 
(see the proof of Theorem~\ref{thm:trim2}).
}
\end{rmk}

After solving for the distinct real roots $u_1,u_2,\ldots$ of the
polynomial $P_i(u)$ given by (\ref{Pslfint}), we must establish the
appropriate correspondences between them by evaluating ${\bf b}(u)$
at each and noting the coincidences ${\bf b}(u_j)={\bf b}(u_k)$ with
$j\not=k$. Some of these may occur ``at infinity,'' {\it i.e.}, we
have $W_b(u_j)=W_b(u_k)=0$ and $(0,0)\not=(X_b(u_j),Y_b(u_j))\propto
(X_b(u_k),Y_b(u_k))$. Exceptionally, {\it three or more\/} of the
parameter values $u_1,u_2,\ldots$ may yield the same point on
${\bf b}(u)$, corresponding to a ``multiple'' self--intersection.

\figg{fig:trimprbla}{True bisectors of a point and a parabola.}{2.75in}

\begin{exmpl}
\label{exmpl:trimprbla}
{\rm
Consider the ``self--intersection'' polynomial (\ref{Pslfint}) for
the case of Example \ref{offsets}.1 --- {\it i.e.}, the bisector of the
point ${\bf p}=(\alpha,\beta)$ and the parabola ${\bf r}(u)=\{u,u^2\}$.
In this case $\Lambda(u)$ is just a constant, while $P_i(u)$ is given by
\be \label{Iprbla}
P_i(u) \,=\, u^4-2\beta u^2+\alpha^2+\beta^2-\beta \,.
\ee
Since (\ref{Iprbla}) is of biquadratic form, its roots can be written
down explicitly as
\be \label{uselfint}
u \,=\, \pm \, \sqrt{ \beta \pm \sqrt{\beta-\alpha^2} } \,.
\ee
All four roots are complex if $\alpha^2>\beta$, {\it i.e.}, the point
$(\alpha,\beta)$ lies ``outside'' the parabola. In this case, however,
it is necessary to trim the untrimmed bisector at the parameter values
(\ref{uinfprbla}) corresponding to its points at infinity.

If $(\alpha,\beta)$ lies ``inside'' the parabola $(\alpha^2<\beta)$,
we have either two or four real roots (and, correspondingly, one or two
real self--intersections) according to whether $\beta$ is less than or
greater than $\sqrt{\beta-\alpha^2}$. The latter condition gives
\be
\alpha^2+\beta^2-\beta \,=\, 0
\ee
as the locus defining the transition between regimes of one and two
self--intersections of the untrimmed bisector as the point $(\alpha,
\beta)$ is varied --- this locus is simply the {\it circle of curvature\/}
at the vertex $(0,0)$ of the parabola. (When $(\alpha,\beta)$ lies on
this circle, $u=0$ becomes a {\it double root\/} of (\ref{Iprbla}) and
the corresponding point $(x,y)=(0,\half)$ is an {\it extraordinary
point\/} of the untrimmed bisector; the remaining roots of (\ref
{Iprbla}) are then simply $u=\pm\sqrt{2\beta}$.)

Figure~\ref{fig:trimprbla} shows the true point/parabola bisectors that
correspond to the untrimmed bisectors seen in Figure~\ref{fig:parabola},
obtained by trimming the latter at the self--intersection points (\ref
{uselfint}) or points at infinity (\ref{uinfprbla}), as appropriate.
} \QED
\end{exmpl}

\figg{fig:trimellps}{True bisectors of a point and an ellipse.}
{5.5in}

\begin{exmpl}
\label{exmpl:trimellps}
{\rm
For the ellipse (\ref{pellipse}), the self--intersection polynomial
$P_i(u)$ is the product of two quartic factors given by
\ba \label{Iellps}
P_{i,1}(u) \! &=& \!
(\alpha^2+\beta^2+2(1-k^2)\alpha+1-2k^2)u^4 \nonumber \\
&& \!\!\!+\ 2(\alpha^2+\beta^2-1)u^2 \,+\, \alpha^2+\beta^2
-2(1-k^2)\alpha+1-2k^2 \,, \\
P_{i,2}(u) \! &=& \!
k(\alpha^2+\beta^2-1)u^4 \,+\, 4(1-k^2)\beta u^3 \nonumber \\
&& \!\!\!+\ 2k(\alpha^2+\beta^2-3+2k^2)u^2
\,+\, 4(1-k^2)\beta u \,+\, k(\alpha^2+\beta^2-1) \,. \nonumber
\ea
In order to clarify the meaning of these equations, note that (\ref
{pellipse}) has radii of curvature $r=k^2$ at the vertices $u=0$ and
$u=\pm\infty$, and $r=1/k$ at the vertices $u=+1$ and $u=-1$. Thus,
the corresponding circles of curvature are given by the equations
\ba \label{ecircles}
C_0(x,y) \! &=& \! (x-1+k^2)^2 + y^2 - k^4 \,=\, 0 \,, \nonumber \\
C_\infty(x,y) \! &=& \! (x+1-k^2)^2 + y^2 - k^4 \,=\, 0 \,, \nonumber \\
C_{+1}(x,y) \! &=& \! x^2 + (y^2-k+1/k)^2 - 1/k^2 \,=\, 0 \,, \nonumber \\
C_{-1}(x,y) \! &=& \! x^2 + (y^2+k-1/k)^2 - 1/k^2 \,=\, 0 \,.
\ea

Consider first $P_{i,1}(u)$, which we regard as a quadratic in $u^2$.
Making use of the polynomials defined in (\ref{ecircles}), we see
that $P_{i,1}(u)$ can be expressed as
\be \label{Iellipse2}
P_{i,1}(u) \,=\, C_\infty(\alpha,\beta)u^4
\,+\, 2(\alpha^2+\beta^2-1)u^2 \,+\, C_0(\alpha,\beta) \,.
\ee
Evidently there are no real values of $u^2$ (and hence of $u$)
satisfying $P_{i,1}(u)=0$ unless its discriminant, given by $\Delta_1
=(\alpha^2+\beta^2-1)^2-C_\infty(\alpha,\beta)C_0(\alpha,\beta)$, is
non--negative. From (\ref{ecircles}) it can be seen that $\Delta_1$
factors to yield
\be \label{discrim1}
\Delta_1 \,=\, 4(1-k^2)(k^2-k^2\alpha^2-\beta^2) \,,
\ee
and hence the condition $\Delta_1 > 0$ may be interpreted geometrically
as follows: either (i) the point $(\alpha,\beta)$ lies {\it inside\/}
the ellipse and the $x$--axis is the major axis of the ellipse ($k<1$),
or (ii) the point $(\alpha,\beta)$ lies {\it outside\/} the ellipse
and the $y$--axis is its major axis ($k>1$). Exceptionally, we have
$\Delta_1=0$ when $(\alpha,\beta)$ lies {\it on\/} the ellipse (note
that $k\not=1$ by assumption).

Case (i): $k<1$ and $k^2\alpha^2+\beta^2<k^2$ (major axis along
$x$--axis, point inside ellipse; note that $C_0(x,y)$ and $C_\infty(x,y)$
overlap if $k>1/\sqrt{2}$, and otherwise are disjoint). Since we are
interested only in non--negative roots $u^2$ to (\ref{Iellipse2}), we
may invoke Descartes' Law of Signs \cite[p.~121]{uspensky48} to deduce
the number of {\it positive\/} roots merely by inspecting its coefficients.
Specifically, when $k<1$ and $k^2\alpha^2+\beta^2<k^2$, the middle term
of (\ref{Iellipse2}) is necessarily negative, and the signature of the
coefficients of $P_{i,1}$ is given by:
\begin{itemize}
\item
$\{+-+\}$, indicating {\it two\/} positive values for $u^2\!$, when
$(\alpha,\beta)$ {\it lies outside both\/} $C_0(x,y)$ {\it and\/}
$C_\infty(x,y)\,$;
\item
or either $\{--+\}$ or $\{+--\}$, indicating {\it one\/} positive
value for $u^2\!$, when $(\alpha,\beta)$ {\it lies inside just one
of\/} $C_0(x,y)$ {\it and\/} $C_\infty(x,y)\,$;
\item
or (if $k>1/\sqrt{2}\,$) $\{---\}$, indicating {\it no\/} positive
values for $u^2\!$, when $(\alpha,\beta)$ {\it lies inside both\/}
$C_0(x,y)$ {\it and\/} $C_\infty(x,y)\,$.
\end{itemize}
Exceptionally, there is a real root $u^2$ to (\ref{Iellipse2}) at
zero or at infinity if $(\alpha,\beta)$ lies {\it on\/} $C_0(x,y)$ or
$C_\infty(x,y)$, respectively.

\smallskip

Case (ii): $k>1$ and $k^2\alpha^2+\beta^2>k^2$ (major axis along
$y$--axis, point outside ellipse --- here the circles of curvature
$C_0(x,y)$ and $C_\infty(x,y)$ intersect, the ellipse being contained
within their common area). In this case the middle term of (\ref
{Iellipse2}) is necessarily positive, and we again have three
possibilities for the number of positive real roots $u^2$:
\begin{itemize}
\item
signature $\{-+-\}$, and thus {\it two\/} positive roots, if
$(\alpha,\beta)$ {\it lies inside both\/} $C_0(x,y)$ {\it and\/}
$C_\infty(x,y)$ (yet outside the ellipse);
\item
or signature $\{-++\}$ or $\{++-\}$, and thus {\it one\/} positive
root, if $(\alpha,\beta)$ {\it lies inside just one of\/} $C_0(x,y)$
{\it and\/} $C_\infty(x,y)\,$;
\item
or signature $\{+++\}$, and thus {\it no\/} positive roots, if
$(\alpha,\beta)$ {\it lies outside both\/} $C_0(x,y)$ {\it and\/}
$C_\infty(x,y)\,$.
\end{itemize}
As before, there is a zero or infinite root $u^2$ to (\ref{Iellipse2})
when $(\alpha,\beta)$ lies {\it on\/} $C_0(x,y)$ or $C_\infty(x,y)$,
respectively.

\smallskip

(In the above we have glossed over the fact that Descartes' Law
actually indicates either {\it two\/} or {\it zero\/} roots in the case
of two sign variations. The latter possibility may be discounted in both
Case (i) and Case (ii) by noting that, in instances with two sign variations,
the extremum of $P_{i,1}(u^2)$ necessarily occurs at a positive value of
$u^2$ and is opposite in sign to $P_{i,1}(0)$ and $P_{i,1}(\infty)$.)

When $\Delta_1>0$, the parameter values that identify self--intersections
of the untrimmed bisector are given explicitly by the formula
\be
u \,=\, \pm \; \sqrt{ 1-\alpha^2-\beta^2 \,\pm\,
\sqrt{\Delta_1} \over C_\infty(\alpha,\beta) } \,,
\ee
which defines zero, one, or two pairs of real values of equal magnitude
and opposite sign, according to the criteria enumerated above.

Consider next $P_{i,2}(u)$. We claim that $P_{i,2}(u)$ has real roots
(other than zero and infinity) only when $P_{i,1}(u)$ has none, and
conversely. To see this, we find by ``completing the square'' that
$P_{i,2}(u)$ may be re--written in the form
\be \label{I2new}
P_{i,2}(u) \,=\, {
\left[\,k(\alpha^2+\beta^2-1)(u^2+1)+2(1-k^2)\beta u\,\right]^2
\,+\, \Delta_1 u^2 \over k(\alpha^2+\beta^2-1) } \,,
\ee
where $\Delta_1$ denotes the discriminant (\ref{discrim1}) introduced
previously. For fixed $\alpha$ and $\beta$, expression (\ref{I2new}) is
evidently of constant sign for all real $u$ when $\Delta_1>0$, and thus
it has no real roots. If $\Delta_1<0$, however, the two terms in the
numerator of (\ref{I2new}) are of opposite sign, and there are real
values of $u$ that will cause them to cancel precisely. These values
identify the self--intersections of the untrimmed bisector when (i)
the point $(\alpha,\beta)$ lies {\it outside\/} the ellipse and the
$x$--axis is the major axis ($k<1$), or (ii) the point $(\alpha,\beta)$
lies {\it inside\/} and the $y$--axis is its major axis ($k>1$).

When $\Delta_1<0$, the real roots of $P_{i,2}(u)$ can be computed using
the two quadratic equations obtained from (\ref{I2new}), namely
\be \label{I2new2}
k(\alpha^2+\beta^2-1)\,(u^2+1) \,+\,
[\,2(1-k^2)\beta \pm \textstyle{\sqrt{-\Delta_1}}\;]\,u \,=\, 0 \,,
\ee
whose real solutions are readily determined. However, it will be
found --- not unexpectedly --- that the number and locations of the
self--intersections identified by (\ref{I2new2}), in relation to the
ellipse and the untrimmed bisector, are actually identical to those
obtained through an appropriate rotation and scaling of the results
for $\Delta_1>0$. Thus, regardless of the orientation of the ellipse,
we may summarize as follows:

When $(\alpha,\beta)$ is located {\it inside\/} the ellipse, we will
have zero, one, or two self--intersections of the untrimmed bisector,
according to whether $(\alpha,\beta)$ lies inside both, just one, or
neither of the circles of curvature to the ellipse at the vertices on
its major axis, respectively (the first case being possible only when
the circles of curvature actually overlap). If $(\alpha,\beta)$ is {\it
outside\/} the ellipse, we have zero, one, or two self--intersections
when $(\alpha,\beta)$ lies inside neither, just one, or both of the
circles of curvature to the ellipse at the vertices on its minor axis,
respectively. In the latter case, the untrimmed bisector must also be
trimmed at its points at infinity, as identified by (\ref{uinfellps}).
} \QED
\end{exmpl}

The preceding discussion clarifies the behavior seen in Figure~\ref
{fig:ellipse} above; the trimmed bisectors corresponding to those
examples, computed by the methods described above, are shown in
Figure~\ref{fig:trimellps}.

\begin{rmk}
{\rm
Note that for the parabola, $P_i(u)$ is a factor of the polynomial
$X_b(u)$ in the parameterization (\ref{Bprbla}) of the untrimmed
bisector, while for the ellipse $P_{i,1}(u)$ and $P_{i,2}(u)$ are
factors of $Y_b (u)$ and $X_b(u)$ in the parametric form (\ref{Bellps}).
This means that the self--intersections of the untrimmed bisector lie on
{\it axes of symmetry\/} of the curves under consideration, namely, $x=0$
in the case of the parabola, and the major or minor axis of the ellipse
(according to whether $(\alpha,\beta)$ lies inside or outside of it).
}
\end{rmk}

The above observation is perhaps not unexpected, since we already know
that for any self--intersection ${\bf q}$ of the untrimmed bisector,
the circle $C_{\bf q}$ (see Definition~\ref{d:Cq}) must meet the given
curve tangentially in at least two points (Proposition~\ref{prop:cri}
and Definition~\ref{d:cri}) --- and the center of any inscribed circle
of a conic must lie on an axis of symmetry of that conic.

Beyond the realm of conic sections, the degree of the self--intersection
polynomial grows very rapidly. For the (polynomial) cubics, for example,
a number of examples suggest that $P_i(u)$ is of degree 26 in general,
while the extraneous factor $\Lambda(u)$ given by (\ref{L}) is of degree
6. Also, the polynomials (\ref{Pinf}) and (\ref{Pcusp}) that define the
``class'' and ``circular'' points of ${\bf r}(u)$ --- {\it i.e.}, those
points that induce points at infinity and cusps on the untrimmed bisector
${\bf b}(u)$ --- are of degree 4 and 8 in the case of cubics.

\begin{exmpl}
{\rm
A sobering example of how complicated the self--intersection polynomial
can grow is provided by the ``simple'' cubic ${\bf r}(u)=\{u,u^3\}$. In
this case the extraneous factor $\Lambda(u)$ is a quartic, while $P_i(u)$
is of degree 24:
\ba \label{P3slfint}
P_i(u)
\!\!\! &=& \!\!\!
    675\,u^{24} \,+\,
    1620\alpha\,u^{23} \,-\,
    3888\alpha^2\,u^{22} \,-\,
    8100\beta\,u^{21}
\nonumber \\
&+& \!\!\!
    540\,(6\alpha\beta-1)\,u^{20} \,+\,
    108\alpha\,(144\alpha\beta-1)\,u^{19}
\nonumber \\
&+& \!\!\!
    108\,(23\alpha^2+235\beta^2)\,u^{18} \,+\,
    108\,[\,3\alpha(11\alpha^2-101\beta^2)-5\beta\,]\,u^{17}
\nonumber \\
&-& \!\!\!
    54\,[\,144\alpha^2(\alpha^2+3\beta^2)-100\alpha\beta-7\,]\,u^{16}
\nonumber \\
&-& \!\!\!
    108\,[\,\beta(241\alpha^2+305\beta^2)+\alpha\,]\,u^{15}
\nonumber \\
&+& \!\!\!
    108\,[\,36\alpha\beta(5\alpha^2+17\beta^2)
      -31\alpha^2+101\beta^2\,]\,u^{14}
\nonumber \\
&+& \!\!\!
    36\,[\,432\alpha^2\beta(\alpha^2+\beta^2)
      +12\alpha(7\alpha^2-57\beta^2)-43\beta\,]\,u^{13}
\nonumber \\
&+& \!\!\!
    2\,[\,27(79\alpha^4+894\alpha^2\beta^2+303\beta^4)
      +5724\alpha\beta-86\,]\,u^{12}
\nonumber \\
&-& \!\!\!
    12\,[\,27\alpha(5\alpha^4+186\alpha^2\beta^2+181\beta^4)
      +18\beta(117\alpha^2+85\beta^2)-53\alpha\,]\,u^{11}
\nonumber \\
&-& \!\!\!
    36\,[\,108\alpha^2(\alpha^2+\beta^2)^2
      -24\alpha\beta(29\alpha^2+45\beta^2)-7\alpha^2+5\beta^2\,]\,u^{10}
\nonumber \\
&-& \!\!\!
    4\,[\,27\beta(253\alpha^4+234\alpha^2\beta^2-19\beta^4)
      +945\alpha(\alpha^2+\beta^2)+293\beta\,]\,u^9
\nonumber \\
&+& \!\!\!
    9\,[\,2664\alpha\beta(\alpha^2+\beta^2)^2
      +12(57\alpha^4+234\alpha^2\beta^2+65\beta^4)+520\alpha\beta+3\,]\,u^8
\nonumber \\
&-& \!\!\!
    108\,[\,\alpha(57\alpha^4+290\alpha^2\beta^2+233\beta^4)
      +\beta(57\alpha^2+41\beta^2)+2\alpha\,]\,u^7
\nonumber \\
&+& \!\!\!
    12\,[\,9(59\alpha^6+77\alpha^4\beta^2-23\alpha^2\beta^4-41\beta^6)
\nonumber \\
&& \qquad\qquad\qquad
  +\ 36\alpha\beta(13\alpha^2+17\beta^2)+63\alpha^2-89\beta^2\,]\,u^6
\nonumber \\
&-& \!\!\!
    36\,[\,99\alpha(\alpha^2+\beta^2)^3
      +3\beta(81\alpha^4+50\alpha^2\beta^2-31\beta^4)
\nonumber \\
&& \qquad\qquad\qquad
  +\ 2\alpha(21\alpha^2-19\beta^2)\,]\,u^5
\nonumber \\
&+& \!\!\!
    18\,[\,324\alpha\beta(\alpha^2+\beta^2)^2
      +105\alpha^4+210\alpha^2\beta^2+41\beta^4\,]\,u^4
\nonumber \\
&+& \!\!\!
    4\,[\,243\beta(\alpha^2+\beta^2)^3
      -54\alpha(7\alpha^4+22\alpha^2\beta^2+15\beta^4)-224\beta^3\,]\,u^3
\nonumber \\
&+& \!\!\!
    12\,[\,9(7\alpha^6-3\alpha^4\beta^2-27\alpha^2\beta^4-17\beta^6)
      +64\alpha\beta^3\,]\,u^2
\nonumber \\
&-& \!\!\!
    72\,(\alpha^2+\beta^2)\,[\,3\alpha(\alpha^2+\beta^2)^2-16\beta^3\,]\,u
\nonumber \\
&+& \!\!\! 27(\alpha^2+\beta^2)^4-256\beta^4 \,.
\ea
The points at infinity and cusps of the untrimmed bisector are identified
by the roots of the polynomials
\[
P_\infty(u) \,=\, 2u^3 - 3\alpha u^2 + \beta \,,
\]
\be
P_c(u) \,=\, 15u^7 - 27\alpha u^6 + 15\beta u^4 - u^3
 + 3\alpha u^2 - 3(\alpha^2+\beta^2)u + \beta \,.
\ee
Examples of the untrimmed bisector are shown in Figure~\ref{fig:cubic}.
} \QED
\end{exmpl}

\figg{fig:cubic}{Untrimmed bisectors of a point and ${\bf r}(u)=\{u,u^3\}$.}
{2.75in}

The reader may care to compare equation (\ref{P3slfint}) with the
self--intersection polynomial for the constant--distance offset to
${\bf r}(u)=\{u,u^3\}$ quoted in \cite[equation~(100)]{farouki90b}. As
a further example, we find for the ``superbola'' ${\bf r}(u)=\{u,u^4\}$
that $P_i(u)$ factors into two components, of degree 10 and 48. Figure~\ref
{fig:sprbla} shows some examples of the untrimmed bisector in this case.

Note that the number of real, affine self--intersections of the untrimmed bisector
cannot exceed one--half the degree of the polynomial $P_i(u)$, since by
definition two distinct parameter values must be associated with each. In
most cases, however, there are far fewer than this, since some roots of
$P_i(u)$ correspond to self--intersections of conjugate branches of the
complex locus of ${\bf b}(u)$, and others to self--intersections ``at
infinity.'' Nevertheless, $P_i(u)$ is irreducible in general --- there
is no simpler representation of just the real, affine self--intersection
parameter values.

\figg{fig:sprbla}{Untrimmed bisectors of a point and ${\bf r}(u)=\{u,u^4\}$.}
{2.75in}

\section{Concluding remarks}
\label{conclusion}

The locus of a variable point ${\bf q}$, which maintains equal distances with
respect to a fixed point ${\bf p}$ and a plane polynomial or rational
curve ${\bf r}(u)$, is amenable to an exact and relatively simple ({\it
i.e.}, rational) parametric representation. Such point/curve bisectors
are thus more compatible with existing computer--aided design systems
than other elementary ``procedurally--defined'' curves --- notably the
fixed--distance {\it offsets\/} to polynomial or rational curves \cite
{farouki90a,farouki90b}, which have no rational parameterizations in
general.

The principal difficulty in computing point/curve bisectors undoubtedly
lies in the  ``trimming'' procedure, {\it i.e.}, identifying the parameter
values which delimit those subsegments of the untrimmed bisector --- see
equations (\ref{pbsctr}) and (\ref{rbsctr}) above --- that constitute the
``true'' bisector. Nevertheless, as shown in Section~\ref{trimming}, it is
possible to attack this problem in an algorithmic manner, and for simple
curves ({\it e.g.}, conics) closed--form analytic expressions for the
trim points may even be written down {\it a priori\/} (see Examples~\ref
{exmpl:trimprbla} and \ref{exmpl:trimellps}).

Once the ordered set of trim points $u_1,u_2,\ldots$ is known, the
bisector can be represented by a single rational expression ${\bf b}(u)$,
restricted to a sequence of disjoint domains $u \in [\,u_1,u_2\,]$, $u
\in [\,u_3,u_4\,]$, $\ldots\,$, or by a set of rational B\'ezier arcs
with matched end points --- all actually derived from the same parent
curve, and mapped individually to the standard domain $u \in [\,0,1\,]$.

The problem of curve/curve bisectors is far more formidable than that
of the point/curve bisectors discussed here, even if the given curves
have simple parameterizations. We hope to address this problem in future
studies.

\begin{thebibliography}{99}

\bibitem{bookstein79}
F.~L.~Bookstein (1979), The line--skeleton, \CGIP{\bf 11}, 123--137.

%\bibitem{bruce84}
%J.~W.~Bruce and P.~J.~Giblin (1984), {\it Curves and Singularities},
%Cambridge University Press.

\bibitem{buck78}
R.~C.~Buck (1978) {\it Advanced Calculus}, Third edition, McGraw--Hill,
New York, 362--367.

%\bibitem{coolidge59}
%J.~L.~Coolidge (1959), {\it A Treatise on Algebraic Plane Curves},
%Dover Publications, New York.

\bibitem{coxeter69}
H.~S.~M.~Coxeter (1969), {\it Introduction to Geometry}, Wiley,
New York.

%\bibitem{docarmo76}
%M.~P.~do~Carmo (1976), {\it Differential Geometry of Curves and
%Surfaces}, Prentice--Hall, Englewood Cliffs, N.J.

\bibitem{DH90}
D.~Dutta and C.~M.~Hoffmann (1990), A geometric investigation of the
skeleton of CSG objects, {\it Technical Report UM--MEAM--90--02},
Dept.\ of Mechanical Engineering, The University of Michigan.

\bibitem{eisenhart60}
L.~P.~Eisenhart (1960), {\it Coordinate Geometry}, Dover Publications,
New York (reprint).

\bibitem{farouki90a}
R.~T.~Farouki and C.~A.~Neff (1990), Analytic properties of plane
offset curves, \CAGD{\bf 7}, 83--99.

\bibitem{farouki90b}
R.~T.~Farouki and C.~A.~Neff (1990), Algebraic properties of plane
offset curves, \CAGD{\bf 7}, 100--127.

\bibitem{farouki90c}
R.~T.~Farouki and T.~Sakkalis (1990), Pythagorean hodographs,
\IBMJRD{\bf 34}, 736--752.

\bibitem{farouki91a}
R.~T.~Farouki (1991), Pythagorean--hodograph curves in practical use,
{\it Geometry Processing for Design and Manufacturing\/} (R.~E.~Barnhill,
ed.), SIAM, Philadelphia, to appear.

\bibitem{farouki91b}
R.~T.~Farouki (1991), Watch your (parametric) speed! {\it The
Mathematics of Surfaces IV\/} (A.~Bowyer, ed.), Oxford University
Press, to appear.

\bibitem{Held91}
M.~Held (1991), {\it On the Computational Geometry of Pocket Machining},
Springer--Verlag, Berlin.

\bibitem{H52}
D.~Hilbert and S.~Cohn--Vossen (1952), {\it Geometry and the Imagination},
Chelsea, New York.

\bibitem{H90}
C.~M.~Hoffmann (1990), A dimensionality paradigm for surface interrogations,
\CAGD{\bf 7}, 517--532.

\bibitem{HV91}
C.~M.~Hoffmann and P.~Vermeer (1991), Eliminating extraneous solutions in
curve and surface operations, \IJCGA{\bf 1}, 47--66.

%\bibitem{hoschek85}
%J.~Hoschek (1985), Offset curves in the plane, \CAD{\bf 17}, 77--82.

\bibitem{hoschek88}
J.~Hoschek (1988), Spline approximation of offset curves, \CAGD{\bf 5},
33--40.

\bibitem{kelly79}
P.~J.~Kelly and M.~L.~Weiss (1979), {\it Geometry and Convexity}, Wiley,
New York.

\bibitem{klass83}
R.~Klass (1983), An offset spline approximation for plane cubic splines,
\CAD{\bf 15}, 297--299.

\bibitem{kreyszig59}
E.~Kreyszig (1959), {\it Differential Geometry}, University of Toronto
Press.

\bibitem{lee82}
D.~T.~Lee (1982), Medial axis transformation of a planar shape,
\IEEETPAMI{PAMI--{\bf 4}}, 363--369.

\bibitem{nackman91}
L.~R.~Nackman and V.~Srinivasan (1991), Bisectors of linearly separable
sets, \DCG{\bf 6}, 263--275.

\bibitem{pham88}
B.~Pham (1988), Offset approximation of uniform B-splines, \CAD{\bf 20},
471--474.

%\bibitem{preparata85}
%F.~P.~Preparata and M.~I.~Shamos (1985), {\it Computational Geometry},
%Springer, New York.

%\bibitem{salmon79}
%G.~Salmon (1879), {\it A Treatise on the Higher Plane Curves}, Chelsea,
%New York (reprint).

\bibitem{sederberg84}
T.~W.~Sederberg (1984), Degenerate parametric curves, \CAGD{\bf 1},
301--307.

\bibitem{sederberg86}
T.~W.~Sederberg (1986), Improperly parameterized rational curves,
\CAGD{\bf 3}, 67--75.

%\bibitem{stoker69}
%J.~J.~Stoker (1969), {\it Differential Geometry}, Wiley, New York.

\bibitem{tiller84}
W.~Tiller and E.~G.~Hanson (1984), Offsets of two-dimensional profiles,
\IEEECGA{\bf 4} (Sept.), 36--46.

\bibitem{uspensky48}
J.~V.~Uspensky (1948), {\it Theory of Equations}, McGraw--Hill,
New York.

\bibitem{W50}
R. J. Walker (1950), {\it Algebraic Curves}, Springer--Verlag, New York.

%\bibitem{winger62}
%R.~M.~Winger (1962), {\it An Introduction to Projective Geometry},
%Dover Publications, New York.

\bibitem{yap87}
C.--K.~Yap (1987), An ${\rm O}(n\log n)$ algorithm for the Voronoi
diagram of a set of simple curve segments, \DCG{\bf 2}, 365--393.

\bibitem{yap89}
C.--K.~Yap and H.~Alt (1989), Motion planning in the {\it
CL\/}--Environment, in {\it Lecture notes in computer science 382\/}
(F.~Dehne, J.--R.~Sack, and N.~Santoro, eds.), Springer, New York,
373--380 (Proceedings of the Workshop on Algorithms and Data
Structures WADS~'89, Ottawa, Canada, August 17--19, 1989).

\end{thebibliography}

\end{document}

