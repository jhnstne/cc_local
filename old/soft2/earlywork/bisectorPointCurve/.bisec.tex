\documentstyle[12pt]{article} 
%
\input{macros}
%
% Page format
%
\DoubleSpace
\setlength{\oddsidemargin}{0pt}
\setlength{\evensidemargin}{0pt}
\setlength{\headsep}{0pt}
\setlength{\topmargin}{0pt}
\setlength{\textheight}{8.75in}
\setlength{\textwidth}{6.5in}
%
%
\title{The bisector of two plane algebraic curves}
\begin{document}

{\bf Abstract.}
We solve the problem of finding the bisector of two plane algebraic curves.
These bisectors are important to the construction of Voronoi diagrams of 
algebraic curves.
The key technique is to build a superset of the bisector and then pare it down.

\section{Previous work}

Yap \cite{Y85} has considered some simple bisectors, as outlined in 
Table~\ref{yap-table} gleaned from Yap \cite[p. 371--72]{Y87}.

\begin{table}
\label{yap-table}
\centering
\begin{tabular}{|l|c|c|}
\hline
			& bisector \\ \hline
two points 		& line \\ \hline
line and point (on line)& line\footnotemark[6] \\ \hline
line and point (not on line) &	parabola \\ \hline
two lines 		& line\footnotemark[7] \\ \hline
two line segments	& piecewise linear/parabolic curve (up to seven 
			  pieces) \\ \hline
two circles (disjoint interiors) & 1 branch of hyperbola\footnotemark[8]
		\\ \hline
two circles (one inside other) & ellipse\footnotemark[9]	\\ \hline
two circles (intersecting) & ellipse + 1 branch of hyperbola\footnotemark[10]
	\\ \hline
circle and point (inside) & ellipse \\
{[degenerate form of two circles]} &  \\ \hline
circle and point (outside) & 1 branch of hyperbola \\
{[degenerate form of two circles]} & \\ \hline
circle and point (on)	  & ray\footnotemark[11] \\ \hline
circle and line (not intersecting)
			& parabola\footnotemark[12]	\\ \hline
circle and line (intersecting)     & 2 parabolas\footnotemark[13] \\ \hline 
\end{tabular}
\caption{Bisectors of points and curves in 2D}

\end{table}
\footnotetext[6]{Through point and normal to line.} 
\footnotetext[7]{Through intersection (possibly at infinity)
			  and bisecting the angle between the lines.}
\footnotetext[8]{The branch loops around
	the smaller circle.  In the degenerate case of equal radii, the 
	hyperbolic branch becomes a line.}
\footnotetext[9]{The foci of the ellipse are the two centers of the circles.}
\footnotetext[10]{Both ellipse and
	hyperbola pass through the two intersections of the circles.
	(Foci of ellipse are again the centers of the circles??)
	Degenerate cases arise when circles are tangent:
	ellipse becomes line segment between two centers when circles 
	are almost disjoint,
	hyperbola becomes ray away from both centers when one circle 
	is almost inside other.}
\footnotetext[11]{From center of circle through point.}
\footnotetext[12]{This bisector is closely related to that of a
		point and a line. In particular, if C is a circle of 
		center $p_{C}$ and radius r and
		L is a line not intersecting C, then bisector(C,L) is 
		equivalent to
		bisector($p_{C}$, line parallel to L but distance r further 
		away from C).}
\footnotetext[13]{Both parabolas pass through 
	the two intersections of circle and line.
	Directrices of two parabolas are the two lines parallel to given 
	line and at 
	a distance of the circle's radius away.
	Degenerate case arises when line is tangent to circle:
	one parabola becomes a ray from the point of tangency away from 
	the circle.}


DISCUSS MEDIAL AXIS LITERATURE.

\section{A superset of the point-curve bisector}
\label{sec-sup1}

% Picture of the bisector of a point and an algebraic curve 
% [with P,C,$\alpha$,Q marked]

Let $B(C,\alpha)$ be the bisector of the plane algebraic curve C and the 
point $\alpha \not \in C$.
By definition, $P \in B(C,\alpha)$ if and only if 
$\mbox{dist}(P,C) = \mbox{dist}(P,\alpha)$.
Since the algebraic curve C is closed in the Euclidean topology of the plane,
dist(P,C) := inf \{dist(P,c) : $c \in C$\} = min \{dist(P,c) : $c \in C$\}.
% If S is closed, then the infimum will be attained by some point,
Therefore, 
\[ P \in B(C,\alpha) \Leftrightarrow \exists\ Q \in C \mbox{ s.t. } 
\mbox{dist}(P,C) = \mbox{dist}(P,Q) = \mbox{dist}(P,\alpha) \]
The goal is to find the locus of all P that satisfy these properties, and 
the way to do this is to translate the conditions on P into a system 
of equations.
This is simple for the criteria $Q \in C$ and 
$\mbox{dist}(P,Q) = \mbox{dist}(P,\alpha)$:
%
\begin{eqnarray}
	f(q_{1},q_{2})=0 \\
%
	(q_{1} - p_{1})^{2} + (q_{2} - p_{2})^{2} - (\alpha_{1} - p_{1})^{2} - 
	(\alpha_{2} - p_{2})^{2} = 0
\end{eqnarray}
%
where $P = (p_{1},p_{2})$, $Q = (q_{1},q_{2})$, 
$\alpha = (\alpha_{1},\alpha_{2})$,
and C is defined by $f(x,y) = 0$.
However, it is hard 
% perhaps impossible except in the theory of the reals, 
% where inequalities are allowed?
to express the minimal condition dist(P,Q) = min \{dist(P,c) : $c \in C$\} 
analytically.
Therefore, it is best to replace this condition, and the following lemma 
offers the means.

\begin{lemma}
\label{lem1}
If Q is the closest point on the curve C to the point P,
then Q is a singularity
or \seg{PQ} is perpendicular to the tangent to C at Q (Figure 3.1).
% Could make stronger statement: ` If ..., then Q is a cusp or \seg{PQ} 
% is perpendicular to each of the tangents to C at Q.'?
\end{lemma}
\Heading{Proof:}
Let Q be the closest point on C to P.
Thus, $B_{\mbox{dist(P,Q)}}(P) \cap C = \emptyset$, 
where $B_{\mbox{dist(P,Q)}}(P)$
is the open ball of radius dist(P,Q) about P.
Assume without loss of generality that Q is a nonsingular point.
If \seg{PQ} is not perpendicular to Q's tangent, then Q's tangent must 
intersect $B_{\mbox{dist(P,Q)}}(P)$.
But, since the tangent is continuous in the neighbourhood of the nonsingular 
point Q,
this implies the contradiction that the curve in the neighbourhood of Q 
intersects $B_{\mbox{dist(P,Q)}}(P)$.
Therefore, \seg{PQ} must be perpendicular to Q's tangent.
\QED

\vspace{.25in}
\noindent This characterization can certainly be translated into an equation.

\begin{corollary}
\label{cor}
If dist(P,Q) = dist(P,C), then 
\begin{equation}
(P-Q) \cdot (f_{y}(Q),-f_{x}(Q)) = 0
\end{equation}
where $f_{x}$ (resp., $f_{y}$) is the derivative of $f$ with respect to $x$ 
(resp., $y$).
\end{corollary}
\Heading{Proof:}
If Q is a singularity, then $f_{x}(Q) = 0 = f_{y}(Q)$ \cite{W50}.
% [p. 54]
Suppose that Q is a nonsingular point.
The tangent of $f(x,y)=0$ at Q is parallel to the line 
$f_{x}(Q)x + f_{y}(Q)y = 0$ \cite[p. 55]{W50}.
Since this line passes through the origin and $(f_{y}(Q),-f_{x}(Q))$, 
the tangent at Q is parallel to the vector $(f_{y}(Q),-f_{x}(Q))$.
Thus, \seg{PQ} is perpendicular to the tangent to C at Q if and only if
$(P-Q) \cdot (f_{y}(Q),-f_{x}(Q)) = 0$.
\QED

\begin{example}
The condition of Lemma~\ref{lem1} is not sufficient, as shown by Figure 3.2 where
$Q'$ satisfies the conditions but is not the closest point of $C$ to $P'$.
However, it is difficult to find a condition that discards $P'$ from the bisector set.
As a result, we shall leave $P'$ in the preliminary set, and only discard it later.
\end{example}

% Example: Figure 3.2 shows a point P' that satisfies the conditions 
% but does not lie on the bisector.

\vspace{.25in}
\noindent Since equation (3) is a necessary but not sufficient condition for 
dist(P,Q) = dist(P,C),
the simultaneous solution set of equations (1)-(3) will be a superset of 
the bisector.
In particular, since these three equations in four variables can be reduced 
to one 
equation in two variables, the solution set of 
equations (1)-(3) is a plane algebraic curve S that contains the bisector.
The next section shows how to strip away the points of S that do not lie on 
the bisector.

The complexity of the computation of the point-curve bisector superset for a 
curve of degree $d$ is the complexity of solving a system of two equations 
of degree $d$ and an equation of degree 2.
Using resultants to reduce the system to a single equation in one variable,
an upperbound is the complexity of solving a univariate equation of 
degree $2d^{2}$.

\section{Collapsing the superset of the point-curve bisector}
\label{sec-col1}

Idea: the superset consists of several segments, each separated by a singularity.
Each segment is either entirely within the bisector or entirely without.
Thus, it only requires checking one point of a segment to determine the status
of the entire segment.

The first observation is that the bisector of $\alpha$ and $C$ will 
not intersect the curve $C$.
Therefore, a point $x \neq \alpha$ of intersection of the superset S and C is 
clearly not on the bisector.
These points of intersection will guide us to the entire extraneous set 
$S \setminus B(C,\alpha)$, since the segments that may potentially have 
to be removed from S are those between two of these intersections.
The following lemma establishes that the segment between two intersections 
is in the bisector if and only if an arbitrary point of this segment is in 
the bisector.

We shall use these observations to winnow S down to the bisector as follows.
The intersections of the curves C and S are computed.
% This step requires resultants [analyze its complexity].
% Discuss methods for curve-curve intersection.
The superset curve S is split into segments between intersections with C, by
sorting the intersections along S \cite{jj}.
The bisector is the union of some of these segments.
A single point of each segment is tested for membership in the bisector
and the appropriate segments are discarded.
In order to test if a point P is in the bisector,
we test if the point is equidistant from $\alpha$ and C, as follows:
find all points Q such that \seg{PQ} is perpendicular to Q's tangent 
(by solving an equation similar to equation (3)),
find the closest one to P, and check if its distance from P is equal to 
dist(P,$\alpha$).

\section{Bisector of point and algebraic curve SEGMENT}

\section{A superset of the bisector of two plane algebraic curves}
\label{sec-sup2}

% Picture of bisector of two curves

The curve-curve bisector computation is analogous to that of the point-curve bisector.

Let $C_{1}$ and $C_{2}$ be two plane algebraic curves defined by $f(x,y)=0$ 
and $g(x,y)=0$, respectively, and consider their bisector $B(C_{1},C_{2})$.
$P \in B(C_{1},C_{2})$ if and only if there exists Q and R such that
\begin{enumerate}
\item
	$Q$ lies on curve $C_{1}$
\item
	$R$ lies on curve $C_{2}$
\item
	dist(P,Q) = dist(P,R)
\item
	dist(P,Q) = dist(P,$C_{1}$) = min \{ dist(P,s) : $s \in C_{1}$ \}
\item
	dist(P,R) = dist (P,$C_{2}$) = min \{ dist(P,s) : $s \in C_{2}$ \}
\end{enumerate} 
Once again, we want to replace the conditions (4) and (5), 
which involve minima, with more tractable conditions.

\begin{enumerate}
\item
	$f(q_{1},q_{2})=0$
\item
	$g(r_{1},r_{2})=0$
\item
	$(q_{1} - p_{1})^{2} + (q_{2} - p_{2})^{2} - (r_{1} - p_{1})^{2} - 
	(r_{2} - p_{2})^{2} = 0$
\item
	$(p_{2} - q_{2}) f_{x}(q_{1},q_{2}) - 
	(p_{1} - q_{1}) f_{y}(q_{1},q_{2}) = 0$
\item
	$(p_{2} - r_{2}) g_{x}(r_{1},r_{2}) - 
	(p_{1} - r_{1}) g_{y}(r_{1},r_{2}) = 0$
\end{enumerate}

Since these conditions are weaker, the resulting solution set will be a 
superset of the bisector.

\section{Bisector of two algebraic curve SEGMENTS}

\section{Solution of system of equations}
\label{sec-system-soln}

Must show that the equations are independent (and thus can be used to eliminate two
variables).
Test whether the matrix of equations is of full rank.
%
The multivariate resultant \cite{} can be used to solve the system of equations (1)-(5).
Unfortunately, with this method, the system of five quadratic equations in six 
variables is reduced to a single equation in two variables of degree $2^{5} = 32$.
We must show that `the solution set is one-dimensional, not two-dimensional'(?).
An alternative is to retain the minimal conditions and express them using inequalities.
The resulting system of equations and inequalities could be solved with the theory of
reals \cite{}, although this requires double-exponential time [or is it single-exponential
since Canny?].

\section{Future directions}

It is interesting to consider the extension of these ideas to several curves.
One should not consider the locus of points equidistant from three or more plane algebraic 
curves, since this is in general a set of points rather than a curve.
The more interesting direction is the one leading to Voronoi diagrams, namely
the locus of points equidistant from at least two curves.
This involves the merging of the bisectors of pairs of curves.
% We are investigating this problem.
A related problem is the computation of the bisector of two segments of algebraic curves,
rather than the bisector of entire curves.
% since it is computed by refining the bisectors of all pairs from the sets IS THIS SO?
A challenging extension is to three dimensions, such as the bisector of two space 
curves or the bisector of two surfaces, since the bisector becomes a surface.

Find medial axis (as opposed to bisector) of curve.
Find medial axis of simple curvegon.

\section{Conclusions}

Originally, it looks as if you need to take min (which is very difficult).
But this is only true if you insist on finding the bisector directly with
necessary and sufficient conditions.
One can also find necessary but not sufficient conditions 
that generate a superset of the bisector, and then find ways of stripping away
the chaff from this superset.

In three dimensions, we also know that the bisector of two skew lines
is a right hyperbolic paraboloid (Johnstone and Shene 1991).
Notice that bisectors are related to distance representations.
Bisectors are a special case: d1 = d2.
Offset surfaces are another special case: d1 = k.
The search for bisectors can be related to the search for 
geometric representations, as follows.
One way to find a geometric representation 
of a complicated curve C is to look for two simple curves whose bisector is C: 
that is, complex = bisector(simple,simple).
The evidence (Yap's upperbounds on bisector degree) is that 
this is a realistic approach, since the bisector of two curves 
is more complex than the original curves.

\section{Appendix}

\begin{lemma}
Point-curve bisectors and curve-curve bisectors are continuous curves
(although they might be disconnected).
[This is important to establish the legality of throwing out an entire segment even though
only one representative point was tested.]
\end{lemma}

\begin{thebibliography}{The longest}

\bibitem{PS85} 
F. Preparata and M. Shamos,
{\em Computational Geometry: An Introduction}, 
Springer Verlag, New York, 1985.

\bibitem{W50}
R. J. Walker,
{\em Algebraic Curves},
Springer Verlag, New York, 1950.

\bibitem{Y85}
C. K. Yap, 
{\em An O(n log n) algorithm for the Voronoi diagram of a set of simple curve segments},
Technical Report No. 161, Courant Institute of Mathematical Sciences, New York University,
NY, 1985.
% May, 1985

\end{thebibliography}

\end{document}
