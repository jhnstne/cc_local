Rida, 

	Here it is finally.   I think we should congratulate ourselves on a 
very elegant paper.  There are some beautiful results and it all ties 
together into an impressive package.  Please pardon the delay: it was tough 
slogging to reread everything (plus the new material).

After wading through my comments scrawled over the paper and 
elsewhere, they reduce to the following list, starting with the more 
interesting points.  I will enclose the paper, with the minor changes 
incorporated, in a subsequent mail.

I have checked most of the formulas through by hand,
but didn't get through some of the longer ones.
I have not checked the following formulas: 
(10), (39), (43), (50) and (51), (58) and (59), (60) and (61), (63) and (64),
(82), (83), (86), (87)-(93).

It would help to have a review of the sections at the end of p. 1.
I will write one and send it to you tomorrow.



p. 21, Lemma 2.1: Should the restriction on the curvature be relaxed
	to `the lowest-order non-vanishing derivative of $\kappa(u)$ is odd'
	rather than `\kappa '(u) \neq 0$?
	This would also affect Lemma 2.2.
	Is the concentration on the cusps `\kappa '(u) \neq 0$'
	due to the remark after 2.2?

p. 35: I have only convinced myself that (75) is a *superset* of the
	coincident normal lines.
	For example, suppose Xi(4,5) = Wi(4,5) = 0
	and Yi(4,3) = Wi(4,3) = 0, but Yi(4,5) \neq 0.
	Then 4 would be a root of (75) but r(4) would not necessarily have
	coincident normals?
	If this is true, it would affect the second paragraph on p. 38.

p. 38, addition to end of Remark 3.5: 
	`However, these exceptional self-intersections should be ignored,
	because a switch from a point on the true bisector to a point not 
	on the true bisector must occur at a critical point whose circle 
	$C_{\bf q}$ is tangent to the curve ${\bf r}(u)$ at two or more 
	distinct points, not simply tangent to the curve at a double point 
	(see the proof of Theorem~\ref{thm:trim2}).'
	
	Note that I could change the definition of critical point
	accordingly (2 or more *distinct* points), 
	but this has repercussions elsewhere that
	I don't want to worry about for now.

p. 38, bottom of page: 
	I'm not sure why `the appropriate correspondences between the 
	distinct real roots *must* be established'.
	The algorithm simply checks each segment [b(u1),b(u2)],
	[b(u2),b(u3)], etc. for validity (last step of algorithm on p. 34),
	so it doesn't seem necessary to pair self-intersections.

% **************************************

The following comments/changes are minor:

p. 3, equation (6): I get a denominator of $W^2$ rather than 
	GCD(W,W') when I take GCD((x/w)',(y/w)').  I'm sure I'm missing
	something.

p. 8: last line of example 1.2: `Moreover, the closest point of 
${\bf r}(u)$ to ${\bf p}$ can be the point ${\bf r}(0)$.'

p. 14, Figure 4: I think it would help to add the marking `d(u)'
to the side pq of the triangle (complementing this mark on the qr(u) side).

p. 14, formula (33): right hand side $\lambda$ should be $|\lambda|$
			(i.e., adding absolute value)

p. 14: directly after (34): `by using the law of cosines.'
	That is, `... we have <(34)> by using the law of cosines.  Since ...'

p. 16 and 22:
	the degrees of formula (36), (58), and (59) seem to be of degree 
	2n-1, 4n-3, and 7n-3, rather than 2n-2, 4n-4, and 7n-6 (as specified)
	respectively.
	I assume that this is because higher degree terms cancel,
	and not a miscalculation.

p. 17: end of Example 2.1: `regardless of the location *of* $(\alpha,\beta)$'

p. 18: end of first paragraph: `The cusps of the untrimmed bisector become
	important for trimming in later sections.'

P. 18, formula (45): I think I mentioned this before, but I forget the answer:
	the expression for curvature that I know about is 
	$\frac{|{\bf r}'(u) \times {\bf r}''(u)|}{|{\bf r}'(u)|^3}$,
	(i.e., $z$ is not involved).
	e.g., Millman and Parker, p. 46, or Kreyszig, p. 35.
	Where does the form in (45) come from?

(51): although (51) is one of the formulae I have not checked (see below),
	the use of 1+2\kappa d looks suspicious, since 1+\kappa d is the norm.
	Perhaps it bears checking.

p. 21, just after (54): `in the direction of $t(u) + n(u) tan \psi(u)$'.
	Note the addition replacing the subtraction.

p. 21, (56): I'm not sure that the second equation adds anything,
	and actually might confuse the issue.
	It's just a substitution from (34) and the important
	thing is the first equality.  If you agree, you could remove it,
	otherwise it's fine for it to remain.

p. 22, sentence before (57): add a clause:
	`Using (35) and (45), the local critical curvature (56) 
		is attained ... etc.'

p. 26, switch order of (2a) and (2b) in Definition 3.2
	(this adds to the clarity)

p. 29: addition just before Defn 3.3: 
	`We are also interested in points where the position of ${\bf p}$
	 relative to the circle of curvature can change.'

p. 30: just before `Note that': I added `(see Remark 3.2 below).'
	Also, since the comment about center and radius of curvature is
	only used at the end of Remark 3.2, I thought it more appropriate
	to put it there (I found that it was confusing encountering this
	comment in its present context).
	I rearranged the discussion there to make it fit.
	By the way, within this comment, I cannot figure out how to establish
	that the curvature is positive when the normal $n(u)$ points away from
	the center of curvature.

p. 34: addition to step 2.

p. 39: `a just` -> `just a'

p. 44: I added parentheses to aid the understanding of the end of Remark 3.6.



I am starting to think about which parts to condense into an extended
abstract for conference submission.  We can talk about that at Tempe.

					John