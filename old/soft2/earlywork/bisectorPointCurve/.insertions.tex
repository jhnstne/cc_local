\documentstyle[12pt]{article}
\begin{document}

\newcommand{\single}{\def\baselinestretch{1.0}\large\normalsize}
\newcommand{\double}{\def\baselinestretch{1.5}\large\normalsize}
\newcommand{\triple}{\def\baselinestretch{2.2}\large\normalsize}

\newcommand{\rd}{{\rm d}}
\newcommand{\re}{{\rm e}}
\newcommand{\ri}{{\rm i}}
\newcommand{\sgn}{{\rm sign}}
\newcommand{\half}{\textstyle{1 \over 2}\displaystyle}
\newcommand{\quarter}{\textstyle{1 \over 4}\displaystyle}
\newcommand{\dist}{{\rm distance}}
\newcommand{\cross}{\!\times\!}
\newcommand{\dotpr}{\!\cdot\!}
\newcommand{\vhat}{\hat {\bf v}}
\newcommand{\what}{\hat {\bf w}}
\newcommand{\zhat}{\hat {\bf z}}
\newcommand{\ldash}{\vrule height 3pt width 0.35in depth -2.5pt}
\newcommand{\be}{\begin{equation}}
\newcommand{\ee}{\end{equation}}
\newcommand{\ba}{\begin{eqnarray}}
\newcommand{\ea}{\end{eqnarray}}
\newcommand{\seg}[1]{\mbox{$\overline{#1}$}}

\newtheorem{dfn}{Definition}[section]
\newtheorem{rmk}{Remark}[section]
\newtheorem{lma}{Lemma}[section]
\newtheorem{propn}{Proposition}[section]
\newtheorem{exmpl}{Example}[section]
\newtheorem{conjec}{Conjecture}[section]
\newtheorem{claim}{Claim}[section]
\newtheorem{notn}{Notation}[section]
\newtheorem{thm}{Theorem}[section]
\newtheorem{crlry}{Corollary}[section]

\newcommand{\prf}{\noindent{{\bf Proof} :\ }}
\newcommand{\QED}{\vrule height 1.4ex width 1.0ex depth -.1ex\ \medskip}

\newcommand{\ACMTOG}{{\sl ACM Trans.\ Graph.\ }}
\newcommand{\AMM}{{\sl Amer.\ Math.\ Monthly\ }}
\newcommand{\BIT}{{\sl BIT\ }}
\newcommand{\CACM}{{\sl Commun.\ ACM\ }}
\newcommand{\CAD}{{\sl Comput.\ Aided Design\ }}
\newcommand{\CAGD}{{\sl Comput.\ Aided Geom.\ Design\ }}
\newcommand{\CJ}{{\sl Computer\ J.\ }}
\newcommand{\DCG}{{\sl Discrete\ Comput.\ Geom.\ }}
\newcommand{\IBMJRD}{{\sl IBM\ J.\ Res.\ Develop.\ }}
\newcommand{\IEEECGA}{{\sl IEEE Comput.\ Graph.\ Applic.\ }}
\newcommand{\IEEETPAMI}{{\sl IEEE Trans.\ Pattern Anal.\ Machine Intell.\ }}
\newcommand{\JACM}{{\sl J.\ Assoc.\ Comput.\ Mach.\ }}
\newcommand{\JAT}{{\sl J.\ Approx.\ Theory\ }}
\newcommand{\MC}{{\sl Math.\ Comp.\ }}
\newcommand{\MI}{{\sl Math.\ Intelligencer\ }}
\newcommand{\NM}{{\sl Numer.\ Math.\ }}
\newcommand{\SIAMJNA}{{\sl SIAM J.\ Numer.\ Anal.\ }}
\newcommand{\SIAMR}{{\sl SIAM Review\ }}

% *************************************************************

\clearpage

INSERTION, ABSTRACT, AFTER FIRST SENTENCE

The bisector of ${\bf p}$ and any segment of $C$ can also be computed.
(Remove ``In such a case''.)

% *************************************************************

\clearpage

P. 3, FINAL PARAGRAPH BEFORE EXAMPLE 1.1

I think that it would be useful to move the paragraph before
Example 1.1 back to a position immediately after Definition 1.1, 
so that the reader will immediately have a geometric characterization of 
regularity.  I think that otherwise there is a potential for confusion 
among readers with an algebraic geometry background, 
since these readers will be used
to equating $f_x = f_y = 0$ (similar to $x' = y' = 0$) with singularities on
an algebraic curve, which of course are not equivalent to irregular points.

% *************************************************************

INSERTION: P.3, penultimate sentence

an ellipse or (one branch of a) hyperbola ...

% *************************************************************

\clearpage

INSERTION JUST BEFORE `NOTE THAT AS $|U| -> \infty$' ON P. 4

For example, consider the point ${\bf p} = (-2,0)$ and the circle
${\bf r}(u) = 
(\frac{1-u^{2}}{1+u^{2}},\frac{2u}{1+u^{2}})$, $u \in (-\infty,\infty)$,
which is the unit circle centered at the origin with 
${\bf r}(\infty) = (-1,0)$.

% *************************************************************

\clearpage

INSERTION, P. 5, MIDDLE, PARAGRAPH BEGINNING `THUS'

at each odd root of (15) and at finite end points ...

% *************************************************************

\clearpage

INSERTION, P. 6, JUST BEFORE `WE NOW NOTE' PARAGRAPH

\begin{exmpl}
For an algebraic curve ${\bf f}(x,y) = 0$,
\be 
\dist({\bf p},{\bf f}(x,y)) \,=\,
\min_{k \,\in\, \{1,\ldots,N\}} \, |\,{\bf p}-{\bf f}(x_k,y_k)\,| \,,
\ee
where $Q_1 = (x_1,y_1),\ldots,Q_N = (x_N,y_N)$ identify all of the roots
$Q = (x,y)$ on the algebraic curve $f(x,y)$ of
\be 
[\,\alpha-x\,]\,f_y(Q) \,-\, [\,\beta-y\,]\,f_x(Q) = 0
\ee
where $f_x$ and $f_y$ are the derivatives of $f$ with respect to $x$ and $y$.
These are the points where the curve is either singular (in the sense
of an algebraic curve) or the line drawn from $p$ meets the curve orthogonally
(since the tangent at a point $Q$ is parallel to the vector $(f_y(Q),-f_x(Q))$
\cite[p. 55]{W50}).
\end{exmpl}

% *************************************************************

\clearpage

INSERTION JUST BEFORE SECTION 2.1 ON P. 8

\begin{dfn}
The bisector ${\bf B}({\bf p},{\bf C})$ of a point ${\bf p}$ and a curve 
${\bf C}$ is the locus
traced by a point that remains equidistant from ${\bf p}$ and ${\bf C}$.
\end{dfn}

% *************************************************************

\clearpage

CORRECTION ON P. 9, STATEMENT OF PROP. 2.1

`denoting the end points of $I$ by $i_{0}$ and $i_{M+1}$'

FIGURE ON P. 9, JUST BEFORE `NOTE THAT TRIMMING' PARAGRAPH

PICTURE OF UNTRIMMED VS. TRIMMED OFFSET OF, SAY, A PARABOLA

% *************************************************************

\clearpage

INSERTION AFTER FIRST SENTENCE FOLLOWING (30) ON P. 10:

% FIGURE: \figg{fig:du}{}{.5in}
Referring to Figure~\ref{fig:du},
by the law of cosines,
\be
	d(u)^{2} = d(u)^{2} + |{\bf p} - {\bf r}(u)|^{2} - 
		   2d(u) \ |{\bf p} - {\bf r}(u)| \ \cos \theta
\ee
	% a^{2}  = b^{2} + c^{2} - 2bc cos A, [A-18]{TF81}
or 
\be
	d(u) = \frac{|{\bf p} - {\bf r}(u)|^{2}}
		    {2|{\bf p} - {\bf r}(u)|\cos \theta}.
\ee
Since $\cos \theta$ is the dot product of its two defining unit vectors,
\be
\label{eq:cos}
	\cos \theta = \frac{({\bf p}-{\bf r}(u)) \cdot {\bf n}(u)}
			   {|{\bf p} - {\bf r}(u)|}.
\ee
	% \cos \theta = (A \cdot B)/(||A|| ||B||)  [20]{LA71}
and
\be
\label{eq:du}
	d(u) = \frac{|{\bf p} - {\bf r}(u)|^{2}}{2 ({\bf p} - {\bf r}(u)) 
		\cdot {\bf n}(u)}.
\ee
(In the figure, we are assuming that $d(u)$ is positive, {\em i.e.}, 
${\bf n}(u)$ points towards
${\bf p}$'s halfplane.
(\ref{eq:du}) still holds if $d(u)$ is negative,
because the sign change of $d(u)$ is balanced by the sign
change of ${\bf n}(u)$ in (\ref{eq:cos}).)

% REMOVE `USING (24), IT MAY BE VERIFIED THAT' and have replaced (31)

Note that (31) is evidently {\em not} (in general) ...

% *************************************************************

\clearpage

CHANGE TO THE PARAGRAPH BEFORE REMARK 2.2 ON P. 11

As before: The curve defined ... (Figure~\ref{fig:du}).
REPLACE 2ND SENTENCE:
We will show below that it is appropriately named, {\em i.e.}, it
is a superset of the bisector ${\bf B}({\bf p},{\bf r}(u))$ 
(Proposition~\ref{p:superset}).

% *************************************************************

\clearpage

INSERTION, P. 11, AT END OF SENTENCE AFTER (32)

(Figure~\ref{fig:sec}).

% FIGURE: \figg{fig:sec}{}{.5in}

% *************************************************************

\clearpage

INSERTION ON P. 11, 4TH LINE OF MIDDLE PARAGRAPHE

({\em i.e.}, ... or, equivalently, the tangent at ${\bf r}(\tau)$ intersects 
${\bf p}$)

% *************************************************************

\clearpage

INSERTION JUST AFTER REMARK 2.3, P. 12

We now show that the untrimmed bisector is a superset of the bisector.

\begin{dfn}
Let $C_{q}$ be the circle with center $q$ and radius $||{\bf p}-q||$.
\end{dfn}

\begin{rmk}
\label{rmk:bis}
$q$ lies on the bisector of ${\bf p}$ and the regular curve 
${\bf r}(u)$ if and only if 
\begin{itemize}
\item
$C_q$ does not contain any 
points of ${\bf r}(u)$ in its interior ($C_q$ is ``empty''), and 
\item
$C_q$ is tangent to ${\bf r}(u)$ in at least one 
point.
\end{itemize}
(Figure~\ref{fig:bis}.)
% 	If ${\bf r}(u)$ is not regular, 
% 	then the second condition should instead be
%  	that the boundary of the circle ${\bf C}_q$ 
%	contains at least one point of ${\bf r}(u)$.
\end{rmk}
\prf
$q$ lies on the bisector if and only if the closest point of ${\bf r}(u)$
to $q$ is at distance $||{\bf p}-q||$.
\QED

% FIGURE: \figg{fig:bis}{}{.5in}

Note that when we are computing the bisector for a curve segment,
the circle $C_q$ may contain points of the curve outside of the segment.

\begin{propn}
\label{p:superset}
The untrimmed bisector of ${\bf p}$ and ${\bf r}(u)$
is a superset of the bisector of ${\bf p}$ and ${\bf r}(u)$.
\end{propn}
\prf
Let $q$ be a point of the bisector of ${\bf p}$ and ${\bf r}(u)$.
By Remark~\ref{rmk:bis}, there exists a point ${\bf r}(u_{0})$ of the curve 
that lies on the circle ${\bf C}_q$,
such that the curve is tangent
to the circle at ${\bf r}(u_{0})$ or, equivalently, 
such that the normal at ${\bf r}(u_{0})$ 
intersects the center $q$ of the circle ${\bf C}_q$.
Thus, ${\bf r}(u_{0}) + d(u_{0}) {\bf n}(u_{0}) = q$.
\QED

Since the second condition of Remark~\ref{rmk:bis}
is satisfied for all points $q$ of the untrimmed bisector (by definition),
a point $q$ of the untrimmed bisector
is a point of the bisector if and only if the interior of ${\bf C}_q$ is empty.

% *************************************************************

\clearpage

CORRECTION, p. 12

$2\alpha u$ should be $4 \alpha u$ in equation for $W_{b}(u)$ in (37).

% *************************************************************

\clearpage

INSERTION TO FIRST SENTENCE OF EXAMPLE 2.2, P. 12

As a simple example of the untrimmed bisector of a point

% *************************************************************

\clearpage

Perhaps the term `critical point' is badly chosen, since I just remembered
that you used it in your `Hierarchical segmentation' paper to mean a 
different thing.

% *************************************************************

\clearpage

INSERTION JUST BEFORE SECTION 3.1 ON P. 16

The untrimmed bisector is trimmed down to the bisector by deleting
a finite number of segments.
Like with the untrimmed offset, this is done by finding
a finite number of special points that identify possible
deviations of ${\bf b}(u)$ from the bisector.
There are four classes of these special points on the untrimmed bisector
(Definitions~\ref{d:trim} and \ref{d:cri}), 
and we split the trimming into two stages.
We now describe the first stage, which removes ``inactive'' segments.

Every point ${\bf r}(u_0)$ of the curve has a corresponding point 
${\bf b}(u_0)$ of its normal on the untrimmed bisector.
However, many of these points ${\bf b}(u_0)$ are clearly not on the bisector,
because ${\bf r}(u_0)$ is clearly not the closest point of the curve to 
${\bf b}(u_0)$ (Figure~\ref{fig:notclosest}).

% FIGURE: \figg{fig:notclosest}{}{.5in}

\begin{dfn}
The points ${\bf r}(u_0)$ and 
${\bf b}(u_0) = {\bf r}(u_0) + d(u_0){\bf n}(u_0)$ are called
{\bf corresponding} points.
\end{dfn}

\begin{dfn}
The point $q = {\bf b}(u_{0})$ of the untrimmed bisector
is {\bf active} if either one of the following conditions holds:
\begin{description}
\item[(1)]
	$q$ has more than one corresponding point on the curve\footnote{There
		are only a finite number of such points, and we will have
		much more to say about them below.}
\item[(2)]
	$q$ has only one corresponding point $q' = {\bf r}(u_0)$ on the curve,
	and either:
\begin{description}
\item[(a)]
	${\bf p}$ and the circle of curvature at $q'$
	lie on opposite sides of the tangent at $q'$, or
\item[(b)]
	the circle of curvature at $q'$ contains ${\bf p}$
	({\em i.e.}, ${\bf p}$ lies on or inside the circle of curvature).
\end{description}
\end{description}
(See Figure~\ref{fig:active}.)
A segment $S$ of ${\bf b}(u)$ is active if every point of $S$ is active.
\end{dfn}

% FIGURE \figg{fig:active}{}{.5in}

\begin{rmk}
\label{r:active}
We observe that an active point appears to lie on the bisector,
at least locally.
To see this, 
let $q$ be an active point of ${\bf b}(u)$ with only one corresponding
point $q'$.
If ${\bf p}$ lies inside or on the circle of curvature at $q'$,
then all of $C_q$ also lies in the circle of curvature at $q'$ 
and, in particular, there exists a neighborhood of $q'$ 
% ({\em i.e.}, the image of a neighborhood of $t'$ where $q' = {\bf r}(t')$)
that lies completely outside of $C_q$ \cite[p. 176]{H52}.
On the other hand, if ${\bf p}$ (and thus $C_q$) lies on the opposite side 
of the tangent at $q'$ from the circle of curvature at $q'$,
then again the curve in a neighborhood of $q'$ lies completely outside of
$C_q$.
Thus, the curve in some neighborhood of an active point (with one corresponding
point) lies completely outside of $C_q$.
In other words, an active point acts at least locally
like a point of the bisector (Remark~\ref{rmk:bis}).
\end{rmk}

\begin{propn}
An inactive point of the untrimmed bisector of ${\bf p}$ and ${\bf r}(u)$
does not lie on the bisector of ${\bf p}$ and ${\bf r}(u)$.
\end{propn}
\prf
Let $q$ be an inactive point of ${\bf b}(u)$ 
and let $q'$ be the point corresponding to $q$.
By definition, ${\bf p}$ and the circle of curvature at $q'$ lie
on the same side of the tangent at $q'$, but ${\bf p}$ lies outside this
circle (Figure~\ref{fig:active}(ii)).
$C_q$ and the circle of curvature at $q'$ are both circles that are
tangent to $q'$ (or equivalently, circles with a center on the normal at $q'$)
and they both lie on the same side of this tangent.
Moreover, $C_q$ must be strictly larger than (and strictly contain)
the circle of curvature at $q'$, because it reaches ${\bf p}$ while the circle
of curvature does not.
But all circles tangent at $q'$ that are larger than the circle of curvature
at $q'$ must lie on one side of the curve in the neighborhood of $q'$
\cite[p. 176]{H52} and consequently $C_q$ must contain some points of the curve
in its interior.
We conclude that $q$ is not on the bisector (Remark~\ref{rmk:bis}).
\QED

Since the definition of an active point depends on the side of the tangent
(at $q'$) that ${\bf p}$ and the circle of curvature lie on, 
we are interested in points where this can change.

\begin{dfn}
\label{d:trim}
A point ${\bf r}(u_{0})$ of a curve is an {\bf inflection} point
if it is not a self-intersection or a cusp, and the tangent at 
${\bf r}(u_{0})$ intersects the curve three or more times at ${\bf r}(u_{0})$.
% concavity of the curve changes at this point
A point of the curve ${\bf r}(u)$ is a {\bf class}
point if its tangent intersects the point ${\bf p}$.\footnote{This
	term is chosen in allusion to the class of a curve, which is the
	number of tangents that hit a typical point not on the curve.}
A point of the curve ${\bf r}(u)$ is a {\bf circular} point
if its circle of curvature intersects ${\bf p}$.
Equivalently, a point ${\bf r}(u_{0})$ is circular 
if its center of curvature is ${\bf b}(u_{0})$.
\end{dfn}

We have already encountered class points: they are the points of the curve 
associated with points at infinity on the untrimmed bisector 
(Section~\ref{sec:untrim}).

\begin{thm}
For a regular polynomial or rational curve or curve segment
${\bf r}(u)$ defined on the interval
$u \in I$, and a point ${\bf p}$ not on ${\bf r}(u)$, 
let ${\bf b}(u)$ be the untrimmed bisector of ${\bf p}$ and ${\bf r}(u)$,
and let $\{i_{1},\ldots,i_{M}\}$ be the ordered set of parameter values on $I$
that correspond to inflection points, class points, and circular points
of ${\bf r}(u)$.
Then, denoting the end points of $I$ by $i_{0}$ and $i_{M+1}$, we have
either
\be
{\bf b}(t) {\rm \ is\ active\ } \quad {\rm for\ all\ } t \in (i_k,i_{k+1})
\ee
or
\be
{\bf b}(t) {\rm \ is\ not\ active\ } \quad {\rm for\ all\ } t \in (i_k,i_{k+1})
\ee
on each span $(i_k,i_{k+1})$ for $k=0,\ldots,M$.
\end{thm}
\prf
Consider moving a point $q$ smoothly along the untrimmed bisector
({\em i.e.}, such that the parameter value changes smoothly).
By definition, in order to change from an active point to an inactive point,
or vice versa, one of the following must occur:
${\bf p}$ must move to a different side of the tangent at $q'$,
the circle of curvature at $q'$ must move to a different side of the tangent
at $q'$, or ${\bf p}$ must move to a different side of the circle of curvature
at $q'$ ({\em i.e.}, from inside to outside or vice versa).
The points associated with these occurrences are, respectively, 
the class points, inflection points, and circular points.
Therefore, if $q$ does not pass through a point with parameter value associated
with one of these special points, its active/inactive status will not change.
\QED

In order to make the second (and final) refinement from active segments 
to segments on the bisector,
recall that a point $q$ of the untrimmed bisector is in the bisector
if and only if its associated circle $C_q$ is empty.
A boundary between the bisector and the rest of the untrimmed bisector
is marked by a transition from points with empty circles to points 
with non-empty circles.
These transitions are associated with critical points.

\begin{dfn}
\label{d:cri}
A point ${\bf b}(u_{0})$ of the untrimmed bisector is a {\bf critical} point 
if the circle ${\bf C}_{{\bf b}(u_0)}$
is tangent to ${\bf r}(u)$ at two or more points (Figure~\ref{fig:critical}).
\end{dfn}

% FIGURE: \figg{fig:critical}{}{.5in}

Although defined differently, critical points are actually self-intersections
of the untrimmed bisector.

\begin{propn}
\label{prop:cri}
A point ${\bf b}(u_0)$ of the untrimmed bisector is a
critical point if and only if it is a self-intersection of ${\bf b}(u)$.
\end{propn}

\prf Let ${\bf q} \in {\bf b}(u)$ be a critical point, and let 
the circle ${\bf C}_{\bf q}$
be tangent to ${\bf r}(u)$ at the two points ${\bf r}(u_1)$ and ${\bf r}(u_2)$.
The normal at ${\bf r}(u_1)$ intersects the center of 
${\bf C}_{\bf q}$ and, since both ${\bf r}(u_1)$
and ${\bf p}$ lie on the boundary of this circle, a point of 
the normal is equidistant
from these two points at the center (Figure~\ref{fig:critical}).
Thus, ${\bf b}(u_{1}) = {\bf q}$.
Similarly, ${\bf b}(u_{2}) = {\bf q}$; so ${\bf q}$ is a self-intersection
of ${\bf b}(u)$.

Let ${\bf b}(u_{1}) = {\bf b}(u_{2})$ be a self-intersection.
${\bf p}$, ${\bf r}(u_{1})$, and ${\bf r}(u_{2})$ are all equidistant from 
${\bf b}(u_{1})$.
Thus, the circle ${\bf C}_{{\bf b}(u_{1})}$
contains ${\bf r}(u_{1})$ and ${\bf r}(u_{2})$.
Moreover, since the normals at ${\bf r}(u_{1})$ and ${\bf r}(u_{2})$
intersect the center ${\bf b}(u_{1})$ of this circle,
the circle is also tangent to the curve at 
${\bf r}(u_{1})$ and ${\bf r}(u_{2})$.
\QED

Proposition~\ref{prop:cri} allows us to trim using self-intersections
while arguing the validity of this trim using critical points.

\begin{thm}
\label{thm:trim2}
For a regular polynomial or rational curve or curve segment
${\bf r}(u)$ defined on the interval
$u \in I$, and a point ${\bf p}$ not on ${\bf r}(u)$, 
let ${\bf b}(u)$ be the untrimmed bisector of ${\bf p}$ and ${\bf r}(u)$,
and let $\{i_{1},\ldots,i_{M}\}$ be the ordered set of parameter values on $I$
that correspond to endpoints of active segments on ${\bf b}(u)$ 
or self-intersections of ${\bf b}(u)$,
{\em i.e.}, ${\bf b}(i_j) = {\bf b}(i_k)$ for some $1 \leq j \neq k \leq M$.
On an active segment $(b(i_j), b(i_k))$, $i_{j} < i_{k}$,
% 	Let $i_{j} < i_{k}$ be the parameter values of the endpoints of an 
% 	active 	segment. Then, either
\be
{\bf b}(t) {\rm \ is\ on\ the\ bisector\ }
\quad {\rm for\ all\ } t \in (i_l,i_{l+1})
\ee
or
\be
{\bf b}(t) {\rm \ is\ not\ on\ the\ bisector\ }
\quad {\rm for\ all\ } t \in (i_l,i_{l+1})
\ee
on each span $(i_l,i_{l+1})$ for $l=j,\ldots,k-1$ between successive
self-intersections on the active segment.
\end{thm}

\prf
Let ${\bf b}(u_{1})$ and ${\bf b}(u_{2})$ be two points
on an active segment of the untrimmed bisector ($u_1 < u_2$),
neither of them self-intersections,
${\bf b}(u_{1})$ on the bisector but
${\bf b}(u_{2})$ not on the bisector.
We shall show that the segment ${\bf b}(u_1,u_2)$ between ${\bf b}(u_{1})$ 
and ${\bf b}(u_{2})$
must contain a self-intersection ({\em i.e.}, a critical point) $\alpha$.
In other words, if an active subsegment of ${\bf b}(u)$ does not contain a 
self-intersection, it either lies entirely on or entirely off the bisector.

Since ${\bf b}(u_{1})$ lies on the bisector,
$C_{{\bf b}(u_{1})}$ touches the curve at the point ${\bf r}(u_{1})$ 
corresponding to ${\bf b}(u_{1})$
{\em and} $C_{{\bf b}(u_{1})}$ is empty (Remark~\ref{rmk:bis}).
As a point $q$ moves along the untrimmed bisector from ${\bf b}(u_{1})$ 
towards ${\bf b}(u_{2})$,
the radius of the circle $C_q$ changes smoothly 
while maintaining contact with $p$,
and at all times the neighborhood of the corresponding point $q'$ lies
completely outside of $C_q$ (Remark~\ref{r:active}).
In order to leave the bisector, the circle $C_q$ must become full, {\em i.e.}, 
the curve must enter $C_q$ (Remark~\ref{rmk:bis}).
We also know that this eventually happens, because ${\bf b}(u_{2})$ 
is not on the bisector.
Since the curve in the neighborhood of $q'$ does not enter $C_q$ 
% this is important because otherwise in the limit the curve might touch
% {\em and} enter at $q'$, e.g., inflection.
and the curve is regular, 
the curve cannot enter $C_q$ at $q'$
and must first enter $C_q$ at $q'' \neq q'$.
In order to enter $C_q$, the curve must first become tangent to $C_q$, 
and this position of $q$ is a critical point of the untrimmed
bisector, since its circle has two points of tangency with the curve, one
at $q'$ and one at the point where the curve is about to enter the circle
(Figure~\ref{fig:critical}).
Thus, the segment ${\bf b}(u_1,u_2)$ contains a critical point.
\QED

\begin{rmk}
If the desired interpretation for the bisector of a point ${\bf p}$ 
and a curve segment $S = r(I)$ of ${\bf r}(u)$ is 
\be
	\{y\ |\ {\rm dist}(y,{\bf p}) = {\rm dist}(y,{\bf r}(u)) 
		{\rm \ and\ the\ closest\ point\ of\ } {\bf r}(u) 
		{\rm \ to \ } y {\rm \ lies\ on\ } S \}
\ee
rather than its usual meaning
\be
	\{y|{\rm \ dist}(y,{\bf p}) = 
	    {\rm \ dist}(y,S\}
\ee
then the statement of Theorem~\ref{thm:trim2} is unchanged except that,
rather than using self-intersections of the segment ${\bf b}(I)$ 
({\em i.e.}, self-intersections ${\bf b}(u_1) = {\bf b}(u_2)$
where $u_1, u_2 \in I$),
one should use all self-intersections of ${\bf b}(u)$ that lie on 
the segment ${\bf b}(I)$ 
({\em i.e.}, self-intersections ${\bf b}(u_1) = {\bf b}(u_2)$
where $u_1 \in I$ but $u_2$ need not lie on $I$).
\end{rmk}

We now have an algorithm for computing the bisector of a point
${\bf p}$ and a regular parametric curve ${\bf r}(u)$.

\begin{enumerate}
\item
	Compute the untrimmed bisector of ${\bf p}$ and ${\bf r}(u)$,
	using (\ref{varoffset}) and (\ref{du}).
	For the polynomial and rational curve, this untrimmed bisector
	is given by (\ref{pbsctr}) and (\ref{rbsctr}), respectively.
\item
	Find the inflection points, class points, and circular points 	
	of the curve ${\bf r}(u)$.
\item
	For each segment on the untrimmed bisector defined by these 
	special points, compare the distance of the midpoint of this
	segment from ${\bf p}$ and ${\bf r}(u)$ (using (\ref{distance2}))
	and discard the segment if these distances are not equal.
\item
	Find the self-intersections of the untrimmed bisector
	(using (\ref{Iresltnt})).
\item
	Split each remaining segment of the untrimmed bisector
	at these self-intersections, and for each of the resulting
	segments: compare the distance of the midpoint of this
	segment from ${\bf p}$ and ${\bf r}(u)$ (using (\ref{distance2}))
	and discard the segment if these distances are not equal.
	The remaining segments are the bisector.
\end{enumerate}

\end{document}