(Message inbox:4547)
Return-Path: rbf@aifh.edinburgh.ac.uk
Return-Path: <rbf@aifh.edinburgh.ac.uk>
Received: from sun2.nsfnet-relay.ac.uk
           by blaze.cs.jhu.edu; Wed, 9 Dec 92 08:18:15 EST
Sender: rbf@aifh.edinburgh.ac.uk
Via: uk.ac.edinburgh.aifh; Wed, 9 Dec 1992 13:16:43 +0000
Date: Wed, 9 Dec 92 13:15:49 GMT
Message-Id: <12992.9212091315@iodine.aifh.ed.ac.uk>
From: Bob Fisher <rbf@aifh.edinburgh.ac.uk>
Subject: Re: Mathematics of Surfaces V paper
To: jj@blaze.cs.jhu.edu
In-Reply-To: jj@edu.jhu.cs.blaze's message of Tue, 08 Dec 92 15:26:47 EST
Organisation: Dept. of Artificial Intelligence, Univ. of Edinburgh.

> 
> I have not yet been able to get in touch with Rida to verify that
> he submitted the final revised version.
> In order to clear this up, I was wondering if you could do
> me a favour:
> could you mail me the latest latex version that you have
> of the paper by Rida Farouki and John Johnstone (Computing Point/Curve
> and Curve/Curve Bisectors)?
> I can then doublecheck that the copy you have is the correct one.
> (If it is not, then only minor--although important--additions 
> would need to be made, meaning that even if typesetting was begun
> the changes would be implementable.)
> 
I enclose below my latest version. I have made some changes to the
start of the file arising from a revision in the book macros.
Otherwise, no changes have been made.

Bob
================================================================
\documentstyle{ima}
\begin{document}

\def\rd{{\rm d}}
\def\re{{\rm e}}
\def\ri{{\rm i}}
\def\ldash{\vrule height 3pt width 0.35in depth -2.5pt}
\newcommand{\half}{\textstyle{1 \over 2}\displaystyle}
\newcommand{\prf}{\noindent{{\bf Proof} :\ }}
\newcommand{\QED}{\vrule height 1.4ex width 1.0ex depth -.1ex\ \medskip}

\newtheorem{dfn}{Definition}[section]
\newtheorem{rmk}{Remark}[section]
\newtheorem{lma}{Lemma}[section]
\newtheorem{propn}{Proposition}[section]
\newtheorem{exmpl}{Example}[section]
\newtheorem{thm}{Theorem}[section]
\newtheorem{crlry}{Corollary}[section]

\chapterb[Computing bisectors]{Computing point/curve and curve/curve bisectors}
{Rida~T.~Farouki}{IBM Thomas J. Watson Research Center}
{John~K.~Johnstone}{The Johns Hopkins University}
\authormark{Farouki and Johnstone}

\section{Introduction}

The {\it bisector\/} of two geometrical elements (points,
curves, surfaces, etc.) is the locus described by a variable
point that moves so as to remain equidistant with respect to
those elements. The parabola is perhaps the simplest non--trivial
example of a bisector --- recall \cite{coxeter69} its descriptive
definition as the locus of a point that maintains equal distances
from a given point (the {\it focus}) and a given straight line
(the {\it directrix}).

Although bisectors arise naturally in a variety of algorithms
\cite{bookstein79,held91,lee82,yap87} concerned with the shape
analysis and decomposition of areas and volumes, the CAD literature
contains no systematic discussion of the {\it computational\/}
problems incurred in dealing with them in the curvilinear realm.
However, there has been a fairly extensive investigation of the
basic geometrical and topological aspects of closely--related
loci --- the medial (symmetric) axis, or ``skeleton'' \cite
{blum67,blum73,blum78}; symmetry sets \cite
{banchoff87,bruce85,bruce86,giblin85}; and cut loci
\cite{wolter85,wolter92}.

We shall focus here on algorithms for computing bisectors in
the plane --- we review recent work \cite{farouki91} on the
formulation of point/curve bisectors, and give preliminary
results on the more--difficult curve/curve bisector problem
that make extensive use of the point/curve bisector machinery.
Specifically, we consider curve/curve bisectors as the {\it
envelopes\/} of one--parameter families of point/curve bisectors
defined by simultaneously considering one curve in its entirety
and a discrete point that moves along the other.

This approach is motivated by an attractive property of the
point/curve bisectors --- namely, they can be represented as
``variable--distance offsets'' to the given curve that (unlike
fixed--distance offsets \cite{farouki90a,farouki90b}) are {\it
generically\/} rational curves if the given curve is polynomial
or rational. They are thus compatible with the canonical
representations of most CAD systems.

Before proceeding, we must clarify what is meant by the
{\it distance\/} between a point and a curve:

\begin{dfn}
The distance of a point ${\bf q}$ from a regular parametric
curve ${\bf r}(u)$ defined on the parameter interval $I$ is
given by
\begin{equation} \label{distance}
{\rm dist}({\bf q},{\bf r}(u)) \,=\,
\inf_{u \,\in\, I} \, |\,{\bf q}-{\bf r}(u)\,| \,.
\end{equation}
\end{dfn}

Although we confine our attention here to parametric curves, most
of the concepts extend straightforwardly to implicitly--defined
curves also. A curve ${\bf r}(u)$ is said to have a {\it regular\/}
parameterization if its derivative ${\bf r}'(u)=\rd{\bf r}/\rd u$
is non--vanishing over the domain of definition, $u \in I$ (this
guarantees that the curve locus is smooth \cite{farouki92},
although it may exhibit self--intersections).

If ${\bf r}(u)$ is a polynomial curve, the bound (\ref{distance})
will always be attained at a {\it finite\/} parameter value $u$
regardless of whether $I$ is finite or infinite. If ${\bf r}(u)$
is a rational curve and $I$ is not finite, however, it is possible
that (\ref{distance}) will be attained only in the limit $|u|\to
\infty$.

\begin{rmk} \label{polydist} {\rm
For the point ${\bf q}=(a,b)$ and the regular polynomial curve
${\bf r}(u)$ $=\{X(u),Y(u)\}$ of degree $n$ defined on $u \in I$,
let $u_1,\ldots,u_N$ be the distinct {\it odd}--multiplicity roots
of the polynomial
\begin{equation} \label{Pperp}
P_\perp(u) \,=\,
[\,a-X(u)\,]\,X'(u) \,+\, [\,b-Y(u)\,]\,Y'(u)
\end{equation}
of degree $2n-1$ on the interior of the interval $I$, augmented by
the finite end points (if any) of $I$. Then the distance function
$(\ref{distance})$ may be expressed as
\begin{equation} \label{distance2}
{\rm dist}({\bf q},{\bf r}(u)) \,=\,
\min_{1 \le k \le N} \, |\,{\bf q}-{\bf r}(u_k)\,| \,.
\end{equation}
} \end{rmk}

An analogous result holds for regular rational curves, provided we
replace the odd roots of the polynomial (\ref{Pperp}) by those of
\begin{eqnarray} \label{Rperp}
P_\perp(u)
&\,=\,& [\,aW(u)-X(u)\,]\,[\,W(u)X'(u)-W'(u)X(u)\,] \nonumber \\
&\,+\,& [\,bW(u)-Y(u)\,]\,[\,W(u)Y'(u)-W'(u)Y(u)\,]
\end{eqnarray}
satisfying $W(u)\not=0$ on the interval $I$. Geometrically,
roots of $P_\perp(u)$ identify points of the curve where lines
drawn from ${\bf q}$ meet ${\bf r}(u)$ orthogonally. The distance
(\ref{distance}) is simply the smallest of the lengths of these
perpendiculars (and the chords drawn from ${\bf q}$ to the affine
end points of ${\bf r}(u)$, if any). Even--multiplicity roots of
$P_\perp(u)$ are ignored since they identify points of ${\bf r}(u)$
where the distance $|\,{\bf q}-{\bf r}(u)\,|$ ``levels off'' but
then continues to increase or decrease ({\it i.e.}, it does not
attain a local extremum).

\begin{propn} {\rm
For a regular curve ${\bf r}(u)$ the function $f({\bf q})=
{\rm dist}({\bf q},{\bf r}(u))$ is everywhere continuous, but
not always differentiable, with respect to ${\bf q}$. }
\end{propn}

\prf See \cite[Proposition~1.2]{farouki91}; also \cite{kelly79}.
\QED

\section{Point/curve bisectors}

Recall \cite{farouki90a,farouki90b} that the ``untrimmed'' offset
at fixed distance $d$ to a regular plane parametric curve ${\bf r}
(u)$ with unit normal ${\bf n}(u)$ is the locus defined by
\begin{equation} \label{offset}
{\bf o}(u) \,=\, {\bf r}(u) + d\,{\bf n}(u) \,.
\end{equation}
The {\it true\/} offset to ${\bf r}(u)$ at distance $d$ is
obtained from (\ref{offset}) by ``trimming'' away certain parametric
subsegments of the latter \cite{farouki90a,farouki90b}. (While, for
each $\xi$, corresponding points ${\bf o}(\xi)$ and ${\bf r}(\xi)$
are distance $d$ apart measured along their mutual normal ${\bf n}
(\xi)$, the point ${\bf o}(\xi)$ of the untrimmed offset is not
necessarily distance $d$ from the {\it entire curve\/} ${\bf r}(u)$
--- see \cite{farouki90a,farouki90b} for further details.)

We define the ``untrimmed'' bisector of a fixed point ${\bf p}$
and a regular plane curve ${\bf r}(u)$ as a generalization of
(\ref{offset}), replacing the fixed offset distance $d$ by a
{\it displacement function\/}:
\begin{equation} \label{varoffset}
{\bf b}(u) \,=\, {\bf r}(u) + d(u) {\bf n}(u) \,.
\end{equation}
For ${\bf b}(u)$ to represent the untrimmed bisector of ${\bf p}$
and ${\bf r}(u)$, the appropriate choice is
\begin{equation} \label{du}
d(u) \,=\, {|\,{\bf p}-{\bf r}(u)\,|^2 \over
2\,({\bf p}-{\bf r}(u))\cdot{\bf n}(u)} \,\,.
\end{equation}
The reader can easily verify that, for each $u$, the displacement
(\ref{du}) identifies the {\it unique\/} point along the curve normal
line at ${\bf r}(u)$ that is equidistant from the given point ${\bf p}$
and the curve point ${\bf r}(u)$. Note also that $d(u)\not=0$ for all
$u$ when ${\bf p}$ does not lie on ${\bf r}(u)$.
Figure~\ref{fig:varoffset} illustrates the formulation of untrimmed
point/curve bisectors as ``variable--distance'' offsets.

\begin{figure}[htbp] \vspace{3in}
\caption[]{Examples of untrimmed point/curve bisectors \\
regarded as ``variable--distance'' offset curves. }
\label{fig:varoffset} \end{figure}

\begin{rmk} {\rm
>From expressions (\ref{varoffset}) and (\ref{du}) we see
that when the given curve ${\bf r}(u)=\{x(u),y(u)\}$ has a
polynomial or rational parameterization, the untrimmed bisector
${\bf b}(u)$ is a {\it rational\/} curve --- since there is an
obvious cancellation of the radical $\sqrt{{x'}^2(u)+{y'}^2(u)}$
incurred in the unit normal ${\bf n}(u)$. }
\end{rmk}

We now briefly enumerate some key properties of untrimmed
point/curve bisectors (full details may be found in \cite
{farouki91}; see also \cite{field92}).

\subsection{Parameterization of the untrimmed bisector}

Let $X(u),Y(u),W(u)$ be polynomials of degree $n$, and let
${\bf b}(u)$ denote the untrimmed bisector of the point ${\bf p}
=(\alpha,\beta)$ and the regular polynomial curve ${\bf r}(u)=
\{X(u),Y(u)\}$ or rational curve ${\bf r}(u)=\{X(u)/W(u),
Y(u)/W(u)\}$.

Writing ${\bf b}(u)=\{X_b(u)/W_b(u),Y_b(u)/W_b(u)\}$, from
(\ref{varoffset}) and (\ref{du}) we can express the homogeneous
coordinates of the untrimmed bisector as
\begin{eqnarray} \label{pbsctr}
X_b &\,=\,& [\,\alpha^2-X^2+(\beta-Y)^2\,]\,Y'
 \,-\, 2(\beta-Y)XX' \,, \nonumber \\
Y_b &\,=\,& 2(\alpha-X)YY'
 \,-\, [\,(\alpha-X)^2+\beta^2-Y^2\,]\,X' \,, \nonumber \\
W_b &\,=\,& 2\,[\,(\alpha-X)Y'-(\beta-Y)X'\,]
\end{eqnarray}
in the case of a polynomial curve, and
\begin{eqnarray} \label{rbsctr}
X_b &\,=\,& [\,\alpha^2W^2-X^2+(\beta W-Y)^2\,]\,V
 \,-\, 2(\beta W-Y)XU \,, \nonumber \\
Y_b &\,=\,& 2(\alpha W-X)YV
 \,-\, [\,(\alpha W-X)^2+\beta^2W^2-Y^2\,]\,U \,, \nonumber \\
W_b &\,=\,& 2W\,[\,(\alpha W-X)V-(\beta W-Y)U\,]
\end{eqnarray}
(where we set $U=WX'-W'X$ and $V=WY'-W'Y$) in the rational case.

\begin{rmk} {\rm
>From the above it may be verified that ${\bf b}(u)$ is of degree
$3n-1$ or $4n-2$, at most, when ${\bf r}(u)$ is a polynomial or
rational curve of degree $n$. }
\end{rmk}

Expressions (\ref{pbsctr}) and (\ref{rbsctr}) can, of course, be
formulated to yield control points for ${\bf b}(u)$ in the standard
B\'ezier representation over any parameter domain of interest when
${\bf r}(u)$ is specified in B\'ezier form. This is an elementary
application of the arithmetic of polynomials in Bernstein form \cite
{farouki88} and is desirable in view of the numerical stability
\cite{farouki87} of this representation.

\subsection{Points at infinity}

The untrimmed bisector has (real) points at infinity corresponding
to {\it finite\/} (real) points of ${\bf r}(u)$ when the condition
$({\bf p}-{\bf r}(u))\cdot{\bf n}(u)=0$ is satisfied, {\it i.e.},
the line joining ${\bf p}$ to ${\bf r}(u)$ is tangential to the
curve at that point. Thus, for regular polynomial and rational
curves, the roots of the polynomials
\begin{equation} \label{Pinf}
P_\infty(u) \,=\,
[\,\alpha-X(u)\,]\,Y'(u) \,-\, [\,\beta-Y(u)\,]\,X'(u)
\end{equation}
and
\begin{eqnarray} \label{Rinf}
P_\infty(u)
&\,=\,& [\,\alpha W(u)-X(u)\,]\,[\,W(u)Y'(u)-W'(u)Y(u)\,] \nonumber \\
&\,-\,& [\,\beta  W(u)-Y(u)\,]\,[\,W(u)X'(u)-W'(u)X(u)\,]
\end{eqnarray}
identify points at infinity on ${\bf b}(u)$. In the rational case
${\bf b}(u)$ has, additionally, points at infinity corresponding to
finite values of $u$ that are roots of $W(u)$.

\subsection{Cusps and higher--order irregular points}

The untrimmed bisector ${\bf b}(u)$ of a point ${\bf p}$ and a
smooth curve ${\bf r}(u)$ is not, in general, a smooth locus itself.
Irregular points of the untrimmed bisector are identified by the
condition $|{\bf b}'(u)|=0$, and by differentiating (\ref{varoffset})
it can be verified \cite{farouki91} that this is satisfied when
the curvature $\kappa=|{\bf r}'|^{-3}\,({\bf r}'\times{\bf r}'')
\cdot{\bf z}$ of the given curve ${\bf r}(u)$ equals the local
``critical'' value
\begin{equation} \label{kappac}
\kappa_{\rm crit}(u) \,=\, -\,{1 \over d(u)} \,.
\end{equation}
Geometrically, equation (\ref{kappac}) indicates that a point
of ${\bf b}(u)$ will be irregular if it coincides with the {\it
center of curvature\/} for the corresponding point of ${\bf r}(u)$.
Thus, irregular points of the untrimmed bisector must lie on the
{\it evolute\/} (the locus of centers of curvature) of the curve
${\bf r}(u)$, regardless of the location of the point ${\bf p}$.
Some examples are shown in Figure~\ref{fig:evolute}.

\begin{figure}[htbp] \vspace{3.6in}
\caption[]{Cusps of ${\bf b}(u)$ lie on the evolute
(dashed curve) of ${\bf r}(u)$.}
\label{fig:evolute} \end{figure}

It is intriguing to note that while (\ref{kappac}) appears
on casual inspection to be a straightforward generalization of
the analogous criterion\footnote{For constant--distance offsets,
it can further be shown \cite{farouki90a} that their cusps meet
the evolute {\it orthogonally}. From the examples of Figure~\ref
{fig:evolute}, however, it is evident that cusps of the untrimmed
bisector ${\bf b}(u)$ are not, in general, orthogonal to the
evolute of ${\bf r}(u)$.} $\kappa(u)=-1/d$ \cite{farouki90a} for
irregular points on the constant--distance offset (\ref{offset}),
this condition is in fact not appropriate to variable--distance
offsets with {\it arbitrary\/} displacement functions $d(u)$ ---
it is specific to the particular form given by (\ref{du}), for
which $\kappa(u)=-1/d(u) \;\Longrightarrow\; d^{\,\prime}(u)=0$.

For polynomial and rational curves, respectively, the parameter
values of irregular points on the untrimmed bisector are roots of
the polynomials
\begin{eqnarray} \label{Pcusp}
P_c &\,=\,& [\,(\alpha-X)^2+(\beta-Y)^2\,]\,(X'Y''-X''Y') \nonumber \\
    &\,+\,& 2\,({X'}^2+{Y'}^2)\,[\,(\alpha-X)Y'-(\beta-Y)X'\,]
\end{eqnarray}
and
\begin{eqnarray} \label{Rcusp}
P_c &\,=\,& W\,[\,(\alpha W-X)^2+(\beta W-Y)^2\,]\,(U_1V_2-U_2V_1)
\nonumber \\
    &\,+\,& 2\,(U_1^2+V_1^2)\,[\,(\alpha W-X)V_1-(\beta W-Y)U_1\,]
\end{eqnarray}
where, for brevity, we have written $(U_1,V_1)=(WX'-W'X,WY'-W'Y)$
and $(U_2,V_2)=(WX''-W''X,WY''-W''Y)$ in the latter case.

\begin{rmk} {\rm
If the curvature of ${\bf r}(u)$ attains the critical value
(\ref{kappac}) without being an extremum $(\kappa'(u)\not=0$),
the corresponding irregular point of ${\bf b}(u)$ is a {\it cusp},
{\it i.e.}, a sudden tangent reversal. However, if the value
(\ref{kappac}) represents a local extremum of $\kappa(u)$ --- a
``vertex'' of the curve ${\bf r}(u)$ --- then ${\bf b}(u)$ will
exhibit a {\it tangent--continuous point of infinite curvature}
(see \cite{farouki91}). }
\end{rmk}

\subsection{Self--intersections of the untrimmed bisector}

It is possible to construct a minimal polynomial $P_i(u)$
whose distinct real roots $u_1,u_2,\ldots$ correspond to
self--intersections of the untrimmed bisector, so that ${\bf b}
(u_j)={\bf b}(u_k)$ for some $j\not=k$. This construction is quite
involved, however, and involves a mass of algebraic details that
are not germane to our present brief survey. Ultimately, it relies
on expanding a modified Sylvester determinant and discarding certain
``extraneous'' factors from the result --- full details may be
found in \cite{farouki91}.

\begin{rmk} {\rm
We can deduce\footnote{Note that
such a simple deduction cannot be made for the self--intersection
polynomial \cite{farouki90b} appropriate to the constant--distance
offsets (\ref{offset}), since these offsets are {\it not\/} rational.}
the degree of $P_i(u)$ indirectly by noting that
the untrimmed bisector ${\bf b}(u)$ is a {\it rational\/} curve,
and invoking the fact \cite{primrose55} that any rational curve
of degree $m$ has precisely $\half(m-1)(m-2)$ double points (or
their equivalent). Consider, for example, the case where ${\bf r}
(u)$ is a polynomial curve of degree $n$. Then ${\bf b}(u)$ is
of degree $m=3n-1$, and thus has $\half(3n-2)(3n-3)$ double
points --- including cusps, which occur at the roots of (\ref
{Pcusp}), a polynomial of degree $4(n-1)$. Thus the number of
nodes is $\half(3n-2)(3n-3)-4(n-1)=\half(9n^2-23n+14)$. Since
two distinct parameters are associated with each node, we infer
that ${\rm deg}(P_i)=9n^2-23n+14$. Hence for point/cubic bisectors
we have ${\rm deg}(P_i)=26$ in general, although in typical examples
one will observe fewer than thirteen real self--intersections. }
\end{rmk}

We assume henceforth that $P_i(u)$, along with the polynomials
$P_\infty(u)$ and $P_c(u)$ defined in the preceding discussion,
have been determined and that their real roots on the parameter
domain of interest have been identified. We emphasize again that
the Bernstein--B\'ezier form is the preferred medium for executing
these calculations \cite{farouki91a}, since even simple curves can
give rise to polynomials of relatively high degree. Having thus
identified all cusps, points at infinity, and self--intersections
of the untrimmed bisector, we can proceed to {\it trim\/} ${\bf b}
(u)$, {\it i.e.}, to determine which parametric subsegments (if
any) must be discarded to give the ``true'' point/curve bisector.

\subsection{The trimming procedure}

>From (\ref{varoffset}) and (\ref{du}) we note that the untrimmed
bisector of ${\bf p}$ and ${\bf r}(u)$ can be regarded as the locus
of centers of a family of circles that pass through ${\bf p}$ and
touch --- {\it i.e.}, are tangent to --- ${\bf r}(u)$ at some point
(see Figure~\ref{fig:circles}).

\begin{figure}[htbp] \vspace{3.9in}
\caption[]{Circles that pass through ${\bf p}$ and touch ${\bf r}(u)$.}
\label{fig:circles} \end{figure}

\begin{rmk} \label{rmk:bis} {\rm
Let $C_\xi$ denote the circle that passes through ${\bf p}$ and
touches ${\bf r}(u)$ at $u=\xi$. Then the point ${\bf b}(\xi)$ of
the untrimmed bisector --- the center of the circle $C_\xi$ ---
belongs to the ``true'' bisector if and only if no point of the
curve ${\bf r}(u)$ lies {\it inside\/} the circle $C_\xi$. }
\end{rmk}

The untrimmed bisector ${\bf b}(u)$ is ``trimmed'' down to the
true bisector by deleting a finite number of continuous segments.
To accomplish this, we must cut ${\bf b}(u)$ at certain ``special''
points that delineate possible deviations of the untrimmed bisector
from the true bisector. These special points are the points at
infinity, cusps, and self--intersections of ${\bf b}(u)$ discussed
above.

A point ${\bf b}(\xi)$ of the untrimmed bisector may be denied
membership in the true bisector under two circumstances: (i) the
curve ${\bf r}(u)$ crosses the circle $C_\xi$ ``locally,'' {\it
i.e.}, at the point ${\bf r}(\xi)$ to which $C_\xi$ is tangent;
or (ii) the curve ${\bf r}(u)$ crosses $C_\xi$ in a ``global''
sense, {\it i.e.}, at some other point ${\bf r}(\omega)$ totally
unrelated to the point of tangency ($\omega\not=\xi$).

We split the trimming process into two stages. In the first
stage ``inactive'' segments of ${\bf b}(u)$, which fall under
category (i) above, are deleted. The second stage eliminates
segments of ${\bf b}(u)$ that fall under category (ii).

\subsubsection{Active and inactive segments}

For a given point ${\bf p}$ and curve ${\bf r}(u)$, we call
${\bf r}(\xi)$ and ${\bf b}(\xi)={\bf r}(\xi)+d(\xi){\bf n}(\xi)$
``corresponding'' points of the curve and the untrimmed point/curve
bisector for each $\xi$. Note that a given {\it geometric\/} point
on the untrimmed bisector may have more than one corresponding
point on the curve ${\bf r}(u)$, {\it i.e.}, for $\xi_1\not=\xi_2$
it is possible that ${\bf r}(\xi_1)+d(\xi_1){\bf n}(\xi_1)={\bf r}
(\xi_2)+d(\xi_2){\bf n}(\xi_2)$.

Now it is clear that certain points ${\bf b}(\xi)$ of the untrimmed
bisector do not belong to the ``true'' bisector, because there are
points along the curve ${\bf r}(u)$ that are closer to these points
${\bf b}(\xi)$ than their corresponding points ${\bf r}(\xi)$.

\begin{dfn} \label{dfn:active}
{\rm
The point ${\bf q}={\bf b}(\xi)$ of the untrimmed bisector
is {\it active\/} if either of the following conditions holds
(see Figure~\ref{fig:active}):
\begin{description}
\item[{\rm (1)}]
        ${\bf q}$ has more than one corresponding point on
        the curve; or
\item[{\rm (2)}]
        ${\bf q}$ has only one corresponding point ${\bf q}'
        ={\bf r}(\xi)$ on the curve, and:
\begin{description}
\item[{\rm (a)}]
        the point ${\bf p}$ lies on or inside the circle of
        curvature at ${\bf q}'$; or
\item[{\rm (b)}]
        the point ${\bf p}$ and the circle of curvature at
        ${\bf q}'$ lie on opposite \\ sides of the tangent
        at ${\bf q}'$.
\end{description}
\end{description}
A segment $S$ of ${\bf b}(u)$ is active if each point of $S$
is active. }
\end{dfn}

Case (1) above corresponds to self--intersections of ${\bf b}(u)$,
which we shall deal with in the following section. For now, we
concentrate on case (2).

\begin{figure}[htbp] \vspace{2.4in}
\caption[]{Active and inactive points on the untrimmed bisector.}
\label{fig:active} \end{figure}

An active point appears --- at least ``locally'' --- to be on
the true bisector. In particular, if ${\bf b}(\xi)$ is an active
point with a unique corresponding point ${\bf r}(\xi)$, then in
some neighborhood of $u=\xi$ the curve ${\bf r}(u)$ lies completely
{\it outside\/} the circle $C_\xi$ (see Remark~\ref{rmk:bis}).

To see this, we note that if ${\bf p}$ lies inside or on the
circle of curvature at ${\bf r}(\xi)$, then it can be shown \cite
[Lemma~3.1]{farouki91} that $C_\xi$ also lies inside or coincides
with the circle of curvature at ${\bf r}(\xi)$ and, in particular,
there exists a neighborhood of $u=\xi$ on ${\bf r}(u)$ that lies
entirely outside the circle $C_\xi$ \cite[p.~176]{H52}. On the
other hand, if ${\bf p}$ (and thus $C_\xi$) lies on the opposite
side of the tangent at ${\bf r}(\xi)$ to the circle of curvature
there, then again ${\bf r}(u)$ in some neighborhood of $u=\xi$
lies entirely outside the circle $C_\xi$.

\begin{rmk} {\rm
An inactive point of the untrimmed bisector ${\bf b}(u)$ of ${\bf p}$
and ${\bf r}(u)$ does not belong to their true bisector. }
\end{rmk}
\prf See \cite[Proposition~3.1]{farouki91}. \QED

Since the definition of whether the point ${\bf b}(\xi)$ of the
untrimmed bisector is active or inactive depends on the location of
${\bf p}$ relative to the tangent line and the circle of curvature
of ${\bf r}(u)$ at $u=\xi$, we are interested in points where these
relative locations can change.

\begin{thm}
\label{thm:active}
Let ${\bf b}(u)$ be the untrimmed bisector of a point ${\bf p}$
and a regular curve ${\bf r}(u)$ defined on the interval $u \in I$,
and let $\{u_1,\ldots,u_M\} \in I$ be the ordered set of parameter
values that correspond to points at infinity or cusps on ${\bf b}
(u)$. Then, denoting the end points of $I$ by $u_0$ and $u_{M+1}$,
we have either
\begin{equation}
{\bf b}(u) {\rm \ is\ active\ for\ all\ } u \in (u_k,u_{k+1})
\end{equation}
or
\begin{equation}
{\bf b}(u) {\rm \ is\ inactive\ for\ all\ } u \in (u_k,u_{k+1})
\end{equation}
on each segment $(u_k,u_{k+1})$ for $k=0,\ldots,M$ of $\,{\bf b}(u)$.
\end{thm}
\prf  See \cite{farouki91}. \QED

In the context of the above theorem,\footnote{Exceptionally,
${\bf p}$ may lie simultaneously on both the tangent line and
the circle of curvature at ${\bf r}(\xi)$ if that point is an
{\it inflection}; but from Definition~\ref{dfn:active} it can
be seen that all points of the untrimmed bisector in some
neighborhood of ${\bf b}(\xi)$ must then be active.}
we remind the reader that
the point ${\bf b}(\xi)$ of the untrimmed bisector is (i) a point
at infinity if ${\bf p}$ lies on the tangent line at ${\bf r}(\xi)$;
and (ii) a cusp if ${\bf p}$ lies on the circle of curvature at
${\bf r}(\xi)$.

\subsubsection{Trimming at self--intersections}
\label{sec:critical}

Recall (Remark~\ref{rmk:bis}) that the point ${\bf b}(\xi)$ of
the untrimmed bisector will belong to the true bisector if and
only if the circle $C_\xi$ centered on ${\bf b}(\xi)$ that passes
through ${\bf p}$ and touches ${\bf r}(u)$ at $u=\xi$ has no point
of ${\bf r}(u)$ in its interior.

Self--intersections ${\bf b}(\xi_1)={\bf b}(\xi_2)$ of the
untrimmed bisector (where $\xi_1\not=\xi_2$) correspond to points
whose circles $C_{\xi_1}$ and $C_{\xi_2}$ coincide --- {\it i.e.},
there exists a single circle $C=C_{\xi_1}=C_{\xi_2}$ that passes
through ${\bf p}$ and touches ${\bf r}(u)$ at {\it two\/} points,
$u=\xi_1$ and $u=\xi_2$. As we move along the untrimmed bisector,
its self--intersections delineate (possible) transitions from a
regime in which the circles $C_\xi$ corresponding to each point
${\bf b}(\xi)$ are ``empty'' to one in which they are ``occupied''
by points of ${\bf r}(u)$, or vice--versa --- see \cite{farouki91}
for full details.

Thus, the second stage of the trimming process consists of
splitting the ``active'' segments of the untrimmed bisector at
its self--intersections, and validating the resulting subsegments
for membership in the true bisector (see Theorem~3.2 in \cite
{farouki91}). As noted above, computing the parameter values
that correspond to self--intersections of ${\bf b}(u)$ is not
a trivial matter --- relatively simple input curves give rise to
polynomials $P_i(u)$ of high degree, reflecting the potentially
complicated topology ({\it e.g.}, Figure~\ref{fig:sprbla}) of
the real locus of ${\bf b}(u)$.

\begin{figure}[htbp] \vspace{3.6in}
\caption[]{Untrimmed bisectors of a point and ${\bf r}(u)=\{u,u^4\}$.}
\label{fig:sprbla} \end{figure}

\subsubsection{The algorithm}

Given a ``robust'' polynomial root--solver, we now outline an
algorithm, based on the above ideas, for computing the bisector
of a point ${\bf p}$ and a regular (polynomial or rational)
parametric curve ${\bf r}(u)\,$:

\begin{enumerate}
\item
        Formulate the untrimmed bisector ${\bf b}(u)$ --- as
        defined by equations (\ref{pbsctr}) and (\ref{rbsctr}),
        for polynomial and rational curves, respectively.
\item
        Compute the points at infinity and cusps of ${\bf b}(u)$,
        {\it i.e.}, the roots of (\ref{Pinf}) and (\ref{Pcusp})
        or (\ref{Rinf}) and (\ref{Rcusp}), as appropriate.
\item
        For each segment of ${\bf b}(u)$ delineated by these points,
        compare distances of the segment parametric midpoint to the
        point ${\bf p}$ and to the curve ${\bf r}(u)$, using (\ref
        {distance2}). Discard segments for which these distances
        are unequal.
\item
        Formulate the self--intersection polynomial $P_i(u)$ ---
        as described in \cite{farouki91} --- and identify its real
        roots that lie on the remaining ``active'' segments.
\item
        Split each active segment at the self--intersections of
        ${\bf b}(u)$ and compare the distances of the parametric
        midpoint of each resulting subsegment to ${\bf p}$ and to
        ${\bf r}(u)\,$: discard if these distances are unequal.
        The remaining rational arcs constitute the true bisector.
\end{enumerate}

\begin{rmk} \label{rmk:convex} {\rm
The true bisector of a point ${\bf p}$ and a (finite or
infinite) curve ${\bf r}(u)$ always encloses a convex,
simply--connected region of the plane, which may be of
finite or infinite area --- see, for example, Figure~\ref
{fig:etrim} below. This is because the region in question
may be regarded as the set--intersection of a continuous
family of half--planes ${\cal S}(u)$ where, for each $u$,
${\cal S}(u)$ contains the given point ${\bf p}$ and is
bounded by the perpendicular bisector of ${\bf p}$ and the
point ${\bf r}(u)$ of the curve. The set--intersection of
a family of half--planes always yields a convex set (see
\cite[Theorem~7 on p.~112]{kelly79}). }
\end{rmk}

In the case of a {\it finite\/} curve ${\bf r}(u)$, defined
on $u\in[\,a,b\,]$ say, it is necessary to ``complete'' the
untrimmed bisector \cite{farouki91} by incorporating the
appropriately directed tangent--extensions to ${\bf b}(u)$
at its end points $u=a$ and $u=b$. The intersections of these
tangent--extensions with ${\bf b}(u)$ for $u\in[\,a,b\,]$
and with each other must then be included in the trimming
process.

\subsection{An illustrative example}

To illustrate the above ideas in a concrete setting, we consider
the bisector of the point ${\bf p}=(\alpha,\beta)$ and the ellipse
\begin{equation} \label{pellipse}
X(u) \,=\, 1-u^2 \,, \quad
Y(u) \,=\, 2k\,u \,, \quad
W(u) \,=\, 1+u^2
\end{equation}
centered on the origin, with semi--axes 1 and $k$. From equations
(\ref{rbsctr}), we see that the untrimmed bisector is a rational
curve of degree {\it six}, defined by
\begin{eqnarray} \label{Bellps}
X_b(u) &\,=\,& (1-u^2)\, [\, k(\alpha^2+\beta^2-1)u^4
               \,+\, 4(1-k^2)\beta u^3 \nonumber \\
       &\,+\,& 2k(\alpha^2+\beta^2-3+2k^2)u^2 \,+\, 4(1-k^2)\beta u
               \,+\, k(\alpha^2+\beta^2-1) \,] \,, \nonumber \\
Y_b(u) &\,=\,& 2u\, [\, (\alpha^2+\beta^2+2(1-k^2)\alpha+1-2k^2)u^4
               \nonumber \\
       &\,+\,& 2(\alpha^2+\beta^2-1)u^2 \,+\, \alpha^2+\beta^2
               -2(1-k^2)\alpha+1-2k^2 \,] \,, \nonumber \\
W_b(u) &=& 2(1+u^2)^2\, [ -\,k(\alpha+1)u^2
                \,+\, 2\beta u \,+\, k(\alpha-1) \,] \,.
\end{eqnarray}
Examples of the curves defined by equation (\ref{Bellps}) are
illustrated in Figure~\ref{fig:euntrim}, for various locations of
the point ${\bf p}$.

\begin{figure}[htbp] \vspace{7.2in}
\caption[]{Examples of untrimmed point/ellipse bisectors \\ (also
shown are circles of curvature to the ellipse at its vertices).}
\label{fig:euntrim} \end{figure}

Up to a constant factor, the polynomial (\ref{Rinf}) in this case
reduces to
\begin{equation}
P_\infty(u) \,=\, (u^2+1)\,
[\,k(1+\alpha)\,u^2\,-\,2\beta\,u\,+\,k(1-\alpha)\,] \,,
\end{equation}
and there are evidently two real points at infinity whenever $\alpha^2+
(\beta/k)^2>1$, {\it i.e.}, ${\bf p}$ lies {\it outside\/} the ellipse.
They correspond to the parameter values
\begin{equation} \label{uinfellps}
u \,=\, {\beta \pm \displaystyle{\sqrt
{k^2\alpha^2+\beta^2-k^2}} \over k(\alpha+1)} \,.
\end{equation}

The cusps of the untrimmed bisector are roots of the polynomial
(\ref{Rcusp}). Substituting from (\ref{pellipse}), this becomes
\begin{eqnarray}
P_c(u) &\,=\,& k\,[\,\alpha^2+\beta^2+2(1-k^2)\alpha+1-2k^2\,]\,u^6
 \nonumber \\
&\,+\,& 3k\,[\,\alpha^2+\beta^2-2(1-k^2)\alpha-3+2k^2\,]\,u^4
 \nonumber \\
&\,+\,& 16\,(1-k^2)\beta\,u^3
 \nonumber \\
&\,+\,& 3k\,[\,\alpha^2+\beta^2+2(1-k^2)\alpha-3+2k^2\,]\,u^2
 \nonumber \\
&\,+\,& k\,[\,\alpha^2+\beta^2-2(1-k^2)\alpha+1-2k^2\,] \,,
\end{eqnarray}
and although this of degree six, it can be shown that there
are actually at most {\it four\/} distinct real roots. However,
the parameter values of the cusps do not (in general) admit a
closed--form expression in terms of radicals.

\begin{figure}[htbp] \vspace{7.2in}
\caption[]{The ``true'' (trimmed) point/ellipse bisectors \\
corresponding to the cases illustrated in Figure~\ref{fig:euntrim}.}
\label{fig:etrim} \end{figure}

Finally, the self--intersection polynomial $P_i(u)$ for the present
example transpires to be the product of two quartic factors
\begin{eqnarray} \label{Iellps}
P_{i,1}(u) &\,=\,&
(\alpha^2+\beta^2+2(1-k^2)\alpha+1-2k^2)u^4 \nonumber \\
&\,+\,& 2(\alpha^2+\beta^2-1)u^2 \,+\, \alpha^2+\beta^2
-2(1-k^2)\alpha+1-2k^2 \,, \\
P_{i,2}(u) &\,=\,&
k(\alpha^2+\beta^2-1)u^4 \,+\, 4(1-k^2)\beta u^3 \nonumber \\
&\,+\,& 2k(\alpha^2+\beta^2-3+2k^2)u^2
\,+\, 4(1-k^2)\beta u \,+\, k(\alpha^2+\beta^2-1) \,. \nonumber
\end{eqnarray}
A detailed analysis \cite{farouki91} reveals that $P_{i,1}(u)$
has real roots only when $P_{i,2}(u)$ has none, and vice--versa.
In particular, for $P_{i,1}(u)$ or $P_{i,2}(u)$ to have real roots
the quantity
\begin{equation} \label{discrim1}
\Delta \,=\, 4\,(1-k^2)(k^2-k^2\alpha^2-\beta^2)
\end{equation}
must be positive or negative, respectively. The number of these
roots is either zero, two, or four (corresponding to zero, one,
or two self--intersections) and is determined by the location of
${\bf p}$ with respect to the circles of curvature
\begin{eqnarray} \label{ecircles}
C_0(x,y) &\,=\,& (x-1+k^2)^2 + y^2 - k^4 \,=\, 0 \,, \nonumber \\
C_\infty(x,y) &\,=\,& (x+1-k^2)^2 + y^2 - k^4 \,=\, 0 \,, \nonumber \\
C_{+1}(x,y) &\,=\,& x^2 + (y^2-k+1/k)^2 - 1/k^2 \,=\, 0 \,, \nonumber \\
C_{-1}(x,y) &\,=\,& x^2 + (y^2+k-1/k)^2 - 1/k^2 \,=\, 0
\end{eqnarray}
of the ellipse at its vertices ($u=0,\infty$ and $u=\pm 1$).

When ${\bf p}$ is {\it inside\/} the ellipse, ${\bf b}(u)$
has two, one, or zero self--intersections, according to whether
${\bf p}$ lies inside neither, just one, or both of the circles
of curvature to the ellipse at the vertices on its {\it major\/}
axis (the last case being possible only if the circles of
curvature actually overlap). If ${\bf p}$ is {\it outside\/}
the ellipse, ${\bf b}(u)$ has zero, one, or two self--intersections
when ${\bf p}$ lies inside neither, just one, or both the circles
of curvature to the ellipse at the vertices on its {\it minor\/}
axis, respectively. In the latter case, of course, ${\bf b}(u)$
also has points at infinity identified by (\ref{uinfellps}).
This behavior is evident in Figure~\ref{fig:euntrim}.

When $\Delta>0$, the parameter values that identify self--intersections
of the untrimmed bisector are given explicitly by the formula
\begin{equation}
u \,=\, \pm \; \sqrt{ 1-\alpha^2-\beta^2 \,\pm\,
\sqrt{\Delta} \over \alpha^2+\beta^2+2(1-k^2)\alpha+1-2k^2} \;,
\end{equation}
which defines zero, one, or two pairs of real values of equal
magnitude and opposite sign, according to the criteria enumerated
above. If $\Delta<0$, the self--intersections can be found by
``completing the square'' in the polynomial $P_{i,2}(u)$. This
gives the two quadratic equations
\begin{equation} \label{I2new2}
k(\alpha^2+\beta^2-1)\,u^2 \,+\,
[\,2(1-k^2)\beta \pm \sqrt{-\Delta}\;]\,u \,+\,
k(\alpha^2+\beta^2-1) \,=\, 0
\end{equation}
whose real solutions are readily determined. Examples of true
point/ellipse bisectors obtained after trimming are shown in
Figure~\ref{fig:etrim}.

\subsection{Point/curve bisectors are antiorthotomics}
\label{antiorth}

Presently we shall make use of point/curve bisectors in
the construction of curve/curve bisectors. Before proceeding,
however, we briefly indulge in some diversionary remarks
concerning a connection --- brought to our attention by Peter
Giblin --- between (untrimmed) point/curve bisectors and a
geometrical optics problem involving smooth mirrors. The
reader whose curiosity is not sufficiently aroused can skip
this interlude with impunity.

Given a smooth curve ${\bf r}(t)$ and a point ${\bf p}$ not
on ${\bf r}(t)$, the {\it orthotomic\/} of ${\bf r}(t)$ with
respect to ${\bf p}$ is the envelope of the family of circles
that are centered on ${\bf r}(t)$ and pass through ${\bf p}$
(envelopes are discussed below in \S\ref{sec:env}). It can be
verified that this envelope comprises the parameterized locus
\begin{equation} \label{ortho}
{\bf p} \,+\,
2\,[\,({\bf r}(t)-{\bf p})\cdot{\bf n}(t)\,]\,{\bf n}(t) \,,
\end{equation}
where ${\bf n}(t)$ is the unit normal to ${\bf r}(t)$, together
with the point ${\bf p}$ itself --- which we discard (see \cite
{bruce84}).

Regarding ${\bf r}(t)$ as a mirror and ${\bf p}$ as a point
source, the optical significance of the orthotomic (\ref{ortho})
is that it describes a hypothetical ``initial'' wavefront which,
when propagated uniformly according to Huygens' principle in
the absence of a reflecting surface, assumes at each instant the
same shape as a spherical wave that was emitted by ${\bf p}$
and reflected at ${\bf r}(t)$. Thus, knowing the orthotomic
(\ref{ortho}) for any source/mirror configuration allows us to
compute the evolution of reflected wavefronts as a sequence of
{\it offset\/} or {\it parallel\/} curves to it. Beware that the
orthotomic goes by a variety of other names \cite{farouki92b}:
the {\it anticaustic\/} (Jakob Bernoulli); the {\it secondary
caustic\/}; and the ``privileged'' or ``archetypal'' wavefront
(corresponding to zero {\it optical path length\/}).

We now ``invert'' the reflection problem, {\it i.e.}, given
a point source ${\bf p}$ and an orthotomic ${\bf r}(t)$ whose
offsets represent desired wavefronts after reflection, we wish
to compute the necessary mirror shape. This is called the {\it
antiorthotomic\/} of ${\bf r}(t)$ with respect to ${\bf p}$,
since it corresponds to the locus of centers of the family of
circles that touch ${\bf r}(t)$ and pass through ${\bf p}$.
It can be shown \cite{bruce84} that the antiorthotomic has
the parameterization
\begin{equation} \label{antiortho}
{\bf r}(t) \,-\,
{|\,{\bf r}(t)-{\bf p}\,|^2 \over
2\,({\bf r}(t)-{\bf p})\cdot{\bf n}(t)} \, {\bf n}(t) \,,
\end{equation}
${\bf n}(t)$ being again the normal to ${\bf r}(t)$, and from
(\ref{varoffset}) and (\ref{du}) we immediately recognize this
as the untrimmed bisector of ${\bf p}$ and ${\bf r}(t)$.

We thus have the following interpretation: if ${\bf p}$ is
a light source, and the untrimmed bisector of ${\bf p}$ and
the curve ${\bf r}(t)$ is a mirror, then after reflection the
spherical waves emitted by ${\bf p}$ will coincide with the
offsets to ${\bf r}(t)$. For ``physical'' applications, of
course, all this will be subject to qualifications accounting
for obscuration, multiple reflections, etc.

\section{Curve/curve bisectors as envelopes}
\label{sec:env}

The bisector of two regular plane parametric curves, ${\bf r}(u)$
and ${\bf s}(v)$, is the point locus defined by
\begin{equation}
{\cal B} \,=\, \{\, {\bf q} \;|\;
{\rm dist}({\bf q},{\bf r}(u))={\rm dist}({\bf q},{\bf s}(v)) \,\} \,.
\end{equation}
Given a starting point ${\bf q}_0$ on ${\cal B}$ and the
corresponding parameter values $u_0$ and $v_0$ that identify the
feet of the perpendiculars from ${\bf q}_0$ to ${\bf r}(u)$ and
${\bf s}(v)$ at which ${\rm dist}({\bf q}_0,{\bf r}(u))$ and
${\rm dist}({\bf q}_0,{\bf s}(v))$ are realized, one can attempt
to trace ${\cal B}$ numerically by discrete steps. To first order,
the increment $\Delta {\bf q}$ to ${\bf q}$ at each step lies
along the tangent to ${\cal B}$, which bisects the angle between
the current perpendiculars from ${\bf q}$ to the two curves.

For the case of polynomial or rational curves, a sophisticated and
more efficient approach along these lines could invoke continuation
methods \cite{morgan87} to track the motion of the feet of the
perpendiculars from ${\bf q}$ to ${\bf r}(u)$ and ${\bf s}(v)$ as
${\bf q}$ traverses ${\cal B}$, rather than resorting to explicit
root--solving at each step.

However, to maintain a reasonable discretization error with
a first--order method would require minuscule steps, and the
formulation of second or higher--order expansions along ${\cal B}$
is not trivial. Furthermore, detecting and accommodating tangent
discontinuities along ${\cal B}$ due to sudden jumps in the feet
of the appropriate perpendiculars from ${\bf q}$ to ${\bf r}(u)$
and ${\bf s}(v)$ is a delicate task, fraught with difficulties
in the context of such numerical curve--tracing procedures
(similar problems arise at points where ${\bf r}(u)$ and
${\bf s}(v)$ cross).

\subsection{Envelopes of families of point/curve bisectors}

Let ${\cal C}(\lambda)$ be a one--parameter family of plane curves
that depend continuously on the variable $\lambda$. There are two
popular definitions for the ``envelope'' of such a family of curves
(see \cite[Chapter~4]{boltyanskii64} and \cite[Chapter~5]{fowler29}
--- also \cite{bruce81}):
\begin{itemize}
\item
The envelope ${\cal E}$ is a plane curve that is tangent, at
each of its points, \\ to {\it some\/} curve in the family
${\cal C}(\lambda)$.
\item
The envelope ${\cal E}$ is the locus, as $\lambda$ varies, of
the intersection points \\ of ``neighboring'' curves ${\cal C}
(\lambda)$ and ${\cal C}(\lambda+\Delta\lambda)$, in the limit
$\Delta\lambda \to 0$.
\end{itemize}
An alternate definition, requiring the introduction of an extra
dimension, offers an attractive geometric interpretation and is
given preference over the above in \cite[Chapter~5]{bruce84}:
\begin{itemize}
\item
If ${\cal S}$ is the surface obtained by ``stacking'' each of
the curves ${\cal C}(\lambda)$ at height $z=\lambda$ above the
$(x,y)$ plane, the envelope ${\cal E}$ is the ``critical set''
--- or {\it silhouette\/} --- of the projection of ${\cal S}$
onto the $(x,y)$ plane.
\end{itemize}

As is well--known, points of ${\cal S}$ that contribute to the
silhouette are those for which the surface normal is orthogonal
to the $z$--axis. We shall choose freely among these envelope
definitions to suit the circumstance, although the reader is
warned that they are not always exactly equivalent \cite
{bruce84}.

\begin{propn}
Let ${\bf b}_\lambda(u)$ denote the point/curve bisector for
the discrete point $v=\lambda$ on ${\bf s}(v)$ and the curve
${\bf r}(u)$ taken in its entirety. Then the bisector ${\cal B}$
of the two curves ${\bf r}(u)$ and ${\bf s}(v)$ is a subset of
the envelope ${\cal E}$ of the family ${\cal C}(\lambda)={\bf b}
_\lambda(u)$ of such point/curve bisectors.
\end{propn}

\noindent{{\bf Proof\ } (sketch):}
Let ${\cal C}(\lambda)$, ${\cal C}(\lambda+\Delta\lambda)$
be the bisectors of the discrete points ${\bf s}(\lambda)$,
${\bf s}(\lambda+\Delta\lambda)$ on ${\bf s}(v)$ and the entire
curve ${\bf r}(u)$, respectively. If ${\bf q}$ is a point
of intersection of ${\cal C}(\lambda)$ and ${\cal C}(\lambda+
\Delta\lambda)$, then by definition we have
\begin{equation}
|\,{\bf q}-{\bf s}(\lambda)\,| \,=\,
|\,{\bf q}-{\bf s}(\lambda+\Delta\lambda)\,| \,=\,
{\rm dist}({\bf q},{\bf r}(u)) \,.
\end{equation}
When $\Delta\lambda \to 0$ the above condition guarantees that,
as $\lambda$ varies, ${\bf q}$ moves so as to remain equidistant
from the entire curve ${\bf r}(u)$ and some neighborhood of
the point $v=\lambda$ on the curve ${\bf s}(v)$. The motion of
${\bf q}$ as $\lambda$ varies will then yield an arc of the true
bisector ${\cal B}$ of ${\bf r}(u)$ and ${\bf s}(v)$ if and
only if no other point of the curve ${\bf s}(v)$ is closer to
${\bf q}$ than ${\bf s}(\lambda)$, {\it i.e.}, $|\,{\bf q}-
{\bf s}(\lambda)\,|={\rm dist}({\bf q},{\bf s}(v))$. If the
intersection ${\bf q}$ is such that $|\,{\bf q}-{\bf s}(\lambda)
\,|>{\rm dist}({\bf q},{\bf s}(v))$, its motion yields an
``extraneous'' arc of the envelope ${\cal E}$ that does not
belong to ${\cal B}$.
\QED

Figure~\ref{fig:envelopes} illustrates the above ideas in
the context of two ellipses, where a family of point/curve
bisectors has been computed by taking a sequence of discrete
points along one of the ellipses. The envelope ${\cal E}$ stands
out as the locus along which the individual point/curve bisectors
``concentrate.'' Note also that, in both examples, there are
portions of the envelope that are clearly {\it not\/} segments
of the true bisector ${\cal B}$ of the two ellipses.

\begin{figure}[htbp] \vspace{7.2in}
\caption[]{Families of point/curve bisectors for two ellipses.}
%\caption[]{Examples of the bisector of two ellipses as
%the envelope of a family of point/ellipse bisectors.}
\label{fig:envelopes} \end{figure}

\subsection{Tracing the envelope}

We know that, for each $\lambda$, the bisector ${\bf b}_\lambda(u)$
of the point ${\bf s}(\lambda)$ and the curve ${\bf r}(u)$ consists
of a finite number of parametric subsegments of a single rational
curve. These arcs are connected at their endpoints so as to bound
a convex, simply--connected region. Mapping each arc of ${\bf b}_
\lambda(u)$ to the unit interval $u\in[\,0,1\,]$, we may represent
them as rational B\'ezier curves, and standard techniques \cite
{farin83} for the subdivision of their control polygons then yield
a convergent sequence of polygonal approximations to ${\bf b}_\lambda
(u)$. Intersecting the polygonal approximations to ${\bf b}_\lambda
(u)$ and ${\bf b}_{\lambda+\Delta\lambda}(u)$, for small increments
$\Delta\lambda$, offers a simple and relatively ``robust'' --- though
rather brutish --- means of approximating (a superset of) the
bisector ${\cal B}$ of ${\bf r}(u)$ and ${\bf s}(v)$.

However, with a little analysis we can do much better. The
derivation of envelope equations is usually discussed \cite
{boltyanskii64,bruce81,fowler29} in the context of families of
curves defined by an {\it implicit\/} (algebraic) equation
\begin{equation} \label{flambda}
f(x,y,\lambda) \,=\, 0 \,.
\end{equation}
A necessary condition for the point $(x,y)$ to lie on the envelope
${\cal E}$ of (\ref{flambda}) is that, for {\it some\/} $\lambda$, it
should satisfy
\begin{equation} \label{fsys}
f(x,y,\lambda) \,=\,
{\partial f \over \partial\lambda}(x,y,\lambda) \,=\, 0 \,.
\end{equation}
Thus, eliminating $\lambda$ between the two equations $f=0$
and $\partial f/\partial\lambda=0$, {\it i.e.}, computing their
``resultant'' \cite{uspensky48} with respect to $\lambda\,$:
\begin{equation} \label{e}
e(x,y) \,=\, {\rm Resultant}_\lambda
\left(f,{\partial f \over \partial\lambda}\,\right) \,,
\end{equation}
defines an algebraic curve $e(x,y)=0$ that contains the envelope
${\cal E}$. However, factors extraneous to the envelope can arise
under various circumstances. For example, if (\ref{flambda})
degenerates such that $f(x,y,\lambda_0) \equiv 0$ at $\lambda
=\lambda_0$, then $\partial f(x,y,\lambda_0)/\partial\lambda$
will appear as a factor in $e(x,y)$ --- and vice--versa. Similarly,
if $f$ and $\partial f/\partial\lambda$ have a non--constant common
factor $\gamma(x,y)$ at $\lambda=\lambda_0$, this will also appear
in $e(x,y)$. Finally, if (\ref{flambda}) exhibits a {\it locus of
singular points\/} ({\it i.e.}, a curve $\sigma(x,y)=0$ such that,
as $\lambda$ varies, $f=\partial f/\partial x=\partial f/\partial
y=0$ at each point of this curve) then $\sigma(x,y)$ will be a
component of $e(x,y)$.

Equation (\ref{fsys}) indicates that the discrete points that
each curve $\lambda$ of the family contributes to the envelope
can be regarded as the intersections of $f(x,y,\lambda)=0$ with
the curve $\partial f(x,y,\lambda)/\partial\lambda=0$. The above
approach could, in principle, be applied to the envelopes of
families of point/curve bisectors, by first ``implicitizing''
\cite{sederberg84} the rational parametric form ${\bf b}_\lambda
(u)$ (of course, the resulting equation $b(x,y,\lambda)=0$ would
represent the {\it untrimmed\/} bisectors). However, even in simple
cases the envelope equations $e(x,y)=0$ generated by this approach
are unwieldy and of little practical use.

We prefer to work directly with the parametric representation
${\bf b}_\lambda(u)$, and we now develop a criterion analogous
to (\ref{fsys}) that is appropriate to this form.

\newpage

\begin{thm}
Let ${\bf b}_\lambda(u)$ be the bisector of the point $v=\lambda$
on ${\bf s}(v)$ and the curve ${\bf r}(u)$. Then, for each $\lambda$,
the smooth points of ${\bf b}_\lambda(u)$ that
contribute to the envelope ${\cal E}$ of the family of point/curve
bisectors ${\cal C}(\lambda)={\bf b}_\lambda(u)$ are identified by
the parameter values $u$ for which
\begin{equation} \label{blambda}
{\partial{\bf b}_\lambda \over \partial\lambda}(u) \,=\, {\bf 0} \,.
\end{equation}
\end{thm}

\noindent{{\bf Proof\ } (sketch):}
The intersections of ``neighboring'' members $\lambda$ and
$\lambda+\Delta\lambda$ of the family of point/curve bisectors
can be identified with parameter values $u$ for which there
exist increments $\Delta u$ such that
\begin{equation} \label{intrsct}
{\bf b}_\lambda(u) \,=\,
{\bf b}_{\lambda+\Delta\lambda}(u+\Delta u) \,.
\end{equation}
Assuming ${\bf b}_\lambda(u)$ is a smooth point ({\it i.e.}, not
the juncture of two distinct arcs belonging to the point/curve
bisector) we can expand the above to obtain
\begin{equation} \label{beqn}
{\bf b}'_\lambda(u)\,\Delta u \,+\,
{\partial {\bf b}_\lambda \over \partial\lambda}(u)\,\Delta\lambda
\,+\, \;\cdots\; \,=\, {\bf 0} \,,
\end{equation}
where, as usual, primes denote differentiation with respect to
$u$. As this is a {\it vector\/} equation, it can be satisfied
(to first order) only when the derivatives ${\bf b}'_\lambda$ and
$\partial{\bf b}_\lambda/\partial\lambda$ are parallel --- or one
of them vanishes. Writing
\begin{equation}
{\bf b}_\lambda(u) \,=\, {\bf r}(u) + d_\lambda(u){\bf n}(u) \,,
\end{equation}
where $d_\lambda(u)$ is defined by substituting ${\bf s}(\lambda)$
for ${\bf p}$ in (\ref{du}), we differentiate the above and invoke
the Frenet relations \cite{kreyszig59} to obtain
\begin{equation} \label{bderivs}
{\bf b}'_\lambda \,=\, |{\bf r}'|\,
(1+\kappa d_\lambda)\,{\bf t}
\,+\, d^{\,\prime}_\lambda\,{\bf n}
\qquad {\rm and} \qquad
{\partial {\bf b}_\lambda \over \partial\lambda} \,=\,
{\partial d_\lambda \over \partial\lambda}\,{\bf n} \,.
\end{equation}
Here ${\bf t}$, ${\bf n}$, and $\kappa$ are the tangent, normal,
and curvature of ${\bf r}(u)$, and $|{\bf r}'|\not=0$ for a
regular curve. Thus we must have either $\kappa=-1/d_\lambda$
or $|d^{\,\prime}_\lambda|\to\infty$ for the derivatives (\ref
{bderivs}) to be parallel if they are non--zero. But the first
condition identifies {\it cusps\/} on the untrimmed bisector of
${\bf s}(\lambda)$ and ${\bf r}(u)$, which cannot belong to the
true bisector (the latter must be convex). Similarly, the second
condition is discounted since it identifies {\it points at
infinity\/} on ${\bf b}_\lambda(u)$.

Hence, affine smooth points of ${\bf b}_\lambda(u)$ that contribute
to the envelope must satisfy ${\bf b}'_\lambda={\bf 0}$ or $\partial
{\bf b}_\lambda/\partial\lambda={\bf 0}$. Again, the first case can
be dropped since it identifies cusps, and we are left with the stated
condition (\ref{blambda}). Equation (\ref{beqn}) then gives $\Delta
u=0$, {\it i.e.}, when (\ref{blambda}) is satisfied at $u=u_\ast$,
say, then to first order the intersection of the neighboring members
${\bf b}_\lambda(u)$ and ${\bf b}_{\lambda+\Delta\lambda}(u)$
corresponds to the {\it same\/} parameter value $u_\ast$ on these
two curves.
\QED

\begin{rmk} \label{penvelope} {\rm
The derivation of (\ref{blambda}) was specific to families of
point/curve bisectors. However, for {\it any\/} family ${\bf r}
(t,\xi)=\{x(t,\xi),y(t,\xi)\}$ of smooth plane parametric curves,
it is true that a point will belong to the envelope if the
derivatives $\partial{\bf r}/\partial t$ and $\partial{\bf r}
/\partial\xi$ with respect to the curve parameter $t$ and the
family parameter $\xi$ are parallel there. To see this, we
introduce the surface ${\bf S}(t,\xi)=\{x(t,\xi),y(t,\xi),\xi\}$
and employ the third envelope definition given above, instead
of relying on the expansion (\ref{beqn}) of equation (\ref
{intrsct}). We have $\partial{\bf S}/\partial t=(\partial x
/\partial t,\partial y/\partial t,0)$ and $\partial{\bf S}/
\partial\xi=(\partial x/\partial\xi,\partial y/\partial\xi,1)$
and the normal to ${\bf S}(t,\xi)$ is thus in the direction of
\begin{equation} \label{normal}
{\partial{\bf S} \over \partial t} \times
{\partial{\bf S} \over \partial\xi} \,=\,
\left({\partial y \over \partial t}\,,\;
-\,{\partial x \over \partial t}\,,\;
{\partial x \over \partial t}{\partial y \over \partial\xi}-
{\partial y \over \partial t}{\partial x \over \partial\xi}
\right) \,.
\end{equation}
The $z$--component of (\ref{normal}) must vanish at silhouette
points, and this occurs only when $\partial{\bf r}/\partial t$
and $\partial{\bf r}/\partial\xi$ are parallel. Because of the
particular form of the derivatives (\ref{bderivs}), this reduces
to (\ref{blambda}) in the case of point/curve bisectors.
}
\end{rmk}

Note that, unlike the numerical schemes mentioned above, condition
(\ref{blambda}) gives points that lie {\it exactly\/} on the envelope
${\cal E}$ of the family ${\cal C}(\lambda)={\bf b}_\lambda(u)$. From
(\ref{bderivs}) we see that condition (\ref{blambda}) amounts to the
requirement that the partial derivative of the displacement function
\begin{equation} \label{dlambdau}
d_\lambda(u) \,=\,
{|\,{\bf s}(\lambda)-{\bf r}(u)\,|^2 \over
2\,({\bf s}(\lambda)-{\bf r}(u))\cdot{\bf n}(u)}
\end{equation}
with respect to $\lambda$ should vanish. This occurs when
\begin{equation} \label{ueqn}
2\,({\bf s}(\lambda)-{\bf r}(u))\cdot{\bf n}(u) \,\,
   ({\bf s}(\lambda)-{\bf r}(u))\cdot
               {\partial{\bf s} \over \partial\lambda} \,=\,
|\,{\bf s}(\lambda)-{\bf r}(u)\,|^2 \,
{\bf n}(u)\cdot{\partial{\bf s} \over \partial\lambda}
\end{equation}
which, for each $\lambda$, amounts to a polynomial equation in
$u$ whose real roots on the domain of definition of the point/curve
bisector ${\bf b}_\lambda(u)$ are desired. We now give a geometric
interpretation of the above equation:

\begin{propn} \label{propn:normal}
For each $\lambda$, the roots of equation $(\ref{ueqn})$ identify
the points of ${\bf b}_\lambda(u)$ that lie on the normal line to
the curve ${\bf s}(v)$ at $v=\lambda$.
\end{propn}

\prf
We divide (\ref{ueqn}) through by $2\,({\bf s}(\lambda)-{\bf r}(u))
\cdot{\bf n}(u)$ and re--write it as
\begin{equation}
\left[\; {|\,{\bf s}(\lambda)-{\bf r}(u)\,|^2 \over
2\,({\bf s}(\lambda)-{\bf r}(u))\cdot{\bf n}(u)}\,{\bf n}(u)
\,-\, ({\bf s}(\lambda)-{\bf r}(u)) \;\right] \!\cdot
{\partial{\bf s} \over \partial\lambda} \,=\, 0 \;.
\end{equation}
The scalar quantity multiplying ${\bf n}(u)$ above is just the
displacement function $d_\lambda(u)$ given by (\ref{dlambdau}),
and since ${\bf b}_\lambda(u)={\bf r}(u)+d_\lambda(u){\bf n}(u)$
we obtain
\begin{equation}
\left[\,{\bf b}_\lambda(u)-{\bf s}(\lambda)\,\right] \cdot
{\partial{\bf s} \over \partial\lambda} \,=\, 0 \,.
\end{equation}
Now $\partial{\bf s}/\partial\lambda\;(\not={\bf 0})$ gives the
tangent direction to the (regular) curve ${\bf s}(v)$ at $v=\lambda$,
so the value of $u$ must be such that the point ${\bf b}_\lambda(u)$
lies on the normal line to ${\bf s}(v)$ at $v=\lambda$ to satisfy
the above.
\QED

Thus, Proposition~\ref{propn:normal} allows to easily pick off
the (smooth) points on each member of the family ${\cal C}(\lambda)
={\bf b}_\lambda(u)$ of point/curve bisectors that contribute to
the envelope ${\cal E}$ of this family: simply intersect ${\bf b}
_\lambda(u)$ with the normal line to ${\bf s}(v)$ at $v=\lambda$. By
the fact that each point/curve bisector is convex (see Remark~\ref
{rmk:convex}), ${\bf b}_\lambda(u)$ contributes at most two affine
points to ${\cal E}$ for each $\lambda$.

\begin{rmk} {\rm
If we could write the roots of (\ref{ueqn}) as {\it explicit\/}
functions $u(\lambda)$ of $\lambda$, these could be substituted
into the point/curve bisector equation to give a (local)
parameterization ${\bf b}_\lambda(u(\lambda))$ of the bisector
${\cal B}$ of ${\bf r}(u)$ and ${\bf s}(v)$. However, this is
impossible even in the simplest non--trivial cases. If ${\bf r}
(u)$ is a polynomial or rational curve of degree $\ge 2$, for
example, equation (\ref{ueqn}) is, respectively, of degree $3n-1
\ge 5$ or $4n-2 \ge 6$ in $u$, and cannot be solved in terms
of radicals. Thus, the curve/curve bisector ${\cal B}$ has
no ``simple'' parameterization, and must be traced by some
numerical scheme. }
\end{rmk}

Finally, we must address the possible contributions of
non--smooth points on the individual point/curve bisectors
${\bf b}_\lambda(u)$ to their overall envelope. These are
the endpoints of the various parametric subsegments that
constitute the true point/curve bisector, which will generally
meet with only $C^0$ continuity. A simple example illustrates
that one cannot afford to simply ignore such points: when a
polygon moves in the plane, the envelope that bounds the area
it sweeps out is generated {\it entirely\/} by non--smooth
points (the vertices) if the sense of motion is never parallel
to any side of the polygon.

We will content ourselves here with the observation that, for
each $\lambda$, one could simply include {\it all\/} non--smooth
points of ${\bf b}_\lambda(u)$ as candidate points for the
curve/curve bisector ${\cal B}$. Typically, there are only a
few such points, and their location is known without further
calculation. Since the smooth points of each ${\bf b}_\lambda(u)$
that contribute to their envelope ${\cal E}$ must be validated
for membership in ${\cal B}$, by testing for equality of their
distances from ${\bf r}(u)$ and ${\bf s}(v)$, one might as well
test all the non--smooth points too.

\begin{figure}[htbp] \vspace{7.2in}
\caption[]{Sampling of curve/curve bisectors.}
\label{fig:etrace} \end{figure}

Figure~\ref{fig:etrace} shows examples of curve/curve bisectors,
for the case of two ellipses, that have been sampled using the
method described above (points of the envelope that do not
belong to the true bisector were suppressed by making distance
comparisons). Note that the bisector has two branches passing
through any intersection of the given curves; the bisector
tangents at such points bisect the angles between the tangents
to the given curves.

\newpage

\section{Concluding remarks}

We now conclude by briefly summarizing all the above results:
\begin{itemize}
\item
The bisector ${\bf b}(u)$ of a point ${\bf p}$ and a polynomial
or rational parametric curve ${\bf r}(u)$ of degree $n$ consists
of a finite number of subsegments of a single rational curve
of degree $3n-1$ or $4n-2$, respectively, that bound a convex,
simply--connected region of the plane. Given a robust polynomial
root--finder, it can be computed in an algorithmic manner.
\item
The bisector ${\cal B}$ of two parametric curves ${\bf r}(u)$
and ${\bf s}(v)$ is a subset of the envelope ${\cal E}$ of
the family of point/curve bisectors ${\bf b}_\lambda(u)$ that
correspond to the discrete point $v=\lambda$ on ${\bf s}(v)$
and the entire curve ${\bf r}(u)$.
\item
The smooth points of each ${\bf b}_\lambda(u)$ that lie on
the envelope ${\cal E}$ are identified by the intersections of
the normal line to ${\bf s}(v)$ at $v=\lambda$ with ${\bf b}_
\lambda(u)$. All non--smooth points of each ${\bf b}_\lambda
(u)$ can be regarded as candidates for ${\cal E}$. Increasing
$\lambda$ by small increments, we can test these points for
membership in ${\cal B}\subset{\cal E}$ by comparing their
distances to ${\bf r}(u)$ and ${\bf s}(v)$, and thus trace
out the bisector of the two curves.
\end{itemize}

Some refinements to the method outlined above are needed to
yield a practical scheme for tracing curve/curve bisectors.
First, if we systematically increase $\lambda$, the envelope
points are not generated with proper ``ordering'' along the
bisector. Thus, a method for sorting them {\it a posteriori\/}
or detecting breakdown of ordering during run--time is
required. Second, the spacing of the envelope points can
be very uneven if uniform increments in $\lambda$ are used.
A scheme that adaptively selects the $\lambda$ increments
to give a more even spacing is desirable. We hope to address
these problems in due course.

\begin{ackn}
We are grateful to Peter Giblin for pointing out that
point/curve bisectors are antiorthotomics (see \S\ref{antiorth})
and that condition (\ref{blambda}) is a special case of the
general criterion given in Remark~\ref{penvelope} that identifies
points on the envelopes of families of parametric curves.
\end{ackn}

\begin{thebibliography}{99}

\bibitem{banchoff87}
T.~F.~Banchoff and P.~J.~Giblin, 1987: Global theorems for
symmetry sets of smooth curves and polygons in the plane,
{\it Proc.\ Roy.\ Soc.\ Edinburgh\/} {\bf 106A}, 221--231.

\bibitem{blum67}
H.~Blum, 1967: A transformation for extracting new descriptors
of shape, in {\it Models for the Perception of Speech and Visual
Form\/} (W.~Wathen--Dunn, ed.), MIT Press, Cambridge MA, 362--380.

\bibitem{blum73}
H.~Blum, 1973: Biological shape and visual science (Part I),
{\it J.\ Theoret.\ Biol.} {\bf 38}, 205--287.

\bibitem{blum78}
H.~Blum and R.~N.~Nagel, 1978: Shape description using weighted
symmetric axis features, {\it Pattern Recognition\/} {\bf 10},
167--180.

\bibitem{boltyanskii64}
V.~G.~Boltyanskii, 1964: {\it Envelopes}, MacMillan, New York.

\bibitem{bookstein79}
F.~L.~Bookstein, 1979: The line--skeleton, {\it Comput.\ Graphics\
Image\ Proc.} {\bf 11}, 123--137.

\bibitem{bruce81}
J.~W.~Bruce and P.~J.~Giblin, 1981: What is an envelope? {\it
Math.\ Gazette\/} {\bf 65}, 186--192.

\bibitem{bruce84}
J.~W.~Bruce and P.~J.~Giblin, 1984: {\em Curves and Singularities},
Cambridge University Press.

\bibitem{bruce86}
J.~W.~Bruce and P.~J.~Giblin, 1986: Growth, motion and 1--parameter
families of symmetry sets, {\it Proc.\ Roy.\ Soc.\ Edinburgh\/}
{\bf 104A}, 179--204.

\bibitem{bruce85}
J.~W.~Bruce, P.~J.~Giblin, and C.~G.~Gibson 1985: Symmetry sets,
{\it Proc.\ Roy.\ Soc.\ Edinburgh\/} {\bf 101A}, 163--186.

\bibitem{coxeter69}
H.~S.~M.~Coxeter, 1969: {\it Introduction to Geometry}, Wiley,
New York.

\bibitem{farin83}
G.~Farin, 1983: Algorithms for rational B\'ezier curves, {\it Comput.\
Aided Design\/} {\bf 15}, 73--77.

\bibitem{farouki91a}
R.~T.~Farouki, 1991: Computing with barycentric polynomials, {\it Math.\
Intelligencer\/} {\bf 13}, 61--69.

\bibitem{farouki92}
\ldash\,, 1992: Watch your (parametric) speed!, in {\it The Mathematics
of Surfaces IV\/} (A.~Bowyer, ed.), Oxford University Press, to appear.

\bibitem{farouki92b}
R.~T.~Farouki and J.--C.~A.~Chastang, 1992: Curves and surfaces
in geometrical optics, in {\it Mathematical Methods in CAGD II\/}
(T.~Lyche and L.~L.~Schumaker, eds.), Academic Press, Boston.

\bibitem{farouki91}
R.~T.~Farouki and J.~K.~Johnstone, 1991: The bisector of a point and
a plane parametric curve, {\it IBM Research Report RC17381}.

\bibitem{farouki90a}
R.~T.~Farouki and C.~A.~Neff, 1990: Analytic properties of plane
offset curves, {\it Comput.\ Aided Geom.\ Design\/} {\bf 7}, 83--99.

\bibitem{farouki90b}
\ldash\,, 1990: Algebraic properties of plane offset curves,
{\it Comput.\ Aided Geom.\ Design\/} {\bf 7}, 101--127.

\bibitem{farouki87}
R.~T.~Farouki and V.~T.~Rajan, 1987: On the numerical condition of
polynomials in Bernstein form, {\it Comput.\ Aided Geom.\ Design\/}
{\bf 4}, 191--216.

\bibitem{farouki88}
\ldash\,, 1988: Algorithms for polynomials in Bernstein form,
{\it Comput.\ Aided Geom.\ Design\/} {\bf 5}, 1--26.

\bibitem{field92}
D.~A.~Field and R.~Field, 1992: A new family of curves for industrial
applications, {\it GM Research Publication GMR--7571}.

\bibitem{fowler29}
R.~H.~Fowler, 1929: {\it The Elementary Differential Geometry of
Plane Curves}, Cambridge Univ.\ Press.

\bibitem{giblin85}
P.~J.~Giblin and S.~A.~Brassett, 1985: Local symmetry of plane curves,
{\it Amer.\ Math.\ Monthly\/} {\bf 92}, 689--707.

\bibitem{held91}
M.~Held, 1991: {\it On the Computational Geometry of Pocket Machining},
Springer--Verlag, Berlin.

\bibitem{H52}
D.~Hilbert and S.~Cohn--Vossen, 1952: {\it Geometry and the Imagination},
Chelsea, New York.

\bibitem{kelly79}
P.~J.~Kelly and M.~L.~Weiss, 1979: {\it Geometry and Convexity}, Wiley,
New York.

\bibitem{kreyszig59}
E.~Kreyszig, 1959: {\it Differential Geometry}, University of Toronto
Press.

\bibitem{lee82}
D.~T.~Lee, 1982: Medial axis transformation of a planar shape, {\it IEEE
Trans.\ Pattern Anal.\ Machine Intell.} PAMI--{\bf 4}, 363--369.

\bibitem{morgan87}
A.~Morgan, 1987: {\it Solving polynomial systems using continuation
for engineering and scientific problems}, Prentice--Hall, Englewood
Cliffs, NJ.

\bibitem{primrose55}
E.~J.~F.~Primrose, 1955: {\it Plane Algebraic Curves}, MacMillan, London.

\bibitem{sederberg84}
T.~W.~Sederberg, D.~C.~Anderson, and R.~N.~Goldman, 1984: Implicit
\linebreak representation of parametric curves and surfaces, {\it
Comput.\ Vision Graphics Image Proc.} {\bf 28}, 72--84.

\bibitem{uspensky48}
J.~V.~Uspensky, 1948: {\it Theory of Equations}, McGraw--Hill,
New York.

\bibitem{wolter85}
F.--E.~Wolter, 1985: Cut loci in bordered and unbordered
Riemannian manifolds, PhD thesis, Technische Universit\"at Berlin.

\bibitem{wolter92}
F.--E.~Wolter, 1992: Cut locus and medial axis in global shape
interrogation and representation, MIT Dept.\ of Ocean Engineering,
Design Laboratory Memorandum 92--2.

\bibitem{yap87}
C.--K.~Yap, 1987: An ${\rm O}(n\log n)$ algorithm for the Voronoi
diagram of a set of simple curve segments, {\it Discrete Comput.\
Geom.} {\bf 2}, 365--393.

\end{thebibliography}

\end{document}


