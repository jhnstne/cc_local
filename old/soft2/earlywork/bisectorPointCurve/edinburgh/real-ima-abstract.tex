% Rida's version
\documentstyle[12pt]{article}

\begin{document}

\title{
Computing the bisector of a point \\
and a plane parametric curve
}

\author{
Rida~T.~Farouki \\
IBM Thomas~J.~Watson Research Center, \\
P.~O.~Box 218, Yorktown Heights, NY 10598. \\ \\
John~K.~Johnstone \\
Department of Computer Science, \\
The Johns Hopkins University, Baltimore, MD 21218.
}

\date{}

\maketitle
\thispagestyle{empty}

\begin{abstract}
The {\it bisector\/} of a fixed point ${\bf p}$ and a
smooth plane curve $C$ ({\it i.e.}, the locus traced by a
point that moves so as to remain equidistant with respect
to ${\bf p}$ and $C$) is investigated in the case that $C$
admits a regular polynomial or rational parameterization.
It is shown that the bisector may be regarded as (a subset
of) a ``variable--distance'' offset curve to $C$, which has
the attractive property --- unlike fixed--distance offsets
--- of being {\it generically\/} a rational curve. In its
entirety, this rational curve (the ``untrimmed'' point/curve
bisector) usually exhibits self--intersections and irregular
points, similar in nature to those seen on fixed--distance
offsets. The conditions that give rise to such points are
described, and a {\it trimming procedure} is formulated,
whereby those parametric subsegments of the untrimmed
bisector that constitute the ``true'' bisector may be
identified. For polynomial and rational curves of degree
$n$, the corresponding untrimmed point/curve bisectors are
rational curves of degree $3n-1$ and $4n-2$, respectively
({\it e.g.}, degree 6 for an ellipse/hyperbola, and degree
8 for ordinary cubics).
\end{abstract}

\end{document}

