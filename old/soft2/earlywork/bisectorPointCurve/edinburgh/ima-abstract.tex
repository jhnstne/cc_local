\documentstyle[12pt,titlepage]{article}
\begin{document}

\newcommand{\ACMTOG}{{\sl ACM Trans.\ Graph.\ }}
\newcommand{\AMM}{{\sl Amer.\ Math.\ Monthly\ }}
\newcommand{\BIT}{{\sl BIT\ }}
\newcommand{\CACM}{{\sl Commun.\ ACM\ }}
\newcommand{\CAD}{{\sl Comput.\ Aided Design\ }}
\newcommand{\CAGD}{{\sl Comput.\ Aided Geom.\ Design\ }}
\newcommand{\CGIP}{{\sl Comput.\ Graph.\ Image Proc.\ }}
\newcommand{\CJ}{{\sl Computer\ J.\ }}
\newcommand{\DCG}{{\sl Discrete\ Comput.\ Geom.\ }}
\newcommand{\IBMJRD}{{\sl IBM\ J.\ Res.\ Develop.\ }}
\newcommand{\IJCGA}{{\sl Int.\ J.\ Comput.\ Geom.\ Applic.\ }}
\newcommand{\IEEECGA}{{\sl IEEE Comput.\ Graph.\ Applic.\ }}
\newcommand{\IEEETPAMI}{{\sl IEEE Trans.\ Pattern Anal.\ Machine Intell.\ }}
\newcommand{\JACM}{{\sl J.\ Assoc.\ Comput.\ Mach.\ }}
\newcommand{\JAT}{{\sl J.\ Approx.\ Theory\ }}
\newcommand{\MC}{{\sl Math.\ Comp.\ }}
\newcommand{\MI}{{\sl Math.\ Intelligencer\ }}
\newcommand{\NM}{{\sl Numer.\ Math.\ }}
\newcommand{\SIAMJNA}{{\sl SIAM J.\ Numer.\ Anal.\ }}
\newcommand{\SIAMR}{{\sl SIAM Review\ }}

\title{
Computing the bisector \\
of a point and a plane curve
}

\author{
Rida~T.~Farouki \\
IBM Thomas~J.~Watson Research Center, \\
P.~O.~Box 218, Yorktown Heights, NY 10598. \\
farouki@watson.ibm.com \\ \\
John~K.~Johnstone \\
Department of Computer Science, \\
The Johns Hopkins University, Baltimore, MD 21218. \\
jj@cs.jhu.edu
}

\date{}

\maketitle

In this paper, 
we investigate the bisector of a fixed point ${\bf p}$ and a smooth
plane curve $C$ --- {\it i.e.}, the locus traced by a point
that remains equidistant with respect to ${\bf p}$ and $C$ ---
in the case that $C$ admits a regular polynomial or rational parameterization. 
Our method also applies to finite segments of plane curves.

The bisector is a basic component of a geometric vocabulary
because of the importance of distance in geometrical applications.
Point/curve bisectors arise in a variety of geometric ``reasoning'' 
and geometric
decomposition problems ({\it e.g.}, planning paths of maximum
clearance in robotics, or computing Voronoi diagrams for areas
with curvilinear boundaries). Although they are much simpler
than other loci --- line/curve and curve/curve bisectors ---
that arise in these contexts, no systematic analysis of the
properties of point/curve bisectors is currently available
in the literature.
Related work on the bisector has been done by Hoffmann and Vermeer \cite{HV91},
Hoffmann \cite{H90}, Dutta and Hoffmann \cite{DH90},
Bookstein \cite{bookstein79}, Lee \cite{lee82},
Yap \cite{yap87}, Yap and Alt \cite{yap89}, and
Nackman and Srinivasan \cite{nackman91}.

The parabola may be regarded as the bisector of a point ${\bf p}$ and
a line $L$.
If we substitute a smooth plane curve $C$ in place of
the straight line $L$, the bisector locus is of a more
subtle nature. This paper is concerned with investigating
the geometric properties of such loci, and formulating
tractable representations for them. 

It is shown that the bisector may be
regarded as (a subset of) a ``variable--distance'' offset curve to
$C$ which has the attractive property, unlike fixed--distance offsets,
of being generically a rational curve. 
A trimming procedure, which identifies the parametric subsegments
of this curve that constitute the true bisector, is described.
It turns out that this trimming should be done
at parameter values associated with inflection points of $C$,
class points of $C$ (points whose tangents pass through ${\bf p}$),
and circular points of $C$ (points whose circles of curvature
pass through ${\bf p}$), as well as at 
self-intersections of the untrimmed bisector,
in order to reveal the true bisector.
We give explicit polynomials whose solution identifies the parameter
values of these special points,
and make some observations about the position of self-intersections.

We show that the time to compute the bisector of a point and a 
regular polynomial parametric curve of degree $n$ is 
$O(9n^{2}) \varphi(2n-1) + \varphi(O(9n^2)) + O(9n^3)$, where 
$\varphi(n)$ is the time to solve an equation of degree $n$.

\clearpage

\begin{thebibliography}{99}

\bibitem{bookstein79}
F.~L.~Bookstein (1979), The line--skeleton, \CGIP{\bf 11}, 123--137.

\bibitem{DH90}
D.~Dutta and C.~M.~Hoffmann (1990), A geometric investigation of the
skeleton of CSG objects, {\it Technical Report UM--MEAM--90--02},
Dept.\ of Mechanical Engineering, The University of Michigan.

\bibitem{H90}
C.~M.~Hoffmann (1990), A dimensionality paradigm for surface interrogations,
\CAGD{\bf 7}, 517--532.

\bibitem{HV91}
C.~M.~Hoffmann and P.~Vermeer (1991), Eliminating extraneous solutions in
curve and surface operations, \IJCGA{\bf 1}, 47--66.

\bibitem{lee82}
D.~T.~Lee (1982), Medial axis transformation of a planar shape,
\IEEETPAMI{PAMI--{\bf 4}}, 363--369.

\bibitem{nackman91}
L.~R.~Nackman and V.~Srinivasan (1991), Bisectors of linearly separable
sets, \DCG{\bf 6}, 263--275.

\bibitem{yap87}
C.--K.~Yap (1987), An ${\rm O}(n\log n)$ algorithm for the Voronoi
diagram of a set of simple curve segments, \DCG{\bf 2}, 365--393.

\bibitem{yap89}
C.--K.~Yap and H.~Alt (1989), Motion planning in the {\it
CL\/}--Environment, in {\it Lecture notes in computer science 382\/}
(F.~Dehne, J.--R.~Sack, and N.~Santoro, eds.), Springer, New York,
373--380 (Proceedings of the Workshop on Algorithms and Data
Structures WADS~'89, Ottawa, Canada, August 17--19, 1989).

\end{thebibliography}

\end{document}