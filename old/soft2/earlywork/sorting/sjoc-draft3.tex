\documentstyle{siam} 

\def\notation{\par{\it Notation}. \ignorespaces}
% \def\@beginexample#1#2{\par\bgroup{\it #1\ #2. }\rm\ignorespaces}
% \def\@opargbeginexample#1#2#3{\par\bgroup{\it #1\ #2\ (#3). }\rm\ignorespaces}
% \def\endexample{{\ \vbox{\hrule\hbox{%
%    \vrule height1.3ex\hskip0.8ex\vrule}\hrule
%   }}\par}
% \newtheorem{example}{Example}[section]
\newcounter{examplectr}[section]
\def\example{\addtocounter{examplectr}{1}\par
	{\it Example~\arabic{section}.\arabic{examplectr}}. \ignorespaces}
\newcommand{\Comment}[1]{\relax}  % makes a "comment" (not expanded)
\def\definition{\par{\it Definition}. \ignorespaces}
\newcommand{\arc}[1]{\mbox{$\stackrel{\frown}{#1}$}}
\newcommand{\lyne}[1]{\mbox{$\stackrel{\leftrightarrow}{#1}$}}
\newcommand{\ray}[1]{\mbox{$\vec{#1}$}}
\newcommand{\seg}[1]{\mbox{$\overline{#1}$}}
\newcommand{\tab}{\hspace*{.2in}}
\newcommand{\se}{\mbox{$_{\epsilon}$}}  % subscript epsilon
\newcommand{\wrt}{\mbox{w.\ r.\ t.\ }}
\newcommand{\ie}{\mbox{i.e.}}
\newcommand{\eg}{\mbox{e.\ g.\ }}
\newcommand{\pq}{\arc{PQ}}
\newcommand{\pqi}{\mbox{$\pq_{in}$}}
\newcommand{\tx}{\mbox{$T_{x}$}}
\newcommand{\e}{\mbox{$\epsilon$}}
\newcommand{\eo}{\mbox{$\epsilon_{1}$}}   % note:\e1 gets confused with \e
\newcommand{\et}{\mbox{$\epsilon_{2}$}}   % \e2 gets confused with \e
\newcommand{\xeo}{\mbox{$x_{\epsilon,1}$}}
\newcommand{\xet}{\mbox{$x_{\epsilon,2}$}}
\newcommand{\xya}{\arc{xy}}
\newcommand{\lep}{\mbox{$L_{\epsilon}$}}  % note: \le is already taken
\newcommand{\xpe}{\mbox{$x_{+\epsilon}$}}
\newcommand{\xme}{\mbox{$x_{-\epsilon}$}}
\newcommand{\ye}{\mbox{$y_{\epsilon}$}}
\newcommand{\xo}{\mbox{$x_{1}$}}
\newcommand{\xt}{\mbox{$x_{2}$}}
\newcommand{\xth}{\mbox{$x_{3}$}}
\newcommand{\yo}{\mbox{$y_{1}$}}
\newcommand{\xotha}{\arc{\xo\xth}}
\newcommand{\fxo}{\mbox{$F_{x1}$}}
\newcommand{\fig}[1]{\begin{figure}[htbp]\vspace{1.5in}\caption{}\label{#1}\end{figure}}
\newcommand{\figg}[3]{\begin{figure}[htbp]\vspace{#3}\caption{#2}\label{#1}\end{figure}}
\newcommand{\wwa}{\mbox{$\arc{W_{1}W_{2}}$}}
\newcommand{\wwh}{\mbox{$\widehat{W_{1}W_{2}}$}}
\newcommand{\wo}{\mbox{$W_{1}$}}
\newcommand{\wt}{\mbox{$W_{2}$}}
\newcommand{\param}{{parameterization} }
\newcommand{\wallpoint}{curve point}
\newcommand{\wallpoints}{curve points}
\newcommand{\wall}{wall}
\newcommand{\walls}{walls}
\newcommand{\cellsegment}{cell segment}
\newcommand{\x}{X}
\newcommand{\y}{Y}
\newcommand{\SSo}{{\cal S}}
\newcommand{\SSt}{\mbox{S\hspace{-0.68em}S\hspace{.25em}}}  
%
\title{Sorting Points along an Algebraic Curve%
	\thanks{Received by the editors July 1, 1988;
	accepted for publication (in revised form) March 4, 1990.}}
\author{John K. Johnstone%
	\thanks{Department of Computer Science, The Johns Hopkins University, 
		Baltimore, Maryland 21218.
	The work of this author was supported in part by National Science Foundation grant 
	IRI-8910366.
	While the author was a graduate
	student in the Department of Computer Science, Cornell University, he was supported
	by a Natural Sciences and Engineering
	Research Council of Canada 1967 Graduate 
	Fellowship and an Imperial Esso Graduate Fellowship.}
\and
	Chanderjit L. Bajaj%
	\thanks{Department of Computer Science, Purdue University, 
		West Lafayette, Indiana  47907.
	The work of this author was supported in part by National Science Foundation grant
	MIP-88-16286, 
	Army Research Office contract DAAG 29-85-C0018 under Cornell MSI, and
	Office of Naval Research grant N00014-88-0402.}}
\begin{document}
\maketitle

\begin{abstract}
An operation that is frequently needed during the creation and manipulation of geometric 
models is the sorting of points along an algebraic curve.
Given a segment \arc{AB} of an algebraic curve, a set of points on the curve is sorted
from A to B along \arc{AB} by putting them into the order that they would be encountered 
in traveling continuously from A to B along \arc{AB}.
A new method for sorting points along a plane or space algebraic curve is presented.
Key steps in this method are the decomposition of a plane algebraic curve into convex 
segments and point location in this decomposition.
% The decomposition is accomplished by the tangents at the singularities and points of 
% inflection of the curve.
This new method can sort points on an arbitrary algebraic curve 
(including points spread over several connected components)
and it is particularly efficient 
because of its preprocessing, both of which make it superior to conventional methods.
The complexity of the new method is analyzed, and execution times of various sorting
methods on a number of algebraic curves are presented.
The theory developed for sorting can also be used to locate points on an arbitrary 
segment of an algebraic curve and to decide whether two points lie on the same connected 
component.
\end{abstract}

\begin{keywords}
sorting, decomposition, point location, convexity, algebraic curves,
geometric modeling, solid modeling
\end{keywords}

{\bf AMS(MOS) subject classifications.} 68U05, 68Q25, 68P10, 14H99

\section{Introduction}
%
The sorting of numbers into increasing order or words 
into alphabetical order is one of the basic problems of computer science.  
The purpose of this paper is to establish that the sorting of points
along a curve is a basic problem in geometric modeling and computational geometry,
and to present a universal and efficient method for this sorting.
This method relies upon the solution of two problems that are very useful in their
own right: convex decomposition of a curve and point location on a segment.

To sort a set of points from A to B along the curve segment \arc{AB} means to
put the points into the order that they would be encountered in traveling
continuously from A to B along \arc{AB} (Fig.~\ref{1.1}).
Points that do not lie on \arc{AB} are never encountered and are thus ignored.
A tangent vector at A is provided to indicate the direction in which the sort is to
proceed from A. 
This vector is especially important when the curve is closed, since
there are then two segments between A and B from which to choose.
All of the points, including A and B, are assumed to be nonsingular,
since otherwise their order might be ambiguous.
%
\figg{1.1}{The sorted order from {\rm A} to {\rm B} is {\rm III, II, IV}.}{1.5in}
% figure 1.1, p. 2

Our treatment shall be of irreducible algebraic plane curves (a curve that 
lies in a plane and is described by an irreducible polynomial\footnote{The coefficient
	domain of the polynomial can be the integers, rationals, algebraic real numbers,
	or any other set of numbers that has a finite representation.}
$f(x,y)=0$); 
in the rest of this paper, all curves are assumed to be of this type and nonlinear.
An extension of the methods to algebraic space curves is possible using 
a suitable projection of the space curve to a plane curve, as described in \cite{jj}.

The next section establishes that sorting is a fundamental operation of 
geometric modeling.
After discussing previous sorting methods in
\S~\ref{sp},
we introduce our new sorting method in
\S~\ref{co}.
Convex decomposition of a curve
and point location on a convex segment 
are discussed in \S\S~\ref{s-dec} 
and \ref{s-loc}.
Complexity issues and execution times of the
various sorting methods are presented in \S\S~\ref{s-c} and 
\ref{data}, and \S~\ref{sco} provides some conclusions.
The paper ends with an Appendix.
% containing several algorithms and lemmas that are important
% to the development of the theory of sorting, but which due to their 
% technical nature are more conveniently placed in an appendix.


\section{The importance of sorting}
The sorting of points along a curve has many applications 
in geometric modeling. 
The following problem is the most natural application.

\vspace{.2in}

{\bf Restriction.}

INSTANCE: A set S of points on a curve C and a segment $\arc{E_{1}E_{2}}$ of C.

QUESTION: Which points of S lie on $\arc{E_{1}E_{2}}$? 

SOLUTION: 
\begin{quote}
	One solution is to sort S $\cup\ \{\mbox{endpoints } E_{1},E_{2}\}$ along the curve
	and discard the points that do not lie between $E_{1}$ and $E_{2}$.
	However, a more efficient solution is simply to sort S along $\arc{E_{1}E_{2}}$, 
	starting at $E_{1}$.  
	The output of this sort is the set of points on $\arc{E_{1}E_{2}}$ (in sorted order).
	During this sort, the points of S that do not lie on $\arc{E_{1}E_{2}}$ are either 
	not encountered or they are encountered but eventually discarded. 
	(This will be easier to understand after a description of the sorting algorithm.)
\end{quote}
\vspace{.25in}
Since an edge of a solid model is often defined by a curve and a pair of endpoints,
restriction is a very basic problem in geometric modeling.
For example, the following edge intersection and bounding box problems are two important 
problems that can be solved with restriction.

\vspace{.2in}

{\bf Edge intersection.}

INSTANCE: Edges E and F on curves C and D, respectively.

QUESTION: What is $E \cap F$?

{\samepage
SOLUTION: 
\begin{quote}
Compute $C \cap D$ by well-known methods and restrict to the edges.
	That is, restrict $C \cap D$ to E and then restrict this $C \cap D \cap E$ to F.
\end{quote}

}
\vspace{.2in}

{\bf Bounding box.}

INSTANCE: Edge E on curve C with endpoints $E_{1}$ and $E_{2}$.

QUESTION: Find the smallest rectangle with sides parallel to the coordinate
\begin{quote}
axes that contains E.
\end{quote}

SOLUTION: 
\begin{quote}
Compute the local extrema of the curve and restrict to the edge, yielding S.
Find the minimum $x$-value ($x_{\mbox{\footnotesize{min}}}$) 
in $S \cup \{E_{1},E_{2}\}$, and so on.
The desired box is defined by the lines $x = x_{\mbox{\footnotesize{min}}}$, 
$x = x_{\mbox{\footnotesize{max}}}$, $y = y_{\mbox{\footnotesize{min}}}$, 
and $y = y_{\mbox{\footnotesize{max}}}$.
\end{quote}
\vspace{.25in}
%
\noindent The bounding box (see \cite[p. 372]{NS}) is useful for interference detection: 
the expensive intersection of edges can be reserved for those situations when the edges
are close enough that their bounding boxes interfere.
Bounding regions are also useful for problems such as the restriction problem, 
because they allow points that clearly do not satisfy a condition to be discarded
quickly.

Another fundamental use of sorting\footnote{In this paper, ``sorting'' will always
	refer to the sorting of points along a curve, not the conventional sorting of 
	numbers or words.} 
is to introduce an even-odd parity to a
set of points, as illustrated by the following problem.

\vspace{.2in}

{\bf Solid model intersection.}

INSTANCE: Two solid models M and N.

QUESTION: What is the intersection of M and N?

SOLUTION: 
\begin{quote}
An important step of this computation
is to find the segments of an edge of one model that lie in the intersection.
This is done by finding and sorting the points of intersection 
of this edge with a face of the other model. 
The segments of the edge between the $i$th and $(i+1)$st intersections,
for $i$ odd, are contained in the intersection of the models.
\end{quote}

\vspace{.2in}

Another application of even-odd parity 
is to decide whether a point lies within a piecewise-algebraic plane patch
(or a piecewise-algebraic convex surface patch).
% see Mortenson, Geometric Modeling, p. 545 for a good picture
This problem, which is fundamental to the display of a geometric model, 
is fully discussed in \cite{jj}.
Having established the importance of sorting, in the next section 
we proceed to a discussion of methods for sorting.

\section{Previous work on sorting}
\label{sp}
%
There is no serious study of sorting in the literature.
This can be explained by the fact that nontrivial sorting problems
arise only with curves of degree three or more, and until recently, 
almost all of the curves in solid models were linear or quadratic.  
However, as the science of geometric modeling matures and grows more
ambitious, curves of degree three and higher are becoming common.
For example, the introduction of blending surfaces \cite{hh87}
into a model creates curves and surfaces of high degree.

The lack of a study of sorting can also be explained 
by the presence of an obvious method for sorting points, 
which tends to obviate a search for any other method.
This obvious method uses a rational parameterization of the curve
(i.e., a parameterization ($x(t)$, $y(t)$) such that both 
$x(t)$ and $y(t)$ can be expressed as the quotient of two polynomials in $t$),
sorting a set of points S along \arc{AB} as follows.

\vspace{.2in}

\begin{center}{\bf The Parameterization Method of Sorting}\end{center}
\begin{description}
\item[{[Preprocessing]}] \ \ \ 
\begin{description}
\item[1.] Parameterize the curve.
\end{description}
%
\item[{[Solve]}] \ \ \ 
\begin{description}
\item[2.]
	Find the parameter values of A and B, say $t_{1}$ and $t_{2}$.
\item[3.]
	Find the parameter value of each point in S.
\end{description}
%
\item[{[Sort numbers]}] \ \ \ 
\begin{description}
\item[4.]
	Sort the parameter values of S from $t_{1}$ to $t_{2}$, discarding
	those outside this interval.
\end{description}
\end{description}

\vspace{.2in}

\noindent We insist upon a rational parameterization because a nonrational 
parameterization is difficult to represent and difficult to solve. 
With a nonrational parameterization (such as $x(t) = \sqrt{t}$ or 
$x(t) = \sin(t)$),
two different points may have the same parameter value, which complicates sorting.
Finally, there is no algorithm for the automatic parameterization of a curve 
that does not have a rational parameterization, whereas there is such an algorithm
for rational curves \cite{abba3}.

There are many reasons to be dissatisfied with the parameterization method.
It is not a universal method, since not all algebraic curves have a 
rational parameterization.
Indeed, a plane algebraic curve has a rational parameterization if and only if its
genus is zero, if and only if it has the maximum number of singularities 
allowable for a curve of its degree \cite{walker}.  
Second, even for those curves that do have rational parameterizations, 
the parameterization method will be slow if the degree of the parameterization 
is high, since the computation of the parameter values of the points will 
be expensive.
Other weaknesses of the parameterization method will become clear as we compare
it with the new method, such as its difficulty with sorting points that are
spread over several connected components of a curve (\S~\ref{sec-cc}).
% namely, not clear that it can distinguish connected components 

There is also a brute-force sorting method, which uses techniques for
tracing along a curve \cite{bhhl}.
The order of the points is the order in which they are encountered during
a trace of the segment.
This method is not satisfactory, because its implementation, although robust, is 
inherently very slow.
Moreover, its complexity depends upon the length of the segment that is being
sorted rather than upon the number of points in the sort, which is 
undesirable.

The weaknesses of the parameterization and tracing methods of sorting 
suggest that another method is necessary: one that will perform more
efficiently on a wider selection of algebraic curves.
The next section presents such a method.
This method works with the implicit representation $f(x,y)=0$ of a curve
(as opposed to the parametric representation),
thus allowing the use of tools from algebraic geometry.

\section{The convex segment method of sorting}
\label{co}

The observation that motivates the new method is that 
a convex segment can be sorted easily.
Since every curve is a collection of convex segments,
this suggests a divide-and-conquer strategy.
A segment of a plane algebraic curve is {\em convex} if no line has more than 
two distinct intersections with it.
Alternatively, a planar segment is convex if it lies entirely on one side of
the closed halfplane determined by the tangent line at any point of
the segment \cite{Do}.
The following theorem shows that sorting a convex segment is simple.

\begin{theorem}
\label{T-s}
Let $p_{1},\cdots,p_{n}$ be points on a convex segment \arc{AB}, 
and let H be the convex hull of A, B, $p_{1},\cdots,p_{n}$ {\rm (}Fig.~{\rm \ref{2.3})}.
The order {\rm (}from A to B{\rm )} of $p_{1},\cdots,p_{n}$ is simply 
the order {\rm (}from A to B{\rm )} of the vertices on the boundary of H.
\end{theorem}

\begin{proof}
See p. 19 of \cite{jj} for the proof.
\end{proof}

\vspace{.2in}

\figg{2.3}{The sorting of a convex segment.}{2.25in}
% Figure 2.3, p. 20

Suppose that a curve can be decomposed into convex segments.
Also suppose that the convex segments in this decomposition can be ordered
so that it is simple to determine the predecessor and successor (if any) of each
convex segment.
Finally, suppose that, given a query point, 
we can identify the convex segment 
that contains the query point (point location in a convex decomposition).
These key problems will be discussed in 
\S\S~\ref{s-dec} and \ref{s-loc}.
The following algorithm shows how to sort a set of points S 
along the segment \arc{AB}.

\vspace{.2in}

\begin{center}{\bf The Convex Segment Method of Sorting}\end{center}
\begin{description}
\item[{[Preprocessing]}] \ \ \ 
\item[\hspace{.3in} 1.] Decompose the curve into convex segments
	(say $S_{1}$, $S_{2}$, $\cdots$, $S_{k}$).
%
\item[{[Locate first convex segment]}] \ \ \ 
\item[\hspace{.3in} 2.]
	Find the convex segment that contains A (say $S_{i} = \arc{\wo\wt}$).
\item[\hspace{.3in} 3.]
	Decide whether \arc{AB} leaves A along \arc{AW_{1}} or \arc{AW_{2}}
	(say \arc{AW_{1}}).\footnote{If V is the vector at A that is given as 
		part of the input, then \arc{AB} leaves A along \arc{AW_{1}} if 
\label{alg-page}
		and only if V points to the halfplane defined by \lyne{AW_{1}} 
		that contains \arc{AW_{1}}.}
\item[\hspace{.3in} 4.]
	PresentConvSeg := \arc{AW_{1}} ;
	SortedSet := $\emptyset$ ;
	FoundB := false.
%
{\samepage
\item[{[Sort one convex segment at a time]}] \ \ \ 
\item[\hspace{.3in} 5.]
	Repeat until FoundB
\begin{tabbing}
{\bf (a)} \= Find the points of S that lie on PresentConvSeg.\\
	\> If B is one of these points, then FoundB := true.\\
{\bf (b)} \> Sort these points along PresentConvSeg, 
	using Theorem~\ref{T-s}.
\end{tabbing}
\pagebreak
\begin{tabbing}
{\bf (c)} \= If \= not FoundB, then\\
	\> \> SortedSet := Append(SortedSet,\{sorted points on PresentConvSeg\})\\
	\> else \\
	\> \> SortedSet := \= Append(SortedSet,\{sorted points on \\ 
	\> \> \> PresentConvSeg before B\})
\end{tabbing}
\begin{tabbing}
{\bf (d)} PresentConvSeg := \= appropriate neighboring segment of \\
	\> PresentConvSeg
\end{tabbing}

}
\item[{[Output]}] \ \ \ 
\item[\hspace{.3in} 6.]
	Return SortedSet.
\end{description}

\vspace{.2in}

The expense of this method is concentrated in the preprocessing phase, which
is done once off-line. 
The run-time operations (locating a point on a convex segment and
sorting a set of points along a convex segment) are usually simple.
Therefore, the efficiency of this method is very competitive.
The coverage of the convex segment method is the entire set of algebraic curves,
since it works directly from the implicit representation of the curve.

\figg{2.12a}{Sorting a curve by convex segments.}{2.75in}
% picture revised from Figure 2.10 of thesis in the way outlined in my thesis copy

\begin{example}
{\rm
Consider the sorting of points $P_{1}, \cdots, P_{6}$ 
along the segment \arc{AB} of Fig.~\ref{2.12a}.
The curve is decomposed into convex segments by the dotted lines 
(\S~\ref{s-dec}).
$A$ lies on \arc{W_{1}W_{8}} and
the vector at $A$ identifies that \arc{AW_{1}} is the first convex segment.
There are no points on \arc{AW_{1}}, so we move on.
The next convex segment is \arc{\wo\wt}.
Only $P_{1}$ lies on \arc{\wo\wt} and it becomes the first element of the 
sorted list.
We jump to the next convex segment \arc{W_{2}W_{3}} and
sort the two points $P_{2}$ and $P_{3}$ 
by creating the convex hull of $W_{2}$, $W_{3}$, $P_{2}$, and $P_{3}$.
$P_{2}$ and $P_{3}$ are added to the global sort.
We move on to the next convex segment \arc{W_{3}W_{4}}, and then \arc{W_{4}W_{5}}.
The presence of $B$ indicates that this is the last convex segment.
Upon sorting $B$ and $P_{4}$, $P_{4}$ is discarded because it comes after $B$.
The final sorted list is $P_{1},P_{2},P_{3}$.
}
\end{example}

It remains to discuss how a curve can be decomposed into convex segments and
how a point can be located in this convex decomposition.
These two problems, which are at the heart of the convex segment method 
of sorting, are solved in the following two sections.

\section{Convex decomposition of a curve}
\label{s-dec}
%
The decomposition of an object into simple objects is an important theme
in computational geometry.
Decomposition proves to be particularly useful in divide-and-conquer algorithms, 
since simple objects are easily conquered.
There has been a good deal of work on the decomposition of 
(simple, multiply connected, or rectilinear) polygons into simple components
(e.g., triangles \cite{CI}, \cite{G}, \cite{H}, \cite{T}, quadrilaterals \cite{S}, 
trapezoids \cite{As}, convex polygons  \cite{cd}, \cite{tm}, and star-shaped 
polygons \cite{Av}), sometimes with added criteria (e.g., minimum decomposition 
\cite{cd}, \cite{keil}, minimum covering \cite{O}, no Steiner points \cite{keil}).
However, all of this work has been in the polygonal (or at best polyhedral) 
domain.
The decomposition of a plane algebraic curve of arbitrary degree into convex 
segments is an extension of decomposition to the curved world.

A version of Bezout's theorem states that two irreducible plane algebraic curves 
of degree $m$ and $n$ have exactly $mn$ intersections (properly counted),
unless the curves are identical \cite{walker}.
Therefore, all plane algebraic curves of degree one (lines) and two (conics) 
are already convex.
For the convex decomposition of curves of degree three and higher, the 
singularities and points of inflection are instrumental.
A {\em singularity} of the curve $f(x,y)=0$ is a point $P$ of the curve
such that $f_{x}(P) = f_{y}(P) = 0$ (where $f_{x}$ is the derivative of $f$ with
respect to $x$).
It is a point where the curve crosses itself or changes direction sharply.
A nonsingular point is also called a {\em simple} point.
A {\em point of inflection} is a simple point $P$ of the curve
whose tangent has three or more intersections with the curve at $P$.  
(It is also a point of zero curvature.)
We restrict our attention to points of inflection $P$ such that $P$'s tangent 
has an odd number of intersections with the curve at $P$, which we call
\label{restriction}
{\em flexes} for short.
Fundamental in algebraic and differential geometry, singularities and flexes
form a skeleton of the curve and can be used in many useful ways.
(For example, singularities can be used to parameterize a plane algebraic curve 
\cite{abba3}.)
Their use in convex decomposition underlines their importance to
computational geometry of higher degrees.

\figg{2.8}{Convex segmentation of limacon of Pascal.}{1.75in}
% Figure 2.8(a), p. 27

The tangents at the singularities and flexes of a curve form an arrangement 
of lines that subdivide the plane of the curve into several regions 
(Figs.~\ref{2.12a} and \ref{2.8}).
The regions, which are closed and two-dimensional, are called {\em cells},
and the entire collection of cells is called a {\em cell partition} of the plane.
For example, the cell partition defined by the curve of Fig.~\ref{2.8} consists of
four unbounded cells.
The line segments that bound a cell are called (cell) {\em walls} and are part of the
cell.

The tangents also split the curve into several segments.
The following theorem establishes that each of these segments is convex.

\begin{theorem}
\label{deke}
The tangents of the singularities and flexes of a plane algebraic curve slice
the curve into convex segments.  That is, if \arc{PQ} is a nonconvex 
segment, then some tangent of a singularity or flex will intersect 
\arc{PQ}.\footnote{The simple points at which a singularity/flex 
	tangent touches, but does not cross, the curve are redundant and should 
	not be treated as convex segment endpoints in the decomposition.}
\end{theorem}
%
% \figg{2.9}{\wt\ should be ignored and this should be treated as a 
% single convex segment \arc{\wo W_{3}}}{.1in}  % thesis, 2.9
%
\begin{proof}
Let \arc{PQ} be a nonconvex segment of an algebraic curve.
Assume without loss of generality that \pq\ does not contain a singularity or 
a flex.
It can be shown that there exists a line L that crosses \pq\ at three 
(or more) distinct points \cite[p. 117]{jj}.\footnote{Already, by the definition
	of convexity, there must exist a line that intersects \arc{PQ} three
	(or more) times.}
% by Lemma~\ref{threecross} 
Let \xo, \xt, and \xth\ be three of these points, such that $\xt \in \xotha$ 
and \mbox{$\xotha \cap L = \{ \xo,\xt,\xth\}$}.
\xotha\ does not change its direction of curvature, since there is no 
singularity or flex on \pq.
\xotha\ is not a line segment, otherwise Bezout's theorem would imply that
the algebraic curve that contains \xotha\ is a line, which it cannot be since
it contains a nonconvex segment.
Therefore, it can be assumed without loss of generality that \xotha\ looks like 
Fig.~\ref{2.7}(a).
Let $R$ be the region bounded by \xotha\ and \seg{\xo\xth}.
We will show that $R$ contains a singularity or a flex.
This will complete the proof, since the tangent
of a point inside $R$ must intersect $\xotha \subset \pq$ at least once.
(The tangent must cross the boundary of $R$ twice, and at most one of these
intersections can be with 
\seg{\xo\xth}.)
The curve lies inside of $R$ as it leaves \xotha\ from \xo\ and outside 
of $R$ as it leaves \xotha\ from \xth.
Therefore, the curve must cross the boundary of $R$ after it leaves \xotha\
from \xo, either because it must join with \xth\ (if the curve is bounded)
or because an infinite segment of an algebraic curve
cannot remain within a bounded region (if the curve is unbounded) \cite[p. 116]{jj}.
The curve cannot intersect the \xotha\ boundary of $R$, since
\mbox{$\xotha \subset \pq$} is nonsingular by assumption.
Therefore, the curve must cross \seg{\xo\xth} after it leaves
\xotha\ from \xo.

\figg{2.7}{{\rm (a)} \xotha\ and R. {\rm (b)} Traveling from \xo\ to \seg{\xo\xth}.}{2.25in}
% Figure 2.6(b) and Figure 2.7, pp. 25-6
% only need 1.75 inches for (c) if we have to split it off
%

As the curve leaves \xotha\ from \xo,
it lies on the opposite side of \xo's tangent from \seg{\xo\xth}.
Therefore, after the curve leaves \xotha\ from \xo\ and before it leaves
$R$, the curve must cross \xo's tangent inside of $R$, in order to reach 
\seg{\xo\xth}.
In order to cross over \xo's tangent, the curve
must cross itself or change its curvature inside of $R$ (Fig.~\ref{2.7}(b)),
otherwise it will spiral around inside $R$ forever.
Therefore, $R$ contains a singularity or a flex.
\end{proof}
\Comment{
\Heading{Sketch of Proof:}

Assume w.l.o.g. that \arc{PQ}\ does
\marginpar{(*)}
not contain a singularity or flex.
Since \arc{PQ}\ is not convex, we can find a line that crosses it at
three distinct points: $x_{1}, x_{2}, x_{3}$.
By (*), \arc{x_{1}x_{3}}\ must be a spiral.
Consider the region R bounded by \arc{x_{1}x_{3}}\ and \seg{x_{1}x_{3}}.
It is sufficient to show that R contains a flex or singularity S,
since then S's tangent must cross the \arc{x_{1}x_{3}}\ boundary of R.
It can be shown that the curve must enter R as it leaves $x_{1}$ and
must eventually leave R via \seg{x_{1}x_{3}}.
However, in order to do this, the curve must cross itself or change its
curvature inside of R.
\QED
}

We include here a word about robustness.
Consider the accuracy required in the computation of the singularities, flexes, 
and their tangents in order to guarantee a true division into convex segments.
Suppose that, in the proof of Theorem~\ref{deke}, the tangent of a 
singularity/flex inside the region $R$ is used to split a nonconvex segment.
Any line through a point in the interior of $R$ would work equally well in splitting
the nonconvex segment.
Thus, in this case the method is robust under slight errors in tangents, singularities,
and flexes.
The other case is if a nonconvex segment $S$ is split into convex segments by a 
singularity or flex lying on $S$. 
The computed convex segment will differ from the actual convex segment by the same
amount as the computed flex (say) differs from the actual flex.
The only points that might be treated improperly are those that lie on the segment
between the computed and actual flex.
In other words, points that are within (some function of) machine precision 
of each other cannot be distinguished by the method and must be considered 
equivalent.
This equivalence of points within machine precision is inherent to any sorting 
algorithm.

Theorem~\ref{deke} does not solve the convex decomposition problem,
because it yields a confused collection of endpoints of convex segments, 
not a collection of convex segments.
The more challenging step of pairing up the endpoints remains, where
two endpoints are {\em partners} if they define a convex segment of the 
decomposition.
This pairing problem will be attacked in \S\S~\ref{ssp} 
and \ref{sspII}, but first the collection of convex segments must be 
refined.

\subsection{Refinement of convex segments I: Singularities}
\label{sec-refine1}

Many of the endpoints of the convex segments created by Theorem~\ref{deke} 
are singularities.  However, singular endpoints
cause problems in pairing.\footnote{Unless otherwise noted, in the rest of the paper
	the term {\em endpoint} will be synonymous with 
	``an endpoint of one of 
	the convex segments created by the cell partition of a curve.''
	Once we have discussed how to refine convex segments, we shall
	assume that all convex segments are refined and subsequently
	the term {\em endpoint} will refer to 
	``an endpoint of one of the refined convex segments,'' while
	{\em original endpoint} will refer to
	an endpoint of a convex segment before refinement.}
Consider a convex segment whose two endpoints are the same point,
which might occur around a singularity (Fig.~\ref{2.8}).
This situation is to be avoided, since pairing will turn out to be easier if 
the two endpoints of a convex segment are different.
It is also possible for a singularity to have more than two
partners and, in particular, two partners in the same cell (e.g., singularity 
$A$ in Fig.~\ref{2.12}).
This situation is also to be avoided, since it is easier to find the partner 
of an endpoint in a cell if this partner is unique.
A third problem with singular endpoints is that the ordering of points about
a singularity can be ambiguous.
Does $P_{2}$ or $P_{3}$ follow $A$ in Fig.~\ref{2.11}?
What is the order of the points on the loop of Fig.~\ref{2.11}: 
$S, P_{1}, P_{2}, P_{3}, S$ or $S, P_{3}, P_{2}, P_{1}, S$?

\figg{2.11}{Ambiguity about a singularity.}{2.5in}
% Fig 2.11 of thesis, with added S

As a result of these problems, all convex segments with singular endpoints 
will be replaced by shorter segments with nonsingular endpoints.
This is only a renaming procedure: although a segment $\arc{AB}$ may be replaced
by a segment $\arc{A'B'} \subset \arc{AB}$,
\arc{A'B'} still represents the convex segment $\arc{AB}$.
This is best seen in the next phase, point location,
where singular endpoints pose the same problems as they
do in pairing endpoints.
Although we work with $\arc{A'B'} \subset \arc{AB}$, if a point $x$ lies 
on $\arc{AB} \setminus \arc{A'B'}$ (a segment of the curve that
one might wrongly worry has been cut out of the curve),
it is still located on the convex segment named $\arc{A'B'}$.

The endpoint $A'$ that replaces $A$ is called a {\em refined} endpoint,
while $A$ is called an {\em original} endpoint.
A link is maintained between $A'$ and $A$.
Before refinement, a convex segment is called an {\em original} convex segment.
After refinement (both of any singular endpoints and of any infinite parts as discussed in 
the next section), a convex segment is called a {\em refined} convex segment.
Note that many segments will not require refinement, so their refined segments
will be identical to their original segments.

The process of replacing convex segments with singular endpoints
by convex segments with nonsingular endpoints 
is achieved by replacing singularities by nonsingular points.
For each branch of the curve that passes through a singularity, 
a pair of nonsingular points are found: one on either side of (and usually close to) 
the singularity.
As we shall see, the exact position of the nonsingular point is not important and
is quite flexible.

\figg{2.12}{The refinement of a singularity.}{3in}
% Figure 2.12(b) of thesis,p. 35

\begin{example}
\label{eg-pseudo}
\rm{
The original convex segments \arc{PA}, \arc{AQ}, \arc{RA}, and \arc{AS}
of Fig.~\ref{2.12} are replaced by the refined convex segments
\arc{PV_{1}}, \arc{V_{2}Q}, \arc{RW_{1}}, and \arc{W_{2}S}.
This is done by refining the singularity $A$ into four points:
$V_{1}$ and $V_{2}$ from one branch, and $W_{1}$ and $W_{2}$ from the other.
A link is maintained between $V_{1}$ and $V_{2}$ (as well as between \wo\ and \wt),
so that it is clear that $\arc{PV_{1}}$ is followed by $\arc{V_{2}Q}$.
Note that this refinement makes it clear that $Q$ (not $S$) must follow $P$.
}
\end{example}

Consider the problem of finding two points on each branch of a singularity, one on either 
side of the singularity.
We would like to do this by tracing \cite{bhhl} a small distance along the branch
in both directions from the singularity.
However, there is no reliable way of tracing along a branch as it passes
through a singularity, because the other branches create too much confusion.
Therefore, each branch of the singularity must be isolated so that it can 
be traced robustly.
This isolation is accomplished by blowing up the curve at the singularity by 
a series of quadratic transformations \cite{bhhl}, \cite{walker}, \cite{abhy}, as follows.
(Section~\ref{ssc}, on the computation
of singularities and flexes, is relevant to this discussion.)

The first step in blowing up a singularity is to translate it to the 
origin.\footnote{Since the quadratic transformation does not map the line $x=0$ 
	properly, the curve should also be rotated if necessary so that it is 
	not tangent to $x=0$ at the origin (see \cite{jj}).}
%	This is ensured by applying a nonsingular linear transformation 
%	$x = \alpha\hat{x}+\beta\hat{y}$ and $y = \delta\hat{x} + \gamma\hat{y}$, 
%	such that neither $\alpha\hat{x}+\beta\hat{y}$ nor 
%	$\delta\hat{x} + \gamma\hat{y}$ are tangents to the curve at the origin.
Let the new equation of the curve be $f(x,y)=0$.
A quadratic transformation is now applied to the curve.
The ({\em affine}) {\em quadratic transformation} $x = x_{1},\ y = x_{1}y_{1}$
\cite{abhy} has three important properties:
\begin{itemize}
\item
It maps the origin to the entire $y_{1}$-axis and the rest of 
the $y$-axis to infinity: $y_{1} = y/x$ so $(0,b)$ maps to 
$(0,b/0)$, which is a point at infinity unless $b=0$.
\item
It is one-to-one for all points $(x,y)$ with $x \neq 0$.
\item
$y = mx$, a line  through the origin, is mapped to the horizontal line $y_{1}=m$:\\
$y=mx \rightarrow  x_{1}y_{1} = mx_{1}  \rightarrow  y_{1}=m$.
\end{itemize}

\figg{2.17}{{\rm (a)} Node. {\rm (b)} Its quadratic transformation.}{1.75in}
% Figure 2.13, p. 38

\noindent Thus, a quadratic transformation maps distinct tangent directions of the
various branches of $f$ at the origin to different points on the 
{\em exceptional line} $x_{1} =0$.
The intersections of the transformed branches with the exceptional line
correspond to the transformed points of the origin (Fig.~\ref{2.17}).
If a point of $f(x_{1},x_{1}y_{1})$ on the exceptional line is singular, 
then the procedure is applied recursively (Fig.~\ref{18}).
The following lemma establishes that the various branches of the curve 
in the neighborhood of the singularity eventually get transformed to separate 
branches.

\begin{lemma}
{\rm \cite{abhy},\cite{abba3},\cite{walker}}.
A singularity can be reduced to a number of simple points by 
a finite number of applications of the quadratic transformation.
An ordinary singularity can be reduced to simple points by a 
single quadratic transformation, where a singularity of multiplicity 
$r$ is ordinary if its $r$ tangents are all distinct.
\end{lemma}

\figg{18}{{\rm (a)} The original singularity. {\rm (b)} After one quadratic transformation.
{\rm (c)} After a second transformation: the original singularity successfully 
transformed into two simple points.}{5.25in}
% Figure 2.14, p. 39

To summarize, each singularity is translated to the origin and transformed 
into a set of nonsingular points through the application of a series of quadratic 
transformations.  Each branch of the transformed curve intersects the 
exceptional line in a simple point, so this image branch can be traced
from the image singularity without confusion.
Therefore, upon each image branch, two points are found by tracing a 
short distance in either direction from the image singularity.
Finally, these points are mapped back to the original curve
to become refined endpoints, replacing the singularity.
These new endpoints clarify the branch connectivity at the singularity
and simplify the job of pairing.

% *******************************************************************

The renaming of singularities has only two purposes: to clarify branch
connectivity at the singularity and to simplify pairing.
Since neither of these purposes depends on the exact position of the refined endpoints,
the position of refined endpoints is quite flexible.
\label{page-flex}
In particular, we must only ensure that no convex segment is refined away.
Thus, during the refinement of a convex segment \arc{AB} through the refinement
of the singularity $A$, we must ensure that $A$ is replaced by 
$A' \in \arc{AB}$.
	% \footnote{This is only a concern if
	% 	\arc{AB} is very short, since $A'$ is always chosen close to $A$.}
Suppose $A \neq B$.
Certainly $A' \in \arc{AB}$ if the trace to $A'$ remains inside
the circle of radius $||A - B||/2$ centered at $A$.
Therefore (not yet knowing $B$), we restrict the trace to the circle
of radius $||A - E||/2$ centered at $A$, where $E \neq A$ is the closest endpoint
to $A$ that is not a refined endpoint associated with $A$.\footnote{The consideration of 
	refined endpoints
	associated with $A$ forces $A'$ artificially close to $A$.
	Note that $B$ is not 
	a refined endpoint associated with $A$, since \arc{AB} 
	existed before the refinement of A began.}
Note that traces can be reversed and made arbitrarily short.
Now suppose $A = B$ (i.e., \arc{AB} is a loop).
In this case, $A$ and $B$ are associated with different images 
of the singularity on the image curve, say $i_{1}$ and $i_{2}$, respectively, and 
the image of \arc{AB} on the image curve is \arc{i_{1}i_{2}}.
Certainly $A' \in \arc{AB}$ if the image of $A'$ is on \arc{i_{1}i_{2}}.
Therefore, we restrict the trace on the image curve to the
inside of the circle of radius
$||i_{1} - i_{j}||/2$ centered at $i_{1}$, 
where $i_{j} \neq i_{1}$ is the closest image of the singularity to $i_{1}$.
\Comment{
	after tracing a short distance from image($A$) to image($A'$) on the image curve and
	mapping back to the curve, if $A'$ lies too far from $A$, we return to the image curve
	and reverse the trace until it is short enough.  
	Note that traces can be made arbitrarily short.
}

\Comment{ (from above paragraph)
Thus, $\arc{AB}$ is a loop and the two sides of this loop are bounded by two tangents at $A$,
say $T_{1}$ and $T_{2}$.
Let $L$ be the line that bisects $T_{1}$ and $T_{2}$.
We can restrict $A'$ to $\arc{AB}$ by restricting it to the appropriate side of $L$ (Figure).
}

\subsection{Refinement of convex segments II: Infinite segments}
\label{sec-refine2}

Convex segments with singular endpoints are not the only ones that must
be refined: infinite convex segments are also problematic.
The pairing process is simplified if each convex segment has two endpoints,
but an infinite convex segment has only one endpoint.
Therefore, a second endpoint is added to each infinite segment, as follows.
Once again, this is only a renaming procedure.

\figg{3.9}{\seg{PQ} and \seg{QR} are artificial walls.}{2.75in}
% Figure 3.9, p. 64 of thesis

We shall artificially bound each unbounded cell by a collection of line segments,
called {\em artificial} ({\em cell}) {\em walls} (Fig.~\ref{3.9}).
These artificial walls are chosen carefully so that they only intersect 
infinite convex segments (if any) in the cell, and each of these exactly once 
(unless the infinite segment does not touch the original cell boundary at all
and thus proceeds
to infinity at both ends, in which case two intersections are allowed).
The walls should also be chosen so that the resulting artificially bounded cell 
is a convex polygon.
The creation of artificial walls that satisfy all of these properties
is discussed in the Appendix (\S~\ref{sec-caa}).

After every unbounded cell has been artificially bounded, a second endpoint
is added to each infinite convex segment by making an ({\em artificial}) {\em endpoint} 
at its point of intersection with an artificial wall (Fig.~\ref{3.egg}).
Subsequently, each infinite convex segment is represented by a finite convex
segment with two endpoints.
Once convex segments have been identified by pairing endpoints, 
a pair that contains an artificial endpoint will be recognized as an infinite convex 
segment.\footnote{A pair that contains two artificial endpoints will be recognized as 
	an entire connected component that does not cross any of the singularity/flex 
	tangents. See the discussion of nude components in \S~\ref{sec-pI}.}

After refining both singularities and infinite segments, there are 
three types of endpoint: 
(a) endpoints defined by the original cell decomposition, called original
endpoints: those that are nonsingular are still endpoints of refined
segments;
(b) endpoints refined from singularities, called refined endpoints,
and (c) endpoints added to infinite segments, called artificial endpoints.
From now on, when we speak of all of the endpoints of the decomposition, 
we are referring only to the collection of 
nonsingular original, refined, and artificial endpoints, 
unless otherwise stipulated.
Singularities are no longer considered to be endpoints.
After refinement, endpoints and cells assume the following normal form:

\begin{tabbing}
\indent (i)\ \ \= Every endpoint has exactly two partners,\\
\indent (ii) \> Every cell is a bounded polygon.
\end{tabbing}

\noindent Refinement will not only make the pairing of endpoints easier.
It will also create
a cleaner set of convex segments that better reflects the curve.
In particular, due to the endpoint normal form, the pairing of endpoints 
will create a collection of convex segments that can be ordered very easily.

\subsection{Pairing of endpoints I: Properties of the partner}
\label{ssp}

We are now ready to show how to pair the endpoints of convex segments.
Consider a convex segment in cell C and an endpoint E of this segment.
E's partner in C must obviously be another endpoint in C.
Therefore, the determination of partners in all single-segment cells is trivial.
Corollary~\ref{Cp} will present other conditions that E's partner must satisfy
and Theorem~\ref{Tpner} will show how to isolate the partner if several endpoints
satisfy all of these conditions.
In preparation, some terminology must be introduced and a crucial lemma proved.
%
\figg{3.2}{The inside of P's tangent.}{1.75in}
	% Figure 3.2, p. 56, originally 2.25in
% \figg{3.3}{E is the outside wallpoint of P's wall w.r.t.\ C}{1.75in}
%
\figg{3.2.5}{P faces both $Q_{1}$ and $Q_{2}$ with respect to C.}{1.75in}
	% Figure 3.4, p. 58, originally 2in
\begin{definition}
Let P be a point on a curve F, and let C be the cell (or one
of the cells) of F's cell partition that contains P 
(Figs.~\ref{3.2} and \ref{3.2.5}).
If P is a singularity or flex, then the 
{\em inside of P's tangent} with respect to (w.r.t.) C is the 
halfplane bounded by P's tangent that contains C (which is unambiguous, because P's 
tangent defines a wall of C).  
If P is neither a singularity nor a flex, 
the inside of P's tangent is the halfplane bounded by P's tangent that 
contains all of the curve in the immediate neighborhood of P.\footnote{The 
	inside of P's tangent
	can be determined by tracing from P.
	The trace is restricted to C
	to guarantee that it stays on the same side of the tangent.}
The inside includes the tangent, while the strict inside does not.

Let P be a flex that lies on the wall W of cell C, and 
let $P_{\epsilon}$ be a point of the curve inside C at distance
$\epsilon > 0$ from P.
($P_{\epsilon}$ may be found by tracing the curve into C from P.)
The {\em outside wallpoint} of W w.r.t.\ C is the endpoint of W that 
lies outside of $P_{\epsilon}$'s tangent, for $\epsilon$ small (E in Fig.~\ref{3.2.5}).

Let P and Q be points on a curve.
If P is not a flex, then P {\em faces} Q if Q lies on the inside of 
P's tangent (Fig.~\ref{3.2}).
If P is a flex lying on a wall of cell C, then
P faces Q w.r.t.\ C if (1) Q lies on the strict inside of P's tangent 
w.r.t.\ C or (2) Q lies on P's tangent and on the opposite side of P from 
the outside wallpoint of P's wall w.r.t.\ C (Fig.~\ref{3.2.5}).
\end{definition}

\begin{notation}
\#\{S\}\ is the number of elements in the set $S$ and
\seg{xy} is the line segment strictly between $x$ and $y$.
It is important to note that 
\seg{xy} does not include its \mbox{endpoints $x$ and $y$}.
\end{notation}

The following lemma captures the fact that, if \seg{\x\y} is a line segment 
satisfying some simple properties, the intersections of the curve with 
\seg{\x\y} must pair up into couples that face each other.

\begin{lemma}
\label{Ls}
Consider the cell partition of a curve F.
Let \x\ and \y\ be two nonsingular points of a convex segment in the cell C.
Then

{\samepage
{\rm (1)} The curve crosses\footnote{If P is a point of intersection of the curve 
	with \seg{\x\y}, then the curve {\em crosses} \seg{\x\y} at P if it lies 
	on both sides of \seg{\x\y} in any neighborhood of P; otherwise it 
	only touches \seg{\x\y} at P.}
\seg{\x\y} at an even number of points, ignoring singularities.

}
{\rm (2)} $\#\{P \in \seg{\x\y} \cap F: P \mbox{ faces \x\ w.r.t.\ C}\} = 
\#\{P \in \seg{\x\y} \cap F: P \mbox{ faces \y\ w.r.t.\ C}\}$.

{\rm (3)} For all\ \ $\alpha \in \seg{\x\y}$,\ \ \nopagebreak 
\begin{quote}
$\#\{P \in\seg{\x\alpha}\cap F: P \mbox{ faces \x\ w.r.t.\ C}\} \leq$
$\#\{P \in \seg{\x\alpha}\cap F: P \mbox{ faces \y\ w.r.t.\ C}\}$.
\end{quote}
\end{lemma}

\vspace{.2in}

\begin{example}
{\rm 
Fig.~\ref{4} is a hypothetical example for Lemma~\ref{Ls}.
The curve $F$ crosses \seg{\x\y} an even number of times.
$\{ P \in \seg{\x\y} \cap F:\mbox{ $P$ faces $\x$\ \} = \{}P_{2},P_{5},P_{6}\}$
is of the same size as 
$\{ P \in \seg{\x\y} \cap F:\mbox{ $P$ faces $\y$ } \} = \{P_{1},P_{3},P_{4}\}$.
Moreover,
$\{ P \in \seg{\x\alpha} \cap F: \mbox{$P$ faces $\x$} \} = \{P_{2}\}$\ 
is smaller than
$\{ P \in \seg{\x\alpha} \cap F: \mbox{ $P$ faces $\y$} \} = \{P_{1},P_{3},P_{4}\}$.
}
\end{example}
%
\figg{4}{An illustration of Lemma 5.3.}{2.25in}
%\begin{figure}[htbp]\vspace{2.25in}\caption{An illustration of Lemma~\ref{Ls}}\label{4}\end{figure}
% Figure 3.5, p. 59

Condition~(3) of Lemma~\ref{Ls} may look ominous, but it is not difficult to test.  
It is only shorthand for the fact that
the number of intersections that face $\y$ must always dominate the number that
face $\x$.  
In other words, we need only calculate information at the points
of intersection $P \in \seg{\x\y} \cap F$, not at an infinite number of 
$\alpha \in \seg{\x\y}$.
%
\par{\it Proof of Lemma~\ref{Ls}}. \ignorespaces
Consider the closed region $R_{\x\y}$ bounded by \seg{\x\y}\ and \arc{\x\y}.
% (Fig.~\ref{3.3.5}).
Since \arc{\x\y} lies in the cell $C$ and $C$ is a convex polygon, \seg{\x\y} 
must also lie in $C$.
Therefore, again by convexity, $R_{\x\y}$ must lie in $C$.
% \figg{3.3.5}{The region $R_{\x\y}$}{1.5in}
%
Since $\x$ and $\y$ are nonsingular and the rest of \arc{\x\y} lies in the
interior of the cell, \arc{\x\y} does not contain a singularity.
Therefore, the curve can only cross into $R_{\x\y}$ through \seg{\x\y}.
If the curve enters $R_{\x\y}$, then it must also leave, since an infinite 
segment cannot remain within a bounded region and an algebraic curve of finite
length is closed (viz., the curve cannot stop short in the middle of $R_{\x\y}$).
We claim that the point of departure $D$ must be distinct from the point of 
entry $E$, unless all of the tangents at $D=E$ are \lyne{XY}, as in 
Fig.~\ref{3.ex}.
In particular, if $D=E$ and some tangent at $D$ is not \lyne{XY}, 
then at least one of the tangents of the singularity $D$ 
will cross into $R_{\x\y}$ and form a wall of the cell partition which will 
split $R_{\x\y}$ in two, contradicting the fact that all of $R_{\x\y}$ 
lies in the same cell.
Therefore, with the exception of the special singularities of 
Fig.~\ref{3.ex}, the crossings of \seg{\x\y} by the curve occur in pairs,
called {\em couples}.
This establishes condition (1) of the lemma.

\figg{3.ex}{The only type of singularity that can lie on \seg{\x\y}.}{1.5in}
% Figure 3.7, p. 61

Consider condition (2).
The special singularities of Fig.~\ref{3.ex} (as well as the points where
the curve only touches \seg{XY}) can be ignored during 
the consideration of conditions (2) and (3), since they face both $\x$ and 
$\y$ and contribute the same amount to the left- and right-hand sides
of the expressions of conditions (2) and (3).
Therefore, we can concentrate on the remaining crossings of \seg{\x\y}:
the distinct couples.
Let $A,B\in \seg{\x\y}$ be a couple and assume, without loss of generality, 
that $A$ lies closer to $\x$ than $B$ does.
\arc{AB} is a convex segment since it lies within a cell of the cell
partition.
Therefore, $A$ and $B$ face each other (w.r.t.\ cell $C$).
Since $A$ faces $B$, $A$ faces $\y$.
Similarly, since $B$ faces $A$, $B$ faces $\x$.
Therefore, one member of each couple faces $\x$ and the other faces $\y$,
yielding condition (2).
Moreover, the point of a couple that faces $\y$ (A) is closer to $\x$ than the point 
that faces $\x$ ($B$), yielding condition (3).
{\ \vbox{\hrule\hbox{%
   \vrule height1.3ex\hskip0.8ex\vrule}\hrule
  }}\par

\begin{corollary}
\label{Cp}
Let $W_{1}$ be an endpoint of a convex segment that lies in the cell C of the cell partition of a curve F.
$W_{1}$'s partner $W_{2}$ in C must satisfy the following properties:

{\rm (1)} $W_{1}$ and $W_{2}$ face each other {\rm (}w.r.t.\ C{\rm )}.

{\rm (2)} $\#\{P \in \seg{W_{1}W_{2}} \cap F: P \mbox{ nonsingular,
	F crosses } \seg{\wo\wt} \mbox{ at } P\} = 2k$, $k \in {\cal Z}$.

\begin{tabbing}
\indent {\rm (3)} \= $\#\{P \in \seg{W_{1}W_{2}} \cap F: P \mbox{ faces \wo\ 
	{\rm (}w.r.t.\ C{\rm )}}\} = $\\
\> $\#\{P \in \seg{W_{1}W_{2}} \cap F: P \mbox{ faces \wt\ {\rm (}w.r.t.\ C{\rm )}}\}$.
\end{tabbing}

{\rm (4)} For all $\alpha \in \seg{\wo\wt}$,\ \ \nopagebreak
\begin{quote}
	$\#\{P \in\seg{\wo\alpha}\cap F: P \mbox{ faces \wo\ }\} \leq $
	$\#\{P \in\seg{\wo\alpha}\cap F: P \mbox{ faces \wt\ }\}$.
\end{quote}
\end{corollary}

\begin{figure}[htbp]\vspace{2.25in}\caption{\wt\ must be \wo's partner.}\label{3.12}\end{figure}
%Figure 3.12, p. 69

\noindent The conditions of Corollary~\ref{Cp} will often isolate the partner.

\begin{example}
{\rm 
Consider the cell partition of Fig.~\ref{3.12} and the cell containing
the convex segments \arc{\wo\wt} and \arc{W_{3}W_{4}}.
Suppose that we wish to find the partner of \wo.
$W_{3}$ violates condition (1) and $W_{4}$ violates condition (2), 
so \wt\ must be \wo's partner.
}
\end{example}

We add two practical notes to Corollary~\ref{Cp}.
In the sequel,
there will be many situations in which the conditions of Corollary~\ref{Cp}
must be checked.
During this computation, it is useful to keep in mind that the exact location of the
refined endpoints is not important (as discussed in the last paragraph of
Section~\ref{sec-refine1}).
This can be used to improve the ease or robustness of pathological computations.
For example, if many refined endpoints are clustered close together about a singularity,
it may be useful to spread these endpoints out.

The second note also concerns the checking of conditions in Corollary~\ref{Cp}.
Conditions~(3) and (4) remain true 
if $\#\{P \in \seg{W_{1}W_{2}} \cap F:\ \mbox{$<$cond$>$} \}$ 
is replaced by $\#\{P \in \seg{W_{1}W_{2}} \cap F:\ \mbox{$<$cond$>$},\,
P \mbox{ nonsingular},\ P \mbox{ is a crossing}\}$.  
Moreover, the restriction to nonsingular crossings does not change any of the 
proofs that
use this corollary.
For conciseness, we do not use the restriction.
However, for practical reasons it is advisable to use it,
because of the large potential for error in computing the direction that
singularities on $\seg{\wo\wt}$, as well as points that touch $\seg{\wo\wt}$, face.
(In theory, these singularities and touching points face both $\wo$ and $\wt$.)

\begin{figure}[htbp]\vspace{3.5in}\caption{Sending refined points to the boundary.}\label{3.J}\end{figure}
% Figure 3.10, p. 66 of dissertation

\subsection{Pairing of endpoints II: Distinguishing between candidates}
\label{sspII}

The remaining question in endpoint-pairing is how to find the partner 
of an endpoint $\wo$ in C if several endpoints in C satisfy all of the 
conditions of Corollary~\ref{Cp}.  
This will be done by sorting the candidates about the cell boundary
(Theorem~\ref{Tpner}).
Unfortunately, the refinement of singularities moved some of the endpoints 
of convex segments into the interior of cells.  
Therefore, in order to allow sorting about the boundary, we must associate 
a point $W'$ on the cell boundary with each refined endpoint W, as follows
(Fig.~\ref{3.J}).
If $W \neq \wo$,  then $W'$ is the intersection of the ray \ray{\wo W} 
with the cell boundary.
%
% \footnote{One might fear that the
% 	distance of a refined endpoint W from its singularity would affect the 
%	location of $W'$, and thus the behaviour of theorems that use $W'$,
%	namely Theorem~\ref{Tpner}.
%	However, the specific position of a refined endpoint upon its convex segment
%	is never important in Theorem~\ref{Tpner}: the fact that W lies somewhere 
%	on the appropriate convex segment is sufficient.}
%
If $W = \wo$, then $W'$ is one of the two intersections of \wo's tangent with
the cell boundary: the one that lies on a tangent of the singularity
from which \wo\ was derived.
% we can assume that W1's tangent does hit T: simply ensure that singularity 
% refinement does not move W1 too far from V
For notational consistency, $W' = W$ if $W$ is not a refined endpoint.

The following theorem shows how to find the partner of $\wo$ when Corollary~\ref{Cp} cannot.

\begin{figure}[htb]\vspace{2.1in}\caption{Partitioning the boundary in two halves.}\label{3.8A}\end{figure}
% Figure 3.11, page 68: switch the order to b,c,a

\begin{theorem}
\label{Tpner}
Let $\wo$ be an endpoint in cell C of the cell partition of a curve F,
R{\rm (}\wo{\rm )} the set of endpoints
in C that satisfy the conditions of 
Corollary~{\rm \ref{Cp}},
%That is, $R(\wo)$ := \{endpoints \mbox{W $\neq \wo$} in cell C $\mid$\nopagebreak
%\begin{enumerate}
%\item W and \wo\ face each other (w.r.t.\ C) 
%\item $\#\{P \in\seg{\wo W}\cap F: P \mbox{ faces \wo\ (w.r.t.\ C)}\}=$\\
%$\#\{P \in \seg{\wo W}\cap F: P \mbox{ faces W (w.r.t.\ C)\} }$\nopagebreak
%\item for all $\alpha \in \seg{\wo W}$, \\\nopagebreak
%$ \#\{P \in\seg{\wo\alpha}\cap F: P \mbox{ faces \wo\ (w.r.t.\ C)}\} \leq$\\
%$ \#\{P \in\seg{\wo\alpha}\cap F: P \mbox{ faces W (w.r.t.\ C)\} }  $
%\}
%\end{enumerate}
%
and S{\rm (}\wo{\rm )} 
the set of endpoints in R{\rm (}\wo{\rm )} that lie {\em strictly} inside of \wo's 
tangent.
% \footnote{Recall that
%	`endpoints' refers to all nonsingular original, refined, and artificial endpoints.}
There are two cases to consider in finding the partner of \wo.

{\rm (a)} Suppose that $S(\wo) \neq \emptyset$. 
Let \mbox{$S'(\wo) := \{\ W':W \in S(\wo)\ \}$} and, 
if \wo\ is not a flex, let $X \neq \wo '$ be the other intersection of \wo's 
tangent with the cell boundary, otherwise let X be the outside wallpoint of \wo's 
wall w.r.t.\ C {\rm (}Fig.~\ref{3.8A}{\rm )}.
$\wo '$ and X split the cell boundary into two halves.
Since every endpoint in $S'(\wo)$ will lie on the same half of the boundary,
a sort of $S'(\wo)$ from $\wo '$ to X is well defined.
Let $S_{1}',S_{2}',\cdots,S_{p}'$ be the result of this sort
{\rm (}i.e., $S_{i}'$ is encountered before $S_{i+1}'$ in a traversal of the cell boundary 
from $\wo '$ to X{\rm )}.
The partner of \wo\ in C is $S_{p}$ {\rm (}the endpoint associated with $S_{p}'${\rm )}.

{\rm (b)} Suppose that $S(\wo) = \emptyset$, and 
let $T(\wo)$ be the set of endpoints in $R(\wo)$ that lie on 
the same wall as \wo.
The partner of \wo\ in C is the element of $T(\wo)$ that is closest to \wo.
\end{theorem}

\begin{example}
{\rm 
Consider the computation of \wo's partner in Fig.~\ref{3.egg}, 
where \wo\ is the endpoint of an infinite convex segment.
\mbox{$R(\wo) = S(\wo) = \{\wt,W_{3},W_{4}\}$} and
\mbox{$S'(\wo) = \{\wt,W_{3},W_{4}'\}$}.
The sorted order of $S'(\wo)$ along the boundary from $\wo '$ to $X$
is $W_{3},W_{4}',\wt$, so \wt\ is the partner of \wo.  
Since \wt\ is an artificial endpoint, \wo\ must be the endpoint of an 
infinite convex segment.

\figg{3.egg}{The partner of \wo\ is \wt.}{2.25in}
% Figure 3.13, page 70

Consider the computation of the partner of \wo\ in Fig.~\ref{3.8},
where $S(\wo) = \emptyset$.
$V_{1}$, $V_{2}$ and $V_{4}$ are ruled out by condition (1) of $R(\wo)$,
while $V_{3}$ and $V_{6}$ are ruled out by condition (2).
Therefore, $T(\wo) = \{V_{5},W_{2}\}$.
\wt\ is the closest element of $T(\wo)$ to \wo, so it is \wo's partner.
}
\end{example}

\figg{3.8}{The partner of \wo\ is again \wt.}{2.125in}
% Figure 3.14, p. 71

\par{\it Proof of Theorem~\ref{Tpner}}. \ignorespaces
A key lemma in this proof (as well as others) is Lemma~\ref{lem-645} of the Appendix.
The reader is urged to refer to this lemma and its proof when it is used below.
%
\figg{6}{\wwh\ is dotted.}{2in}
% (a) is Figure 3.15(a)  KEEP THIS EVEN THOUGH IT LOOKS SIMPLE: 
% 	IT HELPS AS A REFERENCE FOR IMPORTANT THINGS IN THE PROOF
% (b) is forge(a) (additional figure)
% (c) is Figure 3.15(b)
%

(a) Suppose that $S(\wo) \neq \emptyset$.
Let \wt\ be \wo's partner, and let \wwh\ be the boundary of the cell 
from $W_{1}'$ to $W_{2}'$, such that $X \not\in \wwh$ (Fig.~\ref{6}).
It is sufficient to show that (i) $\wt ' \in S'(\wo)$ and (ii) $S'(\wo) \subset \wwh$.
Indeed, suppose that $\wt ' \in S'(\wo) \subset \wwh$
and consider a traversal of the cell boundary from $\wo '$ to $X$. 
Since $\wo '$ and $\wt '$ are extreme points of \wwh\ and
$X \not \in \wwh$ (by definition of $\wwh$), 
% the cell boundary between $\wt '$ and $X$ is not in $\wwh$.
% In particular, it is not in $S'(\wo) \subset \wwh$.
% Therefore, 
$\wt '$ is the last element of $S'(\wo)$ that is met during this traversal.
% from $\wo '$ to $X$.
In other words, $W_{2}' = S_{p}'$ ($W_{2} = S_{p}$) as desired.
(There is no ambiguity in choosing the last member of $S'(\wo)$ or in associating
$S_{i}'$ with $S_{i}$, since it can be shown that $S_{i}' \neq S_{j}'$ 
whenever $i \neq j$ \cite[p. 75]{jj}.)  % p. 75

We will first show that (ii) $S'(\wo) \subset \wwh$. Let $s \in S(\wo)$.
By Lemma~\ref{lem-645} of the Appendix, 
$\ray{\wo s}$ does not cross $\wwa \setminus \{\wt\}$.
In particular, \seg{\wo s'} does not cross $\wwa \setminus \{\wt\}$.
Therefore, $s'$ must either lie outside of \wo's tangent or on \wwh\ (Fig.~\ref{6}).
Since $s$, as a member of S(\wo), lies on the strict inside of \wo's tangent,
so must $s'$.
Therefore, $s' \in \wwh$ and $S'(\wo) \subset \wwh$, as desired.

\Comment{
REPLACED BY LEMMA LEM-645.
Suppose, for the sake of contradiction, that $\ray{\wo s}$ crosses \wwa\
\label{thpr}
at $y \neq \wt$ (Fig.~\ref{6}(b-c)).
There are two cases to consider: $y \in \seg{\wo s}$ and $s \in \seg{\wo y}$.
Suppose that $y \in \seg{\wo s}$ (Fig.~\ref{6}(b)).
By Lemma~\ref{Ls}, 
\[ \#\{P \in \seg{\wo y} \cap F: \mbox{ P faces \wo}\} = 
\#\{P \in \seg{\wo y} \cap F: \mbox{ P faces y}\} \]
But recall that $\seg{\wo y}$ does not contain $y$ and
y faces \wo, since \wo\ and y are on the same convex segment.
Therefore, if we choose $\alpha \in \seg{ys}$ such that \seg{y\alpha} does not
contain any intersections with the curve, then 
\[ \#\{P \in \seg{\wo\alpha} \cap F: \mbox{ P faces \wo}\} 
> \#\{P \in \seg{\wo\alpha} \cap F: \mbox{ P faces s}\} \]
in contradiction of $s \in S(\wo)$ (condition 4).
Now suppose that $s \in \seg{\wo y}$ (Fig.~\ref{6}(c)).
Recall an argument used at the beginning of Lemma~\ref{Ls}.
In particular, since \arc{\wo y} is a convex segment lying in a cell,
the nonsingular points of entry and departure of the curve into
the closed region bounded by $\seg{\wo y}$ and \arc{\wo y}
must be along $\seg{\wo y}$ and must pair up into couples.
Let $s$ and $t$ be a couple.
Since \arc{st} is convex, s faces t; since $s \in S(\wo)$, s faces \wo. 
Therefore, $t \in \seg{\wo s}$.
Since \mbox{$s \in S(\wo)$}, 
\[ \#\{P \in \seg{\wo s} \cap F: \mbox{ P faces \wo}\}
= \#\{P \in \seg{\wo s} \cap F: \mbox{ P faces s}\} \]
Noting that $\seg{\wo s} = \seg{\wo t}\ \cup\ \seg{ts}\ \cup\ \{t\}$ and 
t faces s, this becomes 
\[ \#\{P \in \seg{\wo t} \cap F: \mbox{ P faces \wo}\} +
\#\{P \in \seg{ts} \cap F: \mbox{ P faces \wo}\} + 0 =  \]
\nopagebreak
\[ \#\{P \in \seg{\wo t} \cap F: \mbox{ P faces s}\} +
\#\{P \in \seg{ts} \cap F: \mbox{ P faces s}\} + 1\ \ \  \]
Moreover, by Lemma~\ref{Ls} (\arc{st} is convex),
\[ \#\{P \in \seg{ts} \cap F: \mbox{ P faces s}\} = \]
\[ \#\{P \in \seg{ts} \cap F: \mbox{ P faces t}\} = \]
\[ \#\{P \in \seg{ts} \cap F: \mbox{ P faces \wo} \} \]
Upon cancelling terms in the above equation, we conclude that
\[ \#\{P \in \seg{\wo t} \cap F: \mbox{ P faces \wo}\} > \]
\[ \#\{P \in \seg{\wo t} \cap F: \mbox{ P faces s}\} =\  \]
\[ \#\{P \in \seg{\wo t} \cap F: \mbox{ P faces y}\}\ \  \]
But this contradicts condition (3) of Lemma~\ref{Ls} 
(convex segment \wwa, $\x = \wo$, $\y = y$).
These contradictions lead us to conclude that \ray{\wo s} does not cross 
$\wwa \setminus \{\wt\}$.
}

We now show that $\wt \in S(\wo)$ (which implies (i) $\wt ' \in S'(\wo)$).
$\wt \in R(\wo)$ by Corollary~\ref{Cp},
so it suffices to show that \wt\ lies strictly inside of \wo's tangent.
Since \wwa\ is a convex segment, \wt\ lies on the inside of \wo's 
tangent, and we must only show that \wt\ does not lie on \wo's tangent.
Suppose that \wt\ lies on \wo's tangent.
Since $\seg{\wo\wt}$ is a subsegment of \wo's tangent,
$S(\wo) \cap \seg{\wo\wt} = \emptyset$.  
Since $S'(\wo)$ is found
by following rays from \wo\ through elements of $S(\wo)$,
we also have $S'(\wo) \cap \seg{\wo\wt} = \emptyset$.
Finally, since \seg{\wo\wt} is part of a cell wall (Lemma~\ref{Ll} of the Appendix),
we have $\wwh = \seg{\wo\wt}$.
The combination of $S'(\wo) \cap \seg{\wo\wt} = \emptyset$ and
$S'(\wo) \subset \wwh = \seg{\wo\wt}$ forces $S'(\wo) = \emptyset$, which
contradicts our initial assumption.
Thus, \wt\ does not lie on \wo's tangent and $\wt \in S(\wo)$.

(b) The statement of the theorem has been verified if $S(\wo) \neq \emptyset$.
Now suppose that $S(\wo) = \emptyset$.
We first show that $T(\wo)$ is well defined and contains $\wt$.
If \wo\ does not lie on a cell wall,
then $\wt \in S(\wo)$:
$\wt \in R(\wo)$ (as \wo's partner);
\wt\ does not lie on \wo's tangent (Lemma~\ref{Ll});
and \wt\ lies inside \wo's tangent (because \wwa\ is convex).  
Thus, \wo\ must lie on a cell wall and $T(\wo)$ is well defined.
If \wt\ lies strictly inside \wo's wall (w.r.t.\ C), it also lies strictly
inside \wo's tangent (Lemma~\ref{Ll}).
That is, if $\wt \not \in T(\wo)$, then $\wt \in S(\wo)$.
Therefore, $\wt \in T(\wo)$.

Suppose that \wt\ is not the closest member of $T(\wo)$ to \wo, and let 
\mbox{$U \neq \wt$} be the closest.  
Since \wo\ faces $U$, $U$ must lie on $\seg{\wo\wt}$.
Using a familiar argument from Lemma~\ref{Ls}, the nonsingular points of entry
and departure of the curve into the closed region bounded by $\seg{\wo\wt}$
and \arc{\wo\wt} must pair up along $\seg{\wo\wt}$, such that each pair
defines a convex segment.
In particular, each nonsingular point of the curve on $\seg{\wo U}$ 
that faces \wo\ must pair with a nonsingular point on $\seg{\wo U}$ that faces \wt.
But, 
since $U \in R(\wo)$, the number of nonsingular points on $\seg{\wo U}$ that face \wt\ is
exactly 
equal to the number of nonsingular points on $\seg{\wo U}$ that face \wo\ 
(Corollary~\ref{Cp}).
Thus, each nonsingular point on \seg{\wo U} that faces \wt\ is paired with 
some nonsingular point on $\seg{\wo U}$.
In particular, no nonsingular point on \seg{\wo U} that faces \wt\ is paired with
a point on $\{U\} \cup \seg{U\wt}$.
But this leads to a contradiction:
$U$ is nonsingular (since it is an endpoint after the refinement stage)
and it faces \wo\ (since $U \in R(\wo)$), so it should
pair with a nonsingular point of the curve on $\seg{\wo U}$ that faces \wt.
We conclude that \wo's partner \wt\ must be the 
closest element of $T(\wo)$ to \wo, completing the proof of this pairing theorem.
{\ \vbox{\hrule\hbox{%
   \vrule height1.3ex\hskip0.8ex\vrule}\hrule
  }}\par

A plane algebraic curve can now be properly decomposed into convex segments,
since the endpoints of the convex segments can be properly paired together.
Note that it is now simple to sort the convex segments.
The endpoints form a doubly linked list
that defines the order of the convex segments
and makes it easy to traverse the curve, convex segment by convex segment.

\subsection{Computation of singularities and flexes}
\label{ssc}

The above convex decomposition of an algebraic curve requires the singularities 
and flexes of the curve, as well as their tangents.
The singularities of a curve $f(x,y)=0$ are the solution set of the system
$\{f_{x}=0,f_{y}=0,f=0\}$, while the points of inflection are the nonsingular 
intersections of the curve with its Hessian (the determinant 
of the matrix of double derivatives of the curve's equation) \cite{walker}.
%
% BAJAJ ADDITION
These systems of polynomial equations can be solved efficiently 
for all of their point solutions by using $U$-resultant schemes \cite{Ca}. 
% [p. 74]
Furthermore, using bounds on the minimum gap between roots of a polynomial, 
and bounds on the
size of the coefficients and degree of the original curve, we can straightforwardly
estimate the bits of accuracy required in computing the solution 
to ensure that the error in the value of the curve's derivatives is within the gap. 
This in turn ensures the correctness of the quadratic transformations applied to the 
curve and centered at the singularity.

The restriction of points of inflection to flexes (as described on 
page~\pageref{restriction})
is straightforward \cite[p. 44]{jj}.
The tangents of a singularity of the curve $f=0$ can be found by translating 
the singularity to the origin.
The equations of the tangents are the factors of the translated $f$'s order
form (the polynomial consisting of the terms of lowest degree) \cite{walker}.
Finally, after the curve has been translated to projective space by homogenizing 
its equation to $f(x,y,z)=0$ (where $z$ is the homogenizing variable),
the tangent of a flex P is $f_{x}(P)x + f_{y}(P)y + f_{z}(P)z= 0$ \cite{walker}.
This completes our description of the convex decomposition of an algebraic curve.

\section{Point location}
\label{s-loc}

The second key problem in the convex segment method of sorting is
point location in a convex decomposition: given a point, we must identify
the convex segment that contains it.
This is an extension to the curved domain of the well-known problem of point 
location in a planar subdivision. 
We will show how to locate a point on a convex segment (\S~\ref{sec-pI}), 
a general curve segment (\S~\ref{sec-pII}), and a connected component of a curve
(\S~\ref{sec-cc}).

\subsection{Point location I: On a convex segment}
\label{sec-pI}

A decomposition is not very useful unless it is possible to locate points in it. 
In the case of sorting, point location is necessary in order to partition a set of points
into convex segments.
Since a convex segment is identified by its endpoints, finding the convex 
segment that contains a point is equivalent to finding the endpoints that bound 
this convex segment.
This problem is analogous to finding the partners of a given
endpoint (\S~\ref{sspII}), since both problems are 
instances of the more general question: ``what are the two endpoints associated 
with a given point?''
We continue to use refined segments for point location, 
because it is important for 
an endpoint in a cell $C$ to uniquely identify a convex segment in $C$.
However, recall that points that do not lie on any refined segment are still located
on a convex segment: if $x \in \arc{AB} \setminus \arc{A'B'}$, where $\arc{A'B'}$ 
is the refined segment associated with the original segment \arc{AB}, 
then $x$ is located on $\arc{A'B'}$.

It is easy to locate a point in the proper cell, using well-known algorithms 
for point location in a planar subdivision \cite{kirk}, \cite{PS}.
Artificial walls are ignored during this step: a point
is considered to lie in an artificially bounded cell $C$ as long as it lies
in the unbounded cell associated with $C$.
If, as is often the case, a point lies in a cell with only one convex segment, 
then it is obvious to which convex segment it belongs.
Otherwise, Theorem~\ref{Tps} and Lemma~\ref{Ln} can be used 
to locate a point on the proper convex segment.
Lemma~\ref{Ln} shows how to locate points on a special type of convex segment,
a nude component.
Theorem~\ref{Tps} shows how to locate points on other convex segments.
Before we get to these results, we must define a nude component and the type of
a convex segment.
%
\begin{definition}
A {\em connected component} of a curve is a maximal subset of the curve such that 
there exists a continuous path on the curve between any two points of the subset.
\end{definition}

For example, a hyperbola has two connected components.
A connected component that
lies entirely inside of a cell, intersecting none of the walls (including
artificial walls) of the cell partition, is called a {\em nude component} 
(Fig.~\ref{nude}).
It is nude because, unlike other connected 
components, it does not contain any endpoints of convex segments.
This lack of endpoints makes nude components a special case for point location.
A nude component is convex, since it does not contain any singularities or flexes.
%
\figg{nude}{$x$ lies on a nude component.}{2.25in}
% `nude' picture of nonsingular cubic with x on nude component:
%	use walker picture of nonsingular cubic
%
\begin{definition}
Let \arc{P\infty} be an infinite convex segment before refinement, $x \in \arc{P\infty}$.
$x$'s tangent defines two directions:
the {\em finite {\rm (}respectively, infinite{\rm )} direction from $x$}
is the direction along $x$'s tangent that the curve leaves $x$ in traveling
to the endpoint $P$ (respectively, to infinity).
Let $R$ be the ray along $x$'s tangent in the infinite direction from $x$.
%
The {\em type of an infinite convex segment} is 
clockwise-convex or counterclockwise-convex.
\arc{P\infty} is {\em clockwise-convex} if and only if 
$R$ enters the halfplane defined by the inside of $x$'s tangent as it rotates 
clockwise (Fig.~\ref{fig-cc}).
(Since the clockwise direction of rotation depends on one's perspective,
a perspective from a fixed side of the plane containing the curve must be maintained for 
all type computations.)
\end{definition}

The type of an infinite convex segment is well defined, because it is independent of $x$.
The computation of this type is straightforward.
The only nontrivial step is the computation of the infinite direction from a point,
which is discussed in Lemma~\ref{lem-ty} of the Appendix.

\figg{fig-cc}{A clockwise-convex segment.}{2in}

% ********************************************************************

We are finally ready to show how to locate a point on the proper convex segment.
There are three main cases to this point location:
(i) the point lies inside the bounded cell and on a refined segment,
(ii) the point lies inside the bounded cell and not on a refined segment, and 
(iii) the point lies outside the bounded cell.

\begin{theorem}
\label{Tps}
Let $x$ be a point of curve F that lies in cell C.\footnote{Recall that, 
	if $C$ is an artificially bounded cell,
	a point is considered to lie in $C$ as long as it lies in the unbounded cell
	associated with $C$.}
If $x$ is an endpoint, it is located on the refined segment in C associated
with this endpoint.
If $x$ is a singularity,
it is located on one and only one of the refined segments in C associated with this 
singularity.
Otherwise, let $S(x)$ = \{endpoints W in C $\mid$
\begin{tabbing}
\hspace{.1in} \= {\rm (1)} x lies on the strict inside of W's tangent,\\
\> {\rm (2)} W lies on the strict inside of x's tangent,\\
\> {\rm (3)} \mbox{$\#\{P \in\seg{xW}\cap F: P \mbox{ faces x}\} =
\#\{P \in \seg{xW}\cap F: P \mbox{ faces W} \}$},\\
\> {\rm (4)} For all $\alpha \in \seg{xW}$,
$\#\{P \in\seg{x\alpha}\cap F: P \mbox{ faces x}\} \leq
\#\{P \in \seg{x\alpha}\cap F: P \mbox{ faces W}\}$\},
\end{tabbing}
%
\noindent and let $S''(x) = \{\ W'' : W \in S(x)\ \}$, where $W''$ is the intersection 
of \ray{xW} with the boundary of C.
{\rm (}$S(x)$ should be computed by the method described in \S~{\rm\ref{ss-speeeeed}.)}
If $S(x) = \emptyset$, then x lies on a nude connected component and Lemma~{\rm\ref{Ln}}
is used to locate $x$ on the correct nude component.

{\rm (i)} Suppose that $x$ lies inside the bounded cell C.
Let $x_{1}$ and $x_{2}$ be the two points of intersection of x's tangent with the
boundary of C, and let $S_{1}'',S_{2}'',\cdots,S_{p}''$ be the result of a sort of $S''(x)$ 
from $x_{1}$ to $x_{2}$ along the cell boundary.
$x$ either lies on $S_{1}$'s or $S_{p}$'s convex segment,
% 
% If $S''(x)$ has only one element $S_{1}''$, $x$ lies on $S_{1}$'s convex segment.
% IMPOSSIBLE SINCE INFINITE CONVEX SEGMENTS WILL STILL HAVE TWO ENDPOINTS
%
and there are three cases to consider:
{\rm (a)} if $S_{1}$ and $S_{p}$ are partners,\footnote{We will see that $S_{1}$ and $S_{p}$ 
	are partners whenever $x$ lies on a refined segment.}
$x$ lies on \arc{S_{1}S_{p}};
{\rm (b)} otherwise, if $S_{1}$ is a refined endpoint and 
$S_{p}$ is not, $x$ lies on $S_{1}$'s convex segment;
{\rm (c)} if both $S_{1}$ and $S_{p}$ are refined,
$x$ lies on $S_{1}$'s convex segment if and only if 
$S_{1}''$ is one of the extreme points of the sort
of $\{S_{1}'',S_{p}'',T_{1},T_{p}\}$ from $x_{1}$ to $x_{2}$, where
$T_{1}$ and $T_{p}$ are the singularities
associated with $S_{1}$ and $S_{p}$.

% *******************************************************************************

{\rm (ii)} Suppose that $x$ lies outside of C.
Let $A(x) = \{ W \in S(x) : W \mbox{ is an artificial}$ endpoint$\}$ and
let $Q \in A(x)$ be the unique endpoint in $A(x)$ whose convex segment is of the
same type as $x$'s segment.
Then $x$ lies on $Q$'s convex segment.
%
\Comment{
Let $A(x) = \{ W \in S(x) : W \mbox{ is an artificial endpoint (i.e., an endpoint lying
	on an artificial wall)}\} = \{A_{1},A_{2},\cdots,A_{k}\}$.
If $k=1$, then $x$ lies on $A_{1}$'s convex segment.
If $k \geq 2$,
let $\theta_{1,i} \in [0,2\pi]$ be the angle of the tangent at the artificial
endpoint $A_{i} \in A(x)$ ($i = 1,2,\cdots,k$),
let $\theta_{2,i} \in [0,2\pi]$ be the angle of the wall at $A_{i} \in A(x)$,
and let $\theta$ be the angle of $x$'s tangent (Fig.~\ref{fig.fiddle}).
(Since it is ambiguous whether a tangent defines the angle $\theta$ or $\theta + \pi$,
it is actually necessary to measure rays rather than lines:
for $A_{i}$'s tangent, we use the ray that points out of the artificially bounded cell;
for $A_{i}$'s wall, we use the ray that points into the halfplane defined by $A_{i}$'s 
tangent that does not contain $A_{i}$'s partner (i.e., the halfplane that does not 
contain $A_{i}$'s convex segment);
for x's tangent, we use the 
)
Then $x$ lies on $A_{j}$'s convex segment, where $j$ is the unique one such that
$\theta \in [\theta_{1,j},\theta_{2,j} + \pi]$ (?? could be $\cdots - \pi$ i think ??).
}
% *******************************************************************************
\end{theorem}

\vspace{.1in}

\begin{example}
{\rm 
In Fig.~\ref{nude}, $S(x) = \emptyset$ and $x$ lies on a nude component.

Consider the cell of Fig.~\ref{3.12} that contains the convex segments 
\wwa\ and \arc{W_{3}W_{4}}.
\wo\ does not satisfy condition~(2) of $S(x)$ and \wt\ does not satisfy condition~(3).
Thus, $S(x) = \{W_{3},W_{4}\}$ and $x$ must lie on \arc{W_{3}W_{4}}.

Consider the cell partition of Fig.~\ref{2.12a}.
$S(P_{1}) = \{W_{1},W_{2},W_{5},W_{6}\}$, which does not resolve the question
of $P_{1}$'s convex segment.
Let \xo\ and \xt\ be the two points of intersection of $P_{1}$'s tangent
with the cell boundary.
Since the sort of $S''(P_{1})$ from \xo\ to \xt\ is (\wo, $W_{6},\ W_{5}$, \wt),
and \wo\ and \wt\ are partners, $P_{1}$ must lie on \wwa.

In Fig.~\ref{fig-4a2}, the sort of $S''(x)$ from $x_{1}$ to $x_{2}$ is ($\wo''$,
\wt, $W_{3}$, $W_{4}''$), \wo\ and $W_{4}$ are not partners,
and both of them are refined. 
Since the sort of $\wo''$, $W_{4}''$, $T_{1}$, and $T_{p}$ from $x_{1}$ to
$x_{2}$ is $\{\wo'',W_{4}'',T_{1} = T_{p}\}$, we conclude that $x$ lies on \wo's segment.

In Fig.~\ref{fig-4a3}, $x$ lies outside of the bounded cell and
$A(x) = \{\wo,W_{3}\}$.
$x$'s and $\wo$'s segment are
clockwise-convex, but $W_{3}$'s segment is counterclockwise-convex.
Thus, $x$ must lie on \wo's segment.
}
\end{example}
\figg{fig-4a2}{$x$ lies on \wo's convex segment.}{2.8in}
\figg{fig-4a3}{$x$ lies on \wo's convex segment.}{3in}
%
\par{\it Proof of Theorem~\ref{Tps}}. \ignorespaces
Consider the decomposition of the curve into convex segments before refinement
of singularities and infinite segments.
In particular, consider the convex segment SEG in this decomposition that contains $x$.
After refinement of the convex segments, SEG is represented by a refined segment
$\arc{\wo\wt} \subset \mbox{SEG}$.
When $x \in  \arc{\wo\wt}$, the proof is very similar to that of Theorem~\ref{Tpner}.
(Note that it will usually be the case that $x \in  \arc{\wo\wt}$.
Indeed, in many cases $\arc{\wo\wt} = \mbox{SEG}$.)
However, new ideas are required when $x \in \mbox{SEG} \setminus \arc{\wo\wt}$, 
in which case
$x$ either lies on a segment of the curve that was removed while refining a singularity,
or it lies on a segment that was removed while refining an infinite convex segment.
In the former case, 
assume without loss of generality that $x \in \arc{W_{\mbox{\footnotesize{sing}}}\wo}$,
where $W_{\mbox{\footnotesize{sing}}}$ is the singularity that was refined to make \wo.
In the latter case, assume without loss of generality that \wo\ is the artificial endpoint
of $x$'s infinite segment and $x \in \arc{\wo\infty}$.

\begin{tabbing}
\indent \= We first notice that $\wo,\wt \in S(x)$:\\
\> \ \ (1)--(2) $x$, \wo, and \wt\ lie on the same convex segment and $x \neq \wo,\wt$.\\
\> \ \ (3)--(4) Lemma~\ref{Ls} ($\x = x$, $\y = \wo \mbox{ or } \wt$).
\end{tabbing}

\noindent Thus, if $S(x) = \emptyset$, then $x$ must lie on a nude component containing 
no endpoints.
% (One can also quite easily establish the converse: if x lies on a nude 
% component, then $S(x) =  \emptyset$.)
%
% Let x be a point on a nude component in cell C, and consider an endpoint E of a 
% convex segment in C.
% E certainly does not lie on the nude component.
% If E lies on or outside of x's tangent, then condition (4) of S(x) is violated.
% If E lies strictly inside of x's tangent, then the first intersection of \seg{xE}
% with the curve after x is another point of x's nude component (since this 
% component is closed and convex).  
% But this intersection faces x (again by the convexity of nude components),
% violating condition (3) of S(x).
%
Suppose $S(x) \neq \emptyset$.
Notice that a sort of $S''(x)$ 
from $x_{1}$ to $x_{2}$ along the cell boundary is well defined, 
since all of $S''(x)$ lies on the same side of $x$'s tangent
(condition (2) of $S(x)$).

(i) Suppose that $x$ lies inside $C$.
Thus, $x \in \arc{\wo\wt}$ or $x \in \arc{W_{\mbox{\footnotesize{sing}}}\wo}$.
In either case, 
$\wo''$ is either the first or last element of the sort of $S''(x)$
from $x_{1}$ to $x_{2}$, by Corollary~\ref{cor-645} of the Appendix.
That is, $\wo'' = S_{1}'' \mbox{ or } S_{p}''$; or
$\wo = S_{1} \mbox{ or } S_{p}$, 
since $S_{i}'' = S_{j}''$ if and only if $S_{i}=S_{j}$ \cite[p. 75]{jj}.
We conclude that $x$ lies on $S_{1}$'s convex segment or 
$S_{p}$'s convex segment.
We must decide which one.

If $S_{1}$ and $S_{p}$ are partners, then $S_{1}$'s convex segment is the same as
$S_{p}$'s convex segment, and the decision is easy.
Notice that this is the case whenever $x \in \arc{\wo\wt}$.
Indeed, if $x \in \arc{\wo\wt}$,
Corollary~\ref{cor-645} can be applied to \wt\ just as it was to \wo, yielding
$\wt'' = S_{1}'' \mbox{ or } S_{p}''$.
That is, $\{\wo,\wt\} = \{S_{1},S_{p}\}$, and $S_{1}$ and $S_{p}$ are partners.

Assume that $S_{1}$ and $S_{p}$ are not partners.
Thus, $x \in \arc{W_{\mbox{\footnotesize{sing}}}\wo}$ ($x \not\in \arc{\wo\wt}$)
and  \wo\ is a refined endpoint (i.e., an endpoint derived from a singularity).
If $S_{1}$ is refined and $S_{p}$ is not, then it is clear that $\wo = S_{1}$ and that
$x$ lies on $S_{1}$'s convex segment.

Thus, further assume that both $S_{1}$ and $S_{p}$ are refined
and $x$ lies on $S_{1}$'s segment, $\arc{S_{1}T_{1}}$.
We shall establish that $S_{1}''$ is, while $S_{p}''$ is not, an extreme point
in the sort of $\{S_{1}'',S_{p}'',T_{1},T_{p}\}$ from $x_{1}$ to $x_{2}$.
This will complete the proof of case (i).
First note that $S_{p}''$ is not an extreme point in this sort:
$S_{1}''$ is closer to $x$'s tangent than $S_{p}''$,
because \ray{xs} does not cross $\arc{xS_{1}} \setminus \{S_{1}\}$ 
for any $s \in S(x)$ (Lemma~\ref{lem-645});
similarly, on the other side, $T_{1}$ is closer to $x$'s tangent than $S_{p}''$,
because \ray{xs} does not cross $\arc{xT_{1}} \setminus \{T_{1}\}$ 
for any $s \in S(x)$ (Lemma~\ref{lem-645}).
Finally, note that $S_{1}''$ {\bf is} an extreme point in the sort (Fig.~\ref{fig-too}).
$T_{1}$ clearly does not lie between $S_{1}''$ and $x$'s tangent.
Suppose, for the sake of contradiction, that $T_{p}$ lies between $S_{1}''$ and $x$'s 
tangent.
The original convex segment that contains $x$ divides the cell into two regions:
let $R_{\mbox{\footnotesize{out}}}$ be the region that contains the outside of $x$'s tangent.
The entire segment $\arc{S_{p}T_{p}}$ must lie in $R_{\mbox{\footnotesize{out}}}$, because
$T_{p}$ does and $\arc{S_{p}T_{p}}$ cannot cross
$x$'s segment without making a singularity.
However, $S_{p} \in R_{\mbox{\footnotesize{out}}}$ implies that $S_{p}$ lies outside $x$'s 
tangent
or $x$'s segment crosses $\seg{xS_{p}}$, both of which contradict $S_{p} \in S(x)$
(using condition 2 of $S(x)$ and Lemma~\ref{lem-argh}, respectively).
Therefore,
we conclude that $S_{p}''$, $T_{1}$, and $T_{p}$ do not 
lie between $S_{1}''$ and $x$'s tangent,
forcing $S_{1}''$ to be an extreme point in the sort of $\{S_{1}'',S_{p}'',T_{1},T_{p}\}$.

\figg{fig-too}{An illegal position for $T_{p}$.}{3.5in}

% ***********************************************************************

(ii) Suppose that $x$ lies outside of $C$.
That is, $x$ lies on the infinite segment $\arc{\wo\infty}$,
according to the assumptions at the beginning of the proof.
We wish to show that we can locate $x$ on \wo's convex segment
by finding an artificial endpoint
in $A(x)$ whose convex segment is of the same type as $x$'s segment.
That is, we wish to show that
\mbox{$\{a \in A(x): \mbox{$a$'s convex segment is of the same type as $x$'s}\} =$}
$\{\wo\}$.
One direction is simple: $\wo \in A(x)$ since we have shown that $\wo \in S(x)$,
and \wo's convex segment is certainly of the same type as $x$'s, because it is the same
segment.
For the other direction, let $a \in A(x)$ such that $a$'s convex segment 
is of the same type as $x$'s.
The idea of the proof is to show that $a$'s convex segment
enters a bounded region and cannot legally leave except through $x$.
This will prove $x \in \arc{a\infty}$, implying $a = \wo$ as desired.

The first step of the proof is to find a region that contains
\arc{x\infty} and \arc{a\infty}, where \arc{x\infty} (respectively, 
\arc{a\infty}) is the part of
$x$'s (respectively, $a$'s) segment starting at $x$ (respectively, $a$) and proceeding to infinity.
Note that \arc{x\infty} and \arc{a\infty} lie outside of the artificially bounded cell.
We shall show that a containing region for \arc{x\infty} and \arc{a\infty} is 
$R_{xa}$, the intersection of the inside of $x$'s tangent and the inside of $a$'s tangent 
(Fig.~\ref{fig-ax}).
%
\begin{figure}[tbp]\vspace{2in}\caption{A region that contains \arc{x\infty} and \arc{a\infty.}}\label{fig-ax}\end{figure}
%
$R_{xa}$ is nonempty because $x$ and $a \in A(x) \subset S(x)$ face each other.
We shall only show $\arc{x\infty} \subset R_{xa}$, since the proof of 
$\arc{a\infty} \subset R_{xa}$ is entirely analogous, 
\arc{x\infty} cannot leave $R_{xa}$ through $x$'s tangent, by the convexity of \arc{x\infty}.
Suppose that \arc{x\infty} leaves $R_{xa}$ through
$y$ on $a$'s tangent (Fig.~\ref{fig.chick}).
We shall establish a contradiction, thus proving that $\arc{x\infty}$ is restricted
to $R_{xa}$.
Let $R_{xya}$ be the region bounded by \seg{xa}, \arc{xy}, and $a$'s tangent.
$a$'s segment cannot pass through \arc{xy}
(it would cause a singularity inside the cell), $a$'s tangent (by 
the convexity of $a$'s segment), or \seg{ax} (Lemma~\ref{lem-argh}),
and it cannot double back through $a$
(this would cause a singularity inside the cell again).
Thus, when $a$'s segment enters $R_{xya}$, it cannot leave.
This leads to a contradiction.
If \arc{a\infty} enters $R_{xya}$, we get the contradiction that an infinite, 
unbounded segment lies in the finite, bounded region $R_{xya}$.
If the other half of $a$'s segment enters $R_{xya}$,
we get the contradiction that this segment cannot reach the boundary of the 
cell C ($R_{xya}$ lies in the interior of $C$ by the convexity of $C$), 
although all convex segments originally start on a cell boundary.

\begin{figure}[bp]\vspace{2in}\caption{\arc{x\infty} cannot intersect $a$'s tangent.}\label{fig.chick}\end{figure}

We now try to restrict \arc{x\infty} and \arc{a\infty} to even smaller regions.
$R_{xa}$ is split into two subregions by the line segment \seg{xa}:
let $R_{x}^{\infty}$ (respectively, $R_{a}^{\infty}$) be the subregion that contains 
the infinite direction from $x$ (respectively, $a$), as shown in Fig.~\ref{fig-ax}(b).
$\arc{x\infty}$ is restricted to $R_{x}^{\infty}$:
it starts out in $R_{x}^{\infty}$, it cannot intersect \seg{xa}
(Lemma~\ref{lem-argh}), 
it cannot pass through $x$ again (since this would cause
a singularity inside the cell),
and it cannot intersect $a$ (since the unique intersection of $x$'s convex segment 
with the artificial boundary occurs before \arc{x\infty}).
By the same proof, \arc{a\infty} is restricted from passing out of $R_{a}^{\infty}$,
with one exception: it can escape through $x$.
Thus, $\arc{a\infty} \subset R_{a}^{\infty}$ unless $x \in \arc{a\infty}$.

We shall finish the proof by showing
that $R_{a}^{\infty}$ is a finite, bounded region.
This will establish that $\arc{a\infty} \subset R_{a}^{\infty}$
is impossible (since \arc{a\infty} is an infinite segment) and 
imply $x \in \arc{a\infty}$ as desired.
To show that $R_{a}^{\infty}$ is bounded, 
it is sufficient to show that $R_{x}^{\infty} \neq R_{a}^{\infty}$:
only one of the two subregions of $R_{xa}$ is unbounded (the 
	two lines bounding $R_{xa}$ are not parallel,
	since \arc{x\infty} could not stay both convex
	and infinite while remaining entirely between 
	$x$'s tangent and another parallel line)
and $R_{x}^{\infty}$ is certainly unbounded (since
it contains the infinite segment \arc{x\infty}).
Let $V_{1}$ (respectively, $V_{2}$) be the tangent ray from $x$ (respectively, $a$) 
in the infinite direction, and assume without loss of generality 
that $x$'s and $a$'s convex segments 
(which are of the same type) are counterclockwise-convex.
Thus, as $V_{1}$ and $V_{2}$ rotate counterclockwise, they both enter $R_{xa}$.
If $V_{1}$ and $V_{2}$ point to the same side of \seg{xa},
then as they both rotate counterclockwise, one of them will leave $R_{xa}$.
(Refer to Fig.~\ref{fig-ax}(a).)
Therefore, $V_{1}$ and $V_{2}$ must point to opposite sides of $\seg{xa}$,
implying $R_{x}^{\infty} \neq R_{a}^{\infty}$.

In conclusion, since $R_{a}^{\infty} \neq R_{x}^{\infty}$ and only one of 
the two subregions of $R_{xa}$ is unbounded, 
$R_{a}^{\infty}$ must be bounded and \arc{a\infty} must escape from it through
$x$, implying $a = \wo$.
Therefore, $\{a \in A(x): 
\mbox{$a$'s convex segment is of the same type as $x$'s}\} = \{\wo\}$.
{\ \vbox{\hrule\hbox{%
   \vrule height1.3ex\hskip0.8ex\vrule}\hrule
  }}\par

If there is only one nude component in a cell, then Theorem~\ref{Tps} 
can successfully locate a point on this convex segment.
However, if there is more than one nude component in the cell, then the following
lemma must be used to locate points on these nude components.

\begin{lemma}
\label{Ln}
Let P and Q be points that lie on nude components in the same cell.
P and Q lie on the same nude component if and only if 
Q lies in S{\rm (}P{\rm )}, where S{\rm ()} is as in Theorem~{\rm\ref{Tps}}.
\end{lemma}
\begin{proof}
Let $P$ and $Q$ lie on nude components $M$ and $N$, respectively.
If $M=N$, then $P$ and $Q$ lie on the same convex segment, so $Q \in S(P)$ by
Lemma~\ref{Ls}.
Suppose that $M \neq N$.
Nude components do not intersect, since they do not contain any singularities.
Therefore, there are only three cases to consider: $M$ lies inside $N$, $N$ lies 
inside $M$, and neither lies inside the other.
In all three cases, it is straightforward to show that $Q$ violates one of the conditions
of $S(P)$.
\end{proof}

\subsubsection{Speeding up point location}
\label{ss-speeeeed}

Point location can be made faster if the set $S(x)$ of Theorem~\ref{Tps} can be
computed more quickly.
Indeed, this is the only computation in the theorem that could be expensive.
Since the expense is concentrated in testing conditions (3) and (4) (which involve
line-curve intersections, whereas conditions (1) and (2) are simple to test),
we must avoid testing conditions (3) and (4).
The idea is to eliminate as many endpoints as possible from $S(x)$ without using
conditions (3) and (4).
First, eliminate any endpoints in the cell that do not satisfy conditions (1) and (2) 
of $S(x)$.
Next observe that if $x$ is located on \arc{\wo\wt}
(whether on the refined segment or not), then $\wo,\wt \in S(x)$.
(Refer to the proof of Theorem~\ref{Tps}.)
Thus, if $W \not\in S(x)$, we can eliminate $W$'s partner from $S(x)$ as well,
since $x$ cannot lie on $W$'s segment.
As many endpoints as possible should be eliminated using this observation.
If only two endpoints (say $V$ and $W$) remain after these two rounds of elimination,
we can immediately conclude that $x$ lies on $V$'s segment.
If more than two endpoints remain, we are forced to use conditions (3) and (4) to test
their membership in $S(x)$.
However, we have avoided testing these conditions on many endpoints.
As above, whenever an endpoint fails condition (3) or (4), its partner should also be
immediately eliminated from $S(x)$.

This completes our description of techniques that are needed for sorting by 
the convex segment method.
We digress for a moment to show how the theory that we have developed can be used
to solve two important problems (although they are not needed for sorting):
locating a point on an arbitrary segment and deciding whether two points lie on the
same connected component.

\subsection{Point location II: On an arbitrary segment}
\label{sec-pII}

Once it is known how to locate a point on a convex segment of a curve's
convex decomposition, it is straightforward to solve the more
general problem of locating a point on an arbitrary segment of the curve.
Consider a segment \arc{AB} of curve $C$ and a point $P$ on $C$.
To decide if $P$ lies on \arc{AB}, we decompose $C$ into an ordered set of convex
segments and compute the convex segments 
that contain $P$, $A$, and $B$ (say $C_{P}$, $C_{A}$, and $C_{B}$, respectively).
If $C_{P}$, $C_{A}$, and $C_{B}$ are distinct,
$P$ lies on \arc{AB} if and only if $C_{P}$ lies in between $C_{A}$ and $C_{B}$.
Degenerate cases occur if $C_{P}$, $C_{A}$, and $C_{B}$ are not distinct,
and require more subtlety.
For example, if $C_{P} = \arc{EF} = C_{A} \neq C_{B}$,
then the decision is made by sorting $P$, $A$, $E$, and $F$ along \arc{EF}, 
using Theorem~\ref{T-s}:
$P \in \arc{AB}$ if and only if the order is $E$, $P$, $A$, $F$ 
(respectively, $E$, $A$, $P$, $F$) and
\arc{AB} leaves $A$ towards $E$ (respectively, $F$).
(A method for deciding whether \arc{AB} leaves $A$ towards $E$ or $F$ 
is described in footnote~3.)
%	We assume that we are given a directed tangent at A that indicates the 
%	direction of \arc{AB} from A.
Other cases are similar.
In short, point location on an arbitrary segment is easily reducible to point
location on a convex segment.

%NO USE REPEATING THIS
%Point location could be done with a rational parameterization:
%the points that lie on the curve segment \arc{AB} are simply those
%with parameter values that occur between the parameter values of A and B.
%However, this has the same disadvantages as those of parameterizations for 
%sorting: lack of universality and inefficiency of solving the parameterization.

\subsection{Curves with many connected components}
\label{sec-cc}

It should now be clear that the convex segment method can sort points on any
algebraic curve.
In particular, it can sort points that are strewn over several connected 
components of a curve, with no more difficulty than sorting points on a single
component.
This is another advantage of the convex segment method over the parameterization
method, because it is not clear how the latter method could deal with points
on several components, even if we allow nonrational parameterizations.
%
% Harnack's Theorem~\cite{lang1} states that a nonsingular plane curve of
% genus g can have at most $g+1$ connected components.
% Therefore, a nonsingular plane curve with more than one component is nonrational.
% This doesn't say much, since not many curves are nonsingular.
% Bajaj hypothesizes that a plane curve has most connected components when it
% is nonsingular, so that this result actually applies to all curves (i.e., 
% nonsingular plane curves are the only ones you have to consider because they
% establish the maximum)
%
Would each connected component have a separate parameterization?
If so, how would the single equation of a curve produce several independent 
parameterizations?
If not, how would we determine the range of parameter values that is associated 
with each connected component?
% The second question might be answered by considering the complex curve which
% is the zero set of the curve's equation over the field of complex numbers.
% The connected components of the real curve are, in effect, the different
% components of the real plane's cross-section of the complex curve.
% (The complex curve lies in two-dimensional complex space, which is 
% of dimension four over $\Re$, so the complex curve behaves like a surface
% rather than a curve.)
% The parameterizations of the connected components of the real curve
% might be realized as the different components of the real plane's
% 'cross-section' of the (complex) parameterization of the complex curve.

A very useful test for a curve with several components is whether two points 
lie on the same connected component.
For example, with this capability it is reasonable to define an edge of a solid model 
as a particular connected component of a multicomponent curve, since the test allows 
us to restrict intersections with the curve to this connected component.
The following lemma shows that our decomposition of 
the curve into convex segments makes it simple to decide whether two points lie on
the same connected component.

\begin{lemma}
Let P and Q be two points of a curve.
If both P and Q lie on nude components, use Lemma~{\rm\ref{Ln}}.
Otherwise, let P and Q lie on convex segments \arc{AB} and \arc{CD}, respectively.
Define $v \equiv w$ if \arc{vw} is a convex segment 
or $v$ and $w$ are linked refined endpoints {\rm (}such as $V_{1}$ and $V_{2}$ in 
Example~{\rm\ref{eg-pseudo})}, and extend this relation into an equivalence relation 
on endpoints by taking its reflexive, symmetric, and transitive closure.
Then P and Q lie on the same connected component if and only if $A \equiv C$.
\end{lemma}

Two other decompositions of an algebraic curve, Collins' cylindrical algebraic 
decomposition \cite{Co75}, \cite{arnon83} and Canny's stratification \cite{Ca}, 
can also be used to separate a curve into connected components and thus decide whether
two points lie on the same connected component.
% The topology of an algebraic curve can be determined from this decomposition
% \cite{arnon83,kozen}.
% It should be straightforward to determine boundaries for each connected 
% component from this topological information.
% The weakness of these approaches is the exponential, double-exponential, or 
% parallel-exponential complexity of the algorithms that compute the 
% decompositions.

\subsection{Broad comparison of methods}

Let us compare the convex segment method of sorting with the tracing and 
parameterization methods.
The convex segment method leaps along the curve (by convex segments)
like the brute-force tracing method.
However, its jumps are large while the tracing method's jumps must be very small.
Moreover, once the convex segments have
been computed (which can be done once and for all in a preprocessing step),
each jump of the convex segment method can be done very quickly; whereas, the tracing
method must grope for some time (by applying Newton's method) to find the destination 
of each jump.
In short, the convex segment method 
makes large, bold jumps while the tracing method makes small, timid ones.

The convex segment method is similar to the parameterization method because
they both reduce the sorting problem to an easier one.
The parameterization method observes that the sorting of points 
on a line is simple and tries to unwind the curve into a line by parameterizing it.
Rather than trying to reduce the entire problem, the convex segment method divides 
the problem up into many smaller ones (viz., the sorting of points on a convex 
segment).
We shall see that the many small reductions of the convex segment method
can be done more quickly than the single, large reduction of
the parameterization method.

The convex segment method incorporates preprocessing, since the convex decomposition
of a curve can be precomputed.
As a result, the actual sorting is usually very efficient.
We might consider the parameterization of a curve to be preprocessing, but
the subsequent runtime steps (solving for the parameter value of each point)
are usually more expensive than those for the convex segment method (following pointers
and locating points).

\section{Complexity}
\label{s-c}

In this section, we analyze the complexity of the convex segment method of sorting.
We base our complexity analysis on the RAM model, where basic arithmetic operations
are of unit cost \cite{ahu}.

It must be emphasized that the $n$ of the following analysis is the degree of the curve.
This makes the analysis fundamentally different from those that we
are familiar with, such as $O(n \log n)$ for sorting numbers (where
$n$ is the number of points) or $O(n \log \log n)$ for triangulating a simple
polygon (where $n$ is the number of edges of the polygon).
For example, in the following analysis, $n$ is the constant 1 for all polygons.
As a result, the complexity of an operation such as the convex decomposition of an
algebraic curve can be misleading, since it is very easy (although wrong) to 
compare it with familiar complexities of discrete (rather than continuous) algorithms 
such as number sorting or polygon manipulation.

It should also be noted that the following analysis is pessimistic.
The worst-case time will be reached only by the most pathological curves: the time to 
decompose and sort points on curves that arise in practice 
in geometric modeling is much more reasonable.
This is a major reason why complexity analyses of solid modeling algorithms are rare
and more valuable for their insight into the algorithm than their reflection of its
performance.
For example, a typical endpoint will lie on the boundary of a single-segment cell and its 
partner will be computed in $O(1)$, not $O(k \beta[n])$, time (see below).
In particular, the $O(n^{6}\beta[n])$ term in Theorem~\ref{thm-5} will often be
closer to $n^{3}$ than $n^{6}\beta[n]$ in reality.
This observation has been borne out in practice, with the testing of the algorithms on
various curves (see \S~\ref{data}).
The efficiency will be even further improved by the fact that the singularities and 
flexes of the curve, which are important to other geometric algorithms, may already 
be available in many cases.

\clearpage

\subsection{Complexity of convex decomposition}

\ \ \ 

\vspace{.1in}

\begin{theorem}
\label{thm-5}
A curve of degree $n$ {\rm (}a curve whose defining polynomial is degree $n${\rm )} 
can be decomposed into convex segments in
O{\rm (}$\beta[n^{2}] + n^{2}\beta[{\rm MAX} * n] + n^{6}\beta[n]${\rm )} time, 
where $\beta[n]$ is the time required to find the 
real roots of a univariate polynomial equation of degree $n$, and {\rm MAX} is the 
maximum number of quadratic transformations that are necessary to decompose any 
singularity of the curve into simple points.\footnote{MAX is 1 if each 
	singularity has distinct tangents, and MAX will usually be 1 or 2
	in geometric modeling applications. }
\end{theorem}

\begin{proof}
{\em Computation of singularities, flexes.} 
%
Consider the curve $f(x,y) = 0$ of degree $n$.
Its singularities are found by solving the simultaneous system
of equations \mbox{$\{f_{x} = 0, f_{y}=0, f = 0\}$}.
One method is to use resultants \cite{walker}.
The resultant of two polynomials with respect to the variable $x_{n}$ is a polynomial
whose roots are the projection onto the hyperplane $x_{n} = 0$
of the intersections of the two polynomials.
Let $X$ (respectively, $Y$) be the real roots of the resultant of $f_{x}$ and $f_{y}$ 
with respect to $y$ (respectively, $x$), which is a univariate polynomial in $x$ 
(respectively, $y$)
of degree $O(n^{2})$.\footnote{Since 
	singularities at infinity are not of interest, those roots in $X$ (respectively, $Y$)
	that cause the terms of highest degree of \mbox{$\{f_{x} = 0, f_{y}=0\}$} to 
	simultaneously vanish are not of interest.
	(The terms of highest degree of a polynomial are intimately related to its 
	solutions at infinity, since they dominate the polynomial as solutions get large.)
	Therefore, before computing the roots of the resultant, the 
	greatest common denominator of the leading term
	polynomials of $f_{x}$ and $f_{y}$ is computed and divided out of the 
	resultant, all in $O(n\log^{2}\ n)$ time \cite{ahu}.}
$X$ (respectively, $Y$) is the collection of abscissae (respectively, ordinates)
of the solution set of \mbox{$\{f_{x} = 0, f_{y}=0\}$}.
$X$ (and $Y$) can be computed in $O(n^{4}\ \log^{3}\ n + \beta[n^{2}])$ time, since
the resultant of a pair of polynomials of degree at most $n$ in $r$ variables 
can be computed in $O(n^{2r}\ \log^{3}\ n)$ time \cite{bajj}.
% O($n^{5}\ log\ n + \beta[n^{2}]$) time, since
% the resultant of a pair of polynomials of degree at most $d$ in $v$ variables 
% can be computed in O($d^{2v+1}\ log\ d$) time \cite{col}.
The singularities of the curve are
\mbox{$\{\ (x,y)\ :\ x\in X,\ y \in Y \mbox{ and } f(x,y) = f_{x}(x,y) = f_{y}(x,y) = 
0\ \}$.}
This pairwise substitution takes $O(n^{6})$ 
time, since $X$ and $Y$ are each of size $O(n^{2})$ and the evaluation of an equation of 
degree $n$ requires $O(n^{2})$ time.
Hence, all singularities of the curve can be computed in $O(\beta[n^{2}] + n^{6})$
time.
With similar techniques, the flexes can also be computed in $O(\beta[n^{2}] + n^{6})$
time.

{\em Computation of their tangents.} 
%
Recall from \S~\ref{ssc} that the tangents at a singularity ($a, b$) are computed by 
translating the singularity to the origin and factoring the polynomial consisting of 
the terms of lowest degree of the translated $f(x,y)$ into linear factors.
(For example, the lines $y-x=0$ and $y+x=0$ are the tangents of the curve
$y^{2} + x^{3} - x^{2} = 0$ at the origin.)
A translation is simply a linear substitution $x_{t} = x - a ,\ y_{t} = y - b$,
which takes $O(n^{4})$ time for a bivariate equation of degree $n$.
% Proof: there are O(n^4) terms to compute: there are O(i^3) terms generated by 
% the terms of degree i: consider the terms generated by the terms of degree n:
% there are n terms for (x-a)^n, n-1 terms for (x-a)^n-1 * y, (n-2)*2 terms for
% (x-a)^n-2 * y^2, and in general (n-i) * i terms for (x-a)^n-i * y^i, which sums
% to O(n^3) terms generated by the terms of degree n
The factorization of a homogeneous bivariate polynomial is equivalent to the
solution of a univariate polynomial.
Therefore, the computation of the tangents at a singularity requires
$O(n^{4} + \beta[n])$ time.
A curve of degree $n$ has at most $O(n^2)$ singularities \cite{walker},
so all of the tangents at singularities can be computed in $O(n^{6} + n^{2}\beta[n])$
time.
The computation of the tangent at a flex is easier, only involving the $O(n^{2})$
operation of bivariate (or homogeneous trivariate) polynomial evaluation 
(\S~\ref{ssc}).
A curve of degree $n$ also has at most $O(n^{2})$ flexes \cite{walker},
so all of the tangents at flexes can be computed in $O(n^{4})$ time.

{\em Computation of intersections of singularity/flex tangents with curve.}
%
The intersections of the singularity/flex tangents with the curve are needed to
create the convex decomposition.
Consider the number of tangents.
There are at most $O(n^{2})$ tangents at flexes.
A curve of order $n$ has at most $(n-1)(n-2)/2$ double points, where
a singularity of multiplicity $t$ counts as $t(t-1)/2$ double points 
and has $O(t)$ tangents \cite{walker}.
Consequently, there are $t / t(t-1) / 2 < 2$ tangents per double point,
or at most $O(n^{2})$ singularity tangents.
The intersection of a tangent with the curve involves a linear substitution
and a solution of the resulting polynomial, thus $O(n^{4} + \beta[n])$ time or
$O(n^{6} + n^{2}\beta[n])$ for all tangents.
Note that the $O(n^{2})$ tangents generate $O(n^{3})$ endpoints on the curve, since
each tangent intersects the curve in at most $n$ points (Bezout's theorem).

{\em Refinement of singularities and infinite segments.}
%
A singularity of multiplicity $t$ is refined into $O(2t)$ endpoints, meaning
$2t / t(t-1) / 2 \leq 4$ refined endpoints per double point,
or a total of $O(n^{2})$ refined endpoints at singularities.
Thus, the number of endpoints of convex segments 
(and thus the number of convex segments) remains $O(n^{3})$ after refinement.
Consider the time that is required to refine the singularities.
Each singularity is translated to the origin and subjected to quadratic transformations
(perhaps translating the singularity back to the origin after certain quadratic 
transformations).
$O(n^{2})$ quadratic transformations are sufficient to reduce all of the singularities 
to simple points, since the singularities of a curve of degree $n$ account in total for 
$O(n^{2})$ double points and the application of each quadratic transformation 
removes at least one double point, in a global amortized counting \cite{abba3}. 
We have seen that the translation of a curve requires $O(n^{4})$ time, 
amounting to a total of $O(n^{6})$ translation time. 
% By suitable choice of coordinates, it can be arranged that as long as the 
% multiplicity of a singularity is not reduced on application of a quadratic 
% transformation, no translation needs to be performed.
% Such a choice of coordinates (which takes $O(n^{4})$ time) can be determined 
% {\em a priori} for each singularity of the curve \cite{abhy}.
% As this is going to be applied for at most $O(n^{2})$ singularities representing
% O($n^{2}$) double points,
% the total time taken by all of the translations is O($n^{6}$) time. 
Each quadratic substitution $x = x_{1},\ y = x_{1}y_{1}$ takes $O(n^{2})$ time 
(there are $O(n^2)$ terms in the original equation of the curve).
Therefore, all of the quadratic transformations take $O(n^{4})$ time.

During the reduction of a singularity to simple points, each quadratic
transformation can increase the degree of the curve's equation, since
$x^{i}y^{j}$ becomes $x^{i}(x^{j-d}y^{j}) = x^{i+j-d}y^{j}$, where $d$ is the 
multiplicity of the singularity.\footnote{It might appear that 
	$x^{i}y^{j}$ should become $x^{i}(x^{j}y^{j})$.  However,  
	redundant factors must be removed from the polynomial.
	For example, $x^{2} - y^{3} = 0$
	becomes $1-xy^{3} = 0$, not $x^{2} - x^{3}y^{3} = 0$.
	The equation of a curve with a singularity of multiplicity $d$ 
	at the origin has no terms of degree less than $d$, so a factor of $x^{d}$ can
	always be removed.}
In other words, the degree of the polynomial can increase by $O(j)$, where
$j$ is the highest degree of $y$ in any term of the polynomial undergoing 
quadratic transformation. 
Since $j=n$ for the polynomial of the original curve and the $y$-degree of every term
remains invariant under quadratic transformation (and does not increase under
translation of the curve either), the degree of the polynomial can only increase
by $O(n)$ with each quadratic transformation.
Therefore, by the end of the reduction of a singularity to simple points, 
the curve's equation can be at most degree $O({\rm MAX} * n)$.

Finally, after a quadratic transformation where the multiplicity of the singularity 
drops, we compute the intersections of the new curve of degree $i$ with the $y$-axis, 
which takes $\beta[i]$ time.
Again, since this is computed after at most $O(n^{2})$ quadratic transformations,
the total time taken by all of the intersection computations is at most 
$O(n^{2} \beta[{\rm MAX} * n])$ time.
% Can't get asymptotically better even if you analyze more carefully by considering
% \frac{n^2}{MAX} iterations of \beta[n] + \beta[2n] + ... + \beta[(MAX+1)n].
We conclude that a bound on the time for refining the
convex segment endpoints at singularities is $O(n^{6} + n^{2} \beta[{\rm MAX} * n])$.
The refinement of infinite segments is simple compared to the refinement of singularities.

{\em Pairing endpoints.}
%
Consider the time required to compute the partners of the $O(n^{3})$ endpoints.
The dominating expense is the computation of the set $R(\wo)$ of 
Theorem~\ref{Tpner} for each endpoint \wo.
It takes $O(k\beta[n])$ time to compute $R(\wo)$ for an endpoint in a cell
with $k$ endpoints, $O(k^{2}\beta[n])$ time to compute $R(\wo)$ for every 
endpoint in a cell with $k$ endpoints, and $O(\sum k_{i}^{2}\beta[n])$
time to compute $R(\wo)$ for every endpoint in every cell, where $k_{i}$ is the
number of endpoints in cell $C_{i}$ and the sum is over all cells $C_{i}$.
Since $\sum k_{i} = O(n^{3})$, $O(\sum k_{i}^{2}\beta[n]) = O(n^{6}\beta[n])$.
Therefore, partner computation takes $O(n^{6}\beta[n])$ time.
\end{proof}

\subsection{Complexity of sorting}

We now consider the complexity of sorting points along a curve after its convex
decomposition is available.
This sorting is usually very efficient, because the traversal of a curve by
convex segments has been reduced to the traversal of a doubly linked list, and
it is usually simple to find the points on each convex segment.
Once again, the following worst-case analysis is unrealistically pessimistic
for geometric modeling applications.

\begin{theorem}
After the curve has been decomposed into convex segments,
$m$ points on a plane algebraic curve of degree $n$ can be sorted
by the convex segment method in $O(m n^{3} \beta[n] + m\log m)$ time.
\end{theorem}
%
\begin{proof}
The dominating expense of sorting is to locate every point on a convex segment,
since the convex segments are already implicitly sorted (by endpoint pairing) and 
the sorting of points along a convex segment is simple
(by Theorem~\ref{T-s}, it is equivalent to the $O(k \log k)$ operation of finding
and sorting a set of angles). 
A point can easily be located in the proper cell of the cell partition, as follows.
A vector of size $O(n^{2})$ is associated with each of the $m$ points and each
cell: this vector specifies the side (inside or outside) of each singularity/flex tangent
that the point or cell lies on.
A point lies in a cell if and only if their two vectors match.\footnote{The vector
	of a cell need not, and will not, be complete.  Only the entries for the
	cell's walls are necessary.}
Therefore, the only potentially challenging step is locating the convex segment
that contains the point.

In the worst case, it requires $O(k \beta[n])$ time to compute the set 
S($x$) of Theorem~\ref{Tps} for a point in a cell with $k$ endpoints, since
the intersection of line segments with the curve is required.
There are $O(n^{3})$ endpoints, so point location requires $O(n^{3}\beta[n])$ time per
point and $O(mn^{3}\beta[n])$ time for all points.\footnote{Observe the worst-case
	pessimism of this analysis, which assumes that each point is located in a cell
	with $O(n^{3})$ endpoints.
	It is unlikely that there are $O(n^{3})$
	real endpoints, since many of the $n$ intersections of a singularity/flex
	tangent with the curve will be complex.  
	Moreover, it is extremely unlikely that $O(n^{3})$
	of these endpoints lie in the same cell and that all points to be sorted
	lie in such a cell.
	Finally, it is very unlikely that each point location in this cell will 
	require the more expensive conditions
	of $S(x)$ to test the membership of all $O(n^{3})$ endpoints in $S(x)$.}
After adding $O(m \log m)$ time for sorting the points along the convex segments,
the convex segment method requires worst-case $O(mn^{3}\beta[n] + m \log m)$ 
time to sort $m$ points by traversing $O(n^{3})$ convex segments.
\end{proof}

\section{Execution times}
\label{data}
This section presents execution times for the sorting of some representative
curves by the convex segment and parameterization methods.
These empirical results are a good complement to the complexity analysis 
of \S~\ref{s-c}, since they capture the expected case, rather than 
the worst-case, behavior of the methods.
The source code was written in Common Lisp and
execution times are in seconds on a Symbolics Lisp Machine, 
not including time for disk faults and garbage collection.
Times for the convex segment method are the average of 12 trials, 
while times for the parameterization method are the average of three trials.
Preprocessing time is the time required to create the cell partition 
and find the partners of all of the endpoints.
Five curves are examined: two rational cubics and three nonrational quartics.

We do not consider the time required to find a parameterization of the
curve or to find the flexes and singularities of the curve.
Each of these computations is a preprocessing step that is 
entirely independent of sorting, and often the parameterization, singularities, 
and flexes of a curve will already be available.
Moreover, the computation of a curve's parameterization is of approximately the same
complexity as the computation of a curve's singularities and flexes,
so our comparison of sorting methods should not be biased.

The first example illustrates the superiority of the convex segment method:
even when the preprocessing time is added to the sorting time, it is more efficient.
Also notice that the rate of growth of the convex segment method is much smaller.
The inferiority of the tracing method (see the end of \S~\ref{sp})
is obvious from this example, and we do not consider it further.

\vspace{.2in}

\begin{example}
{\samepage
A semi-cubical parabola.\\
\indent Equation of the curve: $27 y^{2} - 2x^{3} = 0$.\\
\indent Preprocessing time: 0.27 seconds.\\
\indent Parameterization: \{$x(t) = 6t^{2}$, $y(t) = 4t^{3}\ :\ 
t \in (-\infty, +\infty)$\}.

}
\begin{table}[h]
\caption{Semi-cubical parabola.}
\begin{center}
\begin{tabular}{|l|c|c|c|}
\hline
Number of sortpoints & 1 & 2 & 6 \\ \hline \hline
Convex segment &           .01 & .03 & .03 \\ \hline
%\footnote{The results should 
%only be compared vertically, not horizontally.
%The reason that sorting times sometimes decrease as more points are sorted
%is that completely different sets of points may be used in each column
%(e.g., the five points of column 3
%are not a subset of the six points of column 4)
%or different start and end points may be used.
%Thus, for example, sorting many points that are close together on a short
%sort segment may be faster than sorting a few points that are spread out
%on a long sort segment.} 
Convex segment + preprocessing & .28 & .30 & .30 \\ \hline
Parameterization & .47 & .63 & 1.04 \\ \hline
Tracing         & 3.14 & 2.89 & 4.77 \\ \hline
\end{tabular}
\end{center}
\end{table}
\end{example}

The second example illustrates the tradeoff between a very fast sort 
that requires preprocessing (convex segment method) and a moderately fast 
sort that does not require preprocessing (parameterization method).

\vspace{.2in}

\begin{example}
\label{eg-folium}
{\samepage
Folium of Descartes.\\
\indent Equation of the curve: $x^{3} + y^{3} - 15xy = 0$.\\
\indent Preprocessing time: 2.81 seconds.\\
\indent Parameterization: \{$x(t) = \frac{15t}{1+t^{3}}$, $y(t) = 
\frac{15t^{2}}{1+t^{3}}\ :\ t \in (-\infty, +\infty)$\}.

}
\begin{table}[htbp]
\caption{Folium of Descartes.}
\begin{center}
\begin{tabular}{|l|c|c|c|c|} \hline
number of sortpoints & 1 & 2 & 5 & 9 \\ \hline \hline
convex segment &           0.01 & 0.01 & 0.05 & 0.04 \\ \hline
convex segment + preprocessing & 2.82 & 2.82 & 2.85 & 2.85 \\ \hline
parameterization & 1.01 & 1.07 & 1.76 & 3.17 \\ \hline
\end{tabular}
\end{center}
\end{table}
\end{example}

The remaining three curves are nonrational, so they are only 
sorted with the convex segment method.

\vspace{.2in}

\begin{example}
\label{eg-devil}
{\samepage
Devil's curve (with several connected components).\\
\indent Equation of the curve: $y^{4} - 4y^{2} - x^{4} + 9x^{2} = 0$.\\
\indent Preprocessing time: 2.20 seconds.

}
\clearpage

\begin{table}[htbp]
\caption{Devil's curve.}
\begin{center}
\begin{tabular}{|l|c|c|c|} \hline
number of sortpoints & 1 & 4 & 7 \\ \hline \hline
convex segment & 0.09 & 0.09 & 0.10 \\ \hline
convex segment + preprocessing & 2.29 & 2.29 & 2.30 \\ \hline
\end{tabular}
\end{center}
\end{table}
\end{example}

\begin{example}
\label{eg-limacon}
{\samepage
Limacon.\\
\indent Equation of the curve: $x^{4} + y^{4} + 2x^{2}y^{2} - 12x^{3} - 12xy^{2} + 
	27x^{2} - 9y^{2} = 0$.\\
\indent Preprocessing time: 4.62 seconds.

}
\begin{table}[htbp]
\caption{Limacon.}
\begin{center}
\begin{tabular}{|l|c|c|c|} \hline
number of sortpoints & 2 & 5 & 8 \\ \hline \hline
convex segment & .09 & .30 & .55 \\ \hline
convex segment + preprocessing & 4.70 & 4.92 & 5.17 \\ \hline
\end{tabular}
\end{center}
\end{table}
\end{example}

\begin{example}
\label{eg-Cassinian}
{\samepage
Cassinian oval.\\
\indent Equation of the curve: $x^{4} + y^{4} + 2x^{2}y^{2} + 50y^{2} - 50x^{2}-671 = 0$.\\
\indent Preprocessing time: 5.36 seconds.

}
\begin{table}[htbp]
\caption{Cassinian oval.}
\begin{center}
\begin{tabular}{|l|c|c|c|} \hline
number of sortpoints & 2 & 4 & 6 \\ \hline \hline
convex segment & .14 & .17 & .19 \\ \hline
convex segment + preprocessing & 5.50 & 5.53 & 5.55 \\ \hline
\end{tabular}
\end{center}
\end{table}
\end{example}

\vspace{.2in}

We finish this section by considering the relative merits of the parameterization and 
convex segment methods of sorting.
Certain curves cannot, or should 
not, be sorted by the parameterization method: curves that 
do not possess a rational parameterization and curves for which
a rational parameterization cannot be efficiently obtained.
Therefore, the convex segment method is often 
the only viable way to sort points along an algebraic curve.

For those curves that can be sorted in either way, the convex segment method 
is generally far more efficient than the parameterization method at the actual 
sorting of the points.
However, the parameterization method does not have the expense of 
preprocessing that the convex segment method does.
Therefore, when only a few points need to be sorted (over the entire lifetime 
of the curve) and the sorting of these points must be done soon after the 
definition of the (rational) curve, the parameterization method will usually 
be the method of choice.
(However, we have seen an example where the convex segment method is superior
to parameterization even when we include preprocessing time.)
The expense of preprocessing will be warranted whenever sorting time is a 
valuable resource, as in a real-time application, or when the number of points 
that will be sorted is large.
The convex segment method will also be preferable when the curve is defined 
long before it is ever sorted (as with a complex solid model that requires 
several days, weeks, or even months to develop), since the preprocessing can 
be done at any time that processing time becomes available before the sort.
We conclude that the convex segment method is an effective new method for
sorting points along an algebraic curve, and that in many situations it is 
either the only or the best method.

\section{Conclusions}
\label{sco}
We have developed a new method of sorting points along an algebraic curve
that is superior to the conventional methods of sorting.
Many curves that could not be sorted, or that could only be sorted slowly,
can now be sorted efficiently.
% The convex segment method of sorting can even be extended to curves defined by an
% arbitrary non-polynomial function, by truncating the power series expansion
% (e.g., Taylor series expansion) of the function about the points to be sorted,
% thereby yielding piecewise algebraic curve approximations to the original curve.
%%%%  The approximations are controlled by the number of terms considered in the 
%%%%  truncated expansion.
The development of our new method has also illustrated how an algebraic curve can 
be decomposed into convex segments, how to locate points on segments 
of algebraic curves, and how to decide whether two points lie on the same
connected component of an algebraic curve.
These results are of interest in a more general context than sorting.

This work is one of the first solutions
of a computational geometry problem that is applicable to curves of arbitrary degree.
Methods are usually restricted to curves/surfaces of some specific or 
bounded degree, such as polygons/polyhedra or quadrics.
The creation and manipulation of curves and surfaces is of major importance 
to geometric modeling.
A sophisticated geometric modeling system should offer 
a rich collection of tools to aid this manipulation.
This paper has been an examination of one of these tools.
The progress of geometric modeling depends upon the development of more tools and upon 
the extension of more computational geometry algorithms from polygons to curves and 
surfaces of higher degree.

\section{Appendix}
\label{sec-append}
\subsection{Constructing artificial walls}
\label{sec-caa}
Section~\ref{sec-refine2} defined the artificial walls of an unbounded cell,
and discussed the properties that these walls must satisfy.
We now give an algorithm for constructing the artificial walls of an unbounded
cell $C$.
Recall that artificial walls must be chosen so that 
(i)~they intersect any infinite convex segment in $C$ exactly once,
(ii)~they do not intersect any finite convex segments, and
(iii)~the resulting artificially bounded cell is convex.
Let $P$ and $Q$ be two extremal points on the boundary of $C$, 
such that all original endpoints
on $C$'s cell boundary lie between $P$ and $Q$ (Fig.~\ref{fig-ab}).
The artificial boundary that we construct shall consist of two or three walls:
a wall touching $P$, a wall touching $Q$, and occasionally a wall joining these two walls.
We call these added walls $W_{P}$, $W_{Q}$, and $W_{PQ}$, respectively.
We must show how to create $W_{P}$, $W_{Q}$, and $W_{PQ}$.

Consider the set of tangents from points of the curve inside $C$.
Let $T$ be the finite subset consisting of tangents that intersect $P$
and make a larger angle than \ray{PQ} with $P$'s wall 
(measuring from the side of $P$'s wall that contains endpoints).
(A method for finding these tangents is discussed below.)
If $T = \emptyset$ and $P$ and $Q$ lie on different walls,
$W_{P}$ can be $\seg{PQ}$ (Fig.~\ref{fig-ab}(b)), 
because if a curve segment crosses \seg{PQ} 
twice, then the tangent of some point outside of \seg{PQ} intersects $P$ and $Q$.
\Comment{
	If $T = U = \emptyset$ and 
	P and Q lie on the same wall, then three artificial walls are necessary:
	the three walls that complete a square with \seg{PQ} as one side 
	(Fig.~\ref{fig-ab}(a)).
}
If $T = \emptyset$ and $P$ and $Q$ lie on the same wall, 
then $W_{P}$ can be the normal to $P$'s wall.
If $T \neq \emptyset$, $W_{P}$ will be
the tangent in $T$ that makes the largest angle with $P$'s wall,
but rotated about $P$ so that
it makes an even larger angle with $P$'s wall, thus avoiding the creation of a redundant
new endpoint at the point of tangency with the curve (Fig.~\ref{fig-ab}(a)).
$W_{Q}$ is created in an entirely analogous way.
The only wall remaining to define is $W_{PQ}$, the wall joining $W_{P}$ and $W_{Q}$.
If $W_{P}$ and $W_{Q}$ intersect inside the cell $C$, then no $W_{PQ}$ is needed.
% (Fig.~\ref{fig-ab}(c)).
Otherwise, let $V$ be the tangents (from points of the curve inside $C$) that are
parallel to \seg{PQ}.
If $V = \emptyset$, let $W_{PQ}$ be an arbitrary line segment connecting $W_{P}$
and $W_{Q}$.
If $V \neq \emptyset$, $W_{PQ}$ will be a segment of the tangent in $V$ 
that lies furthest from \seg{PQ}, translated some distance away from \seg{PQ}
(Fig.~\ref{fig-ab}(c)).

\figg{fig-ab}{Constructing artificial walls.}{5.75in}

As part of the above construction, it is necessary to compute the tangents of nonsingular
points that intersect a given point, as well as those that are parallel to a given line.
The tangent at the nonsingular point $\alpha$ of the plane
curve \mbox{$f(x,y,w)=0$} (where $w$ is the homogeneous coordinate, placing the curve in
projective space) is $\sum_{i=1}^{3} f_{x_{i}}(\alpha) x_{i}$, where $x_{1} = x$,
$x_{2} = y$, $x_{3} = w$, and $f_{x_{i}}$ is the derivative of $f$ with respect to $x_{i}$
\cite{walker}.
% [p. 55]
Thus, the nonsingular points $\alpha = (\alpha_{1}, \alpha_{2}, 1)$ of a curve 
\mbox{$f(x,y,w)=0$}
whose tangents intersect an arbitrary point $P = (p_{1},p_{2},p_{3}) = (p_{1},p_{2},1)$
of the plane can be computed by solving the pair of equations 
$\{f(\alpha) = 0, \sum_{i=1}^{3} f_{x_{i}}(\alpha) p_{i} = 0 \}$ 
for $\alpha_{1}$ and $\alpha_{2}$, and eliminating singularities.
For the second problem, note that the slope of the tangent at $\alpha$ is 
$-f_{x}(\alpha) / f_{y}(\alpha)$, unless the line is vertical in which case 
$f_{y}(\alpha) = 0$.
Thus, the nonsingular points $\alpha = (\alpha_{1}, \alpha_{2}, 1)$ of a curve $f(x,y,w)=0$
whose tangents are parallel to the vector $(a,b)$
can be computed by solving the pair of equations 
$\{f(\alpha) = 0,\ a f_{x}(\alpha) + b f_{y}(\alpha) = 0 \}$ 
for $\alpha_{1}$ and $\alpha_{2}$ and eliminating singularities.

% ***************************************

\subsection{Three key lemmas}

We present three important lemmas in this section.
The first lemma is used in the proof of the point location theorem (Theorem~\ref{Tps}),
as well as the proof of the second lemma below.
The second lemma is crucial in the proofs of both of the main theorems 
(Theorems~\ref{Tpner} and \ref{Tps}).
The third lemma is used in Theorem~\ref{Tpner}.

\begin{lemma}
\label{lem-argh}
Let $x$ be any nonsingular point of the curve $F$ in cell C, and
let $\SSo(x)$ = \mbox{\{endpoints $W$ in C $\mid$}

% IF YOU ADD A CONDITION, CHANGE NUMBERING IN LAST LINE OF first paragraph of PROOF
{\rm (1)} \mbox{$\#\{P \in\seg{xW}\cap F: P \mbox{ faces x}\} =
\#\{P \in \seg{xW}\cap F: P \mbox{ faces W} \}$},

{\rm (2)} For all $\alpha \in \seg{xW}$,
$\#\{P \in\seg{x\alpha}\cap F: P \mbox{ faces x}\} \leq
\#\{P \in \seg{x\alpha}\cap F: P \mbox{ faces W}\} $\}.

\vspace{.05in}

\noindent {\rm (}Note that 
$\SSo(x)$ is a superset of the $S(x)$ of Theorems~{\rm \ref{Tpner}} and {\rm \ref{Tps}.)}
If $s \in \SSo(x)$, then neither $s$'s convex segment nor $x$'s convex 
segment can cross \seg{sx}.
\end{lemma}
\begin{proof}
Suppose that $s$'s convex segment crosses \seg{sx} at $y$ (Fig.~\ref{fig-27a}).
Then, using Lemma~\ref{Ls}, $\#\{P \in \seg{sy} \cap F: \mbox{ $P$ faces $s$}\} = 
\#\{P \in \seg{sy} \cap F: \mbox{ $P$ faces $y$}\} =$
\mbox{$\#\{P \in \seg{sy} \cap F: \mbox{ $P$ faces $x$}\}$}.
Since $y$ faces $s$ and $y \not\in \seg{sy}$, 
if we choose $\alpha \in \seg{yx}$ such that \seg{y\alpha} 
does not contain any intersections with the curve, 
$\#\{P \in \seg{s\alpha} \cap F: \mbox{ $P$ faces $s$}\} >
\#\{P \in \seg{s\alpha} \cap F: \mbox{ $P$ faces $x$}\}$.
Since $s \in \SSo(x)$, we know that 
$\#\{P \in \seg{sx} \cap F: \mbox{$P$ faces $s$}\} =
\#\{P \in \seg{sx} \cap F: \mbox{ $P$ faces $x$}\}$.
Therefore, the above inequality 
becomes $\#\{P \in \seg{\alpha x} \cap F: \mbox{ $P$ faces $s$}\} <
\#\{P \in \seg{\alpha x} \cap F: \mbox{ $P$ faces $x$}\}$, 
which contradicts $s \in \SSo(x)$ (condition (2)).

\figg{fig-27a}{$s$'s segment cannot cross \seg{sx}.}{1.25in}

The proof for $x$'s convex segment is similar.
Suppose that $x$'s segment crosses \seg{sx} at $y$.
% A similar proof to the one for $s$'s segment, only simpler, establishes the contradiction.
Then $\#\{P \in \seg{xy} \cap F: \mbox{ $P$ faces $x$}\} = 
\#\{P \in \seg{xy} \cap F: \mbox{ $P$ faces $y$}\} = 
\mbox{$\#\{P \in \seg{xy} \cap F$: $P$ faces $s$\}}$.
Since $y$ faces $x$, if we choose $\alpha \in \seg{ys}$ such that \seg{y\alpha} 
does not contain any intersections with the curve, 
$\#\{P \in \seg{x\alpha} \cap F: \mbox{ $P$ faces $x$}\} 
> \mbox{$\#\{P \in \seg{x\alpha} \cap F$: $P$ faces $s$\}}$,
in contradiction of $s \in \SSo(x)$.
\end{proof}

\begin{lemma}
\label{lem-645}
Let $x$ be any nonsingular point of the curve $F$ in cell C,
and let $z$ be any point such that $z$ lies on $x$'s convex segment
and \arc{xz} does not contain any endpoints in its interior 
{\rm (}Fig.~{\rm \ref{fig645})}.
Let $\SSt(x)$ = \{endpoints $W$ in C $\mid$

{\rm (1)} $W$ faces $x$,

{\rm (2)} \mbox{$\#\{P \in\seg{xW}\cap F: P \mbox{ faces x}\} =
\#\{P \in \seg{xW}\cap F: P \mbox{ faces W} \}$},

{\rm (3)} For all $\alpha \in \seg{xW}$,
$\#\{P \in\seg{x\alpha}\cap F: P \mbox{ faces x}\} \leq
\#\{P \in \seg{x\alpha}\cap F: P \mbox{ faces W}\} $\}.

\vspace{.05in}

\noindent {\rm (}Note that 
$\SSt(x)$ is a superset of the $S(x)$ of Theorems~{\rm \ref{Tpner}} and {\rm \ref{Tps}.)}
Then \ray{xs} does not cross $\arc{xz} \setminus \{z\}$ 
for any $s \in \SSt(x)$.
\end{lemma}
%
\figg{fig645}{\ray{xs} does not cross \arc{xz}.}{2.6in}
%
\begin{proof}
Let $s \in \SSt(x)$.
Suppose, for the sake of contradiction, that $\ray{xs}$ crosses 
$\arc{xz} \setminus \{z\}$ at $y$.
\label{thpr}
There are three cases to consider: $y \in \seg{xs}$, $y = s$, and $y \not\in \seg{xs}$
(i.e., $s \in \seg{xy}$).
$y \in \seg{xs}$ contradicts Lemma~\ref{lem-argh}.
$y = s$ is also contradictory, since \arc{xz} does not contain any endpoints in its 
interior.
$s \in \seg{xy}$ is the only nontrivial case (Fig.~\ref{66}).
Recall an argument used at the beginning of Lemma~\ref{Ls}:
since \arc{xy} is part of a convex segment lying in a cell
and $x$ and $y$ are nonsingular points,
the points of entry and departure of the curve into
the closed region bounded by $\seg{xy}$ and \arc{xy}
must be along $\seg{xy}$ and must pair up into couples.
In particular, $s$ must pair with another point, say $t \in \seg{xy}$.
We shall use $t$ to develop a contradiction of Lemma~\ref{Ls} for the convex 
segment \arc{xy}.
Since \arc{st} is convex, $s$ faces $t$; and since $s \in \SSt(x)$, $s$ also faces $x$. 
Therefore, $t \in \seg{xs}$.
Since \mbox{$s \in \SSt(x)$}, 
\[ \#\{P \in \seg{xs} \cap F: \mbox{ $P$ faces $x$}\}
= \#\{P \in \seg{xs} \cap F: \mbox{ $P$ faces $s$}\}. \]
Noting that $\seg{xs} = \seg{xt}\ \cup\ \seg{ts}\ \cup\ \{t\}$ and 
$t$ faces $s$, this becomes 
%
\vspace{.1in}
\begin{tabbing}
\hspace{.7in} \= $\#\{P \in \seg{xt} \cap F: \mbox{ $P$ faces $x$}\} +
\#\{P \in \seg{ts} \cap F: \mbox{ $P$ faces $x$}\} + 0$ \\
\nopagebreak
\hspace{.5in} $=$ \> $\#\{P \in \seg{xt} \cap F: \mbox{ $P$ faces $s$}\} +
\#\{P \in \seg{ts} \cap F: \mbox{ $P$ faces $s$}\} + 1.$
\end{tabbing}
\vspace{.1in}
%
Moreover, by Lemma~\ref{Ls},
%
\vspace{.05in}
\begin{tabbing}
\hspace{.7in} $\#\{P \in \seg{ts} \cap F: \mbox{ $P$ faces } s\}$ 
   \= $= \#\{P \in \seg{ts} \cap F: \mbox{ $P$ faces } t\}$ \\
\> $= \#\{P \in \seg{ts} \cap F: \mbox{ $P$ faces } x\}$.
\end{tabbing}
\vspace{.05in}
%
Upon cancelling terms in the above equation, we conclude that
%
\vspace{.05in}
\begin{tabbing}
\hspace{.7in} $\#\{P \in \seg{xt} \cap F: \mbox{ $P$ faces } x\}$ \=
   $> \#\{P \in \seg{xt} \cap F: \mbox{ $P$ faces } s\}$ \\
\> $= \#\{P \in \seg{xt} \cap F: \mbox{ $P$ faces } y\}$.
\end{tabbing}
\vspace{.05in}
%
But this contradicts condition (3) of Lemma~\ref{Ls} 
($\x = x$, $\y = y$).
These contradictions lead us to conclude that \ray{xs} does not cross 
$\arc{xz} \setminus \{z\}$.
\end{proof}

\figg{66}{$s \in \seg{xy}$.}{1.25in}

\begin{corollary}
\label{cor-645}
Let $x$, $z$, C and $\SSt(x)$ be as in the previous lemma.
Let $z'$ be the intersection of \ray{xz} with the boundary of C,
let $\SSt'(x) =$ \{$s'$:$s'$ is the intersection of \ray{xs} with the 
boundary of C, $s \in \SSt(x)$\},
and let $x_{1},x_{2}$ be the intersections of $x$'s tangent with the cell boundary
{\rm (}Fig.~{\rm\ref{fig645})}.
If $\SSt'(x)$ is sorted along the boundary of C from $x_{1}$ to $x_{2}$,
then $z'$ is either the first or the last element.
\end{corollary}

\vspace{.2in}

\begin{lemma}
\label{Ll}
{\rm \cite[p. 119]{jj}}.
% Lemma B.8, p. 119
Let \wo\ and \wt\ be partners.
If \wt\ lies on \wo's tangent, then \wo\ must be a flex.
In other words, if \wt\ lies on \wo's tangent, 
then \wo\ and \wt\ lie on the same cell wall.
\end{lemma}

\subsection{Computing the infinite direction from a point}

In order to compute the type of an infinite convex segment (\S~\ref{sec-pI}), 
it is necessary to compute the infinite direction from a point of that infinite segment.
As seen in Theorem~\ref{Tps}, for our purposes the point will either be an artificial
endpoint or a point outside the bounded cell.
The following lemma shows how to perform this computation.

\begin{lemma}
\label{lem-ty}
Let $x$ be a point that lies on an infinite convex segment,
and let $T_{x}^{+}$ and $T_{x}^{-}$ be the two rays from $x$ along its tangent.
Let C be the unbounded cell {\rm (}before refinement{\rm )} that contains $x$, and 
let $C' \subset C$ be the associated artificially bounded cell 
{\rm (}after refinement{\rm )}.
%
\begin{tabbing}
\indent \= {\rm (i)} \= If $x$ is an artificial endpoint,
then $T_{x}^{+}$ is the infinite direction from $x$ if 
$T_{x}^{+}$ \\
\> \> points into $C'$.\\
\> {\rm (ii)} If $x$ lies outside $C'$, then 
$T_{x}^{+}$ is the infinite direction from $x$ if and only if
\end{tabbing}
\begin{tabbing}
\indent \= \hspace{.15in} \= {\rm (a)} $T_{x}^{+}$ does not intersect the boundary of C, or\\
\> \> {\rm (b)} Both $T_{x}^{+}$ and $T_{x}^{-}$ intersect the boundary of C, 
	but $T_{x}^{+}$ does not enter $C'$.
\end{tabbing}
\end{lemma}
\begin{proof}
Part (i) is clear, so assume that $x$ lies outside $C'$.
It is simple to show that at least one of the rays must intersect the boundary
of $C$, and we leave this as an exercise for the reader.

(a) Suppose that $T_{x}^{+}$ does not intersect the boundary of $C$ and $T_{x}^{-}$ does.
If $T_{x}^{+}$ is the finite direction,
then the convex segment must travel to a cell wall after leaving $x$ along $T_{x}^{+}$
(Fig.~\ref{fig-shouldIuse}).
But, in so doing, it will block the other half of the convex segment 
(leaving $x$ along $T_{x}^{-}$) from traveling to infinity.
Therefore, $T_{x}^{+}$ must be the infinite direction.

(b) Suppose that both $T_{x}^{+}$ and $T_{x}^{-}$ intersect the boundary of $C$.
Let $T_{x}$ be the entire tangent at $x$, which 
divides $C$ into a bounded and an unbounded half.
Let \arc{P\infty} be $x$'s infinite convex segment before refinement.
Since \arc{P\infty} must lie entirely on one side of $x$'s tangent (by convexity)
and in an unbounded region (because \arc{P\infty} is infinite), it must lie
on the unbounded half.
In particular, the unbounded half must contain the original endpoint $P$, which
lies in $C'$ on an original wall.
Thus, $T_{x}$ must enter $C'$.
If both $T_{x}^{+}$ and $T_{x}^{-}$ enter $C'$, then 
all of $T_{x}$ (and in particular $x$) must lie in $C'$, by the
convexity of $C'$.
Thus, since $x \not\in C'$, exactly one of $T_{x}^{+}$ and $T_{x}^{-}$
enters $C'$.
The unique ray that enters $C'$ must be the finite direction, otherwise 
the segment leaving $x$ in the infinite direction will be blocked, just as in (a).
\end{proof}

\figg{fig-shouldIuse}{$T_{x}^{+}$ cannot be the finite direction.}{2.5in}

\section*{Acknowledgments}
This work formed part of the Ph.D. thesis of John Johnstone, who is very grateful for
the guidance of his advisor, John Hopcroft.
The thorough and insightful reading of the main referee is also much appreciated.
%

\nocite{baj}
\nocite{law}

\bibliography{sorting}
\bibliographystyle{siam}
\end{document}

