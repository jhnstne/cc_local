\documentstyle[11pt]{bletter}
\signature{John K. Johnstone}
\newcommand{\arc}[1]{\mbox{$\stackrel{\frown}{#1}$}}
\pagestyle{empty}
\begin{document}
\begin{letter}
{Fachbereich Mathematik\\
Freie Universtitaet Berlin\\
Arnimallee 2-6\\
D-1000 Berlin 33\\
Federal Republic of Germany}

\opening{Dear Chee:}

Here is the revised version finally.
Please pardon the delay.
I have made all of the clarifications and additions that the main referee asked for.
I address these now one at a time.
(Note: due to Crystal Norris' advice at SIAM,
I have pasted figures into one of your copies and not sent
you the original figures.  See the end of this letter for clarification.)

Page numbers are from the original version.
Page numbers in brackets are from the revised version.
Obvious corrections (e.g., typos) are not mentioned.

\begin{enumerate}
\item
	p. 3: Elaboration on solutions to Restriction and Edge Intersection problems.
\item
	p. 4 comment: The method of the paper does apply to space curves, using projection.
	See comment on p. 2.
\item
	p. 9: Cell and cell partition have been defined more carefully.
\item
	p. 11 [pp. 11-12]: Explanation of singularity refinement has been 
	substantially changed to better motivate this procedure.
	But note that it {\bf is} necessary to refine endpoints for pairing, 
	as in distinguishing the following segments: \\
	\vspace{4in}

	and: 
	\vspace{2in}

	The most natural way to store this information is in refined and artificial 
	endpoints.
	Once pairing is done, refined and artificial endpoints are still necessary.
	For example, locating $x$ in the following example (refinement of singularity):\\
	\vspace{2in}
	
	recognizing when a point lies on an infinite segment so that one can use a different
	point location scheme (see note on [pp. 27-35] and point location below):\\
	\vspace{2in}

	and making the order of points around a loop clear:\\
	\vspace{2in}

	Finally, nonsingular points are often necessary, as in Lemma 5.2.
\item	
	p. 15 [p. 16]: The exact position of refined endpoints is not important.
	[p. 16] has a discussion of the amount of flexibility that is possible.
	Also refer to [p. 12].
	On \mbox{[p. 21]} a practical note has been added, based on this flexibility of refined 
	endpoints.
\item
	p. 15: An explicit algorithm for choosing the artificial walls has been added in 
	an appendix (Section 10.1, [p. 46]).
\item
	p. 15 [pp. 16-17]: Parts extending to infinity after an artificial endpoint
	are not discarded (since we might want to sort points on these segments).
	The infinite segment \arc{A\infty} is simply represented by a finite segment 
	\arc{AB}.  For example, a point on \arc{A\infty} but after the artificial
	endpoint $B$ will still be located on \arc{AB}, 
	since that is the representative of \arc{A\infty}.
	See the discussion on [pp. 16-17].
\item
	p. 16 [p. 18]: Definition of `inside' and `face' has been improved.
\item
	p. 19: 	Restriction to nonsingular crossings is optional.
		See the second note at the top of [p. 22].
\item
	p. 20 [pp. 22-23]:  Theorem 5.2 and its proof does not depend upon the exact position
	of the refined endpoint; it only depends on the fact that the refined endpoint
	lies on the appropriate convex segment incident to the singularity.
	For example, Lemma 5.2 can be applied as long as two points lie on the same convex 
	segment.
\item	
	comment on p. 21: Counting along boundary would work in this case
	but not in others.
\item
	p. 21 [p. 24]: First paragraph of proof clarified.
\item
	p. 22: Proof is invariant under restriction to nonsingular P, although
	this restriction is not necessary in theory. 
	Again see the second note on [p. 22]. 
\item
	p. 22 `Therefore, there exists $\alpha \ldots$': 
	This part of the proof is now in Lemma 10.1 \mbox{[p. 48]} and the choice of $\alpha$
	has been explained more carefully.
\item
	p. 22, `By the argument of Lemma 2': This section of the proof has been moved
	to Lemma 10.2 of the appendix [p. 49], where the argument has been made explicit.
\item
	p. 24 [p. 25]: Paragraph beginning `We now show that $W_{2} \in S(W_{1})$'
	has been rewritten to make the argument clearer (e.g., the equality
	that referee circled is better explained).
\item
	p. 24 [p. 25]: The term `refined singularity' is not used anymore
	(it was undefined as referee noted).
\item
	p. 25 [p. 25]: The argument from Lemma 2 (Lemma 5.2) is now explicitly reviewed.
\item	
	p. 26 [p. 28]: We no longer claim that nude components are always closed.
\item
	p. 26: Since the checking of condition 4 of Theorem 4 [Theorem 6.1] may
	look misleadingly expensive,
	an explanatory note has been added on [p. 19].
\item
	p. 31 [p. 39]: Corrected.  Note that we are using the RAM model, where basic
	arithmetic operations are of unit cost.
\end{enumerate}
\vspace{1in}

\noindent We now enumerate some additions that were not requested by the referees, but
which were motivated in the process of the above changes.

\begin{enumerate}
\item
	Note that the title has been changed.
\item
	{[p. 6]}: Preamble to general algorithm for convex segment method has been 
	slightly changed.
\item
	{[p. 11]}: Footnote added.
\item
	{[p. 17]}: Review of types of endpoint added.
\item
	{[p. 48ff.]} For clarity and elegance, some key facts have been factored out from 
	the proofs into lemmas, and these lemmas placed in an appendix (Section 10.2).
\item
	{[pp. 27-35]}: The section on point location has been elaborated.
	The last two examples of Example 6.1 illustrate the two additions.
	We must deal with points that do not lie on a refined segment, either
	points on an infinite segment outside the bounded cell
	or points between a singularity and one of its refinements.
	For example in Figure 24, $S(x) = \{W_{1},W_{2},W_{3},W_{4}\}$ and 
	the two that are closest to $x$'s tangent are $W_{1}$ and $W_{3}$, 
	but one cannot distinguish whether $x$ lies on $W_{1}$'s or $W_{3}$'s 
	segment without considering the type of the two infinite segments.
\item
	{[p. 26]}: A small addition has been made to first paragraph of Section 5.5
	on computation of singularities and flexes.
\item
	{[p. 37]}: Definition of equivalence relation in Lemma 6.2 has been better defined.
\item
	{[p. 38]}: Discussion of complexity analysis has been moved up before Theorem 7.1.
\end{enumerate}

\vspace{1in}

Some administrative details. 
In talking with Rao Kosaraju, I have discovered that the publishing
process is expedited if I help with the typesetting (which will be simple
since I am already using LaTeX).
I have spoken with Crystal Norris at SIAM, who is in charge of these things,
and she has agreed to let me do this.
Related to this, she has advised me to keep the originals of the figures, and
send them directly to her later.
Therefore, I have pasted copies of the figures directly into the copy that I 
have sent you, and have not sent you the originals.
So that I know when to get in touch with Crystal Norris,
would you please send me e-mail when you send the paper into SIAM?
Thanks.

\closing{All of the best,}
\end{letter}
\end{document}
