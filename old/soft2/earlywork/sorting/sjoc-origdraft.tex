\newif\ifFull
\Fulltrue
% \Fullfalse
\documentstyle[11pt]{article} 
%
\input{macros}
% \newtheorem{definition}{Definition}
%
% Page format
%
\DoubleSpace
%\SingleSpace
\setlength{\oddsidemargin}{0pt}
\setlength{\evensidemargin}{0pt}
\setlength{\headsep}{0pt}
\setlength{\topmargin}{0pt}
\setlength{\textheight}{8.75in}
\setlength{\textwidth}{6.5in}
%
%
\title{ON THE SORTING OF POINTS ALONG AN ALGEBRAIC CURVE}
\author{
JOHN K. JOHNSTONE\thanks{Department of Computer Science, The Johns Hopkins University, 
Baltimore, MD 21218.
The work of this author was supported in part by a Natural Sciences and Engineering
Research Council of Canada 1967 Graduate 
Fellowship and an Imperial Esso Graduate Fellowship while the author was a graduate
student in the Department of Computer Science, Cornell University, Ithaca, NY, 14853.}
\ and \ 
CHANDERJIT L. BAJAJ\thanks{Department of Computer Science, Purdue University, 
W. Lafayette, IN  47907.
The work of this author was supported in part by National Science Foundation grant
85-21356 and ARO contract DAAG 29-85-C0018 under Cornell MSI.} 
}
%
\begin{document}
\maketitle

{\bf Abstract.}
An operation that is frequently needed during the creation and manipulation of geometric 
models is the sorting of points along an algebraic curve.
Given a segment \arc{AB} of an algebraic curve, a set of points on the curve is sorted
from A to B along \arc{AB} by putting them into the order that they would be encountered 
in travelling continuously from A to B along \arc{AB}.
A new method for sorting points along an algebraic curve is presented.
Key steps in this method are the decomposition of a plane algebraic curve into convex 
segments and point location in this decomposition.
% The decomposition is accomplished by the tangents at the singularities and points of 
% inflection of the curve.
This new method can sort an arbitrary algebraic curve and it is particularly efficient 
because of its preprocessing, both of which make it superior to conventional methods.
The complexity of the new method is analyzed, and execution times of various sorting
methods on a number of algebraic curves are presented.
The theory developed for sorting can also be used to locate points on an arbitrary 
segment of an algebraic curve and to decide whether two points lie on the same connected 
component.

{\bf Key words.} Sorting, decomposition, point location, convexity, algebraic curves,
geometric modeling, solid modeling.

{\bf AMS(MOS) subject classifications.} 68U05, 68Q25, 68P10, 14H99.

\section{Introduction}
%
The sorting of numbers into increasing order or words 
into alphabetical order is one of the basic problems of computer science.  
The purpose of this paper is to establish that the sorting of points
along a curve is a basic problem in geometric modeling and computational geometry,
and to present a universal and efficient method for this sorting.
This method relies upon the solution of two problems that are very useful in their
own right: convex decomposition of a curve and point location on a segment.

To sort a set of points from A to B along the curve segment \arc{AB} means to
put the points into the order that they would be encountered in travelling
continuously from A to B along \arc{AB} (Figure~\ref{1.1}).
Points that do not lie on \arc{AB} are never encountered and are thus ignored.
A vector at A is provided to indicate the direction in which the sort is to
proceed from A. 
This vector is especially important when the curve is closed, since
there are then two segments between A and B to choose from.
All of the points, including A and B, are assumed to be nonsingular,
since otherwise their order might be ambiguous.
%
\figg{1.1}{The sorted order from A to B is III, II, IV}{1.5in}
% figure 1.1, p. 2

Our treatment shall be of irreducible algebraic plane curves (a curve that 
lies in a plane and is described by an irreducible polynomial\footnote{The coefficient
	domain of the polynomial can be the integers, rationals, algebraic real numbers,
	or any other set of numbers that has a finite representation.}
$f(x,y)=0$); 
in the rest of this paper, all curves are assumed to be of this type and nonlinear.
An extension of the methods to algebraic space curves is possible using 
a suitable projection of the space curve to a plane curve \cite{jj}.

The next section establishes that sorting is a fundamental operation of 
geometric modeling.
After discussing previous sorting methods in
Section~\ref{sp},
we introduce our new sorting method in
Section~\ref{co}.
Convex decomposition of a curve
and point location on a convex segment 
are discussed in Sections~\ref{s-dec} 
and \ref{s-loc}.
Complexity issues and execution times of the
various sorting methods are presented in
Sections~\ref{s-c} and \ref{data}.
The relative advantages of the sorting methods
are weighed in Section~\ref{sc} 
and Section~\ref{sco} ends with 
some conclusions.


\section{The importance of sorting}
The sorting of points along a curve has many applications 
in geometric modeling. 
The following problem is the most natural application.\\[.25in]
%
{\bf Restriction}\\
INSTANCE: A set S of points on a curve C and a segment \arc{EF} of C.\\
QUESTION: Which points of S lie on \arc{EF}? \\
SOLUTION: Sort S along \arc{EF}.\\[.25in]
%
Since an edge of a solid model is often defined by a curve and a pair of endpoints,
restriction is a very basic problem in geometric modeling.
For example, the following edge intersection and bounding box problems are two important 
problems that can be solved with restriction.\\[.25in]
%
{\bf Edge intersection}\\
INSTANCE: Edges E and F on curves C and D, respectively.\\
QUESTION: What is $E \cap F$?\\
SOLUTION: Compute $C \cap D$ by well-known methods and restrict to the edges.\\[.25in]
%
{\bf Bounding box}\\
INSTANCE: Edge E on curve C with endpoints $E_{1}$ and $E_{2}$.\\
QUESTION: Find the smallest rectangle with sides parallel to the coordinate axes that
contains E.\\
SOLUTION: Compute the local extrema of the curve and restrict to the edge, yielding S.
Find the minimum x-value ($x_{\mbox{min}}$) in $S \cup \{E_{1},E_{2}\}$, and so on.
The desired box is defined by the lines $x = x_{min}$, $x = x_{max}$, $y = y_{min}$, 
and $y = y_{max}$.\\[.25in]
%
The bounding box (see \cite[p. 372]{NS}) is useful for interference detection: 
the expensive intersection of edges can be reserved for those situations when the edges
are close enough that their bounding boxes interfere.
Bounding regions are also useful for problems such as the restriction problem, 
because they allow points that clearly do not satisfy a condition to be discarded
quickly.

Another fundamental use of sorting\footnote{In this paper, `sorting' will always
refer to 
the sorting of points along a curve, not the conventional sorting of numbers or
words.} is to introduce an even-odd parity to a
set of points, as illustrated by the following problem.\\[.25in]
%
{\bf Solid model intersection}\\
INSTANCE: Two solid models M and N.\\
QUESTION: What is the intersection of M and N?\\
SOLUTION: An important step of this computation
is to find the segments of an edge of one model that lie in the intersection.
This is done by finding and sorting the points of intersection 
of this edge with a face of the other model. 
The segments of the edge between the $i^{th}$ and $i$+1$^{st}$ intersections,
for $i$ odd, are contained in the intersection of the models.\\[.25in]
%
Another application of even-odd parity 
is to decide whether a point lies within a piecewise-algebraic plane patch
(or a piecewise-algebraic convex surface patch).
% see Mortenson, Geometric Modeling, p. 545 for a good picture
This problem, which is fundamental to the display of a geometric model, 
is fully discussed in \cite{jj}.
Having established the importance of sorting, in the next section 
we proceed to a discussion of methods for sorting.

\section{Previous work on sorting}
\label{sp}
%
There is no serious study of sorting in the literature.
This can be explained by the fact that nontrivial sorting problems
arise only with curves of degree three or more, and until recently, 
almost all of the curves in solid models were linear or quadratic.  
However, as the science of geometric modeling matures and grows more
ambitious, curves of degree three and higher are becoming common.
For example, the introduction of blending surfaces \cite{hh87}
into a model creates curves and surfaces of high degree.

The lack of a study of sorting can also be explained 
by the presence of an obvious method for sorting points, 
which tends to obviate a search for any other method.
This obvious method uses a rational parameterization of the curve
(i.e., a parameterization ($x(t)$, $y(t)$) such that both 
$x(t)$ and $y(t)$ can be expressed as the quotient of two polynomials in $t$),
sorting a set of points S along \arc{AB} as follows.

\begin{center}{\bf The parameterization method of sorting}\end{center}
\begin{description}
\item[{[Preprocessing]}] \ \ \ 
\begin{description}
\item[1.] Parameterize the curve.
\end{description}
\clearpage
%
\item[{[Solve]}] \ \ \ 
\begin{description}
\item[2.]
	Find the parameter values of A and B, say $t_{1}$ and $t_{2}$.
\item[3.]
	Find the parameter value of each point in S.
\end{description}
%
\item[{[Sort numbers]}] \ \ \ 
\begin{description}
\item[4.]
	Sort the parameter values of S from $t_{1}$ to $t_{2}$, discarding
	those outside this interval.
\end{description}
\end{description}
%
We insist upon a rational parameterization because a nonrational 
parameterization is difficult to represent and difficult to solve. 
With a nonrational parameterization (such as x(t) = $\sqrt{t}$ or x(t) = sin(t)),
two different points may have the same parameter value, which complicates sorting.
Finally, there is no algorithm for the automatic parameterization of a curve 
that does not have a rational parameterization, whereas there is such an algorithm
for rational curves \cite{abba3}.

There are many reasons to be dissatisfied with the parameterization method.
It is not a universal method, since not all algebraic curves have a 
rational parameterization.
Indeed, a plane algebraic curve has a rational parameterization if and only if its
genus is zero, if and only if it has the maximum number of singularities 
allowable for a curve of its degree \cite{walker}.  
Secondly, even for those curves that do have rational parameterizations, 
the parameterization method will be slow if the degree of the parameterization 
is high, since the computation of the parameter values of the points will 
be expensive.
Other weaknesses of the parameterization method will become clear as we compare
it with the new method.
% namely, not clear that it can distinguish connected components 

There is also a brute-force sorting method, which uses techniques for
tracing along a curve \cite{bhhl}.
The order of the points is the order in which they are encountered during
a trace of the segment.
This method is not satisfactory, because its implementation, although robust, is 
inherently very slow.
Moreover, its complexity depends upon the length of the segment that is being
sorted rather than upon the number of points in the sort, which is 
undesirable.

The weaknesses of the parameterization and tracing methods of sorting 
suggest that another method is necessary: one that will perform more
efficiently on a wider selection of algebraic curves.
The next section presents such a method.
This method works with the implicit representation $f(x,y)=0$ of a curve
(as opposed to the parametric representation),
thus allowing the use of tools from algebraic geometry.

\section{The convex segment method of sorting}
\label{co}

The observation that motivates the new method is that 
a convex segment can be sorted easily.
Since every curve is a collection of convex segments,
this suggests a divide and conquer strategy.
A segment of a plane algebraic curve is {\em convex} if no line has more than 
two distinct intersections with it.
(Alternatively, a planar segment is convex if it lies entirely on one side of
the closed halfplane determined by the tangent line at any point of
the segment \cite{Do}.)
The following theorem shows that sorting a convex segment is simple.

\begin{theorem}
\label{T-s}
Let $p_{1},\ldots,p_{n}$ be points on a convex segment \arc{AB}, 
and let H be the convex hull of A, B, $p_{1},\ldots,p_{n}$ (Figure~2).
The order (from A to B) of $p_{1},\ldots,p_{n}$ is simply 
the order (from A to B) of the vertices on the boundary of H.
\end{theorem}
\Heading{Proof:}
\cite[p. 20]{jj}.
\QED

\begin{figure}[htbp]\vspace{2.25in}\label{2.3}\caption{The sorting of a convex 
segment}\end{figure}  % Figure 2.3, p. 20

Suppose that a curve can be decomposed into convex segments.
Also suppose that we can identify the convex segment in this decomposition
that contains a query point (point location in convex decomposition).
These key problems will be discussed in 
Sections~\ref{s-dec} and \ref{s-loc}.
The following algorithm shows how to sort a set of points S 
along the segment \arc{AB}.

\begin{center}{\bf The convex segment method of sorting}\end{center}
\begin{description}
\item[{[Preprocessing]}] \ \ \ 
\begin{description}
\item[1.] Decompose the curve into convex segments
	(say \arc{W_{1}W_{2}}, \arc{W_{2}W_{3}}, $\ldots$, 
	\arc{W_{\beta - 1}W_{\beta}}).
\end{description}
\clearpage
%
\item[{[Locate first convex segment]}] \ \ \ 
\begin{description}
\item[2.]
	Find the convex segment that contains A (say \arc{W_{i-1}W_{i}}).
\item[3.]
	Decide whether \arc{AB} leaves A along \arc{AW_{i-1}} or \arc{AW_{i}}
	(say \arc{AW_{i}}).\footnote{If V is the vector at A that is given as 
		part of the input, then \arc{AB} leaves A along \arc{AW_{i}} if 
\label{alg-page}
		and only if V points to the halfplane defined by \lyne{AW_{i}} 
		that contains \arc{AW_{i}}.}
\item[4.]
	PresentConvexSegment := \arc{AW_{i}} ;
	j := i ;
	SortedSet := $\emptyset$ ;
	FoundB := false
\end{description}
%
\item[{[Sort one convex segment at a time]}] \ \ \ 
\begin{description}
\item[5.]
	Repeat until FoundB
\begin{description}
\item[(a)]
	Find the points of S that lie on PresentConvexSegment.\\
	If B is one of these points, then FoundB := true.
\item[(b)]
	Sort these points along PresentConvexSegment, 
	using Theorem~\ref{T-s}.
\item[(c)]
	If not FoundB,\\
	then SortedSet := Append(SortedSet,\{sorted points on PresentConvexSegment\})\\
	else \mbox{SortedSet := Append(SortedSet,\{sorted points on PresentConvexSegment before B\})}
\item[(d)]
	PresentConvexSegment := \arc{W_{j}W_{j+1}} ; j := $j+1$
\end{description}
\end{description}
%
\item[{[Output]}] \ \ \ 
\begin{description}
\item[6.]
	Return SortedSet.
\end{description}
\end{description}

The expense of this method is concentrated in the preprocessing phase, which
is done once off-line. 
The run-time operations (convex-segment sorting and
locating a point on a convex segment) are usually very simple.
Therefore, the efficiency of this method is very competitive.
The coverage of the convex segment method is the entire set of algebraic curves,
since it works directly from the implicit representation of the curve.

\begin{example}
Consider the sorting of points $P_{1}, \ldots, P_{6}$ 
along the segment \arc{AB} of Figure~\ref{2.12a}.
The curve is decomposed into convex segments by the dotted lines 
(Section~\ref{s-dec}).
A lies on \arc{W_{1}W_{8}} and
the vector at A identifies that \arc{AW_{1}} is the first convex segment.
There are no points on \arc{AW_{1}}, so we move on.
The next convex segment is \arc{\wo\wt}.
Only $P_{1}$ lies on \arc{\wo\wt} and it becomes the first element of the 
sorted list.
We jump to the next convex segment \arc{W_{2}W_{3}} and
sort the two points $P_{2}$ and $P_{3}$ 
by creating the convex hull of $W_{2}$, $W_{3}$, $P_{2}$, and $P_{3}$.
$P_{2}$ and $P_{3}$ are added to the global sort.
We move on to the next convex segment \arc{W_{3}W_{4}}, and then \arc{W_{4}W_{5}}.
The presence of B indicates that this is the last convex segment.
Upon sorting B and $P_{4}$, $P_{4}$ is discarded because it comes after B.
The final sorted list is $P_{1},P_{2},P_{3}$.
\end{example}

\begin{figure}[htbp]\vspace{2.75in}\caption{Sorting a curve by convex segments}\label{2.12a}\end{figure}
% picture revised from Figure 2.10 of thesis in the way outlined in my thesis copy

It remains to discuss how a curve can be decomposed into convex segments and
how a point can be located in this convex decomposition.
These two problems, which are at the heart of the convex segment method 
of sorting, are solved in the following two sections.

\section{Convex decomposition of a curve}
\label{s-dec}
%
The decomposition of an object into simple objects is an important theme
in computational geometry.
Decomposition proves to be particularly useful in divide-and-conquer algorithms, 
since simple objects are easily conquered.
There has been a good deal of work on the decomposition of 
(simple, multiply connected, or rectilinear) polygons into simple components
(e.g., triangles \cite{CI,G,H,T}, quadrilaterals \cite{S}, 
trapezoids \cite{As}, convex polygons  \cite{cd,tm}, and star-shaped 
polygons \cite{Av}), sometimes with added criteria (e.g., minimum decomposition 
\cite{cd,keil}, minimum covering \cite{O}, no Steiner points \cite{keil}).
However, all of this work has been in the polygonal (or at best polyhedral) 
domain.
The decomposition of a plane algebraic curve of arbitrary degree into convex 
segments is an extension of decomposition to the curved world.

A version of Bezout's Theorem states that two irreducible plane algebraic curves 
of degree $m$ and $n$ have exactly $mn$ intersections (properly counted),
unless the curves are identical \cite{walker}.
Therefore, all plane algebraic curves of degree one (lines) and two (conics) 
are already convex.
For the convex decomposition of curves of degree three and higher, the 
singularities and points of inflection are instrumental.
A {\em singularity} of the curve $f(x,y)=0$ is a point P of the curve
such that $f_{x}(P) = f_{y}(P) = 0$ (where $f_{x}$ is the derivative of $f$ with
respect to $x$).
It is a point where the curve crosses itself or changes direction sharply.
A nonsingular point is also called a {\em simple} point.
A {\em point of inflection} is a simple point P of the curve
whose tangent has three or more intersections with the curve at P.  
(It is also a point of zero curvature.)
We restrict our attention to points of inflection P such that P's tangent 
has an odd number of intersections with the curve at P, which we call
\label{restriction}
{\em flexes} for short.
Fundamental in algebraic and differential geometry, singularities and flexes
form a skeleton of the curve and can be used in many useful ways.
(For example, singularities can be used to parameterize a plane algebraic curve 
\cite{abba3}.)
Their use in convex decomposition underlines their importance to
computational geometry of higher degrees.

The tangents at the singularities and flexes of a curve form an arrangement 
of lines that subdivide the plane of the curve into several cells,
called a {\em cell partition} (Figures~\ref{2.12a}-\ref{2.8}).
The tangents also split the curve into several segments.
The following theorem establishes that each of these segments is convex.

\figg{2.8}{Convex segmentation of limacon of Pascal}{1.75in}
% Figure 2.8(a), p. 27

\begin{theorem}
\label{deke}
The tangents of the singularities and flexes of a plane algebraic curve slice
the curve into convex segments.  That is, if \arc{PQ} is a nonconvex 
segment, then some tangent of a singularity or flex will intersect 
\arc{PQ}.\footnote{The simple points at which a singularity/flex 
	tangent touches, but does not cross, the curve are redundant and should 
	not be treated as convex segment endpoints in the decomposition.}
\end{theorem}
\ifFull
\Heading{Proof:}

Let \arc{PQ} be a nonconvex segment of an algebraic curve.
Assume without loss of generality that \pq\ does not contain a singularity or 
a flex.
It can be shown that there exists a line L that crosses \pq\ at three 
(or more) distinct points \cite[p. 117]{jj}.\footnote{Already, by the definition
	of convexity, there must exist a line that intersects \arc{PQ} three
	(or more) times.}
% by Lemma~\ref{threecross} 
Let \xo, \xt, and \xth\ be three of these points, such that $\xt \in \xotha$ 
and \mbox{$\xotha \cap L = \{ \xo,\xt,\xth\}$}.
\xotha\ does not change its direction of curvature, since there is no 
singularity or flex on \pq.
\xotha\ is not a line segment, otherwise Bezout's Theorem would imply that
the algebraic curve that contains \xotha\ is a line, which it cannot be since
it contains a nonconvex segment.
Therefore, it can be assumed without loss of generality that \xotha\ looks like 
Figure~\ref{2.7}(a).
Let R be the closed region bounded by \xotha\ and \seg{\xo\xth}.
We will show that R contains a singularity or a flex.
This will complete the proof, since the tangent
of a point inside R must intersect $\xotha \subset \pq$ at least once.
(The tangent must cross the boundary of R twice, and at most one of these
intersections can be with 
\seg{\xo\xth}.)
The curve lies inside of R as it leaves \xotha\ from \xo\ and outside 
of R as it leaves \xotha\ from \xth.
Therefore, the curve must cross the boundary of R after it leaves \xotha\
from \xo, either because it must join with \xth\ (if the curve is closed)
or because an infinite segment of an algebraic curve
cannot remain within a closed region (if the curve is open) \cite{jj}.
The curve cannot intersect the \xotha\ boundary of R, since
\mbox{$\xotha \subset \pq$} is nonsingular by assumption.
Therefore, the curve must cross \seg{\xo\xth} after it leaves
\xotha\ from \xo.

As the curve leaves \xotha\ from \xo,
it lies on the opposite side of \xo's tangent from \seg{\xo\xth}.
Therefore, after the curve leaves \xotha\ from \xo\ and before it leaves
R, the curve must cross \xo's tangent inside of R, in order to reach 
\seg{\xo\xth}.
In order to cross over \xo's tangent, the curve
must cross itself or change its curvature inside of R (Figure~\ref{2.7}(b)),
otherwise it will spiral around inside R forever.
Therefore, R contains a singularity or a flex.
\QED
%
% \figg{2.9}{\wt\ should be ignored and this should be treated as a 
% single convex segment \arc{\wo W_{3}}}{.1in}  % thesis, 2.9
%
\figg{2.7}{(a) \xotha\ and R (b) travelling from \xo\ to \seg{\xo\xth}}{2.25in}
% Figure 2.6(b) and Figure 2.7, pp. 25-6
% only need 1.75 inches for (c) if we have to split it off
%
\else
\Heading{Sketch of Proof:}

Assume w.l.o.g. that \arc{PQ}\ does
\marginpar{(*)}
not contain a singularity or flex.
Since \arc{PQ}\ is not convex, we can find a line that crosses it at
three distinct points: $x_{1}, x_{2}, x_{3}$.
By (*), \arc{x_{1}x_{3}}\ must be a spiral.
Consider the region R bounded by \arc{x_{1}x_{3}}\ and \seg{x_{1}x_{3}}.
It is sufficient to show that R contains a flex or singularity S,
since then S's tangent must cross the \arc{x_{1}x_{3}}\ boundary of R.
It can be shown that the curve must enter R as it leaves $x_{1}$ and
must eventually leave R via \seg{x_{1}x_{3}}.
However, in order to do this, the curve must cross itself or change its
curvature inside of R.
\QED
\fi

We include here a word about robustness.
Consider the accuracy required in the computation of the singularities, flexes, 
and their tangents in order to guarantee a true division into convex segments.
Suppose that, in the proof of Theorem~\ref{deke}, the tangent of a 
singularity/flex inside the region R is used to split a nonconvex segment.
Any line through a point in the interior of R would work equally well in splitting
the nonconvex segment.
Thus, in this case the method is robust under slight errors in tangents, singularities,
and flexes.
The other case is if a nonconvex segment S is split into convex segments by a 
singularity or flex lying on S. 
The computed convex segment will differ from the actual convex segment by the same
amount as the computed flex (say) differs from the actual flex.
The only points that might be treated improperly are those that lie on the segment
between the computed and actual flex.
In other words, points that are within (some function of) machine precision 
of each other cannot be distinguished by the method and must be considered 
equivalent.
This equivalence of points within machine precision is inherent to any sorting 
algorithm.

Theorem~\ref{deke} does not solve the convex decomposition problem,
because it yields a confused collection of endpoints of convex segments, 
not a collection of convex segments.
The more challenging step of pairing up the endpoints remains, where
two endpoints are {\em partners} if they define a convex segment of the 
decomposition.
This pairing problem will be attacked in Sections~\ref{ssp} 
and \ref{sspII}, but first the collection of convex segments must be 
refined.

\subsection{Refinement of convex segments I: Singularities}
\label{sec-refine1}

Many of the endpoints of the convex segments created by Theorem~\ref{deke} 
are singularities.  However, singular endpoints cause problems in pairing.
Consider a convex segment whose two endpoints are the same point,
which might occur around a singularity (Figure~\ref{2.8}).
This situation is to be avoided, since pairing will turn out to be easier if 
the two endpoints of a convex segment are different.
It is also possible for a singularity to have more than two
partners and, in particular, two partners in the same cell.
This situation is also to be avoided, since it is easier to find the partner 
of an endpoint in a cell if this partner is unique.

Another problem with singular endpoints is that the ordering of points about
a singularity can be ambiguous.
Does $P_{2}$ or $P_{3}$ follow A in Figure~\ref{2.11}(a)?
What is the order of the points in Figure~\ref{2.11}(b): 
$S, P_{1}, P_{2}, P_{3}, S$ or $S, P_{3}, P_{2}, P_{1}, S$?
As a result of these problems, all convex segments with singular endpoints 
will be replaced by convex segments with nonsingular endpoints.

\figg{2.11}{Ambiguity about a singularity}{2.5in}
% Fig 2.11 of thesis, with added S

A pair of points will be found on each branch of the curve
that passes through a singularity, one on either side of (and very close to) 
the singularity.
The added points will receive the convex segments that enter the singularity.
After each singularity of the curve has been decomposed in this manner, every 
convex segment of the curve will be bounded by simple points, as desired.

\begin{example}
\label{eg-pseudo}
Four points are associated with the singularity A of Figure~\ref{2.12}:
$V_{1}$ and $V_{2}$ from one branch, $W_{1}$ and $W_{2}$ from the other.
The convex segments of the two cells are now \arc{PV_{1}}, \arc{V_{1}V_{2}}, 
\arc{V_{2}Q}, \arc{RW_{1}}, \arc{W_{1}W_{2}}, and \arc{W_{2}S}.
Notice that this refinement makes it clear that Q (not S) must follow P.
\end{example}

\figg{2.12}{The refinement of a singularity}{3in}
% Figure 2.12(b) of thesis,p. 35

Consider the problem of finding two points on each branch, one on either 
side of the singularity.
We would like to do this by tracing a small distance along the branch
in both directions from the singularity.
However, there is no reliable way of tracing along a branch as it passes
through a singularity, because the other branches create too much confusion.
Therefore, each branch of the singularity must be isolated so that it can 
be traced robustly.
This isolation is accomplished by blowing up the curve at the singularity by 
a series of quadratic transformations \cite{bhhl,walker}, as follows.

The first step in blowing up a singularity is to translate it to the 
origin.\footnote{Since the quadratic transformation does not map the line $x=0$ 
	properly, the curve should also be rotated (if necessary) so that it is 
	not tangent to $x=0$ at the origin (see \cite{jj}).}
%	This is ensured by applying a nonsingular linear transformation 
%	$x = \alpha\hat{x}+\beta\hat{y}$ and $y = \delta\hat{x} + \gamma\hat{y}$, 
%	such that neither $\alpha\hat{x}+\beta\hat{y}$ nor 
%	$\delta\hat{x} + \gamma\hat{y}$ are tangents to the curve at the origin.
Let the new equation of the curve be $f(x,y)=0$.
A quadratic transformation is applied to the curve.
The {\em affine quadratic transformation} $x = x_{1},\ y = x_{1}y_{1}$
\cite{walker} has three important properties:
\begin{itemize}
\item
It maps the origin to the entire $y_{1}$-axis and the rest of 
the y-axis to infinity: $y_{1} = \frac{y}{x}$ so $(0,b)$ maps to 
$(0,\frac{b}{0})$, which is a point at infinity unless $b=0$.
\item
It is one-to-one for all points $(x,y)$ with $x \neq 0$.
\item
$y = mx$, a line  through the origin, is mapped to the horizontal line $y_{1}=m$:
$y=mx \rightarrow  x_{1}y_{1} = mx_{1}  \rightarrow  y_{1}=m$.
\end{itemize}
Thus, a quadratic transformation maps distinct tangent directions of the
various branches of $f$ at the origin to different points on the 
{\em exceptional line} $x_{1} =0$.
The intersections of the transformed branches with the exceptional line
correspond to the transformed points of the origin (Figure~\ref{2.17}).
If a point of $f(x_{1},x_{1}y_{1})$ on the exceptional line is singular, 
then the procedure is applied recursively (Figure~\ref{18}).
The following lemma establishes that the various branches of the curve 
in the neighbourhood of the singularity eventually get transformed to separate 
branches.

\figg{2.17}{(a) node and (b) its quadratic transformation}{1.75in}
% Figure 2.13, p. 38

\figg{18}{(a) the original singularity (b) after one quadratic transformation
(c) after a second transformation: the original singularity successfully 
transformed into two simple points}{5.25in}
% Figure 2.14, p. 39

\begin{lemma}[{\cite{abba3,walker}}]
A singularity can be reduced to a number of simple points by 
a finite number of applications of the quadratic transformation.
An ordinary singularity can be reduced to simple points by a 
single quadratic transformation, where a singularity of multiplicity 
$r$ is ordinary if its $r$ tangents are all distinct.
\end{lemma}

To summarize, each singularity is translated to the origin and transformed 
into a set of nonsingular points through the application of a series of quadratic 
transformations.  Each branch of the transformed curve intersects the 
exceptional line in a simple point, so this image branch can be traced
from the image singularity without confusion.
Therefore, upon each image branch, two points are found by tracing a very 
short distance in either direction from the image singularity.
Finally, these points are mapped back to the original curve
to become new endpoints, replacing the singularity.
These new endpoints clarify the branch connectivity at the singularity
and simplify the job of pairing.

Care must be taken with the short segment that is essentially sliced
out of the curve during the refinement of the singularity, such as 
\arc{V_{1}V_{2}} in Figure~\ref{2.12}.
It is a special convex segment and points that lie on it are sorted in a special
way, by mapping them to the blown-up, desingularized, image curve and using 
the tracing method.
This is not expensive because the sliced-out segment is very short and
very few steps are needed to trace over it.

\subsection{Refinement of convex segments II: Infinite segments}
\label{sec-refine2}

Convex segments with singular endpoints are not the only ones that must
be refined: infinite convex segments are also problematic.
The pairing process is simplified if each convex segment has two endpoints,
but an infinite convex segment has only one endpoint.
Therefore, an artificial endpoint is added to each infinite segment, as follows.

Every open cell is artificially closed by a collection of line segments 
(Figure~\ref{3.9}).
These line segments are chosen carefully so that they only intersect 
infinite convex segments (if any) in the cell, and each of these exactly once 
(unless the infinite segment is entirely contained in the cell and thus proceeds
to infinity at both ends, in which case two intersections are allowed).
The resulting artificially-closed cell should also be a convex polygon.
A point of intersection of an infinite convex segment with the new boundary
of its cell becomes an {\em (artificial) endpoint} (Figure~\ref{3.egg}).
Thus, infinite convex segments are transformed into finite convex segments
with two endpoints.
After every endpoint has been assigned a partner, pairs that contain an
artificial endpoint are recognized as infinite convex segments.
A pair of artificial endpoints represents an entire connected component that does not 
cross any of the singularity/flex tangents.

After the above two refinements, the set of endpoints of convex segments 
assumes the following normal form:
\clearpage
\begin{itemize}
\item
	 every endpoint has exactly two partners
\nopagebreak
\item
	 every cell is a closed polygon
\end{itemize}
The normalization stage not only makes pairing easier: it also creates 
a cleaner set of convex segments that better reflects the curve.
For example, due to the first normal condition, pairing will create 
a collection of convex segments with an implicit order.

\figg{3.9}{The artificial closure of an open cell}{2in}
% Figure 3.9, p. 64 of thesis

\subsection{Pairing of endpoints I: Properties of the partner}
\label{ssp}

We are now ready to show how to pair the endpoints of convex segments.
Consider a convex segment in cell C and an endpoint E of this segment.
E's partner in C must obviously be another endpoint in C.
Therefore, the determination of partners in all single-segment cells is trivial.
Corollary~\ref{Cp} will present other conditions that E's partner must satisfy
and Theorem~\ref{Tpner} will show how to isolate the partner if several endpoints
satisfy all of these conditions.
In preparation, some terminology must be introduced and a crucial lemma proved.
%
\begin{definition}

If P is a singularity or flex, then P's tangent is a cell wall and the 
{\em inside of P's tangent} w.r.t.\ (with respect to) a cell C is the 
halfplane that contains C.  Otherwise, the inside is the halfplane that 
contains all of the curve in the neighbourhood of P (Figure~\ref{3.2}).
The inside includes the tangent, while the strict inside does not.

Let P be a flex that lies on the wall W of cell C, and 
let $P_{\epsilon}$ be a point of the curve inside cell C at distance
$\epsilon > 0$ from P.
($P_{\epsilon}$ may be found by tracing the curve into C from P.)
The {\em outside wallpoint} of W w.r.t.\ C is the endpoint of W that 
lies outside of $P_{\epsilon}$'s tangent, for $\epsilon$ small (E in Figure~\ref{3.2.5}).

If P is not a flex, then P {\em faces} Q if Q lies on the inside of 
P's tangent (Figure~\ref{3.2}(a)).
Otherwise, P faces Q w.r.t.\ cell C if (1) Q lies strictly inside P's tangent 
w.r.t.\ C or (2) Q lies on P's tangent and on the opposite side of P from 
the outside wallpoint of P's wall w.r.t.\ C (Figure~\ref{3.2.5}).
\end{definition}

\begin{notation}
\#\{S\}\ is the number of elements in the set S and
\seg{xy} is the line segment between x and y.
\seg{xy} does not include its \mbox{endpoints x and y}.
\end{notation}

\figg{3.2}{The inside of P's tangent}{2.25in}
	% Figure 3.2, p. 56
% \figg{3.3}{E is the outside wallpoint of P's wall w.r.t.\ C}{1.75in}
\figg{3.2.5}{P faces both $Q_{1}$ and $Q_{2}$ with respect to C}{2in}
	% Figure 3.4, p. 58

\begin{lemma}
\label{Ls}
Consider the cell partition of a curve F.
Let \x\ and \y\ be two nonsingular points of a convex segment in the cell C.
Then
\begin{enumerate}
\item The curve crosses\footnote{If P is a point of intersection of the curve 
	with \seg{\x\y}, then the curve {\em crosses} \seg{\x\y} at P if it lies 
	on both sides of \seg{\x\y} in any neighbourhood of P; otherwise it 
	only touches \seg{\x\y} at P.}
\seg{\x\y} at an even number of points, ignoring singularities.
\item $\#\{P \in \seg{\x\y} \cap F: P \mbox{ faces \x\ w.r.t.\ C}\} = 
\#\{P \in \seg{\x\y} \cap F: P \mbox{ faces \y\ w.r.t.\ C}\}$
\item $\forall \alpha \in \seg{\x\y}$,\ \ \nopagebreak
$\#\{P \in\seg{\x\alpha}\cap F: P \mbox{ faces \x\ w.r.t.\ C}\} \leq 
\#\{P \in \seg{\x\alpha}\cap F: P \mbox{ faces \y\ w.r.t.\ C}\} $
\end{enumerate}
\end{lemma}
%
\begin{example}
Figure~\ref{4} is a hypothetical example for Lemma~\ref{Ls}.
The curve F crosses \seg{\x\y} an even number of times.
$\{ P \in \seg{\x\y} \cap F:\mbox{ P faces \x\ \} = \{}P_{2},P_{5},P_{6}\}$
is of the same size as \\
$\{ P \in \seg{\x\y} \cap F:\mbox{ P faces \y } \} = \{P_{1},P_{3},P_{4}\}$.
Moreover,
$\{ P \in \seg{\x\alpha} \cap F: \mbox{ P faces \x } \} = \{P_{2}\}$
is smaller than
$\{ P \in \seg{\x\alpha} \cap F: \mbox{ P faces \y } \} = \{P_{1},P_{3},P_{4}\}$.
\end{example}
%
\begin{figure}[htbp]\vspace{2.25in}\caption{}\label{4}\end{figure}
% Figure 3.5, p. 59
%
\Heading{Proof of Lemma~\ref{Ls}:}

Consider the closed region $R_{\x\y}$ bounded by \seg{\x\y}\ and \arc{\x\y}.
% (Figure~\ref{3.3.5}).
Since \arc{\x\y} lies in the cell C and C is a convex polygon, \seg{\x\y} 
must also lie in C.
Therefore, again by convexity, $R_{\x\y}$ must lie in C.
% \figg{3.3.5}{The region $R_{\x\y}$}{1.5in}
%
Since \x\ and \y\ are nonsingular and the rest of \arc{\x\y} lies in the
interior of the cell, \arc{\x\y} does not contain a singularity.
Therefore, the curve can only cross into $R_{\x\y}$ through \seg{\x\y}.
If the curve enters $R_{\x\y}$, then it must also leave, since an infinite 
segment cannot remain within a closed region and an algebraic curve of finite
length is closed (viz., the curve cannot stop short in the middle of $R_{\x\y}$).
We claim that the point of departure $D$ must be distinct from the point of 
entry $E$, unless all of the tangents at $D=E$ are \lyne{XY}, as in 
Figure~\ref{3.ex}.
Otherwise, if $D=E$, then at least one of the tangents of the singularity $D$ 
will cross into $R_{\x\y}$ and form a wall of the cell partition which will 
split $R_{\x\y}$ in two, contradicting the fact that all of $R_{\x\y}$ 
lies in the same cell.
Therefore, with the exception of the special singularities of 
Figure~\ref{3.ex}, the crossings of \seg{\x\y} by the curve occur in pairs,
called {\em couples}.
This establishes condition (1) of the lemma.

\figg{3.ex}{The only type of singularity that can lie on \seg{\x\y} }{1.5in}
% Figure 3.7, p. 61

Consider condition (2).
The special singularities of Figure~\ref{3.ex} (as well as the points where
the curve only touches \seg{XY}) can be ignored during 
the consideration of conditions (2) and (3), since they face both \x\ and 
\y\ and contribute the same amount to the left-hand side and right-hand side 
of the expressions of conditions (2) and (3).
Therefore, we can concentrate on the remaining crossings of \seg{\x\y}:
the distinct couples.
Let $A,B\in \seg{\x\y}$ be a couple and assume, without loss of generality, 
that A lies closer to \x\ than B does.
\arc{AB} is a convex segment since it lies within a cell of the cell
partition.
Therefore, A and B face each other (w.r.t.\ cell C).
Since A faces B, A faces \y.
Similarly, since B faces A, B faces \x.
Therefore, one member of each couple faces \x\ and the other faces \y,
yielding condition (2).
Moreover, the point of a couple that faces \y\ (A) is closer to \x\ than the point 
that faces \x\ (B), yielding condition (3).
\QED
%
\begin{corollary}
\label{Cp}
Let $W_{1}$ be an endpoint in the cell C.
$W_{1}$'s partner $W_{2}$ in C must satisfy the following properties:
\begin{enumerate}
\item 
	$W_{1}$ and $W_{2}$ must face each other (w.r.t.\ C)
\item
	the curve must cross \seg{W_{1}W_{2}} at an even number of points,
	ignoring singularities
\item 	
	the number of these crossings that face $W_{1}$ (w.r.t.\ C)
	is equal to the number that face $W_{2}$ (w.r.t.\ C)
\item
	for any $\alpha \in \seg{W_{1}W_{2}}$, the number of crossings in the
	interval $\seg{W_{1}\alpha}$ that face $W_{1}$ is bounded by
	the number of crossings in this interval that face $W_{2}$
\end{enumerate}
\end{corollary}

These conditions, which capture the fact that the intersections of the curve 
with \seg{W_{1}W_{2}} pair up into couples that face each other, will often isolate 
the partner.

\begin{example}
Consider the cell partition of Figure~\ref{3.12} and the cell containing
the convex segments \arc{\wo\wt} and \arc{W_{3}W_{4}}.
Suppose that we wish to find the partner of \wo.
$W_{3}$ violates condition (1) and $W_{4}$ violates condition (2), 
so \wt\ must be \wo's partner.
\end{example}

\begin{figure}[htbp]\vspace{2.25in}\caption{}\label{3.12}\end{figure}
%Figure 3.12, p. 69

\noindent The following technical lemma is necessary for later proofs.
%
\begin{lemma}[{\cite{jj}}]   % Lemma B.8, p. 119
\label{Ll}
Let \wo\ and \wt\ be partners.
If \wt\ lies on \wo's tangent, then \wo\ must be a flex.
\end{lemma}

\subsection{Pairing of endpoints II: Distinguishing between candidates}
\label{sspII}

The remaining question in endpoint-pairing is how to find the partner 
of an endpoint $\wo$ in C if several endpoints in C satisfy all of the 
conditions of Corollary~\ref{Cp}.  
This will be done by sorting the candidates about the cell boundary
(Theorem~\ref{Tpner}).
Unfortunately, the refinement of singularities moved some of the endpoints 
of convex segments into the interior of cells.  
Therefore, in order to allow sorting about the boundary, we must associate 
a point $W'$ on the cell boundary with each endpoint W that was created 
in the singularity refinement stage, as follows.
If $W \neq \wo$,  then $W'$ is the intersection of the ray \ray{\wo W} 
with the cell boundary (Figure~\ref{3.J}(a)).
If $W = \wo$, then $W'$ is one of the (two) intersections of \wo's tangent with
the cell boundary: the one that lies on a tangent of the singularity
from which \wo\ was derived (Figure~\ref{3.J}(b)).
% we can assume that W1's tangent does hit T: simply ensure that singularity 
% refinement does not move W1 too far from V
For notational consistency, $W' = W$ if $W$ is an endpoint that already 
lies on the cell boundary.

\begin{figure}[htbp]\vspace{3.5in}\caption{The boundary points $W_{i}'$}\label{3.J}\end{figure}
% Figure 3.10, p. 66 of dissertation

\begin{theorem}
\label{Tpner}
Let $\wo$ be an endpoint in cell C of the cell partition of a curve F,
R(\wo) the set of endpoints in C that satisfy the conditions of 
Corollary~\ref{Cp} (w.r.t. \wo),
%That is, $R(\wo)$ := \{endpoints \mbox{W $\neq \wo$} in cell C $\mid$\nopagebreak
%\begin{enumerate}
%\item W and \wo\ face each other (w.r.t.\ C) 
%\item $\#\{P \in\seg{\wo W}\cap F: P \mbox{ faces \wo\ (w.r.t.\ C)}\}=$\\
%$\#\{P \in \seg{\wo W}\cap F: P \mbox{ faces W (w.r.t.\ C)\} }$\nopagebreak
%\item for all $\alpha \in \seg{\wo W}$, \\\nopagebreak
%$ \#\{P \in\seg{\wo\alpha}\cap F: P \mbox{ faces \wo\ (w.r.t.\ C)}\} \leq$\\
%$ \#\{P \in\seg{\wo\alpha}\cap F: P \mbox{ faces W (w.r.t.\ C)\} }  $
%\}
%\end{enumerate}
%
and S(\wo) the set of endpoints in R(\wo) that lie {\em strictly} inside of \wo's 
tangent (w.r.t.\ C).

If $S(\wo) \neq \emptyset$, let \mbox{$S'(\wo) := \{\ W':W \in S(\wo)\ \}$}.
If \wo\ is not a flex, let $X \neq \wo '$ be the other intersection of \wo's 
tangent with the cell boundary, otherwise let X be the outside wallpoint of \wo's 
wall w.r.t.\ C (Figure~\ref{3.8A}).
$\wo '$ and X split the cell boundary into two halves.
Since every endpoint in $S'(\wo)$ will lie on the same half,
a sort of $S'(\wo)$ from $\wo '$ to X is well-defined.
Let $S_{1}',S_{2}',\ldots,S_{p}'$ be the result of this sort
(i.e., $S_{i}'$ is encountered before $S_{i+1}'$ in a traversal of the cell boundary 
from $\wo '$ to X).
The partner of \wo\ in C is $S_{p}$ (the endpoint associated with $S_{p}'$).

If $S(\wo) = \emptyset$, let T(\wo) be the set of endpoints in R(\wo) that lie on 
the same wall as \wo.
The partner of \wo\ in C is the element of T(\wo) that is closest to \wo.
\end{theorem}

\begin{figure}[htb]\vspace{4.5in}\caption{Partitioning the boundary of a cell}\label{3.8A}\end{figure}
% Figure 3.11, page 68: switch the order to b,c,a

\begin{example}
Consider the computation of \wo's partner in Figure~\ref{3.egg}, 
where \wo\ is the endpoint of an infinite convex segment.
\mbox{$R(\wo) = S(\wo) = \{\wt,W_{3},W_{4}\}$} and
\mbox{$S'(\wo) = \{\wt,W_{3},W_{4}'\}$}.
The sorted order of $S'(\wo)$ along the boundary from $\wo ' = \wo$ to X
is $W_{3},W_{4}',\wt$, so \wt\ is the partner of \wo.  
Since \wt\ is an artificial endpoint, \wo\ must be the endpoint of an 
infinite convex segment.

\figg{3.egg}{Computing the partner of the endpoint of an open convex segment}{2.5in}
% Figure 3.13, page 70

Consider the computation of the partner of \wo\ in Figure~\ref{3.8},
where $S(\wo) = \emptyset$.
$V_{1}$, $V_{2}$ and $V_{4}$ are ruled out by condition (1) of R(\wo),
while $V_{3}$ and $V_{6}$ are ruled out by condition (2).
Therefore, $T(\wo) = \{V_{5},W_{2}\}$.
\wt\ is the closest element of T(\wo) to \wo, so it is \wo's partner.
\end{example}

\figg{3.8}{Partner computation when $S(\wo) = \emptyset$}{2.125in}
% Figure 3.14, p. 71

\Heading{Proof of Theorem~\ref{Tpner}:}
Suppose that $S(\wo) \neq \emptyset$.
Let \wt\ be \wo's partner, and let \wwh\ be the boundary of the cell 
from $W_{1}'$ to $W_{2}'$, such that $X \not\in \wwh$ (Figure~\ref{6}(a)).
%
\figg{6}{(a) \wwh\ is dotted (b) $y \in \seg{W_{1}s}$ (c) $s \in \seg{W_{1}y}$}{2.5in}
% (a) is Figure 3.15(a)  KEEP THIS EVEN THOUGH IT LOOKS SIMPLE: 
% 	IT HELPS AS A REFERENCE FOR IMPORTANT THINGS IN THE PROOF
% (b) is forge(a) (additional figure)
% (c) is Figure 3.15(b)
%
I claim that it is sufficient to show that $\wt ' \in S'(\wo) \subset \wwh$.
Suppose that this is true, and consider a traversal of the cell boundary from
$\wo '$ to X. Since $\wt '$ is an endpoint of \wwh\ and
$X \not \in \wwh$ (by definition), $\wt '$ must be the last element of $S'(\wo)$ 
that is met during this traversal.
In other words, $W_{2}' = S_{p}'$ ($W_{2} = S_{p}$) as desired.
(Since it can be shown that $S_{i}' \neq S_{j}'$ whenever $i \neq j$, there 
is no ambiguity in choosing the last member of $S'(\wo)$ or in associating
$S_{i}'$ with $S_{i}$ \cite{jj}.)  % p. 75

We will first show that $S'(\wo) \subset \wwh$. Let $s \in S(\wo)$.
Suppose, for the sake of contradiction, that $\ray{\wo s}$ crosses \wwa\
at $y \neq \wt$ (Figure~\ref{6}(b-c)).
There are two cases to consider: $y \in \seg{\wo s}$ and $s \in \seg{\wo y}$.
Suppose that $y \in \seg{\wo s}$ (Figure~\ref{6}(b)).
By Lemma~\ref{Ls}, 
\[ \#\{P \in \seg{\wo y} \cap F: \mbox{ P faces \wo}\} = 
\#\{P \in \seg{\wo y} \cap F: \mbox{ P faces y}\} \]
But y faces \wo, since \wo\ and y are on the same convex segment.
Therefore, there exists $\alpha \in \seg{\wo s}$ such that
\[ \#\{P \in \seg{\wo\alpha} \cap F: \mbox{ P faces \wo}\} 
> \#\{P \in \seg{\wo\alpha} \cap F: \mbox{ P faces s}\} \]
in contradiction of $s \in S(\wo)$.
Now suppose that $s \in \seg{\wo y}$ (Figure~\ref{6}(c)).
By the argument of the proof of Lemma~\ref{Ls}, the
points of intersection of the curve F with \seg{\wo y} pair up.
Let t be the partner of s .
Since \arc{st} is convex, s faces t; since $s \in S(\wo)$, s faces \wo. 
Therefore, $t \in \seg{\wo s}$.
Since \mbox{$s \in S(\wo)$}, 
\[ \#\{P \in \seg{\wo s} \cap F: \mbox{ P faces \wo}\}
= \#\{P \in \seg{\wo s} \cap F: \mbox{ P faces s}\} \]
Noting that $\seg{\wo s} = \seg{\wo t}\ \cup\ \seg{ts}\ \cup\ \{t\}$ and 
t faces s, this becomes 
\[ \#\{P \in \seg{\wo t} \cap F: \mbox{ P faces \wo}\} +
\#\{P \in \seg{ts} \cap F: \mbox{ P faces \wo}\} + 0 =  \]
\nopagebreak
\[ \#\{P \in \seg{\wo t} \cap F: \mbox{ P faces s}\} +
\#\{P \in \seg{ts} \cap F: \mbox{ P faces s}\} + 1\ \ \  \]
Moreover, by Lemma~\ref{Ls} (\arc{st} is convex),
\[ \#\{P \in \seg{ts} \cap F: \mbox{ P faces s}\} = \]
\[ \#\{P \in \seg{ts} \cap F: \mbox{ P faces t}\} = \]
\[ \#\{P \in \seg{ts} \cap F: \mbox{ P faces \wo} \} \]
Upon cancelling terms in the above equation, we conclude that
\[ \#\{P \in \seg{\wo t} \cap F: \mbox{ P faces \wo}\} > \]
\[ \#\{P \in \seg{\wo t} \cap F: \mbox{ P faces s}\} =\  \]
\[ \#\{P \in \seg{\wo t} \cap F: \mbox{ P faces y}\}\ \  \]
But this contradicts condition (3) of Lemma~\ref{Ls} 
(convex segment \wwa, $\x = \wo$, $\y = y$).
These contradictions lead us to conclude that \ray{\wo s} does not cross 
$\wwa \setminus \{\wt\}$.
In particular, by the definition of $s'$, 
\seg{\wo s'} does not cross $\wwa \setminus \{\wt\}$.
Therefore, $s'$ must either lie outside of \wo's tangent or on \wwh (Figure~\ref{6}(a)).
Since $s$, as a member of S(\wo), lies on the strict inside of \wo's tangent,
so must $s'$.
Therefore, $s' \in \wwh$ and $S'(\wo) \subset \wwh$, as desired.

We now show that $\wt \in S(\wo)$.
$\wt \in R(\wo)$ by Corollary~\ref{Cp},
so it suffices to show that \wt\ lies strictly inside of \wo's tangent.
Suppose, for the sake of contradiction, that \wt\ lies on \wo's tangent.
By Lemma~\ref{Ll}, \wo\ must be a flex (whose tangent is a cell wall).
Thus, the wall segment \seg{\wo\wt} is a subsegment of \wo's tangent and
\mbox{$S(\wo) \cap \seg{\wo\wt} = \emptyset$} (by definition of $S(\wo)$).
Therefore, \mbox{$S'(\wo) \cap \seg{\wo\wt} = \emptyset$}.
But $S'(\wo) \subset \wwh = \seg{\wo\wt}$.
Thus, $S'(\wo) = \emptyset$, which contradicts our initial assumption.
We conclude that \wt\ does not lie on \wo's tangent.
Since \wwa\ is a convex segment, \wt\ lies on the inside of \wo's 
tangent, and thus on the strict inside.

The statement of the theorem has been verified if $S(\wo) \neq \emptyset$.
Now suppose that $S(\wo) = \emptyset$.
If \wo\ is a refined singularity, then $\wt \in S(\wo)$:
$\wt \in R(\wo)$ (as \wo's partner);
\wt\ does not lie on \wo's tangent (Lemma~\ref{Ll});
and \wt\ lies inside \wo's tangent (because \wwa\ is convex).  
This would contradict the $S(\wo) = \emptyset$ assumption, so \wo\ cannot
be a refined singularity.
Therefore, \wo\ must lie on a wall of the cell and T(\wo) is well-defined.
If \wt\ lies strictly inside \wo's wall (w.r.t.\ C), it also lies strictly
inside \wo's tangent (Lemma~\ref{Ll}).
Therefore, if $\wt \not \in T(\wo)$, then $\wt \in S(\wo)$.
But $S(\wo) = \emptyset$, so $\wt \in T(\wo)$.

Suppose that \wt\ is not the closest member of $T(\wo)$ to \wo, and let 
\mbox{$U \neq \wt$} be the closest.  
Since \wo\ faces U, U must lie on $\seg{\wo\wt}$.
By the proof used in Lemma~\ref{Ls}, the nonsingular points of 
intersection of the curve with \seg{\wo\wt} must pair up into couples.
In particular, the endpoints on $\seg{\wo U} \subset \seg{\wo\wt}$ 
(all of which are nonsingular because of refinement) that face 
\wo\ must pair with the equal number of endpoints on \seg{\wo U} that face U.
But U must also pair with an endpoint on \seg{\wo U} that faces U,
and there are no such endpoints remaining without a partner.
This contradiction leads us to conclude that \wo's partner \wt\ must be the 
closest element of T(\wo) to \wo.
\QED

\subsection{Computation of Singularities and Flexes}
\label{ssc}

The above convex decomposition of an algebraic curve requires the singularities 
and flexes of the curve, as well as their tangents.
The singularities of a curve $f(x,y)=0$ are the solution set of the system
$\{f_{x}=0,f_{y}=0,f=0\}$, while the points of inflection are the nonsingular 
intersections of the curve with its Hessian (the determinant 
of the matrix of double derivatives of the curve's equation) \cite{walker}.
The restriction of points of inflection to flexes (see page~\pageref{restriction})
is straightforward \cite{jj}.
The tangents of a singularity of the curve $f=0$ can be found by translating 
the singularity to the origin.
The equations of the tangents are the factors of the translated $f$'s order
form (the polynomial consisting of the terms of lowest degree) \cite{walker}.
Finally, after the curve has been translated to projective space by homogenizing 
its equation to $f(x,y,z)=0$ (where $z$ is the homogenizing variable),
the tangent of a flex P is $f_{x}(P)x + f_{y}(P)y + f_{z}(P)z= 0$ \cite{walker}.
This completes our description of the convex decomposition of an algebraic curve.

\section{Point location}
\label{s-loc}

The second key problem in the convex segment method of sorting is
point location in the convex decomposition: given a point, identify
the convex segment that contains it.
This is an extension to the curved domain of the well-known problem of point 
location in a planar subdivision. 
We show how to locate points on both a convex segment and a general curve segment.

\subsection{Point location I: On a convex segment}

A decomposition is not very useful unless it is possible to locate points in it. 
In the case of sorting, point location is necessary to divide a set of points
into convex segments for conquering.
Since a convex segment is identified by its endpoints, finding the convex 
segment that contains a point is equivalent to finding the endpoints that bound 
this convex segment.
Fortunately, this problem is entirely analogous to finding the partners of a given
endpoint as explained in Section~\ref{sspII}, since both problems are 
instances of the more general question: ``what are the two endpoints associated 
with a given point?''
It is easy to locate a point in the proper cell, using well-known algorithms 
for point location in a planar subdivision \cite{kirk,PS}.\footnote{Artificial 
	boundaries are ignored when locating points in a cell: a point
	is considered to lie in an artificially closed cell C as long as it lies
	in the open cell associated with C.}
If, as is often the case, a point lies in a cell with only one convex segment, 
then it is obvious what convex segment it belongs to.  
Otherwise, Theorem~\ref{Tps} and Lemma~\ref{Ln} can be used 
to locate a point on the proper convex segment.\\[3pt]
%
\Heading{Definition}:
A {\em connected component} of a curve is a maximal subset of the curve such that 
there exists a continuous path on the curve between any two points of the subset.
For example, a hyperbola has two connected components.
A type of connected component that requires special treatment is one that
lies entirely inside of a cell, intersecting none of the walls (including
artificial walls) of the cell partition.
We call this a {\em nude} connected component since, unlike other connected 
components, it does not contain any endpoints of convex segments.
Since it does not contain any singularities or flexes, a nude component is convex.
It must also be closed (i.e., homeomorphic to a circle), otherwise it would 
intersect an artificial wall as it proceeded to infinity.
%
\begin{theorem}
\label{Tps}
Consider a point $x$ of curve F that lies in cell C and is not an endpoint of a 
convex segment.\footnote{If $x$ is an endpoint of a convex segment,
	then Theorem~\ref{Tpner} can be used to determine x's partner in C, 
	and thus its convex segment in C.}
Let $S(x)$ = \{endpoints W in C $\mid$
\begin{quote}
\begin{enumerate}
\item x lies on the strict inside of W's tangent
\item W lies on the strict inside of x's tangent 
\item \mbox{$\#\{P \in\seg{xW}\cap F: P \mbox{ faces x}\} =
\#\{P \in \seg{xW}\cap F: P \mbox{ faces W} \}$}
\item $\forall\ \alpha \in \seg{xW}$,
$\#\{P \in\seg{x\alpha}\cap F: P \mbox{ faces x}\} \leq
\#\{P \in \seg{x\alpha}\cap F: P \mbox{ faces W}\} $
\}
\end{enumerate}
\end{quote}
%
If $S(x) = \emptyset$, then x lies on a nude connected component. 
Otherwise, let $S''(x) = \{\ W'' : W \in S(x)\ \}$, where $W''$ is the intersection 
of \ray{xW} with the cell boundary.
Let $x_{1}$ and $x_{2}$ be the two points of intersection of x's tangent with the
cell boundary.
$x_{1}$ and $x_{2}$ split the cell boundary into two halves,
and every endpoint in $S''(x)$ lies on one of these halves.
Let $S_{1}'',S_{2}'',\ldots,S_{p}''$ be the result of a sort of $S''(x)$ 
from $x_{1}$ to $x_{2}$.
Then $S_{1}$ and $S_{p}$ are partners and x lies on the convex segment 
\arc{S_{1}S_{p}}.
\end{theorem}
%
\Heading{Proof:}
If x does not lie on a nude component, then $S(x) \neq \emptyset$, since it will 
contain the two endpoints of x's convex segment.
(One can also quite easily establish the converse: if x lies on a nude 
component, then $S(x) =  \emptyset$.)
%
% Let x be a point on a nude component in cell C, and consider an endpoint E of a 
% convex segment in C.
% E certainly does not lie on the nude component.
% If E lies on or outside of x's tangent, then condition (4) of S(x) is violated.
% If E lies strictly inside of x's tangent, then the first intersection of \seg{xE}
% with the curve after x is another point of x's nude component (since this 
% component is closed and convex).  
% But this intersection faces x (again by the convexity of nude components),
% violating condition (3) of S(x).
%
The rest of the proof is similar to the proof of Theorem~\ref{Tpner}, and the 
interested reader is referred to \cite{jj}.  % p. 80
\QED
%
\begin{example}

In Figure~\ref{nude}(a), $S(x) = \emptyset$ and x lies on a nude component.

Consider the cell of Figure~\ref{3.12} that contains the convex segments 
\wwa\ and \arc{W_{3}W_{4}}.
\wo\ does not satisfy condition~(2) of $S(x)$ and \wt\ does not satisfy condition~(3).
Thus, $S(x) = \{W_{3},W_{4}\}$ and x must lie on \arc{W_{3}W_{4}}.

Consider the cell partition of Figure~\ref{2.12a}.
$S(P_{1}) = \{W_{1},W_{2},W_{5},W_{6}\}$, which does not resolve the question
of $P_{1}$'s convex segment.
Let \xo\ and \xt\ be the two points of intersection of $P_{1}$'s tangent
with the cell boundary.
The sort of $S''(P_{1})$ from \xo\ to \xt\ is \wo, $W_{6},\ W_{5}$, \wt, 
so $P_{1}$ must lie on \wwa.
\end{example}
%
\figg{nude}{(a) $x$ lies on a nude component (b) two overlapping segments}{3in}
% (a) `nude' picture of nonsingular cubic with x on nude component:
%	use walker picture of nonsingular cubic
% (b) `overlap' picture in additional figures

If there is only one nude component in a cell, then Theorem~\ref{Tps} 
can successfully locate a point on this convex segment.
However, if there is more than one nude component in the cell, then the following
lemma must be used to distinguish these nude components.

\begin{lemma}
\label{Ln}
Let P and Q be points that lie on nude components of a curve and in the same cell.
P and Q lie on the same nude component if and only if 
Q lies in S(P), where S() is as in Theorem~\ref{Tps}.
\end{lemma}

\Heading{Proof:}
Let P and Q lie on nude components M and N, respectively.
If $M=N$, then P and Q lie on the same convex segment, so $Q \in S(P)$ by
Lemma~\ref{Ls}.
Suppose that $M \neq N$.
Nude components do not intersect, since they do not contain any singularities.
Therefore, there are only three cases to consider: M lies inside N, N lies 
inside M, and neither lies inside the other.
In all three cases, it is straightforward to show that Q violates one of the conditions
of S(P).
\QED

Point location can be made faster through two observations, 
both of which make use of the endpoint pairings already computed.
The idea is to eliminate endpoints from $S(x)$ in Theorem~\ref{Tps} faster.
First, as soon as the endpoint W is eliminated, W's partner can also be 
eliminated, since the two desired endpoints are partners.
Second, by convexity, the curve segment between an endpoint $W_{1}$ and its
partner $W_{2}$ lies on one side of $\seg{W_{1}W_{2}}$.
Thus, if $x$ does not lie on the appropriate side of $\seg{W_{1}W_{2}}$, then 
both $W_{1}$ and $W_{2}$ can be eliminated.
These observations should be used along with conditions (1-2)
to eliminate as many endpoints as possible from S(x) (in the best case, leaving 
only two). 
Conditions (3-4) should only be used when absolutely necessary, because they 
involve the expensive solution of an equation of degree $n$ 
(where $n$ is the degree of the curve F).
Fortunately, the only time that conditions (3-4) will be needed to locate a point on a 
convex segment is for a point that lies on one of two overlapping convex segments
\label{page-overlap}
in the same cell, as in Figure~\ref{nude}(b): $x$ lies inside all four endpoint's tangents
and all four endpoints lie inside $x$'s tangent.
Experience with algebraic curves (e.g. Lawrence's catalog of algebraic curves \cite{law}),
combined with experimental evidence, indicates that this situation is very rare:
a wall of the cell partition will almost always separate overlapping parts of the curve.
Therefore, a point can usually be located on a convex segment very cheaply.

This completes our description of techniques that are needed for sorting by 
the convex segment method.
We digress for a moment to show how the theory that we have developed can be used
to solve two important problems (although they are not needed for sorting):
locating a point on an arbitrary segment and deciding whether two points lie on the
same connected component.

\subsection{Point location II: On an arbitrary segment}

Once it is known how to locate a point on a convex segment of a curve's
convex decomposition, it is straightforward to solve the more
general problem of locating a point on an arbitrary segment of the curve.
Recall that every endpoint of a convex segment in our (normalized) convex
decomposition has exactly two partners.
Therefore, every convex segment has a unique predecessor and successor,
and it is trivial to order the convex segments.
Consider a segment \arc{AB} of curve C and a point P on C.
To decide if P lies on \arc{AB}, we compute the convex segments of C's 
decomposition that contain P, A, and B (say $C_{p}$, $C_{a}$, and $C_{b}$, 
respectively).
Then, P lies on \arc{AB} if and only if $C_{p}$ lies in between $C_{a}$ and
$C_{b}$.
If P lies on the same convex segment as A and/or B, then the decision
requires more subtlety.
For example, if P lies on the same convex segment \arc{EF} as A (but not B), then
the decision is made by sorting P, A, E, and F along \arc{EF}, 
using Theorem~\ref{T-s}:
$P \in \arc{AB}$ if and only if the order is E, P, A, F (resp., E, A, P, F) and
\arc{AB} leaves A towards E (resp., F).
(A method for deciding if \arc{AB} leaves A towards E or F is described in a footnote 
on page~\pageref{alg-page}.)
%	We assume that we are given a directed tangent at A that indicates the 
%	direction of \arc{AB} from A.
In short, point location on an arbitrary segment is easily reducible to point
location on a convex segment.

%NO USE REPEATING THIS
%Point location could be done with a rational parameterization:
%the points that lie on the curve segment \arc{AB} are simply those
%with parameter values that occur between the parameter values of A and B.
%However, this has the same disadvantages as those of parameterizations for 
%sorting: lack of universality and inefficiency of solving the parameterization.

\subsection{Curves with many connected components}

It should now be clear that the convex segment method can sort points on any
algebraic curve.
In particular, it can sort points that are strewn over several connected 
components of a curve, with no more difficulty than sorting points on a single
component.
This is another advantage of the convex segment method over the parameterization
method, because it is not clear how the latter method could deal with points
on several components, even if we allow nonrational parameterizations.
%
% Harnack's Theorem~\cite{lang1} states that a nonsingular plane curve of
% genus g can have at most $g+1$ connected components.
% Therefore, a nonsingular plane curve with more than one component is nonrational.
% This doesn't say much, since not many curves are nonsingular.
% Bajaj hypothesizes that a plane curve has most connected components when it
% is nonsingular, so that this result actually applies to all curves (i.e., 
% nonsingular plane curves are the only ones you have to consider because they
% establish the maximum)
%
Would each connected component have a separate parameterization?
If so, how would the single equation of a curve produce several independent 
parameterizations?
If not, how would one determine the range of parameter values that is associated 
with each connected component?
% The second question might be answered by considering the complex curve which
% is the zero set of the curve's equation over the field of complex numbers.
% The connected components of the real curve are, in effect, the different
% components of the real plane's cross-section of the complex curve.
% (The complex curve lies in two-dimensional complex space, which is 
% of dimension four over $\Re$, so the complex curve behaves like a surface
% rather than a curve.)
% The parameterizations of the connected components of the real curve
% might be realized as the different components of the real plane's
% 'cross-section' of the (complex) parameterization of the complex curve.

A very useful test for a curve with several components is whether two points 
lie on the same connected component.
For example, with this capability it is reasonable to define an edge of a solid model 
as a particular connected component of a multi-component curve, since the test allows 
you to restrict intersections with the curve to this connected component.
The following lemma shows that our decomposition of 
the curve into convex segments makes it simple to perform this test.
(Lemma~\ref{Ln} can be used for points on nude components.)

\begin{lemma}
Let P and Q be points of a curve, not both of which lie on a nude component.
Let P and Q lie on convex segments \arc{AB} and \arc{CD}, respectively.\footnote{If 
	P (resp., Q) lies on a nude component, then A and B (resp., C and D) are
	null symbols.}
P and Q lie on the same connected component if and only if $A \equiv C$, where
$v \equiv w$ if and only if \arc{vw} is a convex segment of our cell partition 
or $v \equiv z$ and $w \equiv z$ for some z.
\end{lemma}

\noindent Two other decompositions of an algebraic curve, Collins' cylindrical algebraic 
decomposition \cite{Co75,arnon83} and Canny's stratification \cite{Ca}, 
can also be used to separate a curve into connected components and thus decide whether
two points lie on the same connected component.
% The topology of an algebraic curve can be determined from this decomposition
% \cite{arnon83,kozen}.
% It should be straightforward to determine boundaries for each connected 
% component from this topological information.
% The weakness of these approaches is the exponential, double-exponential, or 
% parallel-exponential complexity of the algorithms that compute the 
% decompositions.

\subsection{Broad comparison of methods}

Let us compare the convex segment method of sorting with the others that were mentioned in
Section~\ref{sp}.
Like the brute-force tracing method, the convex segment method leaps from one 
point to another along the curve (viz., from an endpoint to its partner).
However, its jumps are large while the tracing method's jumps must be very small.
Moreover, once the partner of each convex segment endpoint of the cell partition has
been computed (which can be done once and for all in a preprocessing step),
each jump of the convex segment method can be done very quickly; whereas, the tracing
method must grope for some time (by applying Newton's method) to find the destination 
of each jump.
In short, the convex segment method 
makes large, bold jumps while the tracing method makes small, timid ones.

The convex segment method is similar to the parameterization method because
they both reduce the sorting problem to an easier one.
The parameterization method observes that the sorting of points 
on a line is simple and tries to unwind the curve into a line by parameterizing it.
Rather than trying to reduce the entire problem, the convex segment method divides 
the problem up into many smaller ones (viz., the sorting of points on a convex 
segment).
We shall see that the many small reductions of the convex segment method
can be done more quickly than the single, large reduction of
the parameterization method.

The convex segment method incorporates preprocessing, since the convex decomposition
of a curve can be done at any time.
As a result, the actual sorting is usually very efficient.
One might consider the parameterization of a curve to be preprocessing, but
the subsequent runtime steps (solving for the parameter value of each point)
are usually more expensive than those for the convex segment method (following pointers,
locating points, and sorting convex segments).

\section{Complexity}
\label{s-c}

In this section, we analyze the complexity of the convex segment method of sorting.
We base our complexity analysis on the RAM model, where basic arithmetic operations
are of unit cost \cite{ahu}.

\subsection{Complexity of convex decomposition}

\begin{theorem}
A curve of order $n$ (a curve whose defining polynomial is degree $n$) 
can be decomposed into convex segments in
O($\alpha[n^{2}] + n^{2}\alpha[MAX * n] + n^{6}\alpha[n]$) time, 
where $\alpha[n]$ is the time required to find the 
real roots of a univariate polynomial equation of degree $n$, and MAX is the 
maximum number of quadratic transformations that are necessary to decompose any 
singularity of the curve into simple points.\footnote{MAX is 1 if each 
	singularity has distinct tangents, and MAX will usually be 1 or 2
	in geometric modeling applications. }
\end{theorem}
%
\Heading{Proof:}

{\em Computation of singularities, flexes.} 
%
Consider the curve $f(x,y) = 0$ of order $n$.
Its singularities are found by solving the simultaneous system
of equations \mbox{$\{f_{x} = 0, f_{y}=0, f = 0\}$}.
One method is to use resultants \cite{walker}.
The resultant of two polynomials with respect to the variable $x_{n}$ is a polynomial
whose roots are the projection onto the hyperplane $x_{n} = 0$
of the intersections of the two polynomials.
Let $X$ (resp., $Y$) be the real roots of the resultant of $f_{x}$ and $f_{y}$ 
with respect to $y$ (resp., $x$), which is a univariate polynomial in $x$ (resp., $y$)
of degree O($n^{2}$).
Since singularities at infinity are not of interest, those roots in X (resp., Y)
that cause the terms of highest degree of \mbox{$\{f_{x} = 0, f_{y}=0\}$} to 
simultaneously vanish are not of interest.
(The terms of highest degree of a polynomial are intimately related to its solutions at 
infinity, since they dominate the polynomial as solutions get large.)
Therefore, before computing the roots of the resultant, the GCD of the leading term
polynomials of $f_{x}$ and $f_{y}$ is computed and divided out of the resultant, all in 
O($n\ log^{2}\ n$) time \cite{ahu}.
Now X (resp., Y) is the collection of abscissae (resp., ordinates)
of the finite-solution set of \mbox{$\{f_{x} = 0, f_{y}=0\}$}.
X (and Y) can be computed in O($n^{4}\ log^{3}\ n + \alpha[n^{2}]$) time, since
the resultant of a pair of polynomials of degree at most $n$ in $r$ variables 
can be computed in O($n^{2r}\ log^{3}\ n$) time \cite{bajj}.
% O($n^{5}\ log\ n + \alpha[n^{2}]$) time, since
% the resultant of a pair of polynomials of degree at most $d$ in $v$ variables 
% can be computed in O($d^{2v+1}\ log\ d$) time \cite{col}.
The singularities of the curve are
\mbox{$\{\ (x,y)\ :\ x\in X,\ y \in Y \mbox{ and } f(x,y) = f_{x}(x,y) = f_{y}(x,y) = 
0\ \}$.}
This pairwise substitution takes O($n^{6}$) 
time, since X and Y are each of size O($n^{2}$) and the evaluation of an equation of 
degree $n$ requires O($n^{2}$) time.
Hence, all singularities of the curve can be computed in O($\alpha[n^{2}] + n^{6}$)
time.
With similar techniques, the flexes can also be computed in O($\alpha[n^{2}] + n^{6}$)
time.

{\em Computation of their tangents.} 
%
Recall that the tangents at a singularity ($a, b$) are computed by 
translating the singularity to the origin and factoring the polynomial consisting of 
the terms of lowest degree of the translated f(x,y) into linear factors.
(For example, the lines $x-y=0$ and $x+y=0$ are the tangents of the curve
$x^{3} - x^{2} + y^{2} = 0$.)
A translation is simply a linear substitution $x_{t} = x - a ,\ y_{t} = y - b$,
which takes O($n^{4}$) time for a bivariate equation of order $n$.
% Proof: there are O(n^4) terms to compute: there are O(i^3) terms generated by 
% the terms of degree i: consider the terms generated by the terms of degree n:
% there are n terms for (x-a)^n, n-1 terms for (x-a)^n-1 * y, (n-2)*2 terms for
% (x-a)^n-2 * y^2, and in general (n-i) * i terms for (x-a)^n-i * y^i, which sums
% to O(n^3) terms generated by the terms of degree n
The factorization of a homogeneous bivariate polynomial is equivalent to the
solution of a univariate polynomial.
Therefore, the computation of the tangents at a singularity requires
O($n^{4} + \alpha[n]$) time.
A curve of order $n$ has at most O($n^2$) singularities \cite{walker},
so all of the tangents at singularities can be computed in O($n^{6} + n^{2}\alpha[n]$)
time.
The computation of the tangent at a flex is easier, only involving the O($n^{2}$)
operation of bivariate (or homogeneous trivariate) polynomial evaluation 
(Section~\ref{ssc}).
A curve of order $n$ also has at most O($n^{2}$) flexes \cite{walker},
so all of the tangents at flexes can be computed in O($n^{4}$) time.

{\em Computation of intersections of singularity/flex tangents with curve.}
%
The intersections of the singularity/flex tangents with the curve are needed to
create the convex decomposition.
Consider the number of tangents.
There are at most O($n^{2}$) tangents at flexes.
A curve of order $n$ has at most $\frac{(n-1)(n-2)}{2}$ double points, where
a singularity of multiplicity $t$ counts as $\frac{t(t-1)}{2}$ double points 
and has O(t) tangents \cite{walker}.
Consequently, there are $t / \frac{t(t-1)}{2} < 2$ tangents per double point,
or at most O($n^{2}$) singularity tangents.
The intersection of a tangent with the curve involves a linear substitution
and a solution of the resulting polynomial, thus O($n^{4} + \alpha[n]$) time or
O($n^{6} + n^{2}\alpha[n]$) for all tangents.
Note that the O($n^{2}$) tangents generate O($n^{3}$) endpoints on the curve, since
each tangent intersects the curve in at most $n$ points (Bezout's Theorem).

{\em Refinement of singularities and infinite segments.}
%
A singularity of multiplicity $t$ is refined into O($2t$) endpoints, meaning
$2t / \frac{t(t-1)}{2} \leq 4$ refined endpoints per double point,
or a total of $O(n^{2})$ refined endpoints at singularities.
Thus, the number of endpoints of convex segments 
(and the number of convex segments) remains O($n^{3}$) after refinement.
Consider the time that is required to refine the singularities.
Each singularity is translated to the origin and subjected to quadratic transformations
(perhaps translating the singularity back to the origin after certain quadratic 
transformations).
O($n^{2}$) quadratic transformations are sufficient to reduce all of the singularities 
to simple points, since the singularities of a curve of order $n$ account in total for 
O($n^{2}$) double points and the application of each quadratic transformation 
removes at least one double point, in a global amortized counting \cite{abba3}. 
We have seen that the translation of a curve requires $O(n^{4})$ time, 
amounting to a total O($n^{6}$) translation time. 
% By suitable choice of coordinates, it can be arranged that as long as the 
% multiplicity of a singularity is not reduced on application of a quadratic 
% transformation, no translation needs to be performed.
% Such a choice of coordinates (which takes $O(n^{4})$ time) can be determined 
% {\em a priori} for each singularity of the curve \cite{abhy}.
% As this is going to be applied for at most O($n^{2}$) singularities representing
% O($n^{2}$) double points,
% the total time taken by all of the translations is O($n^{6}$) time. 
Each quadratic substitution $x = x_{1},\ y = x_{1}y_{1}$ takes O($n^{2}$) time 
(there are O($n^2$) terms in the original equation of the curve).
Therefore, all of the quadratic transformations take O($n^{4}$) time.

During the reduction of a singularity to simple points, each quadratic
transformation can increase the degree of the curve's equation, since
$x^{i}y^{j}$ becomes $x^{i}(x^{j-d}y^{j}) = x^{i+j-d}y^{j}$, where $d$ is the 
multiplicity of the singularity.\footnote{It might appear that 
	$x^{i}y^{j}$ should become $x^{i}(x^{j}y^{j})$.  However,  
	redundant factors must be removed from the polynomial.
	For example, $x^{2} - y^{3} = 0$
	becomes $1-xy^{3} = 0$, not $x^{2} - x^{3}y^{3} = 0$.
	The equation of a curve with a singularity of multiplicity $d$ 
	at the origin has no terms of degree less than $d$, so a factor of $x^{d}$ can
	always be removed.}
In other words, the degree of the polynomial can increase by $O(j)$, where
$j$ is the highest degree of $y$ in any term of the polynomial undergoing 
quadratic transformation. 
Since $j=n$ for the polynomial of the original curve and the y-degree of every term
remains invariant under quadratic transformation (and does not increase under
translation of the curve either), the degree of the polynomial can only increase
by $O(n)$ with each quadratic transformation.
Therefore, by the end of the reduction of a singularity to simple points, 
the curve's equation can be of degree $O(MAX * n)$.

Finally, after a quadratic transformation where the multiplicity of the singularity 
drops, one computes the intersections of the new curve of order $i$ with the y-axis, 
which takes $\alpha[i]$ time.
Again, since this is computed after at most O($n^{2}$) quadratic transformations,
the total time taken by all of the intersection computations is at most 
O($n^{2} \alpha[MAX * n]$) time.
% Can't get asymptotically better even if you analyze more carefully by considering
% \frac{n^2}{MAX} iterations of \alpha[n] + \alpha[2n] + ... + \alpha[(MAX+1)n].
We conclude that a (pessimistic) bound on the time for refining the
convex segment endpoints at singularities is O($n^{6} + n^{2} \alpha[MAX * n]$).
There are at most two infinite segments, which are comparatively simple to refine.

{\em Pairing endpoints.}
%
Consider the time required to compute the partners of the O($n^{3}$) endpoints.
The dominating expense is the computation of the set R(\wo) of 
Theorem~\ref{Tpner} for each endpoint \wo.
It takes O($k\alpha[n]$) time to compute R(\wo) for an endpoint in a cell
with $k$ endpoints, O($k^{2}\alpha[n]$) time to compute R(\wo) for every 
endpoint in a cell with $k$ endpoints, and O($\sum k_{i}^{2}\alpha[n]$)
time to compute R(\wo) for every endpoint in every cell, where $k_{i}$ is the
number of endpoints in cell $C_{i}$ and the sum is over all cells $C_{i}$.
Since $\sum k_{i} = O(n^{3})$, $O(\sum k_{i}^{2}\alpha[n]) = O(n^{6}\alpha[n])$.
Therefore, partner computation takes $O(n^{6}\alpha[n])$ time.
\QED

It must be emphasized that the $n$ of the above analysis is the order of the curve.
This makes the analysis fundamentally different from those that we
are familiar with, such as O($n \log n$) for sorting numbers (where
$n$ is the {\em number of points}) or O($n \log \log n$) for triangulating a simple
polygon (where $n$ is the {\em number of edges} of the polygon).
(For example, in the above analysis, $n$ is the constant 1 for all polygons.)
As a result, the complexity of an operation such as the convex decomposition of an
algebraic curve can be misleading, since it is very easy (although wrong) to 
compare it with familiar complexities of discrete (rather than continuous) algorithms 
such as number sorting or polygon manipulation.

It should also be noted that the above analysis is pessimistic.
The worst case time will be reached only by the most pathological curves: the time to 
decompose curves that arise in practice in geometric modeling is much more reasonable.
For example, a typical endpoint will lie on the boundary of a single-segment cell and its 
partner will be computed in O(1), not O($k \alpha[n]$), time.
This observation has been borne out in practice, with the testing of the algorithms on
various curves (see Section~\ref{data}).
The efficiency will be even further improved by the fact that the singularities and 
flexes, which are important to other geometric algorithms, may already be available in 
many cases.

\subsection{Complexity of sorting}

We now consider the complexity of sorting points along a curve after its convex
decomposition is available.
This sorting is usually very efficient, because the traversal of a curve by
convex segments has been reduced to the traversal of a doubly linked list, and
it is usually simple to find the points on each convex segment.
Once again, the following worst-case analysis is unrealistically pessimistic
for geometric modeling applications.

\begin{theorem}
After the curve has been decomposed into convex segments,
$m$ points on a plane algebraic curve of order $n$ can be sorted
by the convex segment method in O($m n^{3} \alpha[n] + m\log m$) time.
If the curve does not have overlapping segments (see page~\pageref{page-overlap}), then
$m$ points can be sorted in O($m n^{3} + m\log m$) time.
\end{theorem}
%
\Heading{Proof}:
The dominating expense of sorting is to locate every point on a convex segment,
since the convex segments are already implicitly sorted (by endpoint pairing) and 
the sorting of points along a convex segment is simple
(by Theorem~\ref{T-s}, it is equivalent to the $O(k \log k)$ operation of finding
and sorting a set of angles). 
A point can easily be located in the proper cell of the cell partition.
A vector of size $O(n^{2})$ is associated with each of the $m$ points and each
cell: this vector specifies the side (inside or outside) of each singularity/flex tangent
that the point or cell lies on.
A point lies in a cell if and only if their two vectors match.\footnote{The vector
	of a cell need not, and will not, be complete.  Only the entries for the
	cell's walls are necessary.}
Therefore, the only potentially challenging step is locating the convex segment
in the cell that contains the point.
In the worst case, it requires $O(k \alpha[n])$ time to compute the set 
S($x$) of Theorem~\ref{Tps} for a point in a cell with $k$ endpoints, since
the intersection of line segments with the curve is required.
There are $O(n^{3})$ endpoints, so point location requires $O(n^{3}\alpha[n])$ time per
point and $O(mn^{3}\alpha[n])$ time for all points.\footnote{Observe the worst-case
	pessimism of this analysis.  It is unlikely that there are $O(n^{3})$
	real endpoints, since many of the $n$ intersections of a singularity/flex
	tangent with the curve will be complex.  It is extremely unlikely that all
	of these endpoints are in the same cell and that none of these endpoints
	would be eliminated by the cheap O(1) conditions of Theorem~\ref{Tps}.}
After adding $O(m \log m)$ time for sorting the points along the convex segments,
the convex segment method requires worst-case O($mn^{3}\alpha[n] + m \log m$) 
time to sort $m$ points by traversing $O(n^{3})$ convex segments.
If the curve does not have overlapping segments, then curve-line intersection can be
avoided in the computation of the set S($x$), thus dropping the $\alpha[n]$ factor.
\QED

\section{Execution times}
\label{data}
This section presents execution times for the sorting of some representative
curves by the convex segment and parameterization methods.
These empirical results are a good complement to the complexity analysis 
of Section~\ref{s-c}, since they capture the expected case, rather than 
the worst case, behaviour of the methods.
The source code was written in Common Lisp and
execution times are in seconds on a Symbolics Lisp Machine, 
not including time for disk faults and garbage collection.
Times for the convex segment method are the average of twelve trials, 
while times for the parameterization method are the average of three trials.
Preprocessing time is the time required to create the cell partition 
and find the partners of all of the endpoints.
Five curves are examined: two rational cubic and three non-rational quartic.

We do not consider the time required to find a parameterization of the
curve or to find the flexes and singularities of the curve.
Each of these computations is a preprocessing step that is 
entirely independent of sorting, and often the parameterization, singularities, 
and flexes of a curve will already be available.
Moreover, the computation of a curve's parameterization is of approximately the same
complexity as the computation of a curve's singularities and flexes,
so our comparison of sorting methods should not be biased.

The first example illustrates the superiority of the convex segment method:
even when the preprocessing time is added to the sorting time, it is more efficient.
Also notice that the rate of growth of the convex segment method is much smaller.
The inferiority of the tracing method (see end of Section~\ref{sp})
is obvious from this example, and we do not consider it further.
%
\begin{example}
A semi-cubical parabola\\
Equation of the curve: $27 y^{2} - 2x^{3} = 0$\\
Preprocessing time: 0.27 seconds\\
Parameterization: \{$x(t) = 6t^{2}$,\vspace{.5in} $y(t) = 4t^{3}\ :\ 
t \in (-\infty, +\infty) \}$\\
% -\infty\ <\ t\ <\ +\infty \}$\\
%
\begin{tabular}{|l|c|c|c|}  \hline
number of sortpoints & 1 & 2 & 6 \\ \hline \hline
convex segment &           .01 & .03 & .03 \\ \hline
%\footnote{The results should 
%only be compared vertically, not horizontally.
%The reason that sorting times sometimes decrease as more points are sorted
%is that completely different sets of points may be used in each column
%(e.g., the five points of column 3
%are not a subset of the six points of column 4)
%or different start and end points may be used.
%Thus, for example, sorting many points that are close together on a short
%sort segment may be faster than sorting a few points that are spread out
%on a long sort segment.} 
convex segment + preprocessing & .28 & .30 & .30 \\ \hline
parameterization & .47 & .63 & 1.04 \\ \hline
tracing         & 3.14 & 2.89 & 4.77 \\ \hline
\end{tabular}
\end{example}

% \figg{"A picture of the semi-cubical parabola"}{Semi-cubical Parabola}{3in}

\clearpage

The second example illustrates the tradeoff between a very fast sort 
that requires preprocessing (convex segment method) and a moderately fast 
sort that does not require preprocessing (parameterization method).
%
\begin{example}
\label{eg-folium}
Folium of Descartes\\
Equation of the curve: $x^{3} + y^{3} - 15xy = 0$\\
Preprocessing time: 2.81 seconds\\
Parameterization: \{$x(t) = \frac{15t}{1+t^{3}}$, \vspace{.5in}$y(t) = 
\frac{15t^{2}}{1+t^{3}}\ :\ t \in (-\infty, +\infty) \}$ \\
% -\infty\ <\ t\ <\ +\infty \}$ \\
%
\begin{tabular}{|l|c|c|c|c|} \hline
number of sortpoints & 1 & 2 & 5 & 9 \\ \hline \hline
convex segment &           0.01 & 0.01 & 0.05 & 0.04 \\ \hline
convex segment + preprocessing & 2.82 & 2.82 & 2.85 & 2.85 \\ \hline
parameterization & 1.01 & 1.07 & 1.76 & 3.17 \\ \hline
\end{tabular}
%
\end{example}
% \figg{"A picture of the folium"}{Folium of Descartes}{2in}
%
The remaining three curves are non-rational, so they are only 
sorted with the convex segment method.

\begin{example}
\label{eg-devil}
Devil's Curve (with several connected components)\\
Equation of the curve: $y^{4} - 4y^{2} - x^{4} + 9x^{2} = 0$\\
Preprocessing time: 2.20 \vspace{.5in}seconds\\
%
\begin{tabular}{|l|c|c|c|} \hline
number of sortpoints & 1 & 4 & 7 \\ \hline \hline
convex segment & 0.09 & 0.09 & 0.10 \\ \hline
convex segment + preprocessing & 2.29 & 2.29 & 2.30 \\ \hline
\end{tabular}
\end{example}

% \figg{"A picture of the devil"}{Devil's curve}{2in}

\begin{example}
\label{eg-limacon}
Limacon\\
Equation of the curve: $x^{4} + y^{4} + 2x^{2}y^{2} - 12x^{3} - 12xy^{2} + 27x^{2} - 9y^{2} = 0$\\
Preprocessing time: 4.62\vspace{.5in} seconds\\
%
\begin{tabular}{|l|c|c|c|} \hline
number of sortpoints & 2 & 5 & 8 \\ \hline \hline
convex segment & .09 & .30 & .55 \\ \hline
convex segment + preprocessing & 4.70 & 4.92 & 5.17 \\ \hline
\end{tabular}
\end{example}

% \figg{"A picture of the limacon"}{Limacon}{1.75in}

\clearpage
\begin{example}
\label{eg-Cassinian}
Cassinian oval\\
Equation of the curve: $x^{4} + y^{4} + 2x^{2}y^{2} + 50y^{2} - 50x^{2}-671 = 0$\\
Preprocessing time: 5.36\vspace{.5in} seconds\\
%
\begin{tabular}{|l|c|c|c|} \hline
number of sortpoints & 2 & 4 & 6 \\ \hline \hline
convex segment & .14 & .17 & .19 \\ \hline
convex segment + preprocessing & 5.50 & 5.53 & 5.55 \\ \hline
\end{tabular}
\end{example}
% \figg{"A picture of the Cassinian oval"}{Cassinian oval}{2.25in}


\section{Comparison of sorting methods}
\label{sc}

In this section, we consider the relative merits of the parameterization and
convex segment methods of sorting.
Certain curves cannot, or should 
not, be sorted by the parameterization method: curves that 
do not possess a rational parameterization and curves for which
a rational parameterization cannot be efficiently obtained.
Therefore, the convex segment method is often 
the only viable way to sort points along a curve.

For those curves that can be sorted in either way, the convex segment method 
is generally far more efficient than the parameterization method at the actual 
sorting of the points.
However, the parameterization method does not have the expense of 
preprocessing that the convex segment method does.
Therefore, when only a few points need to be sorted (over the entire lifetime 
of the curve) and the sorting of these points must be done soon after the 
definition of the (rational) curve, the parameterization method will usually 
be the method of choice.
(However, we have seen an example where the convex segment method is superior
to parameterization even when we include preprocessing time.)
The expense of preprocessing will be warranted whenever sorting time is a 
valuable resource, as in a real-time application, or when the number of points 
that will be sorted is large.
The convex segment method will also be preferable when the curve is defined 
long before it is ever sorted (as with a complex solid model that requires 
several days, weeks, or even months to develop), since the preprocessing can 
be done at any time that processing time becomes available before the sort.
We conclude that the convex segment method is an effective new method for
sorting points along an algebraic curve, and that in many situations it is 
either the only or the best method.

\section{Conclusions}
\label{sco}
We have developed a new method of sorting points along an algebraic curve
that is superior to the conventional methods of sorting.
Many curves that could not be sorted, or that could only be sorted slowly,
can now be sorted efficiently.
% The convex segment method of sorting can even be extended to curves defined by an
% arbitrary non-polynomial function, by truncating the power series expansion
% (e.g., Taylor series expansion) of the function about the points to be sorted,
% thereby yielding piecewise algebraic curve approximations to the original curve.
%%%%  The approximations are controlled by the number of terms considered in the 
%%%%  truncated expansion.
The development of our new method has also illustrated how an algebraic curve can 
be decomposed into convex segments, how to locate points on segments 
of algebraic curves, and how to decide whether two points lie on the same
connected component.

This work is one of the first solutions
of a computational geometry problem that is applicable to curves of arbitrary degree.
Methods are usually restricted to curves/surfaces of some specific or 
bounded degree, such as polygons/polyhedra or quadrics.
The creation and manipulation of curves and surfaces is of major importance 
to geometric modeling.
A sophisticated geometric modeling system should offer 
a rich collection of tools to aid this manipulation.
This paper has been an examination of one of these tools.
The progress of geometric modeling depends upon the development of more tools and upon 
the extension of more computational geometry algorithms from polygons to curves and 
surfaces of higher degree.

\section{Acknowledgements}
This work formed part of the thesis of J. Johnstone, who is grateful for
the guidance of his advisor, John Hopcroft.
%

% ARE THESE ABBREVIATIONS CORRECT?
% CORRECT HERTEL/MEHLHORN REFERENCE (E.G., WHAT DOES FTC STAND FOR? WHAT LNCS # IS IT?)

\begin{thebibliography}{Abhyankar 87b}

% \bibitem{abhy} Abhyankar, S., (1983),
% ``Desingularization of Plane Curves,''
% {\em Proc. of Symposia in Pure Mathematics}, AMS, Vol. 40, No. 1, 1-45.

% \bibitem{abba1} Abhyankar, S., and Bajaj, C. (1987),
% ``Automatic Rational Parameterization of Curves and Surfaces I:
% Conics and Conicoids,'' {\em Computer Aided Design}
% 19:1, 11-14.
% % Jan. 1987

% \bibitem{abba2} Abhyankar, S., and Bajaj, C. (1987),
% ``Automatic Rational Parameterization of Curves and Surfaces II:
% Cubics and Cubicoids,'' {\em Computer Aided Design}
% 19:9, 499-502.
% % Nov. 1987

\bibitem{abba3} S. S. Abhyankar and C. Bajaj,
{\em Automatic parameterization of rational curves and surfaces III:
algebraic plane curves},
Computer Aided Geometric Design,  1988, to appear.
% Tech. Rep. CSD-TR-619, Dept. of CS, Purdue Univ.
% Aug. 1986

% \bibitem{abba4} Abhyankar, S., and Bajaj, C., (1987),
% ``Automatic Rational Parameterization of Curves and Surfaces IV: Algebraic Space Curves,''
% Computer Aided Geometric Design, to appear.
%% Tech. Rept. 703, Comp. Science, Purdue University.

\bibitem{bajj}
C. Bajaj and A. Royappa,
{\em Note on an efficient implementation of Sylvester's resultant for multivariate 
polynomials},
Technical Report CSD-TR-718, Dept. of Computer Science, Purdue University, 1987. 

\bibitem{ahu} 
A. Aho, J. Hopcroft, and J. Ullman, 
{\em The Design and Analysis of Computer Algorithms},
Addison-Wesley, Reading, MA, 1974.

\bibitem{arnon83}
D. S. Arnon, {\em Topologically reliable display of algebraic curves},
J. Computer Graphics, 17(1983), pp. 219--227.

\bibitem{As} 
T. Asano, T. Asano, and H. Imai,
{\em Partitioning a polygonal region into trapezoids},
Res. Mem. RMI84-03, Dept. Math. Eng. and Instrumentation Physics, Univ. of 
Tokyo, 1984.

\bibitem{Av} 
D. Avis and G. T. Touissant,
{\em An efficient algorithm for decomposing a polygon into star-shaped
components},
Pattern Recognition, 13(1981), pp. 395-398.

% \bibitem{bhh} Bajaj, C., Hoffmann, C., and Hopcroft, J.,
% ``Tracing Planar Algebraic Curves,''
% Tech. Rep. CSD-TR-637, Dept. of Computer Science, Purdue Univ,
% September 1987.

\bibitem{bhhl} 
C. Bajaj, C. Hoffmann, J. Hopcroft, and B. Lynch,
{\em Tracing surface intersections},
Computer Aided Geometric Design, 1988, to appear.
% Tech. Rept. 725, Comp. Science, Purdue University. 

% \bibitem{BK87a} Bajaj, C., and Kim, M., (1987),
% Convex Decomposition of Objects Bounded by Algebraic Curves.
% Tech. Rept. 677, Comp. Science, Purdue University.

% \bibitem{bajkim87b}
% Bajaj, C., and Kim, M., (1987),
% Convex Hull of Objects Bounded by Algebraic Curves.
% Tech. Rept. 697, Comp. Science, Purdue University.

\bibitem{Ca}
J. F. Canny, 
{\em The Complexity of Robot Motion Planning},
MIT Press, Cambridge, MA, 1987.
% Ph.D. Thesis, Dept. of Electrical Engineering and Computer Science, MIT, 1987.

\bibitem{cd} 
B. Chazelle and D. P. Dobkin, (1985).
{\em Optimal convex decompositions},
in Computational Geometry, G. T. Toussaint, ed., North-Holland, New York, 1985, 
pp. 63-133.

\bibitem{CI} B. Chazelle and J. Incerpi, 
{\em Triangulation and shape-complexity},
ACM Trans. on Graphics, 3(1984), pp. 135-152.
% April 1984

\bibitem{Co75}
G. E. Collins, 
{\em Quantifier elimination for real closed fields by cylindrical 
algebraic decomposition}, 
Proc. 2nd GI Conference on Automata Theory and Formal Languages, 
Lecture Notes in Computer Science 35, Springer, Berlin, 1975, pp. 134--183.

% \bibitem{col}
% G. E. Collins, ``The Calculation of Multivariate Polynomial Resultants,''
% {\it Journal of the ACM} 18:4 (October 1971), 515--532.
%% especially page 527

\bibitem{Do}
M. P. Do Carmo, 
{\em Differential Geometry of Curves and Surfaces},
Prentice-Hall, Englewood Cliffs, NJ, 1976.

% \bibitem{edelsbrunner} Edelsbrunner, H. (1987).
% Algorithms in Combinatorial Geometry.
% New York: Springer-Verlag.

\bibitem{G} 
M. Garey, D. S. Johnson, F. P. Preparata, and R. E. Tarjan,
{\em Triangulating a simple polygon},
Info. Proc. Lett., 7(1978), pp. 175--80.
% 7(4)

% \bibitem{gj87} Goodrich, M. T., \& Johnstone, J. K. (1987).
% Coordinated Walking: A Method for Intersecting Algebraic Curve Segments.
% Manuscript.

\bibitem{H} 
S. Hertel and K. Mehlhorn, 
{\em Fast triangulation of simple polygons},
Proc. FCT'83, Borgholm, Lecture Notes in Computer Science, 
Springer, Berlin, 1983, pp. 207--218.

\bibitem{hh87}
C. Hoffmann and J. Hopcroft, 
{\em The potential method for blending surfaces and corners},
in Geometric Modeling: Algorithms and New Trends, G. Farin, ed.,
SIAM, Philadelphia, 1987, pp. 347--365.

\bibitem{jj} 
J. Johnstone, 
{\em The sorting of points along an algebraic curve}, 
Technical Report 87-841, Ph.D. thesis, Dept. of Computer Science, 
Cornell University, Ithaca, NY, 1987.

\bibitem{keil} 
J. M. Keil, 
{\em Decomposing polygons into simpler components}, 
Technical Report 163/83, Ph.D. thesis, Dept. of Computer Science, 
Univ. of Toronto, 1983.
% April 1983

\bibitem{kirk} 
D. Kirkpatrick, 
{\em Optimal search in planar subdivisions}, 
this Journal, 12(1983), pp. 28--35.

\bibitem{law}
J. D. Lawrence, 
{\em A Catalog of Special Plane Curves}, 
Dover, New York, 1972.

% \bibitem{levi} Levin, J. (1979):
% {\em Mathematical Models for Determining the Intersections of Quadric Surfaces},
% Computer Graphics and Image Processing, 11, 73-87.

\bibitem{NS}
W. M. Newman and R. F. Sproull, 
{\em Principles of Interactive Computer Graphics},
McGraw-Hill, New York, 1979.

% \bibitem{ock} Ocken, S., Schwartz, S., and Sharir, M. (1986):
% {\em Precise Implementation of CAD Primitives Using Rational Parameterization 
% of Standard Surfaces}, in {\em Planning, Geometry, and Complexity of Robot
% Motion}, Schwartz, Sharir, and Hopcroft, eds., 245-266.

\bibitem{O} 
J. O'Rourke, 
{\em The complexity of computing minimum convex covers for polygons},
Proc. 20th Annual Allerton Conf. on Comm. Control and Comput., 1982, pp. 75--84.

\bibitem{PS} 
F. Preparata and M. Shamos, 
{\em Computational Geometry: An Introduction},
Springer-Verlag, New York, 1985.

\bibitem{S} 
J. R. Sack,
{\em An $O(n \log n)$ algorithm for decomposing simple rectilinear polygons
into convex quadrilaterals}.
Proc. 20th Annual Allerton Conf. on Comm. Control and Comput., 1982, pp. 64--74.

\bibitem{T} 
R. E. Tarjan and C. J. Van Wyk,
{\em An $O(n \log \log n)$-time algorithm for triangulating a simple polygon},
this Journal, 17(1988), pp. 143--178.

\bibitem{tm} 
S. B. Tor and A. E. Middleditch,
{\em Convex decomposition of simple polygons},
ACM Trans. on Graphics, 3(1984), pp. 244--265.

\bibitem{walker} 
R. J. Walker,
{\em Algebraic Curves},
Springer-Verlag, New York, 1950.

\end{thebibliography}
%
\end{document}
FIGURES:
\figg{2}{added endpoints at a sing: RB', B'B'',B''S}{.1in}  % 2nd
	For example, the two segments \arc{RS} and \arc{ST} in Figure~\ref{2}
	will be replaced by \arc{RB'}, \arc{B'B''}, and \arc{B''S}.
\figg{2.13}{Figure 2.13, p. 38}{.1in}                       % 3rd
\figg{3.2}{Figure 3.2, p. 56}{.1in} % not strictly necessary, 5th
\figg{3.4}{Figure 3.4, p. 58 with an example of Q3 
        which P does *not* face}{.1in} % again not necessary, 6th
\figg{3.5}{Figure 3.5, p. 59}{.1in}                         % 7th
\figg{3.12}{Figure 3.12, p. 69}{.1in}                       % 8th
\figg{3.13}{Figure 3.13, p. 70}{.1in}                       % 9th
