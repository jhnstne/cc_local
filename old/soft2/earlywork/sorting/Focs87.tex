\documentstyle{article} \def\baselinestretch{1.75}
\textwidth 421pt %default is 360?
\oddsidemargin 25pt % default is 41.4?
\evensidemargin 25pt
\topmargin 5pt %23pt
\textheight 598pt %580pt
%
\newcommand{\tab}{\hspace*{.2in}}
\newcommand{\arc}[1]{\mbox{$\stackrel{\frown}{#1}$}}
\newcommand{\prooof}{{\bf Proof}\quad}
\newcommand{\blob}{\mbox{\rule[-1.5pt]{5pt}{10.5pt}}}
\newcommand{\qed}{\quad\blob}
\newcommand{\seg}[1]{\mbox{$\overline{#1}$}}
\newcommand{\hence}{\\ \mbox{$.\raisebox{1.5ex}{.}.$}\ }
\newtheorem{theorem}{Theorem}
\newtheorem{definition}{Definition}
\newtheorem{corollary}{Corollary}
\newenvironment{proof}{\prooof}{\qed}
%
\begin{document}
\title{Sorting Along an Algebraic Curve}
\author{$\begin{array}{lcl}
\mbox{Chanderjit Bajaj\thanks{Supported in part by NSF Grant DCI 85-21356.}} & &
\mbox{John K. Johnstone\thanks{Supported by an NSERC 1967 Graduate
Fellowship and an Imperial Esso Graduate Fellowship.}} \\
\mbox{Purdue University} & \mbox{ \ \ \ \ \ \ } & \mbox{Cornell University} \\
\mbox{W. Lafayette, IN\ \  47907} & & \mbox{Ithaca, NY\ \ 14853} \end{array}$}
\date{}
\maketitle
%
\section{Introduction}
%
\subsection{The Problem}

\tab The area of geometric modeling is concerned with the creation of 
computationally-efficient models 
of solid physical objects to facilitate their design, assembly, and analysis.
Geometric models of physical objects are needed in many disciplines, including robotics,
computer vision, computer-aided design/computer-aided 
manufacturing, and graphics.
In a geometric modeling system, a solid such as a robot hand or a coffee cup is modeled by a collection
of points, curves, and surfaces.
The sorting of points on a curve, as evidenced by its numerous applications, is a basic tool for the
manipulation of geometric models.

Curve sorting has a natural definition.
Let C be an irreducible, algebraic, plane curve (i.e., 
a plane curve described by an irreducible polynomial \mbox{$f(x,y)=0$}).
If $S \subset C$ and \arc{AB}\ is a segment of C,
then to sort the points of S from A to B along \arc{AB} means to put 
them into the order that they would be encountered in travelling 
continuously from A to B along \arc{AB}.
Any of the points that do not lie on \arc{AB}\ are ignored.

There is no loss of generality in only considering plane curves,
since a sort along a space curve
(i.e., a curve that does not lie in a plane)
can be mapped into a sort along a plane curve (viz., 
a well-chosen projection of the space curve) \cite{john}.
%
\subsection{Main Results}
%
\tab We present a new sorting method which can sort
$m$ points of an arbitrary algebraic curve of degree $n$ in O($m\ S_{n}$) time,
with O$(n^{6} +n^5\ log\ n\ S_{n}$) preprocessing, where $S_{n}$ is the time required to obtain the
real roots of a 
polynomial equation of degree $n$. For obtaining real roots, the fast procedure
of Jenkins and Traub may be used \cite{jen}.
For a nonsingular curve, the preprocessing can be reduced to 
O($n^5\ log\ n\ S_{n}$).
The sorting of $m$ points of a convex segment of a curve can be
done in O($m$) time, without any preprocessing.
%
\subsection{Relevance to Past Research}
\label{sec-1.3}

\tab A natural way to sort points along an algebraic curve is to use a rational parameterization
of the curve (i.e., a parameterization ( $x(t)$ , $y(t)$ ) such that both
$x(t)$ and $y(t)$ can be expressed as the quotient of two polynomials in $t$).
The parameter values $t_{i}$ of the points $(x_{i},y_{i})$ are computed and sorted by increasing $t_{i}$ values. For a curve segment sort, the points
that occur before the start point or after the end point of the segment
are discarded.
There are general algorithms for the 
automatic parameterization of rational curves (curves with a rational
parameterization) \cite{abba3}.
Various efficient methods have also been given to obtain rational
parameterizations for special low degree algebraic curves \cite{abba1,abba2,levi,ock}.
However, only a subset of algebraic curves are rational \cite{wa}.
Even if the curve is rational, the parameterization method will be slow
if the degree of the parameterization is high, since the computation
of the parameter values of the points will be expensive.
Hence, sorting along algebraic curves by conventional methods has been limited and slow.

There is another sorting method which uses the brute-force technique of
tracing along the curve with Newton's method \cite{bhh}.
The implementation of this method, though robust, is inherently very slow,
and it also has the undesirable property
that its complexity depends upon the length of the segment that is being
sorted rather than upon the number of points in the sort  \cite{john}.

There is no serious study of sorting in the literature.
This can be explained by the fact that,
until recently, almost all of the curves in solid models were
linear or quadratic.
Nontrivial sorting problems arise only with curves of degree 
three and more.

Our sorting method is applicable to an arbitrary degree algebraic curve 
with any possible point singularities.
This new method of sorting is especially valuable for its efficient sorting 
along 
$(i)$ algebraic curves for which a rational parameterization cannot be efficiently obtained, 
$(ii)$ algebraic curves with a rational parameterization of high degree, 
and $(iii)$ algebraic curves which do not possess a rational
parameterization.
%
\subsection{Applications}
\tab Since geometric models consist of point vertices, algebraic-curve edge segments,
and algebraic-surface patches, numerous applications of algebraic curve sorting
arise from the various operations performed on the geometric model.
We consider the following problems:
$(1)$~given a set of points S on a curve C, determine which points of S lie on
an edge of C
$(2)$~compute the intersection of two edges
$(3)$~compute the intersection of two space curve segments C and D on the same surface
$(4)$~determine a bounding box for an edge
$(5)$~determine whether a point lies within a plane piecewise algebraic patch
$(6)$~determine whether a point lies within a convex algebraic surface patch, and
$(7)$~compute the intersection of two solid models.
Each of these operations makes essential use of sorting along an algebraic curve
segment.
%
\section{Convex Segmentation}

\tab One of the fundamental steps of our sorting method is a decomposition of the 
curve into convex segments.
This decomposition allows us to take advantage of the simplicity of sorting
points on a convex segment (Theorem~\ref{thm-1}).
%
A segment \arc{PQ}\ of a plane curve is {\bf convex} iff
no line has more than 
two intersections with \arc{PQ}.
%
\begin{theorem}[{\cite{john}}]
\label{thm-1}
Let $p_{1},\ldots,p_{n}$ be points of a convex segment \arc{AB}, 
and let H be the convex hull of A, B, $p_{1},\ldots,p_{n}$.
Then the vertices of H are A, B, $p_{1},\ldots,p_{n}$.
Moreover, the order of the vertices on the boundary of H is equivalent
to the order of the points on \arc{AB}.
\end{theorem}
%

The decomposition of a curve into convex segments is achieved by
the tangents at certain special points of the curve: the 
singularities and flexes.
A {\bf singularity} of the curve $f(x,y)=0$ is a point P of the curve
such that $f_{x}(P) = f_{y}(P) = 0$.
It is a point where the curve crosses itself or changes direction sharply.
A {\bf flex} (or point of inflection) is a nonsingular point P 
whose tangent has three or more intersections with the curve at P.
Equivalently, it is a point of zero curvature.
The tangents at the singularities and flexes of a curve 
subdivide the plane of the curve into several cells and split the curve 
into several segments.  
Theorem~\ref{thm-2}\ establishes that each of these segments is convex.
%
\begin{theorem}[{\cite{john}}]
\label{thm-2}
Let T be the set of tangents of the singularities and floxes of a curve F,
and let \arc{PQ}\ be a nonconvex segment of F.
Then some tangent of T crosses \arc{PQ}. 
\end{theorem}
\begin{proof}
(A sketch.) Assume w.l.o.g. that \arc{PQ}\ does
\marginpar{(*)}
not contain a singularity or flex.
Since \arc{PQ}\ is not convex, we can find a line that crosses it at
three distinct points: $x_{1}, x_{2}, x_{3}$.
By (*), \arc{x_{1}x_{3}}\ must be a spiral.
Consider the region R bounded by \arc{x_{1}x_{3}}\ and \seg{x_{1}x_{3}}.
It is sufficient to show that R contains a flex or singularity S,
since S's tangent must cross the \arc{x_{1}x_{3}}\ boundary of R.
It can be shown that the curve must enter R as it leaves $x_{1}$ and
must eventually leave R via \seg{x_{1}x_{3}}.
However, in order to do this, the curve must cross itself or change its
curvature inside of R.
\end{proof}

The decomposition of the curve into convex segments is a preprocessing step.
It is necessary to compute the singularities and flexes of the curve, 
as well as their tangents.
The singularities of a curve $f(x,y)=0$ are the solution set of the system
$\{f_{x}=0,f_{y}=0,f=0\}$,
while the flexes of a curve are the intersections of the
Hessian of the curve (the determinant of the matrix of double derivatives of 
the curve's equation) with the curve \cite{wa}.
The tangents of a singularity of the curve $f=0$ 
can be found by translating the curve to
the origin and finding the components of the order form (the polynomial
consisting of the terms of lowest degree in the translated f) \cite{wa}.
Finally,  after the curve has been translated to projective space (normal affine
space with an added line at infinity) by homogenizing its equation to
$f(x,y,z)=0$ (where $z$ is the homogenizing variable),
the tangent of a flex P
is $f_{x}(P) + f_{y}(P) + f_{z}(P) = 0$ \cite{wa}.

The tangents of the singularities and flexes subdivide the plane into cells,
some open and some closed.
This cellular decomposition is called a {\bf cell partition}, 
while the tangents
are called {\bf walls} 
and the intersections of the curve with a wall are called
{\bf wallpoints}.
The wallpoints are the endpoints of the convex segments.
%
\section{Sorting}

\tab After the curve has been decomposed into convex segments,
points can be sorted from A to B by traversing the curve from A to B
by convex segments, stepping from wallpoint to wallpoint.
As each convex segment of \arc{AB}\ is encountered, 
the points which lie on it are found and sorted, and this subsort is appended
to the end of a global sort that is being accumulated.

The traversal of the curve by convex segments is especially challenging in 
the neighbourhood of a singularity.
In particular, it can be ambiguous which branch of the curve should be
followed from a singular wallpoint.
This problem is resolved by finding,
for each branch that passes through a singularity, a pair of points,
one on either side of the singularity.
These two points serve to guide the sort through the singularity along
the proper branch.
Section~\ref{ambiguity} discusses this problem.

A crucial step in the traversal of the curve by convex segments is the
determination of the next convex segment.
If \arc{VW} is a convex segment of the curve, then we say that V and W
are {\bf partners}.
A wallpoint is the endpoint of two convex segments, and so it usually has
two partners. 
Our problem reduces to finding the partners of a wallpoint.
That is, given a convex segment \arc{PQ}, the convex segment that follows
\arc{PQ} is \arc{QR}, where R is the other partner of Q.

The other main step in the traversal of the curve is the computation of the 
points that lie on a given convex segment.
It turns out that this problem can be solved with the theory that must
be developed for the partner problem, so we only consider the latter.

Suppose that Q lies on the boundary of cells C and D.
We wish to find the wallpoint R in cell D that is Q's partner.
Cell D will usually contain only one convex segment, in which case it is 
obvious what R is.
We give an indication of the theory necessary to deal with cells which
have many convex segments by solving a special case of the partner
problem. 
(See  \cite[Chapter 3]{john} for the entire theory.)

\begin{definition}
If P is not a singularity or a flex, 
then {\bf the inside of P's tangent} is the halfplane that contains 
all of the curve in the neighbourhood of x.
The inside includes the tangent, while the {\bf strict inside} does not.
P {\bf faces} Q iff Q lies on the inside of P's tangent.  
Finally, \#\{S\}\ is the number of elements in the set S.
\end{definition}

\begin{theorem}[{\cite{john}}]
\label{thm-assocwpts}
Let F be a plane curve that has been split into convex segments.
Let \arc{QR}\ be a convex segment of the cell C such that 
Q is not a singularity or a flex and R does not lie on the same wall as Q.\\
Let $S^{Q}$ = \{ wallpoints X of C $\mid$
\begin{quote}
\begin{enumerate}
\item X lies on the strict inside of Q's tangent with respect to C; and
\item \mbox{$\forall\ z \in $ \seg{QX}, $\#\{P \in$ \seg{Qz}
$\cap$ F: P faces Q \} $\leq \#\{P \in $ \seg{Qz}\ $\cap$ F: P faces X \}  \}}
\end{enumerate}
\end{quote}
Let $Q_{2}$ be the other intersection of Q's 
tangent with the boundary of C.\footnote{If 
C is an open cell, then a cell wall must be placed across its opening to close it up.}
Sort the points of $S^{Q}$ along the boundary of the cell
from Q to $Q_{2}$.
Then $R = S_{p}$.
Thus, $S^{Q}$ can be used to find the partner of Q.
\end{theorem}
%
\section{Sorting Around Singularities}
\label{ambiguity}
%.PP
%.PP

\tab An algebraic curve may possess a variety of different complicated singularities.
At singular points, the curve may have an abrupt change of normal direction (cusps),
multiple self-intersecting branches (nodes), or self-tangent branches (tacnodes).
For singular algebraic curves, we need to ensure that 
correct branch connectivity is considered at the singular points while sorting
points through the singularity.
This information can be achieved by a local analysis of each singularity of the curve. Analysis
of a singularity may be achieved by use of
the affine quadratic transformations of Noether (1870) or the projective
quadratic transformations of Cremona (1860) \cite{wa}.
We use the affine quadratic transformations,
for then the transformation of the curve $f$ is local, and 
each singularity can be treated in isolation.
Using this analysis of the singularity, we give a procedure to obtain a pair
of points on either side of the singularity, for each branch of the curve passing through the singularity.
But we must first introduce the theory of quadratic transformations.

Let $f(x,y)=0$ be a planar algebraic curve and let $R$ be a singularity of 
the curve.
Without restriction, we can assume that $R$ is the origin, since any point on the curve 
can be translated to the origin by a linear transformation.
The {\bf affine quadratic transformation} $q_{1}$ from the x-y plane 
to the $x_{1}$-$y_{1}$ plane is given by $x_{1} =x$ and $y_{1} = \frac{y}{x}$.
The inverse transformation is $x = x_{1}$ and $y = x_{1}y_{1}$. 
The quadratic transformation
maps the origin to the entire line $x_{1} = 0$, but it is
one-to-one for all points $(x,y)$ with $x \neq 0$. 
The line $y = mx$ is mapped to the horizontal line $y_{1}= m$.
Thus, a quadratic transformation maps distinct tangent directions of the
various branches of $f$ at the singular origin to different
points on the {\bf exceptional line} $x_{1} =0$.
The intersections of the transformed branches with the exceptional
line correspond to the transformed points of the singularity at the origin.
If an intersection point on the exceptional line is singular, 
then the procedure is applied recursively.
Hence, under quadratic transformations, the various branches of the 
curve in the neighborhood of the singularity get transformed to separate
branches.
Once a branch is isolated, it is simple to find two points of it
on either side of the singularity, since there are no other branches
present to cause confusion. 

To summarize, for each singular point of the curve, we translate 
the singularity to the origin
and apply a series of quadratic transformations until the singularity is 
transformed into a set of nonsingular points.
A branch of the transformed curve intersects the exceptional line
in a simple (nonsingular) point.
We compute two points on the branch of the curve on either side of 
the exceptional line $x_{1} ~=~ 0$, and
apply the inverse transformations to these points
to yield the two points on the corresponding branch of the original curve. 
This process is repeated for each branch of the curve.
The pair of points on each branch clarify the branch connectivity at the 
singularity  and allow a robust traversal of the curve by convex segments.
%
%        - blow up the curve at the singularity\\
%	- use the desingularized curve to guide you
%	  safely through the singularity\\
%	- see the new (and incomplete) section 2.5 
%	  of my thesis\\
%	- we essentially slice the singularities out
%	  of the curve by creating two points on
%	  either side of the singularity, for each branch
%	  that passes through the singularity
%	  (these pts are found by crawling on the 
%	   desingularized curve)\\
%	- when you are jumping to the singularity during
%          a sort, you actually jump to the point on one
%	  side of the singularity and then continue jumping
%	  from the point on the other side\\
%        - this section will be hard to present 
%	  clearly and *concisely*
%
\section{Complexity}
   
\begin{theorem}[{\cite{john}}]
m points on an arbitrary algebraic curve of degree n can be sorted by the convex-segment
method in O($m\ S_{n}$) time with O($n^6 + n^5\ log\ n\ S_{n}$) preprocessing, where $S_{n}$ is the time
required to find the real roots of a polynomial equation of degree n.
\end{theorem}
\begin{proof}
(A sketch.)
The computation of the flexes and singularities of a curve takes 
O($n^5\ log\ n\ S_{n}$) time, using the fact that resultants 
can be computed by the method of Collins \cite{col} in O($n^5\ log\ n$) time.
(Resultants are an effective way of solving systems of equations.)

A curve of degree n has O($n^{2}$) flexes and singularities
\cite{wa}.
There are also O($n^{2}$) wallpoints at the 
singularities and flexes.

Consider the time that is required to compute the wallpoints
at the singularities by the procedure of section~\ref{ambiguity}.
The resolution of a singularity requires from 1 to 
O($n^2$) quadratic transformations.
Each quadratic transformation entails an algebraic simplification, which 
with O($n^2$) possible monomials in $f$ will take O($n^2$) time.
There is also the time spent in translating the singularity 
to the origin (a linear substitution $x_{t} = x - a , y_{t} = y - a$), which 
again entails an algebraic simplification with an overall cost of O($n^4$).
Counting the time for intersections with the 
exceptional line, which requires an additional $S_{n}$ time,
the overall time bound for computing all O($n^2$) wallpoints of O($n^2$)
singularities is O($n^6 +n^2 S_{n}$).

It can be shown that there are O($n^{3}$) wallpoints that are not singularities
or flexes.
Therefore, the convex decomposition has O($n^{3}$) wallpoints and
O($n^{3}$) convex segments.

For each point that is being sorted, we must determine which convex
segment it lies on.
In the worst case, this takes O($S_{n}$) time per 
point (Theorem~\ref{thm-assocwpts}).
(However, for most points, this will take O(1) time.)
Finally, the sorting of p points by a variant of Theorem~\ref{thm-1} 
takes O($p$) time.
(The variant does not actually create the convex hull: it simply sorts
the angles that the points make with a central point.)
\hence In the worst case, the convex-segment method requires 
O($m\ S_{n}$) time, with O($n^6 +n^5\ log\ n\ S_{n}$) preprocessing, to step over O($n^{3}$) convex segments
and sort the points on these convex segments.
\end{proof}

\begin{corollary}
m points on a segment of a nonsingular curve of degree n can be sorted 
in O($m\ S_{n}$) time with O($n^5\ log\ n\ S_{n}$) preprocessing.
\end{corollary}

\begin{corollary}
m points on a convex curve segment of degree n can be sorted 
in O($m$) time without preprocessing.
\end{corollary}
%
\section{Applications}

\tab We offer a number of applications of curve sorting.
Let edge E be a segment of an algebraic curve C. 
E is defined by the implicit equation $f(x,y) = 0$ of C and
two endpoints $V_{a}$ and $V_{b}$.

(1) Given a set of points S on C, determine the points of S
which lie on E.\\
%
This is a fundamental problem in geometric modeling which arises
from the implicit representation of the edge E.
One solution is to trace the edge E from $V_{a}$ to $V_{b}$,
marking the points of $S$ which are encountered in the trace \cite{bhh}.
However, this method is slow and has other 
disadvantages (Section~\ref{sec-1.3}).
A more efficient solution is to sort $\{V_a,V_b\} \cup S$
along C. The points between $V_a$ and $V_b$ in the sorted sequence lie on E.

(2) Compute the intersection of two edges C and D.
	  
(3) Compute the intersection of two space curve segments C and D
    on the same surface F.

(4)  Determine a bounding box for an edge $E$.\\
The solutions of (2), (3)  and (4) reduce to applications of the 
method of problem (1), and thus require sorting.
%
%The extreme points of an edge are either local maxima/minima of the curve
%or the endpoints of the edge. The solution of problem (1) is applied to
%Therefore, a bounding box can be created by (i) finding all of the local
%maxima/minima of the curve (ii) finding the subset of these points that lie
%on the edge, using the solution of problem (1), and
%(iii) choosing the points with the 
%
%
%          use df/dx, df/dy, and endpoints of edge to determine
%	  local maxima/minima\\
%	  sort to determine which of these lie on edge\\
%         create bounding box for edge\\

(5) Determine whether a point lies within a planar piecewise algebraic patch\\
%
A planar piecewise algebraic patch is given by a closed boundary 
consisting of a simply connected loop of algebraic curve segments.
The problem of determining whether a point $Q$ lies within the closed boundary
reduces to the problem of sorting by the following mapping.
Consider the straight line $L$ defined by a vertex $V$ on the boundary and the
point $Q$. Its intersection with the algebraic curve segments of the patch boundary yields a set of points $S$, which are determined by applications of
the method of problem (2) and hence algebraic curve sorting.
The points of $S$, the point $Q$, and the point $V$ are now sorted on the
line $L$. By applying the Jordan curve theorem, the points on the line
can be grouped into pairs 
and inside and outside intervals can be determined.
$Q$ lies within the patch iff it lies on an inside interval.

(6) Determine whether a point lies within a convex algebraic surface patch\\
%
The solution is a generalization of the method of (5), with repeated 
applications of (3).

(6) Compute the intersection of two solid models.\\
%
An important step of the algorithm for computing the intersection of two
solid models is to find the segments of an edge of one model which lie in
the intersection.
This is done by finding and sorting the intersections of this edge with a 
face of the other model. 
The segments of the edge between the $i^{th}$ and $i+1^{st}$ intersections,
for $i$ odd, are contained in the intersection of the models.
	  
%	  
\section{Conclusions}

\tab We have developed a new method of sorting points on a curve
which is superior to the conventional methods of sorting.
In the process, we have illustrated how an algebraic curve can 
be decomposed into convex segments, and how to resolve the
ambiguity of sorting through a singularity.
We have also established the importance of curve sorting
to geometric modeling by outlining many of its applications.

Our treatment of algebraic curve sorting has highlighted the need for 
efficient computational geometry algorithms for the increased
geometric coverage of curves and surfaces.

\begin{thebibliography}{The longest is long}
%
\bibitem{abba1} Abhyankar, S., and Bajaj, C. (1987):
{\em Automatic Rational Parameterization of Curves and Surfaces I:
Conics and Conicoids}, Computer Aided Design
19:1, 11-14.
% Jan. 1987
%
\bibitem{abba2} Abhyankar, S., and Bajaj, C. (1986):
{\em Automatic Rational Parameterization of Curves and Surfaces II:
Cubics and Cubicoids},
Technical Report 592, Dept. of Computer Science, 
Purdue University.
% (April 1986).
To appear in Computer Aided Design.
% 
\bibitem{abba3} Abhyankar, S., and Bajaj, C. (1986):
{\em Automatic Parameterization of Rational Curves and Surfaces III: 
Algebraic Plane Curves},
Technical Report 619, Dept. of Computer Science, 
Purdue University.
% (August 1986).
%
\bibitem{bhh} Bajaj, C., Hoffmann, C., and Hopcroft, J. (1987):
{\em Tracing Algebraic Curves I: Planar Curves}, Manuscript.
%
\bibitem{col} Collins, G. (1971):
{\em The Calculation of Multivariate Polynomial Resultants},
Journal of the ACM, 18:4, 515-532.
%
\bibitem{jen} Jenkins, M., and Traub, J. (1975):
{\em Algorithm 493}, ACM Transactions of Mathematical Software, 1:2.
%
\bibitem{john} Johnstone, J. (1987):
{\em The Sorting of Points Along an Algebraic Curve},
Ph.D. thesis, Dept. of Computer Science, Cornell University,
Ithaca, NY. In preparation.  
%
\bibitem{levi} Levin, J. (1979):
{\em Mathematical Models for Determining the Intersections of Quadric Surfaces},
Computer Graphics and Image Processing, 11, 73-87.
%
\bibitem{mo} Mortenson, M. (1985):
{\em Geometric Modeling},
New York: John Wiley.
%
\bibitem{ock} Ocken, S., Schwartz, S., and Sharir, M. (1986):
{\em Precise Implementation of CAD Primitives Using Rational Parameterization 
of Standard Surfaces}, in {\em Planning, Geometry, and Complexity of Robot
Motion}, Schwartz, Sharir, and Hopcroft, eds., 245-266.
%
\bibitem{wa} Walker, Robert J. (1978):
{\em Algebraic Curves},
New York: Springer Verlag.
\end{thebibliography}
\end{document}
