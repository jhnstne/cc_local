\documentstyle[12pt]{article} 
%
\input{macros}
%
% Page format
%
\DoubleSpace
\setlength{\oddsidemargin}{0pt}
\setlength{\evensidemargin}{0pt}
\setlength{\headsep}{0pt}
\setlength{\topmargin}{0pt}
\setlength{\textheight}{8.75in}
\setlength{\textwidth}{6.5in}
%
%
\title{Convex decomposition and point location for algebraic curves}
\author{$\begin{array}{lcl}
\mbox{Chanderjit Bajaj\thanks{Supported in part by NSF Grants 85-21356 and
CCR 86-19817.}} & &
\mbox{John K. Johnstone\thanks{Supported by an NSERC 1967 Graduate
Fellowship and an Imperial Esso Graduate Fellowship.}} \\
\mbox{Purdue University} & \mbox{ and\ \ \ } & \mbox{The Johns Hopkins University} \\
\mbox{W. Lafayette, IN\ \  47907} & & \mbox{Baltimore, MD\ \ 21218} \end{array}$}
\begin{document}
%
\maketitle
%
\begin{summary}{Summary of Results}
In this paper, we present a method for decomposing an arbitrary plane
algebraic
curve into convex segments. This decomposition is then used to solve
the problem of point location on plane algebraic curves. 
\end{summary}
\section{Introduction}
%
The decomposition of an object into simple objects is an important theme
in computational geometry.
It finds application in geometric modeling, robotics,
%(e.g., motion planning \cite{lozano-perez grown-obstacles}), 
vision,
% (e.g., shape recognition), 
and graphics.
% (e.g., interference detection)
Decomposition proves to be particularly
useful in divide-and-conquer algorithms, 
since simple objects are easily conquered.
There has been a good deal of work on the decomposition of 
(simple, multiply connected, rectilinear) polygons 
into simple components
(triangles \cite{garey,hertel,incerpi,tarjan}, 
quadrilaterals \cite{sack}, 
trapezoids \cite{asano}, 
convex polygons  \cite{cd,tm},
star-shaped polygons \cite{avis}), 
sometimes with added criteria (minimum decomposition \cite{cd,keil}, 
minimum covering \cite{orourke}, 
decomposition with no Steiner points \cite{keil}).
However, all of this work has been in the polygonal (or at best polyhedral)
world.
This paper considers a decomposition of  a {\em plane algebraic curve} 
(of arbitrary degree) into convex segments, thus extending the idea of
decomposition from polygons to curved objects.

Another problem that has received considerable attention in computational
geometry is planar point location (see \cite{prepsham85} for a good
survey). 
Here we consider an extension of this problem, again to the curved domain.
Given an edge E of an algebraic curve C,
defined by the implicit equation $f(x,y) = 0$ of C together with
two endpoints $V_{a}$ and $V_{b}$, and
given a set of points S on C, we wish to determine the points of S
which lie on E.
This is a fundamental problem in geometric modeling whose complexity stems
from the implicit representation of the edge E.
One solution is to trace the edge E from $V_{a}$ to $V_{b}$,
marking the points of $S$ which are encountered in the trace \cite{bhhl}.
However, this method is slow and has other 
disadvantages.
We present a more efficient solution which uses the curve decomposition as
a preprocessing step.

Curve decomposition and point location on curve segments are both
operations of a fundamental nature and prove useful in the solution of many 
problems, such as the sorting of points along an algebraic curve \cite{jj},
the intersection of algebraic curve segments \cite{gj87}, and the convex hull 
and convex decomposition of objects bounded by plane algebraic curves 
\cite{bajkim87a,bajkim87b}.
Note that these methods can also extend to space curves defined as the
intersection of two algebraic surfaces \cite{abba4,jj}.

In the next section, we describe our convex decomposition method.
In subsequent sections, this decomposition is used to develop a new
method for point location, the complexity of convex decomposition is analyzed,
and the usefulness of a convex decomposition is motivated by an example.

\section{Convex Decomposition}

A segment is {\em convex} if no line has more than two intersections with it.
We wish to decompose an arbitrary plane algebraic curve into convex segments.
Note that all curves of degree one (lines) and two (conics) are already convex,
by Bezout's Theorem \cite{walker}.
It is implicit in this paper that all curves being discussed are algebraic,
irreducible, and planar.

The singularities and points of inflection (or flexes) of a
curve are instrumental in its convex decomposition.
A {\em singularity} of the curve $f(x,y)=0$ is a point P of the curve
such that $f_{x}(P) = f_{y}(P) = 0$.
It is a point where the curve crosses itself or changes direction sharply.
A {\em flex} (or point of inflection) is a nonsingular point P 
whose tangent has three or more intersections with the curve at P.
Equivalently, it is a point of zero curvature.
These points, which are fundamental in algebraic geometry, form a skeleton
of the curve, and can be used in many useful ways.
(For example, singularities can be used for curve parameterization 
\cite{abba3}).
Their use in convex decomposition underlines their importance to
computational geometry of higher degrees.
The flexes and singularities of an algebraic curve can be thought of as
the analogues of the vertices of a polygon.
They will form many of the endpoints in the convex decomposition.
%
\begin{theorem}
\label{deke}
The tangents of the singularities and flexes of an algebraic curve slice
the curve into convex segments.\footnote{Strictly speaking,
	the tangents of flexes of {\em even} order are not necessary.
	See \cite{jj} for details.}
That is, if \arc{PQ} is a nonconvex segment of the curve, then some tangent of 
a singularity or flex intersects \arc{PQ} (Figure~\ref{2.8}).
\end{theorem}
\Heading{Sketch of Proof:}
Assume w.l.o.g. that \arc{PQ}\ does
\marginpar{(*)}
not contain a singularity or flex.
Since \arc{PQ}\ is not convex, we can find a line that crosses it at
three distinct points: $x_{1}, x_{2}, x_{3}$.
By (*), \arc{x_{1}x_{3}}\ must be a spiral.
Consider the region R bounded by \arc{x_{1}x_{3}}\ and \seg{x_{1}x_{3}}.
It is sufficient to show that R contains a flex or singularity S,
since then S's tangent must cross the \arc{x_{1}x_{3}}\ boundary of R.
It can be shown that the curve must enter R as it leaves $x_{1}$ and
must eventually leave R via \seg{x_{1}x_{3}}.
However, in order to do this, the curve must cross itself or change its
curvature inside of R.
\QED

It is necessary to compute the singularities and flexes of the curve, 
as well as their tangents.
We outline succinctly how this is done.
The singularities of a curve $f(x,y)=0$ are the solution set of the system
$\{f_{x}=0,f_{y}=0,f=0\}$,
while the flexes of a curve are the intersections of the
Hessian of the curve (the determinant of the matrix of double derivatives of 
the curve's equation) with the curve \cite{walker}.
The tangents of a singularity of the curve $f=0$ 
can be found by translating the singularity to
the origin. The factors of the order form (the polynomial
consisting of the terms of lowest degree in the translated $f$) are the equations of the tangents to the singularity \cite{walker}.
The tangent at a flex is computed as follows.
The curve is placed into projective space (normal affine
space with an added line at infinity) by homogenizing its equation to
$f(x,y,z)=0$ (where $z$ is the homogenizing variable).
Then the tangent 
of a flex P is $f_{x}(P) + f_{y}(P) + f_{z}(P) = 0$ \cite{walker}.

Theorem~\ref{deke} does not yet solve our problem, because it only yields a
confused collection of endpoints of convex segments, not a collection of 
convex segments.
(The endpoints of the convex segments are the singularities, the flexes, and
the other points\footnote{In the spirit 
	of polygonal decomposition algorithms and our description 
	of the singularities and flexes as the vertices of the curve,
	these `other points of intersection' play the role of Steiner points.}
of intersection of the tangents with the curve.)
The more challenging step of pairing up the endpoints must yet be described.
Before we can attack this pairing problem, we must pause to
refine our collection of convex segments.

The tangents at the singularities and flexes of a curve form an
arrangement of lines \cite{edelsbrunner}, subdividing the plane
of the curve into several cells, which we collectively call a 
{\em cell partition}.
Some of the cells contain a convex segment of the curve, 
others contain several convex segments, and the rest
do not contain any of the curve (Figure~\ref{2.8}).
Two points of the curve are said to be {\em partners} if they 
define a convex segment of the decomposition.
\ifFull
For each cell of the partition, we record the tangents that bound it
and the convex segment endpoints that lie on its boundary.
\fi

The present collection of convex segments may have to be refined at 
singularities.
Consider a convex segment whose two endpoints are the same 
point (i.e., a loop), which might occur around a 
singularity (Figure~\ref{2.8}(a)).
We want to avoid this situation, since pairing will turn out to be easier if 
the two endpoints of a convex segment are different points.
Consider a singularity at which three or more convex segments meet.
This singularity has several partners and, more importantly,
there exists a cell C such that the singularity has two partners in C.
We want to avoid this situation, as well, since
it is easier to find the partner of a point in a cell if this 
partner is unique.
(There are other reasons to dislike endpoints with identical partners or 
too many partners: in the former case, 
the segment defined by two endpoints is no longer unambiguous, and in the
latter, the order of the convex segments along the curve is no longer
implicit from the endpoint pairings.)
Therefore, all convex segments with singular endpoints shall be replaced by
convex segments with nonsingular endpoints.

For each branch that passes through a singularity S, a pair of points will
be found, one on either side of the singularity.
Consider a branch B and its two added points $B',B''$.
The added points for branch B 
will receive the convex segments that enter the singularity along B.
For example, if there are two such segments \arc{RS} and \arc{ST} (as will
usually be the case), these segments will be replaced by \arc{RB'},
\arc{B'B''}, and \arc{B''S} (Figure~\ref{2}).

The first step in finding the added points is to blow up the curve at the 
singularity, by a series of quadratic transformations \cite{walker}.
This reduces the singularity to a number of simple points and, in particular,
isolates each branch of the singularity (Figure~\ref{2.13}).
Once a branch is isolated, it is simple to 
walk \cite{bhhl} along it robustly.
(It would be unclear how to walk from the singularity along the desired branch
on the original curve. This problem is resolved by walking from the image
singularity along the isolated image branch.)
Therefore, upon each image branch, two points are found by walking (a very 
short distance) in either direction from the image singularity.
Finally, these points are mapped back to the original curve
to become new endpoints, replacing the singularity.
The segment between a pair of added points is not a bonafide convex segment.
It is implicitly split into two parts by the two cells that it straddles,
and each of these halves is treated as 
a subsegment of the convex segment that it neighbours.

%The added points will be chosen very close to the 
%singularity.\footnote{In particular, certainly 
%	close enough that no endpoint is walked
%	over between the singularity and an added point.}
%Therefore, the segment between each pair of added points is very short.
%This segment is implicitly split into two parts:
%one part for each of the two cells that it straddles.
%The segment between added points is not a bonafide convex segment:
%each of its halves is treated as 
%a subsegment of the convex segment that it neighbours.
%In other words, the two added points on a branch 
%are treated as extensions of the singularity.

% The added points should be chosen as close as possible to the singularity,
% since the segments between the added points will be invisible to our
% point-location technique. In particular, suppose that the added point
% is at a (Euclidean) distance \alpha from the singularity.
% Then we shall have to assume that any point within the disc of radius
% \alpha (and in the proper cell) lies on the added segment.
% Note: the added segment adjacent to convex segment C should be considered
% part of C. That is, if S is the set of points on the added segment
% and T is the set of points on C, then we should act as if SuT all lie
% on the same convex segment.
% Therefore, we do not have to worry about mistakenly
% assigning points to an added segment neighbouring C rather than C.
% But we certainly want to ensure that no convex segment other than C
% lies in the disc of radius \alpha about the singularity.
% Therefore, we should choose \alpha to be small.


The final preparation for pairing concerns infinite convex segments.
An infinite convex segment has only one endpoint.
However, it turns out that the pairing process is simplified if each 
convex segment has two endpoints.
Therefore, an artificial endpoint is added to each infinite segment,
as follows.
Infinite segments lie in open cells. 
Every open cell is artificially closed by a line 
segment (Figure~\ref{3.9}).\footnote{The line segment is chosen so that it
	only intersects infinite convex segments in the cell,
	and each of those exactly once.}
The point of intersection of an infinite convex segment and the closing
line segment of its cell is now considered an (artificial) endpoint.
All pairs that include an artificial endpoint are discarded at the end of
the pairing phase.

After the above preparatory steps, 
the set of endpoints of convex segments assumes the following normal form:
\begin{itemize}
\item
	 every endpoint has exactly two partners
\item
	 every cell is a closed polygon
\end{itemize}
Notice that the normalization stage not only makes pairing easier:
it also creates a cleaner set of convex segments that better reflect the curve.
For example, when (and only when) the first normal condition is satisfied,
pairing will create a collection of convex segments with an implicit order.

Let us develop some terminology (viz., what it means for a point to 
`face' another).
Let T be a tangent of a point P that lies on the boundary of cell C.
T splits the plane into two halfplanes.
If P is a singularity or flex, then the {\em inside} of T w.r.t. C 
is the halfplane that contains C, otherwise it is the halfplane that contains
all of the curve in the neighbourhood of P (Figure~\ref{3.2}).
If P is not a flex, then P {\em faces} another point Q w.r.t. C 
if Q lies on the inside of P's tangent w.r.t. C.
If P is a flex, then the definition is more complicated,
but Figure~\ref{3.4} captures its essence.

We are ready to show how to pair up the endpoints.
Let E be an endpoint (of a convex segment).
The problem is to find the partners of E.
Suppose that E is the endpoint of convex segment S in cell C (so E lies on
the boundary of C).
E's partner must certainly lie in C (so the determination of partners 
in all single-segment cells is trivial), but we can say more.
The partner F must satisfy the following conditions (exhibited in 
Figure~\ref{3.5}), which capture the fact that 
the intersections of the curve with \seg{EF}
will pair up into couples that face each other:
\begin{enumerate}
\item 
	E and F face each other (w.r.t. C)
\item
	the curve must cross \seg{EF} at an even number of points (ignoring
	singularities)
\item 	
	the number of these crossings that face E (w.r.t. C)
	is equal to the number that face F (w.r.t. C)
\item
	for any $\alpha \in \seg{EF}$, the number of crossings in the
	interval $\seg{E\alpha}$ that face E is less than or equal to
	the number of crossings in this interval that face F
\end{enumerate}
%
These conditions will often isolate the partner.
%
\begin{example}
Consider the cell partition of Figure~\ref{3.12} and the cell containing
the convex segments \arc{\wo\wt} and \arc{W_{3}W_{4}}.
Suppose that we wish to find the partner of \wo.
$W_{3}$ violates condition (1) and $W_{4}$ violates condition (2), 
so \wt\ must be \wo's partner.
\end{example}
%
If (in the worst case) the cell contains several endpoints that satisfy
the above conditions, then the following theorem must be used
to isolate the partner.

\begin{theorem}
\label{thm-pair}
Let E be an endpoint in cell C.
Let S(E) be the set of endpoints in C that satisfy the above four conditions.
Let $S'(E)$ be the subset of S(E) consisting of endpoints that lie 
strictly inside E's tangent (w.r.t. C).
\begin{enumerate}
\item
	If $S'(E)$ is not empty, 
then sort $S'(E)$ along the boundary of the cell 
and let F be the last endpoint.
(A complete definition of this sort involves details that are beyond 
the scope of this abstract, but it reduces to the sorting of points
around the boundary of a polygon, which is very simply done.)
%
%	ASSOCIATING A POINT ON THE CELL BOUNDARY WITH EVERY PSEUDO CURVE POINT
%\footnote{Care must be
%	taken with the endpoints added near singularities, because they do 
%	not lie on the boundary.  Points on the boundary are associated 
%	with these endpoints as follows:... These boundary points are used 
%	in the sorting.}
%
%	THE TWO BOUNDARIES OF THE CELL
%\footnote{The direction of the sort is defined by 
%	distinguishing two boundaries of the cell, as follows:...}
%
\item
	If $S'(E)$ is empty, then let $S''(E)$ be the subset of S(E) consisting
of those endpoints that lie on E's wall of the cell, and
let F be the element of $S''(E)$ that is closest to E.
\end{enumerate}
F is the partner of E in cell C.
\end{theorem}
\Heading{Proof}:
Omitted due to lack of space.
\QED
%
\begin{example}
Consider the cell of Figure~\ref{3.13} and the computation of \wo's partner
in this cell.
(We need to associate a point $W_{4}'$ with $W_{4}$, since $W_{4}$ does
not lie on the cell boundary.)
$S'(\wo) = \{\wt,W_{3},W_{4}\}$. We sort \wt, $W_{3}$, and $W_{4}'$ 
along the boundary of the cell from \wo\ to X (the intersection
of \wo's tangent with the boundary), yielding $W_{3}$, $W_{4}'$, \wt.
The last element, \wt, must be \wo's partner.
Since \wt\ is an artificial endpoint, we deduce that \wo\ is actually
the endpoint of an infinite segment in this cell.
\end{example}

%WHAT IF E DOES NOT HAVE A PARTNER IN C?
%THIS NEVER HAPPENS: E EITHER LIES ON THE BOUNDARY OF EXACTLY TWO CELLS
%(IF IT IS A FLEX OR INCIDENTAL CURVE POINT) OR IT IS A PSEUDO CURVE POINT
%(ORIGINALLY A SINGULARITY) STRICTLY *INSIDE* A CELL

This completes our outline of the convex decomposition of an
algebraic curve.

\section{Point Location}
%
A decomposition is not very useful unless it is possible to locate points
in it. 
In other words, given a point P on a curve (and a convex decomposition of 
the curve), a method for deciding which convex segment P 
lies on is very important.
(Otherwise, a set of points cannot be divided into convex segments
for conquering.)
Fortunately, this problem 
is entirely analogous to finding the partners of a given endpoint, as explained
in the earlier section.
This becomes clear upon noticing that both problems are instances of 
the more general question:
``what are the two endpoints associated with a given point?''
We add that improvements can be made to the solution of the 
point location problem 
(viz., the avoidance of expensive condition-testing)
that make it even faster in practice.
(Both the convex decomposition and point location algorithms have
been implemented by J. Johnstone on a LispMachine.)

Once it is known how to locate a point on a convex segment of a curve's
convex decomposition, it is quite straightforward to solve the more
general problem of locating a point on an arbitrary segment of the curve.
Recall that every endpoint of a convex segment in our (normalized) convex
decomposition has exactly two partners.
Therefore, every convex segment has a unique predecessor and successor,
and it is very easy to order the convex segments.
Consider a segment \arc{AB} of an algebraic curve C and a point P on C.
To decide if P lies on \arc{AB}, we compute the convex segments
of C's decomposition that contain P, A, and B (say $C_{p}$, $C_{a}$, and
$C_{b}$, respectively).
Then, P lies on \arc{AB} if and only if $C_{p}$ lies in between $C_{a}$ and
$C_{b}$.
If P lies on the same convex segment as A and/or B, then the decision
requires a bit more subtlety.
(For example, if A and P lie on the same convex segment \arc{EF},
the decision is made by sorting A, P, E, and F along the curve--using 
Theorem~\ref{thm-sort} below.
If \arc{AB} leaves A towards E--which is a separate 
decision upon which we do not elaborate, then P lies on \arc{AB} 
if and only if the order is E, P, A, F.)


%We assume that A is nonsingular\footnote{If not, one can perturb it slightly
%using quadratic transformation techniques outlined above}
%and that we are given a directed tangent at A that indicates the direction of 
%\arc{AB} from A.

Before we leave point location, it is important
to explain why the above method is important despite the
existence of other methods.
In particular, a natural way to locate a set of points S on an algebraic 
curve segment is to use a rational parameterization
of the curve (i.e., a parameterization ( $x(t)$ , $y(t)$ ) such that both
$x(t)$ and $y(t)$ can be expressed as the quotient of two polynomials in $t$). 
This point location method proceeds as follows.
The endpoints of the curve segment are given by distinct parameter 
values, say, ${t_a}$ and ${t_b}$, with parameter values $t$ defining the 
curve segment, satisfying ${t_a} \leq t \leq {t_b}$ . 
The parameter values $t_{i}$ of the points $(x_{i},y_{i})$ of S 
are computed.
Then, the points that lie on the curve segment are simply those
with parameter values that occur between $t_a$ and $t_b$.
Since there are general algorithms for the 
automatic parameterization of rational curves (curves with a rational
parameterization) \cite{abba3}, and
various efficient methods for obtaining rational
parameterizations for special low degree algebraic 
curves \cite{abba1,abba2,levi,ock}, this is a reasonable method.
However, it has two disadvantages.
First, only a subset of algebraic curves have rational 
parameterizations \cite{walker}, so this is not a universal method.
Secondly, even if the curve is rational, the parameterization method will be 
slow if the degree of the parameterization is high, since the computation
of the parameter values of the points will be expensive.

%
%
%An assumption must be made about points very close to singularities:
%Note that Theorem 3.2 does not work 
%if the point lies on one of the epsilon-length convex segments 
%that contain singularities (i.e., the segments between 'pseudo curve points').
%It obviously cannot, because the two endpoints of this segment lie in
%different cells and our technique only chooses from endpoints in the
%same segment.
%For these points, we must make the following additions.
%Make sure that the added points
%are chosen extremely close to the singularity, so that they are at a very
%small $\epsilon$ from the singularity.
%Then assume that if a point P is within %$\epsilon$ of a singularity S 
%and in cell C, then it must lie on the added segment of S in C.
%
%finding (the two endpoints of) the convex segment of a given point 
%Thm 3.2 (p.78) and Example 3.3.1 (p.79)
%a method of avoiding expensive computations (p.81)
%
\section{Complexity}

\begin{theorem}
The worst case time for a convex decomposition of a plane algebraic
curve of degree $n$ is 
O($S(n^{2}) + n^{6}S(n) + n^{2}S(2^{\mbox{MAX}}n) + n^{4}2^{\mbox{MAX}}$),
where $S(n)$ is the time required to find the real roots of a univariate
polynomial equation of degree $n$, and MAX is the maximum number
of quadratic transformations that are necessary to decompose any
singularity of the curve into simple points.\footnote{MAX will usually
be 1 or 2.  For example, MAX is 1 if each singularity has distinct
tangents.}
\end{theorem}
\Heading{Sketch of Proof}:
The curve has O($n^{2}$) singularities and flexes, which 
(along with their tangents) can be computed in O($S(n^{2}) + n^{6}$) time.
O($n^{2}S(n)$) time is required to find the O($n^3$) intersections of the
tangents with the curve.
Resolving the O($n^{2}$) singularities into simple points (using quadratic 
transformations) requires O($n^{2} S(2^{\mbox{MAX}}n) + n^{4}2^{\mbox{MAX}}$)
time.
Finally, computing the partners of the O($n^{3}$) endpoints of convex segments
requires O($n^{6}S(n)$) time.
\QED

This complexity analysis is very pessimistic.
That is, the worst case time will only be reached by the most
pathological curves, while the time to decompose 
curves that arise in practice in geometric modeling
is much more reasonable.
This observation has been borne out in practice, with the testing of 
the algorithms on various curves.
\Comment{
For example, the pairing of most endpoints requires O(1) time.
Moreover, once the singularities and flexes of a curve are computed,
they are useful for many other algorithms.}

\section{An application of convex decomposition}
Once the convex decomposition of a curve is available, it is possible
to use it in divide-and-conquer algorithms, since convex segments are
often easily conquered.
We offer the example of sorting points along an algebraic curve.
The following theorem shows that sorting a convex segment is simple.

\begin{theorem}
\label{thm-sort}
Let $p_{1},\ldots,p_{n}$ be points on a convex segment \arc{AB}, 
and let H be the convex hull of A, B, $p_{1},\ldots,p_{n}$.
The order of the points on \arc{AB} is simply 
the order of the vertices on the boundary of H (from A to B).
\end{theorem}

Using this observation and the convex decomposition method, 
it is possible to derive an effective sorting method \cite{jj}.
%
\section{Conclusions}
A method for decomposing an algebraic curve into convex segments 
has been presented.
This continues the work on decomposition of a geometrical object into
simpler objects.
However, this work is unique in that it is one of the first solutions
of a computational geometry problem for curves of arbitrary degree.
We have also outlined a new method for point location on algebraic curves,
based upon the convex decomposition.
There is a need for more work on the extension of computational
geometry algorithms past polygons to curves and surfaces of higher degree.
%
\section{Acknowledgements}
This work formed part of the thesis of J. Johnstone, who acknowledges
the guidance of his advisor, John Hopcroft.
%
\begin{thebibliography}{Abhyankar 87b}
%
\bibitem{abba1} Abhyankar, S., and Bajaj, C. (1987),
``Automatic Rational Parameterization of Curves and Surfaces I:
Conics and Conicoids,'' {\em Computer Aided Design}
19:1, 11-14.
% Jan. 1987

\bibitem{abba2} Abhyankar, S., and Bajaj, C. (1987),
``Automatic Rational Parameterization of Curves and Surfaces II:
Cubics and Cubicoids,'' {\em Computer Aided Design}
19:9, 499-502.
% Nov. 1987

\bibitem{abba3} Abhyankar, S. S., \& Bajaj, C. (1986).
``Automatic Parameterization of Rational Curves and Surfaces III:
Algebraic Plane Curves,''
Tech. Rep. CSD-TR-619, Dept. of CS, Purdue Univ.
% Aug. 1986

\bibitem{abba4} Abhyankar, S., and Bajaj, C., (1987),
``Automatic Rational Parameterization of Curves and Surfaces IV: Algebraic Space Curves,''
Tech. Rept. 703, Comp. Science, Purdue University.

\bibitem{bhhl} Bajaj, C., Hoffmann, C., Hopcroft, J., Lynch, B., (1987),
Tracing Surface Intersections.
Tech. Rept. 725, Comp. Science, Purdue University. 

\bibitem{bajkim87a} Bajaj, C., and Kim, M., (1987),
Convex Decomposition of Objects Bounded by Algebraic Curves.
Tech. Rept. 677, Comp. Science, Purdue University.

\bibitem{bajkim87b}
Bajaj, C., and Kim, M., (1987),
Convex Hull of Objects Bounded by Algebraic Curves.
Tech. Rept. 697, Comp. Science, Purdue University.

\bibitem{asano} Asano, T., Asano, T., \& Imai, H. (1984).
Partitioning a polygonal region into trapezoids.
(Res. Mem. RMI84-03). Dept. Math. Eng. \& Instrumentation Physics, Univ. of 
Tokyo.
%
\bibitem{avis} Avis, D., \& Touissant, G. T. (1981).
An efficient algorithm for decomposing a polygon into star-shaped
components.
{\em Pattern Recognition}, 13, 395-398.

%\bibitem{bhh} Bajaj, C., Hoffmann, C., and Hopcroft, J.,
%``Tracing Planar Algebraic Curves,''
%Tech. Rep. CSD-TR-637, Dept. of Computer Science, Purdue Univ,
%September 1987.

\bibitem{cd} Chazelle, B., \& Dobkin, D. P. (1985).
Optimal Convex Decompositions.
In {\em Computational Geometry}, G.T.Toussaint, editor.

\bibitem{incerpi} Chazelle, B., \& Incerpi, J. (April 1984).
Triangulation and shape-complexity. 
{\em ACM Trans. on Graphics}, 3(2), 135-152.

\bibitem{edelsbrunner} Edelsbrunner, H. (1987).
Algorithms in Combinatorial Geometry.
New York: Springer-Verlag.

\bibitem{garey} Garey, M., Johnson, D. S., Preparata, 
F. P., \& Tarjan, R.E. (1978).
Triangulating a simple polygon.
{\em Info. Proc. Lett.}, 7(4), 175-80.

\bibitem{gj87} Goodrich, M. T., \& Johnstone, J. K. (1987).
Coordinated Walking: A Method for Intersecting Algebraic Curve Segments.
Manuscript.

\bibitem{hertel} Hertel, S., \& Mehlhorn, K. (1983).
Fast triangulation of simple polygons.
{\em Proc. FCT'83}, Borgholm, LNCS Springer-Verlag, 207-218.


\bibitem{jj} Johnstone, J. (June 1987).
The sorting of points along an algebraic curve.
(Tech. Rep. 87-841). Ph.D. Thesis, Cornell Univ.
%
\bibitem{keil} Keil, J. Mark. (April 1983).
Decomposing Polygons into Simpler Components.
(Tech. Rep. 163/83). Ph.D. Thesis, Univ. of Toronto.

\bibitem{levi} Levin, J. (1979):
{\em Mathematical Models for Determining the Intersections of Quadric Surfaces},
Computer Graphics and Image Processing, 11, 73-87.
%
\bibitem{ock} Ocken, S., Schwartz, S., and Sharir, M. (1986):
{\em Precise Implementation of CAD Primitives Using Rational Parameterization 
of Standard Surfaces}, in {\em Planning, Geometry, and Complexity of Robot
Motion}, Schwartz, Sharir, and Hopcroft, eds., 245-266.

\bibitem{orourke} O'Rourke, J. (1982).
The complexity of computing minimum convex covers for polygons.
{\em Proc. 20th Annual Allerton Conf. on Comm. Control and Comput.}, 75-84.

\bibitem{prepsham85} Preparata, F., and Shamos, M., (1985):
{\em Computational Geometry: An Introduction},
Springer Verlag, New York.

\bibitem{sack} Sack, J. R. (1982).
An $O(n \log n)$ algorithm for decomposing simple rectilinear polygons
into convex quadrilaterals.
{\em Proc. 20th Annual Allerton Conf. on Comm. Control, and Comput.},
64-74.
%
\bibitem{tarjan} Tarjan, R. E., \& Van Wyk, C. J. (February 1988).
An $O(n \log \log n)$-time algorithm for triangulating a simple polygon.
{\em SIAM Journal on Computing}, 17, 1.
%
\bibitem{tm} Tor, S.B. and Middleditch, A. E. (October 1984).
Convex Decomposition of Simple Polygons.
{\em ACM Trans. on Graphics}, 3(4), 244-265.
%
\bibitem{walker} Walker, R. J. (1950).
Algebraic Curves.
New York: Springer-Verlag.
\end{thebibliography}
%
\clearpage
FIGURES:
\figg{2.8}{Figure 2.8(a) and (c), p. 27}{.1in}              % 1st
\figg{2}{added endpoints at a sing: RB', B'B'',B''S}{.1in}  % 2nd
\figg{2.13}{Figure 2.13, p. 38}{.1in}                       % 3rd
\figg{3.9}{Figure 3.9, p. 64}{.1in}                         % 4th
\figg{3.2}{Figure 3.2, p. 56}{.1in} % not strictly necessary, 5th
\figg{3.4}{Figure 3.4, p. 58 with an example of Q3 
        which P does *not* face}{.1in} % again not necessary, 6th
\figg{3.5}{Figure 3.5, p. 59}{.1in}                         % 7th
\figg{3.12}{Figure 3.12, p. 69}{.1in}                       % 8th
\figg{3.13}{Figure 3.13, p. 70}{.1in}                       % 9th
\end{document}

