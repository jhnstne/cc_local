\chapter{A Comparison of the Sorting Methods}
\label{chap-results}
%COMMENT: we use our own root-finder in the implementation of the 
%parameterization method.

This chapter examines the efficiency of the convex-segment method
of sorting.
Section~1 analyzes its theoretical complexity, while
Section~2 presents some empirical results for the sorting methods.
Section~3 discusses the advantages of the 
convex-segment method.

\section{Complexity Analysis}

This section presents an analysis of the worst case complexity of 
the convex-segment method.
We include this analysis for the insight that it offers into the algorithm.
However, we must emphasize that it is often unwise to compare
geometric modeling algorithms by their worst case performance,
because worst cases can be misleadingly pessimistic and
a geometric modeler is concerned about the 
treatment of cases that arise in practice rather than the behaviour of the
algorithm on a worst case that occurs very rarely.
For example, in the worst case, Theorem~\ref{thm-parents} will have to be 
applied in order to determine the convex segment that a given sortpoint $x$
lies on, which could involve the solution of several equations of degree $n$.
Yet, the worst case arises only in those rare cases when the sortpoint $x$ 
lies in a multisegment cell and requires the expensive conditions of S($x$)
to determine its convex segment.
(The use of R($x$), as described in Section~\ref{sec3-chap3}, makes this
worst case even more unlikely.)
In all other cases, $x$'s convex segment can be found in $O(1)$ time.
An expected case analysis would be preferable, however it is difficult to
formalize the notion of an expected case for sorting with the convex-segment
method.

The complexity of sorting $m$ points along a rational curve 
with the parameterization method is O($m\alpha[p]$), where $\alpha[d]$ is 
the time required to find the real roots of a polynomial equation of degree
$d$ and $p$ is the degree of the curve's rational parameterization.
The complexity of finding a parameterization depends upon the algorithm used.
However, it should be of the same order of complexity as finding the 
singularities and flexes of the curve, since singularities are 
used in the algorithm for parameterizing rational curves.

The complexity of sorting $m$ points along a sort segment
of a curve with the crawling method is O($\frac{NL}{\e}$), where $N$ 
is the time required to apply Newton's method
(which will vary from application to application), $L$ is the length
of the sort segment, and $\e$ is the size of each jump.
%
\begin{theorem}
$m$ points on a plane algebraic curve of order $n$
can be sorted by the convex-segment
method in O($mn^{3}\alpha[n]$) worst case time after \\
O($\alpha[n^{2}] + n^6\alpha[n] + n^{2}\alpha[2^{MAX}n] + 
n^{4}2^{2*MAX}$) preprocessing,
where $\alpha[n]$ is the time
required to find the real roots\footnote{The
procedure of Jenkins and Traub \cite{jen} for computing real roots
is a good choice.}
of a univariate polynomial equation of degree $n$
and {\em MAX} is the maximum number of quadratic transformations that are 
necessary to decompose any singularity of the curve into simple 
points.\footnote{{\em MAX} will usually be 1 or 2.  For
example, {\em MAX} is 1 if all of the singularities are ordinary.}
\end{theorem}
\begin{proof}
The singularities of a curve are found by solving the simultaneous system
of equations
\mbox{$\{f_{x} = 0, f_{y}=0, f_{z} = 0\}$} (Lemma~\ref{lem-singflex}),
which can be done effectively by using resultants.
Let $X$ be the real roots of the resultant of $f_{x}$ and $f_{y}$ with 
respect to y,
which is a univariate\footnote{We can remove the
homogeneous z-variable by setting it to $1$, since we are only concerned
with finite singularities.}
polynomial in $x$ of degree O($n^{2}$).
Similarly, let $Y$ be the real roots of the resultant of $f_{x}$ and $f_{y}$ with 
respect to x.
Then the singularities of the curve in the real, affine plane are
\mbox{$\{\ (x,y)\ :\ x\in X,\ y \in Y \mbox{ and } f_{z}(x,y) = 0\ \}$.}
A resultant of a pair of polynomials of degree at most $d$ in $v$ variables 
can be computed in O($d^{2v+1}\ log\ d$) time \cite{col}.
Therefore, X (and Y) can be computed in O($n^{5}\ log\ n + \alpha[n^{2}]$).
Since X and Y are of size O($n^{2}$) and  
O($n^{2}$) time suffices 
to evaluate  an equation of degree $n$, the singularities
can be computed in O($n^{5}\ log\ n + \alpha[n^{2}] + n^{6}$) time.
The flexes, which are the intersections of the curve with its
Hessian (Lemma~\ref{lem-singflex}),
can also be computed in  O($\alpha[n^{2}] + n^{6}$) time.

A curve of order $n$ has O($n^{2}$) flexes and 
singularities (Lemma~\ref{lem-maxsing}).
Therefore, it has O($n^{2}$) \wallpoints\ at its flexes.
The bound on the number of singularities is expressed in terms
of the maximum number of double points: a curve of order $n$ can
have at most $\frac{(n-1)(n-2)}{2}$ double points, and
a singularity of multiplicity $t$ counts as $\frac{t(t-1)}{2}$ 
double points.
Since O($2t$) \pseudos\ are created at a singularity of multiplicity $t$,
$\frac{2t}{\frac{t(t-1)}{2}} = \frac{4}{t-1} \leq 4$ \pseudos\ are created
per double point.
Therefore, there are no more than $4*\frac{(n-1)(n-2)}{2} =$ O($n^{2}$) 
\pseudos.

Consider the time required to compute the \pseudos.
It takes O($d^{2}$) time to apply a quadratic transformation or a 
translation to an equation of degree $d$.
It takes $\alpha[d]$ time to compute the intersections of a curve
of order $d$ with the y-axis.
During the reduction of a singularity to simple points, each quadratic
transformation can double the degree of the curve's equation, since
$y^{i}$ becomes $(xy)^{i}$.
Therefore, the equation can become of degree $2^{MAX}n$ during the
reduction of a singularity.
Finally, O($n^{2}$) quadratic transformations are sufficient to reduce
all of 
the singularities (which account for O($n^{2}$) double points) to simple
points \cite{abba3}.
We conclude that a (very pessimistic) bound on the time for computing 
the \pseudos\ is O($n^{2}(2^{MAX}n)^{2}\alpha[2^{MAX}n]$).
%A quadratic transformation reduces the multiplicity of a singularity by
%at least one???, and there are O($n^{2}$) double points.
%O($n^{2}$) quadratic transformations are sufficient to reduce the O($n^{2}$)
%singularities to simple points \cite{abba3}.
%Each quadratic transformation takes O($n^2$) time???
%(there are O($n^2$) terms in $f$ and each of its quadratic transformations??),
%and the translation of the singularity 
%to the origin (a linear substitution $x_{t} = x - a ,\ y_{t} = y - a$) 
%takes O($[n2^{\mbox{max}}]^2$) time, where {\em max} is the maximum number of
%quadratic transformations required to isolate 
%one of the curve's singularities.  
%({\em max} will usually be 1 or 2: for example, {\em max} is 1 whenever a 
%singularity is ordinary.)
%Computing the intersections of the curve with the y-axis after each
%quadratic transformation takes an additional $\alpha[n2^{max}]$ time.
%Thus, the overall bound on the  computation of the \pseudos\
%is O($n^2([n2^{max}]^2 + \alpha[n2^{max}])$).

Finally, consider the incidental \wallpoints.
Since a line intersects a curve of order $n$ at most $n$ 
times (Theorem~\ref{bezout}), 
each of the O($n^{2}$) tangents at singularities and flexes 
can intersect the curve in 
at most $n$ points.
Thus, there are O($n^{3}$) incidental \wallpoints, and they 
can be computed in O($n^{2}\alpha[n]$) time.
We conclude that the cell partition has O($n^{3}$) \wallpoints\ and
O($n^{3}$) convex segments.

Consider the time required to compute the partners of all of the \wallpoints.
The dominating expense is the computation of the set S(\wo)
of Theorem~\ref{thm-partner} for each \wallpoint\ \wo.
It takes O($k\alpha[n]$) time to compute S(\wo) for a \wallpoint\ in a cell
with $k$ \wallpoints, O($k^{2}\alpha[n]$) time to compute S(\wo) for every 
\wallpoint\ in a cell with $k$ \wallpoints, and O($\sum k_{i}^{2}\alpha[n]$)
time 
to compute S(\wo) for every \wallpoint\ in every cell, where $k_{i}$ is the
number of \wallpoints\ in cell $C_{i}$, and the sum is over all cells $C_{i}$.
Since $\sum k_{i} = O(n^{3})$, 
$O(\sum k_{i}^{2}\alpha[n]) = O(n^{6}\alpha[n])$.
Therefore, partner computation takes $O(n^{6}\alpha[n])$ time.
(This is another example of an unrealistically pessimistic worst case: 
a typical
\wallpoint\ will not lie on the boundary of a multisegment cell and its 
partner will be computed in constant time.)
We conclude that preprocessing takes 
$O(\alpha[n^{2}] + n^{6} + n^{2}\alpha[2^{MAX}n] + 
n^{4}2^{2\ MAX} + n^{6}\alpha[n])$ time.

The dominating expense of the actual sorting is the determination of the
convex segment that each sortpoint lies on.
In the worst case, it requires $O(k\alpha[n]) = O(n^{3}\alpha[n])$ time to 
compute the set S($x)$ of Theorem~\ref{thm-parents} for a sortpoint in a cell with $k$ \wallpoints, and thus $O(mn^{3}\alpha[n])$ time for all
sortpoints.
The sorting of $p$ points on a convex segment
takes O($p$) time (Theorem~\ref{thm-2.1}).
Therefore, in the worst case, the convex-segment method requires 
O($mn^{3}\alpha[n]$) time to traverse $O(n^{3})$ convex segments
and sort the points on these convex segments.
\end{proof}

\begin{corollary}
Let C be a plane curve of order $n$. 
If C has no extraordinary singularities and its
cell partition contains no multisegment 
cells, then $m$ points of C can be sorted
in O($m+n^{3}$) time, with  O($n^6 + n^2\alpha[n] + \alpha[n^{2}]$)
preprocessing.
\end{corollary}

\begin{corollary}
$m$ points on a convex segment of a plane curve of order $n$ can be sorted 
in O($m$) time, without preprocessing.
\end{corollary}
%
%        - it is hard to analyze the complexity of the parameterization
%	  method: is there a bound to the complexity of a parameterization
%	  in terms of the order of the curve? if so, what is it?
%	  what is the complexity of the algorithms for finding a param
%	  (for example, the rational parameterization algorithm)\\
%
\section{Empirical Results}

This section presents execution times for the sorting of some representative
curves by the convex-segment and parameterization methods.
These empirical results are a good complement to the complexity analysis 
of Section~1, since they capture the expected case, rather than the worst 
case, behaviour of the methods.

We do not consider the time required to find a parameterization of the
curve or to find the flexes and singularities of the curve.
The computation of a curve's parameterization is of approximately the same
complexity as the computation of a curve's singularities and flexes,
so our comparison of sorting methods should not be biased.
Moreover, each of these computations
is a preprocessing step that is entirely independent of sorting,
and the parameterization, singularities, and flexes of a curve will
(or should) often be computed already.
%We believe that the computation of a parameterization is of the same order of
%complexity as the computation of the flexes and singularities.
%Recall that the most promising method of parameterization already 
%required the singularities of the curve.
%Therefore, 
%the benefit that the first assumption delivers to the parameterization 
%method should balance the benefit that the second assumption delivers to the
%convex-segment method.

Our results are execution times in seconds on a Symbolics Lisp Machine, 
and the time spent in disk faults and garbage collection is not included.
The source code is written in Common Lisp.
The preprocessing time for the convex-segment method 
is the time required to create the cell partition and find the partners of all
of the \wallpoints.
The preprocessing and sort times for the convex-segment
method are the average of twelve trials, while the 
sort times for the parameterization method are the average of three
trials.

%I DOUBT THAT I WILL KEEP THIS PART UNTIL THE LINE LABELLED `FINISH'\\
%For many curves, one of the sorting methods is clearly superior.
%(The remainder of this paragraph is questionable.)
%?Curves with very simple parameterizations should be sorted with the
%parameterization method.  
%The complexity of a parameterization (x(t),y(t)) is dictated
%by the complexity of the less complex of x(t),y(t).
%This can be seen by considering the algorithm that finds the parameter
%value of a point ($x_{0}$,$y_{0}$):
%\begin{quote}
%\begin{enumerate}
%\item solve $x(t) = x_{0}$ ({\em or} $y(t) = y_{0}$) for t
%\item substitute the values of t found in (1) into y(t) (or x(t), whichever
%was not used in (1)) and choose the value that yields $y_{0}$ (or $x_{0}$)
%\end{enumerate}
%\end{quote}
%It is step (1) that can be expensive, so if either x(t) or y(t) is simple,
%then the entire parameterization is considered simple.
%%
%\begin{example}
%\label{ex-eight}
%Consider the curve $x^{4}-x^{2}+y^{2}=0$ (the eight curve).
%A parameterization for this curve is (cos(t) , sin(t)cos(t))$_{-\pi \leq t \leq \pi}$.
%It is trivial to determine the parameter of a curve point.
%The new method cannot compete in this situation.
%Indeed, in order to sort a set of points on a convex segment, one of the
%steps is to find the angle that each point makes with a central point,
%which involves computing the arccos of a value.
%Thus, the convex-segment method will certainly take longer.
%\end{example}
%We stress the fact that the parameterization must be very simple
%before the parameterization method is automatically chosen.
%We shall soon see an example of a curve with a rational parameterization
%of degree two that is sorted more quickly by the convex-segment method.
%The class of very simple parameterizations should include all
%parameterizations of the form 
%(cos(t),\underline{\ \ \ }),
%(sin(t),\underline{\ \ \ }),
%(\underline{\ \ \ },cos(t)),
%(\underline{\ \ \ },sin(t)).
%It is not clear that any other parameterizations should be considered very
%simple.
%
%Just as there are curves with parameterizations so simple that they
%demand the parameterization method, there are curves with parameterizations
%so complicated that it is clear that the parameterization method will
%not be competitive.
%%
%\begin{example}
%An example of a complicated parameterization is offered by the hippopede:
%$(x^{2}+y^{2})^{2} + 24(x^{2}+y^{2}) - 36x^{2} = 0$.
%Its parameterization is \mbox{$(2 cos(t) \sqrt{3-9sin^{2}(t)},
%2 sin(t)\sqrt{3-9sin^{2}(t)} )_{-\pi \leq t \leq \pi}$}.
%Ironically, the hippopede looks very similar to the eight curve (of
%Example~\ref{ex-eight}), which had a very simple parameterization.
%\end{example}
%
%FINISH

%If a \param cannot be found by any known method, then the
%convex-segment method is a clear victor.


%OMIT THIS PARAGRAPH.  WE HAVE ALREADY DISCUSSED THIS.
%In Section~\ref{sec3-chap3}, we explained that the new method might
%be slow if the cell partition contains a cell with many convex segments
%such that the expensive conditions (4)-(6) of $S^{x}$ must be tested
%in order to decide the wallpoints that bound a sortpoint's convex segment.
%Due to the screening conditions that we introduced in 
%Section~\ref{sec3-chap3}, we are confident that this situation will occur
%only very rarely.
%Nevertheless, one should be aware that there is a slight danger of the
%convex-segment 
%method sorting less efficiently when the cell partition contains
%a cell with many convex segments.

%Let us now consider the sorting of the remaining 
%curves, whose parameterizations are
%neither particularly simple nor particularly complicated.

%The choice of the best method for the remaining 
%curves usually depends upon the particular application.\\
%PRESENT EXAMPLES BEFORE DISCUSSING WHAT THEY REVEAL\\
%The convex-segment 
%method will always(???) perform the actual sort much more quickly,
%but there is an overhead to the new sort: the preprocessing required
%to create the cell partition and pair the wallpoints.
%The investment in preprocessing will not always be warranted, in which
%case the parameterization method is a viable alternative.

We consider five examples: two rational cubic curves and three non-rational
quartic curves.
Our first example illustrates the superiority of the convex-segment method.
Even when the preprocessing time is added to the sort time, the 
convex-segment method solves this problem more efficiently 
than the \param method.
The convex-segment method's rate of growth is also much smaller.
The inferiority of the crawling method is obvious from this example, and we do
not consider it further.
%
\begin{example}
A semi-cubical parabola\\
Equation of the curve: $27 y^{2} - 2x^{3} = 0$\\
Preprocessing time: 0.27 seconds\\
Parameterization: \{$x(t) = 6t^{2}$,\vspace{.5in} $y(t) = 4t^{3}\ :\ 
-\infty\ <\ t\ <\ +\infty \}$\\
%
\begin{tabular}{|l|c|c|c|}  \hline
number of sortpoints & 1 & 2 & 6 \\ \hline \hline
convex-segment &           .01 & .03 & .03 \\ \hline
%\footnote{The results should 
%only be compared vertically, not horizontally.
%The reason that sorting times sometimes decrease as more points are sorted
%is that completely different sets of points may be used in each column
%(e.g., the five points of column 3
%are not a subset of the six points of column 4)
%or different start and end points may be used.
%Thus, for example, sorting many points that are close together on a short
%sort segment may be faster than sorting a few points that are spread out
%on a long sort segment.} 
convex-segment + preprocessing & .28 & .30 & .30 \\ \hline
parameterization & .47 & .63 & 1.04 \\ \hline
crawling         & 3.14 & 2.89 & 4.77 \\ \hline
\end{tabular}
\end{example}

\figg{"A picture of the semi-cubical parabola"}{Semi-cubical Parabola}{3in}

The second example illustrates that there is sometimes a 
tradeoff between the convex-segment method
(a very fast sort that requires preprocessing) and the parameterization 
method (a moderately fast sort that does not require preprocessing).
%
\begin{example}
\label{eg-folium}
Folium of Descartes\\
Equation of the curve: $x^{3} + y^{3} - 15xy = 0$\\
Preprocessing time: 2.81 seconds\\
Parameterization: \{$x(t) = \frac{15t}{1+t^{3}}$, \vspace{.5in}$y(t) = 
\frac{15t^{2}}{1+t^{3}}\ :\ -\infty\ <\ t\ <\ +\infty \}$ \\
%
\begin{tabular}{|l|c|c|c|c|} \hline
number of sortpoints & 1 & 2 & 5 & 9 \\ \hline \hline
convex-segment &           0.01 & 0.01 & 0.05 & 0.04 \\ \hline
convex-segment + preprocessing & 2.82 & 2.82 & 2.85 & 2.85 \\ \hline
parameterization & 1.01 & 1.07 & 1.76 & 3.17 \\ \hline
\end{tabular}
%
\end{example}
\figg{"A picture of the folium"}{Folium of Descartes}{2in}
%
The remaining three curves are non-rational, so they are only 
sorted with the convex-segment method.
%
\begin{example}
\label{eg-devil}
Devil's Curve (with several connected components)\\
Equation of the curve: $y^{4} - 4y^{2} - x^{4} + 9x^{2} = 0$\\
Preprocessing time: 2.20 \vspace{.5in}seconds\\
%
\begin{tabular}{|l|c|c|c|} \hline
number of sortpoints & 1 & 4 & 7 \\ \hline \hline
convex-segment & 0.09 & 0.09 & 0.10 \\ \hline
convex-segment + preprocessing & 2.29 & 2.29 & 2.30 \\ \hline
\end{tabular}
%
\end{example}
\figg{"A picture of the devil"}{Devil's curve}{2in}
%
%\begin{example}
%\label{eg-hippopede2}
%Hippopede\\
%Equation of the curve: $x^{4} + y^{4} + 2x^{2}y^{2} - 32x^{2} + 32y^{2}$\\
%Preprocessing time: 2.56\vspace{.5in} seconds\\
%%
%\begin{tabular}{|l|c|c|} \hline
%number of sortpoints & 3 & 6 \\ \hline \hline
%convex-segment &           0.086 & 0.091 \\ \hline
%convex-segment + preprocessing & 2.646 & 2.651 \\ \hline
%\end{tabular}
%\end{example}
%\figg{"A picture of the hippopede"}{Hippopede}
\begin{example}
\label{eg-limacon}
Limacon\\
Equation of the curve: $x^{4} + y^{4} + 2x^{2}y^{2} - 12x^{3} - 12xy^{2} + 27x^{2} - 9y^{2} = 0$\\
Preprocessing time: 4.62\vspace{.5in} seconds\\
%
\begin{tabular}{|l|c|c|c|} \hline
number of sortpoints & 2 & 5 & 8 \\ \hline \hline
convex-segment & .09 & .30 & .55 \\ \hline
convex-segment + preprocessing & 4.70 & 4.92 & 5.17 \\ \hline
\end{tabular}
\end{example}
\figg{"A picture of the limacon"}{Limacon}{1.75in}
%
\begin{example}
\label{eg-Cassinian}
Cassinian oval\\
Equation of the curve: $x^{4} + y^{4} + 2x^{2}y^{2} + 50y^{2} - 50x^{2}-671 = 0$\\
Preprocessing time: 5.36\vspace{.5in} seconds\\
%
\begin{tabular}{|l|c|c|c|} \hline
number of sortpoints & 2 & 4 & 6 \\ \hline \hline
convex-segment & .14 & .17 & .19 \\ \hline
convex-segment + preprocessing & 5.50 & 5.53 & 5.55 \\ \hline
\end{tabular}
\end{example}
\figg{"A picture of the Cassinian oval"}{Cassinian oval}{2.25in}
%
\section{The Superiority of the Convex-Segment Method}

Section~\ref{sec-param} established that certain curves cannot, or should
not, be sorted by the parameterization method: curves that 
do not possess a rational parameterization and curves for which
a rational parameterization cannot be efficiently obtained.
Therefore, the convex-segment method is often 
the only viable way to sort points along a curve.

For those curves that can be sorted in either way, 
the convex-segment method is generally far more
efficient than the parameterization method at the actual sorting of the 
points.
However, the parameterization method does not have the expense of 
preprocessing that the convex-segment method does.
Therefore, when only a few points need to be sorted 
(over the entire lifetime of the curve) and the sorting of these points
must be done soon after the definition of the (rational) curve,
the parameterization method will usually be the method of choice.
The expense of preprocessing will be warranted when
sorting time is a valuable resource, as in a real-time application,
or when the number of points that will be sorted is large.
The convex-segment method will also be preferable 
when the curve is defined long before it is ever sorted
(as with a complex solid model that requires several days, weeks, or even
months to 
develop), since the preprocessing can be done at any time
that processing time becomes available  before the sort.

We conclude that the convex-segment method is an effective new method for
sorting points along an algebraic curve, and that in many situations it is 
either the only or the best method.

%OMIT THE NEXT SECTION?

%\section{An indication of the Time Required to Find Flexes and Singularities}
%
%\tab Section~\ref{sec-flexsing}\ describes how to find the flexes and 
%singularities of a curve, by solving a system of equations.
%We solve these systems with a straightforward application of MACSYMA's
%ALGSYS command.
%More efficient techniques can undoubtedly be found, so the times that we
%present below should be viewed as conservative upperbounds on the time
%required to find flexes and singularities.\\
%%
%\begin{tabular}{l|c|c|c} \hline
%Curve & Equation & \# of sings. & Time to find sings. with ALGSYS (seconds) \\ \hline
%Folium of Descartes & $(x+1)^{3} + (y-2)^{3} - 4(x+1)(y-2)$ & & 1.88,1.53 \\ \hline
%Right strophoid & $x^{3} - 2x^{2} + x-2y^{2}+xy^{2}$ & & 0.83 (incorrect answer) \\ \hline
%Lemniscate of Bernoulli & $(x^{2}+y^{2})^{2}-4(x^{2}-y^{2})$ & & 1.47 \\ \hline
%Deltoid & $(x^{2}+y^{2})^{2} - 8x(x^{2}-3y^{2})+18(x^{2}+y^{2})-27$ & & 2.25 (incorrect answer) \\ \hline
%\end{tabular}\\
%%
%\begin{tabular}{l|c|c|c} \hline
%curve & equation & \# of flexes & time to find flexes \\ \hline
%Witch of Agnesi & $x^{2}y+4y-8$ & & 0.88 \\ \hline
%Kampyle of Eudoxus & $x^{4}-1.21(x^{2}+y^{2})$ & & 60,73 \\ \hline
%Cassinian oval & & & about 5 seconds \\ \hline
%\end{tabular}\\


%We conclude that finding flexes and singularities requires on the order of
%a few seconds for low degree curves, and many seconds for high degree curves.
%Indeed, in order to find these special points for high degree curves,
%an alternative technique may be necessary.
%%

%OMIT THE NEXT SECTION? (There is more reason to omit it than the previous section.)
%
%\section{Projections of Space Curves}
%
%\tab The following results indicate the complexity of finding the equation
%of the projection of a space curve, 
%a computation that must be carried out before a space curve can be sorted
%by the convex-segment method.
%Lemma~\ref{lem-sec5-planeeqn}\ contains the relevant theory.
%The resultant of this theorem is computed using the RESULTANT command
%of MACSYMA.\\
%%
%\begin{tabular}{l|l|c} \\ \hline
%surface 1 & surface 2 & time to find projection (seconds) \\ \hline
%$x^{2} + y^{2} + z^{2}-1$ (sphere) & $(x-.5)^{2}+y^{2}+z^{2}-1$ (sphere) & 1.45 \\ \hline
%$x^{2}+y^{2}-1$ (cylinder)& $y^{2}+z^{2}-1$ (cylinder) & 2.78 \\ \hline
%$x^{2}+y^{2}+\frac{z^{2}}{4}-1$ (oblate spheroid)& $x^{2}+y^{2}+4z^{2}-1$ (prolate spheroid) & 1.28 \\ \hline
%$(x-3)^{2}+(y+4)^{2}-8z$ (paraboloid of revolution)& $(x+1)^{2}+(y-.25)^{2}+z^{2}-1$ (sphere) & 2.50 \\ \hline
%$2x^{2}-y^{2}-3z$ (saddle) & $(x-2)^{2}+(y-1)^{2}-4z$ (paraboloid) & 1.9 \\ \hline
%$x^{3}+y^{3}+6xy$ & $(x-1)^{2}+y^{2}-2$ (cylinder) & 1.28 \\ \hline
%$x^{2}+xz^{2} + y^{2}z$ & $2(x-.4)^{2}-y-30.1-3z$ (saddle) & 3.13 \\ \hline
%$x^{2}y^{2} + x^{2}z^{2} + y^{2}z^{2} -2xyz$ (Steiner surface) & $2x^{2}-y^{2}-3z$ (saddle) & 7.00 \\ \hline
%\end{tabular}
%
%The time needed to compute the equation of the projection must be added on
%to the preprocessing time.
%Thus, the convex-segment method is less competitive for space curves.
%Of course, only the compilation time is affected, not the actual sorting time.
%Recall that the parameterization method may also opt to project the space 
%curve to a plane curve if a parameterization of the space curve is difficult
%to find.
%In this case, both sorting methods suffer the same disadvantage from the
%space curve.
