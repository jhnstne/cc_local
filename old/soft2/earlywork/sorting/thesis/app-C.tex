\chapter{Parameterization Algorithms}
\label{app-D}
%
(NOTE: this may not be exactly like the thesis copy because
I accidentally axed the tex copy and this is a checkpointed version.
Check it over before using it.)

This appendix considers several algorithms that 
have been developed for the parameterization of 
plane curves and surfaces of degree two and three.
The first method proceeds by linearizing the implicit equation
of the curve or surface 
with respect to one of its variables \cite{abba1,abba2}.
For example, 
to parameterize a degree two curve, a linear transformation of the form
\[ X = \frac{a_{1}x + b_{1}y + c_{1}}{a_{3}x + b_{3}y + c_{3}},\ \ \ 
   Y = \frac{a_{2}x + b_{2}y + c_{2}}{a_{3}x + b_{3}y + c_{3}} \]
is applied to the equation of the curve, where $b_{1},\ b_{2},$ and $b_{3}$
are chosen so that the $y^{2}$ term is eliminated, and $a_{i}$ and $c_{i}$
are chosen so as to make the transformation well-defined (by ensuring
that the matrix
\[ \left[ \begin{array}{ccc}
a_{1} & b_{1} & c_{1} \\
a_{2} & b_{2} & c_{2} \\
a_{3} & b_{3} & c_{3}
\end{array}   \right] \]
is nonsingular) well-defined \cite{abba1}.
The resulting equation is linear in $y$ and easy to parameterize.
%(Theoretically, their technique generalizes to degrees four and five.
%However, it is not practical for these degrees.---remove these
%statements if Bajaj hasn't developed the algorithm rigorously for
%degree four and five.)
%
\begin{example}
\label{eg-appC}
Consider the steps that are taken by this algorithm to parameterize 
the circle \mbox{\( x^{2} + y^{2} - 1 = 0 \).}
We linearize the equation with respect to y by applying the transformation
\mbox{\( x = \frac{x_{n} + y_{n}}{y_{n}} \)},
\mbox{\( y = \frac{1}{y_{n}} \)}:
\[ \begin{array}{ll}
x^{2} + y^{2} - 1 = 0 & \rightarrow (\frac{x_{n}+y_{n}}{y_{n}})^{2} + 
  \frac{1}{y_{n}^{2}} - 1 = 0 \\
& \equiv \frac{x_{n}^{2} + y_{n}^{2} + 2x_{n}y_{n} + 1 - y_{n}^{2}}{y_{n}^2}
 = 0 \\
& \equiv x_{n}^{2} + 2x_{n}y_{n} + 1 = 0
\end{array} \]
This equation is simple to parameterize because we can solve for $y_{n}$
in terms of $x_{n}$ (\(y_{n} = \frac{-x_{n}^{2}-1}{2x_{n}} \)),
yielding the parameterization 
\( x_{n} = t \),
\( y_{n} = \frac{-t^{2}-1}{2t} \).
A parameterization of the original circle is found by substituting back
into the transformation equations:
\[ x = \frac{x_{n}}{y_{n}} + 1 = \frac{t(2t)}{-t^{2}-1} + 
  \frac{-t^{2}-1}{-t^{2}-1} = \frac{t^{2}-1}{-t^{2}-1} = 
  \frac{1-t^{2}}{1+t^{2}} \]
\[ y = \frac{1}{y_{n}} = \frac{-2t}{1+t^{2}} \]
\end{example}

The parameterization of plane curves of degree three and of surfaces of
degree two and three by the linearizing technique is analogous.
The linearizing technique becomes impractically slow for 
degrees four and five \cite{bajcomm},
and it does not generalize to higher degrees (because of the lack of a
general formula for the solution of equations of degree five or 
more \cite{hernstein}).

Another method for parameterizing curves of degree two
involves solving for the variables in a 
template parameterization \cite{hopcroft-hoffmann}.
Since a plane curve of order two can be parameterized in homogeneous
coordinates by four polynomials of degree two \cite{abba1},
a template for the parameterization can be created:
\[ x(t) = \frac{a*t^{2} + b*t + c}{d*t^{2} + e*t + f} \]
\[ y(t) = \frac{g*t^{2} + h*t + j}{k*t^{2} + l*t + m} \]
We assume that two points of the curve and the tangents at these
points are known.
These points and tangents, along with some other conditions, are used to
solve for the variables in the template parameterization by
substituting into the equation of the curve.
It is not clear whether this technique can be generalized to higher
degrees, but the preliminary evidence is not encouraging.

A surface of degree two can be parameterized by normalizing the surface's
equation
to one of a number of forms for which a parameterization is already known
(such as \( x^{2} + y^{2} - 1 = 0 \) if the surface 
is an elliptic cylinder) \cite{levin76}.
This technique collapses for surfaces of higher degree,
since no exhaustive classification, and thus no class of normal forms, 
is available for these surfaces.
