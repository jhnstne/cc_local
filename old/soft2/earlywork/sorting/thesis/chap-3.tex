\chapter{Multisegment Cells}
\label{chap-3}

When a cell of a curve's cell partition 
contains more than one convex segment (Figure~\ref{3.1}),
two of the problems associated
%
\figg{3.1}{Some multisegment cells}{2.25in}
%
with curve sorting become nontrivial: determining the partners of a 
\wallpoint\ and determining the convex segment that a sortpoint lies on.
This chapter confronts these problems.
Section~1 introduces the definitions and lemmas that are needed for 
the treatment of multisegment cells,
while Sections~2 and~3 actually solve the problems.

\section{Foundations}
\label{sec-3.1}

%Two wallpoints \wo, \wt\ at an ordinary singularity are not equivalent,
%although they do share the same coordinates.
%In particular, the tangents of the singularity 
%(or, equivalently, the cell walls)
%associated with \wo\ and \wt\ are different.
%We express this relationship as follows: $\wo \neq \wt$ but 
%$\mbox{pt}(\wo)=\mbox{pt}(\wt)$.
%Thus, pt(\wo) refers only to the coordinates of \wo.
%A wallpoint \wo\ of an ordinary singularity
%is implicitly associated with the arc of the curve that is tangent to
%\wo's tangent as it passes through the singularity.
%Moreover, \wo\ is not considered to be a 
%wallpoint of all
%of the cells that meet at the singularity: only the two cells $C_{1}, C_{2}$
%such that $W_{1}$'s arc passes from $C_{1}$ to $C_{2}$ as it travels through
%\wo.
%
%\begin{example}
%\fig{2}{chap-3}
%$W_{i}$ is associated with $T_{i}$.  
%\wo\ is associated with the solid arc, \wt\ with the dotted arc.
%\wo\ is only a wallpoint of cells 1 and 4.
%\wt\ is only a wallpoint of cells 1 and 2.
%\end{example}
%INSERT C.4

Throughout this section, C is a cell of the cell partition of 
a curve F, and P and Q are points of F.
%
\figg{3.1.5}{The inside of P's tangent}{2.25in}
%
\begin{definition}
\label{defn-inside}
If P is not a singularity or a flox,
then {\bf the inside of P's tangent} is the halfplane that contains 
all of the curve in the neighbourhood of P (Figure~\ref{3.1.5}(a)).
Otherwise, P's tangent is a wall of F's cell partition, and
{\bf the inside of P's tangent with respect to the cell C} is the halfplane
that contains C (Figure~\ref{3.1.5}(b)).
The inside includes the tangent, while the {\bf strict inside} does not.

Let P be a flox that lies on the boundary of the cell C.
Let $P_{\epsilon}$ be a point of the curve inside cell C at distance
$\epsilon > 0$ from P.
($P_{\epsilon}$ may be found by crawling into C from P.)
{\bf The outside endpoint of P's \cellsegment\ with respect to C} is the 
endpoint that lies outside of $P_{\epsilon}$'s tangent, for $\epsilon$
small (Figure~\ref{3.3}).
%use the alternative method if JEH complains about crawling
%
\figg{3.3}{E is the outside endpoint of P's \cellsegment}{1.75in}
%

If P is not a flox, then P {\bf faces} 
Q if Q lies on the inside of P's tangent (Figure~\ref{3.1.5}(a)).
Otherwise, {\bf P faces Q with respect to the cell C} if
(1) Q lies strictly inside P's tangent with respect to C,
or (2) Q lies on P's tangent and Q lies on the opposite side of P from 
the outside endpoint of P's \cellsegment\ with respect to C
(Figure~\ref{3.2.5}).
\end{definition}
%
\figg{3.2.5}{P faces both $Q_{1}$ and $Q_{2}$ with respect to C}{2in}
%
%
\begin{notation}
\#\{S\}\ is the number of elements in the set S.
\seg{xy} is the line segment between x and y, 
and it is assumed that \seg{xy} does not include 
its \mbox{endpoints x and y}.
Finally, we use `w.r.t.' as an abbreviation for `with respect to'.
\end{notation}

The following lemma establishes three important properties of the line
segment that joins two points of a convex segment.
(Recall from Section~\ref{sec-divide} what it means for a line and a curve
to cross.)
%
\begin{lemma}
\label{lem-sec31}
Consider the cell partition of a curve F, and a cell C of this partition.
Let \x\ and \y\ be two nonsingular points of a convex segment of the cell C.
Then
\begin{enumerate}
\item the curve crosses \seg{\x\y} at an even number of points (ignoring
singularities)
\item $\#\{P \in \seg{\x\y} \cap F: P \mbox{ faces \x\ w.r.t.\ C}\} = $\\
$\#\{P \in \seg{\x\y} \cap F: P \mbox{ faces \y\ w.r.t.\ C}\}$
\item for all $\alpha \in \seg{\x\y}$,\ \ \nopagebreak
$ \#\{P \in\seg{\x\alpha}\cap F: P \mbox{ faces \x\ w.r.t.\ C}\} \leq $\\
$ \#\{P \in \seg{\x\alpha}\cap F: P \mbox{ faces \y\ w.r.t.\ C}\} $
\end{enumerate}
\end{lemma}
%
\figg{4}{An example of Lemma 3.1}{2.25in}
%
\begin{example}
Figure~\ref{4} offers a hypothetical example for Lemma~\ref{lem-sec31}.
The curve F crosses \seg{\x\y} an even number of times.
\[ \{ P \in \seg{\x\y} \cap F:\mbox{ P faces \x\ \} = \{}P_{2},P_{5},P_{6}\}\]
which is of the same size as
\[ \{ P \in \seg{\x\y} \cap F:\mbox{ P faces \y } \} = \{P_{1},P_{3},P_{4}\}\]
Moreover,
\[ \{ P \in \seg{\x\alpha} \cap F: \mbox{ P faces \x } \} = \{P_{2}\} \]
which is smaller than
\[ \{ P \in \seg{\x\alpha} \cap F: \mbox{ P faces \y } \} = \{P_{1},P_{3},P_{4}\} \]
\end{example}
%
\begin{proof}{\bf (of Lemma~\ref{lem-sec31})}\\
(1) Consider the closed region $R_{\x\y}$ bounded by \seg{\x\y}\ and 
%
\figg{3.3.5}{The region $R_{\x\y}$}{1.5in}
%
\arc{\x\y} (Figure~\ref{3.3.5}).
$R_{\x\y}$ lies in the cell C:\nopagebreak
\begin{quote}\nopagebreak
\arc{\x\y}\ lies in the cell $\Rightarrow$
\seg{\x\y}\ lies in the cell (since the cell is a convex polygon)
$\Rightarrow$ the region bounded
by \arc{\x\y}\ and \seg{\x\y}\ lies in the cell (since the 
cell is a convex polygon)
\end{quote}
\x\ and \y\ are nonsingular, and
\arc{\x\y} does not contain a singularity, since it lies in the
interior of the cell.
Therefore, the curve can only cross into $R_{\x\y}$ through \seg{\x\y}.
If the curve enters $R_{\x\y}$, then it must also leave $R_{\x\y}$, since 
an infinite segment cannot remain within a 
closed region (Lemma~\ref{nonbounded}) and an algebraic curve of finite
length is closed (in particular, the curve cannot stop short in the middle of 
$R_{\x\y}$).
We claim that the 
point of departure $D$ must be distinct from the point of entry $E$,
unless all of the tangents 
at $D=E$ are \lyne{XY}, as in Figure~\ref{3.except}.
Otherwise, if $D=E$, then at least one of 
the tangents of the singularity $D$ will cross into $R_{\x\y}$ and
form a wall of the cell partition which will split $R_{\x\y}$ in two, 
contradicting the fact that all of $R_{\x\y}$ lies in the same cell.
Therefore, with the exception of the extraordinary singularities of
Figure~\ref{3.except},
the crossings of \seg{\x\y} by the curve
occur in pairs, which we shall call {\bf couples}.
This establishes condition (1) of the 
%
\figg{3.except}{The only type of singularity that can lie on \seg{\x\y} }{1.5in}
%
lemma.\\
(2) Now consider condition (2) of the lemma.
Note that the extraordinary singularities of
Figure~\ref{3.except} can be ignored during the consideration of conditions
(2) and (3), since they face both \x\ and \y\ and contribute the same
amount to the left-hand side and right-hand side of the expressions
of conditions (2) and (3).
Therefore, we can concentrate on the remaining crossings of \seg{\x\y}:
the distinct `couples'.
Let $A,B\in \seg{\x\y}$ be a couple and 
assume, without loss of generality, that A lies closer to \x\ than B does
(Figure~\ref{3.4A}(a)).
\arc{AB} is a convex segment since it lies within a cell of the cell
partition.
Therefore, A and B face each other (with respect to the cell C), by 
Lemma~\ref{lem-face}.
Since A faces B, A faces \y.
Similarly, since B faces A, B faces \x.
Therefore, one member of each couple faces \x\ and the other faces \y.
This yields the desired result:
\[ \#\{P \in \seg{\x\y}\ \cap\ F: P \mbox{ faces \x\ w.r.t.\ C}\} = 
\#\{P \in \seg{\x\y}\ \cap\ F: P \mbox{ faces \y\ w.r.t.\ C}\} \]
%
\figg{3.4A}{(a) a couple (b) open cells and open convex segments}{2.75in}
%
\ \\
(3) The point of a couple that faces \y\ (A)
is closer to \x\ than the point that faces \x\ (B).
Therefore, for all $\alpha \in \seg{\x\y}$,
\[ \#\{P \in\seg{\x\alpha}\ \cap\ F: P \mbox{ faces \x}\} \leq
\#\{P \in \seg{\x\alpha}\ \cap\ F: P \mbox{ faces \y}\} \]
\end{proof}
%
%
\section{Finding the Partners of a Curve Point}
\label{sec-3.2}

This section applies the theory of the previous section to the
problem of determining the partners of a \wallpoint.
A solution of the partner problem is needed in order to find 
the next convex segment during a traversal of a curve by convex segments.
(Recall that two \wallpoints\ $W_{1}$ and \wt\ of a cell partition
are partners if \arc{\wo\wt} is a convex segment.)
Since we identify a convex segment by its two endpoints,
if the partners of all of the \wallpoints\ have been computed, then
the two convex segments that leave a given \wallpoint\ can be 
quickly identified.

Let F be a plane curve that has been split
into convex segments by a cell partition.
Consider a multisegment cell C of this cell partition and a \wallpoint\
\wo\ of this cell.
Since singularities have been replaced by \pseudos\ 
(Section~\ref{sec-resolve}), \wo\ is either a flox, an incidental curve
point, or a \pseudo.
Theorem~\ref{thm-partner} shows how to determine whether \wo\ has a partner
in C (i.e., whether \wo\ is the endpoint of a closed convex segment in C)
and, if \wo\ does have a partner in C, how to find this partner.
In preparation for this theorem, we must make some preliminary comments.
%
\begin{definition}
A cell is {\bf closed} (resp., {\bf open}) if it is (resp., is not)
a closed polygon (Figure~\ref{3.4A}(b)).
An open cell is unbounded.
A convex segment of the curve in cell C is {\bf closed} if it is of 
finite length and 
{\bf open} if it proceeds to infinity within C.
Open segments have only one endpoint and must lie in open cells.
\end{definition}

The computation of \wo's partner involves the computation of intersections
of lines with the boundary of cell C and a traversal of the cell boundary.
Therefore, it is necessary for the cell to be closed.
If C is an open cell, then temporary cell segments must be placed across
its opening in order to artificially make it a closed cell
(Figure~\ref{3.leap}).
The added \cellsegments\ are called the {\bf \artificial\ boundary} of C, and
they must be chosen carefully.
The resulting closed cell should be a convex polygon, it should be large 
enough to contain all of the closed convex segments of the original open cell,
and it should have only one intersection with each open convex segment in the
cell.

\figg{3.leap}{A \artificial\ boundary for an open cell}{2in}

If C is an open cell that contains an open convex segment SEG,
then we shall be interested in the intersection of this open segment with the
\artificial\ boundary of C.
If \wo\ is the endpoint of SEG, then the intersection of SEG with the boundary
will behave like a partner of \wo.
Indeed, we shall identify that \wo\ has no partner by noticing that the
partner computed for \wo\ lies on the \artificial\ rather than the
original boundary of the cell.
A point of intersection of the curve with the \artificial\ boundary shall
be called a {\bf \artificialcurvepoint}.

There are now three families of \wallpoints:
(1) \original\ \wallpoints, which include floxes and incidental 
\wallpoints;
(2) \pseudos, which are the points that replace the singularities 
and guarantee a robust traversal of the curve; and
(3) \artificialcurvepoints, which are points on the \artificial\ boundary
of open cells.
%
\begin{figure}[htbp]\vspace{3.75in}\caption{The boundary points $W_{i}'$}\label{3.J}\end{figure}

%
The computation of \wo's partner involves the sorting of \wallpoints\ along
the boundary of the cell C.
However, a \pseudo\ does not lie on the boundary.
Therefore, with each \pseudo\ $W_{i}$ in C, we must associate a point $W_{i}'$
on the cell boundary.
If $W_{i} \neq \wo$,  then $W_{i}'$ is chosen to be the intersection of
the ray \ray{\wo W_{i}} with the cell boundary (Figure~\ref{3.J}(a)).
(A link is maintained between $W_{i}$ and $W_{i}'$ so that it is simple
to retrieve $W_{i}$ from $W_{i}'$.)
If \wo\ is itself a \pseudo, then it has a special associated point.
Let V be the singularity from which \wo\ is derived, let T be the tangent
to the branch of V that contains \wo, and
let $T_{1}$ be the ray of \wo's tangent that intersects 
T (Figure~\ref{3.J}(b)).
$\wo '$ is the intersection of $T_{1}$ with the cell boundary.
For notational consistency, we let $W_{i}' = W_{i}$
if $W_{i}$ is a \wallpoint\ that already lies on the cell boundary
(i.e., a flox, incidental, or \artificial\ \wallpoint).

Finally, we partition the boundary of the cell into two regions.
Let $B_{1}$ be the boundary of C from $\wo '$ to X in one direction
and $B_{2}$ the boundary from $\wo '$ to X in the other direction, where
X is defined as follows.
If \wo\ is a flox, let X be the outside endpoint of \wo's 
\cellsegment\ with respect to C (Definition~\ref{defn-inside} and
Figure~\ref{3.8A}(a)).
Otherwise, let $X \neq \wo '$ be the other intersection of \wo's tangent with
the cell boundary (Figure~\ref{3.8A}(b-c)).
We are now ready for the statement of the theorem.
%
\begin{figure}[htb]\vspace{4.5in}\caption{Partitioning the boundary of a cell}\label{3.8A}\end{figure}
%
\begin{theorem}
\label{thm-partner}
Let $S(\wo)$ = \{(\original, \pseudoalone, and \artificial) \wallpoints\ 
\mbox{W $\neq \wo$} of cell C of the cell partition of the curve F $\mid$\nopagebreak
\begin{enumerate}
\item W lies on the strict inside of \wo's tangent (with respect
to C) \nopagebreak
\item $\#\{P \in\seg{\wo W}\cap F: P \mbox{ faces \wo\ (w.r.t.\ C)}\}=$\\
$\#\{P \in \seg{\wo W}\cap F: P \mbox{ faces W (w.r.t.\ C)\} }$\nopagebreak
\item for all $\alpha \in \seg{\wo W}$, \\\nopagebreak
$ \#\{P \in\seg{\wo\alpha}\cap F: P \mbox{ faces \wo\ (w.r.t.\ C)}\} \leq$\\
$ \#\{P \in\seg{\wo\alpha}\cap F: P \mbox{ faces W (w.r.t.\ C)\} }  $
\item W faces \wo\ (w.r.t.\ C) \}
\end{enumerate}
%
\underline{{\bf Case 1}}: Suppose that $S(\wo) \neq \emptyset$.
Let \mbox{$S'(\wo) = \{\ W':W \in S(\wo)\ \}$}.
Let $B_{1}$ and $B_{2}$ be the appropriate sections of the boundary of C,
as defined above.
Either \mbox{$S'(\wo) \subset B_{1}$} or \mbox{$S'(\wo) \subset B_{2}$}.
(Assume without loss of generality that \mbox{$S'(\wo) \subset B_{1}$}.)
Sort the points of $S'(\wo)$ along $B_{1}$ from $\wo '$ to X.
That is, sort the points of $S'(\wo)$ into $S_{1}',S_{2}',\ldots,S_{p}'$,
where $S_{i}'$ is encountered before $S_{i+1}'$\ in a traversal of the
cell boundary from $\wo '$ to X along $B_{1}$.
If $S_{p}$ (the \wallpoint\ associated with $S_{p}'$) is a \artificial\
\wallpoint, then \wo\ has no partner in C.
Otherwise, $S_{p}$ is \wo's partner in C.\\
%
\underline{{\bf Case 2}}: Suppose that $S(\wo) = \emptyset$.
Then \wo\ is not a \pseudo\ and
\wo's partner lies on \wo's cell segment.
Let T(\wo)~=~\{~\original~\wallpoints\footnote{That is, floxes and incidental 
curve points.}~W~of~C~$\mid$
\begin{quote}
\begin{enumerate}
   \item W lies on \wo's \cellsegment
   \item \wo\ faces W and W faces \wo
   \item \#\{$P \in \seg{\wo W} \cap F$: P faces \wo\ (w.r.t.\ C)\} = \\
\#\{$P \in \seg{\wo W} \cap F$: P faces W (w.r.t.\ C)\} \}
\end{enumerate}
\end{quote}
\wo's partner is the element of T(\wo) that is closest to \wo.
\end{theorem}
%
\figg{3.4B}{Cell partition of a limacon}{2.25in}
%
\begin{example}
Consider the cell partition of a limacon (Figure~\ref{3.4B})
and the multisegment cell containing the convex segments \wwa\ and
\arc{W_{3}W_{4}}.
Suppose that we wish to find the partner of \wo.
$W_{3}$ violates condition (4) of S(\wo) and $W_{4}$ violates condition (2),
so $S(\wo) = \{\wt\}$ and it is clear that \wt\ is \wo's partner.

Consider the multisegment cell of Figure~\ref{3.egg} and the computation
of \wo's partner, where \wo\ is the endpoint of an open convex segment.\\
\mbox{$S(\wo) = \{\wt,W_{3},W_{4}\}$} and
\mbox{$S'(\wo) = \{\wt,W_{3},W_{4}'\}$}.
The sorted order of $S'(\wo)$ along the boundary from $\wo '\ (=\wo)$ to X
(the intersection of \wo's tangent with the boundary) is $W_{3},W_{4}',\wt$.
The last element is \wt, which is a \artificialcurvepoint.
Therefore, \wo\ has no partner.

\figg{3.egg}{Computing the partner of the endpoint of an open convex segment}{2.5in}

Finally, consider the computation of the partner of \wo\ in Figure~\ref{3.8},
where $S(\wo) = \emptyset$.
$V_{1}$, $V_{2}$ and $V_{4}$ are ruled out by condition (2) of T(\wo),
while $V_{3}$ and $V_{6}$ are ruled out by condition (3).
Therefore, $T(\wo) = \{V_{5},W_{2}\}$.
\wt\ is the closest element of T(\wo) to \wo, so it is \wo's partner.
\end{example}
%
\figg{3.8}{Partner computation when $S(\wo) = \emptyset$}{2.125in}
%
\begin{proof}{\bf (of Theorem~\ref{thm-partner})  }\\ \nopagebreak
If \wo\ is the endpoint of an open convex segment SEG, 
then let \wt\ be the intersection of SEG with the \artificial\ boundary
of the cell.
Otherwise, let \wt\ be \wo's partner.\\
\underline{{\bf Case 1}}: Suppose that $S(\wo) \neq \emptyset$.
Let \wwh\ be the boundary of the cell 
from $W_{1}'$ to $W_{2}'$, such that $X \not\in \wwh$
(Figure~\ref{6}(a)).
\wwh\ is a subset of either $B_{1}$ or 
%
\figg{6}{(a) \wwh\ is dotted (b) s and t on \seg{\wo y}}{2.5in}
%
$B_{2}$.
We will show that $S'(\wo) \subset \wwh$.
Let $s \in S(\wo)$.\\
\underline{{\bf Claim}}: \ray{\wo s} does not cross 
$\wwa \setminus \{\wt\}$.\\
\underline{{\bf Proof of claim}}:
Suppose, for the sake of contradiction, that $\ray{\wo s}$ crosses \wwa\
at $y \neq \wt$.
This is impossible if $s = \wt$ (since \wwa\ is convex) so assume that
%
%
$s \neq \wt$.\\
\underline{{\bf Subcase 1}}: Suppose that $s \in \seg{\wo y}$.
By the argument of the proof of Lemma~\ref{lem-sec31}, the
points of intersection of the curve F with \seg{\wo y} pair up.
Let t be the partner of s (Figure~\ref{6}(b)).
The segment \arc{st} is convex, so s and t face each 
other (Lemma~\ref{lem-face}).
s faces \wo\ (condition (4) of S(\wo)) and t, so
$t \in \seg{\wo s}$.
Since \mbox{$s \in S(\wo)$}, 
\[ \#\{P \in \seg{\wo s} \cap F: \mbox{ P faces \wo}\}
= \#\{P \in \seg{\wo s} \cap F: \mbox{ P faces s}\} \]
Since \seg{\wo s} does not include its endpoints,
$\seg{\wo s} = \seg{\wo t}\ \cup\ \seg{ts}\ \cup\ \{t\}$.
Also, t faces s.
Therefore, the above equation becomes 
\[ \#\{P \in \seg{\wo t} \cap F: \mbox{ P faces \wo}\} +
\#\{P \in \seg{ts} \cap F: \mbox{ P faces \wo}\} + 0 =  \]
\[ \#\{P \in \seg{\wo t} \cap F: \mbox{ P faces s}\} +
\#\{P \in \seg{ts} \cap F: \mbox{ P faces s}\} + 1\ \ \  \]
Moreover, by Lemma~\ref{lem-sec31} (\arc{st} is convex),\newpage
\[ \#\{P \in \seg{ts} \cap F: \mbox{ P faces s}\} = \]
\[ \#\{P \in \seg{ts} \cap F: \mbox{ P faces t}\} = \]
\[ \#\{P \in \seg{ts} \cap F: \mbox{ P faces \wo} \} \]
Upon the cancellation of terms in the above equation, we conclude that
\[ \#\{P \in \seg{\wo t} \cap F: \mbox{ P faces \wo}\} > \]
\[ \#\{P \in \seg{\wo t} \cap F: \mbox{ P faces s}\} =\  \]
\[ \#\{P \in \seg{\wo t} \cap F: \mbox{ P faces y}\}\ \  \]
This contradicts condition (3) of Lemma~\ref{lem-sec31} 
($SEG = \wwa$, $\x = \wo$, $\y = y$).\\
%
\underline{{\bf Subcase 2}}: Suppose that $y \in \seg{\wo s}$.
By Lemma~\ref{lem-sec31},
\[ \#\{P \in \seg{\wo y} \cap F: \mbox{ P faces \wo}\} = 
\#\{P \in \seg{\wo y} \cap F: \mbox{ P faces y}\} \]
But y faces \wo, since \wo\ and y are on the same convex segment.
Therefore, there exists $\alpha \in \seg{\wo s}$ such that
\[ \#\{P \in \seg{\wo\alpha} \cap F: \mbox{ P faces \wo}\} 
> \#\{P \in \seg{\wo\alpha} \cap F: \mbox{ P faces s}\} \]
This is a contradiction of $s \in S(\wo)$ (condition (3)).\\
\hspace*{4in}{\bf QED of Claim}\\
We conclude that \ray{\wo s} does not cross $\wwa \setminus \{\wt\}$.
In particular, \seg{\wo s'} does not cross $\wwa \setminus \{\wt\}$.
Therefore, $s'$ must either lie on \wwa, outside of \wo's tangent, or on \wwh.
However, since $s$ lies on the strict inside of \wo's tangent (by 
condition~(1) of $S(\wo)$), so does $s'$.
Moreover, the only curve points on \wwa\ are \wo\ and \wt,\footnote{Recall
the conditions that were placed on the closing boundary.}
both of which lie on \wwh.
Therefore, $s'$ must lie on \wwh, and we have successfully shown that
$S'(\wo) \subset \wwh$.

We now show that $\wt \in S(\wo)$.
\begin{quote}
\begin{description}
   \item[(1)] Suppose, for the sake of contradiction, that \wt\ lies on \wo's tangent.
By Lemma~\ref{lem-lies.on}, \wt\ must lie on \wo's wall.
Thus, \mbox{\wwh\ = \seg{\wo\wt}}, a subsegment of \wo's wall.
Again by Lemma~\ref{lem-lies.on}, \wo\ must be a flox whose tangent is a wall
of the cell partition, so \seg{\wo\wt} is a subsegment of \wo's tangent.
By condition (1) of $S(\wo)$,
\mbox{$S(\wo) \cap \seg{\wo\wt} = \emptyset$}.
Therefore, \mbox{$S'(\wo) \cap \seg{\wo\wt} = \emptyset$}.
But $S'(\wo) \subset \wwh = \seg{\wo\wt}$.
Thus, $S'(\wo) = \emptyset$, which is a contradiction.
We conclude that \wt\ does not lie on \wo's tangent.
\wt\ certainly lies on the inside of \wo's tangent, since
\wwa\ is a convex segment.
   \item[(2-3)] Lemma~\ref{lem-sec31}\ (SEG = \wwa, \x\ = \wo, \y\ = $W_{2}$)
   \item[(4)] Lemma~\ref{lem-face}
\end{description}
\end{quote}

We are now prepared to show that $\wt = S_{p}$.
Since $X$ is not on \wwh\ (by definition),
all of \wwh\ is met in traversing the cell boundary from $\wo '$ to X through
\wwh.
Therefore, 
since $\wt '$ is an endpoint of \wwh\ and all of $S'(\wo)$ is contained
in \wwh, $\wt '$ is the last element
of $S'(\wo)$ that is met in traversing from $\wo '$ to X through \wwh.
In other words, $W_{2}' = S_{p}'$, and $W_{2} = S_{p}$.

We must show that $S_{i}' \neq S_{j}'$ whenever $i \neq j$,
so that there is no ambiguity in choosing the last member of $S'(\wo)$.
Suppose that $i \neq j$ but $S_{i}' = S_{j}'$.
Then $S_{i},S_{j} \in \seg{\wo S_{i}'}$\ and we can assume without loss of
generality that $S_{i} \in \seg{\wo S_{j}}$.
\mbox{$S_{i} \in S(\wo)$} implies that \mbox{$\#\{P \in \seg{\wo S_{i}} \cap F : P 
\mbox{ faces \wo} \}$} 
\mbox{$= \#\{P \in \seg{\wo S_{i}} \cap F : P \mbox{ faces } S_{i}\}$}
\mbox{$=\#\{P \in \seg{\wo S_{i}} \cap F : P \mbox{ faces } S_{j}\}$}.
Moreover, \mbox{$S_{i} \in S(\wo)$} implies that $S_{i}$ faces \wo.
Therefore, there exists \mbox{$\alpha \in \seg{\wo S_{j}}$} such that
\mbox{$\#\{P \in \seg{\wo\alpha} \cap F : P \mbox{ faces \wo} \} $} 
\mbox{$> \#\{P \in \seg{\wo\alpha} \cap F : P \mbox{ faces } S_{j} \}$},
which is a contradiction of $S_{j} \in S(\wo)$.
Therefore, $S_{i}' \neq S_{j}'$ if $i \neq j$, and
the sort of $S'(\wo)$ is well-defined.\\
%
\underline{{\bf Case 2}}: Suppose that $S(\wo) = \emptyset$.
%
%\begin{description}
%   \item[\underline{Case 1}] [\wo\ is an ordinary singularity]
%\begin{quote}
%Suppose, for the sake of contradiction, that \mbox{pt(\wo) $\neq$ pt(\wt)}.
%Since \wo\ lies at the end of a cell wall and \wt\ lies on \wo's wall,
%\wt\ must lie strictly outside of the curve's tangent as the curve enters
%the cell from \wo.
%By Lemma~\ref{lem-forthmsamewalls}, either \mbox{\wwa $\setminus$ \{\wo,\wt\}}
%contains a flox or a singularity, or the line \lyne{\wt\wo}\ hits \wwa\ at
%a third point.
%Each of these is a contradiction of the convexity of \wwa.
%\hence pt(\wo) = pt(\wt): \wo\ and \wt\ are wallpoints of the same
%singularity.
%Clearly, \wt's arc must enter C, since \wwa\ lies in this cell.
%\end{quote}
%   \item[\underline{Case 2}] [\wo\ and \wt\ are not ordinary singularities]
%\begin{quote}
We claim that if \wo\ is a \pseudo, then \wt\ is an element of S(\wo), 
which contradicts $S(\wo) = \emptyset$:
\begin{quote}
\begin{description}
\item[(1) of S(\wo)] Since \wo\ is a \pseudo, \wt\ does not lie on \wo's tangent 
(Lemma~\ref{lem-lies.on}). \wt\ lies inside \wo's tangent because \wwa\
is convex.
\item[(2-3) of S(\wo)] Lemma~\ref{lem-sec31} (\x = \wo, \y = \wt)
\item[(4) of S(\wo)] Lemma~\ref{lem-face}
\end{description}
\end{quote}
Therefore, \wo\ is not a \pseudo\ and it is 
well-defined to speak of \wo's \cellsegment.
We show that $\wt \in T(\wo)$:\nopagebreak
\begin{description}
   \item[(1) of T(\wo)] Suppose 
that \wt\ lies strictly inside \wo's wall (w.r.t.\ C).
Then $\wt \in S(\wo)$, a contradiction of $S(\wo) = \emptyset$:\nopagebreak
\begin{quote}
\begin{description}
\item[(1) of S(\wo)] The segment \wwa\ is convex, so \wt\ lies on the inside of \wo's 
tangent.
Since \wt\ lies strictly inside \wo's wall, it cannot lie
on \wo's tangent (Lemma~\ref{lem-lies.on}).
\item[(2-3) of S(\wo)] Lemma~\ref{lem-sec31}
\item[(4) of S(\wo)] Lemma~\ref{lem-face}
\end{description}
\end{quote}
Therefore, \wt\ lies on \wo's wall.
   \item[(2) of T(\wo)] Lemma~\ref{lem-face}
   \item[(3) of T(\wo)] Lemma~\ref{lem-sec31}
\end{description}
Therefore, $\wt \in T(\wo)$.

Suppose that \wt\ is not the closest
member of $T(\wo)$ to \wo.
Let \mbox{$U \neq \wt$} be the closest.
Since \wo\ faces U, U must lie on $\seg{\wo\wt}$.
By the proof used in Lemma~\ref{lem-sec31}, the nonsingular points of 
intersection of the curve with \seg{\wo\wt} must pair up into couples,
since \wwa\ is a nonsingular, convex segment.
In particular, the original \wallpoints\ on 
$\seg{\wo U} \subset \seg{\wo\wt}$ that face \wo\ must pair
with the equal number of original \wallpoints\ on \seg{\wo U} that face U.
But U must also pair with a \wallpoint\ on \seg{\wo U} that faces U,
and there are no such \wallpoints\ remaining without a partner.
This contradiction leads us to conclude that
\wo's partner \wt\ must be the closest element of T(\wo) to \wo.
%\end{quote}
\end{proof}
%

Since the pairing of the \wallpoints\ of
a curve does not depend upon the sortpoints, it can be done in a 
preprocessing phase.
In particular, the creation of the cell partition and the computation
of partners can be done at any time  between the definition of the curve
and the sorting of points along the curve.

%There is one disadvantage to moving the pairing of wallpoints to a 
%preprocessor: one ends up pairing all of the wallpoints even if the sort
%segment of the sorting problem only contains a small subset of the wallpoints.
%If the partners were computed as they were needed, then only those wallpoints
%on the sort segment would be paired.
%One could argue that this is not an important weakness because even if some
%wallpoints are not involved in one sort, they will probably eventually enter
%into a future one.
%At any rate, the advantages of preprocessing outweigh this disadvantage.
%
\section{Finding the Convex Segment that a Sortpoint Lies On}
\label{sec3-chap3}

In this section, we present a solution to the second of our problems:
determining the convex segment that a sortpoint lies on.
This is a key step in the sorting of a set of points by the convex-segment
method, since it offers a method of determining which points lie on a
given convex segment during a traversal of the curve.

As in the previous section, let F be a plane curve that has been split
into convex segments by a cell partition.
Consider a multisegment cell C of this cell partition and a sortpoint $x$
of the curve in the interior of C.
(If $x$ lies on the boundary of the cell, then Theorem~\ref{thm-partner} 
can be used to determine its partner and thus its convex segment.)
The following theorem shows how to determine the convex segment of
C that contains $x$.
Since a convex segment is identified by its endpoints, determining the
convex segment that $x$ lies on is equivalent to finding the
\wallpoints\ that bound this convex segment.
Therefore, Theorem~\ref{thm-parents} is very similar to 
Theorem~\ref{thm-partner}, since they both
involve finding the endpoints of a given point's convex segment.

As with Theorem~\ref{thm-partner}, an open cell must be artificially closed,
and a \wallpoint\ $W_{i}$ must have an associated boundary point $W_{i}''$.
We choose $W_{i}''$ to be the intersection of \ray{xW_{i}} with the cell boundary.
We also need to partition the cell boundary into two regions again.
Let $B_{1}$ be the boundary of the cell from \xo\ to \xt\ in one direction, 
and let $B_{2}$ be the boundary in the other direction, where
\xo\ and \xt\ are the two points of intersection of 
$x$'s tangent with the cell boundary.
%
\begin{theorem}
\label{thm-parents}
\mbox{Let $S(x)$ = \{(\original, \pseudoalone, \artificial) \wallpoints\ W of C $\mid$}
\begin{quote}
\begin{enumerate}
\item W lies on the strict inside of x's tangent
\item \mbox{$\#\{P \in\seg{xW}\cap F: P \mbox{ faces x}\} =
\#\{P \in \seg{xW}\cap F: P \mbox{ faces W} \}$}
\item $\forall\ \alpha \in \seg{xW}$,
\[ \#\{P \in\seg{x\alpha}\cap F: P \mbox{ faces x}\} \leq
\#\{P \in \seg{x\alpha}\cap F: P \mbox{ faces W}\} \]
\item W faces x \}
\end{enumerate}
\end{quote}
Let $S''(x) = \{\ W'' : W \in S(x)\ \}$.
Either $S''(x) \subset B_{1}$ or $S''(x) \subset B_{2}$.
(Assume without loss of generality that $S''(x) \subset B_{1}$.)
Sort the points of $S''(x)$ into $S_{1}'',S_{2}'',\ldots,S_{p}''$,
where $S_{i}''$ is encountered before $S_{i+1}''$\ in a traversal of the
cell boundary from $x_{1}$ to $x_{2}$ along $B_{1}$.
Then either (i) $S_{1}$ and $S_{p}$ are partners and $x$ lies on the convex segment
\arc{S_{1}S_{p}}, or (ii)
x lies on an open convex segment SEG, one of $S_{1},S_{p}$ is a
\artificialcurvepoint, and the other is the endpoint of SEG.
\end{theorem}
%
\begin{example}
Consider the cell partition of the limacon (Figure~\ref{3.4B})
and the multisegment cell containing the convex segments
\wwa\ and \arc{W_{3}W_{4}}.
Suppose that we wish to know the convex segment that x lies on.
We compute $S(x)$.
\wo\ does not satisfy condition~(1), and \wt\ does not satisfy condition~(3).
Thus, $S(x) = \{W_{3},W_{4}\}$ and x must lie on \arc{W_{3}W_{4}}.

Consider the cell partition of the Cassinian oval (Figure~\ref{2.12a}).
Suppose that we wish to know the convex segment that $P_{1}$ lies on.
Since $S(P_{1}) =$\\
$\{W_{1},W_{2},W_{3},W_{4}\}$, it does not resolve the
question.
Let \xo\ and \xt\ be the two points of intersection of $P_{1}$'s tangent 
with the cell walls.
The sort of $S''(P_{1})$
from \xo\ to \xt\ is \wo, $W_{3},\ W_{4}$, \wt, so $P_{1}$ must lie on \wwa.
\end{example}
%
\begin{proof}{\bf (of Theorem~\ref{thm-parents})  }\\
The proof is entirely analogous to the proof of Theorem~\ref{thm-partner}.
Let SEG be the convex segment that contains $x$.
Let \wo\ and \wt\ be the endpoints of SEG or,
if SEG is an open convex segment, let \wo\ be its one endpoint and
let \wt\ be the intersection of SEG with the closing boundary of the cell.
We first show that $W_{1},W_{2} \in S(x)$:\nopagebreak
\begin{quote}
\begin{description}
   \item[(1)] Since $x \in \wwa$, \wwa\ is convex, and $x \neq \wo,\wt$,
both \wo\ and \wt\ must lie on the strict inside of $x$'s tangent.
    \item[(2-3)] Lemma~\ref{lem-sec31}\ (SEG = \wwa, \x = $x$, \y = \wo\ or \wt)
    \item[(4)] Lemma~\ref{lem-face}
\end{description}
\end{quote}
Let \wwh\ be the boundary of the cell
from $\wo ''$ to $W_{2}''$, 
such that \mbox{$x_{1},x_{2} \not\in \wwh$}.
\wwh\ is a subset of either $B_{1}$ or $B_{2}$.
By the argument used in Theorem~\ref{thm-partner}, $S''(x) \subset \wwh$.
\mbox{$x_{1},x_{2} \not\in \wwh$}
(by definition), so $W_{1}''$ and $W_{2}''$ are the first and last points of
$S''(x) \subset \wwh$ that 
are met in a traversal of the cell boundary 
from $x_{1}$ to $x_{2}$ through \wwh.
Therefore, $\{W_{1}'',W_{2}'' \} = \{ S_{1}'',S_{p}'' \}$, and
$\{W_{1},W_{2} \} = \{ S_{1},S_{p} \}$.
\end{proof}
%

It can be expensive to create the set S($x$).
In particular, conditions (2) and (3) require the points of intersection
of the curve with a line segment,
which involves the solution of an equation of degree $n$, where $n$
is the order of the curve.
This is an expensive operation that we would like to avoid.
Fortunately, it is usually possible to do so.\footnote{The expensive 
conditions cannot be avoided in Theorem~\ref{thm-partner}.
However, partner computation is a one-time preprocessing step.
Moreover, conditions (1) and (4) will often rule out all of the
\wallpoints\ except the partner.}
%Here is an algorithm for computing these intersections:
%\begin{quote}
%\begin{description}
%   \item[(i)] let $f(x,y) = 0$ be the implicit equation of the curve (of
%degree n) and let \seg{xy}\ be the line segment;
%compute a parametric representation $(x(t),y(t))$ of the line
%   \item[(ii)] substitute the parameterization of the line 
%into the curve equation, yielding $g(t) = f(x(t),y(t))$
%   \item[(iii)] solve for the roots of the degree n equation g(t)
%\end{description}
%\end{quote}
%Step (iii) is an expensive operation.

Before the more expensive conditions (2) and (3) of S($x$) are tested,
we would like to eliminate as many \wallpoints\ as possible from contention.
Therefore, we would like to find a collection of inexpensive conditions
that must be satisfied by the endpoints of a sortpoint's
convex segment.
The inexpensive conditions that we choose 
are motivated by conditions (1) and (4)
of S($x$) and the following observations.
First, as soon as the \wallpoint\ W is eliminated,
W's partner can also be eliminated, since 
the endpoints of a sortpoint's convex segment are partners.
Second, all of the curve segment between a \wallpoint\ $W_{1}$ and its
partner $W_{2}$ lies on one side of $\seg{W_{1}W_{2}}$,
since $\arc{W_{1}W_{2}}$ is a convex segment.
Thus, if the sortpoint does not lie on the appropriate side of 
$\seg{W_{1}W_{2}}$ (viz., the inside of the chord \seg{\wo\wt}, a term that
is defined at the end of Appendix~A),
then both $W_{1}$ and $W_{2}$ can be eliminated.

We can now present a more efficient algorithm for
finding the endpoints of a sortpoint x's convex segment.
Let R($x$) = \{\wallpoints\ W of cell C $\mid$\nopagebreak
\begin{quote}
\begin{enumerate}
    \item W and its partner in C lie on the strict inside of x's tangent
    \item $x$ lies on the strict inside of W's tangent and the strict inside
of W's partner's tangent
    \item \mbox{$x$ lies on the inside of the chord of W's convex segment 
in C\}}
\end{enumerate}
\end{quote}
R($x$) contains the desired endpoints of $x$'s convex segment.
Therefore, we are finished if $R(x) = \{\wo,\wt\}$ and \wo\ and 
\wt\ are partners, or if $R(x) = \{\wo\}$.
However, if R($x$) contains more than two \wallpoints\ or
if $R(x) = \{ W_{1},W_{2} \}$ and both \wo\ and \wt\ are endpoints of 
open convex segments, then
we must revert to an application of Theorem~\ref{thm-parents}.
Since we already know R($x$), it should be used to compute S($x$) more 
efficiently:
S($x$) = \{ $W \in R(x)$ $\mid$\nopagebreak
\begin{enumerate}
\item \mbox{$\#\{P \in\seg{xW}\cap F: P \mbox{ faces $x$ }\} =
\#\{P \in \seg{xW}\cap F: P \mbox{ faces W \} }$}
\item
for all $\alpha \in \seg{xW}$, 
\[ \#\{P \in\seg{x\alpha}\cap F: P \mbox{ faces $x$ }\} \leq
\#\{P \in \seg{x\alpha}\cap F: P \mbox{ faces W \}\ \}} \]
\end{enumerate}

%There is strong evidence to suggest that the conditions of R($x$) are almost
%always enough to find the endpoints, and that it is therefore very rare that the more
%expensive conditions of S($x$) must be tested.
%For example, 
%only two of the curves in Lawrence's catalog of curves \cite{lawrence}
%even have multisegment cells, and R($x$) is always sufficient to 
%isolate the endpoints of the convex segment of any point of these curves.

The problems associated with multisegment cells have now been solved.
This completes our discussion of the theory of the convex-segment
method of sorting points along an algebraic curve.
In the next chapter, we turn to an analysis of its behaviour.
