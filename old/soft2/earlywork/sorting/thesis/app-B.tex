\chapter{Lemmas}

This appendix presents various lemmas and theorems that are
important to the development of the theory of sorting.
Due to their technical nature, we find it more convenient
to place these results in an appendix.
%
\begin{theorem}[Bezout's Theorem {\cite[p. 59]{wa}}]
\label{bezout}
If two algebraic plane curves, of orders m and n, have more than mn common
points, then they have a common component.
\end{theorem}
%
\begin{lemma}[{\cite[pp. 25, 26, 59]{wa}}]
\label{resultant}
Let $R(x_{1},\ldots,x_{r-1})$ be the resultant of $f(x_{1},\ldots,x_{r})$
and $g(x_{1},\ldots,x_{r})$ with respect to $x_{r}$.
\mbox{$R(\alpha_{1},\alpha_{2},\ldots,\alpha_{r-1}) = 0$} if and only if
there exists $\alpha_{r}$ such that
\mbox{$f(\alpha_{1},\ldots,\alpha_{r})
= g(\alpha_{1},\ldots,\alpha_{r}) = 0 $}.
%That is, the zero set of the resultant identifies the common roots of f
%and g "modulo" $x_{r}$.
\end{lemma}
%
\begin{lemma}
\label{tangcont}
The tangent of an algebraic curve changes continuously on nonsingular
segments of the curve.
\end{lemma}
\begin{proof}
Since all polynomial functions are continuous, this is a corollary of 
Lemma~\ref{tangenteqn}.
\end{proof}
%
\begin{definition}
A polygon is {\bf simple} if (i) any pair of edges of the polygon are either
disjoint or intersect only at their endpoints, and (ii) no more than two edges
intersect at each endpoint.
\end{definition}

The two common definitions for convexity of a polygon P are:
\begin{quote}
\begin{itemize}
  \item if $v,w \in P$, then $\{tv + (1-t)w \mid 0\leq t \leq 1\} \subset P$
  \item $P = \{ \sum_{i=1}^{n} \lambda _{i}v_{i} \mid 
     \sum_{i=1}^{n} \lambda _{i} = 1,\ 0 \leq \lambda _{i} \leq 1\ \},$ 
     where $v_{1}, \ldots ,v_{n}$ are the 
     vertices of P.
\end{itemize}
\end{quote}

We present an alternative characterization of convexity that works
with the boundary rather than the interior of the polygon.
%
%
\begin{lemma}
\label{conv3}
Let P = $v_{1} \ldots v_{n}$ be a simple polygon.
P is convex if and only if, 
for every edge E = 
\seg{v_{i}v_{j}}, the line \lyne{v_{i}v_{j}} does not intersect
$P \setminus E$.
\end{lemma}
%
\begin{proof}
$\Rightarrow$:\ \ Assume that P is convex.  
Suppose, for the sake of contradiction, that \seg{v_{i}v_{j}}
is an edge of P such that the line \lyne{v_{i}v_{j}}
intersects $P\setminus E$.
Let $w$ be the first such intersection
and assume, without loss of generality, 
that $w$ lies on the ray \ray{v_{i}v_{j}}.
Then there exists $x \in$ \seg{v_{j}w} such that $x \not\in P$,
which contradicts the first definition of polygonal convexity.\\
$\Leftarrow$:\ \ Assume that, for every edge E = 
\seg{v_{i}v_{j}}, the line \lyne{v_{i}v_{j}} does not intersect
$P \setminus E$.
Suppose, for the sake of contradiction, that there exists \mbox{$v,w \in P$} and 
\mbox{$t \in [0,1]$} such that \mbox{\(X = tv + (1-t)w \not\in P\)} 
(Figure~\ref{B.first}).
Since $v \in P$ and $X \not\in P$,
$\seg{vX} \cup \{v\}$ must cross the boundary of P, say at $e$ on edge E.
$v$ (or, if $v=e$, the polygon in the neighbourhood of $v$) 
and $w$ lie on opposite sides of the line defined by E.
Therefore, since $v,w \in P$,  some edge of P must cross over
E's line, say at Y.
Since P is a simple polygon, Y cannot be a point of E.
Thus, the continuation of E intersects $P \setminus E$, a
contradiction.
Therefore, for all $v,w \in P$ and $t \in [0,1]$, 
\mbox{$tv + (1-t)w \in P$}, and P is convex.
\end{proof}
%
\figg{B.first}{An example for Lemma~B.3}{2in}

Recall from Section~\ref{sec-divide} what it means for a line to cross
a curve.
\begin{lemma}
\label{planarcutsthru}
A plane curve C crosses a line L at a nonsingular point P
if and only if the number of intersections of L with C at P is odd.
\end{lemma}
%
\begin{proof}
Assume, without loss of generality, that L
is the x-axis and P is the origin.
Let $N_{\epsilon}(0)$ be a small segment of the curve near the
origin: i.e., $N_{\epsilon}(0)$ = \{$\ c\in C\ |\ $dist$_{C}(c,0) < \epsilon$\ \},
where dist$_{C}(c_{1},c_{2})$ is the length of the curve segment between
two points $c_{1},\ c_{2}$ of C.
Since the directed tangent to the nonsingular origin is ($\pm$1,0)
and the tangent to an algebraic curve
changes continuously on nonsingular segments (Lemma~\ref{tangcont}),
there exists $\epsilon\ > 0$ such that no two points of
$N_{\epsilon}(0)$ have the same abscissa.
Within this neighbourhood, 
the curve can be represented by a function \mbox{$y = g(x)$}.
We expand $g(x)$ into a Taylor series:
\begin{equation}
\label{eq:Taylor}
g(x) = g(0) + g'(0)x + g''(0)x^2 + \ldots  
\end{equation}
Since the lowest order term in (\ref{eq:Taylor})
dominates all other terms \mbox{as $x \rightarrow 0$},
$g(x)$ changes sign as $x$ changes sign if and only if
\mbox{min$\{\ i\ |\ g^{(i)}(0) \neq 0\}$} is odd.
Also, $g(x)$ changes sign as $x$ changes sign if and only if
C crosses L (the x-axis) at P (the origin).
Therefore, C crosses L at P if and only if
\mbox{min$\{\ i\ |\ g^{(i)}(0) \neq 0\}$} is odd.

Since $y-g(x) = 0$ represents the curve C near the origin
and \mbox{$x=t,\ y=0$} is a parameterization of L, the intersections of L
with C near the origin are associated with the roots of $g(t)=0$.
In particular, using (\ref{eq:Taylor}) above, the
multiplicity of the intersection of L with C at the origin is
\mbox{min$\{\ i\ |\ g^{(i)}(0) \neq 0\}$}.
Therefore, C crosses L at P if and only if the multiplicity
of intersection of L with C at P is odd.
\end{proof}
%
\begin{lemma}
\label{nonbounded}
An infinite segment of an algebraic curve cannot be bounded within a
closed region.
\end{lemma}
%
\begin{proof}
Let S be an infinite segment of an algebraic curve and
suppose that S lies in a closed region.
Since S must twist infinitely often to avoid crossing the boundaries
of the region,
a line can be found that intersects S in an arbitrarily large
number of points, contradicting Bezout's Theorem (Theorem~\ref{bezout}).
\end{proof}
%
%\begin{lemma}[Monotonicity of the tangent on convex curves]
%\label{mono}
%The
%change in slope
%of the tangent of a convex segment is strictly monotonic,
%with the exception of a possible discontinuity at $\pm\infty$.
%Equivalently, as one moves along a convex segment in a given direction,
%the angle of the directed tangent (pointing in the direction that one is
%moving) changes strictly monotonically.
%\end{lemma}
%%
%\begin{proof}
%Let C be a convex segment of an irreducible curve.
%Fix a ray in the plane from which to measure angles from.  Let
%the measurement of angles be continuous.\footnote{That is, as an angle
%approaches 0 from the positive side and passes through 0, the angle becomes 
%negative rather than jumping to $2\pi - \epsilon$.}
%Since C is convex and thus nonsingular, the tangent changes smoothly on C
%(Lemma~\ref{tangcont}).
%Suppose that the angle of the tangent is not monotonic.
%Then we can assume, without loss of generality, that
%$\exists$ a point P of C and 
%$\delta > 0$ such that
%\mbox{$\angle$ tangent at Q} $\leq \angle$ tangent at P, $\forall\ Q \in C$
%such that $\mbox{dist}_{c}(P,Q) < \delta$ .
%Pick any $\epsilon$ neighbourhood of P, $0< \epsilon \leq \delta$.
%This neighbourhood will contain two points $R_{\epsilon},S_{\epsilon}
%\in C$ on opposite sides of P that have parallel tangent slopes:
%\begin{quote}
%	Let $P_{-\epsilon}$ and $P_{+\epsilon}$ be the points of C such that
%	\mbox{dist$_{C}(P,P_{-\epsilon})=\epsilon$} \\
%        $= \mbox{dist}_{C}(P,P_{+\epsilon})$.
%	Let $A_{1}$ and $A_{2}$ be the angles of the tangent 
%        at $P_{-\epsilon}$
%	and $P_{+\epsilon}$, respectively. Assume, without loss of generality,
%        that \mbox{$A_{1} \leq A_{2}$}.
%	Since \mbox{$A_{1} \leq A_{2} \leq \angle$ (tangent at P)} and the tangent
%	is continuous, $\exists$ a point of \arc{P_{-\epsilon} P} 
%        whose tangent's
%	angle is $A_{2}$.
%\end{quote}
%Because C is convex, all of it must lie on one side of the tangent
%at $R_{\epsilon}$.  Similarly on one side of $S_{\epsilon}$'s tangent.
%But the tangents at $R_{\epsilon}$ and $S_{\epsilon}$ are parallel, and 
%yet not equal (otherwise this tangent intersects the cubic four times).
%\hence All of C must lie in between $R_{\epsilon}$ and $S_{\epsilon}$'s
%tangents.\\
%dist$(R\se,S\se) \leq \mbox{dist}_{C}(R\se,S\se) 
%= \mbox{dist}_{C}(R\se,P) + \mbox{dist}_{C}
%(P,S\se) \leq \epsilon + \epsilon = 2\epsilon$\\
%But $\epsilon$ was chosen arbitrarily such that $0 < \epsilon \leq \delta$.
%\hence C is sandwiched in between two parallel lines that lie arbitrarily
%close together.
%\hence C is a line, contradicting the convexity of C and the irreducibility
%of the curve.
%\hence The angle of the tangent must change monotonically.\\
%Moreover, the angle of the tangent cannot ever 
%remain constant as one moves along
%the cubic curve, otherwise two points would share the same tangent, 
%contradicting Bezout's Theorem.
%Thus, the angle of the tangent must change strictly
%monotonically.
%\end{proof}

%We translate Lemma~\ref{conv3}'s characterization of polygonal convexity 
%into an analogue for curves.
%
%\begin{lemma}
%\label{convmeans2}
%Let C be a segment of an irreducible plane curve.
%C is convex $\Leftrightarrow$ \  no line has more
%than two intersections with C.
%\end{lemma}
%
%\begin{proof}\\
%$\Leftarrow$:\ \ Trivial: a tangent is a line.
%Moreover, C will be nonsingular: if S is a singularity and $P \in C$,
%then \seg{SP}\ has at least three intersections with C.\\
%$\Rightarrow$:\ \ Suppose that C is convex and line L intersects C 
%three times.
%If any two of these intersections are at the same point, then L is a tangent 
%to the curve and has three intersections with it, contradicting convexity.
%\hence L's intersections with C are at distinct points: say R, S, and T.
%Assume, without loss of generality, that S lies on \arc{RT}.  
%We will show that there is a point of 
%\arc{RS} whose tangent strikes \arc{ST},
%thus contradicting convexity.
%
%Rotate and translate the curve so that \ray{RT} is the positive x-axis.  
%Assume, without loss of generality, that \arc{RS} lies above the x-axis.
%Consider the tangent at R, directed toward \arc{RS}\ (call it $L_{R}$),
%and the tangent at S, directed toward \arc{ST}\ (call it $L_{S}$).
%(Whenever we speak of a tangent, consider it to be pointed in the direction
%of travel from R to T.)
%
%\fig{B.1}
%\fig{3withadditions}{cubic.tex}
%
%If we measure angles counterclockwise from the positive x-axis, then
%\mbox{$\angle L_{R} \in (0,\pi)$}, since \arc{RS} lies above the x-axis,
%and \mbox{$\angle L_{S} \in (-\pi,0)$}.
%By Lemma~\ref{mono}, the angle of the tangent must either increase or 
%decrease monotonically from R to S.
%By the convexity of C, all of \arc{RT}\ lies on the same side of R's
%tangent.
%In particular, as the curve leaves R along \arc{RS}, it lies on
%T's side of the tangent at R.
%Therefore, the angle of the tangent must decrease as the curve leaves R,
%since T lies on the positive x-axis and $\angle L_{R} \in (0,\pi)$.
%Thus, $\exists~p \in \arc{RS}$ whose tangent has angle 0 (again by
%Lemma~\ref{mono}).
%As the tangent sweeps from angle 0 (parallel to \lyne{RT}) at p
%to an angle less than 0 at S, the intersection of the tangent with the 
%x-axis \lyne{RT} sweeps from a point at $\infty$ to S, through T.
%\hence Some tangent of \arc{RS} (actually infinitely many) strikes \seg{ST}.
%Since the curve is continuous, any line separating the two points S and T
%of the curve must intersect the curve somewhere on \arc{ST}.  Hence, some
%tangent of \arc{RS} strikes \arc{ST}, 
%contradicting the convexity of \arc{RT}.
%We conclude that no line can intersect C more than twice.
%\end{proof}
%
\begin{lemma}
\label{threecross}
Let \pq\ be a nonconvex segment of a curve F such that
\pq\ contains no singularities or floxes.
There exists a line L that crosses
\pq\ in at least three distinct points.
\end{lemma}
%
\begin{proof}
Since \pq\ is nonconvex, there exists a line ${\cal L}$ that
intersects \arc{PQ} at least three times, such that not all
of the intersections occur at a flex of even order.
There are three cases to consider.
%
\begin{quote}
\begin{description}
\item[\underline{Case 1}] If ${\cal L}$ crosses \arc{PQ} in at least three
distinct points, then we are done.
\item[\underline{Case 2}] Suppose that ${\cal L}$ intersects \arc{PQ} at less
than three distinct points.  
Since \arc{PQ} contains no floxes, ${\cal L}$ must intersect \pq\ at two
distinct points.
By the pigeon-hole principle, two of the intersections must occur at the
same point. 
That is, ${\cal L}$ is tangent to the curve at some point.
\item[\underline{Case 3}]
Suppose that ${\cal L}$ intersects \pq\ at three or more
distinct points but crosses \pq\ at less than three points.
Thus, ${\cal L}$ touches but does not cross \pq\ at some point $x$.
Since \pq\ is nonsingular, the tangent changes continuously, so ${\cal L}$
must be tangent to \pq\ at $x$ in order to touch but not cross \pq.
\end{description}
\end{quote}
%
Therefore, either ${\cal L}$ already satisfies the requirements or
there exists $x \in \arc{PQ}$ such that $x$'s
tangent $T_{x}$ strikes \arc{PQ} at another point (Figure~\ref{B.eps}).
Let $y \neq x$ be an intersection of \tx\ with \pq\ such that 
\xya\ lies strictly
inside \tx\ (\ie, y is the closest intersection to x).
For any $\e > 0$, let \lep\ be the line such that (i) \lep\ 
is parallel to \tx,
(ii) \lep\ lies inside \tx, and (iii) \lep\ is at a distance of \e\ from \tx.
It can easily be shown that there exists an $\e > 0$ such that 
\lep\ crosses the curve at least three times.
%$\forall\ \e > 0$, if $\lep \cap \xya \neq \emptyset$, then let \xpe\ be the
%intersection of \lep\ with \xya\ such that $\arc{x\xpe} \cap \lep = \{\xpe\}$
%(\ie, \xpe\ is the first intersection of \xya\ with \lep\ after x); and, if 
%$\lep \cap (F \setminus \xya) \neq \emptyset$, then 
%let \xme\ be the intersection
%%
%\fig{B.2}
%%Figure 4 of lemma9.tex
%%
%of \lep\ with $F \setminus \xya$ such that $\arc{x\xme} \cap \lep = \{\xme\}$.\\
%The points of the curve
%\marginpar{    (*)}
%F that have tangents parallel to \tx\ are isolated, otherwise some line
%parallel to \tx\ would strike the curve in an infinite number of points and
%thus be a reducible component of the irreducible curve F.
%Let 
%\[ \mbox{dist} = \left\{ \begin{array}{ll}
%	\mbox{$+\infty$}  &\mbox{if there are no points P of F} \\
%                          &\mbox{with tangents parallel to \tx} \\
%			  &\mbox{such that P lies strictly inside \tx}\\
%	\mbox{min\{$\e >0\ |\ $\lep\ is tangent to the curve\}}&\mbox{otherwise}
%	\end{array}
%	\right.  \]
%By (*), dist is well-defined.\\
%Since x is not a flox or singularity, the curve does not cross \tx\ at x
%(by Lemma~\ref{planarcutsthru}), so there is a neighbourhood of x that lies
%inside \tx.
%By (*), the curve must actually lie {\em strictly} inside \tx\ in some
%neighbourhood of x, and $\exists\ \eo > 0$ such that $\forall\ \e < \eo$,
%\xpe\ and \xme\ exist.
%Thus, \mbox{$\forall\ 0 < \e < $ min\{\eo,dist\}}, \xpe\ and \xme\ exist
%and \lep\ {\em crosses} the curve at \xpe\ and \xme\ (\ie, \lep\ is not
%tangent to \xpe\ or \xme).
%We have two of our desired three crossings.\\
%Now consider the intersections of \lep\ with \xya\ close to y.\\
%$\forall\ \e > 0$, if $\lep \cap \xya \neq \emptyset$, then let \ye\ be the
%intersection of \lep\ with \xya\ such that $\arc{y\ye} \cap \lep = \{\ye\}$.
%\xya\ lies strictly inside \tx, by the choice of y, so
%$\exists\ \et > 0$ such that \ye\ exists $\forall\ \e < \et$.
%Thus, \mbox{$\forall\ 0 < \e <$ min \{\et,dist\} }, \ye\ exists and
%\lep\ {\em crosses} the curve at \ye.
%\hence \mbox{$\forall\ 0 < \e <$ min \{\eo,\et,dist\} }, \xpe, \xme\ and
%\ye\ exist and \lep\ crosses the curve at these points.\\
%Note that if \xya\ crosses \lep\ once, then it must do so again, in order
%to return to \tx.
%Thus, if \lep\ crosses the curve at \xpe\ and \ye, then $\xpe \neq \ye$,
%because \xpe\ is the closest crossing to x, while \ye\ is the closest
%crossing to y.
%Clearly, $\xme \neq \xpe$ and $\xme \neq \ye$.
%Thus, \mbox{$\forall\ 0 < \e <$ min \{\eo,\et,dist\}}, \lep\ crosses F at 
%three {\em distinct} points.\\
%$\xpe, \ye \in \xya \subset \pqi$, but it is possible 
%that $\xme \not\in \pq$.
%However, since $x \neq P$ or Q, if we choose \e\ small enough, \xme\ will
%be close enough to x that it will remain on \pq.
%%\hence $\exists\ \e > 0$ such that line \lep\ crosses \pq\ at 
%three distinct points.
\end{proof}
\figg{B.eps}{Some $L_{\epsilon}$ will cross the curve at least three times}{1.5in}
%
\begin{lemma}
\label{lem-face}
If \arc{\wo\wt}\ is a convex segment in the cell C, 
then \wo\ and \wt\ face each other (with respect to C).
\end{lemma}
%
\begin{proof}
If \wo\ is not a flox, then \wo\ faces \wt\ simply because 
\wt\ must lie inside of \wo's tangent by convexity.

Suppose that \wo\ is a flox and \wt\ lies on \wo's tangent 
(Figure~\ref{fig-flox}).
The convexity of \arc{\wo\wt}\ implies that a point of
\arc{\wo\wt} lies inside the tangent of any other point of \arc{\wo\wt}.
Therefore, $W_{2}$ 
will lie strictly inside of the curve's tangent as the curve
leaves $W_{1}$ along \arc{\wo\wt}.
That is,
\wt\ and the outside endpoint of \wo's \cellsegment\ with respect to C
will lie on opposite sides of \wo.
Therefore, $W_{1}$ faces $W_{2}$ with respect to C.

By symmetry, $W_{2}$ faces $W_{1}$ (with respect to C).
\end{proof}
\figg{fig-flox}{\wt\ lies on \wo's tangent}{1.5in}
%
\begin{lemma}
\label{lem-lies.on}
Let \wo\ and \wt\ be partners.
If \wt\ lies on \wo's tangent, then \wo\ must be a flox.
(Thus, if \wt\ lies on \wo's tangent, then \wt\ must lie on \wo's wall.)
\end{lemma}
\begin{proof}
Assume that \wt\ lies on \wo's tangent.
Suppose that \wo\ is an incidental \wallpoint.
Since \wwa\ is convex, it must look like Figure~\ref{3.K}(a).
Thus, since \wo's tangent is not the same as \wo's wall,
\wo's wall must cross \wwa\ and split \wwa\ over two cells,
which is a contradiction.

Suppose that \wo\ is a \pseudo.
Let V be the singularity from which \wo\ was derived.
Either the singularity's tangents cross \wwa\ (Figure~\ref{3.K}(b)),
a contradiction as in the incidental case, 
or \wwa\ intersects \arc{V\wo} (Figure~\ref{3.K}(c)),
causing a singularity in the
interior of the cell, which is also a contradiction (since singularities
will only occur on the boundaries of a cell, by the definition of a cell
partition).
\end{proof}
%
\figg{3.K}{(a) incidental \wo\ (b-c) pseudo \wo}{2in}
%
%
\begin{lemma}
\label{lem-forthmsamewalls}
Let x be a point of a curve such that the tangent of x strikes the curve
again at y.
Let $z \in \lyne{xy} \setminus \ray{xy}$.
Suppose that y lies strictly outside of the curve's tangent as the curve
leaves x along \arc{xy} (Figure~\ref{B.4}(a)).
Then $\arc{xy}$ must contain a flox, a singularity, or a point 
of \ray{xz}.
\end{lemma}
%
\figg{B.4}{(a) $x$, $y$, and $z$ in Lemma~B.9 (b) $\arc{xy}$ without singularities or floxes}{1.5in}
%
\begin{proof}
Suppose that $\arc{xy}$ does not contain a flox or a singularity.
Thus, the curve cannot cross itself on $\arc{xy}$ and there cannot be
any point on $\arc{xy}$ where the curvature changes from concave to
convex or vice versa.
In other words, $\arc{xy}$ must spiral around in ever larger 
circles (Figure~\ref{B.4}(b)).
Since $y$ lies strictly outside of the curve's tangent as the curve leaves
$x$ along $\arc{xy}$, the curve must spiral by an angle of at least $2\pi$ to 
get from $x$ to $y$.
Therefore, \arc{xy}\  must cross \ray{xz}.
\end{proof}
%
%\begin{lemma}
%\label{Sempty}
%Let x be a point on a connected component CC of a curve such that CC
%has no \wallpoints.
%Then $S(x) = \emptyset$.
%\end{lemma}
%\begin{proof}
%Let $W_{i}$ be a \wallpoint of the curve.
%Assume without loss of generality that $W_{i}$ lies strictly inside of
%x's tangent (otherwise $W_{i}$ violates condition (1) of S(x)).
%CC is
%\end{proof}
%
%
%\begin{lemma}
%\label{lem-not2bachelors}
%Suppose that U and V lie in the cell C and neither of them has a partner in C.
%Then (U,V) is not mistaken for a pair.
%\end{lemma}
%\begin{proof}
%We will show that either U is not the computed partner of V or 
%V is not the computed partner of U.
%Let $S^{U}$ (resp., $S^{V}$) be the S set of Theorem~\ref{thm-assocwpts}\
%with $x = U$ (resp., $x=V$).
%\begin{description}
%   \item[{\underline{Case 1 [$S^{U} \neq \emptyset$]}}]  Suppose that V is
%computed to be the partner of U.
%Thus, $V \in S^{U} \Rightarrow$ V does not lie on U's cell wall (by conditions
%(1)-(2) of $S^{U}$) $\Rightarrow$ U will never be computed as V's partner
%via step (ii) of partner computation.
%Therefore, it is sufficient to show that U is not computed to be V's partner
%in step (iii).
%In other words, we must show that $U \not\in S^{V}$.
%
%Let $U_{2}$ be the intersection of U's open convex segment with a wall that
%has been added to close up the cell and let $\hat{UU_{2}}$ be the appropriate
%portion of the cell boundary, as defined in Theorem~\ref{thm-assocwpts}.
%By the proof of Theorem~\ref{thm-assocwpts}, $S^{U} \subset \hat{UU_{2}}$,
%so $V \in \hat{UU_{2}}$.
%That is, V lies in the region bounded by the cell walls $\hat{UU_{2}}$ and
%the curve segment $\arc{UU_{2}}$.
%
%\fig{5}{app-B}
%
%V's open segment cannot cross $\arc{UU_{2}}$ since this would cause a 
%singularity in the middle of the cell.
%Therefore, the intersection $V_{2}$ of V's open segment with the cell
%boundary (complete with added walls) must lie in $\hat{UU_{2}}$.
%That is, if $\hat{VV_{2}}$ is the appropriate portion of the cell
%boundary, $\hat{VV_{2}} \subset \hat{UU_{2}}$ and, in particular,
%$U \not\in \hat{VV_{2}} \supset S^{V}$.
%   \item[{\underline{Case 2 [$S^{U} = \emptyset$ and U is 
%an ordinary singularity]}}] Consider the two convex segments in cell C 
%that
%are associated with U and V: that is, the two ends of the curve.\\
%INCOMPLETE
%The two ends of an open curve intersect at a point at infinity, since every
%plane curve is closed in the projective plane.
%Therefore, the two ends of the curve are asymptotic to the same line.
%The four ways in which the two ends of a curve can approach the same
%line asymptotically are:
%\fig{6}{app-B}
%\mbox{$.\raisebox{1.5ex}{.}.$}\ It is impossible for the two ends of the
%curve to both lie in the same cell and originate from wallpoints of the
%same ordinary singularity.
%\hence If U is an ordinary singularity, then V will not be a wallpoint 
%of this same singularity.
%\hence V will not be computed to be U's partner.
%   \item[{\underline{Case 3 [$S^{U} = \emptyset$ and U is a flox or incidental
%wallpoint]}}] Let $T^{U}$ (resp., $T^{V}$) be the T set of 
%Theorem~\ref{thm-samewall}\ with $W_{1} = U$ (resp., $W_{1} = V$).
%From the four cases above, it can be seen that if the two ends of the curve
%lie in the same cell
%and originate from the same wall of the cell partition, then these 
%originating wallpoints do not face each other.
%Therefore, $V \not\in T^{U}$ and V is not computed to be U's partner.
%\end{description}
%\end{proof}
%
%\begin{lemma}
%\label{lem-noconfusion}
%Let W be a wallpoint of a cell C that is the source of an open convex segment
%in C, and let x be a sortpoint on this open convex segment.
%Suppose that \mbox{$\{S_{1},S_{p}\}$} are the first and last elements of the
%sorted set $S^{x}$ of Theorem~\ref{thm-assocwpts}.
%Then only one of $S_{1},S_{p}$ lacks a partner in C.
%Thus, there is no confusion in the choice of the wallpoint associated with x.
%\end{lemma}
%
%\begin{proof}
%By Case 2 of Theorem~\ref{thm-assocwpts}, we know that
%$W \in \{S_{1},S_{p}\}$, so at least one of $S_{1},S_{p}$ lacks a 
%partner in C.
%Let V be the other wallpoint of the curve that is an endpoint of an open
%segment, and suppose that V lies in C.
%We must show that $V \not\in \{S_{1},S_{p} \}$.
%
%Let $W_{1}$ be the intersection of W's open convex segment with the cell 
%wall that has been added to close up cell C (see Theorem~\ref{thm-assocwpts}),
%and let $\hat{WW_{1}}$ be the appropriate portion of the boundary of C, as
%defined in Theorem~\ref{thm-assocwpts}.
%By the proof of Theorem~\ref{thm-assocwpts}, $S^{x} \subset \hat{WW_{1}}$.
%Therefore, it is sufficient to show that 
%$V \not\in \hat{WW_{1}} \supset S^{x} \supset \{ S_{1}, S_{p} \}$.\\
%Lemma~\ref{lem-not2bachelors} revealed that the two ends of an open curve 
%are asymptotic to the same line, leaving four cases to consider:
%
%\fig{7 (this is fig 6 with V,W labels}{app-B}
%
%It can be shown in all four cases that $V \not\in \hat{WW_{1}}$.
%INCOMPLETE
%\end{proof}
