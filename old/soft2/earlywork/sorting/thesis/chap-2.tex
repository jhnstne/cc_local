\chapter{The Convex-Segment Method of Sorting}
\section{Sorting a Convex Segment}
\label{sec-2.1}

The observation that motivates the new method is that 
a convex segment can be sorted easily.
Since every curve is a concatenation of convex segments (Figure~\ref{2.2}),
this suggests a divide and conquer strategy.
%
\figg{2.2}{A curve is a collection of convex segments}{1.5in}

The new method only applies directly to plane curves.
However, the sort of a space curve
(\ie, a curve that does not lie in a plane)
can be mapped into a sort of a plane curve,
as we shall see in Section~\ref{sec-spacecurve}, so there is
no loss of generality.

\begin{definition}
A segment \arc{PQ} of a plane curve is {\bf convex} if no line has more than 
two intersections with \arc{PQ}, unless all of the intersections occur at
the same point and are even in number\footnote{The normal definition of 
convexity is that 
\arc{PQ} is convex if no line has more than two intersections with \arc{PQ}.
We amend this definition because, for our purposes, a curve such as 
\mbox{$y=x^{4}$ } can be considered convex.
Thus, in the terminology of Section~\ref{sec-divide}, we allow a convex 
segment to contain flexes of even order.} (Figure~\ref{2.1}).
%
\begin{figure}[htb]\vspace{4.75in}\caption{(a) is convex; (b)-(d) are not}\label{2.1}\end{figure}
%\figg{2.1}{(a) is convex; (b)-(d) are not}
%

There is another useful characterization of convexity:
a plane segment is convex if it lies entirely on one side of
the closed halfplane determined by the tangent line at any point of
the segment \cite{docarmo}. 
\end{definition}
\begin{definition}
A polygon P is {\bf convex} if, for all $A,B \in P$, the line 
segment \seg{AB}\ is entirely contained in P.
The {\bf convex hull} of a set of points S
is the smallest convex polygon that contains S.
\end{definition}
%
\begin{notation}
Let P and Q be points of a curve C.
\pq\ is the segment of the curve joining P to Q.\footnote{If the 
curve is closed, then the context should make clear which segment is
intended.}
\end{notation}

The following theorem establishes that it 
is simple to sort a set of points on a convex segment.

\begin{theorem}
\label{thm-2.1}
Let $p_{1},\ldots,p_{n}$ be points of a convex segment \arc{AB}, 
and let H be the convex hull of A, B, $p_{1},\ldots,p_{n}$.
Then the vertices of H are A, B, $p_{1},\ldots,p_{n}$.
Moreover, the order of the vertices on the boundary of H is exactly
the order of the points on \arc{AB} (Figure~\ref{2.3}).
\end{theorem}

\figg{2.3}{An example of Theorem 2.1}{2.25in}
%
\begin{proof}
It is sufficient to
show that the polygon created by joining the points A, B, 
$p_{1},\ldots,p_{n}$ in sorted
order is their convex hull.  Let this polygon 
be $P = r_{0}r_{1}\ldots r_{n+1}$\ , where $r_{0} = A$, $r_{n+1} = B$, and
$r_{1},\ldots ,r_{n}$ is the sorted order of the $p_{i}$ on the curve.
Let $E = \seg{r_{i}r_{i\oplus 1}}$ be an edge of P (where 
$\oplus$ is addition mod $n+2$), \lyne{E} the infinite
line containing E, and $\arc{E}~=~\arc{r_{i}r_{i\oplus 1}}$.
By convexity, \lyne{E}\ can have only two
intersections with \arc{AB}, those at $r_{i}$ and $r_{i\oplus 1}$.
Therefore, all of \arc{E}\ must lie on the same side of \lyne{E},
with the rest of \arc{AB}\ on the other side.\footnote{A point
of tangency to \lyne{E} counts as two intersections, so \arc{AB} cannot
stay on \arc{E}'s side by being tangent at $r_{i}$
or $r_{i\oplus 1}$.}
Since none of the $r_{j}$ lie in between $r_{i}$ and $r_{i\oplus 1}$,
all of the points $r_{j}$ lie on one side of the halfplane defined
by the edge E, and only the endpoints of E lie on \lyne{E}.
Suppose, for the sake of contradiction,
that the continuation \lyne{E} of E intersects another edge
\seg{r_{j}r_{j\oplus 1}}.
Either $r_{j}$ and $r_{j\oplus 1}$ lie on opposite sides of \lyne{E}
or one of $r_{j},r_{j\oplus 1}$ lies on \lyne{E},
a contradiction in either case.
Therefore, the continuation of any edge of the polygon P does not strike any other
edge of P.
By Lemma~\ref{conv3}, P is convex.
Since a convex polygon is the convex hull of its vertices,
P is the convex hull of the $r_{i}$'s.
\end{proof}

\begin{corollary}
\label{easyconic}
Conics can be sorted easily.
\end{corollary}
\begin{proof}
Conics are convex, since 
a line can have only two intersections with an irreducible curve of order 
two (Theorem~\ref{bezout}).
\end{proof}

When Theorem~\ref{thm-2.1}\ is used to sort a 
set of points on a convex segment,
there is no need to actually create the convex hull.
Let \mbox{$Q = \frac{1}{3}(A+B+p_{1})$},
the barycenter of A, B, and $p_{1}$.
Q will lie in the interior of the convex hull of 
A, B, $p_{1},\ \ldots,\ p_{n}$ (Figure~\ref{2.4}).
Consider the angles that the lines $\lyne{QA},\ \lyne{QB},\ 
\lyne{Qp_{1}},\ \ldots,\ \lyne{Qp_{n}}$ make with the positive x-axis, and
sort these angles from $\angle$\lyne{QA} to $\angle$\lyne{QB}
(where $0 \equiv 2\pi$).
Since A, B, $p_{1},\ \ldots,\ p_{n}$ are the vertices of a convex polygon
and Q is a point in the interior of this polygon, the order of the angles
from $\angle$\lyne{QA} to $\angle$\lyne{QB} is equivalent to the order of the
vertices on the polygon from A to B.
Therefore, $p_{1},\ \ldots,\ p_{n}$ can be sorted on \arc{AB}\ by sorting
the angles that they make with a central point.

\figg{2.4}{Sorting points on a convex segment by sorting angles}{2.25in}

Once it is realized that points on a convex segment can be sorted easily,
the focus on curve sorting can change,
since the problem has essentially been reduced from sorting points to sorting
convex segments.
Of course, the curve must first be divided up into convex segments.
We consider this division problem in the next section.
%
\section{Convex Segmentation}
\label{sec-divide}

The crucial step of the new method is the decomposition of the curve
into convex segments.
This decomposition allows us to take advantage of the simplicity of sorting
points on a convex segment.
The decomposition is achieved by the tangents at certain special 
points of the curve: the singularities (Appendix~\ref{app-defn}) and flexes.
%\footnote{These tangents do not
%necessarily split the curve into convex segments without redundancies.
%That is, perhaps a strict subset of the boundaries
%that the tangents create on the curve would satisfactorily divide the curve
%into convex segments.}

\begin{definition}
A {\bf point of inflection}, or {\bf flex} for short,
of a curve F is a nonsingular point $P \in F$ whose tangent
has three or more intersections with F at P.
(The tangent of most simple points 
intersects the curve twice at the point of tangency.)
The number of intersections of the flex's tangent with the curve at the 
flex is called the {\bf order} of the flex.
A flex of odd order is called a {\bf flox}.\footnote{This is our own term.
It is not used in the literature.}
\end{definition}
%
\begin{example}
The origin of \(y = x^3\) is a flox.
\end{example}

The interesting property of a flex is that the curve can only change its
direction of curvature (from convex to concave or vice versa) at a
flex or a singularity.
In other words, flexes are the only
nonsingular points P such that the curve in any neighbourhood of P can
lie on both sides of P's tangent.
This is an important property because the curve in the neighbourhood
of a point P of a convex segment always lies on one side of P's tangent.
Since the curve in the neighbourhood of a flex of even
order does not lie on both sides of P's tangent 
(Lemma~\ref{planarcutsthru}), we will only be interested in floxes.

%
%Unless explicitly noted, hereinafter `flex' shall refer to a flex of odd
%order.
%We shall try to use the term `flox'; however, since this is rather a clumsy
%term, we shall sometimes refer to floxes simply as flexes.
\figg{2.4A}{(a-b) touch (c) touch and cross}{1.5in}
\begin{definition}
\label{defn-cross}
The {\bf inside} (resp., {\bf outside}) of a plane curve \mbox{f(x,y) = 0}
is the halfplane \mbox{f(x,y) $<$ 0} (resp., \mbox{f(x,y) $>$ 0}).
Two curves {\bf touch} at P if they intersect at P.
A line L {\bf crosses} a plane curve C at P if L touches C at P and L 
lies on both the inside
and outside of C in any neighbourhood of P (Figure~\ref{2.4A}).
\end{definition}
%

The tangents at the singularities and floxes of a curve subdivide the plane
of the curve into several cells and split the curve into several segments.
The following theorem establishes that each of these segments is convex.
%As motivation for the next theorem, notice that as a curve passes
%through an ordinary singularity, it crosses one of the tangents of the
%singularity.
%However, the curve need not cross the tangent of an extraordinary
%singularity.
%
%\fig{6}{chap-2}

\begin{theorem}
\label{tangents-split}
Let T be the set of tangents of the singularities and floxes of a curve F,
and let \arc{PQ}\ be a nonconvex segment of F.
Then some tangent of T touches or crosses \arc{PQ}.
(Moreover, unless \arc{PQ} contains a singularity, some tangent of T
crosses \arc{PQ}.)
%Moreover, if no tangent crosses \arc{PQ}, then \arc{PQ}\ must contain
%an extraordinary singularity SING such that the curve lies entirely on 
%one side of SING's tangent in a small enough neighbourhood of SING.
\end{theorem}
%
\begin{proof}
Assume, without loss of generality, that \pq\  does not contain a flox or 
a singularity.
(If it does, then we are done.)
By Lemma~\ref{threecross}, there exists a line L that crosses \pq\  at 
three (or more) distinct points.
Let \xo, \xt, and \xth\ be three of these points such that $\xt \in \xotha$ 
and \mbox{$\xotha \cap L = \{ \xo,\xt,\xth\}$} 
(\ie, \xo, \xt, and \xth\ are as close together as possible on \pq).
\xotha\ does not change its direction of curvature,
since there is no flox or singularity on \pq.
Moreover, \xotha\ is not a line segment, otherwise \lyne{\xo\xth}\ would be
a component of the curve (Theorem~\ref{bezout}),
contradicting the irreducibility of the curve.
Therefore, without loss of generality, we can assume 
that \xotha\ looks like Figure~\ref{2.7}(a).
Let R be the closed region bounded by \xotha\ 
and \seg{\xo\xth} (Figure~\ref{2.7}(b)).
We will show that R contains a flox or a 
singularity.
This will complete the proof, since the tangent
of a point inside R must cross $\xotha \subset \pq$ at least once.
(The tangent must cross the boundary of R twice, and at most one of these
intersections can be with 
%
\figg{2.7}{(a) \xotha\ (b) the region R}{2in}
%
\seg{\xo\xth}.)\\
%Let 
%\[ \fxo = \left\{ \begin{array}{ll}
%	\mbox{$F \setminus \xotha$}  &\ \ \ \mbox{if F is closed}\\ \\
%	\mbox{the segment of $F\setminus\xotha$}  &\ \ \ \mbox{if F is open}\\
%        \mbox{with endpoint \xo}
%	\end{array}
%	\right.   \]
%
\tab The curve lies inside of R as it leaves \xotha\ from \xo\ and outside 
of R as it leaves \xotha\ from \xth.
Therefore, the curve must cross the boundary of R after it leaves \xotha\
from \xo, either because it must join with \xth\ (if the curve is closed)
or because an infinite segment of an algebraic curve
cannot remain within a closed region (if the curve is open,
using Lemma~\ref{nonbounded}).
The curve cannot cross the \xotha\ boundary of R, since
\mbox{$\xotha \subset \pq$} is nonsingular by assumption.
Therefore, the curve must cross \seg{\xo\xth} after it leaves
\xotha\ from \xo.

%Consider \xo's tangent.
%Since \arc{PQ} crosses L at \xo, the tangent at \xo\ must cross L.
%Thus, \xt\ and \xth\ lie on opposite sides of \xo's tangent.
%In particular, as the curve leaves \xo\ along \xotha, it lies on the
%opposite side of \xo's tangent from \xth.
%Remember that the x1x3 arc only crosses x1's tangent once, since it is
%a spiral and x1,x2,x3 are as close together as possible.
%
%\hence Since \xo\ is not a flex, 
As the curve leaves \xotha\ from \xo,
it lies on the opposite side of \xo's tangent from \seg{\xo\xth}.
Therefore, after the curve leaves \xotha\ from \xo\ and before it leaves
R, the curve must cross \xo's tangent inside of R, in order to reach 
\seg{\xo\xth}.
In order to cross over \xo's tangent, the curve
must cross itself or change its curvature inside of R (Figure~\ref{2.9}),
otherwise it will spiral around inside R forever.
Therefore, R contains a singularity or a flox.
\end{proof}
%
\figg{2.9}{The curve must cross itself or change its curvature in travelling
from \xo\ to \seg{\xo\xth}}{1.75in}

\begin{corollary}
The tangents of the singularities and floxes of a curve divide the curve
into convex segments.
\end{corollary}

\begin{example}
\label{eg-2.2.2}
Figure~\ref{2.10}\ illustrates the division of a curve into
convex segments by the tangents at singularities and floxes.
\end{example}
\begin{figure}[htbp]\vspace{4.75in}
\caption{Convex segmentation of (a) limacon of Pascal (b) serpentine (c) Cassinian oval}
\label{2.10}\end{figure}

Theorem~\ref{tangents-split} establishes that
the segmentation of a curve by its singularities and the 
points where the
curve crosses a singularity/flox tangent will be a convex segmentation.
Therefore, the nonsingular points at which the curve touches (but does not
cross) a singularity/flox tangent can be ignored.
For example, the convex segment from \wo\ in Figure~\ref{2.10A} should be
\arc{\wo W_{3}} rather than \arc{\wo W_{2}}.
\figg{2.10A}{Nonsingular points where the curve merely touches a tangent are ignored}{1.5in}
%
\section{The Cell Partition and the Sorting of Convex Segments}
\subsection{The Cell Partition}

Consider a subdivision of the plane into cells by the
tangents at the singularities and floxes of a curve.
Some of the cells contain a convex segment of the curve,
some of the cells contain several convex segments of the curve,
and the rest of the cells do not contain any of the curve
(Figure~\ref{2.10}).
This subsection develops some terminology to describe this situation.

\begin{definition}
A {\bf cell partition} of a plane curve F is a partition of the plane
into convex polygons, called {\bf cells}, by the tangents at the singularities
and the floxes of F.
A line segment that forms part of the boundary of a cell 
is called a {\bf cell segment}.
A tangent of a singularity or flox of F is called a {\bf \wall} 
of the cell partition.
Consider the points where the curve intersects the cell partition.
%r different \wallpoints are assigned to a 
%singularity with r distinct tangents:
%each tangent has one \wallpoint associated with it.
%All of the r \wallpoints of a singularity share the same coordinates:
%they are only logically different.
Many of these intersections arise 
intentionally: \ie, the intersection is a singularity or a flox through
which a \wall\ of the cell partition was intentionally passed.
The other intersections arise
indirectly and are called {\bf incidental} \wallpoints.
The points of intersection of the curve with the cell partition
are collectively called {\bf curve points}.\footnote{Recall that
the nonsingular points of intersection where the curve
only touches a wall are ignored.}

The curve is decomposed into convex segments by the cell partition.
A {\bf multisegment cell} is a cell that contains more than one 
convex segment.
The endpoints of each convex segment are \wallpoints.
If \ \arc{VW} is a convex segment of the partition, then V is called
a {\bf partner} of W (and vice versa).
A \wallpoint\ separates two convex segments and thus usually has two partners.
However, a \wallpoint\ might only have one partner: a convex
segment that goes off to infinity within a cell C will have only one endpoint,
and this \wallpoint\ will not have a partner with respect to C.
\end{definition}
%
\begin{example}
The cell partition of the serpentine (Figure~\ref{2.10}(b)) has three
walls and three \wallpoints, each of which is a flox.
E and F are partners, as are F and G.
The cell partition of the Cassinian oval (Figure~\ref{2.10}(c)) 
has four \walls,
arising from the four flexes A, B, C, and D. 
There are four incidental \wallpoints: $\wo,\wt,W_{3},W_{4}$.
Cell 1 is the only multisegment cell.
\wo\ and \wt\ are partners in cell 1, as are $W_{3}$ and $W_{4}$.
\end{example}
%$A_{1},A_{2}$, B, and C are \wallpoints.\footnote{Note that we draw the two
%\wallpoints of the singularity offset along the appropriate tangent in order
%to indicate which tangent is associated with each \wallpoint.}
%B and C are incidental \wallpoints.
%Cell 1 is a multisegment cell.
%B is the partner of C with respect to cell 1.
%$A_{2}$ is the partner of C with respect to cell 4.
%$A_{1}$ is the partner of B with respect to cell 4.
%$A_{1}$ is the partner of $A_{2}$ with respect to cell 3.
%
%E has no partner in cell X.
%This cell partition has no incidental \wallpoints.

A cell partition is not simply a collection of \walls.
It is a large data structure that defines interrelationships
between cells, \walls, \wallpoints, and convex segments.
It contains information that is needed in the implementation of the
new sorting method, such as the \walls\ of each cell (in implicit and
parametric representations);  for each of these \walls, the side that the cell
lies on; the \wallpoints\ on each cell segment;
and adjacency information, such as the two cells that border a cell segment.

The creation of the cell partition of a curve is a preprocessing step that
is entirely independent of the sorting of any points on the curve.
%
\subsection{The Sorting of Convex Segments}
We complete our description of the new sorting method
by discussing
how to determine the order of the convex segments on the curve.
The sorting of the convex segments is done in an unusual manner.
In fact, to say that we `sort' the convex segments is a misnomer.
Unlike a normal sort, we do not create a list of convex segments and 
rearrange them into proper order.
Instead, the points from A to B are sorted by traversing the curve from A to
B by convex segments, stepping from \wallpoint\ to \wallpoint.
The next convex segment is determined only when we
need to move to it.\footnote{This technique is reminiscent of lazy evaluation
in compiler theory.
Since we sort the convex segments `lazily',
only those convex segments that lie on the sort segment
are encountered and sorted.}
As each convex segment is encountered, the points that lie on it are found
and sorted (using Theorem~\ref{thm-2.1}), and this subsort is appended to the
end of a global sort which is being accumulated.
Thus, the sorting of the convex segments is interleaved with the sorting
of the sortpoints.

A crucial step in the traversal of a curve by convex segments is the 
determination of the next convex segment.
Given a convex segment \arc{VW}, we must be able to find the
convex segment that follows \arc{VW}\ from W.
Suppose that \arc{VW}\ lies in cell C and W lies on the boundary
of cells C and D.
The problem of finding the convex segment that follows \arc{VW}\
from W reduces to finding the partner of W in D.
If the cell D contains only one convex segment (as is often the case),
then this is trivial.
Chapter~\ref{chap-3}\ discusses how to determine the partner of a 
\wallpoint\ in a multisegment cell.

The other main step in the traversal of a curve by convex segments is
the computation of the points that lie on a given convex segment.
It turns out that the theory that must be developed to solve this problem
is the same as for the solution of the above next-convex-segment problem.
Chapter~\ref{chap-3} discusses how to find the convex segment that 
contains a given point of the curve and, thus, 
how to find the points that lie on a given convex segment.
The first step is to find the cell that contains the point:
a point lies in a cell if and only if it lies on the proper side of each 
of the \walls\ that define the cell.
If the point's cell contains only one convex segment,
then this convex segment contains the point.
However, if the cell contains several convex segments,
then the decision is much more complicated.

The first convex segment is determined by first 
finding the convex segment \arc{VW} that contains the start point S.
Recall that, as part of the definition of a sorting problem, a vector at
S is provided to indicate the direction in which the sort is to proceed
from S.
This vector can be used to 
determine whether the first convex segment is \arc{SV}\ or \arc{SW}:
\arc{SV}\ is the first convex segment if and only if the vector points to the
inside of the chord \lyne{SV}\ (see Appendix~\ref{app-defn}'s definition).

We have now presented the fundamentals of our new method of curve sorting.
We refer to it as the convex-segment method of sorting.
%
\begin{example}
Consider the sorting of points along a Cassinian oval (Figure~\ref{2.12a}).
We determine that the startpoint S 
lies on \arc{\wt C}\ and use the vector at
S to choose the subsegment \arc{\wt S}\ as the first convex segment.
There are no points on \arc{\wt S}, so we move on.
The next convex segment is \arc{\wo\wt}, since \wo\ is \wt's partner in 
cell~1.
There are two sortpoints in cell~1 ($P_{1}$ and $P_{5}$), but only $P_{1}$
lies on \arc{\wo\wt}.
We make $P_{1}$ the first element of the sort.
We jump to the next convex segment \arc{\wo A} and
sort the two points, $P_{2}$ and $P_{3}$, that lie on this convex segment
by sorting the angles that \wo, $P_{2},\ P_{3}$, and A make with a 
central point. 
We add $P_{2}$ and $P_{3}$ to the global sort, and
move on to the next convex segment \arc{AB}.
We immediately move
on to \arc{BW_{3}}, since we find that \arc{AB}\ does not contain any
sortpoints.
Both END and $P_{4}$ lie on \arc{BW_{3}}.
The presence of END indicates that this is the last convex segment 
that needs to be considered.
Upon sorting END and $P_{4}$, $P_{4}$ is discarded because it comes after END.
The final sorted list is $P_{1},P_{2},P_{3}$.
\end{example}
\begin{figure}[htbp]\vspace{2.75in}\caption{Sorting a Cassinian oval}
\label{2.12a}\end{figure}
%
\section{Resolving the Ambiguity at Singularities}
\label{sec-resolve}

The traversal of a curve by convex segments is especially challenging in 
the neighbourhood of a singularity.
In particular, it can be ambiguous which branch of the curve should be
followed from a singular \wallpoint.
%
\begin{example}
Consider the curve in the neighbourhood of A on Figure~\ref{2.13}(a). 
It is ambiguous whether this is 
two semicircles touching or two flexes crossing.
In particular, it is not clear whether $P_{2}$ or $P_{3}$ follows $P_{1}$.

\figg{2.13}{Ambiguity about a singularity}{2.5in}

Consider the sorting of points on a loop around an ordinary 
singularity (Figure~\ref{2.13}(b)).
It is not immediately clear whether the order is \mbox{$P_{1},P_{2},P_{3}$} or
\mbox{$P_{3},P_{2},P_{1}$}.
\end{example}
%

This problem is resolved by finding,
for each branch that passes through a singularity, a pair of points,
one on either side of the singularity.
These two points serve to guide the sort through the singularity along
the proper branch.
Before we discuss how to find these points, we offer an example of how they
are used.
%
\figg{2.15}{Resolving the ambiguity at a singularity}{3in}
%
\begin{example}
\label{eg-pseudo}
We associate four points with the singularity A of Figure~\ref{2.15}(a):
$V_{1},\ V_{2},\ W_{1}, \mbox{ and } W_{2}$.
$V_{1}$ is paired with $V_{2}$ on the solid arc, and 
$W_{1}$ is paired with $W_{2}$ on the dotted arc.
When this curve is sorted, rather than traversing the curve 
from START to A, A to A, and
A to END, the traversal proceeds 
from START to $V_{2}$, $V_{1}$ to $W_{2}$, and 
$W_{1}$ to END.
Notice that if the traversal reaches a point associated with a singularity 
such as $V_{2}$, then the traversal continues from
$V_{2}$'s partner, not $V_{2}$ itself.


Suppose that we have determined that the curve of Figure~\ref{2.15}(b) 
is actually two semicircles touching at A.
We will associate four points with the singularity, as above.
The convex segments of the two cells of Figure~\ref{2.15}(b) 
are now \arc{PV_{1}}, \arc{V_{2}Q}, \arc{RW_{1}}, and \arc{W_{2}S}.
If a traversal of the curve during a sort reaches P, 
then it will proceed to $V_{1}$ and then on from $V_{2}$ to Q.
In particular, there is no danger of the traversal mistakenly
proceeding from P to A to S.
\end{example}
%

Notice that, after each singularity of the curve has been decomposed in this 
manner, every convex segment of the curve is bounded by simple points.
Care must be taken with the sortpoints that lie on any of the segments about
a singularity that are essentially sliced out, such as sortpoints that lie on
\arc{V_{1}V_{2}} in the previous example.
We shall discuss how to sort these points at the end of this section.
%
\subsection{Blowing Up The Curve at a Singularity}

For each branch that passes through a singularity, we wish to find a pair
of points on it, one on either side of the singularity.
We would like to do this by crawling a small distance along the arc in both
directions from the singularity.
However, there is no reliable way of crawling along a given arc as it passes
through a singularity. 
Therefore, we must isolate each arc of the singularity so that we can crawl
along it robustly.
We accomplish this by blowing up the curve at the singularity by a series
of quadratic transformations \cite{bhh,wa}.

Suppose that a curve has been translated so that one of its singularities
is at the origin.
Let its new equation be $f(x,y)=0$.
Consider the {\bf affine quadratic transformation} $x = x_{1},\ y = x_{1}y_{1}$
(of Cremona) \cite{wa} and the associated curve $f(x_{1},x_{1}y_{1})$.
The useful property of this transformation is that it maps the origin
to the entire $y_{1}$-axis and maps the rest of the y-axis to infinity:
$y_{1} = \frac{y}{x}$ so $(0,b)$ 
maps to $(0,\frac{b}{0})$, which is a point at
infinity unless $b=0$.
The quadratic transformation is one-to-one for all points $(x,y)$ with $x \neq 0$.
The line $y = mx$ through the origin is mapped to the horizontal line 
$y_{1}=m$:
$y=mx \rightarrow  x_{1}y_{1} = mx_{1}  \rightarrow  y_{1}=m$.
Thus, a quadratic transformation maps distinct tangent directions of the
various branches of $f$ at the singular origin to different
points on the {\bf exceptional line} $x_{1} =0$.\footnote{The quadratic
transformation does not map the line $x=0$ properly, so we must make sure
that $x=0$ is not a tangent direction to the curve at the origin.
This is done by a nonsingular linear transformation $x = \alpha\hat{x}+\beta\hat{y}$
and $y = \delta\hat{x} + \gamma\hat{y}$, such that
neither $\alpha\hat{x}+\beta\hat{y}$ nor
$\delta\hat{x} + \gamma\hat{y}$ are tangents to the curve at the origin.}
The intersections of the transformed branches with the exceptional
line correspond to the transformed points of the singularity at the 
%
\figg{2.17}{(a) node and (b) its quadratic transformation}{1.75in}
%
origin (Figure~\ref{2.17}).
If an intersection point on the exceptional line is singular, 
then the procedure is applied recursively (Figure~\ref{18}).
Hence, under quadratic transformations, the various branches of the 
curve in the neighbourhood of the singularity eventually 
get transformed to separate branches.
Once a branch is isolated, it is simple to find two points of it
on either side of the singularity, since there are no other branches
present to cause confusion. 

\begin{figure}[htbp]\vspace{5.25in}\caption{(a) the original singularity (b) after one quadratic transformation
(c) after a second transformation: the original singularity has been 
successfully transformed into two simple points}\label{18}\end{figure}
%
%\figg{19}{Another example of blowing up a singularity}
%

\begin{lemma}[{\cite{abba3,wa}}]
A finite number of applications of the quadratic transformation reduces
a singularity to a number of simple points.
\end{lemma}
%

To summarize, for each singular point of the curve, we translate 
the singularity to the origin
and apply a series of quadratic transformations until the singularity is 
transformed into a set of nonsingular points.
Each branch of the transformed curve will intersect the exceptional line
in a simple (nonsingular) point.
For each of these branches, we compute two points on either side of 
the exceptional line $x_{1}$= 0 (by crawling an 
\mbox{$\epsilon$-distance}) and  map this pair back onto
the corresponding branch of the original curve, by applying inverse
transformations.
The pair of points on each branch clarify the branch connectivity at the 
singularity  and allow a robust traversal of the curve by convex segments.

\begin{definition}
The collection of points associated with a singularity are called
{\bf pseudo \wallpoints}.
(For example, $V_{1}$, $V_{2}$, $W_{1}$, and $W_{2}$ of 
Example~\ref{eg-pseudo} are pseudo \wallpoints.)
They replace the singularity \wallpoint\ in the cell partition.
That is, for the purposes of curve traversal, 
the singularity is no longer considered to be a \wallpoint.
\end{definition}

As mentioned above, the sortpoints 
that lie close to a singularity (on one of the segments that is `sliced out')
must be treated as a special case.
These points must be sorted by mapping them to the blown-up, desingularized 
curve and using the crawling method.
%That is, if there are any sortpoints that lie in an \mbox{$\epsilon$-ball}
%about the singularity (where the \mbox{$\epsilon$-ball} is chosen so that
%it is just big enough to contain all of the points associated with the 
%singularity), then these points must be sorted by crawling.
This is not expensive because the sliced-out segment is very short and
very few jumps are needed to crawl over it.

%Once the arcs of a singularity have been isolated,
%they are nonsingular in the neighbourhood of the image of the singularity
%and it is possible to crawl along them safely.
%Let C be the original curve, S the singularity of multiplicity m,
%g(C) the desingularized curve, and $S_{1},\ldots,S_{m}$ the images of S
%on g(C).
%For each $S_{i}$, we crawl to $S_{i}^{+\epsilon}$ and $S_{i}^{-\epsilon}$
%(two points on opposite sides of $S_{i}$ and very close to $S_{i}$)
%and map these points back to the original curve.
%The collection 
%$\{g^{-1}(S_{i}^{+\epsilon}),g^{-1}(S_{i}^{-\epsilon})\}$
%are the points associated with the singularity, 
%and $g^{-1}(S_{i}^{+\epsilon})$ is paired 
%with $g^{-1}(S_{i}^{-\epsilon})$.
%These points are considered to be \wallpoints of the appropriate cell
%of C's cell partition.
%
\section{Of Flexes and Singularities}
\label{sec-flexsing}

This section explains how to find the \walls\ of a cell partition.
We start with a discussion of projective space so that we can show how to
find the singularities and flexes of a curve.
We then examine how to find the tangents of a singularity and a flex,
and how to distinguish a flox from a flex.

We have been working in {\bf affine space}, the familiar n-dimensional
Euclidean space in which points are represented as n-tuples, such as
(0,0) for the origin of the plane.
Projective space is an extension of affine space.
Consider the equivalence relation 
$ (x_{1},\ldots,x_{n+1}) \approx (y_{1},\ldots,y_{n+1})$ if and only if
there exists $t \neq 0$ such that $x_{i} = t*y_{i}$
for all i.
{\bf N-dimensional projective space} (over the field K) is the space of
equivalence classes of tuples of $K^{n+1} \setminus \{(0,0,\ldots,0)\}$
under this equivalence relation.
The point \mbox{$(x_{1},\ldots,x_{n})$} of affine space is identified with the
point \mbox{$(t*x_{1},\ldots,t*x_{n},t)$} \mbox{$= (x_{1},\ldots,x_{n},1)$} 
of projective space.
That is, points in affine space are associated with 
points of the complement of $x_{n+1} = 0$ in projective space.
The plane $x_{n+1} = 0$ of projective space is called the {\bf plane
at infinity}.
It allows the formal treatment of points and components at infinity, such
as the intersection of two curves at infinity.
The point $(x_{1},\ldots,x_{n},0)$ represents the point at infinity along
the vector $(x_{1},\ldots,x_{n})$.
Projective space is the extension of affine space by the plane
at infinity.\footnote{A more spatially oriented and evocative view of 
projective space is offered by Fulton \cite{fulton}.  
The point $(x_{1},\ldots,x_{n})$ is identified with the line in $K^{n+1}$
through $(0,0,\ldots,0)$ and $(x_{1},\ldots,x_{n},1)$.
Projective space is the collection of lines through $(0,0,\ldots,0)$
in $K^{n+1}$.
Affine space is embedded in projective space as the plane $x_{n+1} = 1$.
Two points of $K^{n+1}$ on the same line are equivalent.
The lines through $(0,0,\ldots,0)$ in the plane $x_{n+1} = 0$ correspond
to the points at infinity.}

In projective space, a plane algebraic curve is the zero set of a
{\bf homogeneous} polynomial (i.e., a polynomial whose terms are all of the
same degree) in three variables.\footnote{The equation must be homogeneous,
since solutions must be invariant under the above equivalence relation.}
The curve $F(X,Y,Z) = 0$ in projective space corresponds to the curve 
$f(x,y) := F(x,y,1) = 0$ in affine space.
Conversely, the curve $f(x,y) = 0$ of order $n$ in 
affine space corresponds to the curve $F(X,Y,Z) := Z^{n}f(\frac{X}{Z},
\frac{Y}{Z}) = 0$ in projective space, which is a homogenized version of 
$f(x,y) = 0$.
In both cases, the curve $f(x,y) = 0$ is equivalent to the curve 
$F(X,Y,Z) = 0$ without its points at infinity.
%
\begin{example}
The projective equivalent of the curve \mbox{$x^{2}+2x-4y-3=0$} is
\mbox{$x^{2}+2xz-4yz-3z^{2}=0$}.
Consider the hyperbola \mbox{$x^{2}-y^{2}-1=0$} and one of its asymptotes
\mbox{$x-y=0$}.
Their projective equivalents are\\
\mbox{$x^{2}-y^{2}-z^{2}=0$}
and \mbox{$x-y=0$}, respectively.
Solving for their intersection yields \mbox{$x=y,\ z=0$}.
Therefore, since \mbox{$(x,x,0) \approx (1,1,0)$}, the hyperbola 
intersects its asymptote at the point of infinity \mbox{$(1,1,0)$}.
\end{example}

The following lemma gives the mathematical characterization of a
singularity and a flex.
%
\begin{lemma}[{\cite[pp. 51,71]{wa}}]
\label{lem-singflex}
Let $F(x,y,z) = 0$ be the representation of a plane curve C in projective
space, and let P be a point of projective space.
\begin{quote}
\begin{enumerate}
\item P is a singularity of C if and only if
$F_{x}(P) = F_{y}(P) = F_{z}(P) = 0$
\item P is a flex of C if and only if F(P) = 0, 
P is not a singularity, and \mbox{det($F_{ij}(P)) = 0$},
\end{enumerate}
\end{quote}
%
where $F_{x}$ is the derivative of F with respect to x, etc.;
$F_{11}=F_{xx},$ etc.;
and $(F_{ij}(P))$ is a 3 by 3 matrix.
(The curve \mbox{det($F_{ij}(P)) = 0$} is called the Hessian of the 
curve $F=0$.)
\end{lemma}

Therefore, the computation of the singularities of a curve of order $n$ involves
the solution of a system of three equations of degree $n-1$.
The computation of the flexes involves the solution of
a system of two equations, one of
degree $n$ and the other of \mbox{degree $3(n-2)$}.
Resultants offer a possible method of solution (Appendix~\ref{app-defn}).
The solution of systems of equations is a well-studied problem,
so we do not elaborate further.

The following lemma gives an indication of the worst-case complexity of a 
cell partition.
%
\begin{lemma}[{\cite[pp. 65,120]{wa}}]
\label{lem-maxsing}
A curve of order n has at most $\frac{(n-1)(n-2)}{2}$ singularities.
A curve of order n 
that has no extraordinary\footnote{See Appendix~\ref{app-defn}.}
singularities of multiplicity greater
than two has at most $3n(n-2)-6\delta-8\kappa$ flexes, where
$\delta$ is the number of nodes of the curve (properly counted)
and $\kappa$ is the number of cusps.
\end{lemma}
%
These maxima are not attained simultaneously.
The maximum number of flexes occurs when there are no singularities, and
the number of flexes decreases as the number of singularities increases.

It is simple to find the tangent of a flex.
%
\begin{lemma}[{\cite[p. 55]{wa}}]
\label{tangenteqn}
If P is a nonsingular point of the plane curve \mbox{$F(x,y,z)=0$}
(in projective space), then the equation of the tangent to F at P is
\[ F_{x}(P)x + F_{y}(P)y + F_{z}(P)z = 0 \]
\end{lemma}
%
\begin{definition}
The {\bf order form} of a polynomial $f(x,y)=0$ is the homogeneous
polynomial consisting of the terms of lowest degree in f.
\end{definition}

The tangents of a singularity are found by translating the singularity to the
origin and applying the following lemma.
%
\begin{lemma}[{\cite[p. 54]{wa}}]
If the order form of $f(x,y)=0$ is of degree $r$, 
then the origin is a singularity of $f=0$ of multiplicity $r$, 
and the components of the order form are the tangents to $f$ at the origin.
\end{lemma}
%

We wish to ignore all flexes of even order, so we require a method of
computing the order of a flex.
\begin{lemma}
\label{formatofneweqn}
Let P be a nonsingular point of a plane curve.
%Using Lemma~\ref{newcurveeqn}, 
Let $f(x,y)=0$ be the equation
of the curve after it has been translated and rotated so that P is the
origin and P's tangent is the x-axis.
Then the number of intersections of P's tangent with the curve at P is
\[ \mbox{min } \{\ i\ |\ Ax^{i} \mbox{ is a term of } f(x,y), 
\mbox{ for some A} \neq 0\}\]
\end{lemma}
%
\begin{proof}
(This proof is a variation upon the discussion of \cite[p.\ 53]{wa}.)
The intersections of $f(x,y)$ with the x-axis 
(whose parameterization is \mbox{$x=t,\ y=0$})
are represented by the roots of $f(t,0)=0$.
The number of intersections of $f(x,y)$ with the x-axis at the origin is
equal to the multiplicity of the root $t=0$ in $f(t,0)=0$.
The Taylor expansion of $f(t,0)$ is 
\[ f(0,0) + f_{x}(0,0)*t + \frac{1}{2!}f_{xx}(0,0)*t^{2} +
   \frac{1}{3!}f_{xxx}(0,0)*t^{3} + \ldots \]
and the multiplicity of the root $t=0$ is the lowest nonzero power of t
in this expansion.
That is, the multiplicity of intersection of $f(x,y)$ with the x-axis at
the origin is 
\[ \mbox{min } \{\ i\ |\ (\frac{df}{dx^{i}})(0,0) \neq 0 \} \]
\[ = \mbox{min } \{\ i\ |\ Ax^{i} \mbox{ is a term of } f(x,y), 
\mbox{ for some A} \neq 0\}\]
\end{proof}
%
\section{Space Curves}
\label{sec-spacecurve}

The convex-segment 
method has been presented as a technique for sorting plane curves.
Indeed, its reliance upon a cell partition of the curve's plane suggests that
it can only be used to sort plane curves.
However, we shall now show that space curves can also be sorted by the new 
method.

We solve a sorting problem for a space curve by mapping it to a related
sorting problem for a plane curve.
The plane curve is derived by projecting the space curve.
%
\begin{definition}
Let B be a plane and let q be a point, $q \notin B$.
The {\bf (central) projection of a point} p (with respect to q and B)
is the point of intersection of the line $\lyne{pq}$ with B.
The {\bf (central) projection of a space curve} C (with respect to q 
and B) is the plane curve generated by the projections of the points of C.
B is the {\bf plane of projection} and q is the 
{\bf center of projection}.
\end{definition}
%
\begin{lemma}
\label{lem-sec5-project}
Let $z=0$ be the plane of projection and let $(x_{c},y_{c},z_{c})$
be the center of projection ($z_{c} \neq 0$).
The projection map $P:\Re^{3} \rightarrow \Re^{2}$ is 
\[ P(x_{3d}, y_{3d},z_{3d}) = (\frac{x_{c}z_{3d} - x_{3d}z_{c}}{z_{3d}-z_{c}},
                             \frac{y_{c}z_{3d} - y_{3d}z_{c}}{z_{3d}-z_{c}})\]
The inverse map $P^{-1}:\Re^{2} \rightarrow \Re^{3}$ is 
\[P^{-1}(x_{2d},y_{2d})=(\frac{x_{2d}(z_{c}-z_{3d})+x_{c}z_{3d}}{z_{c}},
                              \frac{y_{2d}(z_{c}-z_{3d})+y_{c}z_{3d}}{z_{c}},
                              z_{3d}) \]
The inverse image of $(x_{2d},y_{2d})$ must be expressed
in terms of $z_{3d}$, since infinitely many points project to a given
point on the plane (viz., all of the points on the line between 
$(x_{2d},y_{2d},0)$ and $(x_{c},y_{c},z_{c})$).
\end{lemma}
%
\begin{proof}
The line through $(x_{c},y_{c},z_{c})$ and $(x_{3d},y_{3d},z_{3d})$ 
can be parameterized by 
$((x_{3d}-x_{c})t + x_{c},(y_{3d}-y_{c})t+y_{c},(z_{3d}-z_{c})t+z_{c})$.
The projection of $(x_{3d},y_{3d},z_{3d})$ lies on this line 
and has a z-coordinate of 0.  
Therefore, it corresponds to the parameter value $\hat{t}$
where $(z_{3d}-z_{c})\hat{t} + z_{c} = 0 \Rightarrow \hat{t} = 
\frac{-z_{c}}{z_{3d}-z_{c}}$.
Therefore, 
\[ \mbox{proj }(x_{3d},y_{3d},z_{3d}) 
= ((x_{3d}-x_{c})(\frac{-z_{c}}{z_{3d}-z_{c}}) + x_{c},\ 
   (y_{3d}-y_{c})(\frac{-z_{c}}{z_{3d}-z_{c}}) + y_{c},\ 0) \]
\[ = (\frac{z_{c}(x_{c}-x_{3d}) + x_{c}(z_{3d}-z_{c})}{z_{3d}-z_{c}},
\ \frac{z_{c}(y_{c}-y_{3d}) + y_{c}(z_{3d}-z_{c})}
   {z_{3d}-z_{c}},\ 0) \]
\[ = (\frac{x_{c} z_{3d} - z_{c} x_{3d}}{z_{3d}-z_{c}},
\ \frac{y_{c} z_{3d} - z_{c} y_{3d}}{z_{3d} - z_{c}},\ 0) \]
%
The inverse map can be derived by solving for $x_{3d}$ and $y_{3d}$
in the equations
\[ x_{2d} = \frac{x_{c}z_{3d} - x_{3d}z_{c}}{z_{3d}-z_{c}},
\ y_{2d} = \frac{y_{c}z_{3d} - y_{3d}z_{c}}{z_{3d}-z_{c}} \]
\end{proof}

\begin{corollary}
\label{cor-chap2-sec5}
Let $z=0$ be the plane of projection and let $(x_{c},y_{c},z_{c})$
be the center of projection ($z_{c} \neq 0$).
Let A be the plane $z = z_{c}$.
The projection map $P:\Re^{3} \rightarrow \Re^{2}$ is continuous on 
$\Re^{3} \setminus A$.
(That is, \mbox{$lim_{\alpha \rightarrow 0}\ P(X+\alpha) = P(X)$} for
$X \in \Re^{3} \setminus A$.)
\end{corollary}
\begin{proof}
If f(x), g(x), and h(x) are polynomials, then the rational map
$(\frac{f(x)}{h(x)},\frac{g(x)}{h(x)})$ is continuous at all points where 
the denominator is not zero.
\end{proof}
%
\begin{definition}
Order on the segment SEG is {\bf preserved} by the projection map
\mbox{proj:$\Re^{3} \rightarrow \Re^{2}$}
if proj(SEG) is connected and,
\mbox{for all $P_{1},P_{2},P_{3} \in $ SEG},
proj($P_{2}$) lies in between proj($P_{1}$) and proj($P_{3}$)
whenever $P_{2}$ lies in between $P_{1}$ and $P_{3}$.
\end{definition}

We want to choose a projection that preserves order on the sort segment
of the space curve.
We do this by ensuring that the projection map is continuous on the 
entire sort segment.
In the rest of this section, F is a space curve defined by 
the intersection of the two
%\footnote{As discussed in 
%Appendix~\ref{app-defn}, we only deal with space curves that are
%defined by the intersection of two surfaces.}
(affine) surfaces 
$f_{1}(x,y,z) = 0$ and $f_{2}(x,y,z) = 0$,
and \arc{PQ} is a finite segment of F (viz., the sort segment).
%
\begin{lemma}
\label{thm-orderpreserve}
Let \mbox{proj$_{C}$} be the projection map for the projection onto the plane
$z=0$ from the center of projection C.
There exists a point $C \in \Re^{3}$ such that order on \arc{PQ}\ is 
preserved by the projection \mbox{proj$_{C}$}.
\end{lemma}
%
\begin{proof}
Let $\alpha = \mbox{max} \{\ z' \mid (x',y',z') \in \arc{PQ} \}$.
$\alpha$ exists because \arc{PQ}\ is a finite segment.
Choose C = $(x_{C},y_{C},z_{C})$ so that $z_{C} > \alpha$.
proj$_{C}$ will be continuous on \arc{PQ}, by 
Corollary~\ref{cor-chap2-sec5}.
proj$_{C}(\arc{PQ})$ is connected, since it is the continuous image of a 
connected segment.
Let $P_{1},P_{2},P_{3}$ be points of \arc{PQ}\ such that $P_{2}$ lies
in between $P_{1}$ and $P_{3}$.
The continuity of proj$_{C}$ ensures that 
$\mbox{proj}_{C}(P_{2})$ lies in between proj$_{C}(P_{1})$ and 
proj$_{C}(P_{3})$.
\end{proof}

In solid modeling applications, \arc{PQ}\ will usually be the finite
segment of a space curve that defines an edge of a solid model.
The model will be bounded and, in particular, it will be bounded in the z 
direction.
Therefore, the 
center of projection can be chosen above this bound to guarantee
the continuity of the projection map on the sort segment, and thus
the preservation of order on the projected sort segment.
In the rest of this section, let $z=0$ be the plane of projection and
let $(x_{c},y_{c},z_{c})$ be the center of projection, chosen so that the
projection map is continuous on \arc{PQ}.

In order to apply the convex-segment 
method to the projection proj(\arc{PQ}) of the
sort segment, we must know the implicit equation 
of the irreducible 
component of the projection that contains proj(\arc{PQ}).
The following lemma shows how this equation 
%can be determined from the equation \mbox{$f_{1}=0 \cap f_{2}=0$}
can be determined from the equation $f_{1}$=0 $\cap$ $f_{2}$=0
of the original space curve.  
It makes use of the resultant of a pair of polynomials, which is defined
in Appendix~\ref{app-defn}.
%
\begin{lemma}
\label{lem-sec5-planeeqn}
The projection of F is contained in the plane curve defined by
the resultant $R(x,y)$ of 
\( g_{1}(x,y,z) := f_{1}(\frac{x(z_{c}-z)+x_{c}z}{z_{c}},
                              \frac{y(z_{c}-z)+y_{c}z}{z_{c}},
                              z) \) and
\( g_{2}(x,y,z) := f_{2}(\frac{x(z_{c}-z)+x_{c}z}{z_{c}},
                              \frac{y(z_{c}-z)+y_{c}z}{z_{c}},
                              z) \) with respect to z.
Moreover, the projection of \arc{PQ}\ is contained in a 
connected component of the curve $R(x,y) = 0$.
\end{lemma}
%
\begin{proof}
Let $\beta = (x_{2d},y_{2d},0)$ be a point of the projection of F.
By Lemma~\ref{lem-sec5-project}, the point of F that projects into 
$\beta$ is 
$(\frac{x_{2d}(z_{c}-z_{3d})+x_{c}z_{3d}}{z_{c}},
                              \frac{y_{2d}(z_{c}-z_{3d})+y_{c}z_{3d}}{z_{c}},
                              z_{3d})$, for some $z_{3d}$.
That is, there exists $z_{3d}$ such that
\[ f_{1}(\frac{x_{2d}(z_{c}-z_{3d})+x_{c}z_{3d}}{z_{c}},
                              \frac{y_{2d}(z_{c}-z_{3d})+y_{c}z_{3d}}{z_{c}},
                              z_{3d}) = 0 \]
and
\[ f_{2}(\frac{x_{2d}(z_{c}-z_{3d})+x_{c}z_{3d}}{z_{c}},
                              \frac{y_{2d}(z_{c}-z_{3d})+y_{c}z_{3d}}{z_{c}},
                              z_{3d}) = 0 \]
Therefore, for every point \mbox{$(x_{0},y_{0})$} of the projection of F,
there exists $z_{0}$ 
such that \mbox{$g_{1}(x_{0},y_{0},z_{0}) = g_{2}(x_{0},y_{0},z_{0}) = 0$}.
Thus, by Lemma~\ref{resultant}, for every point \mbox{$(x_{0},y_{0})$} of the 
projection, \mbox{$R(x_{0},y_{0}) = 0$}.
Therefore, \mbox{$R(x,y)=0$} contains the projection of F.
Since the projection map is continuous on \arc{PQ} and the continuous
image of a connected set is 
connected,
the projection of \arc{PQ}\ is connected.
\end{proof}

The problem of sorting points along a space curve has now been successfully
reduced to a problem of sorting points along a plane curve.
The equation of the plane curve is computed by the method of 
Lemma~\ref{lem-sec5-planeeqn},
and the sortpoints, start point, and end point are the projections of their
counterparts on the space curve.

Let A be the plane that contains the center of projection and lies parallel
to the plane of projection.
By the choice of the center of projection (Lemma~\ref{thm-orderpreserve}),
all of the sort segment lies on
the same side of A; and a sortpoint will lie on the same connected 
component as the sort segment if and only if it lies on this side of A.
Therefore, 
since we want all of the projected sortpoints to lie on the same connected
component of the projected plane curve (Section~\ref{a-chap1-sec}),
we discard all sortpoints on the space curve
that lie on the opposite side of A from the sort segment.
These sortpoints do not lie on the sort segment, so they can be safely
ignored.

If the resultant of Lemma~\ref{lem-sec5-planeeqn}
is factored into irreducible components (perhaps by the method of \cite{berlekamp}
or \cite{wang}),
then it is simple
to determine the component associated with the projection of the sort segment.
Let proj(P) be the projection of a sortpoint of the space curve
(one that has not been discarded).
The desired component is the unique one that contains proj(P).

It is possible that a sortpoint of the space curve could 
project into a singularity.
This must be avoided because the sorting of a set of points that
includes singularities can be ambiguous (Section~\ref{a-chap1-sec}).
Therefore, if a projected sortpoint proj(P) is discovered to be a 
singularity when the singularities and flexes of the projection are
computed, then proj(P) is offset along the appropriate branch 
of the singularity
by crawling a short distance along the space curve from the original 
sortpoint P to a point $P_{\epsilon}$.
That is, if proj(P) is a singularity, then we map P to 
proj($P_{\epsilon}$), which is not a singularity.
(The crawl is stable because P cannot be a singularity.
Care must be taken to make the crawl on the space curve short so that
it does not crawl over another sortpoint, since this would disrupt the order
of the sortpoints.)

Similarly, if two sortpoints $S_{1}$ and $S_{2}$ of the space curve
project to the same point of the projection, then $S_{1}$ should be mapped
to \mbox{proj$(S_{1}^{\epsilon})$} and $S_{2}$ to 
\mbox{proj$(S_{2}^{\epsilon})$}, where $S_{i}^{\epsilon}$ is found by
crawling a short distance from $S_{i}$.
(Care must be taken that $S_{1}^{\epsilon}$ and $S_{2}^{\epsilon}$
do not still map to the same point.)

The computation of the resultant 
of Lemma~\ref{lem-sec5-planeeqn} and its subsequent factorization 
(in order to find the irreducible component that contains the projection)
are expensive operations.
The expense of the factorization is the lesser problem,
since the resultant is often already irreducible.
We conclude that although the convex-segment 
method can indeed be used to sort points on a space curve,
space curve sorting is considerably more expensive than 
plane curve sorting.

Since it can be difficult to find the parameterization of a space curve,
the parameterization method may also decide to sort the projection of the
space curve rather than the space curve itself.
Recall that a rational parameterization of the space curve $S_{1} \cap S_{2}$
is derived from a low degree, rational parameterization of $S_{1}$ or $S_{2}$.
If there is no such parameterization (or no such parameterization
is computable by a known algorithm),
then the only recourse may be to look for a 
parameterization of the projection and sort it instead.
Therefore, the expense of projecting the space curve to a plane curve
may have to be absorbed by the parameterization method as well.
%
\section{A Broad Comparison of the Methods}

We have been introduced to three methods of sorting points along an
algebraic curve.
The crawling method sorts the points by making short jumps along the curve.
The parameterization method observes that the sorting of points 
on a line is simple and
tries to unwind the curve into a line by parameterizing it.
The convex-segment method borrows from both of these methods.

Like the crawling method, the convex-segment method 
leaps from one point to another along the curve
(viz., from a \wallpoint\ to its partner).
However, its jumps are large while the crawling method's jumps must be
very small.
Moreover, once the partner of each \wallpoint\ of the cell partition has been
computed (which can be done once and for all in a preprocessing step),
each jump of the convex-segment method can be
done very quickly; whereas, the crawling method 
must grope for some time (by applying
Newton's method) to find the destination of each jump.
In short, the convex-segment method 
makes large, bold jumps while the crawling method makes small, timid ones.

The convex-segment method is similar to the parameterization method because
they both reduce the sorting problem to an easier one.
However, rather than trying to reduce the entire problem 
(from sorting points
on a curve to sorting real numbers), the convex-segment method
divides the problem up into many
smaller ones and reduces each one of these (from sorting points on a convex 
segment to sorting the angles those points make with a central point).
We shall see that the many small reductions of the convex-segment method
can be done more quickly than the single, large reduction of
the parameterization method.

