\chapter{The Sorting of Points}
\section{Introduction}
\label{chap1-sec1}

The area of geometric modeling is concerned with the creation of 
computationally efficient models 
of solid physical objects to facilitate their design, assembly, and analysis.
Geometric models of physical objects  are needed in many disciplines, including robotics,
computer vision, computer-aided design/computer-aided 
manufacturing, and graphics.
In a geometric modeling system,
a solid such as a robot hand or a coffee cup is modeled by a collection
of points, curves, and surfaces.
The creation and manipulation of a solid model requires 
a variety of high-level operations, such as the combination of models by 
the boolean operations of union, intersection, and
difference; the addition of blending surfaces to a model to smooth off sharp
connections between edges and faces; the detection of interference between
models; and the pictorial display of a model.
The implementation of these high-level operations requires a
number of basic tools,
such as finding tangents to curves and surfaces; 
parameterizing lines, curves, and surfaces; 
measuring distances; and finding roots of equations. 
The development of efficient ways to perform these fundamental
tasks will benefit all of the applications that use them.

The sorting of points along an algebraic curve, as evidenced by its
numerous applications, is a basic tool for the manipulation of
geometric models.
Curve sorting has a natural definition.
If S is a set of points on a curve and \arc{AB}\ is a segment of this curve,
then to sort the points of S from A to B along \arc{AB} 
means to put them into the
order that they would be encountered in travelling continuously from A to B 
along \arc{AB} (Figure~\ref{1.1}).
Any of the points that do not lie on \arc{AB}\ are ignored.
%
\figg{1.1}{The sorted order from A to B is III, II, IV}{1.5in}
%

Curve sorting has many applications in geometric modeling.
We offer two examples and postpone the discussion of
further applications until Chapter~\ref{chap-5}.
An edge of a solid model is defined by a 
curve and a pair of endpoints.
A point lies on the edge if and only if it lies on the curve and 
between the endpoints.
The problem of deciding 
whether a point lies between two other points is a sorting 
problem.
A more elaborate application of sorting arises 
in computing the intersection of two solid models. 
An important step of this computation
is to find the segments of an edge of one model that lie in the intersection.
This is done by finding and sorting the points of intersection 
of this edge with a face of the other model. 
The segments of the edge between the $i^{th}$ and $i+1^{st}$ intersections,
for $i$ odd, are contained in the intersection of the models.

There is no serious study of curve sorting in the literature.
This can be explained by the fact that,
until recently, almost all of the curves in solid models were linear
or quadratic, and the sorting of a curve of these low degrees is 
trivial (Corollary~\ref{easyconic}).
However, as the science of geometric modeling matures and grows more
ambitious, curves of degree three and higher are becoming quite common.
For example, the introduction of blending surfaces
into a model creates curves and surfaces of high degree.
Therefore, the sorting of points along an edge of a solid model has become
an important and difficult problem.

The lack of a study of curve sorting can also be explained 
by the presence of an obvious method for sorting points.
This obvious method, which uses a parameterization of the curve, a tool
with which the geometric modeler is familiar, tends to obviate a
search for any other method.

This thesis will present a thorough investigation of curve sorting.
We will show that there are at least three methods for sorting
points along an algebraic curve.
One of these methods is entirely new and particularly appealing.
It will be shown to be the method of choice in many situations.
The thesis is organized as follows.
The next section gives a formal definition of curve sorting, and the
remainder of the chapter discusses the two conventional methods of
sorting.
Chapters~2 and~3 present our new method.
Chapter~4 is devoted to an analysis of the complexity of the new
method, experimental results, and a comparison of the three methods.
Applications of curve sorting, future research, and conclusions can
be found in Chapter~5.
Finally, there are three appendices, for definitions, technical lemmas,
and a survey of parameterization algorithms.
%
\section{The Sorting of Points Along a Curve}
\label{a-chap1-sec}

In this section, we give a formal
explanation of the sorting of points along a curve.
Let C be an irreducible,\footnote{Terms such as these
are defined in Appendix~\ref{app-defn}.} algebraic
curve described by a polynomial \( f(x,y) = 0 \) or the
intersection of two polynomial surfaces \( f_{1}(x,y,z) = 0 \) and
\( f_{2}(x,y,z) = 0 \).
(The primary representation of curves and surfaces in many solid modelers
is the implicit equation.
This representation is convenient for deciding if a point lies on the
curve or surface, and for applying techniques from algebraic geometry,
as in the creation of blending surfaces~\cite{hof-hop}.)
Let \arc{AB}\ be a segment of C, such that both A and B are
nonsingular points.
If $S \subset C$ is a set of nonsingular points on the curve, then
(as we have said) to sort the
points of S from A to B along \arc{AB} means to put the points into the
order that they would be encountered in travelling continuously
from A to B along \arc{AB}.
Points of S that do not lie on \arc{AB} are never encountered
and are thus ignored.
In order to avoid confusion, a vector at A is provided to indicate the
direction in which the sort is to proceed from A.
This is especially important when the curve is closed, since
there are then two segments between A and B to choose from.

A is called the {\bf start point}, B the {\bf end point}, and 
\arc{AB}\ is the {\bf sort segment}.
The points that are to be sorted (the points of S) are called the
{\bf sortpoints}.
We shall often refer to the sorting of points along a curve 
as {\bf curve sorting}.

The sorting of points along a curve 
is more sophisticated than the sorting of numbers or names.
In particular, start and end points are necessary, and the sorted set is often
a strict subset of the unsorted set.
These changes are necessary in order to resolve the ambiguity of sorting 
on a closed curve, where order is cyclic.
The changes are also useful for geometric modeling applications, since only
a part of the curve (viz., the edge) is of interest.

\begin{figure}[htbp]\vspace{4.5in}\caption{Some sorts are ambiguous}\label{1.1.5}\end{figure}
%\figg{1.1.5}{}

Every curve in this thesis can be assumed to be both algebraic 
and irreducible.
We also assume that all of the sortpoints of a sorting problem
lie on the connected component
that contains the sort segment, since the order of a set of points 
is unclear if the points lie on different connected 
components (Figure~\ref{1.1.5}(a) and Section~\ref{manycomponents}).
Finally, we assume that the start point, the end point, and the
sortpoints are nonsingular points,
because the order of a set of points that includes singularities
can be ambiguous (Figure~\ref{1.1.5}(b-c)).
%
%

%When curve sorting is applied to geometric modeling,
%the sort segment will be an edge of a model.
%A typical edge of a solid model will be smooth
%and will not cross itself.
%For example, the segment \arc{CE}\ of Figure~1B would be split
%into two edges, \arc{CD}\ and \arc{DE}, and 
%the segment \arc{AB}\ of Figure~1A would not be an edge.
%This explains why the sort segment is nonsingular.
%
%\fig{1A}{chap-1}
%\fig{1B}{chap-1}
%
\section{The Parameterization Method}
\label{sec-param}

A natural way to sort points along a curve is to use a rational
parameterization of the curve (i.e., a parameterization \mbox{($x(t)$,$y(t)$)}
or \mbox{($x(t)$,$y(t)$,$z(t)$)}
such that $x(t)$, $y(t)$, and $z(t)$ 
can each be expressed as the quotient of two polynomials in $t$).
The parameter values $t_{i}$ of the points $(x_{i},y_{i})$ are computed and sorted by increasing $t_{i}$ values. 
The points 
that occur before the start point or after the end point of the sort segment
are discarded.
%
\begin{example}
\label{pedalPRAM}
Consider the curve \( x^{3} + xy^{2} + 16x^{2} - 4y^{2} \) (a pedal of 
a parabola) with start point \( A = (1,\sqrt{\frac{17}{3}}) \),
end point \( B = (3,-3\sqrt{19}) \), sortpoints 
\( I = (3,3\sqrt{19}) \), 
\( II = (-1,\sqrt{3}) \),
\( III = (-2,-2\sqrt{\frac{7}{3}}) \), and
\( IV = (1,-\sqrt{\frac{17}{3}}) \),
and direction $V_{A}$ from A, as shown in Figure~\ref{1.1}.
A parameterization of the curve is
\( x = \frac{4t^{2} - 16}{t^{2}+1} \),
\( y = \frac{4t^{3} - 16t}{t^{2}+1} \).
We compute a point d on the curve close to A in the direction 
$V_{A}$.\footnote{d is found by crawling 
from A, a technique that is described in Section~\ref{sec-crawl}.}
The parameter value of d indicates the order (increasing or decreasing)
in which the parameter values should be sorted: the values are sorted
in increasing order if and only if the parameter value of d is greater 
than that of A.
For this example, \( d = (0.9,\sqrt{\frac{13.689}{3.1}}) \).
The parameter value associated with 
\mbox{\( I = (3,3\sqrt{19}) \)} is determined by solving the system
\mbox{\( \{ 3 = \frac{4t^{2} - 16}{t^{2}+1}, 
3\sqrt{19} = \frac{4t^{3} - 16t}{t^{2}+1} \} \),}
yielding \mbox{\( t = \sqrt{19} \)}.
The parameter values
\mbox{[II\ ,\ $t = -\sqrt{3}$],}
\mbox{[III\ ,\ $t = \sqrt{\frac{7}{3}}$],}
\mbox{[IV\ ,\ $t = -\sqrt{\frac{17}{3}}$],}
\mbox{[A\ ,\ $t = \sqrt{\frac{17}{3}}$],}
\mbox{[B\ ,\ $t = -\sqrt{19}$],} and
\mbox{[d\ ,\ $t = \sqrt{\frac{16.9}{3.1}}$]}
are computed in an analogous fashion.
After sorting by parameter values, the point list becomes 
B, IV, II, III, d, A, I.
Sorting from A to B in the appropriate direction yields the
desired sorted list III, II, IV.
\end{example}
%

There are three reasons that we insist that the curve's 
parameterization be rational.
Consider a parameterization that involves $n^{th}$ roots,
such as \mbox{$x(t) = \sqrt{t} + 2 \sqrt[3]{t}$}.
In solving such a parameterization for the parameter value of a
given point, it can be unclear which $n^{th}$ root to use.
Another inconvenient property of non-rational parameterizations is
that their representation is difficult.
They cannot be represented by the coefficients of the parameterization:
they must be represented symbolically (as in a computer algebra system like
MACSYMA), which is less efficient.
For example,
a parameterization of the devil's curve
$y^{4} - x^{4} - y^{2} + 4x^{2} = 0$ is 
\[ x = \mbox{cos}(t)\sqrt{\frac{sin^{2}(t)-4cos^{2}(t)}{sin^{2}(t) - cos^{2}(t)}},
\ y=\mbox{sin}(t)\sqrt{\frac{sin^{2}(t)-4cos^{2}(t)}{sin^{2}(t) - cos^{2}(t)}}\]
%
Finally, there is no algorithm for the automatic parameterization
of curves that do not have a rational parameterization.
Thus, there are problems with all three aspects of a non-rational
parameterization: computation, representation, and solution.
%
\subsection{Parameterization}
\label{subsec-par}

%
\begin{definition}
A curve is {\bf rational} if it has a rational parameterization.
\end{definition}

The translation of an implicit representation of a curve into a parametric
representation, which is one of the key steps of the parameterization method
of sorting, has received some attention in the literature.
There are constructive methods for the parameterization of plane curves of low
order (viz., two and three) \cite{abba1,abba2,hopcroft-hoffmann}
and rational plane curves \cite{abba3}.
There are also constructive methods for the parameterization of surfaces
of low degree (again, two and three) \cite{abba1,abba2,levin76}, but not 
of high degree even if the surface has a rational parameterization.
(The parameterization of surfaces is important because, as shown below,
a space curve's parameterization can be developed from 
a parameterization of one of the surfaces that defines the space curve.)
Appendix~\ref{app-D} presents various 
methods for the parameterization of low degree
curves and surfaces.

A rational plane curve of order $n$ can be parameterized by
establishing a one-to-one correspondence 
between the points of the curve and a one-parameter family of curves 
of degree max($n-2$,1) through well-chosen single and
double points of the curve \cite{abba3}.
%
\begin{example}
Consider the parameterization of the circle \( x^{2}+y^{2}-1=0 \).
Let P be a point of the circle (say $P = (-1,0)$)
and let $L_{t}$ be the line through P of slope t.
There is a one-to-one correspondence between the lines through P and the
points of the circle, which can be used to construct a parameterization:
the parameter value of a point Q is the slope of the line through P that
hits Q.
The equation of $L_{t}$ is \( y = tx + t \).
The point of the circle associated with $L_{t}$ satisfies 
{ \( y = tx + t \mbox{ and } x^{2} + y^{2} - 1 = 0 \)}, so
\( x^{2} + (tx+t)^{2} - 1 = 0 \).
Using the quadratic formula, \( x = -1 \mbox{ or } \frac{1-t^{2}}{1+t^{2}} \).
The latter root is clearly the one of interest.
Therefore, 
\[ y = tx + t = t(\frac{1-t^{2}}{1+t^{2}})+t = \frac{2t}{1+t^{2}} \]
and a parameterization of the circle is
$(\frac{1-t^{2}}{1+t^{2}},\frac{2t}{1+t^{2}})$.
\end{example}
%

A space curve's parameterization can sometimes be derived from a 
parameterization of one of its two constituent 
surfaces~\cite{hopcroft-krafft,levin79}.
We illustrate the method with an example.
%
\begin{example}
Consider the space curve $S \cap T$, where 
$S \equiv x^{2} + y^{2} - 1 = 0$ and
$T \equiv 2x^{2} + y^{2} - z^{2} = 0$ are surfaces.
We find a parameterization for S,
$\{ x = \frac{1-t^{2}}{1+t^{2}}, y = \frac{2t}{1+t^{2}}, z=s\}$, and
substitute these parametric equations into the implicit equation for T, 
yielding $\frac{2(1-t^{2})^{2} + 4t^{2}}{(1+t^{2})^{2}} - s^{2} = 0$.
We then solve for s in terms of t: $s = \frac{\sqrt{2t^{4}+2}}{1+t^{2}}$.
We conclude that a parameterization for 
the intersection of the two surfaces is 
$\{ x = \frac{1-t^{2}}{1+t^{2}}, y = \frac{2t}{1+t^{2}}, 
z= \frac{\sqrt{2t^{4}+2}}{1+t^{2}}\}$.
Notice that this parameterization is not appropriate for sorting, 
because it is not rational.
\end{example}
%
Unfortunately, it may be hard or impossible to find a parameterization
for either of the surfaces, and it may be 
impossible to solve for s in terms of t (and vice versa) after
substituting the parametric equations of one surface into the implicit
equation of the other.
In particular, problems will arise when the degree of s is five or more,
since there is no general formula for the solution of equations of degree 
greater than four~\cite{hernstein}.
Therefore, this technique of space curve parameterization is quite 
restricted.

We conclude that plane curves of low degree, rational plane curves, and
a restricted class of rational space curves can be automatically
parameterized.
%
\subsection{Weaknesses of the Parameterization Method}

The two main steps of the parameterization method of sorting are to find
a rational parameterization of the curve and to find the parameter
values of the sortpoints by solving this parameterization.
We shall show that there are problems with both of these steps.

Only a strict subclass of 
algebraic curves are rational, and the proportion
of algebraic curves that are rational decreases as the order of 
the curve increases.
Therefore, there are many curves that 
cannot be sorted by the parameterization method, simply because they do not
have a rational parameterization.
These facts are established by considering the genus of a curve.
%
\begin{definition}
The {\bf genus} of an irreducible, algebraic, plane curve is
\[ \frac{(n-1)(n-2)}{2} - \sum\frac{r_{i}(r_{i}-1)}{2}\]
where n is the order of the curve,
the sum is over the singularities $P_{i}$ of the curve,\footnote{Singularities
must be counted properly and singularities at infinity must be included.
See \cite[pp. 80-84]{wa}.}
and $r_{i}$ is the multiplicity of the curve at the singularity $P_{i}$.
The genus is nonnegative, and it
is zero if and only if 
the curve has the maximum number of singularities allowed
for a curve of its order.
\end{definition}
%
\begin{theorem}[{\cite[p. 180]{wa}}]
\label{thm-rational}
A plane curve is rational if and only if its genus is 0.
\end{theorem}
%
\begin{example}

It can be shown that any irreducible quartic (degree four) 
space curve that is generated by the intersection
of two degree two surfaces is non-rational.

The genus of any quartic plane curve with an ordinary double point
is 1, so no curve of this type has a rational parameterization.
For example, the lemniscate of Bernoulli
\( (x^{4} + y^{4} + 2x^{2}y^{2} - 4x^{2} + 4y^{2} = 0) \) 
is such a curve, and its parameterization is
\mbox{\( x = \frac{2cos(t)}{1+sin^{2}(t)} \)},
\mbox{\( y = \frac{2sin(t)cos(t)}{1+sin^{2}(t)} \)},
\mbox{\( -\pi \leq t \leq +\pi \)} \cite{lawrence}.
\end{example}

Even if a rational parameterization for the curve can be found, it is still
necessary to solve for the parameter values associated with the set of points
that are to be sorted.
This is another weakness of the parameterization method,
because the solution of a polynomial of high degree is usually expensive.
Even for the tame example of the 
pedal of a parabola (Example~\ref{pedalPRAM}),
%\( x^{3} + xy^{2} + 16x^{2} - 4y^{2} \) (Example~\ref{pedalPRAM}),
which has a parameterization involving degree three polynomials,
each solution for the parameter value of a sortpoint consumes
on the order of 120 milliseconds.\footnote{Using Common Lisp on a 
Symbolics Lisp Machine, and the ZEROIN algorithm of Dekker and Brent
for solving nonlinear equations \cite{forsythe}.}
%Used rootpkg-zeroin on pedal of parabola to get results.
We shall see in Chapter~\ref{chap-results} that this is a nontrivial expense.
%Possible comment: it seems that the computation of any radical (square root,
%cube root, ...) is expensive.
We conclude that the parameterization method of sorting is limited and slow.

There are two fundamental ways of representing a curve: the implicit equation
and the parameterization. 
The difficulties that arise with the 
parameterization method of sorting reflect the difficulty of working
with the parameterization representation of a curve in an environment
where the implicit equation is the original representation.
%
\section{The Crawling Method}
\label{sec-crawl}

Another method of sorting points along a curve is the crawling method.
%(This method is inherently inefficient, so it can be essentially ruled
%out as a viable method.
%However, we present it for two reasons:
%to discourage its use by revealing its weakness, and to explain how to
%crawl from a point to another close to it, a technique that 
%is useful as a subroutine
%in both of the other sorting methods.)
This is a brute-force method that sorts a set of points by tracing along
the sort segment and recording the order in which the sortpoints
are encountered during this trace.
The curve is traced by making small jumps along it, using
Newton's method.
For example, consider a small jump of size $\epsilon$ 
from the point \( P = (x_{0},y_{0}) \) of the 
curve \( f(x,y) = 0 \).
Depending upon the behaviour of the curve in the neighbourhood
of P, $x_{0}$ or $y_{0}$ is incremented or decremented by $\epsilon$.
Suppose that $x_{0}$ is incremented, effectively jumping off of the curve to 
\( (x_{0} + \epsilon\ ,\ y_{0}) \).
A root $y'$ of \( f(x_{0}+\epsilon,y) \) is found by applying
Newton's method, with initial guess $y_{0}$.
\( (x_{0}+\epsilon,y')\) is a point of the curve that
lies close to $(x_{0},y_{0})$, but it is a step
further along the curve from the original point.

%
\figg{1.2}{Tracing a curve from P to Q}{1.5in}
%

Progress is made along the curve by these small jumps.
For example, in Figure~\ref{1.2}, a tracing of the curve from P to Q 
may involve jumping to $P_{1}$, $P_{2}$, $P_{3}$, and $P_{4}$.
The details of tracing along a curve, including a discussion of how to trace
robustly through a singularity, are presented in \cite{bhh}.
%
%WE MAY *CANCEL* THE TECHNICAL DISCUSSION OF THE CRAWLING METHOD ALTOGETHER.
%We shall postpone the technical results related to the crawling method, 
%such as
%how one makes the choice between the four possibilities of step (ii) and 
%exactly how Newton's method works, until Appendix~\ref{chap-technical}.
%


If a jump is made to a point within some ball of radius~$\delta$ about a
sortpoint $x$, then we assume that the trace has reached $x$ and we insert
$x$ into the sorted list that is being accumulated.
If $\delta$ is small, then this is a reasonable assumption.
However, this assumption is not entirely robust (Figure~\ref{1.2A}).
%
\figg{1.2A}{x may be sorted improperly by the crawling method}{1.5in}
%

The major weakness of the crawling method is that it must make small jumps, 
in order to ensure that the crawl proceeds smoothly along the curve and does 
not get confused.
If $\epsilon$ is large, then it is possible to jump discontinuously to 
another part of the curve (Figure~\ref{1.3}(a)), 
or even to jump completely off of the end of the curve (Figure~\ref{1.3}(b)).
The jumps must also be small in order to avoid jumping over two or more 
sortpoints at the same time, which would cause problems in sorting,
and to avoid ignoring a sortpoint x by tracing through it without jumping into
the $\delta$-ball about x that triggers x's insertion into the sort.
As a result, the crawling method is 
very slow unless the sort segment is short.

%
\begin{figure}[htbp]\vspace{2.25in}\caption{Jumps must be small}\label{1.3}\end{figure}
%\figg{1.3}{Jumps must be small}
%

Another undesirable property of the crawling method 
is that its speed depends upon the length of
the sort segment rather than upon the number of points to be sorted.
The crawling method does have the advantage that it does not require any
preprocessing, such as the computation of a parameterization.

The weaknesses that we have observed in the 
parameterization and crawling methods of sorting points along a curve
suggest that another method is necessary: one that will perform more
efficiently on a wider selection of algebraic curves.
The next chapter presents such a method.
