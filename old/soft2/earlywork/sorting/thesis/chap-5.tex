\chapter{Applications, Future Work, and Conclusions}
\label{chap-5}

This chapter creates a context for sorting within
the area of geometric modeling.
Section~1 cites several applications of sorting, thereby
establishing its importance.
Section~2 discusses various research problems that are suggested
by our work on sorting, and
we end with some conclusions.
\section{Applications}

The sorting of points along an algebraic curve has many applications 
in geometric modeling.
The components of a geometric model are faces, edges, and vertices, which
are represented by patches of algebraic surfaces, segments of algebraic
curves, and points, respectively.
The sorting of points is a problem that arises naturally in the
manipulation of these components.
We consider a number of\vspace{.5in} applications.\\
%
{\bf Problem 1}
\ \ Given 
a set S of points on an algebraic curve C, determine the points of
S that lie on an edge E of C.\\
{\bf Solution}\ \ 
E is defined by the implicit equation of C 
and two endpoints $V_{1}$ and $V_{2}$.
If S is sorted along \arc{V_{1}V_{2}}, then the points of S that
do not lie on the edge will be ignored and the sorted list will only
contain the points of S that lie on the \vspace{.5in}edge.\\
Problem (1) is a very basic problem in geometric modeling.
It must be solved regularly in operations ranging from intersection to
display.
We offer two examples of its use.\\
{\bf Problem 2}\ \  
Compute the intersection of two edges.\\
{\bf Solution}\ \ Let $E_{1}$ and $E_{2}$ be edges of the curves
$C_{1}$ and $C_{2}$, respectively.
Once the points of intersection of the two curves have been computed (perhaps
by resultants), Problem (1) can be applied (twice) to determine 
the points 
of $C_{1} \cap C_{2}$ that lie on $E_{1} \cap E_{2}$, since\vspace{.5in}
\mbox{$E_{1} \cap E_{2} = [(C_{1} \cap C_{2}) \cap E_{1}] \cap E_{2} $.}\\
%
{\bf Problem 3}\ \ Determine a bounding box for an edge E.\\
%
{\bf Solution}\ \ In motion planning, it is useful to approximate a
geometric model by a simple superset, because this makes interference
detection simpler.
The more expensive interference detection with the geometric model can
be reserved for those situations when the solid approaches close enough
to an obstacle that interference is detected with the simple superset.
Bounding regions are also useful for problems such as (1) above, 
for they allow
points that clearly do not satisfy a condition to be quickly discarded.
We can define a bounding rectangle for an edge E by the minimum and maximum
x and y values of E.
Consider the computation of the maximum x value of E.
(The other extrema are computed in a similar manner.)
E's x-maximum is either attained at a local x-maximum of E's curve 
or at an endpoint of the edge.
Therefore, in order to determine the maximum x-value of E,
the local x-maxima of the curve must be computed (as solutions of 
$f_{y} = f = 0$, where $f$ is the implicit equation of E's curve), 
and then restricted to the subset that lies on E.
This restriction is an instance of\vspace{.5in} Problem~(1).\\
%
{\bf Problem 4}\ \
Determine if a point lies within a piecewise-algebraic plane patch.\\
{\bf Solution}\ \
This problem is fundamental to the display of a geometric model.
A piecewise-algebraic plane patch is defined by a closed boundary consisting
of a simply-connected collection of plane algebraic curve segments.
The problem of determining whether 
a point $Q$ lies within the closed boundary 
reduces to the problem of sorting points by the following mapping.
Consider the straight line $L$ defined by a vertex $V$ on the boundary
and the point $Q$.
We compute the set ${\cal I}$ of intersections of $L$ with the algebraic curve
segments 
of the patch's boundary, through several applications of Problem~(2).
The points of ${\cal I}$ and the point $Q$ are then sorted along the line $L$.
By applying the Jordan curve theorem, the points of ${\cal I}$ can be grouped into
pairs, and inside/outside intervals can be determined.
$Q$ lies within the patch if and only if it lies on an inside interval.
%
\figg{5.0}{Deciding if Q lies inside the plane patch}{2.25in}
%
\begin{example}
Consider the plane patch of Figure~\ref{5.0}.
${\cal I} = \{L_{1},L_{2},L_{3},V\}$.
The sorted order of ${\cal I} \cup \{Q\}$ is $L_{1},L_{2},V,Q,L_{3}$.
Therefore, the intervals of L that are inside the patch are
\seg{L_{1}L_{2}} and \seg{VL_{3}}.
Since Q lies on one of these inside intervals (\seg{VL_{3}}),
it lies inside of the patch.
\end{example}
%
{\bf Problem 5}\ \
Determine if a point lies within a 
piecewise-algebraic convex surface patch.\\
%
{\bf Solution}\ \ 
A piecewise-algebraic surface patch is defined by a closed, 
simply-connected loop of boundary edges on a primary surface $F$. 
The edges are algebraic space curve segments defined by the intersection
of secondary surfaces $G_{i}$ with the primary surface $F$.
This problem is a direct extension of Problem~(4), with one exception.
Instead of using a line L defined by the point Q and a vertex on the boundary,
we use a planar cross-section of the primary surface, where the plane is defined
by Q and two vertices on the boundary.
The primary surface is assumed to be convex in order to guarantee that the planar cross-section
is a connected curve and that a sort of points along the cross-section is 
therefore well-defined.
Since we are now sorting points along a curve rather than a line, nontrivial
sorting is required both to find the points of intersection with the patch
boundary and to sort\vspace{.5in}~them.\\
%
{\bf Problem 6}\ \
Compute the intersection of two solid models.  
(See Section\vspace{.5in}~\ref{chap1-sec1}.)
%
%
%{\bf Solution}
%A piecewise algebraic surface patch is defined by a closed boundary consisting
%of a simply connected collection of edges on a primary surface $F$.
%The patch is convex if any line hits the patch at most two times.
%The edges are algebraic space curve segments defined by the intersection
%of secondary surfaces $G_{i}$ with the primary surface $F$.
%The method of determining if a point $Q$ lies within the patch is a
%direct generalization of the method of problem (6).
%However, problem (3) is used to compute the intersection set $S$, rather than
%problem (2).
%The surface is assumed to be convex in order to guarantee that a plane
%defined by the point $Q$ and two vertex points $V_1$, $V_2$ 
%on the boundary patch
%will intersect the primary surface $F$ in a convex connected curve $CL$.
%In general, the intersection of an algebraic surface with a plane will
%yield a reducible curve of many pieces.
%The points of $S$ and the point $Q$ are sorted on 
%the convex segment $CL$ rather than the line $L$ of problem (6).

These six problems give an indication of the importance of 
sorting points along an algebraic curve.
They also reveal that there are essentially two ways in which sorting can
be used:
Problems (1)-(3) use sorting as a means of restricting
a set of points to a specific subset, while Problems (4)-(6) use sorting
as a means of introducing an even-odd parity to a set of points.
%
\section{Future Work}
\subsection{Parameterization}

Our investigation of sorting has revealed some problems that require
further attention.
One of the most obvious areas for future 
research is the parameterization of surfaces of higher degree.
Appendix~\ref{app-D} presents 
some methods for parameterizing surfaces of 
degree two and three, but we know of no practical methods for higher degrees.
%
\subsection{Curves with Several Connected Components}
\label{manycomponents}

Curves with several connected components are more challenging
than curves with a single connected component.\footnote{Just as 
multisegment cells are more challenging than cells with a single convex
segment.}
Example~\ref{foobar} illustrates that all of the sortpoints must lie on the
same connected component when the convex-segment method is used to sort points
on a connected component that has no \wallpoints.
This requirement is not necessary (i.e., the sortpoints can be strewn over 
several connected components)
if the points are being sorted on a connected
component CC that contains at least one \wallpoint, since a convex-segment
traversal is then possible.
Any sortpoints that do not lie on CC will 
be ignored (using Theorem~\ref{thm-parents})
in the same way that
any sortpoints on CC that do not lie on the sort segment are ignored.
%
\begin{example}
\label{foobar}
The quartic curve of Figure~\ref{5.1} has four connected components, but 
no flexes or singularities.
Thus, this curve has no walls in its cell partition, and it will appear as if
all of the sortpoints lie on the same convex segment.
This will not be a correct 
conclusion unless all of the sortpoints lie on the same connected component.
\end{example}
%
\figg{5.1}{These sortpoints appear to be on the same convex segment}{2.375in}
% From Arnon83

Unfortunately, the points that are to be sorted
may lie on several connected components.
For example, the sortpoints will often be
generated by intersecting a curve with another curve or surface 
(as evidenced by the previous section), and the 
resulting points may be spread over several connected components.
Therefore, before sorting can proceed, those sortpoints that lie on 
connected components with no \wallpoints\
must be divided into connected components.
The most obvious way of doing this
is to create a boundary about 
%
\figg{5.3}{Component separation of a quartic curve}{2.5in}
%
each connected component (Figure~\ref{5.3}).
The creation of these boundaries turns out to be simple 
for cubic curves,
%
%There is only one class of cubic curve that has several branches:
%the nonsingular cubic, which has two branches.
%The closed component can be separated by finding the two tangents that
%strike one of the flexes of the curve.  (See Figure~\ref{fig4-chap6}.)
%
%Elaborate on cubic branch separation, using end of cubic.tex.
%
however a general solution may be very difficult.\footnote{Hilbert's 
$16^{th}$ problem is to determine the relative position
of the connected components of a nonsingular algebraic plane curve
\cite{kaplansky}.}
%
%"Hilbert's 16th problem asks for a study of the topology of nonsingular
%algebraic curves in the real projective plane ... \\
%... as a topological space, such an algebraic curve is a disjoint union
%of circles."  (p.153 of Kaplansky:QA36K17+)
%
Collins' cylindrical algebraic decomposition provides a possible solution.
This decomposition can be used to determine the topology of an algebraic curve 
\cite{arnon83,kozen}, from which
it should be straightforward to determine boundaries for each connected 
component.
Although the computation of a cylindrical algebraic decomposition of
$\Re^{d}$ is double-exponential (or parallel-exponential) in $d$, we
are only concerned with decompositions of the plane, so $d=2$ and the
method may be tractable.
%The weakness of this approach is the double-exponential (or 
%parallel-exponential) complexity of the algorithm that computes the 
%cylindrical algebraic decomposition.
%It may be necessary to search for a solution that is effective only for curves
%of low degree.



The sorting of points along a curve of several connected components 
is also difficult when the parameterization
method of sorting is used.
The following questions arise:
(1) should each connected component have a separate parameterization?
(2) if so, how does the implicit equation of a curve 
      produce several, independent parameterizations?
(3) if not, how can a single parameterization be split over several
      connected components
(i.e., how can the range of parameter values 
that is associated with each connected component be determined)?
%
%
\subsection{A Theory of Finite Precision}

The algorithms of geometric modeling must be implemented on a
computer of finite precision.
This can cause instabilities that, because of their geometric nature,
are quite different from the instabilities studied in conventional 
numerical analysis.
For example, in the creation of a cell partition, the tangent of a flex 
is intersected with the curve.
Since the curve can be very flat about a flex, a small error 
in the tangent will cause the tangent to have more than one intersection
with the curve near the flex.
Although this particular problem is easily solved (by merging the
intersections), finite precision
problems may eventually impede progress in other
applications.
Therefore, the development of a systematic theory for the resolution of
geometric problems arising from finite precision would be an important
contribution to geometric modeling.
\subsection{The Importance of Flexes, Singularities, and Projections}

The convex-segment method has revealed the importance of the
singularities and flexes of a curve.
Singularities are also used to develop a curve's parameterization 
(Section~\ref{subsec-par}).
It may be advisable to include flexes and singularities
as an integral component of a curve's representation 
in a geometric model.
(For 
a space curve, the singularities and flexes of its projection would be
stored.)
Certainly, the discovery of more efficient algorithms for the computation
of singularities and flexes would be a significant contribution.

Our investigation of sorting has also demonstrated that
the best way to solve a problem for a space curve may be to reduce the
dimension, and hopefully the complexity, of the problem by translating
it to an equivalent problem for a projection of the space curve.
%
\section{Conclusions}
We have developed a new method of sorting points along an algebraic curve
that is superior to the conventional methods of sorting.
Many curves that could not be sorted, or that could only be sorted slowly,
can now be sorted efficiently.
The development of our new method has also
illustrated how an algebraic curve can 
be decomposed into convex segments, and how the
ambiguity of sorting through a singularity can be resolved.

The creation and manipulation of curves and surfaces is of major 
importance to geometric modeling.
A sophisticated geometric modeling system should offer 
a rich collection of tools to aid this manipulation.
Our work on sorting has been an attempt to develop one of these tools.
The progress of geometric modeling depends upon the development of more
tools and upon the extension of more
computational geometry algorithms from polygons to curves and surfaces
of higher degree.
