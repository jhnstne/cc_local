\documentstyle[12pt]{article}
% \usepackage{times}
\pagestyle{empty}

% What does the kernel look like in dual space?

\begin{document}

Title: Kernels from hulls

Author: John K. Johnstone

\begin{abstract}
The kernel of a plane curve $C$ is the locus of points that can see every
point of $C$, a convex connected subset of the curve's interior.
Since a point of the kernel maximizes visibility of the object, the kernel 
is a fundamental structure in visibility analysis.
The kernel of a polygon has long been an object of study in computational
geometry, especially in the context of art gallery theorems.
In this paper, we consider a new algorithm for computing the kernel of a
plane curve.
This analysis also sheds new light on the structure of the kernel,
by considering its relationship with the convex hull in dual space.

The kernel of a curve can be studied in dual space, using the 
geometric duality between lines and points.
An interesting observation is that the kernel in primal space
becomes the convex hull in dual space.
The tangents of a curve dualize to points, and the entire tangent space
of a curve dualizes to a curve.
It is the convex hull of this curve that is related to the kernel.
The tangent space is relevant to the study of kernels
because a point lies in the kernel of a curve if and only if it is not hit 
by any tangent.
Using the dual relationship between kernel and convex hull, 
any kernel computation can be reduced to a convex hull computation.
The structure of the kernel can also be better understood from the structure 
of the hull.

The reduction of kernel to convex hull relies on the prior knowledge of a 
single point of the kernel.
This point is used to adjust the curve before dualization, ensuring the 
finiteness of the curve in dual space.
We discuss how to compute this seed point of the kernel.
\end{abstract}

\end{document}

