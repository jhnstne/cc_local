\documentclass[12pt]{article}
\usepackage[pdftex]{graphicx}
% \usepackage{epsfig}
\usepackage{times}
\input{header}

\newif\ifJournal
\Journalfalse
\newif\ifTalk
\Talkfalse

\newcommand{\plucker}{Pl\"{u}cker\ }
\newcommand{\tang}{tangential surface\ }
\newcommand{\tangs}{tangential surfaces\ }
\newcommand{\Tang}{Tangential surface\ }
\newcommand{\atang}{tangential $a$-surface\ }
\newcommand{\btang}{tangential $b$-surface\ }
\newcommand{\ctang}{tangential $c$-surface\ }
\newcommand{\atangs}{tangential $a$-surfaces\ }
\newcommand{\btangs}{tangential $b$-surfaces\ }
\newcommand{\ctangs}{tangential $c$-surfaces\ }

\setlength{\oddsidemargin}{0pt}
\setlength{\topmargin}{-1in}	% should be 0pt for 1in
% \setlength{\headsep}{.5in}
% \setlength{\textheight}{8.875in}
\setlength{\textheight}{8.6in}
\setlength{\textwidth}{6.875in}
\setlength{\columnsep}{5mm}	% width of gutter between columns

% \DoubleSpace

\title{DRAFT\\Smooth kernels in 2-space:  Finding one point of a smooth kernel\\is as hard as finding the entire kernel}
% Hulls, kernels, and giftwrapping of smooth curves
% A smooth Jarvis march
% The convex hull of a smooth curve
\author{J.K. Johnstone\\
	Geometric Modeling Lab\\
	Computer and Information Sciences\\
	University of Alabama at Birmingham\\
	University Station, Birmingham, AL, USA 35294}
%	Geometric Modeling Lab, Computer and Information Sciences, University
%	Station, Birmingham, AL, USA 35294; Phone: (205) 975-5633; 
%	Fax: (205) 934-5473; jj@cis.uab.edu

\begin{document}
\maketitle

\begin{abstract}
The kernel of a point set S is the subset of S that can see every point of S,
a useful tool in visibility analysis.
The kernel of a polygon is well understood.
Recently, Elber developed an algorithm for the kernel of a plane
parametric curve, expressed as the zero set of a system of equations.
Motivated by this work,
we develop an alternative algorithm for the kernel of a closed plane curve
using duality.

Our starting point is the existence of a 
dual relationship between the convex hull and kernel:
if a single point of the kernel can be found,
the kernel in primal space can be computed 
directly from the convex hull in dual space.
This reduces the problem to finding a seed point of the kernel.
We present a method for computing a seed point,
by finding a small set of candidate points
that is guaranteed to contain a kernel point (if the kernel is not empty).
These candidates are built from the curve's inflection tangents.
The algorithm is efficient, requiring only a small
number of line-curve and line-line intersections.
Finally, we show that the computation of the seed point 
actually computes the entire kernel, allowing us to avoid the
reduction to convex hull.
	% The key idea is that the computation of the seed point yields not only 
	% the seed point, but all of the corners of the kernel as well.
The algorithm leverages the power of dual space,
using the recently developed tangential curve,
a dual image of the curve's tangent space.
Dual space is used both for practical computation
and for theoretical understanding.
\end{abstract}

\ifTalk
The convex hull in dual space can be used to compute the kernel
in primal space, after some massaging.
It is this massaging that is the main topic of this paper.
\fi

Keywords: kernel, convex hull, dual space, tangential curve, bitangent,
inflection point.

% \tableofcontents

\clearpage

%%%%%%%%%%%%%%%%%%%%%%%%%%%%%%%%%%%%%%%%%%%%%%%%%%%%%%%%%%%%%%%%%%%%%%%%%%%%%

\section{Introduction}

We shall develop an efficient 
algorithm for computing the kernel of a closed plane curve.
The kernel is a convex connected subset of the curve's interior,
defined as follows (Figure~\ref{fig:curveob1b}).
% Let us begin with a definition of the kernel.
% Notice that the kernel is convex and connected.

\begin{defn2}
\label{defn:kernel}
Let $C$ be a plane curve.
The point $P$ can {\bf see the point} $Q$ on $C$
if the line segment $\seg{PQ}$ does not intersect $C$.
The {\bf kernel} of $C$ is the locus of points that can see every
point of $C$:
\[
\mbox{kernel}(C) = \{ P \in R^2: \seg{PQ} \ \cap \ C = \emptyset 
		\hspace{.25in} \forall \ Q \in C \}
\]
\end{defn2}

\begin{figure}[b]
\begin{center}
\includegraphics*[scale=.5]{img/jjhu1.jpg}
\includegraphics*[scale=.5]{img/jjke0.jpg}
\end{center}
\caption{A smooth curve and its kernel}
\label{fig:curveob1b}
% kernel -p -d 10 data/ob1b.pts
\end{figure}

The kernel is a classical construct of computational geometry
\cite{preparata85,goodman97}.
\ifJournal
Star-shaped polygons, polygons with nontrivial kernels, are an 
important class.
\fi
It is also a fundamental structure for visibility analysis \cite{orourke87}.
For example, the kernel of a room would be an ideal location to place a camera, 
robot or light, since the entire room is visible from this point.
The kernel problem has long been resolved for polygons,
since Lee and Preparata developed an optimal linear algorithm 
for the kernel of a simple polygon \cite{lee79}.
The kernel of a plane curve has only recently been studied \cite{elber02}.
Elber's algorithm uses an algebraic computation from the zero set of a 
system of nonlinear equations.
Motivated by Elber's work, we propose a new algorithm for the kernel of a
curve that leverages the power of duality.

The structure of our algorithm is as follows.
We observe that the kernel is dually related to the convex hull.
This offers a new way of interpreting the kernel using an analysis
in dual space, especially since the convex hull is well
understood \cite{elber01,jj02hull}.
The first attempt is to reduce the problem to the convex hull in dual space.
However, this requires a single point of the kernel,
to ensure a robust dualization of the curve (Section~\ref{sec:kernel2hull}).
Thus, our focus turns to the problem of finding a seed point of the kernel.
Using the tangents at the inflection points of the curve,
we find a small set of candidate points that is guaranteed to contain
a kernel point.
The seed point is extracted from this candidate set by a simple test
in dual space, using the tangential curve \cite{jj01,jj02}.
This completes the reduction to the convex hull.

We next observe that 
the kernel can be computed directly,
using the very tools that we developed to find the seed point.
In particular, the candidate set will not only contain a seed point of
the kernel: it will contain all of the corner points of the kernel.
The entire kernel can be defined from these corner points.
In short, in calculating one point of the kernel, we actually compute
the entire kernel.

Since our algorithm and its development relies heavily on dual space,
we first introduce this material in Section~\ref{sec:duality}.
Section~\ref{sec:kernel2hull} develops the argument that a kernel computation
can be reduced to a convex hull computation in dual space.
Section~\ref{sec:candidate} shows how a seed point of the kernel can be computed,
which clears the way for this reduction.
Since this process inherently deals with infinite curves in dual space,
we discuss how to guarantee that this computation is robust 
in Section~\ref{sec:trim}, by trimming the infinite curves.
Section~\ref{sec:nohull} presents our preferred algorithm for the kernel,
computing the kernel directly from the seed point construction.
We offer some examples of kernels in Section~\ref{sec:eg} and conclude 
in Section~\ref{sec:conclusions}.

%%%%%%%%%%%%%%%%%%%%%%%%%%%%%%%%%%%%%%%%%%%%%%%%%%%%%%%%%%%%%%%%%%%%%%%%%%%%%

\section{The kernel in dual space}
\label{sec:duality}

In dual space, lines are replaced by points and points by lines.
% Duality is a mechanism for reinterpreting lines as points and points as lines.
We give a short review of duality here.
For a more detailed discussion, the reader is referred to our papers
on tangential curves \cite{jj01,jj02}.
	% We need to use three different dualities, each interpreting the projective 
	% coordinate differently.
	% For the work on kernels, we can restrict our attention to the c-duality.

\begin{defn2}
\label{defn:dual}
The line $ax+by+c=0$ in 2-space is {\bf dual} to the point 
$(\frac{a}{c}, \frac{b}{c})$ in 2-space.\footnote{This is equivalent to 
	the c-duality in \cite{jj02}.}
% (or the point $(a,b,c)$ in projective 2-space).
\end{defn2}

\begin{lemma}
\label{lem:cinfty}
Lines through the origin map to infinity under c-duality.
\end{lemma}

\begin{lemma}
\label{lem:join}
The line between $a$ and $b$
	% The join of two points $a$ and $b$
dualizes to the intersection of the lines $a^*$ and $b^*$.
\end{lemma}

\begin{lemma}
Duality is its own inverse:
$\mbox{dual}(\mbox{dual}(x)) = x$.
In other words, there is a symmetry to duality:
if $y = \mbox{dual}(x)$, then $x = \mbox{dual}(y)$.
\end{lemma}

%%%%%%%%%%%%%%%%%%%%%%

The curve's tangents will play an important role in the kernel.
When the tangents of a curve are mapped to dual space,
they define a tangential curve.
In the rest of the paper,
$C$ will refer to a closed plane curve and $C^*$ its tangential curve.

\begin{defn2}
\label{defn:tangentialcurve}
Let $C(t)$ be a plane curve.
The {\bf tangential curve} of $C(t)$ is the curve $C^*(t) \subset P^2$,
where $C^*(t)$ is the dual of the tangent at $C(t)$.\footnote{This is
	equivalent to the tangential c-curve in \cite{jj02}.
	Unlike most problems involving the tangent space of a curve,
	the tangential c-curve is more natural than
	the tangential a-curve and b-curve for the kernel,
	basically because of Lemma~\ref{lem:kernel}.}
\end{defn2}

%%%%%%%%%%%%%%%%%%%%%%%%%

We can now reinterpret the kernel in dual space.
% We first need to reinterpret the kernel in primal space.
\cite{elber02} characterized the kernel in a fresh way that is more
useful algorithmically than the classical definition.
It shows that the tangent space 
of a curve carves out the kernel (Figure~\ref{fig:kernel}b)
and leads directly to a dual interpretation.
% Lemma~\ref{lem:kernel} can be reinterpreted in dual space.

\begin{lemma}[Elber]
\label{lem:kernel}
$P \in \mbox{kernel}(C)$ if and only if no tangent of $C$ intersects $P$.
\end{lemma}

\begin{corollary}
$P \in \mbox{kernel}(C)$ if and only if dual($P$) does not intersect the 
tangential c-curve $C^*$.
\end{corollary}

\begin{defn2}
A line in dual space is {\bf free} if it does not intersect the 
tangential c-curve $C^*$.
\end{defn2}

\begin{corollary}
\label{cor:free}
$P \in \mbox{kernel}(C)$ if and only if dual($P$) is free.
\end{corollary}

\noindent In other words, points of the kernel are easily recognizable in dual space
as free lines (Figure~\ref{fig:freeline}).
This observation will be central to our computation of the kernel.

\clearpage

\begin{figure}
\begin{center}
\includegraphics*[scale=.5]{img/jjke1.jpg}
\includegraphics*[scale=.5]{img/jjke2.jpg}
\end{center}
\caption{(a) The kernel of a curve (b) The tangent space of a curve, revealing the kernel}
\label{fig:kernel}
% kernel -p -d 10 data/ob1b.pts, with left window grown
\end{figure}

% picture of kernel, points in and out of kernel, and associated dual lines
% including some tangents
\begin{figure}
\begin{center}
\includegraphics*[scale=.4]{img/jjke3a.jpg}
\includegraphics*[scale=.4]{img/jjke3b.jpg}
\includegraphics*[scale=.4]{img/jjke3c.jpg}
\end{center}
\caption{Kernel points in primal space are associated with free lines in dual space}
\label{fig:freeline}
% kernel -p -d 10 -x .5 data/ob1b.pts (origin now in kernel)
\end{figure}

\clearpage

%%%%%%%%%%%%%%%%%%%%%%%%%%%%%%%%%%%%%%%%%%%%%%%%%%%%%%%%%%%%%%%%%%%%%%%%%%%%%

\section{Computing the kernel from the convex hull}
\label{sec:kernel2hull}

The kernel of a curve $C$
is strongly related to its convex hull.
To get the kernel of $C$, 
we shrink $C$ until every point of the new $C$
can see every point of the old $C$.
The convex hull of $C$ is similar, except that 
% This is similar to the convex hull of $C$, where we
instead we expand $C$ until every point of
the new $C$ can see every point of the {\em new} $C$.
Both are closures under the 'see' operation ($\seg{PQ}\ \cap \ C = \emptyset$):
one a contraction and the other an expansion.
	% It is not surprising, therefore, that the convex hull
	% can be used in the computation of the kernel.
The following theorem clarifies the tightness of this relationship.

\begin{theorem}[Wang 02]
\label{thm:hullisdual}
Suppose that the kernel of $C$ contains the origin $(0,0)$.
Then the kernel of $C$ is dual to the convex hull of 
its tangential curve $C^*$.
\end{theorem}
\prf
More precisely, we will show that the boundary of the kernel of $C$ is dual 
to the tangent space of the boundary of the convex hull of $C^*$.
This will show that the kernel of $C$ is immediately available from the 
convex hull of $C^*$, as shown in Algorithm 1.

Since the kernel contains the origin, no tangent will intersect the origin
(Lemma~\ref{lem:kernel}) and so no tangent will dualize to infinity (Lemma~\ref{lem:cinfty}).
Thus, $C^*$ is finite, bounded, and closed,
and its convex hull is well defined.
Since kernel points correspond to free lines (Corollary~\ref{cor:free}),
boundary points of the kernel correspond to 
free lines that are just about to touch $C^*$, or free tangents of $C^*$.
But the free tangents of $C^*$ bound its convex hull (Lemma~\ref{lem:condition}
of the appendix).
\QED

% does not reveal anything new
% more importantly, this figure misleads, since we are really interested in the
% tangent space of the hull
% the previous figure is a better indication of the result (pt goes to line)
% figure of translated ob1b, its kernel, and the convex hull of its tangential c-curve
% \begin{figure}[h]
% \begin{center}
% \includegraphics*[scale=.5]{img/--.jpg}
% \end{center}
% \caption{The kernel and the convex hull are dual, if the origin lies in the kernel}
% \label{fig:kernelob1a}
% \end{figure}

This dual relationship between the kernel and convex hull 
only holds when the kernel contains the origin,
because we implicitly rely on the boundedness of $C^*$.
If the kernel of $C$ does not contain the origin, a reduction to the convex hull
of $C^*$ will not work since $C^*$ will be unbounded
or disconnected, and the convex hull of $C^*$
will be uninteresting (Figure~\ref{fig:originNotInKernel}).

%%%%%%%%%%%%%%%%%%%%%%%%%%%%%%%%%%%%%%%%%%%%%%%%%%%%%%%%%%%%%%%%%%%%%%%%%%%%%

Theorem~\ref{thm:hullisdual} leads to the following algorithm for computing
the kernel.
	
\vspace{.2in}

\centerline{{\bf Algorithm 1: Computing the kernel of $C$ from the hull}}

\begin{enumerate}
\item 	Find a point P of the kernel.  If no such point exists, 
	the kernel is empty.
\item   Translate P to the origin.
\item	Compute the tangential curve $C^*$.
\item	Compute the convex hull of $C^*$.
	This defines the kernel of $C$:
	the curved segments $C^*(t_1,t_2)$ of the hull boundary define
	the curved segments $C(t_1,t_2)$ of the kernel boundary,
	and the bitangents $\seg{C^*(t_1)C^*(t_2)}$ of the hull boundary 
	define the straight line segments $\seg{C(t_1)C(t_2)}$ of the
	kernel boundary.
\end{enumerate}

% figure of ob1b, its kernel, and its unclipped tangential c-curve
\begin{figure}[h]
\begin{center}
\includegraphics*[scale=.5]{img/jjke4.jpg}
\end{center}
\caption{A curve whose kernel does not contain the origin, and its infinite tangential curve}
\label{fig:originNotInKernel}
% kernel -p -d 10 data/ob1b.pts
\end{figure}

% We have solved the kernel problem if the origin lies in the kernel.
% But this will rarely be the case.
% Indeed, for all curves with empty kernels, this will not be the case.

This establishes that the problem of computing the kernel reduces to the 
apparently easier problem of computing a single point of the kernel.
However, this subproblem is slippery.
	% Notice that if the kernel is empty, there isn't even one to find!
In fact, we shall discover that finding one point of the kernel
is just as hard as finding the entire kernel.

%%%%%%%%%%%%%%%%%%%%%%%%%%%%%%%%%%%%%%%%%%%%%%%%%%%%%%%%%%%%%%%%%%%%%%%%%%%%%

\section{Computing a single point of the kernel}
\label{sec:candidate}

% We must find one point of the kernel.
We shall attack this problem by finding 
a finite collection of points that must necessarily contain
a kernel point, if one exists.
It will then be a simple process of testing each of these candidates against
the kernel, using Corollary~\ref{cor:free}.
If none are in the kernel, the kernel must be empty.
	% we shall conclude that the kernel is empty.
	% Otherwise, we shall have one or more seed points for the kernel.

\begin{defn2}
A set of points $S \subset \Re^2$ is a 
% {\bf necessary and sufficient set of kernel candidates} 
{\bf kernel set}
if $S \cap \mbox{kernel}(C) \neq \emptyset$ 
whenever  $\mbox{kernel}(C) \neq \emptyset$.
\end{defn2}

\noindent 
We shall compute the kernel set in primal space as a set of points,
but often interpret it in dual space as a set of lines.

Convex curves are a special case in this development.
Since the kernel of a convex curve $C$ is simply the interior of $C$,
we can assume without loss of generality that $C$ is concave.
Moreover, convex curves are easily recognized as curves without inflection
points.
We can also assume that the kernel of $C$ is
not empty, otherwise the choice of kernel set is arbitrary.
Curves with self-intersections are another trivial case,
since these curves always have empty kernels.
(As the tangent sweeps from the self-intersection $I$ back to $I$ again,
it will sweep out the entire plane, leaving no room for the kernel.)
Therefore, we can assume that $C$ is a simple concave 
curve with nonempty kernel (what we shall call a {\bf nontrivial} curve).

% 	\begin{defn2}
% 	An {\bf inflection point} is a point at which the sign of the curvature changes.
% 	% Not just a zero curvature.
% 	\end{defn2}

Inflection points will be central to our construction of the kernel set.
Thinking of the kernel as the region milled out by the tangent space (Figure~\ref{fig:kernel}b),
it becomes clear why inflection points will play an important role,
as they represent local extrema of the tangent's sweep in a particular
direction.\footnote{The inflection tangents 
	that you see in 
	Figure~\ref{fig:kernel}b are not explicitly drawn.
	They are artifacts of the tangent space, indicating where
	tangents naturally bunch together as they change direction
	at an inflection point.}
Nontrivial curves will have inflection points.
Indeed, every concavity must contain an inflection point.
% Consider walking along a (concave, simple) curve and entering a concavity.
% The curve must eventually change curvature in order to leave the concavity.
% At this change, there is an inflection point.

%	\begin{lemma}
%	If $C$ is convex, $\mbox{kernel}(C) = \mbox{interior}(C)$.
%	\end{lemma}
%	
% 	\begin{lemma}
%	\label{lem:hasinfl}
%	Suppose $\mbox{kernel}(C) \neq \emptyset$.
%	If $C$ is not convex, then $C$ has inflection points.
%	\end{lemma}
%	\prf
%	Consider walking along a curve and entering a concavity.
%	The curve must eventually change curvature in order to leave the concavity.
%	At this change, there is an inflection point.
%	\QED

Here is the punch line of our development.

\begin{theorem}
\label{thm:kernelset}
Let $T$ be the tangents of the inflection points of $C$.
Let $H_1$ be the intersections of $T$.
Let $H_2$ be the intersections of $T$ with $C$.
A kernel set for a concave $C$ is $H_1 \cup H_2$.
\end{theorem}

\clearpage

\begin{figure}
\begin{center}
\includegraphics*[scale=.4]{img/jjke5.jpg}
\end{center}
\caption{Inflection points of $C$ and their tangents, and the associated cusps of the clipped $C^*$}
\label{fig:infl}
% kernel -p -d 10 data/ob1b.pts
\end{figure}

\begin{figure}
\begin{center}
\includegraphics*[scale=.4]{img/jjke6.jpg}
\end{center}
\caption{The first component $H_1$ of the kernel set}
\label{fig:comp1}
% fig 11
\end{figure}

\begin{figure}
\begin{center}
\includegraphics*[scale=.4]{img/jjke7.jpg}
\end{center}
\caption{The second component $H_2$ of the kernel set}
\label{fig:comp2}
\end{figure}

\clearpage

\begin{figure}
\begin{center}
\includegraphics*[scale=.4]{img/jjke8.jpg}
\end{center}
\caption{The kernel points in the kernel set of Figures~\ref{fig:comp1}-\ref{fig:comp2}}
\label{fig:free}
\end{figure}

Figures~\ref{fig:comp1}-\ref{fig:comp2} 
illustrate $H_1$ and $H_2$, as points in primal space
and lines in dual space.
% \footnote{In this and 
%	subsequent figures, $C^*$ is trimmed as described in Section~\ref{sec:trim}.}
To establish that $H_1 \cup H_2$ is a kernel set,
we shift our attention to dual space.
The tangent of an inflection point in primal space becomes a cusp in dual space
(Figure~\ref{fig:infl}).
After all, at an inflection point the tangent reverses direction,
so in dual space the point will reverse direction, yielding a cusp.

\ifTalk
\begin{lemma}
The tangent of an inflection point of $C$ dualizes to a cusp of $C^*$.
\end{lemma}
\fi

% \begin{defn2}
% A {\bf cusp} is a point at which the incoming and outgoing tangents are
% in opposite directions.
% At a cusp, the first derivative is zero.
% \end{defn2}

%	\begin{lemma}
%	\label{lem:inflcusp}
%	An inflection point of $C$ corresponds to a cusp of $C^*$.
%	\end{lemma}
%	\prf
%	As you pass through an inflection point, the tangent reverses direction
%	\QED

In primal space, we are looking for a kernel point.
In dual space, we are looking for a free line (Corollary~\ref{cor:free}).
The next lemma shows that we can restrict our search to bitangents.

\begin{lemma}
\label{lem:existence}
If $C$ is nontrivial,
one of the bitangents of $C^*$ is free.
\end{lemma}
\prf
If $C$ is nontrivial, its kernel is not empty and 
there exists a free line in dual space.
We can move this free line to a bitangent while preserving its freedom.
In particular, we can move the free line until it touches the curve once,
then pivot until it touches the curve again, generating a bitangent.
This pivoting is always possible because the bitangent is a cusp
bitangent (see below).
%
%	This will not necessarily work for conventional bitangents.
%	But it does when we incorporate cusps, as follows.
%	Push the free line until it touches a cusp (what if it can't reach a cusp?).
%	Then rotation is free about this cusp, until we become tangent to the curve.
%	(The problem with rotation about a non-cusp is that you are moving along
%	the curve in order to maintain tangency and you may go to infinity
%	before a second point of tangency is found.)
%
%	Notice that $C^*$ will have a (cusp) bitangent, since 
%	$C$ has an inflection point and $C^*$ has a cusp.
\QED

\vspace{-.3in}

\begin{corollary}
If $C$ is nontrivial, the bitangents of $C^*$ define a kernel set.
% need concave C, since a convex C will have C* with no bitangents
\end{corollary}

% A kernel set is a set of points that necessarily includes a kernel point.
% In dual space, this becomes a 
% collection of lines that necessarily includes a free line (Corollary~\ref{cor:free}).
% Equivalently, there exists a free line if and only if there exists
% a free line that is a bitangent of $C^*$.

Bitangents come in two flavours: conventional bitangents and cusp bitangents.
But for nontrivial $C$, $C^*$ only has cusp bitangents.

\begin{defn2}
\label{defn:bitang}
A {\bf conventional bitangent} of a curve $C$ is a line
that is tangent to $C$ at two or more distinct points.
If a curve contains cusps, tangents of the curve through the cusps
and lines between the cusps are also considered bitangents.
% (Figure~\ref{fig:cusps}).
These will be called {\bf cusp bitangents}.
\end{defn2}

\begin{lemma}
If $C$ is nontrivial,
all of the bitangents of $C^*$ are cusp bitangents.
\end{lemma}
\prf
Conventional bitangents of $C^*$ are associated
with self-intersections of $(C^*)^*$ 
and $(C^*)^* = C$ \cite{jj02}.
Since $C$ is nontrivial, it has no self-intersections
and we conclude that $C^*$ has no conventional bitangents.
The only remaining source of bitangents is cusps.
\QED

\vspace{-.2in}

% What are the points in primal space associated with cusp bitangents in
% dual space?
% A cusp bitangent is either a line between two cusps
% (right side of Figure~\ref{fig:comp1}) 
% or a tangent through a cusp.
% (right side of Figure~\ref{fig:comp2})
% Since cusps dualize to inflection tangents,
A line between two cusps dualizes to the intersection of the associated inflection
tangents (Lemma~\ref{lem:join}).
A tangent of $C^*$ through a cusp dualizes to the intersection of 
the associated inflection tangent with the curve $C$.
% \footnote{Here, we again appeal to the fact that $(C^*)^* = C$.}
We conclude that the (cusp) bitangents of $C^*$ dualize to
the intersections of the inflection tangents of $C$ ($H_1$) 
and the intersections of these inflection tangents with $C$
($H_2$).
This is the kernel set of Theorem~\ref{thm:kernelset}
and proves the theorem.

We can gather this material into an algorithm
for the computation of a kernel set.

\vspace{.2in}

\centerline{{\bf Algorithm 2: Computing the kernel set}}

\begin{enumerate}
\item Compute the inflection points of $C$.
\item If $C$ has none, it is convex and $\mbox{kernel}(C) = \mbox{interior}(C)$.
\item Compute the intersections $H_1$ of the inflection tangents.
\item Compute the intersections $H_2$ of the inflection tangents with $C$.
\item $S = H_1 \cup H_2$ is our kernel set.
\end{enumerate}

% By definition, our kernel set $S$ will contain a kernel point (if the kernel is not empty).
The kernet set will necessarily contain a kernel point.
But this kernel point will not make a good seed point without some massaging.
% We can extract a kernel point directly from the kernel set,
% but we must manipulate this kernel point.
It turns out that our kernel set $S$ will contain many kernel points, not just one.
However, 
all of these points will inherently lie on the boundary of the kernel
(Figure~\ref{fig:free}),
since the kernel set corresponds to extremal free lines (free tangents)
in dual space.
The purpose of finding a point $P$ of the kernel is to create
a finite tangential curve $C^*$ for robust convex hull computation (Algorithm 1).
But a boundary kernel point will not 
lead to the most robust convex hull computation,
since the tangential curve will 'almost' be infinite.
To get a robust, interior point of the kernel,
we compute the centroid of the kernel points in $S$.
Since the kernel is convex, this centroid will lie in the kernel 
(Figure~\ref{fig:mean}).
This leads to the following algorithm for
trimming the kernel set down to a seed point of the kernel.
The tangential curve is used to isolate the kernel points in $S$,
appealing to Corollary~\ref{cor:free}.

\vspace{.2in}

\centerline{{\bf Algorithm 3: Computing a seed point from the kernel set}}

\begin{enumerate} 
\item	Compute the kernel set $S = H_1 \cup H_2$  
	and tangential curve $C^*$, using Algorithm 2.
\item	Compute $K = \{s \in S : \mbox{dual}(s) \cap C^* = \emptyset\}$.\footnote{This
		step will be altered in Section~\ref{sec:trim}.}
\item	$K$ is the set of kernel points in the kernel set.
	If $K = \emptyset$, the kernel is empty.
\item	Let $P$ be the centroid of $K$,
	an interior point of the kernel.
\end{enumerate}

$P$ is the desired seed point of the kernel,
bootstrapping the kernel computation.
Algorithms 2 and 3 implement step (1) of Algorithm 1
and complete the reduction of the kernel to the convex hull.
Notice that the only computational use of dual space in Algorithms
2 and 3 is the filtering of $S$ down to $K$ through intersection
with the tangential curve.

\begin{figure}[h]
\begin{center}
\includegraphics*[scale=.5]{img/jjke9.jpg}
\end{center}
\caption{The centroid of $K$ is a good seed point of the kernel}
\label{fig:mean}
\end{figure}

%%%%%%%%%%%%%%%%%%%%%%%%%%%%%%%%%%%%%%%%%%%%%%%%%%%%%%%%%%%%%%%%%%%%%%%%%%%%%

\subsection{Trimming the infinite tangential curve}
\label{sec:trim}

There is a robustness issue in the use of the tangential curve $C^*$
to find the kernel points $K$ in the kernel set $S$.
$C^*$ will inherently be infinite during the computation of $K$
and the infinite parts of $C^*$ will disrupt the analysis in dual space
(testing whether a line is free).
To make the computation of $K$ robust,
it is crucial that the infinite segments of $C^*$ be trimmed away 
(Figure~\ref{fig:trim}).
% 	The removal of infinite regions
%	from tangential a-curves and b-curves is called 'clipping'
%	because it involves horizontal and vertical line intersections
%	with the curve that resembles clipping by a window.
%	In the context of tangential c-curves, it involves removing curve
%	segments, more suggestive of traditional trimming.
Fortunately, this is a simple task.
$C^*$ goes to infinity as the tangent of $C$
approaches the origin,
so $C^*$ should be trimmed surrounding the tangents through the origin.

\begin{defn2}
The {\bf trimmed tangential curve} $C^*_{\mbox{\tiny{trim}}}$
is $C^*$ with the segments
\[
\{C^*(t_i - \epsilon, t_i + \epsilon),\ \ i=1,\ldots,n\}
\]
trimmed away,
where the tangents at $C(t_1),\ldots,C(t_n)$ pass through the origin
(Figure~\ref{fig:trim}).
$C^*_{\mbox{\tiny{trim}}}$ is a finite curve.
\end{defn2}

% tangents thru origin in primal space, untrimmed tangential curve in dual space
\begin{figure}[h]
\begin{center}
\includegraphics*[scale=.5]{img/jjke10.jpg}
\hspace{.3in}
\includegraphics*[scale=.5]{img/jjke11.jpg}
\end{center}
\caption{(a) Tangents through the origin guide the trimming;
	 (b) the associated trimmed segments in primal space lead to 
	 (c) the trimmed tangential curve.}
\label{fig:trim}
\end{figure}

The tangents of $C$ through the origin can be computed
using tangential curves (but the tangential a- and b-curves
of \cite{jj02} rather than the tangential c-curves of this paper,
which are inherently unstable for this analysis).
See \cite{jj02hull}.
% Step (2) of Algorithm 3 is changed to use the trimmed tangential curve.

% \begin{description}
% \item[(2)] Compute $K =
%	\{s \in S : \mbox{dual}(s) \cap C^*_{\mbox{\tiny{trim}}} = \emptyset\}$.
% \end{description}
% \vspace{.2in}

%%%%%%%%%%%%%%%%%%%%%%%%%%%%%%%%%%%%%%%%%%%%%%%%%%%%%%%%%%%%%%%%%%%%%%%%%%%%%

% \subsubsection{Accounting for trimming}
% \label{sec:trimarea}

The trimming of $C^*$ may lead to another type of mistake in the computation
of $K$, if we are not careful.
	% A kernel point is a point avoided by the tangent space.
Since trimming $C^*$ is equivalent to removing some of the tangent space of $C$,
a nonkernel point may be misdiagnosed as a kernel point because the only
part of tangent space that touched it was trimmed away (Lemma~\ref{lem:kernel}).
We compensate for the trimming of $C^*$
by adding a test (in primal space) of $K$ against the tangent space that was trimmed away,
as follows.
	% This additional test is performed in primal space using trim regions.

% The trimming of $C^*$ may affect the diagnosis of free lines in dual space.
% In particular, the set $K$ of kernel points (in Algorithm 3)
% may include some impostors if we test against $C^*_{\mbox{trim}}$ rather 
% than $C^*$.

\begin{defn2}
Suppose that the tangent at $C(t_i)$ passes through the origin.
	% so that the tangent space of $C(t_i - \epsilon, t_i + \epsilon)$ was removed from $C^*$.
The {\bf trim region associated with $C(t_i)$} 
is the region of primal space covered by the tangent space
of $C(t_i - \epsilon, t_i + \epsilon)$ (Figure~\ref{fig:trimregion}).
The {\bf trim region} is the union of all of the
trim regions associated with points whose tangents pass through the origin.
\end{defn2}

Step (2) of Algorithm 3 becomes:

\begin{description}
\item[(2a)] Compute $\hat{K} = \{s \in S : \mbox{dual}(s) \cap C^*_{\mbox{\tiny{trim}}} = \emptyset\}$.
\item[(2b)] Remove the points from $\hat{K}$ that lie in the trim region,
	yielding $K$.
\end{description}

This is a robust method of finding $K$, the kernel points in the kernel set.
\ifJournal
The trim region is easily computed.
Let $L_1$ and $L_2$ be the tangents at $C(t_i - \epsilon)$ and $C(t_i + \epsilon)$
respectively
and define the inside of $L_1$ (resp., $L_2$) 
to be the side that contains $L_2$'s (resp., $L_1$'s) point of tangency.
The trim region associated with $C(t_i)$ 
is the region inside $L_1$ and outside $L_2$,
or inside $L_2$ and outside $L_1$.
\fi

\begin{figure}[h]
\begin{center}
\includegraphics*[scale=.5]{img/jjke12.jpg}
\end{center}
\caption{A trim region}
\label{fig:trimregion}
\end{figure}

%%%%%%%%%%%%%%%%%%%%%%%%%%%%%%%%%%%%%%%%%%%%%%%%%%%%%%%%%%%%%%%%%%%%%%%%%%%%

\section{Computing the kernel without the convex hull}
\label{sec:nohull}

In preparation for computing the kernel from the convex hull, 
a finite collection of kernel points $K$ was computed.
In this section, we observe that $K$ already defines the entire kernel
and no subsequent convex hull computation is required.
This yields the preferred algorithm for the smooth kernel.

	% In this section, we present our preferred algorithm for the smooth kernel,
	% which does not require any convex hull computation.
	% It does rely, however, on the tools that we have developed along the way
	% to the kernel-from-hull algorithm.

We have observed that the points of $K$ lie on the boundary of the kernel.
Actually, they lie at the {\em corners} of the boundary because
$K$ corresponds to free bitangents in dual space.
Free lines correspond to kernel points, free tangents to kernel
points on the boundary, and free bitangents to kernel points at the
corner of the boundary (since they are extremal in two ways).
This leads to the following algorithm for computing the kernel from $K$.

\vspace{.2in}

\centerline{{\bf Algorithm 4: Computing the kernel of $C$ without the hull}}

\begin{enumerate}
\item Compute $K$ using Algorithms 2 and 3 (including the trimming refinements of step 2).
\item Sort $K$ about the sample mean of $K$.
\item The kernel of $C$ is defined from $K$ as follows.
      Two consecutive points of the sorted $K$ should be joined by:
\begin{enumerate}
\item a curve segment of $C$ if both points are from $H_2$ 
(intersections of inflection tangents with $C$)
\item a straight line if either point is from $H_1$ 
(intersections of inflection tangents).
\end{enumerate}
\end{enumerate}

Figure~\ref{fig:nohull} gives an example.

\ifJournal
These connections of the points of $K$ mimic the behaviour of the
convex hull in dual space.
% give an argument why
\fi

\begin{figure}
\begin{center}
\includegraphics*[scale=.5]{img/jjke13.jpg}
\end{center}
\caption{The points in $K$ fully define the kernel}
\label{fig:nohull}
\end{figure}

We conclude that computing a single point of the kernel is as difficult
as computing the entire kernel,
since the computation of $K$ needed to find a single point can be used
instead to compute all of the kernel.

% \begin{lemma}
% These seed kernel points fully define the kernel (without any subsequent
% convex hull computation), as follows.
% Let $\{C^*(t_1), \ldots, C^*(t_n)\}$ be the cusps of $C^*$, sorted along $C^*$.
% Let $\{C^*(t_i) C^*(u_i): i = 1,\ldots,n\}$ be the tangents through the cusps.
% Kernel is curve segments $C(u_i) C(u_{i+1})$ and line segments
% $C(t_i) C(u_i)$(?).
% \end{lemma}

%%%%%%%%%%%%%%%%%%%%%%%%%%%%%%%%%%%%%%%%%%%%%%%%%%%%%%%%%%%%%%%%%%%%%%%%%%%%%

\section{Examples}
\label{sec:eg}

Algorithm 4 has been fully implemented and we offer 
several examples in this section.
Figure~\ref{fig:banana} shows that 
the kernel of a banana will vary depending on the intersection
point of its inflection tangents.
Figure~\ref{fig:kernelob1a} shows a kernel
that contains the origin, showing the clear relationship with
the convex hull in dual space.
Figure~\ref{fig:complex} illustrates a more complicated multi-sided kernel.

	%	kernel data/banana1.pts; 
	%	kernel data/banana4.pts.
\begin{figure}
\begin{center}
\includegraphics*[scale=.5]{img/jjke15.jpg}
\includegraphics*[scale=.5]{img/jjke14.jpg}
\end{center}
\caption{The kernel of two bananas, one empty and one not}
\label{fig:banana}
\end{figure}

\begin{figure}
\begin{center}
\includegraphics*[scale=.5]{img/jjke16.jpg}
\end{center}
\caption{A kernel clearly dual to the convex hull}
\label{fig:kernelob1a}
\end{figure}

\begin{figure}
\begin{center}
\includegraphics*[scale=.5]{img/jjke17.jpg}
\end{center}
\caption{The kernel of a cloverleaf}
\label{fig:complex}
\end{figure}


%%%%%%%%%%%%%%%%%%%%%%%%%%%%%%%%%%%%%%%%%%%%%%%%%%%%%%%%%%%%%%%%%%%%%%%%%%%%%

\section{Conclusions}
\label{sec:conclusions}

Through an analysis of the tangential curve in dual space,
we have discovered that the corners of a curve's kernel are defined
by intersections of the inflection tangents with the curve and each other.
The appropriate intersections are associated with lines that do not
intersect the tangential curve in dual space.
This leads to a simple and efficient algorithm for defining the kernel.
Dual space and tangential curves are integral to the development
of this algorithm,
offering yet another example of the power of tangential curves for 
analyzing problems that involve tangent spaces,
including many visibility problems.
The kernel, which consists of points not struck by any tangent,
is certainly such a problem.
We are looking into the extension of these ideas to the kernel of a surface.

% A comparison to Gershon Elber's zero-set algorithm.

\ifTalk
Our search for a seed kernel point is equivalent to searching
for a free bitangent of the tangential curve.
\fi

\ifTalk
We have developed algorithms for the convex hull of a closed curve
and the kernel of a closed curve.
Both involve dual space (to find bitangents or to find free lines).
Both use bitangents, the hull in primal space and the kernel in dual space.
Both require a seed point before the algorithm can begin
(the seed point of the kernel is considerably harder to find).
Both use the convex hull (or at least can)!
Both use free lines.
\fi

\ifTalk
We can reveal which inflection tangents combine to define boundary points
of the kernel by mapping to dual space.
Lines between the duals establish
pairings and the correct pairings are recognized by being free.
\fi

\ifTalk
Once the bitangents of a collection of curves are known, 
it becomes feasible to build many secondary structures at a reasonable cost,
such as the convex hull, the visibility graph, and the kernel.
These algorithms have been classically developed for polygons.
We consider the construction of all of these structures for smooth curves.
The calculation of these structures in a smooth environment,
as opposed to a polygonal environment built from a discrete sampling of the
smooth environment, can significantly simplify the structures
and improve their accuracy.
\fi

% The convex hull of a curve is built out of bitangents.
% And the kernel set in dual space is built out of bitangents.
% Thus, no matter how we compute the kernel, as a convex hull
% or through the kernel set, its computation is based upon bitangents.

% Notice that our algorithm for finding the seed point of the kernel
% is similar to finding the hull bitangents that define the
% convex hull in dual space.

%%%%%%%%%%%%%%%%%%%%%%%%%%%%%%%%%%%%%%%%%%%%%%%%%%%%%%%%%%%%%%%%%%%%%%%%%%%%%

\ifJournal
\section{Acknowledgements}

Wenping Wang made the observation of Theorem~\ref{thm:hullisdual} 
	at the 2002 Dagstuhl
	Seminar on Geometric Modeling, at which I gave 
	a talk on tangential curves and the convex hull of a curve
	and Gershon Elber gave a talk on the kernel of a curve.
	I am indebted to Wenping and Gershon for
	motivating me to explore this relationship.
I appreciate talks with Wenping and Gershon Elber on the kernel,
and the opportunity for research and a stimulating
environment offered by the Dagstuhl seminars.
Gershon Elber suggested the bananas of Section~\ref{sec:eg}.
This work was supported in part by the National Science Foundation
under grant CCR-0203586.
\fi

%%%%%%%%%%%%%%%%%%%%%%%%%%%%%%%%%%%%%%%%%%%%%%%%%%%%%%%%%%%%%%%%%%%%%%%%%%%%%

\bibliographystyle{plain}
\begin{thebibliography}{Johnstone 02a}

\bibitem[Elber 01]{elber01}
Elber, G., M.-S. Kim and H.-S. Heo (2001)
The Convex Hull of Rational Plane Curves.
Graphical Models 63, 151--162.

\bibitem[Elber 02]{elber02}
Elber, G. (2002)
The kernel of freeform planar parametric curves.
Presented at Dagstuhl 2002.

\bibitem[Goodman 97]{goodman97}
Goodman, J. and J. O'Rourke, editors (1997)
Handbook of Discrete and Computational Geometry.
CRC Press (New York).

\bibitem[Johnstone 01]{jj01}
Johnstone, J. (2001)
A Parametric Solution to Common Tangents.
International Conferenced on Shape Modelling and Applications (SMI2001),
Genoa, Italy, IEEE Computer Society, 240--249.

\bibitem[Johnstone 02a]{jj02}
Johnstone, J. (2002)
The tangential curve.
Manuscript.

\bibitem[Johnstone 02b]{jj02hull}
Johnstone, J. (2002)
Giftwrapping smooth hulls.
Manuscript.

\bibitem[Lee 79]{lee79}
Lee, D.T. and F. Preparata (1979)
An Optimal Algorithm for Finding the Kernel of a Polygon.
Journal of the ACM 26(3), 415--421.

\bibitem[O'Rourke 87]{orourke87}
O'Rourke, J. (1987)
Art Gallery Theorems and Algorithms.
Oxford University Press (New York).

\bibitem[Preparata 85]{preparata85}
Preparata, F. and M. Shamos (1985)
Computational Geometry: An Introduction.
Springer-Verlag (New York).

\bibitem[Wang 02]{wang02}
Wang, W. (2002) Personal communication.

% \bibitem[Catmull 74]{catmull74}
% Catmull, E. (1974)
% A Subdivision Algorithm for Computer Display of Curved Surfaces.
% Ph.D. thesis, University of Utah.

\end{thebibliography}

\section{Appendix}

%	\begin{defn2}
%	A line is {\bf free} if it does not intersect $C$.\footnote{Of course,
%	we do not count a point of tangency as an intersection.}
%	\end{defn2}

\begin{lemma}
\label{lem:condition}
Let $C$ be a closed plane curve.
The point $P \in C$ lies on the boundary of the convex hull of $C$ 
if and only if the tangent at $P$ does not intersect $C$.
\end{lemma}
\prf
The tangent of a point inside a concavity will clearly
intersect the curve.
But the tangent of a point on the convex hull will define
a halfplane that contains the curve 
(since the convex hull of a point set is the intersection of the halfplanes 
that contain the point set),
so it will not intersect the curve.
%	More formally but no better,
%	A tangent defines a containing halfplane if and only if it is free.
%	The halfplane defined by the tangent at $P$ is clearly a limit halfplane
%	in this direction and therefore a boundary point of the hull if it is free.
%	Conversely, if $P$ lies on the convex hull boundary, 
%	then some halfplane through $P$ is containing.
%	But the only possible containing halfplane through $P$
%	is its tangent halfplane.
\QED

%%%%%%%%%%%%%%%%%%%%%%%%%%%%%%%%%%%%%%%%%%%%%%%%%%%%%%%%%%%%%%%%%%%%%%%%%%%%%

\ifJournal
Our kernet set is sharp, in the sense that an upperbound is sharp:
there is no smaller kernel set that would always define the kernel.
There are examples where the intersections of inflection tangents
define the kernel (if the convex hull is built out of lines between cusps,
such as a 3-cusped concave triangle) and other examples where the
intersections of inflection tangents with the curve define the kernel
(if the tangential curve has one cusp only),
so both components of the kernel set are necessary in some cases.
\fi

\ifTalk
In our reduction of the kernel to the convex hull,
we need to find a seed point in the kernel.
The algorithm for querying a point's inclusion in a star-shaped
polygon \cite[p. 44]{preparata85} 
also relies on finding a seed point in the kernel to
act as the origin.
\fi

\ifJournal
Possible extension to open curves:

Our method won't work as is on an open concave curve,
because it assumes that K defines a closed set
and open curves will have open kernels that go off to infinity.
However, it could probably be easily changed to work:
at least to define the kernel inside the convex hull of
the open curve.
For example, add the line segment between the endpoints as
a pseudo-inflection tangent.
\fi

\end{document}

%%%%%%%%%%%%%%%%%%%%%%%%%%%%%%%%%%%%%%%%%%%%%%%%%%%%%%%%%%%%%%%%%%%%%%%%%%%%%

\begin{lemma}
The kernel is convex and connected.
\end{lemma}

{\bf No need to compute bitangents in this algorithm, and thus no need
for the following discussion.}

\subsection{Bitangents}

Tangential curves are excellent at computing bitangents
of curves.

The bitangents of a curve will be important tools in this paper.
In \cite{jj01}, we showed how to compute the bitangents of a smooth curve
efficiently using a dual representation of the tangent space of the curve, 
called the tangential curve.

We use a dual space to compute bitangents.
Bitangents of $C$ are self-intersections of the tangential curve system of $C$
(i.e., self-intersections of the clipped tangential a-curve $C_a^*$
and self-intersections of the clipped tangential b-curve $C_b^*$).

Bitangents of a plane curve \cite{jj01,jj02}.

Convex hull of closed plane curve \cite{jj02,elber01}.

%%%%%%%%%%%%%%%%%%%%%%%%%%%%%%%%%%%%%%%%%%%%%%%%%%%%%%%%%%%%%%%%%%%%%%%%%%%%%

The kernel is robustly computed using tangential c-curves,
although bitangents are robustly computed using tangential a-curves and b-curves,
packaged into the tangential curve system.
The c-curve is appropriate after the seed point of the kernel point
has been moved to the origin, because the tangent space of the curve 
will not intersect the origin and so the tangential c-curve is finite
(whereas the a-curve and b-curve are inherently infinite since a closed
curve must have horizontal and vertical tangents).
Even before the seed point is calculated, the c-curve works better
than the a-curve or b-curve.
%
We guarantee that the tangent space of the curve does not intersect the origin.
	% since we shall compute a convex hull in dual space, which is inherently
	% closed and therefore inherently straddles both a-dual and b-dual spaces,
	% which complicates the hull computation.

\begin{rmk}
Once we find a point of the kernel and translate it to the origin,
we can exclusively use tangential c-curves for any problem (e.g., bitangent),
since they have no problems with infinity, and ignore the subtlety of
clipping and tangential curve systems (Section~\ref{sec:duality}).
\end{rmk}

