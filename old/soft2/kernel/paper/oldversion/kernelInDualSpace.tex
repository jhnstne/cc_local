\documentclass[12pt]{article}
\usepackage[pdftex]{graphicx}
\usepackage{times}
\input{header}

\newif\ifJournal
\Journalfalse
\newif\ifTalk
\Talkfalse

\setlength{\oddsidemargin}{0pt}
\setlength{\topmargin}{-1in}	% should be 0pt for 1in
% \setlength{\headsep}{.5in}
\setlength{\textheight}{8.6in}
\setlength{\textwidth}{6.875in}
\setlength{\columnsep}{5mm}	% width of gutter between columns

% \DoubleSpace

\title{The smooth kernel, dually}
% Computing a smooth kernel purely in dual space}
% A purely dual computation of the smooth kernel
% Smooth kernels in 2-space:  Finding one point of a smooth kernel\\is as hard as finding the entire kernel}
\author{J.K. Johnstone\\
	Geometric Modeling Lab\\
	Computer and Information Sciences\\
	University of Alabama at Birmingham\\
	University Station, Birmingham, AL, USA 35294}
%	Geometric Modeling Lab, Computer and Information Sciences, University
%	Station, Birmingham, AL, USA 35294; Phone: (205) 975-5633; 
%	Fax: (205) 934-5473; jj@cis.uab.edu

\begin{document}
\maketitle

\begin{abstract}
We show that the kernel of a plane curve can be computed purely in dual space.
No recourse is made to primal structures such as inflection points.
This disciplined abnegation of primal space, although perhaps appearing
to be a form of self-flagellation, does have its advantages.
For example, the dual solution is more easily generalizable to 
surfaces in 3-space than the primal solution.
\end{abstract}

Solution structure: kernel point to free line to free bitangent to 
			free cusp bitangent to (free line through cusps
			OR free tangent through cusp) to ...

\section{Introduction}

The kernel of a polygon can be computed as the intersection of its interior
halfplanes.
Analogously, the kernel of a plane curve can be computed as the intersection
of the interior tangent halfplanes of its inflection points
with the curve's interior.
These are computations in primal space.
Suppose that, instead, we want to compute the kernel in dual space.
This can be done without recourse to any primal computations,
once the curve % (or actually its tangent space) 
is mapped to dual space.
In this paper, we consider the kernel problem through this dual lens.
In 2-space, this is a problem of academic curiosity: what does a
computation of the kernel look like in dual space?
Moving to 3-space, a dual interpretation of the kernel 
takes on practical importance.
Since the inflection point in 2-space does not have a clean generalization
to 3-space, it is not clear how to generalize the primal solution.
The dual solution of curve kernels will have better promise for generalization
to surface kernels.

The kernel of a polygon is the locus of points that sees the entire 
polygon.\footnote{P sees Q if $\seg{PQ}$ does not intersect the polygon.}
This definition, which applies equally well to curves,
highlights the importance of kernels for visibility analysis.
However, for curves, Elber observed that there is a more useful 
characterization of the kernel \cite{elber02}.

\begin{lemma}[Elber]
\label{lem:kernel}
Let $C$ be a closed plane curve.
$P \in \mbox{kernel}(C)$ if and only if no tangent of $C$ intersects $P$.
\end{lemma}

Notice that this clarifies that the kernel is the region
carved out by the tangent space (Figure~\ref{fig:kernel}b).
Lemma~\ref{lem:kernel} also leads directly to a dual interpretation of the 
kernel.
This interpretation involves the tangential curve,
a dual representation of the curve's tangent space.
For more details on the tangential curve, see \cite{jj01,jj02}.

\begin{figure}
\begin{center}
\includegraphics*[scale=.5]{img/jjke1.jpg}
\includegraphics*[scale=.5]{img/jjke2.jpg}
\end{center}
\caption{(a) The kernel of a curve (b) The tangent space of a curve, revealing the kernel}
\label{fig:kernel}
% kernel -p -d 10 data/ob1b.pts, with left window grown
\end{figure}

\begin{defn2}
\label{defn:tangentialcurve}
Let $C(t)$ be a plane curve.
The {\bf tangential curve} of $C(t)$ is the curve $C^*(t)$,
where $C^*(t)$ is the dual of the tangent at $C(t)$.
\end{defn2}

\begin{corollary}
Let $C$ be a closed plane curve.
$P \in \mbox{kernel}(C)$ if and only if dual($P$) does not intersect the 
tangential curve $C^*$.
\end{corollary}

\begin{defn2}
A line in dual space is {\bf free} if it does not intersect the 
tangential curve $C^*$.
\end{defn2}

\begin{corollary}
\label{cor:free}
$P \in \mbox{kernel}(C)$ if and only if dual($P$) is free.
\end{corollary}

% picture of kernel, points in and out of kernel, and associated dual lines
% including some tangents
\begin{figure}
\begin{center}
\includegraphics*[scale=.4]{img/jjke3a.jpg}
\includegraphics*[scale=.4]{img/jjke3b.jpg}
\includegraphics*[scale=.4]{img/jjke3c.jpg}
\end{center}
\caption{Kernel points in primal space are associated with free lines in dual space}
\label{fig:freeline}
% kernel -p -d 10 -x .5 data/ob1b.pts (origin now in kernel)
\end{figure}

\noindent 
The essence of duality is that lines in dual space are associated
with points in primal space.
(The tangential curve associates tangents in dual space with points in
primal space.)
Corollary~\ref{cor:free} says that free lines in dual space
are associated with kernel points in primal space (Figure~\ref{fig:freeline}).
Continuing this argument, free tangents (that is, extremal free lines) 
in dual space must be associated with boundary kernel points in primal space.
And free bitangents (free lines extremal in at least two ways) in dual space 
are associated with corner kernel points in primal space 
(Figures~\ref{fig:kernelob1a} and \ref{fig:complex}).

\begin{lemma}
$P$ is a corner of the kernel if and only if dual($P$) is a free bitangent of $C^*$.
\end{lemma}

\begin{figure}
\begin{center}
\includegraphics*[scale=.5]{img/jjke16.jpg}
\end{center}
\caption{Corners of the kernel are associated with free bitangents in dual space}
\label{fig:kernelob1a}
\end{figure}

\begin{figure}
\begin{center}
\includegraphics*[scale=.5]{img/jjke17.jpg}
\end{center}
\caption{The kernel of a cloverleaf}
\label{fig:complex}
\end{figure}

The remaining task is to properly connect the corners into the kernel.
Since the kernel is convex, the order of the corners is obvious.
The type of connection is also clear: if the corner point lies in the strict
interior of $C$, connect it with its neighbouring corner by a straight line (why?).
If two consecutive points lie on the boundary of $C$, connect
them by a curve segment.
The type of bitangent will reveal whether the point lies on the boundary of $C$,
as discussed below.
	% if bitangent is tangent through cusp, then dual point lies on $C$
	% if bitangent is line through cusps, dual point does not lie on $C$
This leads to a dual algorithm for computing the smooth kernel.

{\bf Don't introduce algorithm yet: stay pure.}

\centerline{{\bf Computing the kernel of a closed plane curve $C$ in dual space}}

\begin{enumerate}
\item	Compute the free bitangents of the tangential curve system $C^*$.
	The associated points in primal space are the corners of the kernel.
\item   Radially sort the kernel corners about their sample mean.
\item	Join consecutive kernel corners by a curve segment if they both lie
	on $C$ and by a line segment otherwise.
	This defines the boundary of the convex kernel.
\end{enumerate}

A purer algorithm that does not require any computation in primal space
can be developed once we better understand the relationship of the
problem to cusp bitangents in the next section.

\section{Bitangents in dual space}

In this section, we explore the bitangents of the tangential curve system,
and discover that they must all be cusp bitangents when the kernel is nonempty.

\Comment{
We shall compute the kernel of a closed plane curve $C$ 
in dual space, using only the tangential curve system $C^*$.
A point set in primal space becomes a line set in dual space.
By Corollary~\ref{cor:free}, it is the lines that do not intersect $C^*$
(the free lines) that characterize the kernel.
The first observation is that a search for free lines can begin
with a search for free bitangents.
}

\begin{defn2}
\label{defn:bitang}
Let $C^*$ be a tangential curve system.
A {\bf $C^*$-tangent} is a conventional tangent of $C^*$
or a line through an endpoint of $C^*$.

A {\bf conventional bitangent} of $C^*$ is a line
that is tangent to $C^*$ at two or more distinct points.
In nontrivial cases, the tangential curve system will contain cusps.
A {\bf cusp bitangent} of $C^*$ is a tangent of $C^*$ that passes
through a cusp, or a line between two cusps.
A {\bf degenerate bitangent} is a tangent that passes through 
an endpoint of $C^*$.
In the rest of this paper, a {\bf $C^*$-bitangent} is a 
conventional, cusp, or degenerate bitangent of $C^*$.
\end{defn2}

\Comment{
\begin{lemma}
\label{lem:existence}
Let $C$ be a closed plane curve.
If $\mbox{kernel}(C)$ is nonempty, one of the $C^*$-bitangents is free.
\end{lemma}
\prf
Since the kernel is nonempty,
there exists a free line in dual space.
By continuous movements, this free line can be continuously moved 
until it becomes $C^*$-tangent to the tangential curve system,
while preserving its freedom.
(This movement is coordinated in the two dual spaces so that the
two lines represent the same point.)
Then, this free $C^*$-tangent can be moved to a free $C^*$-bitangent.
In particular, we can move the free line until it touches the curve once,
then move the tangent along the tangential curve 
until it touches the curve again at an endpoint or a second point of tangency.
\QED

\vspace{-.3in}
}

% can restrict to cusp and degenerate bitangents

As discussed in \cite{jj02},
the conventional bitangents of a curve are associated
with the self-intersections of its tangential curve system.\footnote{If 
	a line is tangent to the curve at two distinct points,
	the tangents at the two points dualize to an intersection.}
But we also show in that paper that $(C^*)^* = C$ (which is consistent
with the fact that the dual map is its own inverse).
Therefore, conventional bitangents of $C^*$ are associated
with self-intersections of $C$.
But if the kernel of $C$ is nonempty, $C$ cannot have any self-intersections.
(The curve's interior is divided into two parts at a self-intersection,
and these two parts cannot see each other.  Consider a figure 8.)
We conclude that the free $C^*$-bitangents
are cusp or degenerate bitangents.

{\bf Establish that degenerate bitangents are never free, 
so that we can restrict to cusp bitangents.}

\begin{lemma}
$P$ is a corner of the kernel if and only if dual($P$) is a free cusp bitangent of $C^*$.
\end{lemma}

Since all free bitangents of $C^*$ are cusp bitangents,
we can classify the associated kernel corners in primal space.
Kernel corners associated with tangents through a cusp lie
on $C$, since a tangent of $C^*$ dualizes to a point of $C$.
Kernel corners associated with lines through two cusps lie
inside $C$.

\centerline{{\bf Computing the kernel of a closed plane curve $C$ in dual space}}

\begin{enumerate}
\item	Compute the free cusp bitangents of the tangential curve system $C^*$.
\item
\begin{enumerate}
\item	If $T$ is a free cusp bitangent in dual space between two cusps $c_1$ 
	and $c_2$,
	connect its associated corner $T^*$ by a line segment to any corner
	associated with a free cusp bitangent involving $c_1$ or $c_2$.
\item	If $T_1$ is a free cusp bitangent with point of tangency $p_1$
	and $T_2$ is a free cusp bitangent with point of tangency $p_2$,
	connect $T_1$'s associated corner 
	with $T_2$'s associated corner by a curve segment
	if there are no cusps or other points of tangency on the 
	curve segment between $p_1$ and $p_2$.
	(PROOF?)
\end{enumerate}
\end{enumerate}

\Comment{
\begin{enumerate}
\item	Compute the free cusp bitangents of the tangential curve system $C^*$.
	(The associated points in primal space are the corners of the kernel.)
\item	(The kernel corners are connected together to define the boundary
	of the kernel as follows. 
	Recall that cusp bitangents come in two flavours:
	lines between two cusps and tangents through cusps.)
	
	If $T$ is a free line in dual space between two cusps $c_1$ and $c_2$,
	connect its associated corner $T^*$ by a line segment to any corner
	associated with cusps $c_1$ or $c_2$.
	If $T_1$ is a free tangent through a cusp with point of tangency $p_1$
	and $T_2$ is a free tangent through a cusp with point of tangency $p_2$,
	connect $T_1$'s associated corner 
	with $T_2$'s associated corner by a curve segment
	if there are no cusps or other points of tangency on the 
	curve segment between $p_1$ and $p_2$.
	(PROOF?)
\end{enumerate}
}

\section{An explanation in primal space}

Now that we have developed an algorithm purely in dual space,
we can pause and interpret this algorithm in light of primal space.
The cusps of $C^*$ are associated with tangents through the inflection points
of $C$ (Figure~\ref{fig:infl}).  (Why?)
A line through 2 cusps is associated with an intersection of two
inflection tangents (Figure~\ref{fig:comp1}),
and a tangent through a cusp is associated with an intersection of an
inflection tangent with the curve $C$ (Figure~\ref{fig:comp2}).
In short, cusp bitangents are associated with intersections of 
inflection tangents with $C$ and themselves.

\begin{figure}
\begin{center}
\includegraphics*[scale=.4]{img/jjke5.jpg}
\end{center}
\caption{Inflection points of $C$ and their tangents, and the associated cusps of the clipped $C^*$}
\label{fig:infl}
% kernel -p -d 10 data/ob1b.pts
\end{figure}

\begin{figure}
\begin{center}
\includegraphics*[scale=.4]{img/jjke6.jpg}
\end{center}
\caption{The first type of cusp bitangent, interpreted in primal space}
\label{fig:comp1}
% fig 11
\end{figure}

\begin{figure}
\begin{center}
\includegraphics*[scale=.4]{img/jjke7.jpg}
\end{center}
\caption{The second type of cusp bitangent, interpreted in primal space}
\label{fig:comp2}
\end{figure}

We are basically computing the convex hull of $C^*$ in dual space,
except that $C^*$ is a messy disconnected curve and so the convex hull
is not traditional (Figure~\ref{fig:banana}).
In fact, the kernel and the convex hull can be equated,
if we can first find one point of the kernel \cite{wang02}.
({\bf Elaborate.})

	%	kernel data/banana1.pts; 
	%	kernel data/banana4.pts.
\begin{figure}
\begin{center}
\includegraphics*[scale=.5]{img/jjke15.jpg}
\includegraphics*[scale=.5]{img/jjke14.jpg}
\end{center}
\caption{The kernel of two bananas, one empty and one not}
\label{fig:banana}
\end{figure}

In this light, finding free cusp bitangents is equivalent
to finding intersections of the inflection tangents (with $C$ and themselves)
that lie inside the kernel.
In primal space, this could be done by comparing against the
interior halfplanes of the inflection tangents.
The algorithm in dual space points out that the kernel is fully defined
by these interior intersections (Figure~\ref{fig:free}).

\begin{figure}
\begin{center}
\includegraphics*[scale=.4]{img/jjke8.jpg}
\end{center}
\caption{Certain intersections of the inflection tangents fully define the kernel}
\label{fig:free}
\end{figure}

\section{Reduction to finding a seed point of the kernel}

The problem with finding a seed point is curves with empty kernel.
We must find a set that provably contains a kernel point, if one exists.
This must be a rather comprehensive set, since it is must be sufficient.

\section{The tangential curve system and dual space}

{\bf Refinement to free lines in two dual spaces using the tangential curve system,
for robust treatment of tangents that map to infinity.}

\section{Cusps}

The computation of the cusps of a rational curve is discussed in
\cite{manocha92} and \cite{li97}.
On a curve with a proper parametrization (see Sederberg 1986, basically
a curve that cannot be degree reduced),
a point is a cusp if and only if its tangent vector vanishes \cite{manocha92}. % Section 5, p. 11
(On improper curves, a vanishing tangent vector is only a necessary
condition for a cusp.)
Finding cusps reduces to root finding of a univariate polynomial.

\Comment{
On a plane algebraic curve $f(x,y,w)=0$ expressed in projective space, 
the tangent through a simple point $P$ of the curve
is the line $f_x(P) x + f_y(P) y + f_w(P) w = 0$.
A singular point is a point with $f_x(P) = f_y(P) = 0$ (and possibly
other vanishing higher derivatives).
If all of the first $r-1$ derivatives vanish at $P$ but at least one of the
$r^{th}$ derivatives does not vanish, 
then the singular point has $r$ tangents.
A cusp is a singular point where these $r$ tangents are identical.
}

\section{Finding cusps}

Why not find roots of 'length of tangent = 0'?
$x'(t)^2 + y'(t)^2 = 0$.
This would have the surface analog of 'length of normal = 0'.
Similar to finding the silhouette as intersection of normal surface
with z=0 plane.
Zeros of $f(u,v) = x_u^2 + y_u^2 + z_u^2 = 0$ are intersections of
$z = f(u,v)$ with $z=0$ ($\frac{\partial S}{\partial u} = 0$).
Zeros of $g(u,v) = x_v^2 + y_v^2 + z_v^2 = 0$ 
($\frac{\partial S}{\partial v} = 0$).
And zeroes of $(x_u - x_v)^2 + (y_u - y_v)^2 + (z_u - z_v)^2 = 0$
($\frac{\partial S}{\partial u} = \frac{\partial S}{\partial v}$).

For each segment of a rational Bezier curve, 
you must solve for all roots of $I(t) = (R(t) \times R'(t)) \cdot R''(t)$
where $R(t) = (X(t),Y(t),W(t))$,
then isolate the multiple roots that satisfy 
$X'(t) : X(t) = Y'(t) : Y(t) = W'(t) : W(t)$.
How will we find multiple roots, since they may violate Descartes'
rule of signs?
Find $G(t) = gcd(I(t),I'(t))$ using the Euclidean algorithm
and solve this function for its roots (could these be multiple roots? yes).
Notice that if $R(t)$ is degree 5 (as in tangential curves),
$I(t)$ is of degree 3(3) = 9 and 
$G(t)$ is of degree at most 4 (at most 4 double roots in 9 roots).
Recursively find multiple roots of $G(t)$
until you finally get a constant gcd or degree 1 or 2 gcd that
can be directly solved.

References:

'Identification of inflection points and cusps on rational curves'
Yong-Ming Li and Robert J. Cripps (1997)
CAGD 14, 491--497.

'Detecting Cusps and Inflection Points'
D. Manocha and J. Canny (1992)
CAGD 9, 1--24.

\section{Moving to surfaces in 3-space}

{\bf Motivation for writing up tangential surface system paper.}

{\bf Give example of silhouette in tangential surface paper
	to show how lazy evaluation works.}

Computing the kernel of a surface $S$ in dual space
is related to the convex hull of the tangential surface system $S^*$.
The convex hull of a surface is defined by free bitangent developables.
If the surface kernel is nonempty, $S$ has no self-intersections
so the tangential surface system has no conventional bitangent planes.
Thus, we are looking for cusp bitangent planes.
In particular, we are looking for free cusp bitangent developables.
The kernel in primal space will be defined by these developables
along with portions of $S^*$ between the developables.

\begin{defn2}
A {\bf surface cusp} is a point where the normal vector disappears.
A {\bf cusp bitangent plane} is a tangent plane that passes through a cusp.
A {\bf cusp bitangent developable} is the envelope of a one-parameter family
of cusp bitangent planes.
\end{defn2}

Surface cusps are computed as follows.
The normal of a surface $S(u,v)$ at a simple point is 
$\frac{\partial S}{\partial u} \times \frac{\partial S}{\partial v}$.
This normal vanishes if $\frac{\partial S}{\partial u} = 0$,
$\frac{\partial S}{\partial v} = 0$, or 
$\frac{\partial S}{\partial u} = \frac{\partial S}{\partial v}$.

The tangent planes of a surface $S$ through a point $p$ are computed 
by intersecting the plane $p^*$ with the tangential surface system $S^*$.
Thus, the tangent planes of $S^*$ through a cusp $p$ are computed 
in primal space (the dual of dual space) by intersecting the plane $p^*$ 
with $S$.

\begin{itemize}
\item	Compute cusp curves.
\end{itemize}

What is the mapping of a cusp curve to primal space?
Vanishing normal.
Use $(S^*)^* = S$.
A tangential surface represents the tangent space of a surface,
which can be interpreted as the normal space (since an element
of the tangent space is a tangent plane, which is identified with
a normal).
Think about how the tangential surface relates to the normal surface.

\end{document}
