% curveKernel.tex
% Last updated: 31 January 2003
% Alternate ACM SIG Proceedings document using LaTeX2e
\documentclass{sig-alternate}

\newcommand{\Comment}[1]{\relax}  % makes a "comment" (not expanded)
\newcommand{\QED}{\vrule height 1.4ex width 1.0ex depth -.1ex\ \vspace{.3in}} % square box
\newcommand{\lyne}[1]{\mbox{$\stackrel{\leftrightarrow}{#1}$}}
\newcommand{\seg}[1]{\mbox{$\overline{#1}$}}
\newcommand{\prf}{\noindent{{\bf Proof}:\ \ \ }}
\newcommand{\choice}[2]{\mbox{\footnotesize{$\left( \begin{array}{c} #1 \\ #2 \end{array} \right)$}}}      
\newcommand{\scriptchoice}[2]{\mbox{\scriptsize{$\left( \begin{array}{c} #1 \\ #2 \end{array} \right)$}}}
\newcommand{\tinychoice}[2]{\mbox{\tiny{$\left( \begin{array}{c} #1 \\ #2 \end{array} \right)$}}}

\newtheorem{theorem}{Theorem}	
\newtheorem{rmk}[theorem]{Remark}
\newtheorem{example}[theorem]{Example}
\newtheorem{conjecture}[theorem]{Conjecture}
\newtheorem{claim}[theorem]{Claim}
\newtheorem{notation}[theorem]{Notation}
\newtheorem{lemma}[theorem]{Lemma}
\newtheorem{corollary}[theorem]{Corollary}
\newtheorem{defn2}[theorem]{Definition}
\newtheorem{observation}[theorem]{Observation}

\title{Kernels from hulls}
\numberofauthors{1}
\author{
\alignauthor J.K. Johnstone\thanks{This work was supported by the National Science Foundation under grant CCR-0203586.}\\
\affaddr{Geometric Modeling Lab}\\
\affaddr{Computer and Information Sciences}\\
\affaddr{University of Alabama at Birmingham}\\
\affaddr{University Station, Birmingham, AL, USA 35294}\\
\email{jj@cis.uab.edu}
}

\begin{document}
\conferenceinfo{ACMSE}{'03 Savannah, Georgia USA}
\CopyrightYear{2003}
\maketitle

\begin{abstract}
The kernel of a polygon has long been an object of study in computational
geometry, especially in the context of art gallery theorems.
In this paper, we present two new algorithms for computing the kernel of a
plane curve.
We show that the computation of the kernel of a plane curve
can be reduced to the computation of the convex hull of a plane curve,
revealing the computational and structural similarity of these two fundamental
geometric objects.
The kernel of a curve is studied in dual space, by applying the 
geometric duality between lines and points to the curve's tangent space.
The kernel in primal space becomes the convex hull in dual space.
\end{abstract}

\category{I.3.5}{Computer Graphics}{Computational Geometry and Object Modeling}[Geometric algorithms, languages, and systems, splines.]
\keywords{Kernel, convex hull, duality, plane curve, tangential curve.}

%%%%%%%%%%%%%%%%%%%%%%%%%%%%%%%%%%%%%%%%%%%%%%%%%%%%%%%%%%%%%%%%%%%%%%%%%%%%%%%%%%%

\section{Introduction}

The dual relationship between Voronoi diagram and Delaunay triangulation
\cite{preparata85} is often leveraged to compute Voro\-noi diagrams from 
Delaunay triangulations,
allowing all attention to be placed on the development of Delaunay triangulation
algorithms.
This paper explores a similar dual relationship between the kernel and convex hull
of closed plane curves.
This relationship binds the computation of these two important geometric constructs
and can be used to compute kernels from convex hull algorithms. 

The {\bf kernel} of a plane curve $C$ is the locus of points that can see every
point of $C$:
\[
\mbox{kernel}(C) = \{ P \in R^2: \seg{PQ} \ \cap \ C = \emptyset 
		\hspace{.25in} \forall \ Q \in C \}
\]
It is a convex connected subset of the curve's interior (Figure~\ref{fig:curveob1b}).
Since a point of the kernel maximizes visibility, the kernel 
is an important structure in visibility analysis,
including problems in lighting, rendering, and motion planning.
For example, interpreting the curve as the boundary of a room,
a point of the kernel would be an ideal location to place a light, camera, or robot,
since the entire room would be visible.

\begin{figure}[h]
\begin{center}
\includegraphics*[scale=.25]{img/jjke0.jpg}
\end{center}
\caption{A smooth curve and its kernel}
\label{fig:curveob1b}
\end{figure}

The kernel of a polygon is a classical construct of computational geometry
\cite{goodman97,preparata85} and visibility analysis \cite{orourke87}.
Lee and Preparata developed an optimal linear algorithm 
for the kernel of a simple polygon \cite{lee79}.
Recently, Elber computed the kernel of a plane curve using 
an algebraic computation from the zero set of a system of nonlinear equations
\cite{elber02}.
He observed that 
an alternative characterization of the kernel is available for curves:
a kernel point is not hit by any tangent.

\begin{lemma}[Elber]
\label{lem:primalkernelchar}
Let $C$ be a closed plane curve.
$P$ lies in the interior of $C$'s kernel if and only if 
no tangent of $C$ intersects $P$.
\end{lemma}

Elber's characterization establishes a relationship between the kernel 
and the tangent space,
which carves out the kernel (Figure~\ref{fig:kernelFromTangSpace}).
This can be better explored in dual space, where the tangent space
becomes a curve.
The rest of the paper develops this idea and is organized as follows.
Some theory of geometric duality is introduced in Section~\ref{sec:duality},
especially the dualization of the tangent space of a curve.
This theory is applied to the kernel problem in Section~\ref{sec:reduction},
where it is shown that the kernel reduces to the convex hull.
This reduction relies on the prior knowledge of a 
single point of the kernel:
Section~\ref{sec:first} addresses the computation of this seed point,
using bitangents in dual space.
Section~\ref{sec:first} includes several examples of kernels computed using
this dual algorithm.
Our computation of a seed point also suggests a second new algorithm
for the kernel, which is presented in Section~\ref{sec:second}.
Some parting thoughts are given in Section~\ref{sec:conclude}.

\begin{figure}
\begin{center}
\includegraphics*[scale=.25]{img/jjke2.jpg}
\end{center}
\caption{The tangent space of Figure~\ref{fig:curveob1b}, revealing the kernel}
\label{fig:kernelFromTangSpace}
\end{figure}

%%%%%%%%%%%%%%%%%%%%%%%%%%%%%%%%%%%%%%%%%%%%%%%%%%%%%%%%%%%%%%%%%%%%%%%%%%%%%%%%%%%

\section{Duality}
\label{sec:duality}

To understand the dual relationship between kernels and hulls,
we must first understand duality.
While Voronoi diagrams and Delaunay triangulations are 
related by a graph (face-vertex) duality, 
kernels and convex hulls are related by a geometric duality.
This geometric duality identifies points and lines in 2-space \cite{pedoe70}.
In this section, we only have room for 
a cursory introduction to geometric duality.
The reader is referred to \cite{jj01,jj02} for a full discussion of geometric duality
and complete details on tangential curves, including their Bezier representation.

Classically, the line $ax+by+c=0$ is dual to the point
$(\frac{a}{c},\frac{b}{c})$ in Cartesian 2-space
(or, more accurately, the point $(a,b,c)$ in projective 2-space)
\cite{hartshorne77}.
The following definition presents three variations on this duality.

\begin{defn2}
The line $ax+by+c=0$ is 
{\bf a-dual} to the point $(\frac{c}{a},\frac{b}{a})$,
{\bf b-dual} to the point $(\frac{a}{b},\frac{c}{b})$, and
{\bf c-dual} to the point $(\frac{a}{c},\frac{b}{c})$.
The point $(a,b)$ is 
{\bf a-dual} to the line $x+by+a=0$,
{\bf b-dual} to the line $ax+y+b=0$, and
{\bf c-dual} to the line $ax+by+1=0$.
\end{defn2}

\noindent By dualizing the tangent at $C(t)$ to the point $C^*(t)$,
the tangent space of a plane curve $C(t)$ can be mapped to a
plane curve $C^*(t)$ in dual space.

\begin{defn2}
Let $C(t)$ be a plane curve.
The {\bf tangential a-curve} of $C$ is the plane curve $C^*_a$ where the point $C^*_a(t)$ 
is the a-dual of the tangent at $C(t)$.  Similarly,
the {\bf tangential b-curve} and {\bf tangential c-curve} of $C$ are the plane curves 
$C^*_b$ and $C^*_c$ where the points $C^*_b(t)$ and $C^*_c(t)$ are
the b-dual and c-dual of the tangent at $C(t)$, respectively.
\end{defn2}

A curve and its tangential c-curve are
illustrated in the first pair of images of Figure~\ref{fig:banana}.
In this pair, as well as the image pairs of Figures~\ref{fig:freeline}-\ref{fig:complex},
the left image is a curve in primal space
and the right image is its tangential c-curve in dual space.
Figures~\ref{fig:freeline}-\ref{fig:complex} 
also include some extra points in primal space
and extra lines in dual space (their duals).

In general, a single tangential curve is not enough to represent the tangent space robustly,
because some tangents will map to infinity.
In particular, horizontal lines, vertical
lines, and lines through the origin are mapped to infinity by a-duality, b-duality
and c-duality, respectively.
Two cooperating dualities are typically needed for a robust representation
\cite{jj01}.

\begin{defn2}
\label{defn:tcsystem}
The tangential a-curve $C^*_a$ within $y \in [-1,1]$ and
the tangential b-curve $C^*_b$ within $x \in (-1,1)$ are collectively 
called the {\bf tangential curve system} 
($C_a^*, C_b^*$) of $C$.
The tangential curve system is always a robust representation of the tangent space of $C$ \cite{jj01,jj02}.
\end{defn2}

When the kernel of $C$ is not empty, there is a special case
when one tangential curve suffices.
This will become important in the reduction to the convex hull.

\begin{lemma}
\label{lem:crobust}
The tangential c-curve is a robust finite representation of $C$'s tangent space
whenever the origin lies inside $C$'s kernel.\footnote{This 
	is not a general solution to the representation of tangent spaces
	because many curves have empty kernels.}
\end{lemma}
\begin{proof}
Lines through the origin are mapped to infinity by the c-duality.
If the origin lies inside the kernel,
no tangent passes through the origin (Lemma~\ref{lem:primalkernelchar})
so no tangent maps to infinity.
\end{proof}

%%%%%%%%%%%%%%%%%%%%%%%%%%%%%%%%%%%%%%%%%%%%%%%%%%%%%%%%%%%%%%%%%%%%%%%%%%%%%%%%%%%

\section{A reduction to the convex hull}
\label{sec:reduction}

We are now ready to reduce the kernel in primal space to the convex hull in dual space.
We begin by reinterpreting Elber's characterization of a kernel point 
(Lemma~\ref{lem:primalkernelchar}) in dual space.
Any duality may be used in the following lemma.

\begin{defn2}
A line in dual space is {\bf free} if it does not intersect the tangential curve.
\end{defn2}

\begin{lemma}
\label{lem:dualkernelchar}
The point $P$ lies in the interior of $C$'s kernel if and only if 
its dual line $P^*$ is free.
\end{lemma}
\begin{proof}
By Lemma~\ref{lem:primalkernelchar},
$P$ lies in the interior of $C$'s kernel if and only if 
'no tangent of $C$ intersects $P$'.
The latter statement dualizes to 'no point of $P^*$ intersects the tangential curve $C^*$'.
\end{proof}

This lemma identifies our desired goal of kernel points with free lines (Figure~\ref{fig:freeline}),
shifting our attention to intersection with the tangential curve.
Notice that Lemma~\ref{lem:dualkernelchar} is a much simpler kernel test than 
Lemma~\ref{lem:primalkernelchar},
involving a single line-curve intersection.

% picture of kernel, points in and out of kernel, and associated dual lines
% including some tangents
\begin{figure}[h]
\begin{center}
\includegraphics*[scale=.25]{img/jjke3a.jpg}
\includegraphics*[scale=.25]{img/jjke3b.jpg}
\includegraphics*[scale=.25]{img/jjke3c.jpg}
\end{center}
\caption{Kernel points in primal space are associated with free lines in dual space}
\label{fig:freeline}
% kernel -p -d 10 -x .5 data/ob1b.pts (origin now in kernel)
\end{figure}

Since the free lines of a curve bound its convex hull,
Lemma~\ref{lem:dualkernelchar} suggests a relationship between the kernel 
and convex hull.

\begin{theorem}
\label{thm:kerneldual}
The kernel of $C$ is dual to the convex hull of its tangential c-curve $C^*_c$
whenever the origin lies inside $C$'s kernel.
\end{theorem}
\begin{proof}
Suppose that the origin lies inside $C$'s kernel.
Then $C^*$ is a closed, finite curve (Lemma~\ref{lem:crobust}) 
and its convex hull is well defined.
Since free lines in dual space correspond to kernel points of $C$,
free tangents of $C^*$ (extremal free lines)
in dual space correspond to boundary points of the kernel of $C$ in primal space.
But free tangents of a curve correspond to boundary points of its convex hull.
Therefore, the boundary points of the convex hull of $C^*$ correspond to the 
boundary points of the kernel of $C$.
\end{proof}

More concretely, the kernel can be defined from the convex hull as follows.

\newpage

\centerline{{\bf Algorithm 1: Computing the kernel of $C$}}

\begin{enumerate}
\item 	Find a point P of the kernel (Section~\ref{sec:first}).  If no such point exists, 
	the kernel is empty.
\item   Translate P to the origin.
\item	Compute the tangential curve $C^*$ of the translated $C$ \cite{jj01}.
\item	Compute the convex hull of $C^*$.
\item	The convex hull is built from curve segments and bitangents.
	These define the kernel as follows (see Figure~\ref{fig:kernelob1a}):
\begin{enumerate}
\item
	The curve segments $C^*(t_1,t_2)$ of the hull boundary define
	the curve segments $C(t_1,t_2)$ of the kernel boundary.
\item
	The bitangents of the hull boundary define the corner points of the kernel.
\item
	Two of these kernel corners are connected by a line segment if the intersection of the
	associated bitangents lies on the convex hull.
\end{enumerate}
\end{enumerate}

\begin{figure}[h]
\begin{center}
\includegraphics*[scale=.4]{img/jjke16.jpg}
\end{center}
\caption{The kernel can be made dual to the convex hull}
\label{fig:kernelob1a}
\end{figure}

This algorithm shows that, if we can find a single point of the kernel,
the computation of the kernel can be reduced to the computation of the convex hull.
The computation of a seed point is addressed in the next section.

%%%%%%%%%%%%%%%%%%%%%%%%%%%%%%%%%%%%%%%%%%%%%%%%%%%%%%%%%%%%%%%%%%%%%%%%%%%%%%%%%%%

\section{The first point of the kernel}
\label{sec:first}

The reduction of kernel to convex hull relies on the prior knowledge
of a single point of the kernel.
This point is used to adjust the curve before dualization, ensuring
the finiteness of the tangential c-curve.
This section discusses the computation of a first point of the kernel.
For consistency, we shall use dual space to guide the search for a kernel point,
finding a special free line.
This will also reveal the structure of the convex hull associated with nontrivial kernels.

Before proceeding, the kernel characterization of Lemma~\ref{lem:dualkernelchar}
must be adapted to the tangential curve system.
Since there is no longer any way to guarantee finiteness in mapping to dual space,
the computation of this section 
requires the use of the tangential curve system for robustness
(Definition~\ref{defn:tcsystem}).

\begin{defn2}
\label{defn:linesystem}
Let $P$ be a point.
The {\bf line system} of $P$ is the combination 
of its a-dual and b-dual lines, $(P^*_a, P^*_b)$.
The line system $(P^*_a, P^*_b)$ is {\bf free} 
if the a-dual line $P^*_a$ 
does not intersect the tangential a-curve of $C$ inside $y \in [-1,1]$
and the b-dual line $P^*_b$ 
does not intersect the tangential b-curve of $C$ inside $x \in (-1,1)$.
\end{defn2}

\begin{lemma}
\label{lem:robustdualkernelchar}
The point $P$ lies in the interior of $C$'s kernel if and only if 
its line system $(P^*_a,P^*_b)$ is free.
\end{lemma}

\noindent Lemma~\ref{lem:robustdualkernelchar} reduces the search for a kernel point
in primal space to the search for a free line system in dual space.
We will now show that this search can be limited to a very small set of lines:
the cusp bitangents.

\begin{defn2}
\label{defn:cusp}
A {\bf cusp} of a (proper) curve is a a point whose tangent vector vanishes.
A conventional bitangent of a plane curve is a line that
is tangent to the curve at two or more distinct points.
A {\bf cusp bitangent} is a degenerate\footnote{Any
	line through a cusp may be considered a degenerate tangent because,
	as the tangent reverses direction at the cusp,
	it sweeps out all lines through the cusp.}
bitangent defined by
a tangent that passes through a cusp (type 1) or a line between two cusps
(type 2).
\end{defn2}

\noindent The right of Figure~\ref{fig:kernelob1a} illustrates three cusp bitangents,
two of type 1 and one of type 2.
The following lemmas apply to any dual space (a, b, or c).

\begin{lemma}
\label{lem:allcusp}
Let $C$ be a curve with nonempty kernel.
All of the bitangents of its tangential curve $C^*$ are cusp bitangents.
\end{lemma}
\begin{proof}
Since a curve with a self-intersection has an empty kernel,
$C$ does not have any self-intersections.
But conventional bitangents of a curve are associated
with self-intersections of its tangential curve \cite{jj01}.
Thus, $C^*$ has no conventional bitangents, since the tangential curve of $C^*$ is $C$ \cite{jj02}.
\end{proof}

Lemma~\ref{lem:allcusp} reveals the structure of the hulls in kernel computation:
all of the bitangents that define the hull are associated with cusps.
This is illustrated in Figure~\ref{fig:complex}.
Note that a convex hull is built by bridging concavities with bitangents.

The following lemma allows the search for a free line to be restricted
to a search for a free cusp bitangent,
reducing its domain from a doubly infinite family of lines to a finite set.

\begin{lemma}
\label{lem:cuspisenough}
Let $C$ be a concave curve with tangential curve $C^*$.
If any line in dual space is free,
one of the cusp bitangents of $C^*$ is free.
\end{lemma}
\begin{proof}
Consider a free line in dual space.
While preserving its freedom, we can move this free line 
until it touches the curve once,
then pivot until it touches the curve again, generating a bitangent,
which is necessarily a cusp bitangent.
The tangential curve of a convex curve will not contain any cusp bitangents
for this process to relax to, but the tangential curve of a concave curve will:
a concave curve contains inflection points and
cusps are dual to the tangents of inflection points.
\end{proof}

For example, the tangential curve on the top of Figure~\ref{fig:banana}
has no free cusp bitangents, while the tangential curve on the bottom
has three (which are shown).
Lemma~\ref{lem:cuspisenough} leads to the following algorithm for computing a seed point of the kernel.

\vspace{.2in}

\centerline{{\bf Algorithm 2: Computing a seed point of the kernel}}

\begin{enumerate}
\item	Compute the cusp bitangents of $(C_a^*, C_b^*)$.
	Each cusp bitangent of $C^*_a$ naturally pairs with a cusp bitangent
	of $C^*_b$, yielding a line system.
	If there are no cusp bitangents, $C$ is convex and its kernel is the interior of $C$.
\item	Compute the free cusp bitangents (Definition~\ref{defn:linesystem}).
	These free line systems of $(C^*_a, C^*_b)$ correspond
	to kernel points of $C$.
	If there are none, the kernel is empty.
\end{enumerate}

Since the kernel points found by Algorithm~2
will inherently lie on the boundary of the kernel,
their sample mean is a more robust choice for the seed point of the kernel.
This is still a kernel point since the kernel is convex.
In step 1, the cusps of $C^*$ can be computed directly \cite{manocha92}
or by computing inflection points of $C$ and dualizing.
The tangents that pass through a cusp (type 1 cusp bitangents)
can be computed using \cite{jj01ACMSE}.

Algorithm 2 completes the reduction to the convex hull,
by implementing the first step of Algorithm 1.

%%%%%%%%%%%%%%%%%%%%%%%%%%%%%%%%%%%%%%%%%%%%%%%%%%%%%%%%%%%%%%%%%%%%%%%%%%%%%%%%%%%

\begin{figure}[h]
\begin{center}
\includegraphics*[scale=.4]{img/jjke17.jpg}
\end{center}
\caption{The kernel of a cloverleaf}
\label{fig:complex}
\end{figure}

	%	kernel data/banana1.pts; 
	%	kernel data/banana4.pts.
\begin{figure}[h]
\begin{center}
\includegraphics*[scale=.4]{img/jjke15.jpg}
\includegraphics*[scale=.4]{img/jjke14.jpg}
\end{center}
\caption{The kernel of two bananas, one empty and one not}
\label{fig:banana}
\end{figure}

%%%%%%%%%%%%%%%%%%%%%%%%%%%%%%%%%%%%%%%%%%%%%%%%%%%%%%%%%%%%%%%%%%%%%%%%%%%%%%%%%%%

\section{A streamlined algorithm}
\label{sec:second}

We end with a streamlined algorithm for the kernel 
that combines the features of Algorithms~1 and 2.
This algorithm follows from the observation that the 
free cusp bitangents found in the seed point algorithm
already define the convex hull, and can be used directly
to compute the kernel.

To see this idea in primal space,
recall that the free tangents of $C^*$ correspond to the boundary of the kernel 
(proof of Theorem~\ref{thm:kerneldual}).
Pushing this further, the free {\em bi}tangents (as extremal free tangents) 
define the corners of this kernel, which are sufficient to define the entire convex kernel as follows.
(Figure~\ref{fig:kernelob1a} offers a good illustration.)

\vspace{.2in}

\centerline{{\bf Algorithm 3: A streamlined kernel algorithm}}
\nopagebreak
\begin{enumerate}
\item Compute the cusp bitangents and their associated kernel points $K$, using
	Algorithm 2.
\item Sort $K$ about the sample mean of $K$.
\item The kernel of $C$ is defined from $K$ as follows.
      Two consecutive points of the sorted $K$ should be joined by:
\begin{enumerate}
\item a curve segment of $C$ if both points are associated with a type 1
	cusp bitangent.
\item a straight line if either point is associated with a type 2 cusp bitangent.
\end{enumerate}
\end{enumerate}

\noindent This algorithm uses the convex hull, but indirectly.
The kernels of Figures~\ref{fig:kernelob1a}-\ref{fig:banana}
are generated using this algorithm.

%%%%%%%%%%%%%%%%%%%%%%%%%%%%%%%%%%%%%%%%%%%%%%%%%%%%%%%%%%%%%%%%%%%%%%%%%%%%%%%%%%%

\section{Conclusions}
\label{sec:conclude}

Our goal in this paper was to establish the structural relationship between
the kernel and convex hull.
This study also yielded efficient new algorithms for the construction
of the kernel, although this is less important since efficient algorithms already exist.
Section~\ref{sec:second} shows that finding a single point of the kernel 
is almost as hard as finding the entire kernel.
The dual relationship between kernel and hull also pays dividends
in the generalization to surfaces, where the hull can again be used
to compute the kernel.
We are presently working on this extension \cite{jj03}.
To stress the main point of this paper, we end with an 
informal view of the duality between kernel and hull, as revealed by visibility.
The kernel of $C$ is found by shrinking $C$ until every point of the new $C$ 
can see every point of the old $C$.
The convex hull is found by expanding $C$ until every point can see every other point.

\section{Acknowledgements}

Wenping Wang suggested a connection between kernel and convex hull at Dagstuhl 2002.
Gershon Elber suggested the bananas of Figure~\ref{fig:banana}.
I appreciate discussions with both, and the congenial research
atmosphere offered by the Dagstuhl Seminars on Geometric Modeling.
This work was supported by the National Science Foundation under 
grant CCR-0203586.

%%%%%%%%%%%%%%%%%%%%%%%%%%%%%%%%%%%%%%%%%%%%%%%%%%%%%%%%%%%%%%%%%%%%%%%%%%%%%%%%%%%%%

\bibliographystyle{plain}
\begin{thebibliography}{99}

\bibitem{elber02}
Elber, G. (2002)
The kernel of freeform planar parametric curves.
Dagstuhl Seminar on Geometric Modeling 2002.

\bibitem{jj03}
Elber, G., J. Johnstone, M.-S. Kim, and J.-K. Seong (2003)
The convex hull of parametric surfaces.
In preparation.

\bibitem{goodman97}
Goodman, J. and J. O'Rourke, editors (1997)
Handbook of Discrete and Computational Geometry.
CRC Press (New York).

\bibitem{hartshorne77}
Hartshorne, R. (1977)
Algebraic Geometry.
Springer-Verlag (New York).

\bibitem{jj01}
Johnstone, J. (2001)
A Parametric Solution to Common Tangents.
International Conference on Shape Modelling and Applications (SMI2001),
Genoa, Italy, IEEE Computer Society, 240--249.

\bibitem{jj01ACMSE}
Johnstone, J. (2001)
Smooth Visibility from a Point.
39th Annual ACM Southeast Conference, Athens, 296--302.

\bibitem{jj02}
Johnstone, J. (2002)
The tangential curve.
Technical Report, Department of Computer Science, UAB.

\bibitem{lee79}
Lee, D.T. and F. Preparata (1979)
An Optimal Algorithm for Finding the Kernel of a Polygon.
Journal of the ACM 26(3), 415--421.

\bibitem{manocha92}
Manocha, D. and J. Canny (1992)
Detecting Cusps and Inflection Points.
CAGD 9, 1--24.

\bibitem{orourke87}
O'Rourke, J. (1987)
Art Gallery Theorems and Algorithms.
Oxford University Press (New York).

\bibitem{pedoe70}
Pedoe, D. (1970)
Geometry: A Comprehensive Course.
Dover Publications (New York).

\bibitem{preparata85}
Preparata, F. and M. Shamos (1985)
Computational Geometry: An Introduction.
Springer-Verlag (New York).

\end{thebibliography}

\balancecolumns

\end{document}


